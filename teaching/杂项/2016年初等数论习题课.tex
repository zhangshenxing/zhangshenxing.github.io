% -*- coding: utf-8 -*-
\documentclass[12pt,a4paper,reqno]{amsart}
%%% 用 xeLaTeX 编译
\usepackage{amsmath}
\usepackage{latexsym}
\usepackage{amssymb, amscd, amsthm}
\usepackage{fontspec}               %设置字体
\usepackage[all]{xy}                %画交换图
\usepackage{graphicx}               %插图

\usepackage[CJKnumber]{xeCJK}                  % 中英文混排
\usepackage{hyperref}
\usepackage{url}
\usepackage{dsfont}
\usepackage{titlesec}

%%% 设置字体
%\setmainfont{CMU Serif}
%\setsansfont{CMU Sans Serif}
%\setmonofont{CMU Typewriter Text}
\setCJKmainfont[BoldFont={SimHei},ItalicFont={[simkai.ttf]}]{SimSun}
\setCJKsansfont{SimHei}
\setCJKmonofont{FZYTK.ttf}

\newtheorem{thm}{定理}[section]  % 按 section 编号
\newtheorem{prop}[thm]{命题}
\newtheorem{cor}[thm]{推论}
\newtheorem{lemma}[thm]{引理}
\newtheorem*{mainthm}{主定理}

\theoremstyle{remark}
\newtheorem{defn}[thm]{定义}
\newtheorem{remark}[thm]{注}
\newtheorem{exam}[thm]{例}

\makeatletter
\makeatother

%%%%%%%%%% 一些重定义 %%%%%%%%%%
\renewcommand{\contentsname}{\large\bf{目录}}     % 将Contents改为目录
\renewcommand{\abstractname}{\bf{摘要}}           % 将Abstract改为摘要
\renewcommand{\refname}{\bf{参考文献}}            % 将References改为参考文献
\renewcommand{\appendixname}{\bf{附录}}
\renewcommand{\indexname}{\bf{索引}}
\renewcommand{\figurename}{图}
\renewcommand{\tablename}{表}
\renewcommand{\proofname}{证明}
\renewcommand{\datename}{日期}
\renewcommand{\today}{\number\year 年 \number\month 月 \number\day 日}

\titleformat{\section}{\centering\large}{第\CJKnumber{\thesection{}}章}{1em}{\bf}
\titleformat*{\subsection}{\bf}

\newcommand{\Z}{\mathbb{Z}}
\newcommand{\N}{\mathbb{N}}
\newcommand{\F}{\mathbb{F}}
\newcommand{\Q}{\mathbb{Q}}
\newcommand{\R}{\mathbb{R}}
\newcommand{\BC}{\mathbb{C}}
\newcommand{\fp}{\mathfrak{p}}
\newcommand{\leg}[2]{\left(\frac{#1}{#2}\right)}    %勒让德符号
\font\cyr=wncyr10 \newcommand{\Cheb}{\hbox{\cyr Qebywev}}
\renewcommand{\le}{\leqslant}
\renewcommand{\ge}{\geqslant}
\DeclareMathOperator{\ud}{\mathrm{d\!}}
\DeclareMathOperator{\li}{\mathrm{li}}

\begin{document}
\title[初等数论习题课]{\LARGE 初等数论习题课}
\author[张神星]{\large 张神星}
\email{zsxqq@mail.ustc.edu.cn}
\date{\today}

\begin{abstract}
\noindent 本文为2016年春中国科学院大学课程《初等数论》习题课讲义, 课程所用教材为《华罗庚文集 数论卷II》的《数论导引》.
\end{abstract}
\maketitle

本文中所用记号:
\begin{itemize}
  \item $v_p(x)$ 表示非零有理数 $x$ 的素数 $p$ 幂次;
  \item $p^e\| n$ 表示 $p^e\mid n$ 且 $p^{e+1}\nmid n$;
  \item $\mu$ 表示 Mobi\"us 函数;
  \item $\#$ 表示集合的大小;
  \item $\N,\Z,\Q,\R,\BC,\F_p$ 分别表示自然数集, 整数环, 有理数域, 实数域, 复数域, $p$ 元有限域.
\end{itemize}

%\tableofcontents

\section{整数之分解}

\subsection*{1.1}
由于
  \[\begin{split}
                         &[\alpha]\le \alpha<[\alpha]+1\\
    \implies\qquad&n[\alpha]\le n\alpha<n[\alpha]+n\\
    \implies\qquad&n[\alpha]\le [n\alpha]<n[\alpha]+n\\
    \implies\qquad&[\alpha]\le \frac{[n\alpha]}{n}<[\alpha]+1\\
    \implies\qquad&[\alpha]\le \left[\frac{[n\alpha]}{n}\right]<[\alpha]+1
  \end{split}\]
故 $\left[\frac{[n\alpha]}{n}\right]=[\alpha]$.

\subsection*{1.2}
设 $k=[n\alpha]-n[\alpha]$, 则
  \[
    \frac{k}{n}\le \alpha-[\alpha]<\frac{k+1}{n},
    \quad 0\le k\le n-1,
  \]
于是
  \[
    [\alpha]+\frac{k+i}{n}
    \le \alpha+\frac{i}{n}
    <[\alpha]+\frac{k+i+1}{n}.
  \]
因此
  \[
     \left[\alpha+\frac{i}{n}\right]
    =\begin{cases}
        [\alpha],&\ \text{若}\ i<n-k;\\
        [\alpha]+1,&\ \text{若}\ i\geqslant n-k.
    \end{cases}\]
故
  \[
    [\alpha]+\left[\alpha+\frac{1}{n}\right]+\cdots+\left[\alpha+\frac{n-1}{n}\right]
    =n[\alpha]+k=[n\alpha].
  \]

\subsection*{1.3}
设 $x=\alpha-[\alpha],y=\beta-[\beta]$, 则
  \[([2\alpha]+[2\beta])-([\alpha]+[\alpha+\beta]+[\beta])=[2x]+[2y]-[x+y].\]
若 $x>\frac{1}{2}$ 或 $y>\frac{1}{2}$, 则 $[2x]+[2y]\geqslant 1\geqslant [x+y]$; 若 $x,y$ 均 $<\frac{1}{2}$, 则 $x+y<1$, $[2x]+[2y]-[x+y]=0$. 综上所述
  \[[2\alpha+2\beta]>[\alpha]+[\alpha+\beta]+[\beta].\]

\subsection*{4补充}
设 $a>1,m,n$ 为正整数, 证明 $(a^m-1,a^n-1)=a^{(m,n)}-1.$
\begin{proof}
利用辗转相除法,我们有
  \[\begin{split}
          m&=n q_0+r_0,\quad 0<r_0<n;\\
          n&=r_0q_1+r_1,\quad 0<r_1<r_1;\\
           &\cdots\\
    r_{k-2}&=r_{k-1}q_k+r_k,\quad 0<r_k<r_{k-1};\\
    r_{k-1}&=r_k q_{k+1},
  \end{split}\]
则 $r_k=(m,n)$. 于是
  \[\begin{split}
            a^m-1&=a^{r_0}(a^n-1)(a^{n(q_0-1)}+\cdots+1)+a^{r_0}-1;\\
            a^n-1&=a^{r_1}(a^{r_0}-1)(a^{r_0(q_1-1)}+\cdots+1)+a^{r_1}-1;\\
                 &\cdots\\
    a^{r_{k-2}}-1&=a^{r_k}(a^{r_{k-1}}-1)(a^{r_{k-1}(q_k-1)}+\cdots+1)+a^{r_k}-1;\\
    a^{r_{k-1}}-1&=(a^{r_k}-1)(a^{r_k(q_{k+1}-1)}+\cdots+1),
  \end{split}\]
因此 $(a^m-1,a^n-1)=a^{r_k}-1=a^{(m,n)}-1$.
\end{proof}
\begin{proof}[另证]
我们对 $m+n$ 归纳.

若 $m+n=2$, 显然.

假设命题对于 $m+n\le k-1$ 成立. 对于 $m+n=k$, 不妨设 $m\geqslant n$. 若 $n=0$ 或 $m=n$, 显然成立; 否则 $0<n<m$,
  \[a^m-1=a^{m-n}(a^n-1)+a^{m-n}-1,\]
因此 $(a^m-1,a^n-1)=(a^{m-n}-1,a^n-1)$. 由归纳假设, 此为
  \[a^{(m-n,n)}-1=a^{(m,n)}-1.\qedhere\]
\end{proof}

\subsection*{6.2}
以第一个等式为例. 令 $a_k=p_1^{e_{1,k}}\cdots p_s^{e_{s,k}}$, $p_1<p_2<\cdots<p_s, e_{i,k}\ge 0$. 则
  \[\begin{split}
    v_{p_i}(a_1\cdots a_n)&=\sum_{k=1}^n e_{i,k}\\
    v_{p_i}(a_1\cdots a_{j-1}a_{j+1}\cdots a_n)& =\sum_{k=1}^n e_{i,k}-e_{i,j}
  \end{split}\]
  \[
     v_{p_i}(\text{右边})
    =\sum_{k=1}^n e_{i,k}-\max_j \{\sum_{k=1}^n e_{i,k}-e_{i,j}\}
    =\min_j \{e_{i,j}\}=v_{p_i}(\text{左边}).
  \]
因此两边相等.

\subsection*{8.2}
由于
  \[n=bcx+cay+abz=bc(x+at+as)+ca(y-bs)+ab(z-ct),\]
我们不妨设 $0\le y<b, 0\le z<c$, 于是
  \[x=\frac{n-cay-abz}{bc}\geqslant \frac{n-ca(b-1)-ab(c-1)}{bc}=\frac{n-2abc+ac+ab}{bc}.\]
若 $n>2abc-ac-ab-bc$, 则上式大于 $-1$, 因此必然 $\geqslant 0$. 也就是说, 任意大于 $2abc-ab-bc-ca$ 的整数 $n$ 均可由此表出.

若 $n=2abc-ab-bc-ca=bcx+cay+abz$, 则
  \[bc(x+1)+ca(y+1)+ab(z+1)=2abc.\]
若 $x,y,z\geqslant 0$, 则 $x+1,y+1,z+1\geqslant 1$ 且 $a\mid x+1,b\mid y+1,c\mid z+1$, 因此
  \[x+1\geqslant a,y+1\geqslant b,z+1\geqslant c,\]
  \[bc(x+1)+ca(y+1)+ab(z+1)\geqslant 3abc,\]
这不可能!

\subsection*{8.3}
设该方程的解数为 $a_n$, 则我们有
  \[\begin{split}
    f(T)=&\frac{1}{(1-T)(1-T^2)(1-T^3)}\\
        =&(1+T+T^2+\cdots)(1+T^2+T^4+\cdots)(1+T^3+T^6+\cdots)\\
        =&\sum_{x,y,z\geqslant 0} T^{x+2y+3z}=\sum_n a_nT^n.
  \end{split}\]
通过待定系数, 我们有
  \[\begin{split}
     &f(T)\\
    =&\frac{1}{6(1-T)^3}+\frac{1}{4(1-T)^2}+\frac{17}{72(1-T)}
        +\frac{1}{8(1+T)}+\frac{1}{9}(\frac{1}{1-\omega T}\\
     &\qquad {}+\frac{1}{1-\bar\omega T})\\
    =&\sum_{n} (\frac{1}{6}\frac{(n+1)(n+2)}{2}+\frac{1}{4}(n+1)+\frac{17}{72}
        +\frac{(-1)^n}{8}+\frac{2}{9}\cos\frac{2n\pi}{3})T^n\\
    =&\sum_n (\frac{(n+3)^2}{12}-\frac{7}{72}+\frac{(-1)^n}{8}+\frac{2}{9}\cos\frac{2n\pi}{3})T^n.
  \end{split}\]

\subsection*{9.2}
设 $n=\prod\limits_{i=1}^s p_i^{e_i}$, 则
  \[\begin{split}
  \prod_{d\mid n}d=&\prod_{x_1=0}^{e_1}\prod_{x_2=0}^{e_2}\cdots\prod_{x_s=0}^{e_s} \prod_{i=1}^s p_i^{x_i}\\
  =&\prod_{i=1}^s (\prod_{x_i=0}^{e_i} p_i^{x_i})^{\tau(n)/(e_i+1)}\\
  =&\prod_{i=1}^s p_i^{\tau(n)e_i/2}=n^2,
  \end{split}\]
其中 $\tau(n)=\prod\limits_{i=1}^s (e_i+1)$ 为 $n$ 的正因子个数. 因此 $\tau(n)=4$,
  \[ s=1,e_1=3\ \text{或}\ s=2,e_1=e_2=1,\]
$n$ 为一素数之立方或两不同素数之积.

\begin{proof}[另证]
由于
  \[
     \prod_{d\mid n}d=\prod_{d\mid n}\frac{n}{d}
    =\frac{n^{\tau(n)}}{\prod_{d\mid n}d},\]
因此 $\prod\limits_{d\mid n} d=n^{\tau(n)/2}$, $\tau(n)=4$. 其余与上文相同.
\end{proof}

\subsection*{9 补充}
记 $n$ 的因子个数为 $\tau(n)$.

(1) 证明 $\tau(n)\le 2\sqrt{n}$.

(2) 证明 $\tau(n)\le \sqrt{3n}$.

(3) 证明 $\tau(n)\le 8\sqrt[3]{\frac{3n}{35}}$.

\begin{proof}
(1) 由于 $d\mid n$ 等价于 $\frac{n}{d}\mid n$, 因此
  \[\tau(n)\le 2\lceil\sqrt{n}\rceil+1\le 2\sqrt{n},\]
其中 $\lceil x\rceil$ 为不小于 $x$ 的最小的整数.

(2) 对任意 $\lambda>0$ 和素数 $p$, 我们考虑 $f_p(v)=\frac{p^{\lambda v}}{v+1}$ 的最小值, 其中 $v$ 是自然数. 由于
  \[f_p(v)\le f_p(v-1)\iff
  \frac{p^{\lambda v}}{v+1}\le \frac{p^{\lambda {v-1}}}{v}
  \iff  v\le\frac{1}{\sqrt{p}-1},\]
  \[f_p(v)\le f_p(v+1)\iff v+1\geqslant \frac{1}{\sqrt{p}-1},\]
因此
  \[\min_{v\in\mathbb{N}} f_p(v)=f_p([\frac{1}{\sqrt{p}-1}]).\]
若 $\lambda=\frac{1}{2}$, 则
  \[
    \frac{\sqrt{n}}{\tau(n)}=\prod_p \frac{\sqrt{p}^{e_p}}{e_p+1}
    \ge f_2(2)f_3(1)=\frac{2}{3}\times \frac{\sqrt{3}}{2}=\frac{1}{\sqrt{3}}.
  \]
若 $\lambda=\frac{1}{3}$, 则
  \[
    \frac{\sqrt[3]{n}}{\tau(n)}
    \ge f_2(3)f_3(2)f_5(1)f_7(1)
    =\frac{1}{2}\times\frac{\sqrt[3]{3^2}}{3}
     \times\frac{\sqrt[3]{5}}{2}\times\frac{\sqrt[3]{5}}{2}
    =\frac{1}{8}\sqrt[3]{\frac{35}{3}}.
  \]
\end{proof}

\subsection*{10补充}
证明若正整数 $m>n$, 则 $F_n\mid (F_m-2)$ 且 $(F_m,F_n)=1$. 由此证明素数有无穷多个.
\begin{proof}
由于
  \[
    F_m-2
   =2^{2^m}-1
   =(2^{2^n}-1)\prod_{k=n}^{m-1}\frac{2^{2^{k+1}}-1}{2^{2^k}-1}
   =(F_n-2)\prod_{k=n}^{m-1}F_k,\]
因此 $F_n\mid (F_m-2)$ 且 $(F_m,F_n)=(2,F_n)=  1$. 由此可知 $F_n$ 两两互素, 它们含有不同的素因子, 因此素数有无穷多个.
\end{proof}

\subsection*{11.2}
${1000\choose 500}$ 的 $5$ 的幂次为
  \[\begin{split}
    &v_5({1000\choose 500})
   =\sum_{k\ge 1}(\left[\frac{1000}{5^k}\right]-2\left[\frac{500}{5^k}\right])\\
   =&(200-2\times 100)+(40-2\times 20)+(8-2\times 4)+1=1.
  \end{split}\]

\subsection*{11补充1}
设 $m,n$ 为正整数, 证明 $\dfrac{(2m+2n)!}{(m+n)!m!n!}$ 为整数.
\begin{proof}
由习题 1.3 可知
  \[[\frac{2m+2n}{p^t}]\geqslant [\frac{m+n}{p^t}]+[\frac{m}{p^t}]+[\frac{n}{p^t}],\]
因此
  \[v_p\left(\dfrac{(2m+2n)!}{(m+n)!m!n!}\right)\ge 0.\]
由于 $p$ 是任意的, 因此原命题成立.
\end{proof}

\subsection*{11补充2}
设 $a,b$ 为不同的正整数, $n$ 为正整数. 如果 $n\mid a^n-b^n$, 则 $n\mid \dfrac{a^n-b^n}{a-b}$.
\begin{proof}
设 $p^e\|n$, 即 $n$ 的 $p$ 的幂次为 $e$, 则 $a^{p^e}-b^{p^e}\mid a^n-b^n$. 设 $p^f\| i\geqslant 1$, 则 $p^{e-f}\|{p^e\choose i}$. 若 $p\nmid b$, 则
  \[a^{p^e}-b^{p^e}=\sum_{i=1}^{p^e} {p^e\choose i}(a-b)^i b^{p^e-i}\]
的 $p$ 之方次数 $\geqslant e-f+i\geqslant e+1$, 即 $p^e\mid \frac{a^n-b^n}{a-b}$.

若 $p\mid b$, 设 $p^f\| a-b$, 则
  \[a^{p^e}-b^{p^e}=\sum_{i=1}^{p^e}{p^e\choose i}(a-b)^i b^{p^e-i}\]
的 $p$ 之方次数 $\geqslant fi+p^e-i\geqslant fi+e+1-i\geqslant e+f$,  即 $p^e\mid \frac{a^n-b^n}{a-b}$.
\end{proof}

\subsection*{12.1}
凡 $k$ 次 $n$ 元之整值多项式必可表为
  \[\sum_{\lambda_1+\cdots+\lambda_n\le k} \alpha_{\lambda_1,\ldots,\lambda_n}{x_1\choose \lambda_1}\cdots {x_n\choose \lambda_n},\]
式中 $\alpha_{\lambda_1,\ldots,\lambda_n}$ 皆为整数, 且对任何整数 $\alpha_{\lambda_1,\ldots,\lambda_n}$, 此皆整值多项式.
\begin{proof}
1) 如此之多项式显然是整值多项式.

2) $n=1$ 时由定理2知成立. $n\geqslant 2$ 时, 若命题对 $n-1$ 已成立, 由于任一 $k$ 次 $n$ 元整值多项式 $f(x_1,\ldots,x_n)$ 可写成
  \[ f(x_1,\ldots,x_n)=\sum_{i=0}^k \alpha_i(x_2,\ldots,x_n){x_1\choose i}, \]
由归纳假设可得.
\end{proof}

\subsection*{12.3}
设 $k$ 为正整数, 如果 $k=m+\frac{1}{2}(m+n-1)(m+n-2)$, 则
  \[\frac{1}{2}(m+n-1)(m+n-2)\le k-1< m+n-1+\frac{1}{2}(m+n-1)(m+n-2),\]
整理可得
  \[\begin{split}
     m+n=e&=\left[\sqrt{2k-\frac{7}{4}}+\frac{1}{2}\right]+1,\\
     1\le m&=k-\frac{1}{2}e(e-1)<e,\end{split}\]
即 $k$ 可以唯一地写成题述形式.

另证: 将正整数按下图顺序排在第一象限, 则直线 $x+y=e$ 上最大的数为 $e(e-1)/2$, 坐标 $(m,n)$ 处的数为 $\frac{1}{2}(m+n-1)(m+n-2)+m$, 它们无重复无遗漏地取遍所有正整数.

\begin{figure}
 \begin{minipage}[b]{0.5\linewidth} \centering
  \includegraphics[width=60mm]{fig1.12.3.png}
 \end{minipage}
\end{figure}

\subsection*{12.4}
设 $k$ 次多项式 $f(x)$ 在 $a,a+1,\ldots,a+k$ 处取整数值, 令 $g(x)=f(x+a)$, 则 $g(0),g(1),\ldots,g(k)\in\Z$. 设 $g(x)=\sum\limits_{i=0}^k \alpha_i{x\choose i}$, 则
  \[\alpha_k=\Delta^k g\mid_{x=0}=\sum_{i=0}^k (-1)^{k-i} {k\choose i}g(i)\in\Z,\]
因此 $g(x)$ 是整值多项式, $f(x)$ 也是.

\subsection*{13}
设 $f(x)=x^6+x^3+1$, 则
  \[f(x+1)=x^6+6x^5+15x^4+21x^3+18x^2+9x+3.\]
令 $p=3$, 由 Eisenstein 判别法可知 $f$ 不可约.

\subsection*{13 补充1}
设 $a_1,\ldots,a_n$ 为两两不同的整数, 证明 $(x-a_1)\cdots(x-a_n)-1$ 不可约.
\begin{proof}
假设 $f(x)=(x-a_1)\cdots (x-a_n)-1$ 可约,由于 $f$ 首一, 存在首一的非常数整系数多项式 $g(x), h(x)$, $f(x)=g(x)h(x)$. 于是 $g(a_i)=\pm 1$. 不妨设 $g(a_1)=\cdots=g(a_t)=1, g(a_{t+1})=\cdots=g(a_n)=-1$, 则 $\deg g\geq \max\{t,n-t\}$, 同理 $\deg h\geq \max\{t,n-t\}$. 因此 $t=n-t=\deg g=\deg h$,
\[g(x)=(x-a_1)\cdots(x-a_t)+1,\ h(x)=(x-a_1)\cdots(x-a_t)-1,\]
\[f(x)=g(x)h(x)=(x-a_1)^2\cdots(x-a_t)^2-1,\]
矛盾!
\end{proof}

\subsection*{13 补充2}
设 $f(x)=x^n+a_{n-1}x^{n-1}+\cdots+a_0\in\Z[x],a_0\neq 0$.   证明: 如果 $|a_{n-1}|>1+|a_0|+\cdots+|a_{n-2}|$, 则 $f(x)$ 不可约.
\begin{proof}
$f$ 的常数项模长大于等于1, 因此存在根 $x_0$, $|x_0|\geq 1$. 设
  \[ g(x)=\frac{f(x)}{x-x_0}=x^{n-1}+\sum_{i=0}^{n-2} b_i x^i, \]
则
  \[ \begin{split}
        a_0&=-x_0b_0,\\
        a_i&=-x_0b_i+b_{i-1},\quad 1\le i\le n-2,\\
    a_{n-1}&=-x_0+b_{n-2}, \end{split}\]
于是
  \[ \begin{split}
    |b_{n-2}|+|x_0|&\geq |b_{n-2}-x_0|=|a_{n-1}|\\
                   &>1+\sum_{i=0}^{n-2}|a_i|\\
		  	       &=1+\sum_{i=0}^{n-2}|x_0b_i-b_{i-1}|\\
			       &\geq 1+\sum_{i=0}^{n-2}(|x_0b_i|-|b_{i-1}|)\\
			       &=(|x_0|-1)\sum_{i=0}^{n-2}|b_i| +|b_{n-2}|+1
  \end{split} \]
  \[ (|x_0|-1)(\sum_{i=0}^{n-2}|b_i|-1)<0, \]
因此 $|x_0|>1, \sum_{i=0}^{n-2}|b_i|<1$.

设 $g(x)=(x-y_0)(x^{n-2}+\sum_{i=0}^{n-3}c_i x^i)$, 则类似地, 我们有
\[ 1>\sum_{i=0}^{n-2}|b_i|\geq (|y_0|-1)\sum_{i=0}^{n-3}|c_i|+|y_0|,\]
\[ (|y_0|-1)(\sum_{i=0}^{n-3}|c_i|+1)<0, \]
$|y_0|<1$. 因此 $f$ 只有一个根模长大于1, 其余模长均小于1. 如果 $f=gh$  可约, 则 $g,h$ 的常数项均非零, 均存在模长大于等于1 的根, 矛盾! 因此 $f$ 不可约.

实际上, 如果我们利用复变函数中儒歇定理, 可以知道在 $S^1$ 上 $|f-a_{n-1}x^{n-1}|<|a_{n-1}x^{n-1}|$, 从而 $f$ 和 $a_{n-1}x^{n-1}$ 在 $S^1$ 内部有相同多的根, 即 $n-1$ 个.
\end{proof}

\section{同余式}

\subsection*{2 补充}
(1) 设 $n$ 是整数, 证明 $n^2\equiv 0,1\bmod 4, n^3\equiv 0,1\bmod 9, n^4\equiv 0,1\bmod{16}$.

(2) 设 $a$ 是奇数, $n$ 是正整数, 证明 $a^{2^n}\equiv 1\bmod{2^{n+2}}$.
\begin{proof}
(1) 若 $n=2k$, 则 $n^2=4k^2\equiv 0\bmod 4$; 若 $n=2k+1$, 则 $n^2=4k^2+4k+1\equiv1\bmod 4$.

若 $n=3k$, 则 $n^3=27k^3\equiv0\bmod 9$; 若 $n=3k\pm 1$, 则 $n^3=27k^3\pm27k^2+9k\pm1\equiv 1\bmod9$.

若 $n=2k$, 则 $n^4=16k^4\equiv 0\bmod {16}$; 若 $n=4k\pm1$, 则 $n^4=256k^4\pm 256k^3+96k^2\pm 16k+1\equiv 1\bmod{16}$.

(2) 由于
  \[a^{2^n}-1=(a+1)(a-1)\prod_{i=2}^{n-1}(a^{2^i}+1),\]
而 $a+1$ 和 $a-1$ 有一个为 $4$ 的倍数, 因此 $2^{n+2}\mid a^{2^n}-1$.
\end{proof}

\subsection*{3 补充1}
设 $m,n$ 是正整数, $(m,n)=1$. 证明:
\[m^{\varphi(n)}+n^{\varphi(m)}\equiv1\bmod {mn}.\]
\begin{proof}
由于 $m^{\varphi(n)}\equiv1 \bmod n$, 因此 $m^{\varphi(n)}+n^{\varphi(m)}\equiv1\bmod m$. 同理 $m^{\varphi(n)}+n^{\varphi(m)}\equiv1\bmod n$, 因此原命题成立.
\end{proof}

\subsection*{3 补充2}
设 $p,q$ 为不同的奇素数, $a$ 与 $p,q$ 互素, 证明 \[a^{\varphi(pq)/2}\equiv1\bmod{pq}.\]
\begin{proof}
由于 $p,q$ 为奇数, $a^{p-1}\equiv 1\bmod p$, 因此 $a^{(p-1)(q-1)/2}\equiv 1\bmod p$. 同理对 $q$ 成立.
\end{proof}

\subsection*{5.1}
设 $m=\prod p^e, d=\prod p^f,f\le e$, 则
  \[ \varphi(d)=\prod\varphi(p^f), \]
  \[\sum_{d\mid m}\varphi(d)=\prod \sum_{0\le f\le e} \varphi(p^f)=\prod (1+\sum_{1\le f\le e} (p^f-p^{f-1}))=\prod p^e=m.\]

\subsection*{5 补充1}
令 \[\mu(n)=\begin{cases}
  1,\quad & n=1;\\
  (-1)^k, & n=p_1\cdots p_k, p_i\ \text{两两不同};\\
  0,& n\ \text{有平方因子},
\end{cases}\]
则 \[\varphi(n)=\sum_{d\mid n} d\mu(\frac{n}{d}), n=\sum_{d\mid n} \varphi(d).\]

一般地, 若 \[f(n)=\sum_{d\mid n} g(d),\] 则 \[g(n)=\sum_{d\mid n}f(d)\mu(\frac{n}{d}).\]

\subsection*{5.2}
由
  \[\begin{split}
     \frac{\varphi(mn)}{\varphi(m)\varphi(n)}
    &=\frac{mn\prod\limits_{p\mid mn}(1-\frac{1}{p})}
        {m\prod\limits_{p\mid m}(1-\frac{1}{p})
            \times n\prod\limits_{p\mid n}(1-\frac{1}{p})}\\
    &=\frac{1}{\prod\limits_{p\mid(m,n)}(1-\frac{1}{p})}
     =\frac{1}{\prod\limits_{p\mid P}(1-\frac{1}{p})}=\frac{P}{\varphi(P)}
  \end{split}\]
可得.

\subsection*{5 补充2}
(1) 当 $n\ge 3$ 时, $\varphi(n)$ 是偶数.

(2) 当 $n\ge 2$ 时, 不超过 $n$ 且与 $n$ 互素的正整数之和是 $\frac{1}{2}n\varphi(n)$.

\begin{proof}
(1) 由于 $(a,n)=(n-a,n)=1$ 且若 $n$ 为偶数时, $(\dfrac{n}{2},n)\neq1$, 因此
  \[\varphi(n)=2\#\{1\le a< \dfrac{n}{2}\mid (a,n)=1\}\]
为偶数.

或者: 若存在奇素数 $p\mid n$, 则 $(p-1)\mid \varphi(n)$ 为偶数. 若不然, 则 $n=2^k$, $k\geq 2$, 因此 $\varphi(n)=2^{k-1}$ 为偶数.

(2) 由于 $(a,n)=(n-a,n)=1$, 因此
  \[\sum_{1\le a\le n,(a,n)=1} a=\sum_{1\le a\le n,(a,n)=1} (n-a)=\frac{1}{2}\sum_{1\le a\le n,(a,n)=1} n=\frac{1}{2}n\varphi(n).\qedhere\]
\end{proof}

\subsection*{5 补充3}
(1) 设 $f(n),g(n)$ 是积性函数, 证明
  \[(f*g)(n)=\prod_{d\mid n}f(d)g(n/d)\]
也是积性函数.

(2) 证明所有不恒为 $0$ 的积性函数关于 $*$ 构成交换群, 且 $\mathds{1}^{-1}=\mu$, 这里 $\mathds{1}$ 表示常值函数 $1$.

\begin{proof}
(1) 若 $d\mid m, e\mid n$, $(m,n)=1$, 则 $(d,e)=1,(m/d,n/e)=1$, 因此
  \[\begin{split}
       &(f*g)(mn)
      =\prod_{d\mid mn}f(d)g(mn/d)
      =\prod_{d\mid m,e\mid n}f(de)g(mn/de)\\
      =&\prod_{d\mid m,e\mid n}f(d)f(e)g(m/d)g(n/e)
      =\prod_{d\mid m}f(d)g(m/d)\times \prod_{e\mid n}f(e)g(n/e)\\
      =&(f*g)(m)\times (f*g)(n).
  \end{split}\]

(2) 设 $e(1)=1,e(n)=0,n\ge 2$, 则易见 $f*g=g*f,f*e=e*f$.
  \[\begin{split}
     &((f*g)*h)(n)=\prod_{d\mid n}(f*g)(d)h(n/d)\\
    =&\prod_{e\mid d\mid n}f(e)g(d/e)h(n/d)
    =\prod_{e\mid n,d'\mid n/e}f(e)g(d')h(n/d'e)\\
    =&\prod_{e\mid n}f(e)(g*h)(n/e)
    =(f*(g*h))(n).
  \end{split}\]
由 $f(1)f(n)=f(n)$ 知若 $f$ 不恒为 $0$, 则 $f(1)\neq 0$, 定义
  \[g(1)=f(1)^{-1},g(n)=-f(1)^{-1} \sum_{d\mid n,d\neq n} f(n/d)g(d),\]
则 $g$ 为积性函数且 $f*g=e$.

易知 $(\mu*\mathds{1})(n)=\prod_{d\mid n}\mu(d)=e(n)$.
\end{proof}

\subsection*{9 补充}
设 $G$ 是有限 Abel 群, 则 \[\prod_{g\in G}g=\prod_{a\in G,a^2=1} a.\]

由此, 对正整数 $m$, 计算
  \[\prod_{1 \le a\le m,(a,m)=1} a\bmod m.\]
\begin{proof}
由于 $g^2\neq 1$ 当且仅当 $g\neq g^{-1}$, 这样的 $g$ 可以两两配对成互为逆的对, 于是命题可得.

令 $G=(\Z/m\Z)^\times$, 设 $x^2\equiv 1\bmod m$.

若 $m=2^k$, 则 $x=\pm 1,\pm1+2^{k-1}$.

若 $m=p^k$, 则 $x=\pm 1$.

因此当 $4\mid m$ 且 $m$ 有奇素因子, 或 $m$ 有至少两个不同的奇素因子时, 上式为 $1$.

当 $m=p^k,2p^k$ 且 $k\ge 1$, $p$ 为素数时, 上式为 $-1$.
\end{proof}
\section{二次剩余}

\subsection*{2}
设 $p=4n+3$, $q=8n+7$ 为素数, 则
  \[2^p=2^{\frac{q-1}{2}}\equiv \leg{2}{q}=1\mod q,\]
因此 $q\mid 2^p-1$. 令 $p$ 为相应的素数即可得到题目中的关于 Mersenne 素数的结论.

\subsection*{6 补充1}
设\ $p$ 是素数, $p\equiv 1\pmod 4$. 证明

(1) $\sum\limits_{r=1,\leg{r}{p}=1}^{p-1} r=\frac{p(p-1)}{4}$;

(2) $\sum\limits_{r=1}^{p-1} r\leg{r}{p}=0$;

(3) $\sum\limits_{r=1}^{\frac{p-1}{2}} \left[\frac{r^2}{p}\right]=\frac{(p-1)(p-5)}{24}.$

\begin{proof}
(1) 由于 $\leg{p-r}{p}=\leg{-r}{p}=\leg{-1}{p}\leg{r}{p}=\leg{r}{p}$, 因此
  \[
     \sum_{r=1,\leg{r}{p}=1}^{p-1} r
    =\frac{1}{2}\sum_{r=1,\leg{r}{p}=1}^{p-1} p
    =\frac{p(p-1)}{4}.
  \]

(2) 我们有
  \[
     \sum_{r=1}^{p-1} r\leg{r}{p}
    =2\sum_{r=1,\leg{r}{p}=1}^{p-1} r-\sum_{r=1}^{p-1} r=0.
  \]

(3) 由于此时 $r^2$ 取遍 $\bmod p$ 的二次非剩余, 因此
  \[\begin{split}
     \sum_{r=1}^{\frac{p-1}{2}} \left[\frac{r^2}{p}\right]
    &=\sum_{r=1}^{\frac{p-1}{2}}\frac{r^2}{p}
      -\sum_{r=1,\leg{r}{p}=1}^{p-1}\frac{r}{p}\\
    &=\frac{1}{6}\times\frac{p-1}{2}\times \frac{p+1}{2}-\frac{p-1}{4}\\
    &=\frac{(p-1)(p-5)}{24}.\qedhere
  \end{split}\]
\end{proof}

\subsection*{6 补充2}
设\ $p>3$ 是素数, $p\equiv 3\pmod 4$. 证明

(1) $\sum\limits_{r=1,\leg{r}{p}=1}^{p-1} r\equiv 0\pmod p$;

(2) $\sum\limits_{a=1}^{p-1} a\leg{a}{p}\equiv 0\pmod p.$

\begin{proof}
(1) 由于 $1^2,2^2,\ldots,(\dfrac{p-1}{2})^2$ 取遍 $\bmod p$ 的所有二次剩余, 因此
  \[
       \sum_{r=1,\leg{r}{p}=1}^{p-1} r
      \equiv \sum_{s=1}^{(p-1)/2} s^2
      =\frac{p(p^2-1)}{24}
      \equiv 0\pmod p.
  \]

(2) 我们有
  \[
     \sum_{r=1}^{p-1} r\leg{r}{p}
    =2\sum_{r=1,\leg{r}{p}=1}^{p-1} r-\sum_{r=1}^{p-1} r
    \equiv \frac{(p-1)p}{2}
    \equiv 0\pmod p.\qedhere
  \]
\end{proof}

\subsection*{6 补充3}
证明方程 $x^2-y^2\equiv n\bmod p$ 在 $\bmod p$ 意义下的解的个数为 $p-1$, 若 $p\nmid n$; 为 $2p-1$, 若 $p\mid n$.

\begin{proof}
若 $p\mid n$, 则所有解为
  \[ (s,\pm s),(-s,\pm s),(0,0),\quad s=1,2,\ldots,p-1,\]
共 $2p-1$ 个解. 若 $p\nmid n$, 令 $s=x+y$, 则 $x-y\equiv ns^{-1}\bmod p$, 所有解为
  \[(\frac{s+ns^{-1}}{2},\frac{s-ns^{-1}}{2}),\quad s=1,2,\ldots,p-1,\]
共 $p-1$ 个解.
\end{proof}

\subsection*{6 补充4}
设\ $p$ 是奇素数, $f(x)=ax^2+bx+c$ 且\ $p\nmid a$. 记
  \[D=b^2-4ac.\]
证明
  \[\sum_{x=0}^{p-1} \leg{f(x)}{p}=\begin{cases}
    -\leg{a}{p},&\text{如果}\ p\nmid D,\\
    (p-1)\leg{a}{p},\quad&\text{如果}\ p\mid D.
  \end{cases}\]
这里 $\leg{0}{p}=0$.

\begin{proof}
易知方程 $x^2\equiv -D\bmod p$ 的解的个数为 $1+\leg{-D}{p}$, 因此由上一题结论知
  \[\sum_{y=0}^{p-1}(1+\leg{y^2-D}{p})=\begin{cases}
    p-1,&\text{如果}\ p\nmid D,\\
    2p-1,\quad&\text{如果}\ p\mid D.
  \end{cases}\]

由于 $4af(x)=(2ax+b)^2-D,$ 因此
  \[\begin{split}
    \sum_{x=0}^{p-1} \leg{f(x)}{p}&=\leg{a}{p}\sum_{x=0}^{p-1} \leg{x^2-D}{p}\\
         &=\begin{cases}
            -\leg{a}{p},&\text{如果}\ p\nmid D,\\
            (p-1)\leg{a}{p},\quad&\text{如果}\ p\mid D.
         \end{cases}\qedhere
  \end{split}\]
\end{proof}

\subsection*{6 补充5}
方程 $x^2-ay^2\equiv n\bmod p$ 在 $\bmod p$ 意义下的解的个数为多少?
\begin{proof}[解]
方程 $x^2-ay^2\equiv n\bmod p$ 的解的个数为
  \[\begin{split}
   \sum_{y=0}^{p-1}(1+\leg{ay^2+n}{p})
   &=p+\leg{a}{p}\sum_{y=0}^{p-1}\leg{y^2+na^{-1}}{p}\\
   &=\begin{cases}
      p-\leg{a}{p},&\text{如果}\ p\nmid D,\\
      p+(p-1)\leg{a}{p},\quad&\text{如果}\ p\mid D.
     \end{cases}
   \qedhere
   \end{split}\]
\end{proof}
\subsection*{6 补充6}
试问方程
  \[a_1x_1^2+\cdots+a_kx_k^2=c\]
在 $\F_p$ 上有多少个解? 这里 $a_i\neq 0$.

\subsection*{6 补充7}
若非零整数 $a$ 对所有素数都不是二次非剩余, 则 $a$ 是平方数.
\begin{proof}
不妨设 $a$ 无平方因子. 由于
  \[\leg{-1}{3}=-1,\quad \leg{\pm2}{5}=-1,\]
故 $a\neq -1,\pm2$. 若 $a$ 存在奇素因子 $p$. 设 $a=\pm 2^\varepsilon pn$, 由中国剩余定理知存在 $m\equiv 1\bmod{8n}$ 且 $m$ 是模 $p$ 的二次剩余. 由狄利克雷定理知存在素数 $q\equiv m\bmod{8pn}$, 于是
  \[\leg{a}{q}=\leg{pn}{q}=\leg{q}{pn}=\leg{m}{pn}=\leg{m}{p}=-1,\]
矛盾! 因此 $a=1$.
\end{proof}

\subsection*{8 补充1}
设 $p$ 是奇素数. 证明: 模 $p$ 的任意两个原根之积不是模 $p$ 的原根.
\begin{proof}
设 $a,b$ 是模 $p$ 的原根, 则 $b=a^e$ 且 $(e,p-1)=1$. 因此 $e$ 为奇数, $e+1$ 为偶数,
  \[(ab)^{\frac{p-1}{2}}=(a^{\frac{e+1}{2}})^{p-1}\equiv 1\bmod p,\]
故 $ab$ 不是原根.
\end{proof}

\subsection*{8 补充2}
设 $p$ 与 $q=2p+1$ 都是素数. 证明

(1) 当 $p\equiv1\pmod4$ 时, $2$ 是模 $q$ 的原根;

(2) 当 $p\equiv3\pmod 4$ 时, $-2$ 是模 $p$ 的原根.

\begin{proof}
(1) 由于 $q\equiv 3\bmod 8$,
  \[2^2=4\not\equiv 1,\quad 2^p=2^{\frac{q-1}{2}}\equiv\leg{2}{q}=-1\bmod q,\]
而 $2$ 的阶整除 $\varphi(q)=2p$, 因此 $2$ 的阶为 $2p=q-1$, 即 $2$ 是模 $q$ 的原根.

(2) 由于 $q\equiv 7\bmod 8$,
  \[(-2)^2=4\not\equiv 1,\quad (-2)^p=(-2)^{\frac{q-1}{2}}\equiv\leg{-2}{q}=-1\bmod q,\]
而 $-2$ 的阶整除 $\varphi(q)=2p$, 因此 $-2$ 的阶为 $2p=q-1$, 即 $-2$ 是模 $q$ 的原根.
\end{proof}

\subsection*{8 补充3}
设 $n,a$ 都是正整数且 $a>1$. 证明 $n\mid \varphi(a^n-1)$.
\begin{proof}
设 $d$ 为 $a$ 模 $a^n-1$ 的阶, 则由
  \[a^n\equiv 1\mod (a^n-1)\]
可知 $d\le n$. 另一方面, 若 $d<n$, 则 $a^d-1<a^n-1$, $a^d\not\equiv 1\mod (a^n-1)$, 因此 $d\ge n$. 故 $d=n$. 再由欧拉定理知 \[n=d\mid \varphi(a^n-1).\qedhere\]
\end{proof}

\subsection*{9}
由题设知 $n$ 是奇数. 令 $d$ 为 $2$ 模 $n$ 的阶, 则 $d\nmid k, d\mid n-1=kp^2$, 因此 $p\mid d$. 而 $d\mid \varphi(n)$, 因此 $p\mid \varphi(n)$, 于是由欧拉函数的公式可知, 存在素数 $q\mid n$ 且 $q\equiv 1\mod p$.

若 $n$ 不是素数, 则存在正整数 $u,v$ 使得
 \[n=kp^2+1=(up+1)(vp+1)=uvp^2+(u+v)p+1,\]
于是 $p\mid u+v$, $u+v\ge p$. 因此 $uv=(u-1)(v-1)+u+v-1\ge p-1$,
  \[k=uv+(u+v)/p\ge p-1+1=p,\]
矛盾! 因此 $n$ 是素数.

\section{多项式之性质}

\subsection*{4.1}
$R=\Z[x_1,\ldots,x_n]$ 的理想为均为有限生成的.
\begin{proof}
假设命题对于 $n<k$ 成立. 对于 $n=k$, 我们令 $\deg$ 表示 $R$ 中多项式关于 $x_k$ 的次数. 设 $I\subseteq R$ 是非零理想. 记 $I$ 中多项式关于 $x_k$ 的首项系数形成的集合为 $J$, 则 $J$ 是 $\Z[x_1,\ldots,x_{k-1}]$ 的理想. 由归纳假设, $J$ 是有限生成的, 不妨设由 $a_1,\ldots,a_m$ 生成. 设对应 $I$ 中的多项式为
  \[f_i=a_i x_k^{d_i}+\cdots.\]
不妨设 $d=d_1\ge d_2\ge \cdots\ge d_m$. 我们断言任一 $f\in I$ 可表为
  \[f=\sum_{i=1}^m q_i f_i+r,\]
其中 $\deg r<d$. 若 $\deg f<d$, 则已经成立. 若不然, $f$ 关于 $x_k$ 的首项系数可表为
  \[\sum_{i=1}^m c_i a_i,\quad c_i\in\Z[x_1,\ldots,x_{k-1}],\]
令
  \[r:=f-\sum_{i=1}^m x_k^{\deg f-d_i} c_i f_i,\]
则 $\deg r<\deg f$. 依此法进行下去, 即可得到该结论.

记 $I$ 中次数小于 $d$ 的元素的首项系数行成的集合为 $J'$, 则 $J'$ 也是有限生成的理想, 记其生成元对应的 $I$ 中多项式为 $g_1,\ldots$, 其 $x_k$ 最高次数为 $d'<d$. 则类似地, $I$ 中任一次数小于 $d$ 的多项式均可表为
  \[r'+q_1' g_1+\cdots\]
且 $\deg r'<d'$. 按此法进行下去, $d,d',\ldots$ 会越来越小, 直至为 $0$. 因此 $I$ 被这些 $f_i,g_j,\ldots$ 生成, 故为有限生成.

又因为 $n=0$ 命题显然成立, 因此由归纳法知原命题成立.
\end{proof}

\subsection*{4.2}
设 $R$ 为所有 $\Q[x]$ 中整值多项式形成的环. 令 $I$ 为所有 ${x\choose k}, k\ge 1$ 形成的理想. 如果 $I$ 是有限生成的, 不妨设 $I$ 由
  \[f_1,\ldots,f_m\]
生成, 且它们的次数为 $d=d_1\ge d_2\ge\cdots d_m$. 则 $I$ 可由
  \[{x\choose 1},{x\choose 2},\ldots,{x\choose d}\]
生成. 对于素数 $p>d$,
  \[{x\choose p}=\sum_{i=1}^d {x\choose i}g_i(x),\quad g_i\in R.\]
令 $x=p$, 则
  \[1=\sum_{i=1}^d {p\choose i} g_i(p),\]
由于 $1\le i\le d< p$, 因此右式是 $p$ 的倍数, 这不可能! 因此 $I$ 不是有限生成的.

\subsection*{4 补充1}
对于 $n\in\Z$, 证明 $x^n+x^{-n}$ 是 $x+x^{-1}$ 的整系数多项式.
\begin{proof}
我们只需对 $n\in\N$ 证明. $n=0,1$ 显然. 若命题对于 $n\le k$ 均成立, 则
  \[x^{k+1}+x^{-k-1}=(x+x^{-1})(x^k+x^{-k})-(x^{k-1}+x^{-k+1})\]
也是 $x+x^{-1}$ 的整系数多项式. 由归纳法知对任意 $n\in\N$, $x^n+x^{-n}$ 是 $x+x^{-1}$ 的整系数多项式.
\end{proof}

\subsection*{4 补充2}
设 $x_1,x_2,x_3$ 是整系数三次方程 $x^3+ax^2+bx+c=0$ 的根. 记 $a_n=x^n_1+x_2^n+x_3^n$. 证明对 $n\in\N$, $a_n$ 是整数.
\begin{proof}
$n=0,1,2$ 时, $a_0=3,a_1=-a,a_2=a^2-2b$ 是整数. 若命题对 $n\le k\le 2$成立, 则
  \[x_i^{k+1}+ax_i^k+bx_i^{k-1}+cx_i^{k-2}=0,\quad k\ge 2,\]
于是 $a_{k+1}=-a a_k-b a_{k-1}-c a_{k-2}$ 是整数. 由归纳法知对任意 $n\in\N$, $a_n$ 是整数.
\end{proof}

\subsection*{4 补充3}
设\ $f(x)\in\Q[x]$ 是一个\ $n$ 次多项式, 满足
  \[f(k)=2^k\ (k=1,2,\cdots,n+1).\]
求\ $f(n+2)$.
\begin{proof}[解]
由 Lagrange 插值公式知
  \[ f(x)=\sum_{k=1}^{n+1}f(k)\prod_{i=1,i\neq k}^{n+1} \frac{x-i}{k-i},\]
于是
  \[\begin{split}
    f(n+2)&=\sum_{k=1}^{n+1}2^k\prod_{i=1,i\neq k}^{n+1} \frac{n+2-i}{k-i}\\
    &=\sum_{k=1}^{n+1}2^k (-1)^{n+1-k}\frac{(n+1)!}{(k-1)!(n+2-k)!}\\
    &=\sum_{k=0}^{n}2^{k+1}(-1)^{n-k}{n+1\choose k}\\
    &=2^{n+2}-2\sum_{k=0}^{n+1}2^{k}(-1)^{n+1-k}{n+1\choose k}\\
    &=2^{n+2}-2\times(2-1)^{n+1}=2^{n+2}-2.\qedhere
  \end{split}\]
\end{proof}

\subsection*{4 补充4}
设\ $f(x)\in\F_p[x],\, \deg f=p-2$. 若对所有\ $\alpha\in\F_p\,(\alpha\neq 0)$ 有\ $f(\alpha)=\alpha^{-1}$, 试确定\ $f(x)$.
\begin{proof}
由题设知 $1,2,\ldots,p-1$ 是 $xf-1$ 的根, 而 $xf-1$ 是 $p-1$ 次多项式, 因此
  \[ xf-1=a \prod_{\alpha=1}^{p-1}(x-\alpha)=a(x^{p-1}-1),\]
而 $xf-1$ 常数项为 $-1$, 因此 $a=1$, $f(x)=x^{p-2}$.
\end{proof}

\subsection*{5 补充}
(1) 求有理系数多项式\ $\alpha(x)$ 和\ $\beta(x)$, 使得
  \[x^3\alpha(x)+(1-x)^2\beta(x)=1.\]

(2) 求有理系数多项式\ $\alpha(x)$ 和\ $\beta(x)$, 使得
  \[x^m\alpha(x)+(1-x)^n\beta(x)=1,\]
其中 $m,n$ 为正整数.

\begin{proof}
利用辗转相除法,
  \[x^3=(x+2)(1-x)^2+3x-2,\]
  \[(1-x)^2=(3x-4)(3x-2)/9+1/9,\]
于是
  \[\begin{split}
  1&=9(1-x)^2-(3x-4)(3x-2)\\
  &=9(1-x)^2-(3x-4)(x^3-(x+2)(1-x)^2)\\
  &=(4-3x)x^3+(3x^2+2x+1)(1-x)^2.\end{split}\]
因此取 $\alpha(x)=4-3x,\beta(x)=3x^2+2x+1$ 即可.

(2) 由
  \[(1-x^m)^n=(1-x)^n(\frac{x^m-1}{x-1})^n=1+\sum_{k=1}^{n}{n\choose k}(-1)^{k}x^{km}\]
知可取
  \[\alpha(x)=\sum_{k=0}^{n-1}{n\choose k+1}(-1)^kx^{km},\beta(x)=(\frac{x^m-1}{x-1})^n.\qedhere\]
\end{proof}

\subsection*{6 补充1}
设 $f(x)\in\R[x]$ 是实系数多项式, $a\in\R$. 假设 $f^{(n)}(a)\neq 0,\forall n$, 试决定 $a$ 在下述多项式的零点重数:
\begin{enumerate}
  \item $f(x)-f(a)-f'(a)(x-a)-\frac{f''(a)}{2}(x-a)^2;$
  \item $f(x)-f(a)-\frac{x-a}{2}(f'(x)+f'(a)).$
\end{enumerate}

\subsection*{6 补充2}
设 $n\ge 2$. 证明 $1$ 是多项式 $x^{2n}-nx^{n+1}+nx^{n-1}-1$ 的 $3$ 重零点.

\subsection*{6 补充3}
证明多项式 $f(x)=\sum\limits_{k=0}^n \dfrac{x^k}{k!}$ 无重根.

\subsection*{9 补充}
证明
  \[f_n(x)=\frac{1}{n}\sum_{d\mid n} \mu(d) x^{n/d}\]
为整值多项式, 其中 $\mu$ 是 Mobi\"us 函数.
\begin{proof}
设 $g_n(x)=nf_n(x)\in\Z[x]$. 当 $n=1$ 时, $f_1(x)=x$ 显然成立. 假设 $f_1,\ldots,f_{n-1}$ 均是整值多项式. 设 $n=p^\alpha n'$, $p\nmid n'$, 则
  \[\begin{split}
    g_n(x)&=\sum_{d\mid n} \mu(d) x^{n/d}\\
          &=\sum_{d\mid n'} \mu(d) x^{p^\alpha n'/d}+\sum_{d'\mid n'} \mu(d'p) x^{p^{\alpha-1} n'/d'}\\
          &=g_{n'}(x^{p^\alpha})-g_{n'}(x^{p^{\alpha-1}}).
    \end{split}\]
而 $x\in\Z$ 时,
\[p^\alpha\mid x^{p^\alpha}-x^{p^{\alpha-1}}\mid g_{n'}(x^{p^\alpha})-g_{n'}(x^{p^{\alpha-1}}),\]
因此 $p^\alpha\mid g_n(x)$. 又由归纳假设 $n'\mid g_{n'}(x)$, 因此 $n\mid g_n(x)$.
\end{proof}

\subsection*{8 补充}
设 $R$ 是含幺交换环. 试定义 Euler 函数并陈述 Euler 定理, Wilson 定理, 二次剩余.
\begin{proof}[解]
设 $I\subseteq R$ 为一理想. 定义
  \[\varphi(I)=|(R/I)^\times|\in\N\cup\{\infty\}.\]
当 $\varphi(I)$ 有限时, 若 $a\in R$ 在 $R/I$ 中的像 $\bar a$ 可逆, 则
  \[a^{\varphi(I)}\equiv 1\mod I.\]
当 $I$ 为极大理想且 $\varphi(I)$ 有限时, $R/I$ 为有限域, 于是
  \[\prod_{0\neq x\in R/I}x=-1. \]
此时若 $\varphi(I)$ 为偶数, 则 $(R/I)^\times$ 为 $\varphi(I)$ 阶循环群, 不妨设 $a$ 为一生成元(原根), 则 $(R/I)^\times$ 中的平方元一定具有形式 $a^{2k}$, 这意味着 $a^{\varphi(I)/2}=1$; $(R/I)^\times$ 中的非平方元一定具有形式 $a^{2k+1}$, 这意味着 $a^{\varphi(I)/2}=-1$.
\end{proof}

令 $R=\Z[x]$, $I=(p,\alpha(x))$, 其中 $p$ 为素数, $\bar\alpha=\alpha\mod p$ 为 $\F_p[x]$ 中 $n$ 次多项式. 则
  \[R/I\cong \F_p[x]/(\bar \alpha)\cong \prod_i \F_{p^{n_i}}\]
为整环, 其中 $n_i$ 为 $\bar\alpha$ 的各个不可约多项式因子 $\beta_i$ 的次数. 故
  \[\varphi(I)=\prod_i (p^{n_i}-1),\]
此即群 $(R/I)^\times$ 的阶, 由此可得重模的 Euler 定理:

设 $f(x)\in\Z[x]$ 满足 $\bar f$ 与 $\bar \alpha$ 在 $\F_p[x]$ 中互素, 则
  \[f(x)^{\varphi((p,\alpha))}\equiv 1\bmod \alpha.\]

\subsection*{8}
设 $\psi(x)$ 及 $\varphi(x)$ 都是 $\mod p$ 不可约多项式. 证明重模 $(p,\varphi(x))$ 方程 $\psi(X)\equiv 0$ 有解当且仅当 $\deg \psi\mid \deg \varphi$, 且此时 $\psi(x)$ 可分解为一次因子之积.
\begin{proof}
我们固定 $\Z[x]/(p,\varphi(x))\cong \F_{p^n}$. 由于 $\psi(x)$ 在 $\F_p$ 上不可约, 由有限域基本理论知其在 $\F_{p^n}$ 上可解当且仅当  $m=\deg \psi\mid n$, 且此时所有根均落在 $\F_{p^m}$ 上.
\end{proof}

\subsection*{10 补充}
若 $f(x)$ 对重模 $(p,\varphi(x)$ 的次数为 $\ell$, 则 $\ell\mid p^n-1$, 其中 $n$ 为 $\varphi(x)$ 的次数.
\begin{proof}
由有限循环群的结构可知.
\end{proof}

\section{素数分布之概况}
\subsection*{4.1}
若不然, 设 $p_1,\ldots,p_m$ 为所有 $6n-1$ 型素数. 令 $N=6p_1\cdots p_m-1$, 则 $2,3,p_i$ 均不整除 $N$, 因此 $N$ 只含有 $6n+1$ 型素数. 但是 $6n+1$ 型素数乘积一定为 $6n+1$ 型, 这与 $N$ 是 $6n-1$  型矛盾! 因此有无穷多 $6n-1$ 型素数.

\subsection*{4.2}
若不然, 设 $p_1,\ldots,p_m$ 为所有 $4n-1$ 型素数. 令 $N=4p_1\cdots p_m-1$, 则 $2,p_i$ 均不整除 $N$, 因此 $N$ 只含有 $4n+1$ 型素数. 但是 $4n+1$ 型素数乘积一定为 $4n+1$ 型, 这与 $N$ 是 $4n-1$  型矛盾! 因此有无穷多 $4n-1$ 型素数.

\subsection*{4.3}
由
  \[\begin{split}
    \frac{\pi^2}{6}&=\sum_{n=1}^\infty \frac{1}{n^2}\\
                   &=\prod_p (1+p^{-2}+p^{-4}+\cdots)\\
                   &=\prod_p (1-p^{-2})^{-2}\\
                   &=\prod_p \frac{p^2}{p^2-1}
  \end{split}\]
可得.

\begin*{remark}
此即 Riemann $\zeta$ 函数在 $2$ 处的取值, 一般地
  \[\zeta (2n)={\frac {(-1)^{n+1}B_{2n}(2\pi )^{2n}}{2(2n)!}},\quad n\ge 1,\]
这里 $B_n$ 是 Bernoulli 数, 即
  \[ \frac {t}{e^{t}-1}=\sum _{m=0}^{\infty} B_{m}{\frac {t^{m}}{m!}}.\]
\end*{remark}

\subsection*{7}
令 $x=\sqrt{2n}>0$, 原式等价于
  \[\begin{split}
    2^{x^2/3}&<x^{2(x+1)}\\
    \frac{1}{6}x^2 \ln 2&<(x+1)\ln x.
    \end{split}\]
令 $f(x)=(x+1)\ln x-\frac{1}{6}x^2 \ln 2$, 则
  \[\begin{split}
    f'(x)&=\ln x+\frac{x+1}{x}-\frac{\ln 2}{3}x,\\
    f''(x)&=\frac{1}{x}-\frac{1}{x^2}-\frac{\ln 2}{3},\\
    f'''(x)&=-\frac{1}{x^2}+\frac{2}{x^3}.
  \end{split}\]
由于 $x<2$ 时 $f'''(x)>0$; $x>2$ 时 $f'''(x)<0$, 因此 $f''(x)$ 在 $(0,2)$ 上单调增, 在 $(2,+\infty)$ 上单调减.

由
  \[f''(1)=-\frac{\ln 2}{3}<0,\ f''(2)=\frac{1}{4}-\frac{\ln 2}{3}>0,\ f''(3)=\frac{2}{9}-\frac{\ln 2}{3}<0\]
知 $f''(x)$ 在 $(0,+\infty)$ 上恰有两个零点 $1<x_1<2<x_2<3$. 因此 $f'(x)$ 在 $(0,x_1)$ 和 $(x_2,+\infty)$ 上单调减; 在 $(x_1,x_2)$ 上单调增.

由于 $0<x<3$ 时
  \[f'(x)\ge \frac{4}{3}-\ln 2>0,\]
因此 $f'(x)$ 恰有一个零点 $x_3>3$. 故 $f(x)$ 在 $(0,x_3)$ 上单调增, 在 $(x_3,+\infty)$ 上单调减.

由于
  \[f(\sqrt{2})=\frac{3\sqrt{2}+1}{6}\ln 2>0,\]
  \[ f(\sqrt{2\times467})\approx 0.032>0,\ f(\sqrt{2\times 468})\approx-0.054<0,\]
因此当且仅当 $1\le n \le 467$ 时, $f(\sqrt{2n})>0$.

\subsection*{8.1}
分别将 $\xi$ 和 $\xi+1$ 代入并相减得
  \[(\xi+1)^\lambda=\frac{(\xi+1)^{\lambda+1} -\xi^{\lambda+1}}{\lambda+1} +c((\xi+1)^\lambda-\xi^\lambda)+O(\xi^{\lambda-2}),\]
化简可得 $c=1/2$.

\subsection*{8.2}
由
  \[\int \log\log x\ud x=x\log\log x-\li x\]
知
  \[\begin{split}&\sum_{3\le 3\le \xi} \log \log n=\int_3^\xi \log\log x\ud x+O(\log\log \xi)\\
  =&\xi\log\log\xi-\li\xi+O(\log\log\xi) =\xi\log\log\xi+O(\frac{\xi}{\log\xi}).  \end{split}\]

\subsection*{8.1'}
令 $f(x)=\frac{\log x}{x}$, 则
  \[f'(x)=\frac{1-\log x}{x^2},\]
因此 $x\ge 3$ 时 $f'(x)<0$, $f(x)$ 单调减且趋于$0$. 由定理2知
  \[\lim_{N\rightarrow \infty} (\sum_{n=3}^N f(n)-\int_3^N f(x)\ud x)=\alpha\]
存在, 且
  \[|\sum_{3\le n\le \xi} f(n)-\int_3^\xi f(x)\ud x-\alpha|\le f(\xi-1)=O(f(\xi)),\]
即
  \[\sum_{3\le n\le \xi} f(n)=\int_3^\xi f(x)\ud x+\alpha+O(f(\xi)).\]
由 $\int f(x)\ud x=\frac{1}{2}\log^2 x$ 知
  \[\sum_{1\le n\le \xi}\frac{\log n}{n}=\frac{1}{2}\log^2 \xi-\frac{1}{2}\log^2 3+\frac{\log 2}{2}+\alpha+O(\frac{\log \xi}{\xi}).\]


\subsection*{8.2'}
令 $f(x)=\frac{1}{x\log x}$, 则
  \[f'(x)=-\frac{1+\log x}{(x\log x)^2},\]
因此 $x\ge 2$ 时 $f'(x)<0$, $f(x)$ 单调减且趋于$0$. 由定理2知
  \[\lim_{N\rightarrow \infty} (\sum_{n=2}^N f(n)-\int_2^N f(x)\ud x)=\alpha\]
存在, 且
  \[|\sum_{2\le n\le \xi} f(n)-\int_2^\xi f(x)\ud x-\alpha|\le f(\xi-1)=O(f(\xi)),\]
即
  \[\sum_{2\le n\le \xi} f(n)=\int_2^\xi f(x)\ud x+\alpha+O(f(\xi)).\]
由 $\int f(x)\ud x=\log\log x$ 知
  \[\sum_{2\le n\le \xi} \frac{1}{n\log n}=\log\log \xi-\log\log2+\alpha+O(\frac{1}{\xi\log\xi}).\]

\subsection*{9.1}
由 $\Cheb$ 定理知
  \[\frac{1}{8}\le \frac{n\log p_n}{p_n}\le 12,\]
因此
  \[ \frac{p_n}{n\log n}> \frac{p_n}{n\log p_n}\ge c_1=\frac{1}{12}.\]
由于 $\log p_n< 2\sqrt{p_n}$,
  \[\frac{1}{8}\le \frac{n\log p_n}{p_n}< \frac{2n}{\sqrt{p_n}},\]
  \[\log n>\frac{1}{2}\log p_n-\log 16,\]
  \[\frac{n\log n}{p_n}>\frac{1}{2}\frac{n\log p_n}{p_n}-\frac{n\log 16}{p_n}
  \ge \frac{1}{16}-\frac{12\log 16}{\log p_n}.\]
设素数 $p_k>16^{192}$, 则 $n\ge k$ 时上式 $\ge \frac{\log (p_k/16^{192})}{16\log p_k}>0$.

令 \[c_0=\min\{\frac{2\log 2}{p_2},\ldots,\frac{(k-1)\log (k-1)}{p_{k-1}},\frac{\log (p_k/16^{192})}{16\log p_k}\},\]
$c_2=2/c_0$, 则
  \[ c_1 n\log n<p_n< c_2 n\log n.\]

\subsection*{9.2}
设 $n=\prod\limits_{i=1}^k q_i^{e_i}$, 令
  \[f(n)=\frac{\varphi(n)\log\log n}{n}=\log(\sum e_i\log q_i)\prod (1-\frac{1}{q_i}).\]
设 $k\ge 2$, 则 $n\ge 3$. 记 $p_i$ 为第 $i$ 个素数, 则
  \[f(n)\ge \log(\sum \log p_i)\prod (1-\frac{1}{p_i}).\]
由 Stirling 公式, 存在 $c_0>0$ 使得 $n!>c_0 (n/e)^n$, 于是
  \[p_1\cdots p_k> k!>c_0(k/e)^k,\]
  \[f(n)\ge \log (k\log(k/e)+\log c_0)\prod (1-\frac{1}{p_i}).\]
当 $n$ 充分大时, 存在 $c_1>0$ 使得 $f(n)\ge c_1\log k/\log p_k\ge c>0.$

$k=1$ 时易得 $f(n)\ge \min\{f(3),f(4)\}$.

\subsection*{9.3}
由
  \[\sum_p \frac{1}{p(\log\log p)^h}\]
和
  \[ \sum_{n\ge 1} \frac{1}{n\log n(\log \log n)^h}\]
相互控制可得, 而它的敛散性和
  \[\int_a^{+\infty} \frac{\ud x}{x\log x(\log \log x)^h}=\int_{\log\log a}^{+\infty} \frac{\ud t}{t^h}\]
相同.

\subsection*{11.1}
设 $c\in\Z$ 满足 $|f(c)|\ge 2$, 设
  \[g(x)=f(x+c)=a_nx^n+\cdots+a_1x+f(c),\]
则
  \[g(mf(c))=(\sum_{i=1}^n a_i m^i f(c)^{i-1} +1)f(c),\]
当 $m$ 充分大时 $f(mf(c)+c)=g(mf(c))$ 是合数.

\subsection*{11.2}
设
  \[f(n)=c_1(n)+c_2(n)2^n+\cdots+c_m(n)m^n.\]
若不然, 存在 $f(a)=p>m$. 由
   \[f(a+p(p-1)t)\equiv f(a)\bmod p\]
知 $p\mid f(a+p(p-1)t)$. 由 $n\rightarrow\infty$ 时 $f(n)\rightarrow\infty$ 可知存在无穷多 $t$ 使得 $f(a+p(p-1)t)$ 是复合数.

\subsection*{12}
若素数 $p\mid x^2+y^2,xy\neq0$, 则 $-1\equiv(x/y)^2\bmod p$, 因此 $\leg{-1}{p}=1, p\equiv 1\bmod 4$.

若只有有限个 $8n+5$ 型素数, 设为 $p_1,\ldots,p_k$, 令
  \[q=(p_1\cdots p_k)^2+2^2,\]
则 $p_i\nmid q$, 因此 $q$ 只含 $8n+1$ 型素因子, 从而 $q\equiv 1\bmod 8$, 这与  $q\equiv 5\bmod 8$ 矛盾.

\section{数论函数}
\subsection*{4.1}
由第二章5补充3知
  \[g*f_1=(f*E_0)*f_1=f*(f_1*E_0)=f*g_1.\]

\subsection*{4.2}
由于
  \[g(e)g_1(e)=\sum_{d,d_1\mid e}f(d)f_1(d_1),\]
因此 $gg_1$ 的 M\"obius 变换为
  \[\begin{split}
    h(n)&=\sum_{d,d_1\mid e\mid n} f(d)f_1(d_1)\mu(\frac{n}{e})\\
        &=\sum_{d,d_1\mid n} f(d)f_1(d_1)\sum_{f\mid \frac{n}{[d,d_1]}}\mu(\frac{n}{f[d,d_1]})\\
        &=\sum_{[d,d_1]=n} f(d)f_1(d_1)
  \end{split}\]

\subsection*{4.3}
由 \[(E_0*E_0)(n)=\sum_{d\mid n}E_0(d)E_0(\frac{n}{d})=d(n)\] 立得.

\subsection*{5.1}
我们有
  \[\sum_{1\le n\le \xi} \frac{d(n)}{n}=\sum_{1\le n\le \xi}\sum_{u\mid n}\frac{1}{n}=\sum_{1\le uv\le \xi}\frac{1}{uv}.\]
该区域可分为 $(0,\sqrt{\xi}]^2$ 和其余两块, 于是
  \[\begin{split}
    &\sum_{1\le uv\le \xi}(uv)^{-1}\\
   =&(\sum_{1\le u\le\sqrt{\xi}}u^{-1})^2+2\sum_{1\le u\le\sqrt{\xi}}u^{-1} \sum_{\sqrt{\xi}< v\le\xi/u}v^{-1}\\
   =&-(\sum_{1\le u\le\sqrt{\xi}}u^{-1})^2+2\sum_{1\le u\le\sqrt{\xi}}u^{-1} \sum_{1\le v\le\xi/u}v^{-1}\\
   =&\sum_{1\le u\le\sqrt{\xi}}u^{-1}(2 \sum_{1\le v\le\xi/u}v^{-1}-\sum_{1\le v\le\sqrt{\xi}}v^{-1})\\
   =&\sum_{1\le u\le\sqrt{\xi}}u^{-1}(2\log \xi-2\log u+2\gamma-\frac{1}{2}\log \xi-\gamma+O(\xi^{-\frac{1}{2}}))\\
   =&-2\sum_{1\le u\le\sqrt{\xi}}\frac{\log u}{u}+(\log \sqrt{\xi}+\gamma+O(\xi^{-\frac{1}{2}}))(\frac{3}{2}\log \xi+\gamma+O(\xi^{-\frac{1}{2}}))\\
   =&-2(\frac{1}{2}\log^2\sqrt{\xi}+c_1+O(\xi^{-\frac{1}{2}}\log \xi))+\frac{3}{4}\log^2\xi+2\gamma\log \xi+O(\xi^{-\frac{1}{2}}\log\xi)\\
   =&\frac{1}{2}\log^2\xi+2\gamma\log \xi+c+O(\xi^{-\frac{1}{2}}\log\xi).
  \end{split}\]

\subsection*{5.2}
由
  \[\sigma(n)=\prod \frac{p^{e+1}-1}{p-1}=\prod O(p^{e(1+\varepsilon)})=n^{1+\varepsilon}\]
可得.

\subsection*{5.3}
令 \[b_v=\sum_{1\le u\le \xi/v} u=\frac{\xi^2}{2v^2}+(1-2\lambda_v)\frac{\xi}{v}+\lambda_v(\lambda_v-1),\]
其中 $\lambda_v=\frac{\xi}{v}-[\frac{\xi}{v}]$, 则
\[\begin{split}
   &\sum_{1\le n\le \xi}\sigma(n) =\sum_{1\le n\le \xi}\sum_{u\mid n}u=\sum_{1\le uv\le \xi} u=\sum_{1\le u\le \xi} u[\frac{\xi}{u}]\\
  =&\sum_{1\le v\le \xi}v(b_v-b_{v+1})=\sum_{1\le k\le \xi}b_k-[\xi]b_{[\xi]}\\
  =&\frac{\xi^2}{2}\sum_{1\le k\le \xi}\frac{1}{v^2}+O(\xi\log\xi)=\frac{\xi^2\pi^2}{12}+O(\xi\log\xi).
\end{split}\]

\subsection*{9.1}
设椭圆的长轴和短轴长为 $2a,2b$, 则由定理2及 $A=\pi ab$, $l\le 2\pi(a+b)$ 得 $N=\pi ab+O(a+b)$.

\subsection*{9.2}
该数为
  \[\begin{split}
     &\sum_{w^2\le x} (\pi (x-w^2)+O(\sqrt{x-w^2}))\\
    =&\pi x(2[\sqrt{x}]+1)-\pi \frac{[\sqrt{x}]([\sqrt{x}]+1)(2[\sqrt{x}]+1)}{3}+O(x)\\
    =&\frac{4}{3}\pi x^{3/2}+O(x).
  \end{split}\]

\subsection*{9.3}
假设 $n$ 维球 $x_1^2+\cdots+x_n^2\le x$ 内整点个数为 \[f_n=c_n x^{\frac{n}{2}}+O(x^{\frac{n-1}{2}}),\]
则
  \[\begin{split}
    f_{n+1} =&\sum_{w^2\le x} (c_n(x-w^2)^{\frac{n}{2}}+O((x-w^2)^{\frac{n-1}{2}}))\\
    =&2c_nd_n  x^{\frac{n+1}{2}}+O(x^{\frac{n}{2}}),
  \end{split}\]
其中
  \[\begin{split}
    d_n&=1-{n/2\choose 1}\frac{1}{3}+{n/2\choose 2}\frac{1}{5}-{n/2\choose 3}\frac{1}{7}+\cdots\\
    &=\int_0^1 (1-x^2)^{n/2}\ud x=\frac{\sqrt{\pi} \Gamma(\frac{n+2}{2})}{2\Gamma(\frac{n+3}{2})},
  \end{split}\]
因此由数学归纳法
  \[c_n=\pi \prod_{k=2}^{k-1} \frac{\sqrt{\pi}\Gamma(\frac{k+2}{2})}{\Gamma(\frac{k+3}{2})}=\frac{\pi^{\frac{n}{2}}}{\Gamma(\frac{n}{2}+1)},\]
   \[f_n=\frac{(\pi x)^{\frac{n}{2}}}{\Gamma(\frac{n}{2}+1)}+O(x^{\frac{n-1}{2}}).\]
\begin{remark}
由此可知半径为 $r$ 的 $n$ 维球的体积为 $\frac{\pi^{n/2}r^n}{\Gamma(\frac{n}{2}+1)}$.
\end{remark}

\subsection*{9.4}
我们有
\[\begin{split}
  \sum_{1\le n\le x} r^2(x)&=16\sum_{1\le n\le x\atop d_1,d_2\mid n} \chi(d_1d_2)\\
  &=16\sum_{1\le d_1,d_2\le x}\chi(d_1d_2)[\frac{x}{[d_1,d_2]}]\\
  &=16\sum_{1\le s\le x}\sum_{1\le u,v\le x/s\atop (u,v)=1} \chi(s^2uv)[\frac{x}{suv}]\\
  &=16\sum_{1\le s\le x}\chi(s^2) \sum_{1\le t\le x/s} \chi(t)[\frac{x}{st}]d(t)\\
  &=16\sum_{1\le s\le x}\chi(s^2)f(x/s)
\end{split}\]
其中
\[\begin{split}
  f(x)&=\sum_{1\le t\le x} \chi(t)[\frac{x}{t}]d(t)\\
  &=x\prod_{1\le p\le x}(1-\chi(p)p^{-1})^{-2}+O(\log x)\\
  &=x(\prod_{n\ge 1}\frac{\chi(n)}{n})^2+O(\log x)\\
  &=\frac{\pi^2 x}{16}+O(\log x),
\end{split}\]
因此
\[\begin{split}
  \sum_{1\le n\le x} r^2(x)&=16\sum_{1\le s\le x}\chi(s^2)f(x/s)\\
  &=\pi^2 x\sum_{1\le t\le \frac{x+1}{2}}\frac{1}{2t-1}+O(\log x)\\
  &=\pi^2 x(\log x-\frac{1}{2}\log x)+O(x)\\
  &=\frac{\pi^2 x\log x}{2}+O(x)
\end{split}\]

\subsection*{9.5}
令 \[s(x)=\sum_{1\le n\le x}r(n),\]
则要求的数为
 \[\begin{split}
    &\sum_{1\le d\le \sqrt{x}} s(x/d^2)\mu(d)\\
   =&\sum_{1\le d\le \sqrt{x}} (\frac{\pi x}{d^2}+O(\frac{\sqrt{x}}{d}))\mu(d)\\
   =&\pi x(\prod_{1\le p\le \sqrt{x}}(1-p^{-2}))+O(\sqrt{x}\prod_{1\le p\le \sqrt{x}}(1-p^{-1}))\\
   =&\frac{6}{\pi} x+O(\sqrt{x}\log x).
 \end{split}\]
  






\vspace{0.4cm}
\begin{thebibliography}{999}
\bibitem[冯]{FY}
冯克勤, 余红兵, \emph{整数与多项式}, 高等教育出版社, 施普林格出版社, 1999

\bibitem[华]{H}
华罗庚, \emph{华罗庚文集 数论卷II}, 科学出版社, 2010

\bibitem[欧阳]{O}
欧阳毅, \emph{代数学基础}, 中国科学技术大学
\end{thebibliography}
\end{document}
