\section{初等函数}

\subsection{指数函数}
\begin{frame}{指数函数}
	\onslide<+->
	我们将实变函数中的初等函数推广到复变函数.
	\onslide<+->
	多项式函数和有理函数的解析性质已经介绍过, 这里不再重复.
	\onslide<+->
	现在我们来定义指数函数.

	\onslide<+->
	指数函数有多种等价的定义方式:
	\begin{enumerate}
		\item $\exp z=e^x(\cos y+i\sin y)$ (欧拉恒等式);
		\item $\exp z=\lim\limits_{n\to\infty}\left(1+\dfrac zn\right)^n$ (极限定义);
		\item $\exp z=1+z+\dfrac{z^2}{2!}+\dfrac{z^3}{3!}+\cdots
		=\lim\limits_{n\to\infty}\sum\limits_{k=0}^n\dfrac{z^k}{k!}$ (级数定义);
		\item $\exp z$ 是唯一的一个处处解析的函数, 使得当 $z=x\in\BR$ 时, $\exp z=e^x$ ($e^x$ 的解析延拓).
	\end{enumerate}
\end{frame}


\begin{frame}{指数函数}
	\onslide<+->
	有些人会从 $e^x,\cos x,\sin x$ 的泰勒展开
	\begin{align*}
		e^x&=1+x+\frac{x^2}{2!}+\frac{x^3}{3!}+\cdots\\
		\cos x&=1-\frac{x^2}{2!}+\frac{x^4}{4!}+\cdots\\
		\sin x&=x-\frac{x^3}{3!}+\frac{x^5}{5!}\cdots
	\end{align*}
	形式地带入得到欧拉恒等式 $e^{ix}=\cos x+i\sin x$.
	\onslide<+->
	事实上我们可以把它当做复指数函数的定义, 而不是欧拉恒等式的证明.
	\onslide<+->
	我们将在第四章说明\enumnum1、\enumnum3和\enumnum4是等价的.
\end{frame}



\begin{frame}{指数函数的定义}
	\onslide<+->
	我们来证明\enumnum1和\enumnum2等价.
	\onslide<+->
	\begin{align*}
		\lim_{n\to\infty}\abs{1+\frac zn}^n
		&=\lim_{n\to\infty}\left(1+\frac{2x}n+\frac{x^2+y^2}{n^2}\right)^{\frac n2}\quad
		\visible<+->{(1^\infty\ \text{型不定式})}\\
		&\visible<+->{=\exp\left[\lim_{n\to\infty}\frac n2
		\left(\frac{2x}n+\frac{x^2+y^2}{n^2}\right)\right]=e^x.}
	\end{align*}
	\onslide<+->
	不妨设 $n>\abs{z}$, 这样 $1+\dfrac zn$ 落在右半平面,
	\onslide<+->
	\[\lim_{n\to\infty} n\arg{\left(1+\frac zn\right)}
	=\lim_{n\to\infty} n\arctan \frac y{n+x}
	\visible<+->{=\lim_{n\to\infty}\frac{ny}{n+x}=y.}\]
	\vspace{-\baselineskip}
	\onslide<+->
	故 $\exp z=e^x(\cos y+i\sin y)$.
\end{frame}


\begin{frame}{指数函数的性质}
	\onslide<+->
	\begin{third}{指数函数}
		\[\exp z:=e^x(\cos y+i\sin y).\]
	\end{third}
	\onslide<+->
	为了方便, 我们也记 $\emphm{e^z=\exp z}$.
	\onslide<+->
	指数函数有如下性质:
	\begin{itemize}\bf
		\item $\exp z$ 处处解析, 且 $(\exp z)'=\exp z$.
		\item $\exp z\neq 0$.
		\item $\exp(z_1+z_2)=\exp z_1\cdot \exp z_2$.
		\item $\exp(z+2k\pi i)=\exp z$, 即 $\exp z$ 周期为 $2\pi i$.
		\item $\exp z_1=\exp z_2$ 当且仅当 $z_1=z_2+2k\pi i,k\in\BZ$.
		\item 指数函数将直线族 $\Re z=c$ 映为圆周族 $\abs{w}=e^c$, 将直线族 $\Im z=c$ 映为射线族 $\Arg w=c$.
	\end{itemize}
\end{frame}


\begin{frame}{指数函数的性质}
	\onslide<+->
	\begin{example}
		计算函数 $f(z)=\exp(z/6)$ 的周期.
	\end{example}

	\onslide<+->
	\begin{solution}
		设 $f(z_1)=f(z_2)$, 则 $e^{z_1/6}=e^{z_2/6}$.
		\onslide<+->{%
			因此存在 $k\in\BZ$ 使得
			\[\frac{z_1}6=\frac{z_2}6+2k\pi i,\]
		}\onslide<+->{%
			从而 $z_1-z_2=12k\pi i$.
		}\onslide<+->{%
			所以 $f(z)$ 的周期是 $12\pi i$.
		}
	\end{solution}

	\onslide<+->
	一般地, $\exp(az+b)$ 的周期是 $\dfrac{2\pi i}a$ (或写成 $-\dfrac{2\pi i}a$), $a\neq 0$.
\end{frame}


\subsection{对数函数}

\begin{frame}{对数函数}
	\onslide<+->
	对数函数 $\Ln z$ 定义为指数函数 $\exp z$ 的反函数.
	\onslide<+->
	为什么我们用大写的 $\Ln$ 呢? 
	\onslide<+->
	在复变函数中, 很多函数是多值函数.
	\onslide<+->
	为了便于研究, 我们会固定它的一个单值分支.
	\onslide<+->
	我们将多值的这个开头字母大写, 而对应的单值的则是开头字母小写.
	\onslide<+->
	例如 $\Arg z$ 和 $\arg z$.
\end{frame}


\begin{frame}{对数函数及其主值}
	\onslide<+->
	设 $z\neq0,e^w=z=re^{i\theta}=e^{\ln r+i\theta}$,
	\onslide<+->
	则
	\[w=\ln r+i\theta+2k\pi i,\quad k\in\BZ.\]
	\vspace{-\baselineskip}
	\onslide<+->
	\begin{third}{对数函数}
		\begin{enumerate}
			\item 定义\emph{对数函数}
				\[\Ln z=\ln\abs{z}+i\Arg z.\]
				它是一个多值函数.
			\item 定义\emph{对数函数主值}
				\[\ln z=\ln\abs{z}+i\arg z.\]
				\vspace{-\baselineskip}
		\end{enumerate}
	\end{third}

	\onslide<+->
	对于每一个整数 $k$, $\ln z+2k\pi i$ 都给出了 $\Ln z$ 的一个单值分支.
	\onslide<+->
	特别地, 当 $z=x>0$ 是正实数时, $\ln z$ 就是实变的对数函数.
\end{frame}


\begin{frame}{例: 对数函数的计算}
	\onslide<+->
	\begin{example}
		求 $\Ln 2,\Ln(-1)$ 以及它们的主值.
	\end{example}

	\onslide<+->
	\begin{solution}
		\begin{enumerate}
			\item
			\[\Ln2=\ln2+2k\pi i, k\in\BZ,\]
			\onslide<+->{%
				主值为 $\ln 2$.
			}
			\item
			\[\Ln(-1)=\ln1+i\Arg(-1)=(2k+1)\pi i, k\in\BZ,\]
			\onslide<+->{%
				主值为 $\pi i$.
			}
		\end{enumerate}
	\end{solution}
\end{frame}


\begin{frame}{例: 对数函数的计算}
	\beqskip{5pt}
	\onslide<+->
	\begin{example}
	求 $\Ln(-2+3i),\Ln(3-\sqrt3 i)$.
	\end{example}

	\onslide<+->
	\begin{solution}
		\begin{enumerate}
			\item
			\begin{align*}
				\Ln(-2+3i)&=\ln\abs{-2+3i}+i\Arg(-2+3i)\\
				&\visible<+->{=\frac 12\ln 13+\left(-\arctan\frac 32+\pi+2k\pi\right)i,\quad k\in\BZ.}
			\end{align*}
			\item
			\begin{align*}
				\Ln(3-\sqrt3i)&=\ln\abs{3+\sqrt 3i}+i\Arg(3-\sqrt 3i)\\
				&\visible<+->{=\ln 2\sqrt 3+\left(-\frac\pi6+2k\pi\right)i
				=\ln 2\sqrt 3+\left(2k-\frac16\right)\pi i,\quad k\in\BZ.}
			\end{align*}
		\end{enumerate}
		\vspace{-.5\baselineskip}
	\end{solution}
	\endgroup
\end{frame}


\begin{frame}{例: 对数函数的计算}
	\onslide<+->
	\begin{example}
		解方程 $e^z-1-\sqrt 3i=0$.
	\end{example}

	\onslide<+->
	\begin{solution}
		由于 $1+\sqrt 3 i=2e^{\frac{\pi i}3}$,
		\onslide<+->{因此
		\[z=\Ln(1+\sqrt 3i)=\ln 2+\left(2k+\frac13\right)\pi i,\quad k\in\BZ.\]}
		\vspace{-\baselineskip}
	\end{solution}

	\onslide<+->
	\begin{exercise}
		求 $\ln(-1-\sqrt3 i)=$\fillblankframe[2cm][3mm]{$\ln 2-\dfrac{2\pi i}3$}.
	\end{exercise}
\end{frame}


\begin{frame}{对数函数的性质}
	\onslide<+->
	对数函数与其主值的关系是
	\[\emphn{\Ln z=\ln z+\Ln 1=\ln z+2k\pi i,\quad k\in\BZ}.\]

	\onslide<+->
	根据辐角以及辐角主值的相应等式, 我们有
	\[\Ln(z_1\cdot z_2)=\Ln z_1+\Ln z_2,\quad
	\Ln\frac{z_1}{z_2}=\Ln z_1-\Ln z_2,\]
	\[\Ln \sqrt[n]z=\dfrac1n\Ln z.\]
	\onslide<+->
	而当 $\abs{n}\ge 2$ 时, \alert{$\Ln z^n=n\Ln z$ 不成立}.
	\onslide<+->
	以上等式换成 $\ln z$ 均不一定成立.
\end{frame}


\begin{frame}{对数函数的导数}
\beqskip{1pt}
	\onslide<+->
	设 $x$ 是正实数, 
	\onslide<+->
	则
	\[\ln (-x)=\ln x+\pi i,\quad
	\lim_{y\to0^-}\ln (-x+yi)=\ln x-\pi i,\]
	\onslide<+->
	因此 $\ln z$ 在负实轴和零处不连续.

	\onslide<+->
	而在其它地方 $-\pi<\arg z<\pi$, $\ln z$ 是 $e^z$ 在区域 $-\pi<\Im z<\pi$ 上的单值反函数, 
	\onslide<+->
	从而
	\alert{$(\ln z)'=\dfrac 1z$},
	\alert{$\ln z$ 在除负实轴和零处的区域解析}.

	\onslide<+->
	也可以通过C-R方程来得到 $\ln z$ 的解析性和导数: 当 $x>0$ 时,
	\[\ln z=\half \ln(x^2+y^2)+i\arctan \frac yx,\]
	\onslide<+->
	\[u_x=v_y=\frac x{x^2+y^2},\qquad v_x=-u_y=-\frac y{x^2+y^2},\]
	\[(\ln z)'=(x-yi)/(x^2+y^2)=\frac 1z.\]
	其它情形可取虚部为 $\arccot\dfrac xy$ 或 $\arccot\dfrac xy-\pi$ 类似证明.
\endgroup
\end{frame}


\subsection{幂函数}

\begin{frame}{幂函数的性质: $a$ 为整数时}
	\onslide<+->
	\begin{third}{幂函数}
		\begin{enumerate}
			\item 设 $a\neq 0$, $z\neq 0$, 定义\emph{幂函数}
			\[w=z^a=e^{a\Ln z}
			=\exp\bigl[a\ln\abs{z}+ia(\arg z+2k\pi)\bigr],\quad k\in\BZ.\]
			\item \emph{幂函数的主值}为
			\[w=e^{a\ln z}=\exp\bigl(a\ln\abs{z}+ia\arg z\bigr).\]
		\end{enumerate}
		\vspace{-\baselineskip}
	\end{third}
	\onslide<+->
	根据 $a$ 的不同, 这个函数有着不同的性质.

	\onslide<+->
	当 $a$ 为整数时, 因为 $e^{2ak\pi i}=1$, 所以 $w=z^a$ 是单值的.
	\onslide<+->
	此时 $z^a$ 就是我们之前定义的乘幂.

	\onslide<+->
	当 $a$ 是非负整数时, $z^a$ 在复平面上解析;
	\onslide<+->
	当 $a$ 是负整数时, $z^a$ 在 $\BC-\set0$ 上解析.
\end{frame}


\begin{frame}{幂函数的性质: $a$ 为分数时}
	\onslide<+->
	当 $a=\dfrac pq$ 为分数, $p,q$ 为互质的整数且 $q>1$ 时,
	\onslide<+->
	\[z^{\frac pq}=\abs{z}^{\frac pq}\exp\left[\frac{ip(\arg z+2k\pi)}q\right],\quad k=0,1,\dots,q-1\]
	具有 $q$ 个值.
	\onslide<+->
	去掉负实轴和 $0$ 之后, 它的主值 $w=\exp(a\ln z)$ 是处处解析的.
	\onslide<+->
	事实上它就是 $\sqrt[q]{z^p}=(\sqrt[q]z)^p$.
	\onslide<+->
	\begin{center}
		\begin{tikzpicture}[scale=.8]
			\coordinate [label=below left:{$0$}] (O) at (0,0);
			\coordinate [label=below:{$x$}] (X) at (2,0);
			\coordinate [label=left:{$y$}] (Y) at (0,2);
			\draw[cstaxis] (O)--(X);
			\draw[cstaxis] (0,-2)--(Y);
			\draw[draw=white,cstfille1] (O) circle (1.3);
			\draw[cstdash,main] (0,0)--(-2,0);
			\draw[cstdash,cstra,third] (1.7,.7)to [bend left] (4.5,.7);
			\draw (3,1.7) node[third] {$w=z^{2/9}$};
			\begin{scope}[xshift=5cm]
				\coordinate [label=below left:{$0$}] (O) at (0,0);
				\fill[cstfille2,pattern color=second] (O)--({1.4*cos(40)},{1.4*sin(40)}) arc (40:-40:1.4)--cycle;
				\draw[cstdash,second] ({1.4*cos(40)},{-1.4*sin(40)})--(0,0)--({1.4*cos(40)},{1.4*sin(40)});
				\coordinate [label=below left:{$0$}] (O) at (0,0);
				\coordinate [label=below:{$u$}] (X) at (2,0);
				\coordinate [label=left:{$v$}] (Y) at (0,2);
				\draw[cstaxis] (-2,0)--(X);
				\draw[cstaxis] (0,-2)--(Y);
			\end{scope}
		\end{tikzpicture}
	\end{center}
\end{frame}


\begin{frame}{幂函数的性质: $a$ 为其他情形}
\onslide<+->
	对于其它的 $a$, $z^a$ 具有无穷多个值.
	\onslide<+->
	这是因为此时当 $k\neq0$ 时, $2k\pi a i$ 不可能是 $2\pi i$ 的整数倍. 
	\onslide<+->
	从而不同的 $k$ 得到的是不同的值.
	\onslide<+->
	去掉负实轴和 $0$ 之后,
	\onslide<+->
	它的主值 $w=\exp(a\ln z)$ 也是处处解析的.
	\onslide<+->
	\begin{center}
		\arrayrulecolor{second}
		\begin{tabular}{cccc} \toprule
			$a$& $z^a$ 的值& $z^a$ 的解析区域\\ \midrule
			&&$n\ge0$ 时处处解析\\
			\multirow{-2}*{整数 $n$}&\multirow{-2}*{单值}&$n<0$ 时除零点外解析\\ \midrule
			分数 $p/q$&$q$ 值&除负实轴和零点外解析\\ \midrule
			无理数或虚数&无穷多值&除负实轴和零点外解析\\ \bottomrule
		\end{tabular}
	\end{center}
\end{frame}


\begin{frame}{典型例题: 幂函数的计算}
	\onslide<+->
	\begin{example}
		求 $1^{\sqrt 2}$ 和 $i^i$.
	\end{example}
	\onslide<+->
	\begin{solution}
		\[
			1^{\sqrt2}=e^{\sqrt2\Ln1}
			\visible<+->{=e^{\sqrt 2\cdot 2k\pi i}}
			\visible<+->{=\cos(2\sqrt 2k\pi)+i\sin(2\sqrt 2k\pi), k\in\BZ.}\]
		\onslide<+->{%
		\[
			i^i=e^{i\Ln i}
			\visible<+->{=\exp\left[i\cdot\left(2k+\half\right)\pi i\right]}
			\visible<+->{=\exp\left(-2k\pi-\half\pi\right), k\in\BZ.}\]}
		\vspace{-\baselineskip}
	\end{solution}
	\onslide<+->
	\begin{exercise}
		填空题: $3^i$ 的辐角主值是\fillblankframe{$\ln 3$}.
	\end{exercise}
\end{frame}


\begin{frame}{幂函数的性质}
	幂函数与其主值有如下关系:
	\onslide<+->
	\[\emphn{z^a=e^{a\ln z}\cdot 1^a
	=e^{a\ln z}\cdot e^{2ak\pi i},\quad k\in\BZ.}\]

	\onslide<+->
	对于幂函数的主值,
	\[\alertn{(z^a)'}=\left(e^{a\ln z}\right)'=\frac{ae^{a\ln z}}z=\alert{az^{a-1}}.\]

	\onslide<+->
	一般而言, $z^a\cdot z^b=z^{a+b}$ 和 $(z^a)^b=z^{ab}$ 都是不成立的.

	\onslide<+->
	最后, 注意 $e^a$ 作为指数函数 $f(z)=e^z$ 在 $a$ 处的值和作为 $g(z)=z^a$ 在 $e$ 处的值是\alert{不同}的.
	\onslide<+->
	因为后者在 $a\not\in\BZ$ 时总是多值的.
	\onslide<+->
	前者实际上是后者的主值.
	\onslide<+->
	为避免混淆, 以后我们总\alert{默认 $e^a$ 表示指数函数 $\exp a$}.
\end{frame}


\subsection{三角函数和反三角函数}

\begin{frame}{三角函数的定义}
	\onslide<+->
	我们知道
	\[\cos x=\frac{e^{ix}+e^{-ix}}2,\quad
	\sin x=\frac{e^{ix}-e^{-ix}}{2i}\]
	对于任意实数 $x$ 成立,
	\onslide<+->
	我们将其推广到复数情形.
	\onslide<+->
	\begin{definition}
		定义余弦和正弦函数
		\[\displaystyle\cos z=\frac{e^{iz}+e^{-iz}}2,\quad
		\sin z=\frac{e^{iz}-e^{-iz}}{2i}.\]
	\end{definition}

	\onslide<+->
	那么欧拉恒等式 \emph{$e^{iz}=\cos z+i\sin z$ 对任意复数 $z$ 均成立}.
\end{frame}


\begin{frame}{三角函数的性质}
	\onslide<+->
	不难得到
	\[
		\cos(iy)=\dfrac{e^y+e^{-y}}2,\qquad
		\visible<+->{\sin(iy)=i\dfrac{e^y-e^{-y}}2.}
	\]
	\onslide<+->
	当 $y\to\infty$ 时, $\cos(iy)$ 和 $\sin(iy)$ 都 $\to\infty$.
	\onslide<+->
	因此 \alert{$\sin z$ 和 $\cos z$ 并不有界}. 
	\onslide<+->
	这和实变情形完全不同.

	\onslide<+->
	容易看出 $\cos z$ 和 $\sin z$ 的零点都是实数.
	\onslide<+->
	于是我们可类似定义其它三角函数
	\begin{align*}
		\alertn{\tan z}&
		\alertn{=\frac{\sin z}{\cos z},z\neq\left(k+\half\right)\pi,}&
		\alertn{\cot z}&
		\alertn{=\frac{\cos z}{\sin z},z\neq k\pi,}\\
		\alertn{\sec z}&
		\alertn{=\frac{1}{\cos z},z\neq\left(k+\half\right)\pi,}&
		\alertn{\csc z}&
		\alertn{=\frac{1}{\sin z},z\neq k\pi.}
	\end{align*}
\end{frame}


\begin{frame}{三角函数的性质}
	\onslide<+->
	这些三角函数的奇偶性, 周期性和导数与实变情形类似,
	\[\alertn{(\cos z)'=-\sin z,\quad
	(\sin z)'=\cos z,}\]
	\onslide<+->
	且在定义域范围内是处处解析的.

	\onslide<+->
	三角函数的各种恒等式在复数情形也仍然成立,
	\onslide<+->
	例如
	\begin{itemize}
		\item $\cos(z_1\pm z_2)=\cos z_1 \cos z_2\mp \sin z_1 \sin z_2$,
		\item $\sin(z_1\pm z_2)=\sin z_1 \cos z_2\pm\cos z_1 \sin z_2$,
		\item $\sin^2z+\cos^2z=1$.
	\end{itemize}
\end{frame}


\begin{frame}{双曲函数}
	\onslide<+->
	类似的, 我们可以定义双曲函数:
	\onslide<+->
	\[\alertn{\ch z=\frac{e^z+e^{-z}}2=\cos iz,}\]
	\onslide<+->
	\[\alertn{\sh z=\frac{e^z-e^{-z}}2=-i\sin iz,}\]
	\onslide<+->
	\[\alertn{\tanh z=\frac{e^z-e^{-z}}{e^z+e^{-z}}
	=-i\tan iz,\quad z\neq \left(k+\half\right)\pi i.}\]
	\onslide<+->
	它们的奇偶性和导数与实变情形类似, 在定义域范围内是处处解析的.

	\onslide<+->
	$\ch z,\sh z$ 的周期是 $2\pi i$, $\tanh z$ 的周期是 $\pi i$.
\end{frame}


\begin{frame}{反三角函数和反双曲函数}
	\onslide<+->
	设 $z=\cos w=\dfrac{e^{iw}+e^{-iw}}2$,
	\onslide<+->
	则
	\[e^{2iw}-2ze^{iw}+1=0,\quad
	\visible<+->{e^{iw}=z+\sqrt{z^2-1}\ \text{(双值)}.}\]
	\onslide<+->
	因此\emph{反余弦函数}为
	\[w=\Arccos z=-i\Ln(z+\sqrt{z^2-1}).\]
	\onslide<+->
	显然它是多值的.
	\onslide<+->
	同理, 我们有:
	\begin{itemize}
		\item \emph{反正弦函数} $\Arcsin z=-i\Ln(iz+\sqrt{1-z^2})$;
		\item \emph{反正切函数} $\Arctan z=-\dfrac i2\Ln\dfrac{1+iz}{1-iz}, z\neq \pm i$;
		\item \emph{反双曲余弦函数} $\Arch z=\Ln(z+\sqrt{z^2-1})$;
		\item \emph{反双曲正弦函数} $\Arsh z=\Ln(z+\sqrt{z^2+1})$;
		\item \emph{反双曲正切函数} $\Arth z=\dfrac12\Ln\dfrac{1+z}{1-z}, z\neq \pm1$.
	\end{itemize}
\end{frame}


\begin{frame}{例: 解三角函数方程}
	\onslide<+->
	\begin{example}
		解方程 $\sin z=2$.
	\end{example}

	\onslide<+->
	\begin{solution}
	由于 $\sin z=\dfrac{e^{iz}-e^{-iz}}{2i}=2$,
	\onslide<+->{我们有
	\[e^{2iz}-4ie^{iz}-1=0.\]
	}\onslide<+->{于是 $e^{iz}=(2\pm\sqrt 3)i$,
	}\onslide<+->{
		\[z=-i\Ln[(2\pm\sqrt 3)i]=\left(2k+\half\right)\pi\pm i\ln(2+\sqrt3),\quad k\in\BZ.\]}
	\vspace{-\baselineskip}
	\end{solution}
\end{frame}

\begin{frame}{例: 解三角函数方程}
	\onslide<+->
	\begin{solution}[另解]
	由 $\sin z=2$ 可知
	\[\cos z=\sqrt{1-\sin^2 z}=\pm\sqrt 3i.\]
	\onslide<+->{于是 $e^{iz}=\cos z+i\sin z=(2\pm\sqrt 3)i$,
	}\onslide<+->{
		\[z=-i\Ln[(2\pm\sqrt 3)i]=\left(2k+\half\right)\pi\pm i\ln(2+\sqrt3),\quad k\in\BZ.\]}
	\vspace{-\baselineskip}
	\end{solution}
	\onslide<+->
	我们总有形式
	\begin{align*}
		\Arcsin z&=(2k+\half)\pi\pm \theta,\\
		\Arccos z&=2k\pi\pm \theta,\\
		\Arctan z&=k\pi+\theta.
	\end{align*}
\end{frame}

