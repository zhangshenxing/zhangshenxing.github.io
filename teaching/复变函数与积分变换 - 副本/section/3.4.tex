\section{柯西积分公式}

\subsection{柯西积分公式}

\begin{frame}{柯西积分公式}
	\onslide<+->
	柯西-古萨定理是解析函数理论的基础, 但在很多情形下它由柯西积分公式表现.

	\onslide<+->
	\begin{algorithm}{柯西积分公式}
		设
		\begin{itemize}[<*>]
			\item 函数 $f(z)$ 在(复合)闭路 $C$ 及其内部(围成的区域) $D$ 解析,
			\item $z_0\in D$,
		\end{itemize}
		\onslide<+->{则
			\[f(z_0)=\frac1{2\pi\ii}\oint_C\frac{f(z)}{z-z_0}\diff z.\]}
	\end{algorithm}

	\onslide<+->
	如果 $z_0\notin \ov D$, 由柯西-古萨定理, 右侧的积分是 $0$.
\end{frame}


\begin{frame}{柯西积分公式: 注记}
	\onslide<+->
	解析函数可以用一个积分
	\[f(z)=\frac1{2\pi\ii}\oint_C\frac{f(\zeta)}{\zeta-z}\diff\zeta,\quad z\in D
	\]
	来表示, 这是研究解析函数理论的强有力工具.

	\onslide<+->
	解析函数在闭路 $C$ 内部的取值完全由它在 $C$ 上的值所确定. 这也是解析函数的特征之一.
	\onslide<+->
	特别地, 解析函数在圆心处的值等于它在圆周上的平均值.
	\onslide<+->
	设 $z=z_0+R\ee^{\ii\theta}$, 则 $\diff z=iR\ee^{\ii\theta}\diff\theta$,
	\onslide<+->
	\[f(z_0)=\frac1{2\pi\ii}\oint_C\frac{f(z)}{z-z_0}\diff z=\frac1{2\pi}\int_0^{2\pi}f(z_0+R\ee^{\ii\theta})\diff\theta.
	\]
\end{frame}


\begin{frame}{柯西积分公式: 证明}
	\onslide<+->
	\begin{proof}
		由连续性可知, $\forall\varepsilon>0,\exists\delta>0$ 使得当 $|z-z_0|\le\delta$ 时, $z\in D$ 且 $|f(z)-f(z_0)|<\varepsilon$.
		\onslide<+->{%
			设 $\Gamma:|z-z_0|=\delta$,
		}\onslide<+->{%
			则
			\begin{align*}
				\abs{\oint_C\frac{f(z)}{z-z_0}\diff z-2\pi\ii f(z_0)}
				&\xeq{\text{复合闭路定理}}\abs{\oint_\Gamma\frac{f(z)}{z-z_0}\diff z-2\pi\ii f(z_0)}\\
				&\visible<+->{=\abs{\oint_\Gamma\frac{f(z)}{z-z_0}\diff z-\oint_\Gamma\frac{f(z_0)}{z-z_0}\diff z}
				=\abs{\oint_\Gamma\frac{f(z)-f(z_0)}{z-z_0}\diff z}}\\
				&\visible<+->{\le\frac\varepsilon \delta\cdot 2\pi \delta
				=2\pi \varepsilon.}
			\end{align*}
		}\onslide<+->{%
			由 $\varepsilon$ 的任意性可知 
			$\displaystyle\oint_C\frac{f(z)}{z-z_0}\diff z=2\pi\ii f(z_0)$.\qedhere
		}
	\end{proof}
\end{frame}


\begin{frame}{典型例题: 柯西积分公式的应用}
	\onslide<+->
	从柯西积分公式可以看出, 被积函数分子解析而分母形如 $z-z_0$ 时, 绕闭路的积分可以使用该公式计算.

	\onslide<+->
	\begin{example}
		求 $\displaystyle\oint_{|z|=4}\frac{\sin z}z\diff z$.
	\end{example}

	\onslide<+->
	\begin{solution}
		函数 $\sin z$ 处处解析.
		\onslide<+->{%
			取 $f(z)=\sin z, z_0=0$ 并应用柯西积分公式得
			\[
				\oint_{|z|=4}\frac{\sin z}z\diff z
				=2\pi\ii \sin z|_{z=0}=0.
	\]
		}\vspace{-.5\baselineskip}
	\end{solution}
\end{frame}


\begin{frame}{典型例题: 柯西积分公式的应用}
	\onslide<+->
	\begin{example}
		求 $\displaystyle\oint_{|z|=2}\frac{\ee^z}{z-1}\diff z$.
	\end{example}

	\onslide<+->
	\begin{solution}
		由于函数 $\ee^z$ 处处解析,
		\onslide<+->{
			取 $f(z)=\ee^z, z_0=1$ 并应用柯西积分公式得
			\[
				\oint_{|z|=2}\frac{\ee^z}{z-1}\diff z
				=2\pi\ii \ee^z|_{z=1}=2\pi ei.
	\]
		}\vspace{-.5\baselineskip}
	\end{solution}

	\onslide<+->
	\begin{exercise}
		求 $\displaystyle\oint_{|z|=2\pi}\frac{\cos z}{z-\pi}\diff z=$\fillblankframe{$-2\pi\ii$}.
	\end{exercise}
\end{frame}


\begin{frame}{典型例题: 柯西积分公式的应用}
	\onslide<+->
	\begin{example}
		设 $f(z)=\displaystyle\oint_{|\zeta|=\sqrt3}\frac{3\zeta^2+7\zeta+1}{\zeta-z}\diff \zeta$, 求 $f'(1+\ii)$.
	\end{example}

	\onslide<+->
	\begin{solution}
		当 $|z|<\sqrt3$ 时,由柯西积分公式得
	\[
			f(z)=\oint_{|\zeta|=\sqrt3}\frac{3\zeta^2+7\zeta+1}{\zeta-z}\diff \zeta
			\visible<+->{=2\pi\ii(3\zeta^2+7\zeta+1)|_{\zeta=z}=2\pi\ii(3z^2+7z+1).}
	\]
		\onslide<+->{%
			因此 $f'(z)=2\pi\ii(6z+7)$,
		}\onslide<+->{
			$f'(1+\ii)=2\pi\ii(13+6\ii)=-12\pi+26\pi\ii$.
		}
	\end{solution}
	\onslide<+->
	注意当 $|z|>\sqrt3$ 时, $f(z)\equiv0$.
\end{frame}


\begin{frame}{典型例题: 柯西积分公式的应用}
	\onslide<+->
	\begin{example}
		求 $\displaystyle\oint_{|z|=3}\frac{\ee^z}{z(z^2-1)}\diff z$.
	\end{example}
	\onslide<+->
	\begin{solution}
		\begin{minipage}{.66\textwidth}
			被积函数的奇点为 $0,\pm1$.
			\onslide<+->{%
				设 $C_1$, $C_2$, $C_3$ 分别为绕 $0$, $1$, $-1$ 的分离圆周.
			}\onslide<+->{%
				由复合闭路定理和柯西积分公式
				\begin{align*}
					&\oint_{|z|=3}\frac{\ee^z}{z(z^2-1)}\diff z
					=\oint_{C_1+C_2+C_3}\frac{\ee^z}{z(z^2-1)}\diff z\\
					\visible<+->{=}&\visible<.->{2\pi\ii\Bigg(\frac{\ee^z}{z^2-1}\bigg|_{z=0}+\frac{\ee^z}{z(z+1)}\bigg|_{z=1}+\frac{\ee^z}{z(z-1)}\bigg|_{z=-1}\Bigg)}\\
					\visible<+->{=}&\visible<.->{2\pi\ii\Bigg(-1+\frac e2+\frac{\ee^{-1}}2\Bigg)=\pi\ii(e+\ee^{-1}-2).}
				\end{align*}
			}\vspace{-1.2\baselineskip}
		\end{minipage}
		\begin{minipage}{.32\textwidth}
			\begin{tikzpicture}[scale=.85]
				\draw[cstaxis] (-2.5,0)--(2.5,0);
				\draw[cstaxis] (0,-2.5)--(0,2.5);
				\draw[cstcurve,main] (0,0) circle (1.8);
				\draw[cstcurve,main,visible on=<3->] (1.2,0) circle(0.5);
				\draw[cstcurve,main,visible on=<3->] (-1.2,0) circle(0.5);
				\draw[cstcurve,main,visible on=<3->] (0,0) circle(0.5);
				\fill[cstdot,second,visible on=<2->] (0,0) circle;
				\fill[cstdot,second,visible on=<2->] (1.2,0) circle;
				\fill[cstdot,second,visible on=<2->] (-1.2,0) circle;
				\draw
					(1.6,1.6) node[main] {$C$}
					(-0.4,0.8) node[second,visible on=<3->] {$C_1$}
					(1.2,-.8) node[second,visible on=<3->] {$C_2$}
					(-1.2,-.8) node[second,visible on=<3->] {$C_3$};
			\end{tikzpicture}
		\end{minipage}
	\end{solution}
\end{frame}


\subsection{高阶导数的柯西积分公式}

\begin{frame}{高阶导数的柯西积分公式}
	\onslide<+->
	解析函数可以由它的积分所表示.
	\onslide<+->
	不仅如此, 通过积分表示, 还可以说明\alert{解析函数是任意阶可导的}.

	\onslide<+->
	\begin{algorithm}{柯西积分公式}
		设函数 $f(z)$ 在闭路或复合闭路 $C$ 及其内部 $D$ 解析, 则对任意 $z_0\in D$,
	\[
		f^{(n)}(z_0)=\frac{n!}{2\pi\ii}\oint_C\frac{f(z)}{(z-z_0)^{n+1}}\diff z.
	\]
	\end{algorithm}

	\onslide<+->
	假如 $f(z)$ 有泰勒展开
	\[f(z)=f(z_0)+f'(z_0)(z-z_0)+\cdots+\frac{f^{(n)}(z_0)}{n!}(z-z_0)^n+\cdots
	\]
	\onslide<+->
	那么由 $\displaystyle\oint_C \frac{\diff z}{(z-z_0)^n}$ 的性质可知上述公式右侧应当为 $f^{(n)}(z_0)$.
\end{frame}


\begin{frame}{高阶导数的柯西积分公式}
	\onslide<+->
	\begin{solution}[证明]
		先证明 $n=1$ 的情形.
		\onslide<+->{设 $\delta$ 为 $z_0$ 到 $C$ 的最短距离.
		}\onslide<+->{当 $|h|<\delta$ 时, $z_0+h\in D$.
		}\onslide<+->{由柯西积分公式,
			\[f(z_0)=\frac1{2\pi\ii}\oint_C\frac{f(z)}{z-z_0}\diff z,\ 
			f(z_0+h)=\frac1{2\pi\ii}\oint_C\frac{f(z)}{z-z_0-h}\diff z.
	\]
		}\onslide<+->{两式相减得到
			\[\frac{f(z_0+h)-f(z_0)}h=\frac1{2\pi\ii}\alert{\oint_C\frac{f(z)}{(z-z_0)(z-z_0-h)}\diff z}.
	\]
		}\onslide<+->{当 $h\to 0$ 时, 左边的极限是 $f'(z_0)$.
		}\onslide<+->{因此我们只需要证明右边的极限等于 
			$\displaystyle\frac1{2\pi\ii}\alert{\oint_C\frac{f(z)}{(z-z_0)^2}\diff z}$.}
	\end{solution}
\end{frame}


\begin{frame}{高阶导数的柯西积分公式}
	\onslide<+->
	\begin{proof}[续证]
		二者之差 $=\displaystyle\frac1{2\pi\ii}\oint_C\frac{h f(z)}{(z-z_0)^2(z-z_0-h)}\diff z$.
		\onslide<+->{由于 $f(z)$ 在 $C$ 上连续, 故存在 $M$ 使得 $|f(z)|\le M$.
		}\onslide<+->{注意到 $z\in C$, $|z-z_0|\ge \delta$, $|z-z_0-h|\ge\delta-|h|$.
		}\onslide<+->{由长大不等式,
			\[\abs{\oint_C\frac{h f(z)}{(z-z_0)^2(z-z_0-h)}\diff z}\le\frac{M|h|}{\delta^2(\delta-|h|)}\cdot L,
	\]
		其中 $L$ 是闭路 $C$ 的长度.
		}\onslide<+->{当 $h\to0$ 时, 它的极限为 $0$, 因此 $n=1$ 情形得证.}

		\onslide<+->{对于一般的 $n$, 我们通过归纳法将 $f^{(n)}(z_0)$ 和 $f^{(n)}(z_0+h)$ 表达为积分形式.
		}\onslide<+->{比较 $\dfrac{f^{(n)}(z_0+h)-f^{(n)}(z_0)}h$ 与积分公式右侧之差, 并利用长大不等式证明 $h\to 0$ 时, 差趋于零.
		}\onslide<+->{具体过程省略.\qedhere}
	\end{proof}
\end{frame}


\begin{frame}{典型例题: 使用高阶导数的柯西积分公式计算积分}
	\onslide<+->
	\alert{柯西积分公式不是用来计算高阶导数的, 而是用高阶导数来计算积分的.}

	\onslide<+->
	\begin{example}
		求 $\displaystyle\oint_{|z|=2}\frac{\cos(\pi z)}{(z-1)^5}\diff z.$
	\end{example}

	\onslide<+->
	\begin{solution}
		由于 $\cos(\pi z)$ 处处解析,
		\onslide<+->{%
			因此由柯西积分公式,
			\[
				\oint_{|z|=2}\frac{\cos(\pi z)}{(z-1)^5}\diff z
				=\frac{2\pi\ii}{4!}[\cos(\pi z)]^{(4)}\big|_{z=1}
				\visible<+->{=\frac{2\pi\ii}{24}\cdot \pi^4\cos \pi=-\frac{\pi^5 i}{12}.}
	\]
		}\vspace{-.5\baselineskip}
	\end{solution}
\end{frame}


\begin{frame}{典型例题: 使用高阶导数的柯西积分公式计算积分}\small
	\beqskip{6pt}
	\onslide<+->
	\begin{example}
		求 $\displaystyle\oint_{|z|=2}\frac{\ee^z}{(z^2+1)^2}\diff z.$
	\end{example}
	\onslide<+->
	\begin{solution}
		被积函数在 $|z|<2$ 的奇点为 $z=\pm i$.
		\onslide<+->{%
			取 $C_1,C_2$ 为以 $i,-\ii $ 为圆心的分离圆周.
		}\onslide<+->{%
			\[
				 \oint_{C_1}\frac{\ee^z}{(z^2+1)^2}\diff z
				=\frac{2\pi\ii}{1}\biggl(\frac{\ee^z}{(z+\ii)^2}\biggr)'\Big|_{z=i}
				\visible<+->{=2\pi\ii\biggl(\frac{\ee^z}{(z+\ii)^2}-\frac{2\ee^z}{(z+\ii)^3}\biggr)\Big|_{z=i}
				=\frac{(1-\ii )\ee^i\pi}2.}
	\]
		}\onslide<+->{%
			类似地, $\displaystyle\oint_{C_2}\frac{\ee^z}{(z^2+1)^2}\diff z=\frac{-(1+\ii)\ee^{-\ii }\pi}2$.
		}\onslide<+->{%
			故
			\begin{align*}
				 \oint_{|z|=2}\frac{\ee^z}{(z^2+1)^2}\diff z
			 &=\biggl(\oint_{C_1}+\oint_{C_2}\biggr)\frac{\ee^z}{(z^2+1)^2}\diff z\\
			 &=\frac{(1-\ii )\ee^i\pi}2+\frac{-(1+\ii)\ee^{-\ii }\pi}2
			  =\pi\ii(\sin1-\cos1).
			\end{align*}
		}\vspace{-\baselineskip}
	\end{solution}
	\endgroup
\end{frame}


\begin{frame}{典型例题: 使用高阶导数的柯西积分公式计算积分}
	\onslide<+->
	\begin{example}
		求 $\displaystyle\oint_{|z|=1}z^n\ee^z\diff z$, 其中 $n$ 是整数.
	\end{example}

	\onslide<+->
	\begin{solution}
		\begin{enumerate}
			\item 当 $n\ge 0$ 时, $z^n\ee^z$ 处处解析.
				\onslide<+->{%
					由柯西-古萨定理, 
					\[
						\oint_{|z|=1}z^n\ee^z\diff z=0.
					
	\]
				}
			\item 当 $n\le-1$ 时, $\ee^z$ 处处解析.
				\onslide<+->{%
					由柯西积分公式,
					\[
						\oint_{|z|=1}z^n\ee^z\diff z
						=\frac{2\pi\ii}{(-n-1)!}(\ee^z)^{(-n-1)}\big|_{z=0}
						=\frac{2\pi\ii}{(-n-1)!}.
					
	\]
				}
		\end{enumerate}
		\vspace{-\baselineskip}
	\end{solution}
\end{frame}


\begin{frame}{典型例题: 使用高阶导数的柯西积分公式计算积分}\small
	\beqskip{0pt}
	\onslide<+->
	\begin{example}
		求 $\displaystyle\oint_{|z-3|=2}\frac1{(z-2)^2z^3}\diff z$ 和 $\displaystyle\oint_{|z-1|=2}\frac1{(z-2)^2z^3}\diff z$.
	\end{example}
	\onslide<+->
	\begin{solution}
		\begin{enumerate}
			\item $\dfrac1{(z-2)^2z^3}$ 在 $|z-3|<2$ 的奇点为 $z=2$.
			\onslide<+->{%
				由柯西积分公式,
				\[
					\oint_{|z-3|=2}\frac1{(z-2)^2z^3}\diff z
					=\frac{2\pi\ii}{1!}\Bigl(\frac1{z^3}\Bigr)'\bigg|_{z=2}
					=-\frac{3\pi\ii}8.
				
	\]
			}
			\item $\dfrac1{(z-2)^2z^3}$ 在 $|z-1|<3$ 的奇点为 $z=0,2$.
			\onslide<+->{%
				取 $C_1,C_2$ 为以 $0,2$ 为圆心的分离圆周.
			}\onslide<+->{
				\begin{align*}
					&\oint_{|z-1|=3}\frac1{(z-2)^2z^3}\diff z=\oint_{C_1}\frac1{(z-2)^2z^3}\diff z+\oint_{C_2}\frac1{(z-2)^2z^3}\diff z\\
					\visible<+->{=}&\visible<.->{\frac{2\pi\ii}{2!}\left[\frac1{(z-2)^2}\right]''\Big|_{z=0}+\frac{2\pi\ii}{1!}\Bigl(\frac1{z^3}\Bigr)'\Big|_{z=2}=0.}
				\end{align*}
			}
		\end{enumerate}
		\vspace{-\baselineskip}
	\end{solution}
	\endgroup
\end{frame}


\begin{frame}{莫累拉定理}
	\beqskip{2pt}
	\onslide<+->
	\begin{exercise}
		$\displaystyle\oint_{|z-2\ii|=3}\frac1{z^2(z-\ii )}\diff z=$\fillblankframe{$0$}.
	\end{exercise}
	\onslide<+->
	\begin{example}[莫累拉定理]
		设 $f(z)$ 在单连通区域 $D$ 内连续, 且对于 $D$ 中任意闭路 $C$ 都有 $\displaystyle\oint_Cf(z)\diff z=0$, 则 $f(z)$ 在 $D$ 内解析.
	\end{example}
	\onslide<+->
	\begin{proof}
		由题设可知 $f(z)$ 的积分与路径无关.
		\onslide<+->{固定 $z_0\in D$, 则
			\[F(z)=\int_{z_0}^zf(z)\diff z
	\]
		定义了 $D$ 内的一个函数.
		}\onslide<+->{类似于原函数的证明可知 $F'(z)=f(z)$.
		}\onslide<+->{故 $f(z)$ 作为解析函数 $F(z)$ 的导数也是解析的.\qedhere}
	\end{proof}
	\endgroup
\end{frame}


\begin{frame}{解析函数与实函数的差异}
	\onslide<+->
	高阶柯西积分公式说明解析函数的导数与实函数的导数有何不同?
	\onslide<+->
	高阶柯西积分公式说明, 函数 $f(z)$ 只要在区域 $D$ 中处处可导, 它就一定无限次可导, 并且各阶导数仍然在 $D$ 中解析.
	\onslide<+->
	\alert{这一点与实变量函数有本质的区别.}

	\onslide<+->
	同时我们也可以看出, 如果一个二元实函数 $u(x,y)$ 是一个解析函数的实部或虚部, 则 $u$ 也是具有任意阶偏导数.
	\onslide<+->
	这便引出了调和函数的概念.
\end{frame}


