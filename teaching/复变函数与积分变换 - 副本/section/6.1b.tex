\subsection{傅里叶变换的性质}


\begin{frame}{傅里叶变换的性质: 线性性质}
	\onslide<+->
	我们不可能也没必要每次都对需要变换的函数从定义出发计算傅里叶变换.
	\onslide<+->
	通过研究傅里叶变换的性质, 结合常见函数的傅里叶变换, 我们可以得到很多情形的傅里叶变换.

	\onslide<+->
	\begin{block}{线性性质}
		\[\msf[\alpha f+\beta g]=\alpha F+\beta G,\quad
		\msf^{-1}[\alpha F+\beta G]=\alpha f+\beta g.\]
	\end{block}

	\onslide<+->
	\begin{block}{位移性质}
		\[\msf[f(t-t_0)]=e^{-j\omega t_0}F(\omega),\quad
		\msf^{-1}[F(\omega-\omega_0)]=e^{j\omega_0 t}f(t).\]
		\vspace{-\baselineskip}
	\end{block}
\end{frame}


\begin{frame}{傅里叶变换的性质: 位移性质}
	\onslide<+->
	\begin{proof}
		\begin{align*}
			\msf[f(t-t_0)]&=\int_{-\infty}^{+\infty}f(t-t_0)e^{-j\omega t}\diff t\\
			&=\int_{-\infty}^{+\infty}f(t)e^{-j\omega (t+t_0)}\diff t=e^{-j\omega t_0}F(\omega).
		\end{align*}
		\onslide<+->{逆变换情形类似可得.\qedhere}
	\end{proof}

	\onslide<+->
	由此可得
	\[\msf[\delta(t-t_0)]=e^{-j\omega t_0},\quad
	\msf^{-1}[\delta(\omega-\omega_0)]=\dfrac1{2\pi}e^{j\omega_0 t}.\]
\end{frame}


\begin{frame}{傅里叶变换的性质: 微分性质}
	\onslide<+->
	\begin{block}{微分性质}
		\[\msf[f'(t)]=j\omega F(\omega),\quad
		\msf^{-1}[F'(\omega)]=-jtf(t),\]
		\vspace{-\baselineskip}
		\[\msf[f^{(k)}(t)]=(j\omega)^k F(\omega),\quad
		\msf^{-1}[F^{(k)}(\omega)]=(-jt)^kf(t).\]
		这里, 被变换的函数要求在 $\infty$ 处趋于 $0$.
	\end{block}

	\onslide<+->
	\begin{proof}
		\begin{align*}
			\msf[f']&=\int_{-\infty}^{+\infty}f'(t)e^{-j\omega t}\diff t\\
			&=-\int_{-\infty}^{+\infty}f(t)(e^{-j\omega t})'\diff t=j\omega F(\omega)
		\end{align*}
		\onslide<+->{逆变换情形类似可得. 一般的 $k$ 归纳可得.\qedhere}
	\end{proof}
\end{frame}


\begin{frame}{傅里叶变换的性质}
	\onslide<+->
	由微分性质可得
	\begin{block}{乘多项式性质}
		\[\msf[tf(t)]=jF'(\omega),\quad
		\msf^{-1}[\omega F(\omega)]=-jf'(t),\]
		\[\msf[t^kf(t)]=j^kF^{(k)}(\omega),\quad
		\msf^{-1}[\omega^kF(\omega)]=(-j)^kf^{(k)}(t).\]
	\end{block}

	\onslide<+->
	\begin{block}{积分性质}
		\[\msf\left[\int_{-\infty}^t f(\tau)\diff\tau\right]=\frac1{j\omega}F(\omega).\]
	\end{block}

	\onslide<+->
	由变量替换易得
	\begin{block}{相似性质}
		\[\msf[f(at)]=\frac1{|a|}F\left(\frac\omega a\right),\quad
		\msf^{-1}[F(a\omega)]=\frac1{|a|}f\left(\frac t a\right).\]
	\end{block}
\end{frame}


\begin{frame}{典型例题: 计算傅里叶变换}
	\onslide<+->
	\begin{example}
		求 $\msf[t^k e^{-\beta t}u(t)],\beta>0$.
	\end{example}

	\onslide<+->
	\begin{solution}
			由于
			\[\msf[e^{-\beta t}u(t)]=\frac{1}{\beta+j\omega},\]
		\onslide<+->{因此
			\begin{align*}
				\msf[t^ke^{-\beta t}u(t)]&=j^k\left(\frac1{\beta+j\omega}\right)^{(k)}
				=\frac{k!}{(\beta+j\omega)^{k+1}}.
			\end{align*}
		}
		\vspace{-\baselineskip}
	\end{solution}
\end{frame}


\begin{frame}{典型例题: 计算傅里叶变换}
	\onslide<+->
	\begin{example}
		求 $\sin{\omega_0 t}$ 的傅里叶变换.
	\end{example}

	\onslide<+->
	\begin{solution}
			由于 $\msf[1]=2\pi\delta(\omega)$,
		\onslide<+->{因此 $\msf[e^{j\omega_0t}]=2\pi\delta(\omega-\omega_0)$,
		}
		\onslide<+->{\begin{align*}
				\msf[\sin{\omega_0 t}]&=\frac1{2j}\left[\msf[e^{j\omega_0t}]-\msf[e^{-j\omega_0t}]\right]&\\
				&\visible<+->{=\frac1{2j}[2\pi\delta(\omega-\omega_0)-2\pi\delta(\omega+\omega_0)]}&\\
				&\visible<+->{=j\pi[\delta(\omega+\omega_0)-\delta(\omega-\omega_0)].}
			\end{align*}
		}
		\vspace{-1.3\baselineskip}
	\end{solution}

	\onslide<+->
	\begin{exercise}
		$\msf[\cos{\omega_0 t}]=$\fillblank[5cm][1mm]{\visible<+->{$\pi[\delta(\omega+\omega_0)+\delta(\omega-\omega_0)]$}}.
	\end{exercise}
\end{frame}


\begin{frame}{常见傅里叶变换汇总}
	\onslide<+->
	\begin{alertblock}{常见傅里叶变换汇总 I}
		\[\msf[\delta(t)]=1,\quad \msf[\delta(t-t_0)]=e^{-j\omega t_0},\]
		\[\msf[1]=2\pi\delta(\omega),\quad \msf[e^{j\omega_0 t}]=2\pi\delta(\omega-\omega_0).\]
		\begin{align*}
		\msf[\sin \omega_0t]&=j\pi[\delta(\omega+\omega_0)-\delta(\omega-\omega_0)],\\
		\msf[\cos{\omega_0 t}]&=\pi[\delta(\omega+\omega_0)+\delta(\omega-\omega_0)].
		\end{align*}
	\end{alertblock}

	\onslide<+->
	\begin{block}{常见傅里叶变换汇总 II}
		\[\msf[u(t)e^{-\beta t}]=\dfrac1{\beta+j\omega},\quad
		\msf[e^{-\beta t^2}]=\sqrt{\dfrac\pi\beta}e^{-\omega^2/(4\beta)},\]
		\[\msf[u(t)]=\dfrac1{j\omega}+\pi\delta(\omega).\]
	\end{block}
\end{frame}


\subsection{卷积*}
\begin{frame}{卷积\noexer}
	\onslide<+->
	\begin{definition}
		$f_1(t),f_2(t)$ 的\emph{卷积}是指
		\[\alert{(f_1\ast f_2)(t)=\int_{-\infty}^{+\infty} f_1(\tau)f_2(t-\tau)\diff \tau.}\]
	\end{definition}

	\onslide<+->
	容易验证卷积满足如下性质:
	\begin{itemize}
		\item $f_1\ast f_2=f_2\ast f_1,\ (f_1\ast f_2)\ast f_3=f_1\ast(f_2\ast f_3)$;
		\item $f_1\ast(f_2+f_3)=f_1\ast f_2+f_1\ast f_3$;
		\item $f\ast\delta=f$;
		\item $(f_1\ast f_2)'=f_1'\ast f_2=f_1\ast f_2'$.
	\end{itemize}
\end{frame}


\begin{frame}{例题: 计算卷积\noexer}
	\onslide<+->
	\begin{example}
		设 $f_1(t)=u(t),f_2(t)=e^{-t}u(t)$. 求 $f_1\ast f_2$.
	\end{example}

	\onslide<+->
	\begin{solution}
			\[(f_1\ast f_2)(t)=\int_{-\infty}^{+\infty} f_2(\tau)f_1(t-\tau)\diff \tau=\int_0^{+\infty} e^{-\tau}u(t-\tau)\diff \tau.\]
		\onslide<+->{当 $t<0$ 时, $(f_1\ast f_2)(t)=0$.
		}\onslide<+->{当 $t\ge0$ 时, 
			\[(f_1\ast f_2)(t)=\int_0^t e^{-\tau}\diff \tau=1-e^{-t}.\]
		}\onslide<+->{故 $(f_1\ast f_2)(t)=(1-e^{-t})u(t)$.
		}
	\end{solution}
\end{frame}


\begin{frame}{卷积定理\noexer}
	\onslide<+->
	\begin{block}{卷积定理}
	\[\msf[f_1\ast f_2]=F_1\cdot F_2,\quad
	\msf^{-1}[F_1\ast F_2]=\frac1{2\pi}f_1\cdot f_2.\]
	\vspace{-\baselineskip}
	\end{block}

	\onslide<+->
	\begin{proof}
		\vspace{-\baselineskip}
		\begin{align*}
			\msf[f_1\ast f_2]&=\int_{-\infty}^{+\infty}\int_{-\infty}^{+\infty}f_1(\tau)f_2(t-\tau)\diff \tau \cdot e^{-j\omega t}\diff t\\
			&\visible<+->{=\int_{-\infty}^{+\infty}\int_{-\infty}^{+\infty}f_1(\tau)e^{-j\omega \tau}\cdot f_2(t-\tau)e^{-j\omega (t-\tau)}\diff t\diff \tau}\\
			&\visible<+->{=\int_{-\infty}^{+\infty}\int_{-\infty}^{+\infty}f_1(\tau)e^{-j\omega \tau}\cdot f_2(t)e^{-j\omega t}\diff t\diff \tau}\\
			&\visible<+->{=\int_{-\infty}^{+\infty}f_1(\tau)e^{-j\omega \tau}\diff\tau\int_{-\infty}^{+\infty}f_2(t)e^{-j\omega t}\diff t}\\
			&\visible<+->{=\msf[f_1]\msf[f_2].\qedhere}
		\end{align*}
	\end{proof}
\end{frame}


\begin{frame}{卷积的应用\noexer}
	\onslide<+->
	\begin{example}
		求 $\displaystyle	I=\int_{-\infty}^{+\infty}\frac{\sin \omega}{\omega}\cdot \frac{\sin (\omega/3)}{\omega/3}\diff\omega$.
	\end{example}

	\onslide<+->
	\begin{solution}
			设 $F(\omega)=\dfrac{\sin\omega}{\omega},G(\omega)=\dfrac{\sin(\omega/3)}{\omega/3}$,%
		\onslide<+->{则
			\[\msf^{-1}[FG]=\frac1{2\pi}\int_{-\infty}^{+\infty}F(\omega)G(\omega)e^{j\omega t}\diff\omega,\]
		}\onslide<+->{
			\[\msf^{-1}[FG](0)=\frac1{2\pi}\int_{-\infty}^{+\infty}F(\omega)G(\omega)\diff\omega=\frac1{2\pi}I.\]
		}
	\end{solution}
\end{frame}


\begin{frame}{卷积的应用\noexer}
	\onslide<+->
	\begin{solutionc}
			我们之前计算过
			\[\msf^{-1}[F(\omega)]=f(t)=\begin{cases}
				1/2, & |t|<1,\\
				1/4, & |t|=1,\\
				0, & |t|>1.
			\end{cases}\]
		\onslide<+->{所以 $\msf^{-1}[G(\omega)]=g(t)=3f(3t)$,
		}\onslide<+->{
			\begin{align*}
				\msf^{-1}[FG](0)&=(f\ast g)(0)\\
				&=\int_{-\infty}^{+\infty}f(-t)g(t)\diff t=\half\int_{-1}^1 g(t)\diff t=\half ,
			\end{align*}
		}\onslide<+->{故 $I=2\pi\msf^{-1}[FG](0)=\pi$.}
	\end{solutionc}
\end{frame}


\subsection{傅里叶变换的应用*}
\begin{frame}{使用傅里叶变换解微积分方程\noexer}
	\begin{center}
		\begin{tikzpicture}[node distance=40pt]
			\node[cstnodeg] (a){微分方程或积分方程};
			\node[cstnodeb,right=110pt of a] (b){象函数的代数方程};
			\node[cstnodeg,below=of a] (c){原象函数(方程的解)};
			\node[cstnodeb,below=of b] (d){象函数};
			\draw[cstnarrow,dcolorb] (a)--node[above]{傅里叶变换 $\msf$}(b);
			\draw[cstnarrow,dcolorb] (d)--node[below]{傅里叶逆变换 $\msf^{-1}$}(c);
			\draw[cstnarrow,dcolorb] (b)--(d);
			\draw[cstnarrow,dcolorc] (a)--(c);
		\end{tikzpicture}
	\end{center}
\end{frame}


\begin{frame}{例题: 使用傅里叶变换解微积分方程\noexer}
	\beqskip{1pt}
	\onslide<+->
	\begin{example}
		解方程 $y'(t)-\displaystyle\int_{-\infty}^t y(\tau)\diff \tau=2\delta(t)$.
	\end{example}

	\onslide<+->
	\begin{solution}
			设 $\msf[y]=Y$.
		\onslide<+->{两边同时作傅里叶变换得到
			\[j\omega Y(\omega)-\frac1{j\omega}Y(\omega)=2,\]
		}\onslide<+->{
			\[Y(\omega)=-\frac{2j\omega}{1+\omega^2}=\frac1{1+j\omega}-\frac1{1-j\omega},\]
		}\onslide<+->{
			\begin{align*}
				y(t)=\msf^{-1}\left[\frac1{1+j\omega}-\frac1{1-j\omega}\right]
				&\visible<+->{=\begin{cases}
				0-e^t=-e^t,&t<0,\\
				0,&t=0,\\
				e^{-t}-0=e^{-t},&t>0.
				\end{cases}}
			\end{align*}
		}
	\end{solution}
	\endgroup
\end{frame}


\begin{frame}{例题: 使用傅里叶变换解微积分方程\noexer}
	\onslide<+->
	\begin{example}
		解方程 $y''(t)-y(t)=0$.
	\end{example}

	\onslide<+->
	\begin{solution}
			设 $\msf[y]=Y$,
		\onslide<+->{则
			\[\msf[y''(t)-y(t)]=[(j\omega)^2-1]Y(\omega)=0,\]
		}
		\vspace{-\baselineskip}
		\onslide<+->{\[Y(\omega)=0,\quad y(t)=\msf^{-1}[Y(\omega)]=0.\]}
		\vspace{-\baselineskip}
	\end{solution}

	\onslide<+->
	显然这是不对的, 该方程的解应该是 $y(t)=C_1e^t+C_2e^{-t}$.

	\onslide<+->
	原因在于使用傅里叶变换要求函数是绝对可积的, 而 $e^t,e^{-t}$ 并不满足该条件.
	\onslide<+->
	我们需要一个对函数限制更少的积分变换来解决此类方程, 例如拉普拉斯变换.
\end{frame}

