\section{复数及其代数运算}

\subsection{复数的概念}

\begin{frame}{复数的定义}
	\onslide<+->
	现在我们来正式介绍复数的概念.
	\onslide<+->
	为了避免记号 $\sqrt{-1}$ 带来的歧义, 我们先引入抽象符号 $\ii$, 再通过定义它的运算来构造复数.
	\onslide<+->
	\begin{definition}
		固定一个记号 $\ii$, \emph{复数}就是形如 $z=x+y\ii$ 的元素, 其中 $x,y$ 均是实数, 且不同的 $(x,y)$ 对应不同的复数.
	\end{definition}
	\onslide<+->
	实数 $x$ 可以自然地看成复数 $x+0\ii$.
\end{frame}


\begin{frame}{复平面}
	\onslide<+->
	将\emph{全体复数记作 $\BC$}.
	\onslide<+->
	那么 $\BC$ 自然构成一个二维实线性空间, 且 $\{1,\ii\}$ 是一组基. 
	\onslide<+->
	因此它和平面上的点可以建立一一对应, 并将建立起这种对应的平面称为\emph{复平面}.
	\onslide<+->
	\begin{center}
		\begin{tikzpicture}
			\begin{scope}
				\draw[cstaxis] (-.5,0)--(3,0);
				\draw[cstaxis] (0,-.5)--(0,2.5);
				\coordinate [label=below left:$0$] (O) at (0,0);
				\coordinate [label=above:\textcolor{second}{$z=x+y\ii$}] (A) at (2,1.5);
				\coordinate (B) at (2,0);
				\coordinate (C) at (0,1.5);
				\draw[cstdash] (B)--(A)--(C);
				\fill[cstdot,second] (A) circle;
				\draw[third,Latex-Latex,line width=.5mm] (2.8,1)--(4,1) node[midway,below,third] {一一对应};
			\end{scope}
			\begin{scope}[xshift=5cm]
				\coordinate [label=below left:$O$] (O) at (0,0);
				\coordinate [label=above:\textcolor{second}{$Z(x,y)$}] (A) at (2,1.5);
				\coordinate (B) at (2,0);
				\coordinate (C) at (0,1.5);
				\draw[cstdash] (B)--(A)--(C);
				\fill[cstdot,second] (A) circle;
				\draw[decorate,decoration={brace,amplitude=5},main,cstfill1] (O)--(B) node[midway,above=2mm] {$x$};
				\draw[decorate,decoration={brace,amplitude=5},main,cstfill1] (C)--(O) node[midway,right=2mm] {$y$};
				\draw[third,Latex-Latex,line width=.5mm] (2.8,1)--(4,1) node[midway,below,third] {一一对应};
				\draw[cstaxis] (-.5,0)--(3,0);
				\draw[cstaxis] (0,-.5)--(0,2.5);
			\end{scope}
			\begin{scope}[xshift=10cm]
				\draw[cstaxis] (-.5,0)--(3,0);
				\draw[cstaxis] (0,-.5)--(0,2.5);
				\coordinate [label=below left:$O$] (O) at (0,0);
				\coordinate [label=above:\textcolor{second}{$\overrightarrow{OZ}=(x,y)$}] (A) at (2,1.5);
				\draw[cstcurve,cstra,second] (O)--(A);
			\end{scope}
		\end{tikzpicture}
	\end{center}
\end{frame}


\begin{frame}{实部和虚部, 虚数和纯虚数}
	\onslide<+->
	\begin{itemize}
		\item $x,y$ 轴分别对应复平面的\emph{实轴}和\alert{虚轴}.
		\item 称 $z=x+y\ii$ 中 $x=\Re z$ 为 $z$ 的\emph{实部}; $y=\Im z$ 为 $z$ 的\alert{虚部}.
		\item 当虚部 $\Im z=0$ 时, $z$ 为实数, 它落在实轴上.
		\item 不是实数的复数是\textcolor{third}{\bf 虚数}.
		\item 当实部 $\Re z=0$ 且 \alert{$z\neq0$} 时, $z$ 为\alert{纯虚数}, 它落在虚轴上.
	\end{itemize}
	\onslide<1->
	\begin{figure}[hbpt]
		\centering
		\begin{minipage}{.48\textwidth}
			\raggedleft
			\begin{tikzpicture}
				\coordinate [label=below left:$0$] (O) at (0,0);
				\coordinate (B) at (2,0);
				\coordinate (C) at (0,1.5);
				\draw[cstaxis] (-.5,0)--(3,0);
				\draw[cstaxis] (0,-.5)--(0,2.5);
				\begin{scope}[visible on=<3->]
					\draw[decorate,decoration={brace,amplitude=5},main,cstfill1] (B)--(O) node[midway,below=1.5mm] {$\Re z$};
					\draw[decorate,decoration={brace,amplitude=5},second,cstfill2] (C)--(O) node[midway,right=1.5mm] {$\Im z$};
					\coordinate [label=above:\textcolor{third}{$z=x+y\ii$}] (A) at (2,1.5);
					\draw[cstdash] (B)--(A)--(C);
					\fill[cstdot,third] (A) circle;
				\end{scope}
				\begin{scope}[visible on=<2->]
					\coordinate [label=above:\textcolor{main}{实轴}] (R) at (3,0);
					\coordinate [label=right:\textcolor{second}{虚轴}] (I) at (0,2.5);
					\draw[cstaxis,main] (-.5,0)--(R);
					\draw[cstaxis,second] (0,-.5)--(I);
				\end{scope}
				\draw[main,->,thick,visible on=<4->] (-2,.2)-|(.6,0);
				\draw[second,->,thick,visible on=<6->] (-2,1.3)--(0,1.3);
				\draw 
					(-2,.1) node[cstnode,draw=main,text=main,visible on=<4->] {实数}
					(-2,1.2) node[align=center,cstnode,draw=second,text=second,visible on=<6->] {纯虚数\\不含原点};
			\end{tikzpicture}
		\end{minipage}
		\begin{minipage}{.48\textwidth}
			\centering
			\begin{tikzpicture}
				\filldraw[cstcurve,cstfill] (.8,0) circle (2.6 and 2);
				\coordinate (R) at (0,-.8);
				\filldraw[cstcurve,main,fill=white,visible on=<4->] (R) circle (1.2 and .7);
				\coordinate (I) at (0,.8);
				\draw (R) node[align=center,main,visible on=<4->] {实数 \\$0,1,\sqrt2,\pi,\ee$};
				\draw (I) node[align=center,second,visible on=<6->] {纯虚数 \\$\ii,-\ii ,\pi\ii$};
				\draw[cstcurve,second,visible on=<6->] (I) circle (1.2 and .7);
				\draw 
					(3.7,0) node[align=center] {全\\体\\复\\数}
					(2,0) node[align=center,third,visible on=<5->] {虚数 \\$\ii,\pi\ii,\frac{-1+\sqrt 3 \ii}2$};
			\end{tikzpicture}
		\end{minipage}
	\end{figure}
\end{frame}


\begin{frame}{例题:判断实数和纯虚数}
	\onslide<+->
	\begin{example}
		实数 $x$ 取何值时, $z=(x^2-3x-4)+(x^2-5x-6)\ii$ 是:
		\begin{enumerate}[<*>]
			\item 实数;
			\item 纯虚数.
		\end{enumerate}
	\end{example}
	\onslide<+->
	\begin{solution}
		\begin{enumerate}
			\item $\Im z=x^2-5x-6=0$, 即 $x=-1$ 或 $6$.
			\item $\Re z=x^2-3x-4=0$, 即 $x=-1$ 或 $4$.
				\onslide<+->{%
					但同时要求 $\Im z=x^2-5x-6\neq 0$, 因此 $x\neq -1$.
				}\onslide<+->{%
					故 $x=4$.
				}
		\end{enumerate}
	\end{solution}
	\onslide<+->
	\begin{exercise}
		若 $x^2(1+\ii)+x(5+4\ii)+4+3\ii$ 是纯虚数, 则实数 $x=$\fillblankframe{$-4$}.
	\end{exercise}
\end{frame}


\subsection{复数的代数运算}

\begin{frame}{复数的加法与减法}
	\onslide<+->
	设 $z_1=x_1+y_1\ii,z_2=x_2+y_2\ii$.
	\onslide<+->
	定义复数的加法和减法:
	\onslide<+->
	\[
		z_1+z_2=(x_1+x_2)+(y_1+y_2)\ii,\quad
		\visible<+->{z_1-z_2=(x_1-x_2)+(y_1-y_2)\ii.}
	\]
	\onslide<+->
	复数的加减法与其对应的向量 $\overrightarrow{OZ}$ 的加减法是一致的.
	\onslide<1->
	\begin{center}
		\begin{tikzpicture}[scale=.8]
			\draw[cstaxis] (-2,0)--(4,0);
			\draw[cstaxis] (0,-3)--(0,2.5);
			\coordinate (O) at (0,0);
			\coordinate [label=right:\textcolor{main}{$z_1$}] (Z1) at (2.5,-1);
			\coordinate [label=above:\textcolor{main}{$z_2$}] (Z2) at (1.5,2);
			\begin{scope}[visible on=<3->]
				\coordinate [label=above right:\textcolor{second}{$z_1+z_2$}] (P) at ($(Z1)+(Z2)$);
				\draw[cstcurve,cstra,second] (O)--(P);
				\draw[cstdash] (Z2)--(P)--(Z1);
			\end{scope}
			\begin{scope}[visible on=<4->]
				\coordinate [label=below:\textcolor{third}{$z_1-z_2$}] (M) at ($(Z1)-(Z2)$);
				\coordinate [label=left:{$-z_2$}] (neg) at ($(O)-(Z2)$);
				\draw[cstcurve,cstra,third] (O)--(M);
				\draw[cstdash,cstra] (O)--(neg);
				\draw[cstdash] (Z1)--(M)--(neg);
			\end{scope}
			\draw[cstcurve,cstra,main] (O)--(Z1);
			\draw[cstcurve,cstra,main] (O)--(Z2);
		\end{tikzpicture}
	\end{center}
\end{frame}


\begin{frame}{复数的乘除法}
	\onslide<+->
	\alert{规定 $\ii\cdot \ii=-1$} 并定义复数的乘法:
	\onslide<+->
	\begin{align*}
		z_1\cdot z_2&=(x_1+y_1\ii)(x_2+y_2\ii)=x_1\cdot x_2+x_1\cdot y_2\ii+y_1\ii\cdot x_2+y_1\ii\cdot y_2\ii\\
		&=(x_1x_2-y_1y_2)+(x_1y_2+x_2y_1)\ii,
	\end{align*}
	\onslide<+->
	由此可得复数的除法:
	\[\frac{z_1}{z_2}=\frac{(x_1+y_1\ii)(x_2-y_2\ii)}{x_2^2+y_2^2}=\frac{x_1x_2+y_1y_2}{x_2^2+y_2^2}+\frac{x_2y_1-x_1y_2}{x_2^2+y_2^2}\ii.
	\]

	\onslide<+->
	对于正整数 $n$, 定义 $z$ 的 \emph{$n$ 次幂}为 $n$ 个 $z$ 相乘.

	\onslide<+->
	当 $z\neq 0$ 时, 还可以定义 $z^0=1,z^{-n}=\dfrac1{z^n}$.
\end{frame}


\begin{frame}{例: 单位根}
	\onslide<+->
	\begin{example}
		\begin{enumerate}
			\item $\ii^2=-1,\ii^3=-\ii ,\ii^4=1$.
			\onslide<+->{%
			一般地, 对于整数 $n$, 
			\[
				\ii^{4n}=1,\quad \ii^{4n+1}=i,\quad
				\ii^{4n+2}=-1,\quad \ii^{4n+3}=-\ii.
			\]
			}
			\vspace{-\baselineskip}
			\item 令 $\omega=\dfrac{-1+\sqrt 3\ii}2$, 则 $\omega^2=\dfrac{-1-\sqrt3\ii}2,\omega^3=1$.
			\item 令 $z=1+\ii$, \onslide<+->{则
			\[z^2=2\ii,\quad z^3=-2+2\ii,\quad z^4=-4,\quad z^8=16=2^4.\]}
		\end{enumerate}
		\vspace{-\baselineskip}
	\end{example}
	\onslide<+->
	将满足 $z^n=1$ 的复数 $z$ 称为 \emph{$n$ 次单位根}.
	\onslide<+->
	那么 $1,\ii,-1,-\ii $ 是 $4$ 次单位根, $1,\omega,\omega^2$ 是 $3$ 次单位根, $-\omega$ 是 $6$ 次单位根.
\end{frame}


\begin{frame}{例: 单位根}
	\onslide<+->
	\begin{example}
		化简 $1+\ii+\ii^2+\ii^3+\ii^4$.
	\end{example}
	\onslide<+->
	\begin{solution}
		根据等比数列求和公式,
		\[
			1+\ii+\ii^2+\ii^3+\ii^4
			=\frac{\ii^5-1}{\ii-1}
			\visible<+->{=\frac{\ii-1}{\ii-1}=1.}
		\]
	\end{solution}
	\onslide<+->
	\begin{exercise}
		化简 $\Bigl(\dfrac{1-\ii}{1+\ii}\Bigr)^{2026}$=\fillblankframe{$-1$}.
	\end{exercise}
\end{frame}


\begin{frame}{复数域\noexer}
	\onslide<+->
	复数全体构成一个\emph{域}.
	\onslide<+->
	所谓的域, 是指带有如下内容和性质的集合
	\begin{itemize}\bf
		\item 包含 $0,1$, 且有四则运算;
		\item 满足加法结合/交换律, 乘法结合/交换/分配律;
		\item 对任意 $a$, $a+0=a\times 1=a$.
	\end{itemize}
	\onslide<+->
	有理数全体 $\BQ$, 实数全体 $\BR$ 也构成域, 它们是 $\BC$ 的子域.
	\onslide<+->
	与有理数域和实数域有着本质不同的是, 复数域是\emph{代数闭域}:
	\onslide<+->
	对于任何次数 $n\ge 1$ 的复系数多项式
	\[
		p(z)=z^n+c_{n-1}z^{n-1}+\cdots+c_1z+c_0,
	\]
	都存在复数 $z_0$ 使得 $p(z_0)=0$.
	\onslide<+->
	由此不难知道, 复系数多项式可以因式分解成一次多项式的乘积.
	\onslide<+->
	我们会在第五章证明该结论.
\end{frame}


\begin{frame}{复数域不是有序域\noexer}
	\onslide<+->
	在 $\BQ,\BR$ 上可以定义出一个好的大小关系,
	\onslide<+->
	换言之它们是有序域, 即存在一个满足下述性质的 $>$:
	\begin{itemize}\bf
		\item 若 $a\neq b$, 则要么 $a>b$, 要么 $b>a$;
		\item 若 $a>b$, 则对于任意 $c$, $a+c>b+c$;
		\item 若 $a>b,c>0$, 则 $ac>bc$.
	\end{itemize}
	\onslide<+->
	而 \alert{$\BC$ 却不是有序域}.
	\onslide<+->
	如果 $\ii>0$, 则
	\[
		-1=\ii\cdot \ii>0,\quad -\ii =-1\cdot \ii>0.
	\]
	\onslide<+->
	于是 $0>\ii$, 矛盾! 同理 $\ii<0$ 也不可能.
\end{frame}


\subsection{共轭复数}

\begin{frame}{共轭复数}
	\onslide<+->
	\begin{definition}
		称 $z$ 在复平面关于实轴的对称点为它的\emph{共轭复数 $\ov z$}.
		换言之, $\ov{x+y\ii}=x-y\ii$.
	\end{definition}
	\onslide<+->
	\begin{exercise}
		$z$ 关于虚轴的对称点是\fillblankframe{$-\ov z$}.
	\end{exercise}
	\onslide<+->
	从定义出发, 不难验证共轭复数满足如下性质:
	\begin{enumerate}\bf
		\item $z$ 是 $\ov z$ 的共轭复数.
		\item $\ov{z_1\pm z_2}=\ov{z_1}\pm\ov{z_2},\ 
		\ov{z_1\cdot z_2}=\ov{z_1}\cdot\ov{z_2},\ 
		\ov{z_1/z_2}=\ov{z_1}/\ov{z_2}$.
		\item $z\ov{z}=(\Re z)^2+(\Im z)^2$.
		\item $z+\ov z=2\Re z,\ z-\ov z=2\ii\Im z$.
		\item $z=\ov z\iff z$ 是实数; $z=-\ov z\iff z$ 是纯虚数或 $z=0$.
	\end{enumerate}
	\onslide<+->
	\enumnum4表明了 $x,y$ 可以用 $z,\ov z$ 表出.
	\onslide<+->
	\enumnum2表明共轭复数和四则运算交换.
	\onslide<+->
	这意味着使用共轭复数进行计算和证明,往往比直接使用 $x,y$ 表达的形式更简单.
\end{frame}


\begin{frame}{例题:共轭复数证明等式}
	\onslide<+->
	\begin{example}
		证明 $z_1\cdot\ov{z_2}+\ov{z_1}\cdot z_2=2\Re(z_1\cdot\ov{z_2})$.
	\end{example}
	\onslide<+->
	我们可以设 $z_1=x_1+y_1\ii,z_2=x_2+y_2\ii$, 然后代入等式两边化简并比较实部和虚部得到.
	\onslide<+->
	但我们利用共轭复数可以更简单地证明它.
	\onslide<+->
	\begin{proof}
		\onslide<+->{%
			由于 $\ov{z_1\cdot\ov{z_2}}=\ov{z_1}\cdot\ov{\ov{z_2}}=\ov{z_1}\cdot z_2$, 
		}\onslide<+->{%
			因此
			\[z_1\cdot\ov{z_2}+\ov{z_1}\cdot z_2
				=z_1\cdot\ov{z_2}+\ov{z_1\cdot\ov{z_2}}
				=2\Re(z_1\cdot\ov{z_2}).\qedhere
	\]
		}
		\vspace{-\baselineskip}
	\end{proof}
\end{frame}


\begin{frame}{例题:共轭复数判断实数}
	\onslide<+->
	\begin{example}
		设 $z=x+y\ii$ 且 $y\neq 0,\pm1$. 证明: $x^2+y^2=1$ 当且仅当 $\dfrac z{1+z^2}$ 是实数.
	\end{example}
	\onslide<+->
	\begin{proof}
		$\dfrac z{1+z^2}$ 是实数当且仅当
	\[
		\frac z{1+z^2}=\ov{\Bigl(\frac z{1+z^2}\Bigr)}=\frac{\ov z}{1+{\ov z}^2},
	\]
		\onslide<+->{%
			即
			\[z(1+{\ov z}^2)=\ov z(1+z^2),\quad (z-\ov z)(z\ov z-1)=0.
	\]
		}\onslide<+->{%
			由 $y\neq0$ 可知 $z\neq \ov z$.
		}\onslide<+->{%
			故上述等式等价于 $z\ov z=1$, 即 $x^2+y^2=1$.\qedhere
		}
	\end{proof}
\end{frame}

\begin{frame}{例: 复数的代数计算}
	\onslide<+->
	由于 $z\ov z$ 是一个实数,
	\onslide<+->
	因此在做复数的除法运算时, 可以利用下式将其转化为乘法:
	\[
		\dfrac{z_1}{z_2}=\dfrac{z_1\ov{z_2}}{z_2\ov{z_2}}=\dfrac{z_1\ov{z_2}}{x_2^2+y_2^2}.
	\]
	\vspace{-.5\baselineskip}
	\onslide<+->
	\begin{example}
		$z=-\dfrac1\ii-\dfrac{3\ii}{1-\ii }$, 求 $\Re z,\Im z$ 以及 $z\ov z$.
	\end{example}
	\onslide<+->
	\begin{solution}
	\[
		z=-\frac1\ii-\frac{3\ii}{1-\ii }
		\onslide<+->{=\ii-\frac{3\ii-3}2=\frac32-\half \ii,}
	\]
		\onslide<+->{%
			因此
				\[\Re z=\frac32,\quad\Im z=-\half ,\quad
				z\ov z=\Bigl(\frac32\Bigr)^2+\Bigl(-\half\Bigr)^2=\frac52.
	\]
		}\vspace{-\baselineskip}
	\end{solution}
\end{frame}


\begin{frame}{例: 复数的代数计算}
	\onslide<+->
	\begin{example}
		设 $z_1=5-5\ii,z_2=-3+4\ii$, 求 $\ov{\Bigl(\dfrac{z_1}{z_2}\Bigr)}$.
	\end{example}
	\onslide<+->
	\begin{solution}
		\begin{align*}
			\frac{z_1}{z_2}&=\frac{5-5\ii}{-3+4\ii}
			\onslide<+->{=\frac{(5-5\ii)(-3-4\ii)}{(-3)^2+4^2}}\\
			&\onslide<+->{=\frac{(-15-20)+(-20+15)\ii}{25}}
			\onslide<+->{=-\frac75-\frac15\ii,}
		\end{align*}
		\onslide<+->{%
			因此 $\ov{\Bigl(\dfrac{z_1}{z_2}\Bigr)}=-\dfrac75+\dfrac15\ii$.
		}
	\end{solution}
\end{frame}

