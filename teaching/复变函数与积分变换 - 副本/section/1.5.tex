\section{复变函数}

\subsection{复变函数的定义}
\begin{frame}{复变函数的定义}
	\onslide<+->
	所谓的\emph{映射}, 就是两个集合之间的一种对应 $f:A\to B$, 使得对于每一个 $a\in A$, 有一个唯一确定的 $b=f(a)$ 与之对应.
	\begin{itemize}
		\item 当 $A$ 和 $B$ 都是实数集合的子集时, 它就是一个实变函数.
		\item 当 $A$ 和 $B$ 都是复数集合的子集时, 它就是一个\emph{复变函数}.
	\end{itemize}

	\onslide<+->
	\begin{example}
		$f(z)=\Re z,\arg z,|z|$, $z^n$ ($n$ 为整数), $\dfrac{z+1}{z^2+1}$ 都是复变函数.
	\end{example}

	\onslide<+->
	\begin{definition}
		\begin{itemize}
			\item 称 $A$ 为 函数 $f$ 的\emph{定义域}.
			\item 称 $\set{w=f(z)\mid z\in A}$ 为它的\emph{值域}.
		\end{itemize} 
	\end{definition}
	\onslide<+->
	上述函数的定义域和值域分别是什么?
\end{frame}


\begin{frame}{多值复变函数}
	\onslide<+->
	在复变函数理论中, 我们常常会遇到\emph{多值的复变函数}, 也就是说一个 $z\in A$ 可能有多个 $w$ 与之对应.
	\onslide<+->
	例如 $\Arg z,\sqrt[n]z$ 等.
	\onslide<+->
	为了方便研究, 我们常常需要对每一个 $z$, 选取固定的一个 $f(z)$ 的值.
	\onslide<+->
	这样我们得到了这个多值函数的一个\emph{单值分支}.
	\onslide<+->
	\begin{example}
		$\arg z$ 是无穷多值函数 $\Arg z$ 的一个单值分支.
	\end{example}

	\onslide<+->
	在考虑多值的情况下, 复变函数总有反函数.
	\onslide<+->
	如果 $f$ 和 $f^{-1}$ 都是单值的, 则称 $f$ 是\emph{一一对应}.
	\onslide<+->
	\begin{example}
		$f(z)=z^n$ 的反函数就是 $f^{-1}(w)=\sqrt[n]{w}$.
		\onslide<+->{%
			当 $n=\pm1$ 时, $f$ 是一一对应.
		}
	\end{example}

	\onslide<+->
	若无特别声明, 本课程中\alert{复变函数总是指单值的复变函数}.
\end{frame}


\subsection{映照}
\begin{frame}{映照}
	\onslide<+->
	大部分复变函数的图像无法在三维空间中表示出来.
	\onslide<+->
	为了直观理解和研究, 我们用两个复平面($z$ 复平面和 $w$ 复平面)之间的\emph{映照}来表示这种对应关系,
	\onslide<+->
	其中 
	\[w=u+\ii v=u(x,y)+\ii v(x,y)
	\]
	的实部和虚部是两个二元实变函数.
	\onslide<+->
	\begin{center}
		\begin{tikzpicture}
			\begin{scope}[xshift=-25mm]
				\draw[cstaxis] (-2,0)--(2,0);
				\draw[cstaxis] (0,-1.5)--(0,1.5);
				\draw
					(2,0) node[above] {$x$}
					(0,1.5) node[left] {$y$}
					(0,-1.5) node[below,main] {$z$ 复平面};
				\draw[cstcurve,main,smooth] plot coordinates {(-1.5,0) (-1.7,-.4) (-.3,-.9) (.5,-.7) (.9,0) (1.1,1) (-.3,1.2) (-.7,1) (-1.5,0)};
				\coordinate (a) at (.5,.8);
				\coordinate (b) at (.5,.5);
				\coordinate (c) at (-.3,.3);
			\end{scope}
			\begin{scope}[xshift=25mm]
				\draw[cstaxis] (-2,0)--(2,0);
				\draw[cstaxis] (0,-1.5)--(0,1.5);
				\draw
					(2,0) node[above] {$u$}
					(0,1.5) node[left] {$v$}
					(0,-1.5) node[below,second] {$w$ 复平面};
				\draw[cstcurve,smooth,second] plot coordinates {(-1.3,0) (-.5,-.5) (0,-.8) (.5,-.5) (1,0) (1.3,.9) (.8,1.2) (-.5,.8) (-1.3,0)};
				\coordinate (A) at (.3,.7) circle;
				\coordinate (B) at (-.3,-.3) circle;
			\end{scope}
			\draw[cstdash,smooth,third,cstra] (a) to [bend left=25] (A);
			\draw[cstdash,smooth,third,cstra] (b)to [bend right=15] (B);
			\draw[cstdash,smooth,third,cstra] (c) to [bend right=25] (B);
			\fill[cstdot,main] (a) circle;
			\fill[cstdot,main] (b) circle;
			\fill[cstdot,main] (c) circle;
			\fill[cstdot,second] (A) circle;
			\fill[cstdot,second] (B) circle;
		\end{tikzpicture}
	\end{center}
\end{frame}


\begin{frame}{例: 映照}
	\onslide<+->
	\begin{example}
		函数 $w=\ov z$.
		\onslide<+->{%
			如果把 $z$ 复平面和 $w$ 复平面重叠放置, 则这个映照对应的是关于 $z$ 轴的翻转变换.
		}\onslide<+->{%
			它把任一区域映成和它全等的区域, 且 $u=x,v=-y$.
			\begin{center}
				\begin{tikzpicture}
					\begin{scope}[xshift=-25mm]
						\draw[cstaxis] (-2,0)--(2,0);
						\draw[cstaxis] (0,-1.5)--(0,1.5);
						\draw
							(2,0) node[above] {$x$}
							(0,1.5) node[left] {$y$}
							(0,-1.5) node[below,main] {$z$ 复平面};
						\draw[cstcurve,main,smooth] plot coordinates {(-1.5,0) (-1.7,-.4) (-.3,-.9) (.5,-.7) (.9,0) (1.1,1) (-.3,1.2) (-.7,1) (-1.5,0)};
						\coordinate (a) at (-1.2,-.3);
						\coordinate (b) at (.6,.9);
						\coordinate (c) at ($.8*(a)+.2*(b)$);
						\coordinate (d) at ($.5*(a)+.5*(b)$);
						\coordinate (e) at ($.2*(a)+.8*(b)$);
						\draw[cstcurve,main] (a)--(b);
					\end{scope}
					\begin{scope}[xshift=25mm]
						\draw[cstaxis] (-2,0)--(2,0);
						\draw[cstaxis] (0,-1.5)--(0,1.5);
						\draw
							(2,0) node[above] {$u$}
							(0,1.5) node[left] {$v$}
							(0,-1.5) node[below,second] {$w$ 复平面};
						\draw[cstcurve,second,smooth] plot coordinates {(-1.5,0) (-1.7,.4) (-.3,.9) (.5,.7) (.9,0) (1.1,-1) (-.3,-1.2) (-.7,-1) (-1.5,0)};
						\coordinate (A) at (-1.2,.3);
						\coordinate (B) at (.6,-.9);
						\coordinate (C) at ($.8*(A)+.2*(B)$);
						\coordinate (D) at ($.5*(A)+.5*(B)$);
						\coordinate (E) at ($.2*(A)+.8*(B)$);
						\draw[cstcurve,second] (A)--(B);
					\end{scope}
					\draw[cstdash,smooth,third,cstra] (c) to[bend right=15] (C);
					\draw[cstdash,smooth,third,cstra] (d) to[bend right=25] (D);
					\draw[cstdash,smooth,third,cstra] (e) to[bend left=45] (E);
					\fill[cstdot,main] (c) circle;
					\fill[cstdot,main] (d) circle;
					\fill[cstdot,main] (e) circle;
					\fill[cstdot,second] (C) circle;
					\fill[cstdot,second] (D) circle;
					\fill[cstdot,second] (E) circle;
				\end{tikzpicture}
			\end{center}
		}
	\end{example}
\end{frame}


\begin{frame}{例: 映照}
	\onslide<+->
	\begin{example}
		函数 $w=az$.
		\onslide<+->{%
			设 $a=r\ee^{\ii\theta}$, 则这个映照对应的是一个旋转映照(逆时针旋转 $\theta$)和一个相似映照(放大为 $r$ 倍)的复合.
		}\onslide<+->{%
			它把任一区域映成和它相似的区域.
			\begin{center}
				\begin{tikzpicture}
					\begin{scope}[xshift=-25mm]
						\draw[cstaxis] (-2,0)--(2,0);
						\draw[cstaxis] (0,-1.5)--(0,1.5);
						\draw
							(2,0) node[above] {$x$}
							(0,1.5) node[left] {$y$}
							(0,-1.5) node[below,main] {$z$ 复平面};
						\draw[cstcurve,main,smooth] plot coordinates {(-1.5,0) (-1.7,-.4) (-.3,-.9) (.5,-.7) (.9,0) (1.1,1) (-.3,1.2) (-.7,1) (-1.5,0)};
						\coordinate (a) at (-1.2,-.3);
						\coordinate (b) at (.6,.9);
						\coordinate (c) at ($.8*(a)+.2*(b)$);
						\coordinate (d) at ($.5*(a)+.5*(b)$);
						\coordinate (e) at ($.2*(a)+.8*(b)$);
						\draw[cstcurve,main] (a)--(b);
					\end{scope}
					\begin{scope}[xshift=25mm]
						\draw[cstaxis] (-2,0)--(2,0);
						\draw[cstaxis] (0,-1.5)--(0,1.5);
						\draw
							(2,0) node[above] {$u$}
							(0,1.5) node[left] {$v$}
							(0,-1.5) node[below,second] {$w$ 复平面};
						\draw[cstcurve,second,smooth,scale=.8,rotate=90] plot coordinates {(-1.5,0) (-1.7,-.4) (-.3,-.9) (.5,-.7) (.9,0) (1.1,1) (-.3,1.2) (-.7,1) (-1.5,0)};
						\coordinate (A) at (.24,-.96);
						\coordinate (B) at (-.72,.48);
						\coordinate (C) at ($.8*(A)+.2*(B)$);
						\coordinate (D) at ($.5*(A)+.5*(B)$);
						\coordinate (E) at ($.2*(A)+.8*(B)$);
						\draw[cstcurve,second] (A)--(B);
					\end{scope}
					\draw[cstdash,smooth,third,cstra] (c) to[bend right=25] (C);
					\draw[cstdash,smooth,third,cstra] (d) to[bend right=10] (D);
					\draw[cstdash,smooth,third,cstra] (e) to[bend left=20] (E);
					\fill[cstdot,main] (c) circle;
					\fill[cstdot,main] (d) circle;
					\fill[cstdot,main] (e) circle;
					\fill[cstdot,second] (C) circle;
					\fill[cstdot,second] (D) circle;
					\fill[cstdot,second] (E) circle;
				\end{tikzpicture}
			\end{center}
		}
	\end{example}
\end{frame}


\begin{frame}{例: 映照}
	\onslide<+->
	\begin{example}
		函数 $w=z^2$.
		\onslide<+->{%
			这个映照把 $z$ 的辐角增大一倍, 因此它会把角形区域变换为角形区域, 并将夹角放大一倍.
		}\onslide<+->{
			\begin{center}
				\begin{tikzpicture}
					\begin{scope}[xshift=-25mm]
						\draw[cstaxis] (-2,0)--(2,0);
						\draw[cstaxis] (0,-1.5)--(0,1.5);
						\draw
							(2,0) node[above] {$x$}
							(0,1.5) node[left] {$y$}
							(0,-1.5) node[below,main] {$z$ 复平面};
						\fill[cstfille1] (0,0)--({1.5*cos(37.5)},{1.5*sin(37.5)}) arc(37.5:7.5:1.5)--cycle;
						\draw[cstcurve,main] (0,0)--({1.5*cos(37.5)},{1.5*sin(37.5)});
						\draw[cstcurve,main] (0,0)--({1.5*cos(7.5)},{1.5*sin(7.5)});
						\coordinate (a) at (0,1);
						\coordinate (b) at (.8,1.2);
						\coordinate (c) at (-.6,-.3);
					\end{scope}
					\begin{scope}[xshift=25mm]
						\draw[cstaxis] (-2,0)--(2,0);
						\draw[cstaxis] (0,-1.5)--(0,1.5);
						\draw
							(2,0) node[above] {$u$}
							(0,1.5) node[left] {$v$}
							(0,-1.5) node[below,second] {$w$ 复平面};
						\fill[cstfille2] (0,0)--({1.8*cos(75)},{1.8*sin(75)}) arc(75:15:1.8)--cycle;
						\draw[cstcurve,second] (0,0)--({1.8*cos(75)},{1.8*sin(75)});
						\draw[cstcurve,second] (0,0)--({1.8*cos(15)},{1.8*sin(15)});
						\coordinate (A) at (-1,0);
						\coordinate (B) at (-.8,1.92);
						\coordinate (C) at (.27,.36);
					\end{scope}
					\draw[cstdash,smooth,third,cstra] (a) to[bend left=10] (A);
					\draw[cstdash,smooth,third,cstra] (b) to[bend left=20] (B);
					\draw[cstdash,smooth,third,cstra] (c) to[bend right=25] (C);
					\fill[cstdot,fill=main] (a) circle;
					\fill[cstdot,fill=main] (b) circle;
					\fill[cstdot,fill=main] (c) circle;
					\fill[cstdot,fill=second] (A) circle;
					\fill[cstdot,fill=second] (B) circle;
					\fill[cstdot,fill=second] (C) circle;
				\end{tikzpicture}
			\end{center}
		}
	\end{example}
\end{frame}


\begin{frame}{例: 映照}
	\onslide<+->
	\begin{example}[续]
		由于 $u=x^2-y^2,v=2xy$.
		\onslide<+->{%
			因此它把 $z$ 复平面上两族分别以直线 $y=\pm x$ 和坐标轴为渐近线的等轴双曲线 $x^2-y^2=c_1,2xy=c_2$
		}\onslide<+->{%
			分别映射为 $w$ 复平面上的两族平行直线 $u=c_1,v=c_2$.
		}\onslide<2->{
			\begin{center}
				\begin{tikzpicture}
					\begin{scope}[xshift=-25mm]
						\draw[cstaxis] (-2,0)--(2,0);
						\draw[cstaxis] (0,-1.5)--(0,1.5);
						\begin{scope}[cstcurve,main,smooth]
							\draw (-1.2,-1.2)--(1.2,1.2);
							\draw (-1.2,1.2)--(1.2,-1.2);
							\draw[domain=-35:35]
								plot ({sec(\x)},{tan(\x)})
								plot ({-sec(\x)},{tan(\x)})
								plot ({tan(\x)},{sec(\x)})
								plot ({tan(\x)},{-sec(\x)});
							\draw[domain=-46:46]
								plot ({(.8*sec(\x))},{0.8*tan(\x)})
								plot ({(-.8*sec(\x))},{0.8*tan(\x)})
								plot ({0.8*tan(\x)},{0.8*sec(\x)})
								plot ({0.8*tan(\x)},{0.8*-sec(\x)});
							\draw[domain=-57:57]
								plot ({(.6*sec(\x))},{0.6*tan(\x)})
								plot ({(-.6*sec(\x))},{0.6*tan(\x)})
								plot ({0.6*tan(\x)},{0.6*sec(\x)})
								plot ({0.6*tan(\x)},{0.6*-sec(\x)});
							\draw[domain=-68:68]
								plot ({(.4*sec(\x))},{0.4*tan(\x)})
								plot ({(-.4*sec(\x))},{0.4*tan(\x)})
								plot ({0.4*tan(\x)},{0.4*sec(\x)})
								plot ({0.4*tan(\x)},{0.4*-sec(\x)});
						\end{scope}
						\begin{scope}[cstcurve,second,smooth,rotate=45,visible on=<4->]
							\draw (-1.2,-1.2)--(1.2,1.2);
							\draw (-1.2,1.2)--(1.2,-1.2);
							\draw[domain=-35:35]
								plot ({sec(\x)},{tan(\x)})
								plot ({-sec(\x)},{tan(\x)})
								plot ({tan(\x)},{sec(\x)})
								plot ({tan(\x)},{-sec(\x)});
							\draw[domain=-46:46]
								plot ({(.8*sec(\x))},{0.8*tan(\x)})
								plot ({(-.8*sec(\x))},{0.8*tan(\x)})
								plot ({0.8*tan(\x)},{0.8*sec(\x)})
								plot ({0.8*tan(\x)},{0.8*-sec(\x)});
							\draw[domain=-57:57]
								plot ({(.6*sec(\x))},{0.6*tan(\x)})
								plot ({(-.6*sec(\x))},{0.6*tan(\x)})
								plot ({0.6*tan(\x)},{0.6*sec(\x)})
								plot ({0.6*tan(\x)},{0.6*-sec(\x)});
							\draw[domain=-68:68]
								plot ({(.4*sec(\x))},{0.4*tan(\x)})
								plot ({(-.4*sec(\x))},{0.4*tan(\x)})
								plot ({0.4*tan(\x)},{0.4*sec(\x)})
								plot ({0.4*tan(\x)},{0.4*-sec(\x)});
						\end{scope}
					\end{scope}
					\begin{scope}[xshift=25mm,visible on=<3->]
						\draw[cstaxis] (-2,0)--(2,0);
						\draw[cstaxis] (0,-1.5)--(0,1.5);
						\begin{scope}[cstcurve,second,visible on=<5->]
							\draw (-1.3,-1.2)--(1.3,-1.2);
							\draw (-1.3,-0.9)--(1.3,-.9);
							\draw (-1.3,-0.6)--(1.3,-.6);
							\draw (-1.3,-0.3)--(1.3,-.3);
							\draw (-1.3,0)--(1.3,0);
							\draw (-1.3,0.3)--(1.3,.3);
							\draw (-1.3,0.6)--(1.3,.6);
							\draw (-1.3,0.9)--(1.3,.9);
							\draw (-1.3,1.2)--(1.3,1.2);
						\end{scope}
						\begin{scope}[cstcurve,main,rotate=90]
							\draw (-1.3,-1.2)--(1.3,-1.2);
							\draw (-1.3,-0.9)--(1.3,-.9);
							\draw (-1.3,-0.6)--(1.3,-.6);
							\draw (-1.3,-0.3)--(1.3,-.3);
							\draw (-1.3,0)--(1.3,0);
							\draw (-1.3,0.3)--(1.3,.3);
							\draw (-1.3,0.6)--(1.3,.6);
							\draw (-1.3,0.9)--(1.3,.9);
							\draw (-1.3,1.2)--(1.3,1.2);
						\end{scope}
					\end{scope}
				\end{tikzpicture}
			\end{center}
		}
	\end{example}
\end{frame}


\begin{frame}{例: 映照的像}\small
	\onslide<+->
	\begin{example}
		求下列集合在映照 $w=z^2$ 下的像.
		\begin{enumerate}
			\item 线段 $0<|z|<2,\arg z=\pi/2$.
			\item 双曲线 $x^2-y^2=4$.
			\item 扇形区域 $0<\arg z<\pi/4,0<|z|<2$.
		\end{enumerate}
	\end{example}
	\onslide<+->
	\begin{solution}
		\begin{enumerate}
			\item 设 $z=r\ee^{\frac{\pi\ii}2}=ir$, 则 $w=z^2=-r^2$.
				\onslide<+->{%
					因此它的像是线段 $0<|w|<4,\arg w=\pi$.
				}
			\item 由于 $w=u+\ii v=z^2=(x^2-y^2)+2xyi$.
				\onslide<+->{%
					因此 $u=x^2-y^2=4,v=2xy$.
				}\onslide<+->{%
					可以说明当 $u=4$ 时, 对任意 $v$, $u+\ii v$ 都是该双曲线上某一点的像.
				}\onslide<+->{%
					所以这条双曲线的像是直线 $\Re w=4$.
				}
			\item 设 $z=r\ee^{\ii\theta}$, 则 $w=r^2\ee^{2\ii\theta}$.
				\onslide<+->{%
					因此它的像是扇形区域 $0<\arg w<\pi/2,0<|w|<4$.
				}
		\end{enumerate}
	\end{solution}
\end{frame}


\begin{frame}{例: 映照的像}
	\onslide<+->
	\begin{example}
		求圆周 $|z|=2$ 在映照 $w=\dfrac{z+1}{z-1}$ 下的像.
	\end{example}

	\onslide<+->
	\begin{solution}
		由于 $z=\dfrac{w+1}{w-1}$, $\abs{\dfrac{w+1}{w-1}}=2$,
		\onslide<+->{%
			因此
			\[|w+1|=2|w-1|,\quad w\ov w+w+\ov w+1=4w\ov w-4w-4\ov w+4,\]}
		\onslide<+->{%
			\[w\ov w-\frac53 w-\frac53\ov w+1=0,\quad \abs{w-\frac53}^2=\dfrac{16}9,
	\]
		}\onslide<+->{%
			即 $\abs{w-\dfrac53}=\dfrac43$, 是一个圆周.
		}
	\end{solution}
\end{frame}

