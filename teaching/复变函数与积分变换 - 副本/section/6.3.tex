\section{拉普拉斯变换}


\begin{frame}{拉普拉斯变换}
\onslide<+->
傅里叶变换对函数要求过高, 这使得在很多时候无法应用它, 或者要引入复杂的广义函数.
\onslide<+->
对于一般的 $\varphi(t)$, 为了让它绝对可积, 我们考虑
\[\varphi(t)u(t)e^{-\beta t},\quad\beta>0.\]
\onslide<+->
它的傅里叶变换为
\[\msf[\varphi(t)u(t)e^{-\beta t}]=\int_0^{+\infty}\varphi(t)e^{-(\beta+j\omega)t}\diff t=\int_0^{+\infty}\varphi(t)e^{-st}\diff t,\]
其中 $s=\beta+j\omega$.
\onslide<+->
这样的积分在我们遇到的多数情形都是存在的, 只要选择充分大的 $\beta=\Re s$.
\onslide<+->
我们称之为 $\varphi(t)$ 的\emph{拉普拉斯变换}.
\end{frame}


\begin{frame}{拉普拉斯变换存在定理}
\begin{block}{拉普拉斯变换存在定理}
若定义在 $[0,+\infty)$ 上的函数 $f(t)$ 满足
\begin{itemize}
\item $f(t)$ 在任一有限区间上至多只有有限多间断点;
\item 存在 $M,c$ 使得 $|f(t)|\le Me^{ct}$,
\end{itemize}
\onslide<+->
则 $F(s)=\msl[f(t)]$ 在 $\Re s>c$ 上存在且为解析函数.
\end{block}
\onslide<+->
\begin{center}
\begin{tikzpicture}[node distance=-20pt]
  \node[cstnodeg,align=center] (a){原象函数\\$f(t)\vphantom{\displaystyle\int}$};
  \node[cstnodeg,align=center, right=116pt of a](b){象函数\\\alert{$F(s)=\displaystyle\int_0^{+\infty}f(t)e^{-st}\diff t$}};
  \node[above=of a] (a1){\hphantom{原象函数}};
  \node[below=of a] (a2){\hphantom{原象函数}};
  \node[above=of b] (b1){\hphantom{$F(s)=\displaystyle\int_0^{+\infty}f(t)e^{-st}\diff t$}};
  \node[below=of b] (b2){\hphantom{$F(s)=\displaystyle\int_0^{+\infty}f(t)e^{-st}\diff t$}};
  \draw[cstnarrow, dcolora] (a1) -- node[above]{拉普拉斯变换 $\msl$} (b1);
  \draw[cstnarrow, dcolorc] (b2) -- node[below]{拉普拉斯逆变换 $\msl^{-1}$} (a2);
\end{tikzpicture}
\end{center}
\end{frame}


\begin{frame}{例题: 求拉普拉斯变换}
\begin{example}
求 $\msl[e^{kt}]$.
\end{example}
\begin{solution}
\vspace{-\baselineskip}
\begin{align*}
\msl[e^{kt}]&=\int_0^{+\infty}e^{kt}e^{-st}\diff t\\
&\visible<+->{=\int_0^{+\infty}e^{-(s-k)t}\diff t
=-\frac1{s-k}e^{-(s-k)t}\big|_0^{+\infty}}\\
&\visible<+->{=\frac1{s-k}, \quad\Re s>\Re k.}
\end{align*}
\onslide<+->
即 \abox{$\msl[e^{kt}]=\dfrac1{s-k}$}.
\onslide<+->
特别地 \abox{$\msl[1]=\dfrac1s$}.
\end{solution}
\end{frame}


\begin{frame}{拉普拉斯变换的性质}
\begin{block}{线性性质}
\[\msl[\alpha f+\beta g]=\alpha F+\beta G,\quad
\msl^{-1}[\alpha F+\beta G]=\alpha f+\beta g.\]
\end{block}

\begin{block}{延迟性质和位移性质}
\[\msl[e^{s_0t}f(t)]=F(s-s_0),\quad
\msl[f(t-t_0)]=e^{-st_0}F(s), t_0\ge 0.\]
\end{block}

\onslide<+->
例如
\[\msl[1]=\frac1s,\quad\msl[e^{kt}]=\frac1{s-k}.\]
\end{frame}


\begin{frame}{拉普拉斯变换的性质}
\begin{block}{微分性质}
\vspace{-\baselineskip}
\[\aboxeq{\msl[f'(t)]=sF(s)-f(0)},\]
\[\aboxeq{\msl[f''(t)]=s^2F(s)-sf(0)-f'(0)},\]
\[\msl[f^{(n)}(t)]=s^nF(s)-s^{n-1}f(0)-s^{n-2}f'(0)-\cdots-f^{(n-1)}(0).\]
\end{block}

\begin{block}{积分性质}
\[\msl\left[\int_0^t f(\tau)\diff\tau\right]=\frac 1s F(s).\]
\end{block}

\begin{block}{乘多项式性质}
\[\msl[tf(t)]=-F'(s),\quad\msl[t^kf(t)]=(-1)^kF^{(k)}(s).\]
\end{block}
\end{frame}


\begin{frame}{典型例题: 求拉普拉斯变换}
\begin{example}
求 $\msl[\sin{kt}]$.
\end{example}
\begin{solution}
由于
\[\sin{kt}=\frac1{2j}(e^{jkt}-e^{-jkt}),\]
\onslide<+->
因此
\begin{align*}
&\peq\aboxeq{\msl[\sin{kt}]}=\frac1{2j}\left(\msl[e^{jkt}]-\msl[e^{-jkt}]\right)\\
&\visible<+->{=\frac1{2j}\left(\frac1{s-jk}-\frac1{s+jk}\right)}
\visible<+->{\aboxeq{=\frac k{s^2+k^2}},\quad \Re s>\abs{\Im k}.}
\end{align*}
\end{solution}
\end{frame}


\begin{frame}{典型例题: 求拉普拉斯变换}
\begin{exercise}
求 $\msl[\cos{kt}]$.
\end{exercise}
\begin{answer}
\abox{$\msl[\cos{kt}]=\dfrac s{s^2+k^2}$}, $\Re s>\abs{\Im k}$.
\end{answer}
\end{frame}


\begin{frame}{典型例题: 求拉普拉斯变换}
\begin{example}
求 $\msl[t^m]$, 其中 $m$ 是正整数.
\end{example}
\begin{solution}
设 $f(t)=t^m$. 由于 $f(0)=f'(0)=\cdots=f^{(m-1)}(0)=0$,
\onslide<+->
因此
\[\msl[f^{(m)}(t)]=s^m\msl[f(t)]=\msl[m!]=\frac{m!}s,\]
\onslide<+->
\vspace{-\baselineskip}
\[\aboxeq{\msl[t^m]}=\msl[f(t)]\aboxeq{=\frac{m!}{s^{m+1}}},\quad \Re s>0.\]
\onslide<+->
或由
\vspace{-\baselineskip}
\[\msl[t^m]=(-1)^m \msl[1]^{(m)}
=(-1)^m\left(\frac1s\right)^{(m)}=\frac{m!}{s^{m+1}}.\]
\vspace{-\baselineskip}
\end{solution}
\end{frame}


\begin{frame}{拉普拉斯逆变换}
\onslide<+->
拉普拉斯逆变换可以由下述反演公式给出:
\[\msl^{-1}[F(s)]=\frac1{2\pi j}\int_{\beta-j\infty}^{\beta+j\infty}F(s)e^{st}\diff s,\quad t>0\]
\onslide<+->
通过构造恰当的闭路, 可以证明这等价于
\begin{block}{拉普拉斯逆变换定理}
设 $F(s)$ 的所有奇点为 $s_1,\dots,s_k$, 且 $F(\infty)=0$, 则
\[\msl^{-1}[F(s)]=\sum_{k=1}^n \Res\left[F(s)e^{st},s_k\right].\]
\end{block}
\onslide<+->
不过我们只要求掌握如何利用常见函数的拉普拉斯变换来计算逆变换.
\end{frame}


\begin{frame}{例题: 求拉普拉斯逆变换}
\begin{example}
求 $F(s)=\dfrac1{s^2(s^2+1)}$ 的拉普拉斯逆变换.
\end{example}
\begin{solution}
注意到
\[F(s)=\frac1{s^2}-\frac1{s^2+1}.\]
\onslide<+->
因此
\[\msl^{-1}[F(s)]=\msl^{-1}\left[\frac1{s^2}\right]
-\msl^{-1}\left[\frac1{s^2+1}\right]
=t-\sin t.\]
\end{solution}
\end{frame}


\begin{frame}{卷积定理}
\onslide<+->
由于在拉普拉斯变换中, 我们考虑的函数在 $t<0$ 时都是零.
\onslide<+->
因此此时函数的卷积变成了
\[f_1(t)\ast f_2(t)=\int_0^t f_1(\tau)f_2(t-\tau)\diff \tau,\quad t\ge 0.\]
\onslide<+->
此时我们有如下的卷积定理.
\begin{block}{卷积定理}
\[\msl[f_1(t)\ast f_2(t)]=F_1(s)\cdot F_2(s).\]
\end{block}
\end{frame}


\begin{frame}{例题: 求拉普拉斯逆变换}
\beqskip{6pt}
\begin{example}
求 $F(s)=\dfrac1{s(s-1)^2}$ 的拉普拉斯逆变换.
\end{example}
\begin{solution}
通过待定系数, 我们可以得到如下的拆分
\[F(s)=\frac1s-\frac1{s-1}+\frac1{(s-1)^2}.\]
\onslide<+->
而
\[\msl^{-1}\left[\frac1s\right]=1,\quad \msl^{-1}\left[\frac1{s-1}\right]=e^t,\]
\onslide<+->
\[\msl^{-1}\left[\frac1{s^2}\right]=t\implies\msl^{-1}\left[\frac1{(s-1)^2}\right]=te^t,\]
\end{solution}
\endgroup
\end{frame}


\begin{frame}{例题: 求拉普拉斯逆变换}
\begin{solutionc}
因此
\[\msl^{-1}[F(s)]=1+(t-1)e^t.\]
\onslide<+->
我们也可以利用卷积定理来计算.
\onslide<+->
\begin{align*}
\msl^{-1}[F(s)]&=\msl^{-1}\left[\frac1s\right]\ast\msl^{-1}\left[\frac1{(s-1)^2}\right]\\
&\visible<+->{=1\ast te^t=\int_0^t\tau e^\tau\diff \tau}\\
&\visible<+->{=(t-1)e^t+1.}
\end{align*}
\end{solutionc}
\end{frame}


\begin{frame}{使用拉普拉斯变换解微积分方程}
\begin{center}
\begin{tikzpicture}[node distance=40pt]
\node[cstnodeg] (p1){微分方程或积分方程};
\node[cstnodeb,right=110pt of p1] (p2){象函数的代数方程};
\node[cstnodeg,below=of p1] (p3){原象函数(方程的解)};
\node[cstnodeb,below=of p2] (p4){象函数};
\draw[cstnarrow,dcolorb] (p1)--node[above]{拉普拉斯变换 $\msl$}(p2);
\draw[cstnarrow,dcolorb] (p4)--node[below]{拉普拉斯逆变换 $\msl^{-1}$}(p3);
\draw[cstnarrow,dcolorb] (p2)--(p4);
\draw[cstnarrow,dcolorc] (p1)--(p3);
\end{tikzpicture}
\end{center}
\end{frame}


\begin{frame}{例题: 使用拉普拉斯变换解微分方程}
\begin{example}
解方程 
  $\begin{cases}
    x''(t)+4x(t)=3\cos t,& \\
    x(0)=x'(0)=0.&
  \end{cases}$
\end{example}
\begin{solution}
设 $\msl[x]=X$,
\onslide<+->
则
\[s^2X+4X=\frac{3s}{s^2+1},\quad
\visible<+->{X(s)=\frac{3s}{(s^2+1)(s^2+4),}}\]
\onslide<+->
\vspace{-\baselineskip}
\[x(t)=\msl^{-1}\left(\frac{s}{s^2+1}-\frac{s}{s^2+4}\right)
\visible<+->{=\cos t-\cos(2t).}\]
\end{solution}
\end{frame}


\begin{frame}{例题: 使用拉普拉斯变换解微分方程}
\beqskip{0pt}
\begin{example}
解方程 
  $\begin{cases}
    x'(t)+2x(t)+2y(t)=10e^{2t},& \\
    -2x(t)+y'(t)+3y(t)=13e^{2t},& \\
    x(0)=1,y(0)=3.&
  \end{cases}$
\end{example}
\vspace{-7pt}
\begin{solution}
设 $\msl[x]=X,\msl[y]=Y$,
\onslide<+->
则
  \[\begin{cases}
    sX-1+2X+2Y=10/(s-2),& \\
    -2X+sY-3+3Y=13/(s-2).&
  \end{cases}\]
\onslide<+->
于是
\[X(s)=\frac1{s-2},\quad Y(s)=\frac3{s-2},\]
\onslide<+->
\[x(t)=\msl^{-1}\left[\frac1{s-2}\right]=e^{2t},\quad
y(t)=\msl^{-1}\left[\frac3{s-2}\right]=3e^{2t}.\]
\end{solution}
\endgroup
\end{frame}


\begin{frame}{例题: 使用拉普拉斯变换解微分方程}
\begin{example}
解方程 $x''(t)-x(t)=0$.
\end{example}
\begin{solution}
设 $a=x(0),b=x'(0),\msl[x]=X$,
\onslide<+->
则
\[\msl[x''(t)]=s^2X(s)-as-b,\]
\onslide<+->
\vspace{-\baselineskip}
\[s^2X(s)-as-b-X(s)=0,\]
\onslide<+->
\vspace{-0.8\baselineskip}
\[X(s)=\frac{as-b}{s^2-1}=\frac{a+b}2\cdot\frac1{s-1}+\frac{a-b}2\cdot\frac1{s+1},\]
\onslide<+->
\vspace{-0.8\baselineskip}
\[x(t)=\msl^{-1}[X(s)]=\frac{a+b}2e^t+\frac{a-b}2e^{-t}.\]
\end{solution}
\end{frame}

% {
% \homework
% \begin{frame}[<*>]{作业}
%   \begin{homeworks}
%     \item(2020年A卷) 用拉普拉斯变换解微分方程
%     \[\begin{cases}y''+4y'+3y=e^{-t},&\\y(0)=y'(0)=1.\end{cases}\]
%     \item(2020年B卷) 用拉普拉斯变换解微分方程
%     \[\begin{cases}y''+y=t,&\\y(0)=1,\quad y'(0)=-2.\end{cases}\]
%     \item(2021年A卷) 用拉普拉斯变换解微分方程
%     \[\begin{cases}y''+4y'+3y=e^t,&\\y(0)=0,\quad y'(0)=2.\end{cases}\]
%     \item(2021年B卷) 用拉普拉斯变换解微分方程
%     \[\begin{cases}y''-3y'+2y=2e^{-t},&\\y(0)=2,\quad y'(0)=-1.\end{cases}\]
%     \item(2022年A卷) 用拉普拉斯变换解微分方程
%     \[\begin{cases}y''+2y=\sin t,&\\y(0)=0,\quad y'(0)=2.\end{cases}\]
%   \end{homeworks}
% \end{frame}
% }
