\section{极限和连续性}


\subsection{无穷远点}
\begin{frame}{数列极限}
	\onslide<+->
	类似于实变函数情形, 我们可以定义复变函数的极限.
	\onslide<+->
	我们先来看数列极限的定义.
	\onslide<+->
	\begin{definition}
		设 $\{z_n\}_{n\ge 1}$ 是一个复数列.
		如果 $\forall \varepsilon>0,\exists N$ 使得当 $n\ge N$ 时 $|z_n-z|<\varepsilon$, 则称 $z$ 是\emph{数列 $\{z_n\}$ 的极限}, 记作 \emph{$\lim\limits_{n\to\infty}z_n=z$}.
	\end{definition}
	\onslide<+->
	如果 $\forall X>0,\exists N$ 使得当 $n\ge N$ 时 $|z_n|>X$, 则记 $\lim\limits_{n\to\infty}z_n=\infty$.
\end{frame}


\begin{frame}{数列极限的等价定义}
	\onslide<+->
	如果我们称
	\[\Uc(\infty,X)=\set{z\in\BC\mid|z|>X}
	\]
	为 \emph{$\infty$ 的(去心)邻域},
	\onslide<+->
	\begin{center}
		\begin{tikzpicture}
			\fill[cstfille1] (0,0) circle (1.2);
			\filldraw[cstcurve,main,fill=white] (0,0) circle (.5);
			\draw[cstaxis] (-1.5,0)--(1.5,0);
			\draw[cstaxis] (0,-1.5)--(0,1.5);
		\end{tikzpicture}
	\end{center}
	\onslide<+->
	那么上述定义可统一表述为:

	\onslide<+->
	\begin{third}{数列极限的等价定义}
		对 $z$ 的任意邻域 $U$, $\exists N$ 使得当 $n\ge N$ 时 $z_n\in U$.
	\end{third}
\end{frame}


\begin{frame}[b]{复球面}
	\onslide<+->
	那么有没有一种看法使得 $\infty$ 的邻域和普通复数的邻域没有差异呢?
	\onslide<+->
	我们将介绍复球面的概念, 它是复数的一种几何表示且自然包含无穷远点 $\infty$.
	\onslide<+->
	这种思想是在黎曼研究多值复变函数时引入的.

	\onslide<+->
	取一个与复平面相切于原点 $z=0$ 的球面.
	\onslide<+->
	过 $O$ 做垂直于复平面的直线, 并与球面相交于另一点 $N$, 称之为北极.

	\onslide<4->
	\begin{center}
		\begin{tikzpicture}
			\coordinate [label=right:\textcolor{white}{$z_1$}] (z1) at (1.65,-.75);
			\fill[cstfill1] (-3.65,-.804)--(-1.85,.804)--(3.65,.804)--(1.85,-.804)--cycle;
			\filldraw[cstcurve,cstfill] (0,1) circle (1);
			\draw[cstdash] (0,1) circle (1 and 0.3);
			\draw[cstaxis] (0,0)--(2.5,0);
			\draw[cstaxis] (0,0)--(-.8,-.9);
			\begin{scope}[visible on=<5->]
				\coordinate [label=above:\textcolor{third}{$N$}] (N) at (0,2);
				\draw[cstdash] (0,0)--(N);
				\fill[cstdot,third] (N) circle;
			\end{scope}
		\end{tikzpicture}
	\end{center}
\end{frame}


\begin{frame}[b]{复球面}
	\begin{itemize}
		\item 对于平面上的任意一点 $z$, 连接北极 $N$ 和 $z$ 的直线一定与球面相交于除 $N$ 以外的唯一一个点 $Z$.
		\item 反之, 球面上除了北极外的任意一点 $Z$, 直线 $NZ$ 一定与复平面相交于唯一一点.
	\end{itemize}
	\onslide<+->
	这样, 球面上除北极外的所有点和全体复数建立了一一对应.

	\onslide<1->
	\begin{center}
		\begin{tikzpicture}
			\fill[cstfill1] (-3.65,-.804)--(-1.85,.804)--(3.65,.804)--(1.85,-.804)--cycle;
			\filldraw[cstcurve,cstfill] (0,1) circle (1);
			\draw[cstdash] (0,1) circle (1 and 0.3);
			\draw[cstdash,third] (0,0) circle (2 and 0.6);
			\coordinate [label=above:\textcolor{third}{$N$}] (N) at (0,2);
			\draw[cstdash] (0,0)--(N);
			\draw[cstaxis] (0,0)--(2.5,0);
			\draw[cstaxis] (0,0)--(-.8,-.9);
			\coordinate [label=right:\textcolor{main}{$z_1$}] (z1) at (1.65,-.75);
			\coordinate [label=left:\textcolor{main}{$Z_1$}] (Z1) at (.6,1);
			\draw[cstcurve,main,cstra] (N)--(z1);
			\fill[cstdot,main] (Z1) circle;
			\begin{scope}[visible on=<2->]
				\coordinate [label=left:\textcolor{second}{$z_2$}] (z2) at (-1,0);
				\coordinate [label=below right:\textcolor{second}{$Z_2$}] (Z2) at (-.7,.6);
				\draw[cstcurve,cstra,second] (N)--(z2);
				\fill[cstdot,second] (Z2) circle;
			\end{scope}
			\fill[cstdot,third] (N) circle;
		\end{tikzpicture}
	\end{center}
\end{frame}


\begin{frame}[b]{复球面: 无穷远点}
	\onslide<+->
	当 $|z|$ 越来越大时, 其对应球面上点也越来越接近 $N$.
	\onslide<+->
	如果我们在复平面上添加一个额外的"点"——\emph{无穷远点}, 记作 \emph{$\infty$}.
	\onslide<+->
	那么\emph{扩充复数集合 $\BC^*=\BC\cup\set\infty$} 就正好和球面上的点一一对应.
	\onslide<+->
	称这样的球面为\emph{复球面}, 称包含无穷远点的复平面为\emph{扩充复平面}或\emph{闭复平面}.
	\onslide<1->
	\begin{center}
		\begin{tikzpicture}
			\fill[cstfill1] (-3.65,-.804)--(-1.85,.804)--(3.65,.804)--(1.85,-.804)--cycle;
			\filldraw[cstcurve,cstfill] (0,1) circle (1);
			\draw[cstdash] (0,1) circle (1 and 0.3);
			\draw[cstdash,third] (0,0) circle (2 and 0.6);
			\coordinate [label=above:\textcolor{third}{$N$}] (N) at (0,2);
			\draw[cstdash] (0,0)--(N);
			\draw[cstaxis] (0,0)--(2.5,0);
			\draw[cstaxis] (0,0)--(-.8,-.9);
			\coordinate [label=right:\textcolor{main}{$z_1$}] (z1) at (1.65,-.75);
			\coordinate [label=left:\textcolor{main}{$Z_1$}] (Z1) at (.6,1);
			\draw[cstcurve,main,cstra] (N)--(z1);
			\fill[cstdot,main] (Z1) circle;
			\coordinate [label=left:\textcolor{second}{$z_2$}] (z2) at (-1,0);
			\coordinate [label=below right:\textcolor{second}{$Z_2$}] (Z2) at (-.7,.6);
			\draw[cstcurve,cstra,second] (N)--(z2);
			\fill[cstdot,second] (Z2) circle;
			\fill[cstdot,third] (N) circle;
		\end{tikzpicture}
	\end{center}
\end{frame}


\begin{frame}{复球面: 与实数无穷的联系}
	\onslide<+->
	它和实数列极限符号中的 $\infty$ 有什么联系呢?
	\onslide<+->
	选取上述图形的一个截面来看, 实轴可以和圆周去掉一点建立一一对应.
	\onslide<+->
	于是实数列极限符号中的 $\infty$ 在复球面上就是 $\infty$.

	\onslide<2->
	\begin{center}
		\begin{tikzpicture}
			\filldraw[cstcurve,cstfill] (0,1) circle (1);
			\coordinate [label=above:\textcolor{third}{$N$}] (N) at (0,2);
			\draw[cstdash] (0,0)--(N);
			\draw[cstaxis] (-2,0)--(2.5,0);
			\coordinate [label=below:\textcolor{main}{$x_1$}] (x1) at (2.2,0);
			\coordinate [label=above right:\textcolor{main}{$X_1$}] (X1) at (1,1.1);
			\coordinate [label=below:\textcolor{second}{$x_2$}] (x2) at (-1,0);
			\coordinate [label=left:\textcolor{second}{$X_2$}] (X2) at (-.8,.4);
			\draw[cstcurve,cstra,main] (0,2)--(x1);
			\fill[cstdot,main] (X1) circle;
			\draw[cstcurve,cstra,second] (0,2)--(x2);
			\fill[cstdot,second] (X2) circle;
			\fill[cstdot,third] (N) circle;
		\end{tikzpicture}
	\end{center}

	\onslide<+->
	朴素地看, 复球面上任意一点可以定义 $\delta$ 邻域为与其距离小于 $\delta$ 的所有点.
	\onslide<+->
	特别地, $\infty$ 的开邻域通过前面所说的对应关系, 可以对应到扩充复平面上 $\infty$ 的一个邻域.
	\onslide<+->
	所以在复球面上, 我们将普通复数和 $\infty$ 的开邻域可以视为相同的概念.
\end{frame}


\subsection{数列的极限}

\begin{frame}{数列收敛的等价刻画}
	\onslide<+->
	下述定理保证了我们可以使用实数列的敛散性判定技巧.

	\onslide<+->
	\begin{theorem}
		设 $z_n=x_n+y_ni,z=x+y\ii$, 则
	\[
		\lim_{n\to\infty}z_n=z\iff \lim_{n\to\infty}x_n=x,\lim_{n\to\infty}y_n=y.
	\]
	\end{theorem}

	\onslide<+->
	\begin{proof}
		由三角不等式
	\[
		|x_n-x|,|y_n-y|\le|z_n-z|\le|x_n-x|+|y_n-y|
	\]
		易证.
	\end{proof}
\end{frame}


\begin{frame}{极限的四则运算}
	\onslide<+->
	由此可知极限的四则运算法则对于数列也是成立的.
	\onslide<+->
	\begin{theorem}
		设 $\lim\limits_{n\to\infty}z_n=z,\lim\limits_{n\to\infty}w_n=w$, 则
		\begin{enumerate}
			\item $\lim\limits_{n\to\infty}(z_n\pm w_n)=z\pm w$;
			\item $\lim\limits_{n\to\infty} z_nw_n=zw$;
			\item 当 $w\neq 0$ 时, $\lim\limits_{n\to\infty}\dfrac{z_n}{w_n}=\dfrac zw$.
		\end{enumerate}
	\end{theorem}
\end{frame}


\begin{frame}{例: 数列的敛散性}
	\onslide<+->
	\begin{example}
		设 $z_n=\Bigl(1+\dfrac1n\Bigr)\ee^{\frac{\pi\ii}n}$. 数列 $\{z_n\}$ 是否收敛?
	\end{example}

	\onslide<+->
	\begin{solution}
		由于
	\[
		x_n=\Bigl(1+\frac1n\Bigr)\cos\frac\pi n\to 1,\quad
		y_n=\Bigl(1+\frac1n\Bigr)\sin\frac\pi n\to 0.
	\]
		\onslide<+->{%
			因此 $\{z_n\}$ 收敛且 $\lim\limits_{n\to\infty}z_n=1$.
		}
	\end{solution}
\end{frame}


\subsection{函数的极限}
\begin{frame}{函数的极限}
	\onslide<+->
	\begin{definition*}
		设函数 $f(z)$ 在点 $z_0$ 的某个去心邻域内有定义.
		\onslide<+->{%
			如果存在复数 $A$, 使得对 $A$ 的任意邻域 $U(A,\varepsilon),\exists\delta>0$ 使得
			\[z\in\Uc(z_0,\delta)\implies f(z)\in U(A,\varepsilon),
	\]
		}\onslide<+->{%
			则称 $A$ 为 \emph{$f(z)$ 当 $z\to z_0$ 时的极限}, 记为 \emph{$\lim\limits_{z\to z_0}f(z)=A$} 或 \emph{$f(z)\to A (z\to z_0)$}.
		}
	\end{definition*}
	\onslide<+->
	此时我们称\emph{极限存在}.

	\onslide<+->
	上述定义中的 $z_0$ 和 $A$ 可换成 $\infty$, 从而得到 $z\to\infty$ 的极限定义, 以及 $\lim f(z)=\infty$ 的含义.
\end{frame}


\begin{frame}{与实函数极限之联系}
	\onslide<+->
	不难看出, 复变函数的极限和二元实函数的极限定义是类似的:
	\onslide<+->
	即 $z\to z_0$ 沿任一曲线趋向于 $z_0$ 的极限都是相同的.

	\onslide<+->
	\begin{theorem}
		设 $f(z)=u(x,y)+\ii v(x,y),z_0=x_0+y_0\ii,A=u_0+v_0\ii$, 则
	\[
		\lim_{z\to z_0}f(z)=A\iff
		\lim_{\substack{x\to x_0\\y\to y_0}}u(x,y)=u_0,\quad
		\lim_{\substack{x\to x_0\\y\to y_0}}v(x,y)=v_0.
	\]
	\end{theorem}

	\onslide<+->
	\begin{proof}
		由三角不等式
	\[
		|u-u_0|,|v-v_0|\le|f(z)-A|\le|u-u_0|+|v-v_0|
	\]
		易证.
	\end{proof}
\end{frame}


\begin{frame}{极限的四则运算}
	\onslide<+->
	由此可知极限的四则运算法则对于复变函数也是成立的.
	\onslide<+->
	\begin{theorem}
		设 $\lim\limits_{z\to z_0}f(z)=A,\lim\limits_{z\to z_0}g(z)=B$, 则
		\begin{enumerate}
			\item $\lim\limits_{z\to z_0}(f\pm g)(z)=A\pm B$;
			\item $\lim\limits_{z\to z_0}(fg)(z)=AB$;
			\item 当 $B\neq 0$ 时, $\lim\limits_{z\to z_0}\Bigl(\dfrac fg\Bigr)(z)=\dfrac AB$.
		\end{enumerate}
	\end{theorem}

	\onslide<+->
	在学习了复变函数的导数后, 我们也可以使用等价无穷小替换、洛必达法则等工具来计算极限.
\end{frame}


\begin{frame}{例: 判断函数极限是否存在}
	\onslide<+->
	\begin{example}
		证明: 当 $z\to0$ 时, 函数 $f(z)=\dfrac{\Re z}{|z|}$ 的极限不存在.
	\end{example}

	\onslide<+->
	\begin{proof}
		令 $z=x+y\ii$, 则 $f(z)=\dfrac x{\sqrt{x^2+y^2}}$.
		\onslide<+->{%
			因此
			\[u(x,y)=\frac x{\sqrt{x^2+y^2}},\quad v(x,y)=0.
	\]
		}\onslide<+->{%
			当 $z$ 在实轴原点两侧分别趋向于 $0$ 时, $u(x,y)\to\pm1$.
		}\onslide<+->{%
			因此 $\lim\limits_{\substack{x\to 0\\y\to0}}u(x,y)$ 不存在,
		}\onslide<+->{%
			从而 $\lim\limits_{z\to z_0}f(z)$ 不存在.\qedhere
		}
	\end{proof}
\end{frame}


\subsection{函数的连续性}
\begin{frame}{函数的连续性}
	\onslide<+->
	\begin{definition}
		\begin{itemize}
			\item 如果 $\lim\limits_{z\to z_0}f(z)=f(z_0)$, 则称 $f(z)$ 在 \emph{$z_0$ 处连续}.
			\item 如果 $f(z)$ 在区域 $D$ 内处处连续, 则称 $f(z)$ 在 \emph{$D$ 内连续}.
		\end{itemize}
	\end{definition}

	\onslide<+->
	根据前面的极限判定定理可知:
	\onslide<+->
	\begin{theorem}
		函数 $f(z)=u(x,y)+\ii v(x,y)$ 在 $z_0=x_0+\iiy_0$ 处连续当且仅当 $u(x,y)$ 和 $v(x,y)$ 在 $(x_0,y_0)$ 处连续.
	\end{theorem}

	\onslide<+->
	\begin{example}
		设 $f(z)=\ln(x^2+y^2)+\ii(x^2-y^2)$.
		\onslide<+->{%
			$u(x,y)=\ln(x^2+y^2)$ 除原点外处处连续, $v(x,y)=x^2-y^2$ 处处连续.
		}\onslide<+->{%
			因此 $f(z)$ 在 $z\neq0$ 处连续.
		}
	\end{example}
\end{frame}


\begin{frame}{连续函数的性质}
	\onslide<+->
	\begin{theorem}
		\begin{itemize}
			\item 在 $z_0$ 处连续的两个函数 $f(z)$, $g(z)$ 之和、差、积、商($g(z_0)\neq 0$) 在 $z_0$ 处仍然连续.
			\item 如果函数 $g(z)$ 在 $z_0$ 处连续, 函数 $f(w)$ 在 $g(z_0)$ 处连续, 则 $f(g(z))$ 在 $z_0$ 处连续.
		\end{itemize}
	\end{theorem}

	\onslide<+->
	显然 $f(z)=z$ 是处处连续的,
	\onslide<+->
	故多项式函数
	\[P(z)=a_0+a_1z+a_2z^2+\cdots+a_nz^n
	\]
	也处处连续,
	\onslide<+->
	有理函数 $\dfrac{P(z)}{Q(z)}$ 在 $Q(z)$ 的零点以外处处连续.
\end{frame}


\begin{frame}{例: 函数连续性的判定}
	\onslide<+->
	\begin{example}
		证明: 如果 $f(z)$ 在 $z_0$ 连续, 则 $\ov{f(z)}$ 在 $z_0$ 也连续.
	\end{example}

	\onslide<+->
	\begin{proof}
		设 $f(z)=u(x,y)+\ii v(x,y),z_0=x_0+\iiy_0$.
		\onslide<+->{%
			那么 $u(x,y),v(x,y)$ 在 $(x_0,y_0)$ 连续.
		}\onslide<+->{%
			从而 $-v(x,y)$ 也在 $(x_0,y_0)$ 连续.
		}\onslide<+->{%
			所以 $\ov{f(z)}=u(x,y)-\ii v(x,y)$ 在 $(x_0,y_0)$ 连续.\qedhere
		}
	\end{proof}
	\onslide<+->
	\begin{proof}[另证]
		函数 $g(z)=\ov z=x-\ii y$ 处处连续,
		\onslide<+->{%
			从而 $g(f(z))=\ov{f(z)}$ 在 $z_0$ 处连续.\qedhere
		}
	\end{proof}
\end{frame}


\begin{frame}{注记}
	\onslide<+->
	可以看出, 在极限和连续性上, 复变函数和两个二元实函数没有什么差别.
	\onslide<+->
	那么复变函数和多变量微积分的差异究竟是什么导致的呢?
	\onslide<+->
	归根到底就在于 $\BC$ 是一个域, 上面可以做除法.

	\onslide<+->
	这就导致了复变函数有\alert{导数}, 而不是像多变量实函数只有偏导数.
	\onslide<+->
	这种特性使得可导的复变函数具有整洁优美的性质, 我们将在下一章来逐步揭开它的神秘面纱.
\end{frame}

