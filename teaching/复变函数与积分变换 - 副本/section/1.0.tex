\subsection{复数的引入}

\begin{frame}{复数的引入\noexer}
	\onslide<+->
	复数起源于多项式方程的求根问题.
	\onslide<+->
	考虑一元二次方程 $x^2+bx+c=0$,
	\onslide<+->
	配方可得
		\[\left(x+\frac b2\right)^2=\frac{b^2-4c}4.\]
	\onslide<+->
	于是得到求根公式
		\[x=\frac{-b\pm\sqrt\Delta}2,\quad \Delta=b^2-4c.\]
	\begin{enumerate}
		\item 当 $\Delta>0$ 时, 有两个不同的实根;
		\item 当 $\Delta=0$ 时, 有一个二重的实根;
		\item 当 $\Delta<0$ 时, 无实根.
	\end{enumerate}
	\onslide<+->
	可以看出, 在一元二次方程中, 我们可以舍去包含\alert{负数开方}的解.
	\onslide<+->
	然而在一元三次方程中, 即便只考虑实数根也会不可避免地引入负数开方.
\end{frame}


\begin{frame}{三次方程的根\noexer}
	\onslide<+->
	\begin{example}
		解方程 $x^3+6x-20=0$.
	\end{example}
	\onslide<+->
	\begin{solution}
		\begin{minipage}{.74\textwidth}
			设 $x=u+v$, 则
			\[
				u^3+v^3+3uv(u+v)+6(u+v)-20=0.
			\]
			\onslide<+->{%
				我们希望 $u^3+v^3=20, uv=-2$,
			}\onslide<+->{%
				则 $u^3,v^3$ 满足一元二次方程 $X^2-20X-8=0$.
			}\onslide<+->{%
				解得
				\[u^3=10\pm\sqrt{108}\visible<+->{=(1\pm\sqrt3)^3.}\]
			}\onslide<+->{%
				所以 $u=1\pm\sqrt3, v=1\mp\sqrt 3$,
			}\onslide<+->{%
				$x=u+v=2$.
			}
		\end{minipage}
		\onslide<+->{%
			\begin{minipage}{.22\textwidth}
				\centering
				\begin{tikzpicture}
					\draw (-.2,1.8) node[second] {$y'=3x^2+6$};
					\begin{scope}[visible on=<2->]
						\filldraw[cstcurve,main,domain=-1.2:3.1,smooth,fill=white] plot ({\x*.35},{(\x*\x*\x+6*\x-20)*.07});
						\draw[cstaxis] (-1.5,0)--(1.5,0);
						\draw[cstaxis] (0,-2)--(0,1.5);
						\coordinate [label=above left:$2$] (B) at (.7,0);
						\fill[cstdot,second] (B) circle;
					\end{scope}
				\end{tikzpicture}
			\end{minipage}
		}
	\end{solution}
\end{frame}


\begin{frame}{三次方程的根\noexer}
	\beqskip{0pt}
	\onslide<+->
	\begin{example}
		解方程 $x^3-7x+6=0$.
	\end{example}
	\onslide<+->
	\begin{solution}
		同样地我们有 $x=u+v$, 其中
		\[u^3+v^3=-6,\quad uv=\dfrac73.\]
	\onslide<+->{%
		于是 $u^3,v^3$ 满足一元二次方程 $X^2+6X+\dfrac{343}{27}=0$.
	}\onslide<+->{%
		然而这个方程没有实数解.
	}\onslide<+->{%
		我们可以强行解得 
	}\onslide<+->{%
		\[
			u^3=-3+\dfrac{10}9\sqrt{-3},\qquad
			\visible<+->{u=\frac{3+2\sqrt{-3}}3,\frac{-9+\sqrt{-3}}6,\frac{3-5\sqrt{-3}}6,}
		\]
	}\onslide<+->{%
		相应地
		\[
			v=\frac{3-2\sqrt{-3}}3,\frac{-9-\sqrt{-3}}6,\frac{3+5\sqrt{-3}}6,\qquad
			\visible<+->{x=u+v=2,-3,1.}
		\]
	}
	\end{solution}
	\endgroup
\end{frame}


\begin{frame}{三次方程的根\noexer}\small
	\onslide<+->
	对于一般的三次方程 $x^3+px+q=0$ 而言, 类似可得:
		\[x=u-\frac p{3u},\quad u^3=-\frac q2+\sqrt{\Delta},\quad \Delta=\frac{q^2}4+\frac{p^3}{27}.\]
	\onslide<+->
	由于 $p=0$ 情形较为简单, 所以我们不考虑这种情形.
	\onslide<+->
	通过分析函数图像的极值点可以知道:
	\begin{enumerate}
		\item 当 $\Delta>0$ 时, 有 $1$ 个实根.
		\item 当 $\Delta=0$ 时, 有 $2$ 个实根 $x=-\sqrt[3]{4q},\half\sqrt[3]{4q}$ ($2$重).
		\item 当 $\Delta<0$ 时, 有 $3$ 个实根.
	\end{enumerate}
	\begin{figure}[hbpt]
		\centering
		\begin{minipage}{.32\textwidth}
			\centering
			\begin{tikzpicture}[scale=.7,visible on=<4->]
				\draw[cstaxis] (-2,0)--(2,0);
				\draw[cstaxis] (0,-2)--(0,2);
				\draw[cstcurve,main,domain=-2.6:3.3,smooth] plot ({(\x)*0.35},{(\x*\x*\x-3*\x-10)*0.1});
				\fill[cstdot,second] (.92,0) circle;
			\end{tikzpicture}
		\end{minipage}
		\begin{minipage}{.32\textwidth}
			\centering
			\begin{tikzpicture}[scale=.7,visible on=<5->]
				\draw[cstaxis] (-2,0)--(2,0);
				\draw[cstaxis] (0,-2)--(0,2);
				\draw[cstcurve,main,domain=-3.1:2.8,smooth] plot ({(\x)*0.35},{(\x*\x*\x-3*\x+2)*0.1});
				\fill[cstdot,second] (.37,0) circle;
				\fill[cstdot,second] (-.69,0) circle;
			\end{tikzpicture}
		\end{minipage}
		\begin{minipage}{.32\textwidth}
			\centering
			\begin{tikzpicture}[scale=.7,visible on=<6->]
				\draw[cstaxis] (-2,0)--(2,0);
				\draw[cstaxis] (0,-2)--(0,2);
				\draw[cstcurve,main,domain=-4:3.9,smooth] plot ({(\x)*0.3},{(\x*\x*\x-7*\x+1)*0.05});
				\fill[cstdot,second] (.76,0) circle;
				\fill[cstdot,second] (.05,0) circle;
				\fill[cstdot,second] (-.78,0) circle;
			\end{tikzpicture}
		\end{minipage}
	\end{figure}
\end{frame}


\begin{frame}{三次方程的根\noexer}
	\onslide<+->
	所以我们想要使用求根公式的话, 就\alert{必须接受负数开方}.
	\onslide<+->
	那么为什么当 $\Delta<0$ 时, 从求根公式一定能得到 $3$ 个实根呢?
	\onslide<+->
	在学习了第一章的内容之后我们就可以回答这个问题了.

	\onslide<+->
	尽管在十六世纪, 人们已经得到了三次方程的求根公式, 然而对其中出现的虚数, 却是难以接受.

	\onslide<+->
	\begin{tcolorbox}[
		borderline={0pt}{0pt}{fourth,cstdash},
		colbacktitle=fourth,
		fontlower=\itshape,
		halign lower=flush right,
		lower separated=true]
		圣灵在分析的奇观中找到了超凡的显示, 这就是那个理想世界的端兆, 那个介于存在与不存在之间的两栖物, 那个我们称之为虚的 $-1$ 的平方根。
		\tcblower
		莱布尼兹 (Leibniz)
	\end{tcolorbox}

	\onslide<+->
	我们将在下一节使用更为现代的语言来解释和运用复数.
\end{frame}
