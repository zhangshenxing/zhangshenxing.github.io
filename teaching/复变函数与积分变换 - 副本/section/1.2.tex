\section{复数的三角与指数形式}

\subsection{复数的模和辐角}
\begin{frame}{复数的极坐标形式}
	\onslide<+->
	由平面的极坐标表示, 我们可以得到复数的另一种表示方式.
	\onslide<+->
	以 $0$ 为极点, 正实轴为极轴, 逆时针为极角方向可以自然定义出复平面上的极坐标系.
	\onslide<+->
	\begin{center}
		\begin{minipage}{.4\textwidth}
			\centering
			\begin{tikzpicture}
				\coordinate [label=below left:$0$] (O) at (0,0);
				\coordinate [label=above:\textcolor{second}{$z=x+y\ii$}] (Z) at (3,2);
				\coordinate (X) at (3,0);
				\coordinate (Y) at (0,2);
				\draw[decorate,decoration={brace,amplitude=5},main,cstfill1] (X)--(O) node[midway,below=2mm] {$x$};
				\draw[decorate,decoration={brace,amplitude=5},main,cstfill1] (O)--(Y) node[midway,left=2mm] {$y$};
				\draw[third,thick,cstra] pic [cstfill3,draw=third, "$\theta$", angle eccentricity=1.3, angle radius=0.8cm] {angle=X--O--Z};
				\draw[cstaxis] (-.5,0)--(4,0);
				\draw[cstaxis] (0,-.5)--(0,3);
				\draw[cstcurve,third,cstra] (O)--(Z) node[midway,above,third] {$r$};
				\draw[cstdash] (X)--(Z)--(Y);
				\fill[cstdot,second] (Z) circle;
			\end{tikzpicture}
		\end{minipage}
		\begin{minipage}{.56\textwidth}
			\centering
			\onslide<4->{
			\[ x=r\cos\theta,\qquad y=r\sin\theta,
	\]
			\[r=\sqrt{x^2+y^2},\qquad \theta=\arctan\dfrac yx\text{ 或 }\arctan\dfrac yx\pm\pi.\]}
		\end{minipage}
	\end{center}
	\vspace{-\baselineskip}

	\onslide<+->
	\onslide<+->
	\begin{definition}
		\begin{itemize}
			\item 称 $r$ 为 $z$ 的\emph{模}, 记为 \emph{$|z|=r$}.
			\item 称 $\theta$ 为 $z$ 的\emph{辐角}, 记为 \emph{$\Arg z=\theta$}.
			\onslide<+->{约定 \alert{$0$ 的辐角没有定义}.}
		\end{itemize}
	\end{definition}
\end{frame}


\begin{frame}{辐角主值}
	\onslide<+->
	任意 $z\neq 0$ 的辐角有无穷多个.
	\onslide<+->
	我们固定选择其中位于 $(-\pi,\pi]$ 的那个, 并称之为\emph{辐角主值}或\emph{辐角主值}, 记作 $\emphm{\arg z}$.
	\onslide<+->
	那么 $\emphm{\Arg z=\arg z+2k\pi, k\in\BZ}$.

	\onslide<2->
	\begin{figure}[hbpt]
		\centering
		\begin{minipage}{.46\textwidth}
			\onslide<4->{
			\[\arg z=\begin{cases}
				\visible<4->{\emphn{\arctan\dfrac yx,}}&\visible<4->{\emphn{x>0;}}\vspace{1ex}\\
				\visible<5->{\alertn{\arctan\dfrac yx+\pi,}}&\visible<5->{\alertn{x<0,y\ge0;}}\vspace{1ex}\\
				\visible<6->{\color{third}{\arctan\dfrac yx-\pi,}}&\visible<6->{\color{third}{x<0,y<0;}}\\
				\visible<7->{\color{fourth}{\dfrac\pi2,}}&
				\visible<7->{\color{fourth}{x=0,y>0;}}\\
				\visible<7->{\color{fourth}{-\dfrac\pi2,}}&
				\visible<7->{\color{fourth}{x=0,y<0.}}
				\end{cases}\]}
		\end{minipage}
		\begin{minipage}{.52\textwidth}
			\begin{tikzpicture}
				\draw[cstaxis](-2.5,0)->(2.5,0); 
				\draw[cstaxis](0,-2)->(0,2);
				\draw[cstaxis,main,cstwla] (-1.5,0) arc(180:-180:1.5);
				\filldraw[cstdote,draw=main] (-1.5,-.07) circle;
				\begin{scope}[visible on=<4->]
					\coordinate [label=above:\textcolor{main}{$0$}] (A) at (1.7,0);
					\fill[cstdot,main] (A) circle;
					\coordinate [label=above right:\textcolor{main}{$\arctan\dfrac yx$}] (B) at (.9,.9);
					\fill[cstdot,main] (B) circle;
					\coordinate [label=right:\textcolor{main}{$\arctan\dfrac yx$}] (C) at (1.4,-.9);
					\fill[cstdot,main] (C) circle;
				\end{scope}
				\begin{scope}[visible on=<5->]
					\coordinate [label=above left:\textcolor{second}{$\arctan\dfrac yx+\pi$}] (D) at (-1.1,.8);
					\fill[cstdot,second] (D) circle;
					\coordinate [label=below:\textcolor{second}{$\pi$}] (E) at (-.9,0);
					\fill[cstdot,second] (E) circle;
				\end{scope}
				\begin{scope}[visible on=<6->]
					\coordinate [label=left:\textcolor{third}{$\arctan\dfrac yx-\pi$}] (F) at (-1.6,-.7);
					\fill[cstdot,third] (F) circle;
				\end{scope}
				\begin{scope}[visible on=<7->]
					\coordinate [label=left:\textcolor{fourth}{$\dfrac\pi2$}] (G) at (0,.5);
					\fill[cstdot,fourth] (G) circle;
					\coordinate [label=right:\textcolor{fourth}{$-\dfrac\pi2$}] (H) at (0,-.6);
					\fill[cstdot,fourth] (H) circle;
				\end{scope}
			\end{tikzpicture}
		\end{minipage}
	\end{figure}
	\vspace{-\baselineskip}
	\onslide<9->
	注意 \alert{$\arg \ov z=-\arg z$ 未必成立}, 仅当 $z$ 不是负实数和 $0$ 时成立.
\end{frame}


\begin{frame}{复数模的性质}\small
	\onslide<+->
	复数的模满足如下性质:
	\begin{figure}[hbpt]
		\centering
		\begin{minipage}{.48\textwidth}
			\begin{itemize}\bf
				\item $z\ov z=|z|^2=|\ov z|^2$;
				\item $\abs{\Re z},\abs{\Im z}\le |z|\le\abs{\Re z}+\abs{\Im z}$;
			\end{itemize}
		\end{minipage}
		\begin{minipage}{.48\textwidth}
			\begin{itemize}\bf
				\item $\big||z_1|-|z_2|\big|\le|z_1\pm z_2|\le|z_1|+|z_2|$;
				\item $|z_1+z_2+\cdots+z_n|\le|z_1|+|z_2|+\cdots+|z_n|$.
			\end{itemize}
		\end{minipage}
	\end{figure}
	\begin{center}
		\begin{tikzpicture}[visible on=<3->,scale=.88]
			\draw[cstaxis] (-4.4,0)--(4.3,0);
			\draw[cstaxis] (0,-3)--(0,3);
			\coordinate (O) at (0,0);
			\coordinate (Z) at (-2.5,1.5);
			\coordinate (R) at (-2.5,0);
			\draw[decorate,decoration={brace,amplitude=5},main,cstfill1] (O)--(R) node[midway,below=2mm,main] {$\abs{\Re z}$};
			\draw[decorate,decoration={brace,amplitude=5},main,cstfill1] (R)--(Z) node[midway,left=2mm,main] {$\abs{\Im z}$};
			\draw[cstcurve,second] (O)--(Z) node[midway,above,second] {$|z|$};
			\draw[cstcurve,main] (Z)--(R)--(O);
			\draw[thick] (R) ++(0,.3)--++(.3,0)--++(0,-.3);
	
			\begin{scope}[visible on=<4->]
				\coordinate [label=below right:\textcolor{main}{$z_1$}] (Z1) at (2.8,-.4);
				\coordinate (Z2) at (1.2,2);
				\coordinate [label=above right:\textcolor{main}{$z_1+z_2$}] (P) at ($(Z1)+(Z2)$);
				\coordinate [label=below:\textcolor{third}{$z_1-z_2$}] (M) at ($(Z1)-(Z2)$);
				\draw[decorate,decoration={brace,amplitude=5},main] (Z1)--(O) node[midway,below,sloped] {$|z_1|$};
				\draw[decorate,decoration={brace,amplitude=5},main] (P)--(Z1) node[midway,below,sloped] {$|z_2|$};
				\draw[decorate,decoration={brace,amplitude=5},second] (O)--(P) node[midway,above,sloped] {$|z_1+z_2|$};
				\draw[decorate,decoration={brace,amplitude=5},main] (Z1)--(M) node[midway,below,sloped] {$|z_2|$};
				\draw[decorate,decoration={brace,amplitude=5},third] (M)--(O) node[midway,below,sloped] {$|z_1-z_2|$};
				\begin{scope}[cstcurve,cstra]
					\draw[main] (O)--(Z1);
					\draw[main] (Z1)--(P);
					\draw[second] (O)--(P);
					\draw[third] (O)--(M);
					\draw[main] (Z1)--(M);
				\end{scope}
			\end{scope}

			\begin{scope}[visible on=<5->]
				\coordinate (A) at (2.7,2.4);
				\draw[decorate,decoration={brace,amplitude=5},main] (A)--(P) node[midway,above,sloped] {$|z_3|$};
				\draw[decorate,decoration={brace,amplitude=5},fourth] (O)--(A) node[midway,above=2mm,sloped] {$|z_1+z_2+z_3|$};
				\begin{scope}[cstcurve,cstra]
					\draw[main] (P)--(A);
					\draw[fourth] (O)--(A);
				\end{scope}
			\end{scope}
		\end{tikzpicture}
	\end{center}
\end{frame}


\begin{frame}{例题:共轭复数解决模的等式}
	\beqskip{0pt}
	\onslide<+->
	\begin{example}
		证明
		\begin{enumerate}
			\item $|z_1z_2|=|z_1\ov{z_2}|=|z_1|\cdot|z_2|$;
			\item $|z_1+z_2|^2=|z_1|^2+|z_2|^2+2\Re(z_1\ov{z_2})$.
		\end{enumerate}
	\end{example}

	\onslide<+->
	\begin{proof}
		\begin{enumerate}
			\item 因为
				\[|z_1z_2|^2=z_1z_2\cdot\ov{z_1}\ov{z_2}
				=z_1z_2\ov{z_1}\ov{z_2}=|z_1|^2\cdot|z_2|^2,
	\]
				\onslide<+->{%
					所以 $|z_1z_2|=|z_1|\cdot|z_2|$.
				}\onslide<+->{%
					因此 $|z_1\ov{z_2}|=|z_1|\cdot|\ov{z_2}|=|z_1|\cdot|z_2|$.
				}
			\item 因为
				\begin{align*}
					\text{左边}&=(z_1+z_2)(\ov{z_1}+\ov{z_2})
					=z_1\ov{z_1}+z_2\ov{z_2}+z_1\ov{z_2}+\ov{z_1}z_2,\\
					\text{右边}&=z_1\ov{z_1}+z_2\ov{z_2}+z_1\ov{z_2}+\ov{z_1\ov{z_2}},
				\end{align*}
				\onslide<+->{%
					而 $\ov{z_1\ov{z_2}}=\ov{z_1}z_2$, 所以两侧相等.\qedhere
				}
		\end{enumerate}
	\end{proof}
	\endgroup
\end{frame}


\subsection{复数的三角形式和指数形式}
\begin{frame}{复数的三角形式和指数形式}
	\onslide<+->
	由 $x=r\cos\theta,y=r\sin\theta$ 可得
	\onslide<+->
	\begin{definition}[复数的三角形式]
	\[
		z=r(\cos\theta+\ii\sin\theta).\]	
	\end{definition}
	\onslide<+->
	定义 \alert{$\ee^{\ii\theta}=\exp(\ii\theta):=\cos\theta+\ii\sin\theta$} (欧拉恒等式),
	\onslide<+->
	则我们得到
	\begin{definition}[复数的指数形式]
	\[
		z=r\ee^{\ii\theta}=r\exp(\ii\theta).
	\]
	\end{definition}
	\onslide<+->
	这两种形式的等价的, 指数形式可以认为是三角形式的一种缩写方式.

	\onslide<+->
	求复数的三角和指数形式的\alert{关键在于计算模和辐角}.
\end{frame}


\begin{frame}{例: 求复数的三角和指数形式}
	\onslide<+->
	\begin{example}
		将 $z=-\sqrt{12}-2\ii$ 化成三角形式和指数形式.
	\end{example}

	\onslide<+->
	\begin{solution}
		$r=|z|=\sqrt{12+4}=4$.
		\onslide<+->{%
			由于 $z$ 在第三象限,
		}\onslide<+->{%
			因此
			\[\arg z=\arctan\frac{-2}{-\sqrt{12}}-\pi=\frac\pi6-\pi=-\frac{5\pi}6.
	\]
		}\onslide<+->{%
			故
			\[z=4\left[\cos\Bigl(-\frac{5\pi}6\Bigr)+\ii\sin\Bigl(-
			\frac{5\pi}6\Bigr)\right]=4\ee^{-\frac{5\pi\ii}6}.
	\]
		}
	\end{solution}
\end{frame}


\begin{frame}{例: 求复数的三角和指数形式}
	\beqskip{0pt}
	\onslide<+->
	\begin{example}
		将 $z=\sin\dfrac\pi5+\ii\cos\dfrac\pi5$ 化成三角形式和指数形式.
	\end{example}
	\onslide<+->
	\begin{solution}
		$r=|z|=1$.
		\onslide<+->{%
		由于 $z$ 在第一象限, 因此
		\[\arg z=\arctan\frac{\cos(\pi/5)}{\sin(\pi/5)}=\arctan\cot\frac\pi 5=\frac\pi2-\frac\pi5=\frac{3\pi}{10}.
		\]}\onslide<+->{%
		故
		\[
			z=\cos\frac{3\pi}{10}+\ii\sin\frac{3\pi}{10}=\ee^{\frac{3\pi\ii}{10}}.
		\]}
	\end{solution}
	\onslide<+->
	\begin{solution}[另解]
		\[
			z=\sin\frac\pi5+\ii\cos\frac\pi5
			\visible<+->{=\cos\Bigl(\frac\pi2-\frac\pi5\Bigr)+\ii\sin\Bigl(\frac\pi2-\frac\pi5\Bigr)}
			\visible<+->{=\cos\frac{3\pi}{10}+\ii\sin\frac{3\pi}{10}=\ee^{\frac{3\pi\ii}{10}}.}
		\]
	\end{solution}
	\endgroup
\end{frame}


\begin{frame}{例: 求复数的三角和指数形式}
	\onslide<+->
	求复数的三角或指数形式时, 我们只需要任取一个辐角就可以了, 不要求必须是辐角主值.

	\onslide<+->
	\begin{exercise}
		将 $z=\sqrt 3-3\ii$ 化成三角形式和指数形式.
	\end{exercise}

	\onslide<+->
	\begin{answer}
		$\displaystyle z=2\sqrt3\Bigl(\cos\frac{-\pi}3+\ii\sin\frac{-\pi}3\Bigr)
		=2\sqrt3\ee^{-\frac{\pi\ii}3}$, 写成 $\dfrac{5\pi}3$ 也可以.
	\end{answer}
\end{frame}


\begin{frame}{模为 $1$ 的复数}
	\onslide<+->
	两个模相等的复数之和的三角和指数形式形式较为简单.
	\onslide<+->
	\[
		\ee^{\ii\theta}+\ee^{\ii\varphi}
		=2\cos\frac{\theta-\varphi}2\ee^{\frac{\theta+\varphi}2\ii}.
	\]
	\vspace{-\baselineskip}
	\onslide<+->
	\begin{center}
		\begin{tikzpicture}[scale=.8]
			\coordinate [label=below left:0] (O) at (0,0);
			\coordinate [label=right:\textcolor{main}{$\ee^{\ii\varphi}$}] (Z1) at ({3*cos(18)},{3*sin(18)});
			\coordinate [label=left:\textcolor{main}{$\ee^{\ii\theta}$}] (Z2) at ({3*cos(130)},{3*sin(130)});
			\coordinate [label=above:\textcolor{second}{$\ee^{\ii\theta}+\ee^{\ii\varphi}$}] (P) at ($(Z1)+(Z2)$);
			\coordinate (M) at ($0.5*(P)$);
			\coordinate (X) at (2,0);
			\draw[thick,main] pic [cstfill1, draw=main,"$\varphi$", angle eccentricity=1.4, angle radius=0.7cm] {angle=X--O--Z1};
			\draw[thick,second] pic [cstfill2, draw=second, "$\frac{\theta-\varphi}2$", angle eccentricity=1.7] {angle=Z1--O--P};
			\draw[cstaxis] (-3,0)--(3,0);
			\draw[cstaxis] (0,-.4)--(0,3.5);
			\draw[cstcurve,cstra,main] (O)--(Z1);
			\draw[cstcurve,cstra,main] (O)--(Z2);
			\draw[cstcurve,cstra,second] (O)--(P);
			\draw[cstdash] (Z2)--(Z1)--(P)--(Z2);
			\draw[thick] (M)--++({.3*cos(16)},{-.3*sin(16)})--++({.3*sin(16)},{.3*cos(16)})--++({-.3*cos(16)},{.3*sin(16)});
		\end{tikzpicture}
	\end{center}
	\onslide<+->
	\begin{example}
		如果 $|z|=1,\arg z=\theta$, 则 $z+1=2\cos\dfrac\theta2 \ee^{\frac{\theta \ii}2}$.
	\end{example}
\end{frame}
