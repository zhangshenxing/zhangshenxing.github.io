\section{复数的乘除、乘幂与方根}

\subsection{复数的乘除与三角、指数表示}
\begin{frame}{复数的乘除与三角、指数表示}
	\onslide<+->
	三角和指数形式在进行复数的乘法、除法和幂次计算中非常方便.

	\onslide<+->
	\begin{theorem}
		设
		\[z_1=r_1(\cos\theta_1+i\sin\theta_1)=r_1e^{i\theta_1},\]
		\[z_2=r_2(\cos\theta_2+i\sin\theta_2)=r_2e^{i\theta_2}\neq 0,\]
		则
		\begin{align*}
			z_1z_2&=r_1r_2[\cos(\theta_1+\theta_2)+i\sin(\theta_1+\theta_2)]=r_1r_2e^{i(\theta_1+\theta_2)},\\
			\frac{z_1}{z_2}&=\frac{r_1}{r_2}[\cos(\theta_1-\theta_2)+i\sin(\theta_1-\theta_2)]=\frac{r_1}{r_2}e^{i(\theta_1-\theta_2)}.
		\end{align*}
	\end{theorem}
\end{frame}


\begin{frame}{复数的乘除与三角、指数表示}
	\onslide<+->
	换言之,
	\[\alertn{|z_1z_2|=|z_1|\cdot|z_2|,\quad\abs{\frac{z_1}{z_2}}=\frac{|z_1|}{|z_2|},}\]
	\onslide<+->
	\[\alertn{\Arg(z_1z_2)=\Arg z_1+\Arg z_2,\quad
	\Arg\left(\frac{z_1}{z_2}\right)=\Arg z_1-\Arg z_2.}\]
	\onslide<+->
	\emph{多值函数相等是指两边所能取到的值构成的集合相等.}

	\onslide<+->
	注意上述等式中 $\Arg$ 不能换成 $\arg$, 也就是说
	\[\arg(z_1z_2)=\arg z_1+\arg z_2,\quad
	\arg\left(\frac{z_1}{z_2}\right)=\arg z_1-\arg z_2\]
	\alert{不一定成立}.
	\onslide<+->
	这是因为 $\arg z_1\pm\arg z_2$ 有可能不落在区间 $(-\pi,\pi]$ 上.
	\onslide<+->
	事实上, 当且仅当等式右侧落在区间 $(-\pi,\pi]$ 内时才成立, 否则等式两侧会相差 $\pm2\pi$.
\end{frame}


\begin{frame}{复数的乘除与三角和指数表示}
	\onslide<+->
	\begin{proof*}
		根据和差的正弦、余弦公式可知
		\begin{align*}
			z_1z_2&=r_1(\cos\theta_1+i\sin\theta_1)\cdot
			r_2(\cos\theta_2+i\sin\theta_2)\\
			&\visible<+->{=r_1r_2\bigl[(\cos\theta_1\cos\theta_2-\sin\theta_1\sin\theta_2)
			+i(\cos\theta_1\sin\theta_2+\sin\theta_1\cos\theta_2)\bigr]}\\
			&\visible<+->{=r_1r_2\bigl[\cos(\theta_1+\theta_2)+i\sin(\theta_1+\theta_2)\bigr]}
		\end{align*}
		\onslide<+->{%
			因此乘法情形得证.
		}

		\onslide<+->{%
			设 $\dfrac{z_1}{z_2}=re^{i\theta}$,
		}\onslide<+->{%
			则由乘法情形可知
			\[rr_2=r_1,\quad \theta+\Arg z_2=\Arg z_1.\]
		}\onslide<+->{%
			因此 $r=\dfrac{r_1}{r_2}$, $\theta$ 可取 $\theta_1-\theta_2$.\qedhere
		}
	\end{proof*}
\end{frame}


\begin{frame}{乘积的几何意义}
	从该定理可以看出, 乘以复数 $z=re^{i\theta}$ 可以理解为模放大为 $r$ 倍, 并沿逆时针旋转角度 $\theta$.
	\begin{center}
		\begin{animateinline}[width=6.5cm]{10}
			\begin{tikzpicture}
				\coordinate [label=below:\textcolor{main}{$1$}] (X) at (1.6,0);
				\coordinate (O) at (0,0);
				\coordinate [label=right:\textcolor{main}{$z=re^{i\theta}$}] (Z) at ({2*cos(50)},{2*sin(50)});
				\draw[cstcurve,main] pic [cstfill1,draw=main, "$\theta$", angle eccentricity=1.4] {angle=X--O--Z};
				\coordinate [label=above:\textcolor{second}{$z_1$}] (Z1) at ({2.4*cos(80)},{2.4*sin(80)});
				\coordinate (ZZ1) at ({3*cos(130)},{3*sin(130)});
				\coordinate [label=above:\textcolor{second}{$zz_1$}] (2) at ({3*cos(130)},{3*sin(130)});
				\draw[cstcurve,second] pic [cstfill2,draw=second, "$\theta$", angle eccentricity=1.4] {angle=Z1--O--ZZ1};
				\draw[cstcurve,second,cstra] (O)--(Z1);
				\draw[cstcurve,second,cstra] (O)--(ZZ1);
				\draw[cstaxis] (-3,0)--(3.5,0);
				\draw[cstaxis] (0,-.5)--(0,3);
				\draw[cstcurve,main,cstra] (O)--(X);
				\draw[cstcurve,main,cstra] (O)--(Z);
			\end{tikzpicture}
			\newframe
			\multiframe{41}{r=0+0.5}{
				\begin{tikzpicture}
					\coordinate [label=below:\textcolor{main}{$1$}] (X) at (1.6,0);
					\coordinate (O) at (0,0);
					\coordinate [label=right:\textcolor{main}{$z=re^{i\theta}$}] (Z) at ({2*cos(50)},{2*sin(50)});
					\draw[cstcurve,main] pic [cstfill1,draw=main, "$\theta$", angle eccentricity=1.4] {angle=X--O--Z};
					\coordinate [label=above:\textcolor{second}{$z_1$}] (Z1) at ({2.4*cos(80)},{2.4*sin(80)});
					\coordinate (ZZ1) at ({(2.4+0.03*\r)*cos(80+2.5*\r)},{(2.4+0.03*\r)*sin(80+2.5*\r)});
					\coordinate [label=above:\textcolor{second}{$zz_1$}] (2) at ({3*cos(130)},{3*sin(130)});
					\draw[cstcurve,second] pic [cstfill2,draw=second, "$\theta$", angle eccentricity=1.4] {angle=Z1--O--ZZ1};
					\draw[cstcurve,second,cstra] (O)--(Z1);
					\draw[cstcurve,second,cstra] (O)--(ZZ1);
					\draw[cstaxis] (-3,0)--(3.5,0);
					\draw[cstaxis] (0,-.5)--(0,3);
					\draw[cstcurve,main,cstra] (O)--(X);
					\draw[cstcurve,main,cstra] (O)--(Z);
				\end{tikzpicture}
			}
		\end{animateinline}
	\end{center}
\end{frame}

	% \begin{center}
	% 	\begin{tikzpicture}
	% 		\coordinate [label=below:\textcolor{main}{$1$}] (X) at (1.6,0);
	% 		\coordinate (O) at (0,0);
	% 		\coordinate [label=right:\textcolor{main}{$z=re^{i\theta}$}] (Z) at ({2*cos(50)},{2*sin(50)});
	% 		\draw[cstcurve,main] pic [cstfill1,draw=main, "$\theta$", angle eccentricity=1.4] {angle=X--O--Z};
	% 		\begin{scope}[visible on=<3->]
	% 			\coordinate [label=right:\textcolor{second}{$z_1$}] (Z1) at ({2.4*cos(80)},{2.4*sin(80)});
	% 			\coordinate [label=right:\textcolor{second}{$zz_1$}] (ZZ1) at ({3*cos(130)},{3*sin(130)});
	% 			\draw[cstcurve,second] pic [cstfill2,draw=second, "$\theta$", angle eccentricity=1.4] {angle=Z1--O--ZZ1};
	% 			\draw[cstcurve,second,cstra] (O)--(Z1);
	% 			\draw[cstcurve,second,cstra] (O)--(ZZ1);
	% 		\end{scope}
	% 		\draw[cstaxis] (-3,0)--(3,0);
	% 		\draw[cstaxis] (0,-.5)--(0,3);
	% 		\draw[cstcurve,main,cstra] (O)--(X);
	% 		\draw[cstcurve,main,cstra] (O)--(Z);
	% 	\end{tikzpicture}
	% \end{center}


\begin{frame}{例: 复数解决平面几何问题}
	\onslide<+->
	\begin{example}
		已知正三角形的两个顶点为 $z_1=1$ 和 $z_2=2+i$, 求它的另一个顶点.
	\end{example}

	\onslide<+->
	\begin{solution}
		由于 $\overrightarrow{Z_1Z_3}$ 为 $\overrightarrow{Z_1Z_2}$ 顺时针或逆时针旋转 $\dfrac\pi3$,
		\begin{center}
			\begin{tikzpicture}
				\coordinate [label=below:\textcolor{third}{$z_1$}] (Z1) at (1.5,0);
				\coordinate [label=right:\textcolor{third}{$z_2$}] (Z2) at (3,1.5);
				\coordinate [label=left:\textcolor{main}{$z_3$}] (Z3) at ({1.5*(1.5-sqrt(3)/2)},{1.5*(.5+sqrt(3)/2)});
				\coordinate [label=right:\textcolor{second}{$z'_3$}] (Z3p) at ({1.5*(1.5+sqrt(3)/2)},{1.5*(.5-sqrt(3)/2)});
				\draw[cstcurve,main] pic [cstfill1,draw=main, "$\pi/3$", angle eccentricity=1.7,angle radius=4mm] {angle=Z2--Z1--Z3};
				\draw[cstcurve,second] pic [cstfill2,draw=second, "$\pi/3$", angle eccentricity=1.8] {angle=Z3p--Z1--Z2};
				\draw[cstaxis] (-.5,0)--(4,0);
				\draw[cstaxis] (0,-.5)--(0,2.5);
				\draw[cstcurve,third] (Z1)--(Z2);
				\draw[cstcurve,main] (Z2)--(Z3)--(Z1);
				\draw[cstdash,second] (Z1)--(Z3p)--(Z2);
			\end{tikzpicture}
		\end{center}
	\end{solution}
\end{frame}


\begin{frame}{例: 复数解决平面几何问题}
	\begin{solution}[续解]
		因此
		\begin{align*}
			z_3-z_1&=(z_2-z_1)\exp\left(\pm\frac{\pi i}3\right)
			\visible<+->{=(1+i)\left(\half\pm\frac{\sqrt3}2i\right)}\\
			&\visible<+->{=\frac{1-\sqrt3}2+\frac{1+\sqrt3}2i\ \text{或}\ \frac{1+\sqrt3}2+\frac{1-\sqrt3}2i,}
		\end{align*}
		\onslide<+->{
			\[z_3=\frac{3-\sqrt3}2+\frac{1+\sqrt3}2i\ \text{或}\ \frac{3+\sqrt3}2+\frac{1-\sqrt3}2i.\]
		}
	\end{solution}
\end{frame}

\begin{frame}{例: 复数解决平面几何问题}
	\onslide<+->
	\begin{example}
		设 $AD$ 是 $\triangle ABC$ 的角平分线, 证明 $\dfrac{AB}{AC}=\dfrac{DB}{DC}$.
	\end{example}

	\onslide<+->
	\begin{solution}[证明]
		不妨设 $A=0,B=z,C=1,D=w$, 设
		$\displaystyle \lambda=\dfrac{DC}{BC}=\dfrac{w-1}{z-1}\in(0,1)$.
		\begin{center}
			\begin{tikzpicture}
				\coordinate [label=below left:\textcolor{main}{$A$}] (A) at (0,0);
				\coordinate [label=right:\textcolor{main}{$B=z$}] (B) at ({3*cos(60)},{3*sin(60)});
				\coordinate [label=below:\textcolor{main}{$C=1$}] (C) at (2,0);
				\coordinate [label=right:\textcolor{second}{$D=w$}] (D) at ($0.4*(B)+0.6*(C)$);
				\draw[cstcurve] pic [cstfill2,draw=second] {angle=C--A--D};
				\draw[cstcurve] pic [cstfill3,draw=third,angle radius=4mm] {angle=D--A--B};
				\draw[cstaxis] (-.3,0)--(3,0);
				\draw[cstaxis] (0,-.4)--(0,2);
				\draw[cstcurve,main] (B)--(A)--(C)--cycle;
				\draw[cstcurve,second] (A)--(D);
			\end{tikzpicture}
		\end{center}
	\end{solution}
\end{frame}


\begin{frame}{例: 复数解决平面几何问题}\small
	\beqskip{5pt}
	\onslide<+->
	\begin{proof}[续证]
		那么
		\[w=1+\lambda(z-1)=\lambda z+(1-\lambda).\]
		\onslide<+->{%
			由于 $\angle BAD=\angle DAC$, 根据复数乘法的几何意义,
			$\dfrac{z-0}{w-0}$ 是 $\dfrac{w-0}{1-0}$ 的正实数倍,
		}\onslide<+->{%
			即
			\[\frac{w^2}z=\lambda^2 z+2\lambda(1-\lambda)+\frac{(1-\lambda)^2}z\in\BR,\]
		}\onslide<+->{%
			于是
			\[\lambda^2z+\dfrac{(1-\lambda)^2}z=\lambda^2\ov z+\dfrac{(1-\lambda)^2}{\ov z},\qquad
			\bigl(\lambda^2|z|^2-(1-\lambda)^2\bigr)(z-\ov z)=0.\]
		}\onslide<+->{%
			显然 $z\neq \ov z$. 又因为 $0<\lambda<1$, 故
			\[\frac{AB}{AC}=|z|=\frac{1-\lambda}{\lambda}
			=\frac{BC-DC}{DC}=\frac{DB}{DC}.\qedhere\]
		}
	\end{proof}
	\endgroup
\end{frame}



\subsection{复数的乘幂}

\begin{frame}{复数的乘幂}
	\onslide<+->
	设 $z=r(\cos\theta+i\sin\theta)=re^{i\theta}\neq0$.
	\onslide<+->
	根据复数三角和指数形式的乘法和除法运算法则, 我们有
	\onslide<+->
	\begin{theorem}[复数的乘幂]
		\[z^n=r^n(\cos{n\theta}+i\sin{n\theta})=r^ne^{in\theta},\quad\forall n\in\BZ.\]
	\end{theorem}
	\onslide<+->
	特别地, 当 $r=1$ 时, 我们得到\emph{棣莫弗公式}
	\[(\cos\theta+i\sin\theta)^n=\cos{n\theta}+i\sin{n\theta}.\]
\end{frame}


\begin{frame}{复数的乘幂\noexer}
	\onslide<+->
	对棣莫弗公式左侧进行二项式展开可以得到
	\onslide<+->
	\begin{align*}
		\cos(2\theta)&=\hphantom{1}2\cos^2\theta-\hphantom{1}1,\\
		\cos(3\theta)&=\hphantom{1}4\cos^3\theta-\hphantom{1}3\cos\theta,\\
		\cos(4\theta)&=\hphantom{1}8\cos^4\theta-\hphantom{1}8\cos^2\theta+1,\\
		\cos(5\theta)&=16\cos^5\theta-20\cos^3\theta+5\cos\theta.
	\end{align*}
	\onslide<+->
	一般地, 可以证明 $\cos{n\theta}$ 是 $\cos\theta$ 的 $n$ 次多项式,
	\onslide<+->
	这个多项式
	\[g_n(T)=2^{n-1}T^n-n2^{n-3}T^{n-2}+\cdots\]
	叫做\emph{切比雪夫多项式}.
	\onslide<+->
	它在计算数学的逼近理论中有着重要作用.
\end{frame}


\begin{frame}{典型例题: 复数乘幂的计算}
	\onslide<+->
	\begin{example}
		求 $(1+i)^n+(1-i)^n$.
	\end{example}

	\onslide<+->
	\begin{solution}
		由于
		\[1+i=\sqrt2\left(\cos\frac\pi4+i\sin\frac\pi4\right),\ 
		1-i=\sqrt2\left(\cos\frac\pi4-i\sin\frac\pi4\right),\]
		\onslide<+->{%
			因此
			\[
				(1+i)^n+(1-i)^n
				=2^{\frac n2}\left(\cos\frac{n\pi}4+i\sin\frac{n\pi}4+\cos\frac{n\pi}4-i\sin\frac{n\pi}4\right)
				\visible<+->{=2^{\frac n2+1}\cos\frac{n\pi}4.}
			\]
		}\vspace{-\baselineskip}
	\end{solution}

	\onslide<+->
	\begin{exercise}
		求 $(\sqrt3+i)^{2022}=$\fillblankframe[2cm]{$-2^{2022}$}.
	\end{exercise}
\end{frame}


\subsection{复数的方根}
\begin{frame}{复数的方根}
	\onslide<+->
	我们利用复数乘幂公式来计算复数 $z$ 的 \emph{$n$ 次方根 $\sqrt[n]z$}.
	\onslide<+->
	设
	\[w^n=z=re^{i\theta}\neq0,\quad w=\rho e^{i\varphi},\]
	\onslide<+->
	则
	\[w^n=\rho^n(\cos{n\varphi}+i\sin{n\varphi})=r(\cos\theta+i\sin\theta).\]
	\onslide<+->
	比较两边的模可知 $\rho^n=r,\rho=\sqrt[n]r$.
	\onslide<+->
	为了避免记号冲突, 当 $r$ 是正实数时, $\sqrt[n]r$ 默认表示 $r$ 的唯一的 $n$ 次正实根, 称之为\emph{算术根}.

	\onslide<+->
	由于 $n\varphi$ 和 $\theta$ 的正弦和余弦均相等, 因此存在整数 $k$ 使得
	\[n\varphi=\theta+2k\pi,\quad \varphi=\frac{\theta+2k\pi}n.\]
\end{frame}


\begin{frame}{复数的方根}
	\onslide<+->
	故 $w=w_k=\sqrt[n]r\exp\Bigl(\dfrac{\theta+2k\pi}ni\Bigr)$.
	\onslide<+->
	不难看出, $w_k=w_{k+n}$, 而 $w_0,w_1,\dots,w_{n-1}$ 两两不同.
	因此只需取 $k=0,1,\dots,n-1$.
	\onslide<+->
	\begin{theorem}[复数的方根]
		任意一个非零复数 $z$ 的 $n$ 次方根有 $n$ 个值:
		\[
			\sqrt[n]z=\sqrt[n]r\exp\Bigl(\dfrac{\theta+2k\pi}ni\Bigr)\\
				=\sqrt[n]r\Bigl(\cos\frac{\theta+2k\pi}n+i\sin\frac{\theta+2k\pi}n\Bigr),\quad k=0,1,\dots,n-1.
		\]
	\end{theorem}
	\onslide<+->
	这些根的模都相等, 且 $w_k$ 和 $w_{k+1}$ 辐角相差 $\dfrac{2\pi}n$.
	\onslide<+->
	因此\alert{它们是以原点为中心, $\sqrt[n]r$ 为半径的圆的内接正 $n$ 边形的顶点}.
\end{frame}


\begin{frame}{例: 复数方根的计算}
	\onslide<+->
	\begin{example}
		求 $\sqrt[4]{1+i}$.
	\end{example}

	\onslide<+->
	\begin{solution}
		由于 $1+i=\sqrt2\exp\left(\dfrac{\pi i}4\right)$,
		\onslide<+->{%
			因此
			\[\sqrt[4]{1+i}=\sqrt[8]2\exp\frac{(\frac\pi4+2k\pi)i}4,\quad k=0,1,2,3.\]
		}\onslide<+->{%
			于是该方根所有值为
			\[w_0=\sqrt[8]2e^{\frac{\pi i}{16}},\quad
			w_1=\sqrt[8]2e^{\frac{9\pi i}{16}},\quad
			w_2=\sqrt[8]2e^{\frac{17\pi i}{16}},\quad
			w_3=\sqrt[8]2e^{\frac{25\pi i}{16}}.\]
		}
		\vspace{-\baselineskip}
	\end{solution}
\end{frame}


\begin{frame}{典型例题: 复数方根的计算}
	\onslide<+->
	显然 $w_{k+1}=iw_k$,
	\onslide<+->
	所以 $w_0,w_1,w_2,w_3$ 形成了一个正方形.
	\onslide<+->
	\begin{center}
		\begin{tikzpicture}[scale=.9]
			\draw[cstaxis] (-2.3,0)--(2.3,0);
			\draw[cstaxis] (0,-2.3)--(0,2.3);
			\coordinate [label=below left:$0$] (O) at (0,0);
			\draw[cstcurve,thick,third,cstra] (0,0) circle (1.6);
			\coordinate (W0) at ({1.6*cos(11.25)},{1.6*sin(11.25)});
			\coordinate (W1) at ({1.6*cos(101.25)},{1.6*sin(101.25)});
			\coordinate (W2) at ({1.6*cos(191.25)},{1.6*sin(191.25)});
			\coordinate (W3) at ({1.6*cos(281.25)},{1.6*sin(281.25)});
			\draw[cstcurve,thick,second,cstra] (O)--(W0)
				node[right] {$w_0$};
			\draw[cstcurve,thick,second,cstra] (O)--(W1)
				node[above] {$w_1$};
			\draw[cstcurve,thick,second,cstra] (O)--(W2)
				node[left] {$w_2$};
			\draw[cstcurve,thick,second,cstra] (O)--(W3)
				node[below] {$w_3$};
			\draw[cstcurve,main] (W0)--(W1)--(W2)--(W3)--cycle;
		\end{tikzpicture}
	\end{center}
	\vspace{-\baselineskip}
	\onslide<+->
	\begin{exercise}
		计算 $\sqrt[6]{-1}=$\fillblankframe[5cm][2mm]{$\pm\dfrac{\sqrt3+i}2,\pm i,\pm\dfrac{\sqrt3-i}2$}.
	\end{exercise}
\end{frame}


\begin{frame}{方幂和方根的辐角等式}
	\onslide<+->
	注意当 $|n|\ge 2$ 时, \alert{$\Arg(z^n)=n\Arg z$ 不成立}.
	\onslide<+->
	这是因为
	\[\Arg(z^n)=n\arg z+2k\pi,\quad k\in\BZ,\]
	\[n\Arg z=n\arg z+2nk\pi,\quad k\in\BZ.\]
	\onslide<+->
	不过我们总有
	\[\alertn{\Arg \sqrt[n]z=\dfrac1n\Arg z}=\dfrac{\arg z+2k\pi}n,\quad k\in\BZ,\]
	\onslide<+->
	其中左边表示 $z$ 的所有 $n$ 次方根的所有辐角.
\end{frame}


\begin{frame}{应用: 三次方程的求根问题\noexer}
	\onslide<+->
	现在我们来看三次方程 $x^3+px+q=0$ 的根, $p\neq 0$.
	\onslide<+->
	回顾求根公式:
	\[x=u+v,\quad u^3=-\frac q2+\sqrt{\Delta},\quad uv=-\frac p3,\quad \Delta=\frac{q^2}4+\frac{p^3}{27}.\]
	\vspace{-\baselineskip}
	\begin{enumerate}
		\item 如果 $\Delta>0$, 设 $\omega=e^{2\pi i/3}$, 设实数 $\alpha$ 满足 $\alpha^3=-\dfrac q2+\sqrt{\Delta}$,
			\onslide<+->
			则
				\[
					u=\alpha,\alpha\omega,\alpha\omega^2,\qquad
					x=\alpha-\frac p{3\alpha},\ 
						\alpha\omega-\frac p{3\alpha} \omega^2,\ 
						\alpha\omega^2-\frac p{3\alpha} \omega.
				\]
			\onslide<+->
			容易证明后两个根都是虚数.
		\item 如果 $\Delta\le 0$, 则 $p<0$, $|u|^2=-\dfrac p3>0$. 
			\onslide<+->
			从而 $v=\ov u$.
			\onslide<+->
			设 $\sqrt[3]{-\dfrac q2+\sqrt{\Delta}}=u_1,u_2,u_3$,
			\onslide<+->
			则我们得到 $3$ 个实根
				\[x=u_1+\ov{u_1},\ u_2+\ov{u_2},\ u_3+\ov{u_3}.\]
			\onslide<+->
			不难验证, 若有重根则 $\Delta=0$.
	\end{enumerate}
\end{frame}

