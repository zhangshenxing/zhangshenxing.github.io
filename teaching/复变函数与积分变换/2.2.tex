\section{函数解析的充要条件}


\begin{frame}{可导函数的特点}
\onslide<+->
通过对一些简单函数的分析, 我们会发现可导的函数往往可以直接表达为 $z$ 的函数的形式, 而不解析的往往包含 $x,y,\ov z$ 等内容.
\onslide<+->
这种现象并不是孤立的.
\onslide<+->
我们来研究二元实变量函数的可微性与复变函数可导的关系.

\onslide<+->
为了简便我们用 $u_x,u_y,v_x,v_y$ 等记号表示偏导数.
\end{frame}


\begin{frame}{可导的等价刻画: 形式推导}
\onslide<+->
设 \emph{$f$ 在 $z$ 处可导}, $f'(z)=a+bi$,
\onslide<+->
则
\[\Delta u+i\Delta v=\Delta f=(a+bi)(\Delta x+i\Delta y)+o(\Delta z).\]
\onslide<+->展开可知
\begin{align*}
\Delta u&=a\Delta x-b\Delta y+o(\Delta z),\\
\Delta v&=b\Delta x+a\Delta y+o(\Delta z).
\end{align*}
\onslide<+->
由于 $o(\Delta z)=o(|\Delta z|)=o(\sqrt{x^2+y^2})$,
\onslide<+->因此 
\begin{block@}
\begin{center}
$u,v$ 可微且 $u_x=v_y=a,v_x=-u_y=b$.
\end{center}
\end{block@}
\end{frame}


\begin{frame}{可导的等价刻画: 形式推导}
\onslide<+->反过来, 假设 \emph{$u,v$ 可微且 $u_x=v_y, v_x=-u_y$}.
\onslide<+->由全微分公式
\begin{align*}
\diff u&=u_x\diff x+u_y\diff y
\visible<+->{=u_x\diff x-v_x\diff y,}\\
\visible<+->{\diff v}&\visible<.->{=v_x\diff x+v_y\diff y}\visible<+->{=v_x\diff x+u_x\diff y,}\\
\visible<+->{\diff f}&
\visible<.->{=\diff (u+iv)=(u_x+i v_x)\diff x+(-v_x+i u_x)\diff y}\\
&\visible<+->{=(u_x+i v_x)\diff(x+iy)}\\
&\visible<+->{=(u_x+i v_x)\diff z=(v_y-i u_y)\diff z.}
\end{align*}
\onslide<+->故
\begin{block@}
\begin{center}
$f(z)$ 在 $z$ 处可导, 且 $f'(z)=u_x+i v_x=v_y-i u_y$.
\end{center}
\end{block@}
\end{frame}


\begin{frame}{可导的等价刻画: 柯西-黎曼方程}
\onslide<+->由此我们得到
\onslide<+->
\begin{alertblock}{柯西-黎曼方程 (C-R方程)}
$f(z)$ 在 $z$ 可导当且仅当在 $z$ 点 $u,v$ 可微且满足C-R方程:
\[u_x=v_y,\quad u_y=-v_x.\]
此时
\[f'(z)=u_x+iv_x=v_y-iu_y.\]
\end{alertblock}
\onslide<+->
\begin{center}
\includegraphics[width=1.7cm]{misc/Cauchy.jpeg}
\hspace{2cm}\includegraphics[width=1.7cm]{misc/Riemann.jpeg}
\end{center}
\end{frame}


\begin{frame}{可导的等价刻画: 柯西-黎曼方程}
\onslide<+->由于有时判断可微性不太方便, 因此我们常常用如下定理来判断.
\onslide<+->
\begin{theorem}
\begin{itemize}
	\item 如果 $u_x,u_y,v_x,v_y$ 在 $z$ 处连续, 且满足C-R方程, 则 $f(z)$ 在 $z$ 可导.
	\item 如果 $u_x,u_y,v_x,v_y$ 在区域 $D$ 上处处连续, 且满足C-R方程, 则 $f(z)$ 在 $D$ 上可导(从而解析).
\end{itemize}
\end{theorem}
\end{frame}


% \begin{frame}{可导的判定方法}
% \onslide<+->
% 具体应用时, 我们先求 $u,v$ 的偏导数, 判断其是否存在.
% \onslide<+->
% 然后把偏导数按次序依次写出
% \[u_x=\cdots,\quad u_y=\cdots\]
% \[v_x=\cdots,\quad v_y=\cdots\]
% 列出C-R方程, 求出所有的可导点.

% \onslide<+->
% 如果一个点的一个邻域内都可导, 那么这个点是解析点.
% \onslide<+->
% 如果一个区域 $D$ 内所有点都可导, 那么区域 $D$ 是解析区域.

% \onslide<+->
% 当然, 如果能用求导法则直接说明 $f(z)$ 在 $D$ 内处处可导, 则也可以知道 $f(z)$ 是 $D$ 内的解析函数.
% \end{frame}


% \begin{frame}{柯西-黎曼方程的极坐标形式*}
% \onslide<+->
% 设 $x=r\cos \theta,y=r\sin\theta$,
% \onslide<+->
% 则
% \[\frac{\partial x}{\partial r}=\cos\theta,\ 
% \frac{\partial y}{\partial r}=\sin\theta,\ 
% \frac{\partial x}{\partial \theta}=-r\sin\theta,\ 
% \frac{\partial y}{\partial \theta}=r\cos\theta.\]
% \onslide<+->
% 因此
% \[u_r=u_x\cos\theta+u_y\sin\theta,\quad
% u_\theta=r(-u_x\sin\theta+u_y\cos\theta).\]
% \onslide<+->
% 将C-R方程代入后, 我们可以得到C-R方程的极坐标形式:
% \[ru_r=v_\theta,\quad rv_r=-u_\theta.\]
% \onslide<+->
% 且
% \[f'(z)=e^{-i\theta}(u_r+iv_r)=\frac{v_\theta-i u_\theta}z.\]
% \end{frame}


% \begin{frame}{柯西-黎曼方程的极坐标形式*}
% \onslide<+->
% 设 $w=f(z)=\rho e^{i\varphi}$,
% \onslide<+->
% 则类似可得C-R方程
% \[\rho_r=\frac{\rho \varphi_\theta}r,\quad
% \varphi_r=-\frac{\rho_\theta}{\rho r}.\]
% \onslide<+->
% 此时
% \[f'(z)=\frac wz(\varphi_\theta+ir\varphi_r)=\frac w{z\rho}(r\rho_r-i\rho_\theta).\]
% \end{frame}


\begin{frame}{柯西-黎曼方程的 $z,\ov z$ 形式*}
\onslide<+->注意到 $x=\dfrac12z+\dfrac12\ov z,y=-\dfrac i2z+\dfrac i2\ov z$.
\onslide<+->如果我们定义 $f$ 对 $z$ 和 $\ov z$ 的偏导数为
\[\left\{
\begin{aligned}
	\frac{\partial f}{\partial z}&
=\frac{\partial x}{\partial z}\frac{\partial f}{\partial x}
	+\frac{\partial y}{\partial z}\frac{\partial f}{\partial y}
=\frac12\frac{\partial f}{\partial x}-\frac i2\frac{\partial f}{\partial y},\\
	\frac{\partial f}{\partial \ov z}&
=\frac{\partial x}{\partial \ov z}\frac{\partial f}{\partial x}
	+\frac{\partial y}{\partial \ov z}\frac{\partial f}{\partial y}
=\frac12\frac{\partial f}{\partial x}+\frac i2\frac{\partial f}{\partial y}.
\end{aligned}\right.\]
\onslide<+->
那么C-R方程等价于
\[\frac{\partial f}{\partial \ov z}
=\frac12\frac{\partial f}{\partial x}+\frac i2\frac{\partial f}{\partial y}
=\frac{u_x+iv_x+iu_y-v_y}2=0.\]
\onslide<+->
所以我们也可以把 $\emph{\dfrac{\partial f}{\partial \ov z}=0}$ 叫做C-R方程.
\end{frame}


\begin{frame}{典型例题: 利用C-R方程判断可导和解析}
\beqskip{4pt}
\onslide<+->
\begin{example}
\enumnum1 函数 $f(z)=\ov z$ 在何处可导, 在何处解析?
\end{example}
\onslide<+->
\begin{solution*}
由 $u=x,v=-y$ 可知
\begin{align*}
u_x&=1,&u_y&=0,\\
v_x&=0,&v_y&=-1.
\end{align*}
\onslide<+->{因为 $u_x=1\neq v_y=-1$, 所以该函数处处不可导, 处处不解析.}
\end{solution*}
\onslide<+->也可以从 $\dfrac{\partial f}{\partial \ov z}=1\neq0$ 看出.
\onslide<+->不过这种方法由于课本上没有, 所以考试的时候最好只把它作为一种验算手段.
\endgroup
\end{frame}


\begin{frame}{典型例题: 利用C-R方程判断可导和解析}
\beqskip{5pt}
\onslide<+->
\begin{example}[续]
\enumnum2 函数 $f(z)=z\Re z$ 在何处可导, 在何处解析?
\end{example}
\onslide<+->
\begin{solution*}
由 $f(z)=x^2+ixy,u=x^2,v=xy$
\onslide<+->{可知
\begin{align*}
u_x&=2x,&u_y&=0,\\
v_x&=y, &v_y&=x.
\end{align*}}%
\onslide<+->{由 $2x=x,0=-y$ 可知只有 $x=y=0,z=0$ 满足C-R方程.}
\onslide<+->{因此该函数只在 $0$ 可导, 处处不解析且
\[f'(0)=\left.(u_x+iv_x)\middle|_{z=0}=0.\right.\]}
\vspace{-\baselineskip}
\end{solution*}
\onslide<+->也可从 $f(z)=\dfrac{z(z+\ov z)}2,\dfrac{\partial f}{\partial\ov z}=\dfrac z2$ 看出.
\endgroup
\end{frame}


\begin{frame}{典型例题: 利用C-R方程判断可导和解析}
\onslide<+->
\begin{example}[续]
\enumnum3 函数 $f(z)=e^x(\cos y+i\sin y)$ 在何处可导, 在何处解析?
\end{example}
\onslide<+->
\begin{solution}
由 $u=e^x\cos y,v=e^x\sin y$
\onslide<+->{可知
\begin{align*}
u_x&=e^x\cos y,&u_y&=-e^x\sin y,\\
v_x&=e^x\sin y,&v_y&=e^x\cos y.
\end{align*}}
\onslide<+->{因此该函数处处可导, 处处解析, 且
\[f'(z)=u_x+iv_x=e^x(\cos y+i\sin y)=f(z).\]}
\end{solution}
\onslide<+->
实际上, 这个函数就是复变量的指数函数 $e^z$.
\end{frame}


\begin{frame}{典型例题: 利用C-R方程判断可导和解析}
\onslide<+->
\begin{exercise}
求 $f(z)=3x^2+y^2-2xyi$ 的可导点和解析点.
\end{exercise}
\onslide<+->
\begin{answer}
可导点为 $\Re z=0$, 没有解析点.
\end{answer}
\end{frame}


\begin{frame}{例题: 利用C-R方程判断可导和解析}
\beqskip{4pt}
\onslide<+->
\begin{example}
设函数 $f(z)=(x^2+axy+by^2)+i(cx^2+dxy+y^2)$ 在复平面内处处解析. 求实常数 $a,b,c,d$ 以及 $f'(z)$.
\end{example}
\onslide<+->
\begin{solution}
由于
\begin{align*}
u_x&=2x+ay,&u_y=ax+2by,\\
v_x&=2cx+dy,&v_y=dx+2y,
\end{align*}
\onslide<+->{因此
\[2x+ay=dx+2y,\quad ax+2by=-(2cx+dy),\]}
\vspace{-\baselineskip}
\onslide<+->{\[a=d=2,\quad b=c=-1,\]}
\vspace{-\baselineskip}
\onslide<+->{\[f'(z)=u_x+iv_x=2x+2y+i(-2x+2y)=(2-2i)z.\]}
\end{solution}
\endgroup
\end{frame}


\begin{frame}[<*>]{例题: 利用C-R方程证明解析函数结论}
\onslide<+->
\begin{example}
如果 $f'(z)$ 在区域 $D$ 内处处为零, 则 $f(z)$ 在 $D$ 内是一常数.
\end{example}
\onslide<+->
\begin{proof}
由于
\vspace{-\baselineskip}
\[f'(z)=u_x+iv_x=v_y-iu_y=0,\]
\onslide<+->{因此 $u_x=v_x=u_y=v_y=0$, $u,v$ 均为常数, }
\onslide<+->{从而 $f(z)=u+iv$ 是常数.}
\end{proof}
\onslide<+->
类似地可以证明, 若 $f(z)$ 在 $D$ 内解析, 则下述条件等价:
\onslide<+->
\begin{columns}
	\column{0.48\textwidth}
		\begin{itemize}
		\item $f(z)$ 是一常数,
		\item $|f(z)|$ 是一常数,
		\item $\Re{f(z)}$ 是一常数,
		\item $v=u^2$,
		\end{itemize}
		\onslide<+->
	\column{0.48\textwidth}
		\begin{itemize}
		\item $f'(z)=0$,
		\item $\arg{f(z)}$ 是一常数,
		\item $\Im{f(z)}$ 是一常数,
		\item $u=v^2$.
		\end{itemize}
\end{columns}
\end{frame}


\begin{frame}{例题: 利用C-R方程证明解析函数结论}
\beqskip{5pt}
\onslide<+->
\begin{example}
如果 $f(z)$ 解析且 $f'(z)$ 处处非零, 则曲线族 $u(x,y)=c_1$ 和曲线族 $v(x,y)=c_2$ 互相正交.
\end{example}
\onslide<+->
\begin{proof*}
由于 $f'(z)=u_x-iu_y$, 因此 $u_x,u_y$ 不全为零.
\onslide<+->{对 $u(x,y)=c_1$ 使用隐函数求导法则得
\[u_x\diff x+u_y\diff y=0,\]}
\onslide<+->{从而 $(u_x,-u_y)$ 是该曲线在 $z$ 处的非零切向量.}

\onslide<+->{同理 $(v_x,-v_y)$ 是 $v(x,y)=c_2$ 在 $z$ 处的非零切向量.}
\onslide<+->{由于
\[u_xv_x+u_yv_y=-u_xu_y+u_yu_x=0,\]}
\onslide<+->{因此二者正交.\qedhere}
\end{proof*}
\endgroup
\end{frame}


\begin{frame}{解析函数的保角性*}
\onslide<+->
当 $f'(z_0)\neq 0$ 时, 
\onslide<+->
经过 $z_0$ 的两条曲线 $C_1,C_2$ 的夹角和它们的像 $f(C_1),f(C_2)$ 在 $f(z_0)$ 处的夹角总是相同的.
\onslide<+->
这种性质被称为\emph{保角性}.

\onslide<+->
这是因为 $\diff f=f'(z_0)\diff z$.
\onslide<+->
局部来看 $f$ 把 $z_0$ 附近的点以 $z_0$ 为中心放缩 $f'(z_0)$ 倍并逆时针旋转 $\arg{f'(z_0)}$.
\onslide<+->
上述例子是该结论关于 $w$ 复平面上曲线族 $u=c_1,v=c_2$ 的一个特殊情形.

\onslide<+->
最后我们来看复数在求导中的一个应用.
\onslide<+->
\begin{example}
求 $f(x)=\dfrac1{1+x^2}$ 的各阶导数.
\end{example}
\end{frame}


\begin{frame}{复变函数在实变函数导数的应用*}
\onslide<+->
\begin{solution}
设 $f(z)=\dfrac1{1+z^2}$, 则它在除 $z=\pm i$ 外处处解析.
\onslide<+->{当 $z=x$ 为实数时,}
\onslide<+->{
\begin{align*}
f^{(n)}(x)&=\frac i2\left[\frac1{x+i}-\frac1{x-i}\right]^{(n)}\\
&\visible<+->{=\frac i2\cdot(-1)^n n!\left[\frac1{(x+i)^{n+1}}-\frac1{(x-i)^{n+1}}\right]}\\
&\visible<+->{=(-1)^{n+1}n!\Im\frac1{(x+i)^{n+1}}}\\
&\visible<+->{=\frac{(-1)^nn!\sin[(n+1)\arccot x]}{(x^2+1)^{\frac{n+1}2}}.}
\end{align*}}
\end{solution}
\end{frame}

