\section{复数的三角与指数形式}

\subsection{复数的模和辐角}
\begin{frame}{复数的极坐标形式}
\onslide<+->
由平面的极坐标表示, 我们可以得到复数的另一种表示方式.
\onslide<+->
以 $0$ 为极点, 正实轴为极轴, 逆时针为极角方向可以自然定义出复平面上的极坐标系.
\onslide<+->
\begin{center}
\begin{tikzpicture}
\draw[cstaxis] (-0.4,0)--(2,0);
\draw[cstaxis] (0,-0.4)--(0,2);
\draw[cstdash,dcolora] (1.6,0)--(1.6,1.2);
\draw[cstdash,dcolora] (0,1.2)--(1.6,1.2);
\fill[cstdot,dcolora] (1.6,1.2) circle;
\fill[cstdot,dcolora] (7.6,1.2) circle;
\draw[dcolorb,line width=0.1cm,cstarrow1double] (3.2,0.8)--(4.8,0.8);
\draw
  (1.6,1.4) node[dcolora] {$z=x+yi$}
  (-0.2,-0.2) node {$0$}
  (7.6,1.4) node[dcolora] {$z=x+yi$}
  (5.8,-0.2) node {$O$}
  (7.0,-0.5) node {极坐标系}
  (4,0.3) node[dcolorb] {\small 一一对应}
  (1.2,-0.5) node[dcolorb] {复平面}
  (6.7,0.8) node[dcolorc,visible on =<5->] {$r$};
\coordinate (A) at (9,0);
\coordinate (B) at (6,0);
\coordinate (C) at (7.6,1.2);
\draw[cstaxis] (B)--(A);
\draw[cstcurve,dcolorc,cstarrowto,visible on =<5->] (B)--(C);
\draw[dcolorb,thick,visible on =<6->,cstarrowto] pic [draw, "$\theta$", angle eccentricity=1.3, angle radius=0.8cm] {angle};
\end{tikzpicture}
\end{center}

\onslide<+->
\begin{definition}
\begin{itemize}
\item 称 $r$ 为 $z$ 的\emph{模}, 记为 \emph{$|z|=r$}.
\item 称 $\theta$ 为 $z$ 的\emph{辐角}, 记为 \emph{$\Arg z=\theta$}.
\visible<+->{\alert{$0$ 的辐角没有意义}.}
\end{itemize}
\end{definition}
\end{frame}


\begin{frame}{极坐标和直角坐标的对应}
\onslide<+->
由极坐标和直角坐标的对应可知
\onslide<+->
\[x=r\cos\theta,\ y=r\sin \theta,\quad
\visible<+->{|z|=\sqrt{x^2+y^2}}\]
\onslide<+->
\[\Arg z=\begin{cases}
\arctan\dfrac yx+2k\pi,&x>0;\vspace{1ex}\\
\arctan\dfrac yx+(2k+1)\pi,&x<0;\\
\dfrac\pi2+2k\pi,&x=0,y>0;\\
-\dfrac\pi2+2k\pi,&x=0,y<0;\\
\text{任意/无意义},&z=0,
\end{cases}\]
其中 $k\in\BZ$.
\end{frame}


\begin{frame}{主辐角}
\onslide<+->
任意 $z\neq 0$ 的辐角有无穷多个.
\onslide<+->
我们固定选择其中位于 $(-\pi,\pi]$ 的那个, 并称之为\emph{主辐角}, 记作 $\emph{\arg z}$.
\begin{tikzpicture}[overlay,xshift=3cm,yshift=-3cm]
\draw[cstaxis,visible on=<3->](-2.5,0)->(2,0); 
\draw[cstaxis,visible on=<3->](0,-2)->(0,2);
\fill[cstdot,visible on=<4->,dcolora] (1,0) circle;
\fill[cstdot,visible on=<4->,dcolora] (0.8,0.8) circle;
\fill[cstdot,visible on=<4->,dcolora] (1.0,-0.8) circle;
\fill[cstdot,visible on=<5->,dcolorc] (-1.0,0.6) circle;
\fill[cstdot,visible on=<5->,dcolorc] (-0.7,0) circle;
\fill[cstdot,visible on=<6->,dcolorb] (-0.9,-0.5) circle;
\fill[cstdot,visible on=<7->] (0,0.5) circle;
\fill[cstdot,visible on=<7->] (0,-0.5) circle;
\draw
	(1.1,0.2) node[dcolora,visible on=<4->] {$0$}
	(1.0,1.1) node[dcolora,visible on=<4->] {$\arctan\dfrac yx$}
	(1.1,-1.2) node[dcolora,visible on=<4->] {$\arctan\dfrac yx$}
	(-1.2,1.2) node[dcolorc,visible on=<5->] {$\arctan\dfrac yx\colorbox{yellow}{$+\pi$}$}
	(-0.6,-0.2) node[dcolorc,visible on=<5->] {$\pi$}
	(-1.2,-1.1) node[dcolorb,visible on=<6->] {$\arctan\dfrac yx\colorbox{yellow}{$-\pi$}$}
	(-0.2,0.5) node[black,visible on=<7->] {$\dfrac\pi2$}
	(0.4,-0.5) node[black,visible on=<7->] {$-\dfrac\pi2$};
\end{tikzpicture}
\onslide<+->

\begin{flalign*}
&\colorbox{yellow}{$\displaystyle\arg z=\begin{cases}
\visible<4->{\alert{\arctan\dfrac yx,}}&\visible<4->{\alert{x>0;}}\vspace{1ex}\\
\visible<5->{\color{dcolorc}{\arctan\dfrac yx+\pi,}}&\visible<5->{\color{dcolorc}{x<0,y\ge0;}}\vspace{1ex}\\
\visible<6->{\emph{\arctan\dfrac yx-\pi,}}&\visible<6->{\emph{x<0,y<0;}}\\
\visible<7->{\dfrac\pi2,}&\visible<7->{x=0,y>0;}\\
\visible<7->{-\dfrac\pi2,}&\visible<7->{x=0,y<0.}
\end{cases}$}&
\end{flalign*}

\onslide<8->
显然 \emph{$\Arg z=\arg z+2k\pi, k\in\BZ$}.
\end{frame}


\begin{frame}{复数模的性质}
\onslide<+->
复数的模满足如下性质:
\onslide<+->
\begin{block}{模的性质汇总}
\begin{itemize}
\item $z\ov z=|z|^2=|\ov z|^2$;
\item $\abs{\Re z},\abs{\Im z}\le |z|\le\abs{\Re z}+\abs{\Im z}$;
\item $\big||z_1|-|z_2|\big|\le|z_1\pm z_2|\le|z_1|+|z_2|$;
\item $|z_1+z_2+\cdots+z_n|\le|z_1|+|z_2|+\cdots+|z_n|$.
\end{itemize}
\end{block}
\onslide<+->
这些不等式均可以用三角不等式证明, 也可以用代数方法证明.
\onslide<+->
\vspace{-0.5\baselineskip}
\begin{center}
\begin{tikzpicture}
\draw[cstaxis] (-3,0)--(3.5,0);
\draw[cstaxis] (0,-0.5)--(0,2);
\draw[cstcurve,dcolorc] (-1.6,0)--(0,0);
\draw[cstcurve,dcolorb] (-1.6,0)--(-1.6,1.2);
\draw[cstcurve,dcolora] (-1.6,1.2)--(0,0);
\fill[cstdot,dcolora] (-1.6,1.2) circle;
\draw[thick] (-1.6,0.2)--(-1.4,0.2)--(-1.4,0);
\draw 
  (-0.6,0.9) node[dcolora] {$|z|$}
  (-0.8,-0.4) node[dcolorc] {$\abs{\Re z}$}
  (-2.2,0.6) node[dcolorb] {$\abs{\Im z}$}
  (1.3,0.2) node[dcolora,visible on=<8->] {$|z_1|$}
  (0.3,1.4) node[dcolora,visible on=<8->] {$|z_2|$}
  (2.8,1.3) node[dcolorc,visible on=<8->] {$|z_1+z_2|$}
  (2.5,0.8) node[dcolorb,visible on=<8->] {$|z_1-z_2|$};
\draw[cstcurve,visible on=<8->,dcolora] (0,0)--(1.5,0.6);
\draw[cstcurve,visible on=<8->,dcolora] (0,0)--(0.8,1.5);
\draw[cstcurve,visible on=<8->,dcolorc] (0,0)--(2.3,2.1);
\draw[cstcurve,visible on=<8->] (1.5,0.6)--(2.3,2.1);
\draw[cstcurve,visible on=<8->] (0.8,1.5)--(2.3,2.1);
\draw[cstcurve,visible on=<8->,dcolorb] (1.5,0.6)--(0.8,1.5);
\end{tikzpicture}
\end{center}
\end{frame}


\begin{frame}{例题:共轭复数解决模的等式}
\onslide<+->
\begin{example}
证明 \enumnum1 $|z_1z_2|=|z_1\ov{z_2}|=|z_1|\cdot|z_2|$;

\enumnum2 $|z_1+z_2|^2=|z_1|^2+|z_2|^2+2\Re(z_1\ov{z_2})$.
\end{example}
\onslide<+->
\begin{proof}
\enumnum1 因为
\[|z_1z_2|^2=z_1z_2\cdot\ov{z_1z_2}
=z_1z_2\ov{z_1}\ov{z_2}=|z_1|^2\cdot|z_2|^2,\]
\onslide<+->
所以 $|z_1z_2|=|z_1|\cdot|z_2|$.
\onslide<+->
因此 $|z_1\ov{z_2}|=|z_1|\cdot|\ov{z_2}|=|z_1|\cdot|z_2|$.

\enumnum2
\vspace{-0.6\baselineskip}
\begin{align*}
|z_1+z_2|^2&=(z_1+z_2)(\ov{z_1}+\ov{z_2})\\
&\visible<+->{=z_1\ov{z_1}+z_2\ov{z_2}+z_1\ov{z_2}+\ov{z_1\ov{z_2}}}\\
&\visible<+->{=|z_1|^2+|z_2|^2+2\Re(z_1\ov{z_2}).\qedhere}
\end{align*}
\vspace{-1.5\baselineskip}
\end{proof}
\end{frame}


\subsection{复数的三角形式和指数形式}
\begin{frame}{复数的三角形式和指数形式}
\onslide<+->
由 $x=r\cos\theta,y=r\sin\theta$ 可得复数的\emph{三角形式}
\[\alert{z=r(\cos\theta+i\sin\theta)}.\]
\onslide<+->
定义 $e^{i\theta}=\exp(i\theta):=\cos\theta+i\sin\theta$ (欧拉恒等式),
\onslide<+->
则我们得到复数的\emph{指数形式}
\[\alert{z=re^{i\theta}=r\exp(i\theta)}.\]
\onslide<+->
这两种形式的等价的, 指数形式可以认为是三角形式的一种缩写方式.
\end{frame}


\begin{frame}{典型例题: 求复数的三角/指数形式}
\onslide<+->
\begin{example}
将 $z=-\sqrt{12}-2i$ 化成三角形式和指数形式.
\end{example}
\onslide<+->
\begin{solution}
$r=|z|=\sqrt{12+4}=4$.
\onslide<+->
由于 $z$ 在第三象限,
\onslide<+->
因此
\[\arg z=\arctan\frac{-2}{-\sqrt{12}}-\pi=\frac\pi6-\pi=-\frac{5\pi}6.\]
\onslide<+->
故
\[z=4\left[\cos\left(-\frac{5\pi}6\right)+i\sin\left(-
\frac{5\pi}6\right)\right]=4\exp\left(-\frac{5\pi i}6\right).\]
\end{solution}
\end{frame}


\begin{frame}{典型例题: 求复数的三角/指数形式}
\onslide<+->
\begin{example}
将 $z=\sin\dfrac\pi5+i\cos\dfrac\pi5$ 化成三角形式和指数形式.
\end{example}
\onslide<+->
\begin{solution}
\vspace{-\baselineskip}
\begin{align*}
z&=\sin\frac\pi5+i\cos\frac\pi5\\
&\visible<+->{=\cos\left(\frac\pi2-\frac\pi5\right)+i\sin\left(\frac\pi2-\frac\pi5\right)}\\
&\visible<+->{=\cos\frac{3\pi}{10}+i\sin\frac{3\pi}{10}=\exp\left(\frac{3\pi i}{10}\right).}
\end{align*}
\end{solution}
\end{frame}


\begin{frame}{典型例题: 求复数的三角/指数形式}
\onslide<+->
求复数的三角或指数形式时, 我们只需要任取一个辐角就可以了, 不要求必须是主辐角.
\onslide<+->
\begin{exercise}
将 $z=\sqrt 3-3i$ 化成三角形式和指数形式.
\end{exercise}
\onslide<+->
\begin{answer}
$\displaystyle z=2\sqrt3\left[\cos\left(\frac{5\pi}3\right)+i\sin\left(\frac{5\pi}3\right)\right]
=2\sqrt3\exp\left(\frac{5\pi i}3\right)$.
\end{answer}
\end{frame}


\begin{frame}{模为 $1$ 的复数}
\onslide<+->
两个模相等的复数之和的三角/指数形式形式较为简单.
\onslide<+->
\[e^{i\theta}+e^{i\varphi}=2\cos\frac{\theta-\varphi}2e^{\frac{\theta+\varphi}2i}.\]
\onslide<+->
\vspace{-0.5\baselineskip}
\begin{center}
\begin{tikzpicture}
\draw[cstaxis] (-1.5,0)--(2.5,0);
\draw[cstaxis] (0,-0.6)--(0,2.5);
\draw[cstcurve,cstarrowto,dcolora] (0,0)--(-0.8,1.72);
\draw[cstdash] (-0.8,1.72)--(1,2.32)--(1.8,0.6)--(0.5,1.16);
\draw[thick] (0.7,1.074)--(0.786,1.274)--(0.586,1.36);
\coordinate (A) at (2,0);
\coordinate (B) at (0,0);
\coordinate (C) at (1.8,0.6);
\draw[thick,dcolorb] pic [draw, "$\varphi$", angle eccentricity=1.4, angle radius=0.7cm] {angle};
\coordinate (A) at (1.8,0.6);
\coordinate (C) at (1,2.32);
\draw[cstcurve,dcolora,cstarrowto] (B)--(A);
\draw[cstcurve,dcolorb,cstarrowto] (B)--(C);
\draw[thick,dcolorb] pic [draw, "$\frac{\theta-\varphi}2$", angle eccentricity=1.7] {angle};
\draw
  (-0.2,-0.3) node {$0$}
  (2.4,0.8) node[dcolora] {$e^{i\varphi}$}
  (-1.2,1.7) node[dcolora] {$e^{i\theta}$};
\end{tikzpicture}
\end{center}
\vspace{-0.3\baselineskip}

\onslide<+->
\begin{example}
\vspace{-\baselineskip}
\begin{align*}
z&=1-\cos\alpha+i\sin \alpha
\visible<+->{=e^0+e^{(\pi-\alpha)i}}\\
&\visible<+->{=2\cos\frac{\pi-\alpha}2 e^{\frac{\pi-\alpha}2i}}
\visible<+->{=2\sin\frac\alpha2 e^{\frac{\pi-\alpha}2i}.}
\end{align*}
\end{example}
\end{frame}
