\section{复数的三角与指数形式}


\begin{frame}{复数的极坐标形式}
\onslide<+->
由平面的极坐标表示, 我们可以得到复数的另一种表示方式.
\onslide<+->
以 $0$ 为极点, 正实轴为极轴, 逆时针为极角方向可以自然定义出复平面上的极坐标系.
\onslide<+->
\begin{center}
\begin{tikzpicture}
\draw[cstaxis] (-0.4,0)--(2,0);
\draw[cstaxis] (0,-0.4)--(0,2);
\draw[cstdash,alecolor] (1.6,0)--(1.6,1.2);
\draw[cstdash,alecolor] (0,1.2)--(1.6,1.2);
\fill[cstdot,alecolor] (1.6,1.2) circle;
\draw[cstaxis] (6,0)--(9,0);
\fill[cstdot,alecolor] (7.6,1.2) circle;
\draw[notcolor,line width=0.1cm,cstarrow1double] (3.2,0.8)--(4.8,0.8);
\draw
  (1.6,1.4) node[alecolor] {$z=x+yi$}
  (-0.2,-0.2) node {$0$}
  (7.6,1.4) node[alecolor] {$z=x+yi$}
  (5.8,-0.2) node {$O$}
  (7.0,-0.5) node {极坐标系}
  (4,0.3) node[notcolor] {\small 一一对应}
  (1.2,-0.5) node[notcolor] {\textbf{复平面}}
  (6.7,0.8) node[defcolor,visible on =<4->] {$r$}
  (7,0.3) node[notcolor,visible on =<5->] {$\theta$};
\draw[cstcurve,defcolor,cstarrowto,visible on =<4->] (6,0)--(7.6,1.2);
\draw[notcolor,thick,visible on =<5->,cstarrowfrom] (6.6,0.45) arc(37:0:0.75);
\end{tikzpicture}
\end{center}

\begin{definition}
\begin{itemize}
\item 称 $r$ 为 $z$ 的\markdef{模}, 记为 \markdef{$|z|=r$}.
\onslide<+->
\item 称 $\theta$ 为 $z$ 的\markdef{辐角}, 记为 \markdef{$\Arg z=\theta$}.
\visible<+->{\markatt{$0$ 的辐角没有意义}.}
\end{itemize}
\end{definition}
\end{frame}


\begin{frame}{极坐标和直角坐标的对应}
\onslide<+->
由极坐标和直角坐标的对应可知
\onslide<+->
\[x=r\cos\theta,\ y=r\sin \theta,\quad
\visible<+->{|z|=\sqrt{x^2+y^2}}\]
\onslide<+->
\[\Arg z=\begin{cases}
\arctan\dfrac yx+2k\pi,&x>0;\vspace{1ex}\\
\arctan\dfrac yx+(2k+1)\pi,&x<0;\\
\dfrac\pi2+2k\pi,&x=0,y>0;\\
-\dfrac\pi2+2k\pi,&x=0,y<0;\\
\text{任意/无意义},&z=0,
\end{cases}\]
其中 $k\in\BZ$.
\end{frame}


\begin{frame}{主辐角}
\onslide<+->
任意 $z\neq 0$ 的辐角有无穷多个.
\onslide<+->
我们固定选择其中位于 $(-\pi,\pi]$ 的那个, 并称之为\markdef{主辐角}, 记作 $\markdef{\arg z}$.
\begin{tikzpicture}[overlay,xshift=1.7cm,yshift=-3cm]
\draw[cstaxis,visible on=<3->](-2,0)->(2,0); 
\draw[cstaxis,visible on=<3->](0,-2)->(0,2);
\fill[cstdot,visible on=<4->,alecolor] (1,0) circle;
\fill[cstdot,visible on=<4->,alecolor] (0.8,0.8) circle;
\fill[cstdot,visible on=<4->,alecolor] (1.0,-0.8) circle;
\fill[cstdot,visible on=<5->,defcolor] (-1.0,0.6) circle;
\fill[cstdot,visible on=<5->,defcolor] (-0.7,0) circle;
\fill[cstdot,visible on=<6->,notcolor] (-0.9,-0.5) circle;
\fill[cstdot,visible on=<7->] (0,0.5) circle;
\fill[cstdot,visible on=<7->] (0,-0.5) circle;
\draw
	(1.1,0.2) node[alecolor,visible on=<4->] {$0$}
	(1.0,1.1) node[alecolor,visible on=<4->] {$\arctan\dfrac yx$}
	(1.1,-1.2) node[alecolor,visible on=<4->] {$\arctan\dfrac yx$}
	(-1.2,1.2) node[defcolor,visible on=<5->] {$\arctan\dfrac yx+\pi$}
	(-0.6,-0.2) node[defcolor,visible on=<5->] {$\pi$}
	(-1.2,-1.1) node[notcolor,visible on=<6->] {$\arctan\dfrac yx-\pi$}
	(-0.2,0.5) node[black,visible on=<7->] {$\dfrac\pi2$}
	(0.4,-0.5) node[black,visible on=<7->] {$-\dfrac\pi2$};
\end{tikzpicture}
\onslide<+->

\begin{flalign*}
&\arg z=\begin{cases}
\visible<4->{\markatt{\arctan\dfrac yx,}}&\visible<4->{\markatt{x>0;}}\vspace{1ex}\\
\visible<5->{\markdef{\arctan\dfrac yx+\pi,}}&\visible<5->{\markdef{x<0,y\ge0;}}\vspace{1ex}\\
\visible<6->{\marknot{\arctan\dfrac yx-\pi,}}&\visible<6->{\marknot{x<0,y<0;}}\\
\visible<7->{\dfrac\pi2,}&\visible<7->{x=0,y>0;}\\
\visible<7->{-\dfrac\pi2,}&\visible<7->{x=0,y<0.}
\end{cases}&
\end{flalign*}

\onslide<8->
$z$ 是实数当且仅当 $\arg z=0,\pi$ 或 $z=0$.
\onslide<9->
$z$ 是纯虚数当且仅当 $\arg z=\pm\dfrac\pi2$.
\end{frame}


\begin{frame}{复数模的性质}
\begin{conclusion}
复数的模满足如下性质:
\begin{itemize}
\item $z\ov z=|z|^2=|\ov z|^2$;
\item $\abs{\Re z},\abs{\Im z}\le |z|\le\abs{\Re z}+\abs{\Im z}$;
\item $\big||z_1|-|z_2|\big|\le|z_1\pm z_2|\le|z_1|+|z_2|$;
\item $|z_1+z_2+\cdots+z_n|\le|z_1|+|z_2|+\cdots+|z_n|$.
\end{itemize}
\end{conclusion}
\onslide<+->
这些不等式均可以用三角不等式证明, 也可以用代数方法证明.
\onslide<+->
\vspace{-\baselineskip}
\begin{center}
\begin{tikzpicture}
\draw[cstaxis] (-3,0)--(3.5,0);
\draw[cstaxis] (0,-0.5)--(0,2);
\draw[cstcurve,defcolor] (-1.6,0)--(0,0);
\draw[cstcurve,notcolor] (-1.6,0)--(-1.6,1.2);
\draw[cstcurve,alecolor] (-1.6,1.2)--(0,0);
\fill[cstdot,alecolor] (-1.6,1.2) circle;
\draw[thick] (-1.6,0.2)--(-1.4,0.2)--(-1.4,0);
\draw 
  (-0.6,0.9) node[alecolor] {$|z|$}
  (-0.8,-0.4) node[defcolor] {$\abs{\Re z}$}
  (-2.2,0.6) node[notcolor] {$\abs{\Im z}$}
  (1.3,0.2) node[alecolor,visible on=<8->] {$|z_1|$}
  (0.3,1.4) node[alecolor,visible on=<8->] {$|z_2|$}
  (2.8,1.3) node[defcolor,visible on=<8->] {$|z_1+z_2|$}
  (2.5,0.8) node[notcolor,visible on=<8->] {$|z_1-z_2|$};
\draw[cstcurve,visible on=<8->,alecolor] (0,0)--(1.5,0.6);
\draw[cstcurve,visible on=<8->,alecolor] (0,0)--(0.8,1.5);
\draw[cstcurve,visible on=<8->,defcolor] (0,0)--(2.3,2.1);
\draw[cstcurve,visible on=<8->] (1.5,0.6)--(2.3,2.1);
\draw[cstcurve,visible on=<8->] (0.8,1.5)--(2.3,2.1);
\draw[cstcurve,visible on=<8->,notcolor] (1.5,0.6)--(0.8,1.5);
\end{tikzpicture}
\end{center}
\end{frame}


\begin{frame}{例题:共轭复数解决模的等式}
\begin{example}
证明 (1) $|z_1z_2|=|z_1\ov{z_2}|=|z_1|\cdot|z_2|$;

(2) $|z_1+z_2|^2=|z_1|^2+|z_2|^2+2\Re(z_1\ov{z_2})$.
\end{example}
\begin{proof}
(1) 因为
\[|z_1z_2|^2=z_1z_2\cdot\ov{z_1z_2}
=z_1z_2\ov{z_1}\ov{z_2}=|z_1|^2\cdot|z_2|^2,\]
\onslide<+->
所以 $|z_1z_2|=|z_1|\cdot|z_2|$.
\onslide<+->
因此 $|z_1\ov{z_2}|=|z_1|\cdot|\ov{z_2}|=|z_1|\cdot|z_2|$.

\onslide<+->
(2) \vspace{-\baselineskip}
\begin{align*}
|z_1+z_2|^2&=(z_1+z_2)(\ov{z_1}+\ov{z_2})\\
&=z_1\ov{z_1}+z_2\ov{z_2}+z_1\ov{z_2}+\ov{z_1\ov{z_2}}\\
&=|z_1|^2+|z_2|^2+2\Re(z_1\ov{z_2}).\qedhere
\end{align*}
\vspace{-1.2\baselineskip}
\end{proof}
\end{frame}


\begin{frame}{复数的三角形式和指数形式}
\onslide<+->
由 $x=r\cos\theta,y=r\sin\theta$ 可得复数的\markdef{三角形式}
\[\markatt{z=r(\cos\theta+i\sin\theta)}.\]
\onslide<+->
定义 $e^{i\theta}=\exp(i\theta)=\cos\theta+i\sin\theta$ (欧拉恒等式),
\onslide<+->
则我们得到复数的\markdef{指数形式}
\[\markatt{z=re^{i\theta}=r\exp(i\theta)}.\]
\onslide<+->
这两种形式的等价的, 指数形式可以认为是三角形式的一种缩写方式.
\end{frame}


\begin{frame}{典型例题: 求复数的三角/指数形式}
\begin{example}
将 $z=-\sqrt{12}-2i$ 化成三角形式和指数形式.
\end{example}
\begin{solution}
$r=|z|=\sqrt{12+4}=4$.
\onslide<+->
由于 $z$ 在第三象限,
\onslide<+->
因此
\[\arg z=\arctan\frac{-2}{-\sqrt{12}}-\pi=\frac\pi6-\pi=-\frac{5\pi}6.\]
\onslide<+->
故
\[z=4\left[\cos\left(-\frac{5\pi}6\right)+i\sin\left(-
\frac{5\pi}6\right)\right]=4\exp\left(-\frac{5\pi i}6\right).\qedhere\]
\end{solution}
\end{frame}


\begin{frame}{典型例题: 求复数的三角/指数形式}
\begin{example}
将 $z=\sin\dfrac\pi5+i\cos\dfrac\pi5$ 化成三角形式和指数形式.
\end{example}
\begin{solutions}
\vspace{-\baselineskip}
\begin{align*}
z&=\sin\frac\pi5+i\cos\frac\pi5\\
&\visible<+->{=\cos\left(\frac\pi2-\frac\pi5\right)+i\sin\left(\frac\pi2-\frac\pi5\right)}\\
&\visible<+->{=\cos\frac{3\pi}{10}+i\sin\frac{3\pi}{10}=\exp\left(\frac{3\pi i}{10}\right).\mqed}
\end{align*}
\end{solutions}
\end{frame}


\begin{frame}{典型例题: 求复数的三角/指数形式}
\begin{exercise}
将 $z=\sqrt 3-3i$ 化成三角形式和指数形式.
\end{exercise}
\begin{answer}
$\displaystyle z=2\sqrt3\left[\cos\left(\frac{5\pi}3\right)+i\sin\left(\frac{5\pi}3\right)\right]
=2\sqrt3\exp\left(\frac{5\pi i}3\right)$.
\end{answer}
\onslide<+->
求复数的三角或指数形式时, 我们只需要任取一个辐角就可以了, 不要求必须是主辐角.
\begin{example}
将 $z=1-\cos\alpha+i\sin \alpha$ 化成三角形式和指数形式, 并求出它的主辐角, 其中 $0<\alpha\le \pi$.
\end{example}
\end{frame}


\begin{frame}{典型例题: 求复数的三角/指数形式}
\beqskip{8pt}
\begin{solution}
\[|z|^2=(1-\cos\alpha)^2+(\sin\alpha)^2
\visible<+->{=2-2\cos\alpha}
\visible<+->{=4\sin^2\frac\alpha2,}\]
\onslide<+->
因此 $|z|=2\sin\dfrac\alpha2$.
\onslide<+->
由于
\[\frac{\sin\alpha}{1-\cos \alpha}
=\frac{2\sin\dfrac\alpha2\cos\dfrac\alpha2}
{2\sin^2\dfrac\alpha2}
\visible<+->{=\cot\frac\alpha2=\tan\frac{\pi-\alpha}2,}\]
\onslide<+->
且 $\Re z=1-\cos\alpha>0$,
\onslide<+->
因此 $\arg z=\dfrac{\pi-\alpha}2$,
\onslide<+->
\[z=2\sin\frac\alpha2\left(\cos\frac{\pi-\alpha}2+i\sin\frac{\pi-\alpha}2\right)
=2\sin\frac\alpha2 e^{\frac{(\pi-\alpha)i}2}.\qedhere\]
\end{solution}
\endgroup
\end{frame}


\begin{frame}{模为 $1$ 的复数}
\onslide<+->
两个模相等的复数之和的三角/指数形式形式较为简单.
\onslide<+->
\[e^{i\theta}+e^{i\varphi}=2\cos\frac{\theta-\varphi}2\exp\left[\frac{i(\theta+\varphi)}2\right].\]
\onslide<+->
\begin{center}
\begin{tikzpicture}
\draw[cstaxis] (-1.5,0)--(2.5,0);
\draw[cstaxis] (0,-0.6)--(0,2.5);
\draw[notcolor,thick] (0.6,0.2) arc(19:0:0.632);
\draw[notcolor,thick] (0.2,0.464) arc(66:19:0.505);
\draw[cstcurve,cstarrowto,alecolor] (0,0)--(1.8,0.6);
\draw[cstcurve,cstarrowto,alecolor] (0,0)--(-0.8,1.72);
\draw[cstcurve,cstarrowto,notcolor] (0,0)--(1,2.32);
\draw[cstdash] (-0.8,1.72)--(1,2.32)--(1.8,0.6)--(0.5,1.16);
\draw[thick] (0.7,1.074)--(0.786,1.274)--(0.586,1.36);
\draw
  (-0.2,-0.3) node {$0$}
  (2.4,0.8) node[alecolor] {$e^{i\varphi}$}
  (-1.2,1.7) node[alecolor] {$e^{i\theta}$}
  (0.9,0.15) node[notcolor] {$\varphi$}
  (0.7,0.6) node[notcolor] {$\frac{\theta-\varphi}2$};
\end{tikzpicture}
\end{center}
\end{frame}


