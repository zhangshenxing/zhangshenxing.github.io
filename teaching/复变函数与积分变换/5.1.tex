\section{孤立奇点}


\begin{frame}{孤立奇点}
\onslide<+->
我们先根据奇点附近洛朗展开的形式来对其进行分类, 以便于分类计算留数.
\begin{example}
考虑函数 $f(z)=\dfrac1{\sin\left(\dfrac1z\right)}$, 显然 $0,z_k=\dfrac1{k\pi}$ 是奇点, $k$ 是非零整数.
\onslide<+->
因为 $\lim\limits_{k\to+\infty} z_k=0$, 所以 $0$ 的任何一个去心邻域内都有奇点.
\onslide<+->
此时无法选取一个圆环域 $0<|z|<\delta$ 作 $f(z)$ 的洛朗展开.
\onslide<+->
因此我们不考虑这类奇点.
\onslide<3->
\vspace{-9pt}
\begin{center}
\begin{tikzpicture}
\draw[cstcurve] (0,0) circle (1.3);
\fill[cstdot] (0,0) circle;
\fill[cstdot,dcolorb] (2,0) circle;
\fill[cstdot,dcolora] (1,0) circle;
\fill[cstdot,dcolorb] (0.6667,0) circle;
\fill[cstdot,dcolora] (0.5,0) circle;
\fill[cstdot,dcolorb] (0.4,0) circle;
\draw
  (-0.3,0) node {$0$}
  (2,-0.3) node[dcolorb] {$z_1$}
  (1,0.3) node[dcolora] {$z_2$}
  (0.6667,-0.3) node[dcolorb] {$z_3$}
  (0.5,0.3) node[dcolora] {$z_4$}
  (0.4,-0.3) node[dcolorb] {$z_k$};
\end{tikzpicture}
\end{center}
\vspace{-7pt}
\end{example}
\end{frame}


\begin{frame}{孤立奇点的定义}
\begin{definition}
如果 $z_0$ 是 $f(z)$ 的一个奇点, 且 $z_0$ 的某个邻域内没有其它奇点, 则称 $z_0$ 是 $f(z)$ 的一个\markdef{孤立奇点}.
\end{definition}
\begin{example}
\begin{itemize}
\item $z=0$ 是 $e^{\frac1z},\dfrac{\sin z}z$ 的孤立奇点.
\item $z=-1$ 是 $\dfrac1{z(z+1)}$ 的孤立奇点.
\item $z=0$ 不是 $\dfrac1{\sin\left(\dfrac1z\right)}$ 的孤立奇点.
\end{itemize}
\end{example}
\onslide<+->
如果函数 $f(z)$ 只有有限多个奇点, 那么这些奇点都是孤立奇点.
\end{frame}


\begin{frame}{孤立奇点的分类}
\onslide<+->
如果 $f(z)$ 在孤立奇点 $z_0$ 的去心邻域 $0<|z-z_0|<\delta$ 内解析, 则可以作 $f(z)$ 的洛朗级数.
\onslide<+->
根据该洛朗级数主要部分的项数, 我们可以将孤立奇点分为三种:
\onslide<+->
\begin{center}
\renewcommand\arraystretch{2}
\defaultrowcolor
\begin{tabular}{|c|c|c|}
\tht 孤立奇点类型&\tht 洛朗级数特点&\tht $\lim\limits_{z\to z_0}f(z)$\\
可去奇点&没有主要部分&存在且有限\\
$m$ 阶极点&\makecell[c]{主要部分只有有限项非零\\最低次为 $-m$ 次}&$\infty$\\
本性奇点&主要部分有无限项非零&不存在且不为 $\infty$\\
\end{tabular}
\end{center}
\end{frame}


\begin{frame}{孤立奇点的分类: 可去奇点}
\begin{definition}
如果 $f(z)$ 在孤立奇点 $z_0$ 的去心邻域的洛朗级数没有主要部分, 即
\[f(z)=c_0+c_1(z-z_0)+c_2(z-z_0)^2+\cdots\]
是幂级数, 则称 $z_0$ 是 $f(z)$ 的\markdef{可去奇点}.
\end{definition}
\onslide<+->
设 $g(z)$ 为右侧幂级数的和函数, 则 $g(z)$ 在 $|z-z_0|<\delta$ 上解析,
\onslide<+->
且除 $z_0$ 外 $f(z)=g(z)$.
\onslide<+->
通过补充或修改定义 $f(z_0)=g(z_0)=c_0$, 可使得 $f(z)$ 也在 $z_0$ 解析.

\begin{conclusion}[可去奇点判定方法]
$z_0$ 是 $f(z)$ 的可去奇点 $\iff\lim\limits_{z\to z_0}f(z)$ 存在且有限 $\iff\lim\limits_{z\to z_0}(z-z_0)f(z)=0$.
\end{conclusion}
\end{frame}


\begin{frame}{例题: 可去奇点}
\beqskip{2pt}
\begin{example}
\vspace{-\baselineskip}
\[\frac{\sin z}z=1-\dfrac{z^2}{3!}+\dfrac{z^4}{5!}+\cdots\]
没有负幂次项, 因此 $0$ 是可去奇点.
\onslide<+->
也可以从
\[\lim_{z\to0}\frac{\sin z}z=\lim_{z\to0}\frac{\sin z-\sin 0}{z-0}=(\sin z)'|_{z=0}=\cos 0=1\]
看出.
\end{example}
\begin{example}
\vspace{-\baselineskip}
\[\frac{e^z-1}z=1+\dfrac z{2!}+\dfrac{z^2}{3!}+\cdots\]
没有负幂次项, 因此 $0$ 是可去奇点.
\onslide<+->
也可以从
\[\lim_{z\to0}\frac{e^z-1}z=(e^z)'|_{z=0}=1\]
看出.
\end{example}
\endgroup
\end{frame}


\begin{frame}{孤立奇点的分类: 本性奇点}
\begin{definition}
如果 $f(z)$ 在孤立奇点 $z_0$ 的去心邻域的洛朗级数主要部分有无限多项非零, 则称 $z_0$ 是 $f(z)$ 的\markdef{本性奇点}.
\end{definition}
\begin{example}
$\displaystyle e^{\frac1z}=1+\frac1z+\frac1{2z^2}+\cdots$, $0$ 是本性奇点.
\end{example}

\begin{theorem}
$z_0$ 是 $f(z)$ 的本性奇点 $\iff\lim\limits_{z\to z_0}f(z)$ 不存在也不是 $\infty$.
\end{theorem}

\onslide<+->
事实上我们有\marknot{皮卡大定理}: 对于本性奇点 $z_0$ 的任何一个去心邻域, $f(z)$ 的像取遍所有复数, 至多有一个取不到.

\onslide<+->
可去奇点的性质比较简单, 而本性奇点的性质又较为复杂, 因此我们主要关心的是极点的情形.
\end{frame}


\begin{frame}{孤立奇点的分类: 极点}
\begin{definition}
如果 $f(z)$ 在孤立奇点 $z_0$ 的去心邻域的洛朗级数主要部分只有有限多项非零, 即
\[f(z)=c_{-m}(z-z_0)^{-m}+\cdots+c_0+c_1(z-z_0)+\cdots,\]
其中 $c_{-m}\neq 0,m\ge 1$, 则称 $z_0$ 是 $f(z)$ 的\markdef{$m$ 阶极点}(\markdef{$m$ 级极点}).
\end{definition}
\end{frame}


\begin{frame}{典型例题: 函数的极点}
\onslide<+->
令 $g(z)=c_{-m}+c_{-m+1}(z-z_0)+c_{-m+2}(z-z_0)^2+\cdots$, 则 $g(z)$ 在 $z_0$ 解析且非零,
\onslide<+->
且当 $z\neq z_0$ 时, $f(z)=\dfrac{g(z)}{(z-z_0)^m}$.

\begin{theorem}
\begin{itemize}
\item $z_0$ 是 $f(z)$ 的 $m$ 阶极点 $\iff\lim\limits_{z\to z_0}(z-z_0)^mf(z)$ 存在且非零.
\item $z_0$ 是 $f(z)$ 的极点 $\iff\lim\limits_{z\to z_0}f(z)=\infty$.
\end{itemize}
\end{theorem}
\end{frame}


\begin{frame}{典型例题: 函数的极点}
\begin{example}
$f(z)=\dfrac{3z+2}{z^2(z+2)}$,
\onslide<+->
由于 $\lim\limits_{z\to 0}z^2f(z)=1$, 因此 $0$ 是 $2$ 阶极点.
\onslide<+->
同理 $-2$ 是 $1$ 阶极点.
\end{example}

\begin{exercise}
求 $f(z)=\dfrac1{z^3-z^2-z+1}$ 的奇点, 并指出极点的阶.
\end{exercise}
\begin{answer}
$-1$ 是 $1$ 阶极点, $1$ 是 $2$ 阶极点.
\end{answer}
\end{frame}


\begin{frame}{函数的零点}
\onslide<+->
现在我们来研究极点与零点的联系.
\begin{definition}
如果 $f(z)$ 在解析点 $z_0$ 处的泰勒级数最低次项幂次是 $m\ge1$, 即
\[f(z)=c_m(z-z_0)^m+c_{m+1}(z-z_0)^{m+1}+\cdots,\]
其中 $c_m\neq 0$, 则称 $z_0$ 是 $f(z)$ 的 \markdef{$m$ 阶零点}.
\end{definition}
\onslide<+->
此时 $f(z)=(z-z_0)^mg(z)$, $g(z)$ 在 $z_0$ 解析且 $g(z_0)\neq 0$.

\begin{theorem}
如果 $f(z)$ 在 $z_0$ 解析, 则 $z_0$ 是 $m$ 阶零点 $\iff$
\[f(z_0)=f'(z_0)=\cdots=f^{(m-1)}(z_0)=0,\quad
f^{(m)}(z_0)\neq 0.\]
\end{theorem}
\end{frame}


\begin{frame}{函数的零点}
\begin{example}
$f(z)=z(z-1)^3$
\onslide<+->
有 $1$ 阶零点 $0$ 和 $3$ 阶零点 $1$.
\end{example}
\begin{example}
$f(z)=\sin z-z$.
\onslide<+->
由于
\[f(z)=\frac{z^3}{3!}-\frac{z^5}{5!}+\cdots\]
因此 $0$ 是 $3$ 阶零点.
\end{example}
\end{frame}


\begin{frame}{函数的零点}
\begin{theorem}
非零的解析函数的零点总是孤立的.
\end{theorem}
\begin{proof}
\indent
设 $f(z)$ 是区域 $D$ 上的非零解析函数, $z_0\in D$ 是 $f(z)$ 的一个零点.
\onslide<+->
由于 $f(z)$ 不恒为零, 因此存在 $m\ge 1$ 使得在 $z_0$ 的一个邻域内 $f(z)=(z-z_0)^m g(z)$, $g(z)$ 在 $z_0$ 处解析且非零.

\indent
\onslide<+->
对于 $\varepsilon=\dfrac12|c_m|$, 存在 $\delta>0$ 使得当 $z\in \stackrel{\circ}{U}(z_0,\delta)\subseteq D$ 时, $|g(z)-g(z_0)|<\varepsilon$.
\onslide<+->
从而 $|g(z)|>\dfrac12|c_m|>0$, $f(z)\neq 0$.
\end{proof}

\onslide<+->
设 $f(z)$ 是处处解析函数, 且当 $x\in\BR$ 时, $f(x)=e^x$.
\onslide<+->
那么 $f(z)-e^z$ 也是处处解析的, 且全体实数都是它的零点.
\onslide<+->
所以 $f(z)-e^z\equiv0,f(z)=e^z$.
\onslide<+->
这说明了该性质足以定义指数函数.
\end{frame}


\begin{frame}{函数的零点, 极点和阶}
\onslide<+->
为了统一地研究零点和极点, 我们引入下述记号.
\onslide<+->
设 $z_0$ 是 $f(z)$ 的可去奇点、极点或解析点.
\onslide<+->
记 $\ord(f,z_0)$ 为 $f(z)$ 在 $z_0$ 的洛朗展开的最低次项幂次.

\onslide<+->
如果 $\ord(f,z_0)=-m<0$, 则 $z_0$ 是 $m$ 阶极点.
\onslide<+->
如果 $\ord(f,z_0)\ge 0$, 则 $z_0$ 是可去奇点或解析点.

\begin{conclusion}[可去奇点和极点判定方法]
\boxatt{如果 $\ord(f,z_0)=m,\ord(g,z_0)=n$, 那么}
\[\eqatt{\ord\left(\frac fg,z_0\right)=m-n,\quad\ord(fg,z_0)=m+n.}\]
\end{conclusion}
\end{frame}


\begin{frame}{函数的零点, 极点和阶}
\beqskip{7pt}
\begin{proof}
设 $f_0(z)$ 为幂级数 $(z-z_0)^{-m}f(z)$ 的和函数, $g_0(z)$ 为幂级数 $(z-z_0)^{-n}g(z)$ 的和函数,
\onslide<+->
则 $f_0(z),g_0(z)$ 在 $z_0$ 解析且非零.
\onslide<+->
因此 $\dfrac{f_0(z)}{g_0(z)},f_0(z)g_0(z)$ 在 $z_0$ 解析且非零.
\onslide<+->
由
\[\frac{f(z)}{g(z)}=(z-z_0)^{m-n}\frac{f_0(z)}{g_0(z)},\quad
f(z)g(z)=(z-z_0)^{m+n}f_0(z)g_0(z)\]
可知命题成立.
\end{proof}

\begin{corollary}
设 $z_0$ 是 $f(z)$ 的 $m$ 阶零点, 是 $g(z)$ 的 $n$ 阶零点.
\onslide<+->
当 $m\ge n$ 时, $z_0$ 是 $\dfrac{f(z)}{g(z)}$ 的可去奇点; 当 $m<n$ 时, $z_0$ 是 $\dfrac{f(z)}{g(z)}$ 的 $n-m$ 阶极点.
\end{corollary}
\endgroup
\end{frame}


\begin{frame}{典型例题: 函数的极点}
\begin{example}
函数 $\dfrac1{\sin z}$ 有哪些奇点? 如果是极点指出它的阶.
\end{example}
\begin{solution}
函数 $\dfrac1{\sin z}$ 的奇点是 $\sin z=0$ 的点, 即 $z=k\pi$.
\onslide<+->
这些点处
\[(\sin z)'|_{z=k\pi}=\cos{k\pi}=(-1)^k\neq 0.\]
\onslide<+->
从而是 $\sin z$ 的一阶零点, 是 $\dfrac1{\sin z}$ 的一阶极点.
\end{solution}
\begin{example}
$z=0$ 是 $\dfrac{e^z-1}{z^2}$ 的几阶极点?
\end{example}
\end{frame}


\begin{frame}{典型例题: 函数的极点}
\begin{solution}
由于 $e^z-1=z+\dfrac{z^2}{2!}+\cdots$,
\onslide<+->
所以 $0$ 是 $e^z-1$ 的 $1$ 阶零点,
\onslide<+->
从而是 $\dfrac{e^z-1}{z^2}$ 的 $1$ 阶极点.
\end{solution}

\begin{exercise}
求 $f(z)=\dfrac{(z-5)\sin z}{(z-1)^2z^2(z+1)^3}$ 的奇点.
\end{exercise}
\begin{answer}
$1$ 是 $2$ 阶极点, $0$ 是 $1$ 阶极点, $-1$ 是 $3$ 阶极点.
\end{answer}
\end{frame}


\begin{frame}{函数在 $\infty$ 的性态}
\onslide<+->
当我们把复平面扩充成闭复平面后, 从几何上看它变成了一个球面.
\onslide<+->
这样的一个球面是一种封闭的曲面, 它具有某些整体性质.

\onslide<+->
当我们需要计算一个闭路上函数的积分的时候,
\onslide<+->
我们需要研究闭路内部每一个奇点处的洛朗展开,
\onslide<+->
从而得到相应的小闭路上的积分.
\onslide<+->
如果在这个闭路内部的奇点比较多, 而外部的奇点比较少时, 这样计算就不太方便.
\onslide<+->
此时如果通过变量替换 $z=\dfrac1t$, 转而研究闭路外部奇点处的洛朗展开,\onslide<+->
便可减少所需考虑的奇点个数, 从而降低所需的计算量.
\onslide<+->
因此我们需要研究函数在 $\infty$ 的性态.
\end{frame}


\begin{frame}{函数在 $\infty$ 的性态}
\begin{definition}
如果函数 $f(z)$ 在 $\infty$ 的去心邻域 $R<|z|<+\infty$ 内没有奇点, 则称 $\infty$ 是 $f(z)$ 的\markdef{孤立奇点}.
\end{definition}
\onslide<+->
设 $g(t)=f\left(\dfrac1t\right)$, 则研究 $f(z)$ 在 $\infty$ 的性质可以转为研究 $g(t)$ 在 $0$ 的性质.
\onslide<+->
$g(t)$ 在圆环域 $0<|t|<\dfrac1R$ 上解析, $0$ 是它的孤立奇点.
\begin{definition}
如果 $0$ 是 $g(t)$ 的可去奇点 ($m$ 阶极点、本性奇点), 则称 $\infty$ 是 $f(z)$ 的\markdef{可去奇点 ($m$ 阶极点、本性奇点).}
\end{definition}
\end{frame}


\begin{frame}{函数在 $\infty$ 的性态}
\onslide<+->
设 $f(z)$ 在圆环域 $R<|z|<+\infty$ 的洛朗展开为
\[f(z)=\cdots+\frac{c_{-2}}{z^2}+\frac{c_{-1}}{z}+c_0+c_1z+c_2z^2+\cdots\]
\onslide<+->
则 $g(t)$ 在圆环域 $0<|t|<\dfrac1R$ 的洛朗展开为
\[g(t)=\cdots+\frac{c_2}{t^2}+\frac{c_1}t+c_0+c_{-1}t+c_{-2}t^2+\cdots\]
\vspace{-\baselineskip}
\onslide<+->
\begin{center}
\renewcommand\arraystretch{1.8}
\defaultrowcolor
\begin{tabular}{|c|c|c|}
\tht $\infty$ 类型&\tht 洛朗级数特点&\tht $\lim\limits_{z\to\infty}f(z)$\\
可去奇点&没有正幂次部分&存在且有限\\
$m$ 阶极点&\makecell[c]{正幂次部分只有有限项非零\\最高次为 $m$ 次}&$\infty$\\
本性奇点&正幂次部分有无限项非零&	不存在且不为 $\infty$\\
\end{tabular}
\end{center}
\end{frame}


\begin{frame}{例题: $\infty$ 的奇点类型}
\begin{example}
$f(z)=\dfrac z{z+1}$.
\onslide<+->
由 $f(\infty)=\lim\limits_{z\to\infty}f(z)=1$ 可知 $\infty$ 是可去奇点.
\onslide<+->
事实上此时 $f(z)$ 在 $1<|z|<+\infty$ 内的洛朗展开为
\[f(z)=\frac{1}{1+\dfrac1z}=1-\frac1z+\frac1{z^2}-\frac1{z^3}+\cdots\]
\end{example}
\end{frame}


\begin{frame}{例题: $\infty$ 的奇点类型}
\begin{example}
函数 $f(z)=z^2+\dfrac1z$
\onslide<+->
含有正次幂项且最高次为 $2$, 因此 $\infty$ 是 $2$ 阶极点.
\end{example}
\begin{example}
设 $p(z)$ 是 $n\ge1$ 次多项式,
\onslide<+->
则 $\infty$ 是 $p(z)$ 的 $n$ 阶极点.
\end{example}
\end{frame}


\begin{frame}{例题: $\infty$ 的奇点类型}
\begin{example}
函数 
\[\sin z=z-\frac{z^3}{3!}+\frac{z^5}{5!}-\frac{z^7}{7!}+\cdots\]\onslide<+->
含有无限多正次幂项,
\onslide<+->
因此 $\infty$ 是本性奇点.
\onslide<+->
事实上, 如果函数 $f(z)$ 在复平面上处处解析, 且 $f(z)$ 不是多项式, 则 $\infty$ 是它的本性奇点.
\end{example}
\end{frame}


\begin{frame}{典型例题: 奇点的类型}
\begin{example}
函数 $f(z)=\dfrac{(z^2-1)(z-2)^3}{(\sin{\pi z})^3}$ 在扩充复平面内有哪些什么类型的奇点, 并指出奇点的阶.
\end{example}
\begin{solution}
\begin{itemize}
\item 整数 $z=k\neq \pm1,2$ 是 $\sin{\pi z}$ 的 $1$ 阶零点, 因此是 $f(z)$ 的 3 阶极点.
\item $z=\pm1$ 是 $z^2-1$ 的 $1$ 阶零点, 因此是 $f(z)$ 的 $2$ 阶极点.
\item $z=2$ 是 $(z-2)^3$ 的 $3$ 阶零点, 因此是 $f(z)$ 的可去奇点.
\item 由于奇点 $1,2,3,\cdots\to \infty$, 因此 $\infty$ 不是孤立奇点.
\end{itemize}
\end{solution}
\end{frame}


\begin{frame}{典型例题: 奇点的类型}
\begin{exercise}
函数 $f(z)=\dfrac{z^2+4\pi^2}{z^3(e^z-1)}$ 在扩充复平面内有哪些什么类型的奇点, 并指出奇点的阶.
\end{exercise}
\begin{answer}
\begin{itemize}
\item $z=2k\pi i,k\neq \pm1,0$ 是 $1$ 阶极点.
\item $z=0$ 是 $4$ 阶极点.
\item $z=\pm 2\pi i$ 是可去奇点.
\item $z=\infty$ 不是孤立奇点.
\end{itemize}
\end{answer}
\end{frame}


\begin{frame}{例题: 证明复数域是代数封闭的*}
\begin{example}
证明非常数复系数多项式 $p(z)$ 总有复零点.
\end{example}
\vspace{-1pt}
\begin{proof}
\indent
假设多项式 $p(z)$ 没有复零点, 那么 $f(z)=\dfrac1{p(z)}$ 在复平面上处处解析, 
\onslide<+->
从而 $f(z)$ 在 $0$ 处可以展开为幂级数.

\indent
\onslide<+->
由于 $\infty$ 是 $p(z)$ 的极点, $\lim\limits_{z\to\infty}p(z)=\infty$.
\onslide<+->
因此 $\lim\limits_{z\to\infty}f(z)=0$, $\infty$ 是 $f(z)$ 的可去奇点.
\onslide<+->
这意味着 $f(z)$ 在 $0$ 处的洛朗展开没有正幂次项.
\onslide<+->
二者结合可知 $f(z)$ 只能是常数, 矛盾!
\end{proof}
\onslide<+->
设 $z_1$ 是 $n$ 次多项式 $p(z)$ 的零点, 则 $\dfrac{p(z)}{z-z_1}$ 是 $n-1$ 次多项式.
\onslide<+->
归纳可知, $p(z)$ 可以分解为
\[p(z)=(z-z_1)\cdots(z-z_n).\]
\end{frame}

%
%\begin{frame}{例题: 扩充复平面的整体性质*}
%\begin{example}
%如果函数 $f(z)$ 在扩充复平面上的奇点都是极点, 则 $f(z)$ 是有理函数, 且
%\[\sum_{z\in\BC^*}\ord(f,z)=0.\]
%\end{example}
%\begin{proofs}
%\indent
%如果 $f(z)$ 有无限多个奇点, 则存在一串奇点趋向于某个 $z_0\in\BC^*$.
%\onslide<+->
%于是 $z_0$ 是 $f(z)$ 的非孤立奇点, 矛盾!
%\onslide<+->
%因此 $f(z)$ 只有有限多个奇点.
%\end{proofs}
%\end{frame}
%
%
%\begin{frame}{例题: 扩充复平面的整体性质*}
%\beqskip{8pt}
%\begin{proofc}
%\indent
%设 $f(z)$ 在复平面内的奇点为 $z_1,\dots,z_n$, 其中 $z_k$ 为 $d_k$ 阶极点,
%\onslide<+->
%则
%\[g(z)=\prod_{k=1}^n(z-z_k)^{d_k}f(z)\]
%在复平面上处处解析.
%\onslide<+->
%因此 $g(z)$ 在 $0$ 处的洛朗展开没有负幂次项.
%
%\indent
%\onslide<+->
%由于 $\infty$ 是 $g(z)$ 的解析点或极点, 因此 $g(z)$ 在 $0$ 处的洛朗展开至多只有有限多非零正幂次项.
%\onslide<+->
%从而 $g(z)$ 是一个多项式, $f(z)=g(z)/h(z)$ 是一个有理函数.
%\onslide<+->
%\begin{align*}
%&&\sum_{z\in\BC^*}\ord(f,z)&%=\ord(f,\infty)+\sum_{z\in\BC}\ord(g,z)-\sum_{z\in\BC}\ord(h,z)\\
%&\visible<+->{=\deg h-\deg g+\deg g-\deg h=0.}
%\end{align*}
%\vspace{-\baselineskip}
%\end{proofc}
%\endgroup
%\end{frame}
%
