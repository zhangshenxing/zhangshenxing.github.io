\section{傅里叶变换的性质和应用}


\begin{frame}{傅里叶变换的性质}
\onslide<+->
我们不可能也没必要每次都对需要变换的函数从定义出发计算傅里叶变换.
\onslide<+->
通过研究傅里叶变换的性质, 结合常见函数的傅里叶变换, 我们可以得到很多情形的傅里叶变换.

\begin{theorem}[线性性质]
\vspace{-3pt}
\[\msf[\alpha f+\beta g]=\alpha F+\beta G,\quad
\msf^{-1}[\alpha F+\beta G]=\alpha f+\beta g.\]
\end{theorem}

\begin{theorem}[位移性质]
\vspace{-\baselineskip}
\vspace{-3pt}
\[\msf[f(t-t_0)]=e^{-j\omega t_0}F(\omega),\quad
\msf^{-1}[F(\omega-\omega_0)]=e^{j\omega_0 t}f(t).\]
\vspace{-\baselineskip}
\end{theorem}
\onslide<+->
位移性质通过变量替换容易证明.
\onslide<+->
由此可得
\[\markatt{\msf[\delta(t-t_0)]=e^{-j\omega t_0},\quad
\msf^{-1}[\delta(\omega-\omega_0)]=\dfrac1{2\pi}e^{j\omega_0 t}.}\]
\end{frame}


\begin{frame}{傅里叶变换的性质}
\begin{theorem}[微分性质]
\[\msf[f'(t)]=j\omega F(\omega),\quad
\msf^{-1}[F'(\omega)]=-jtf(t),\]
\onslide<+->
\vspace{-\baselineskip}
\[\msf[f^{(k)}(t)]=(j\omega)^k F(\omega),\quad
\msf^{-1}[F^{(k)}(\omega)]=(-jt)^kf(t).\]
\end{theorem}
\onslide<+->
这里, 被变换的函数都要求在 $\infty$ 处趋于 $0$, 下同. 
\onslide<+->
这由
\[\msf[f']=\pair{f'(t),e^{-j\omega t}}=-\pair{f(t),(e^{-j\omega t})'}=j\omega F(\omega)\]
可得.

\begin{theorem}[积分性质]
\[\msf\left[\int_{-\infty}^t f(\tau)\diff\tau\right]=\frac1{j\omega}F(\omega).\]
\end{theorem}
\end{frame}


\begin{frame}{傅里叶变换的性质}
由微分性质可得
\begin{theorem}[乘多项式性质]
\[\msf[tf(t)]=jF'(\omega),\quad
\msf^{-1}[\omega F(\omega)]=-jf'(t),\]
\onslide<+->
\[\msf[t^kf(t)]=j^kF^{(k)}(\omega),\quad
\msf^{-1}[\omega^kF(\omega)]=(-j)^kf^{(k)}(t).\]
\end{theorem}
\onslide<+->
由变量替换易得
\begin{theorem}[相似性质]
\[\msf[f(at)]=\frac1{|a|}F\left(\frac\omega a\right),\quad
\msf^{-1}[F(a\omega)]=\frac1{|a|}f\left(\frac\omega a\right).\]
\end{theorem}
\end{frame}


\begin{frame}{典型例题: 计算傅里叶变换}
\begin{example}
求 $\msf[t^k e^{-\beta t}u(t)],\beta>0$.
\end{example}
\begin{solution}
\vspace{-\baselineskip}
\begin{align*}
\msf[e^{-\beta t}u(t)]&=\frac{1}{\beta+j\omega},\\
\visible<+->{\msf[t^ke^{-\beta t}u(t)]}&\visible<.->{=j^k\left(\frac1{\beta+j\omega}\right)^{(k)}}
  \visible<+->{=\frac{k!}{(\beta+j\omega)^{k+1}}.}
\end{align*}
\end{solution}
\end{frame}


\begin{frame}{典型例题: 计算傅里叶变换}
\begin{example}
求 $\sin{\omega_0 t}$ 的傅里叶变换.
\end{example}
\begin{solution}
由于 $\msf[1]=2\pi\delta(\omega)$,
\onslide<+->
因此 $\msf[e^{j\omega_0t}]=2\pi\delta(\omega-\omega_0)$,
\onslide<+->
\begin{align*}
\markatt{\msf[\sin{\omega_0 t}]}&=\frac1{2j}\left[\msf[e^{j\omega_0t}]-\msf[e^{-j\omega_0t}]\right]&\\
&\visible<+->{=\frac1{2j}[2\pi\delta(\omega-\omega_0)-2\pi\delta(\omega+\omega_0)]}&\\
&\visible<+->{\markatt{=j\pi[\delta(\omega+\omega_0)-\delta(\omega-\omega_0)].}}
\end{align*}
\end{solution}
\end{frame}


\begin{frame}{典型例题: 计算傅里叶变换}
\begin{exercise}
求 $\cos{\omega_0 t}$ 的傅里叶变换.
\end{exercise}
\begin{answer}
\markatt{$\msf[\cos{\omega_0 t}]=\pi[\delta(\omega+\omega_0)+\delta(\omega-\omega_0)]$}.
\end{answer}
\end{frame}


\begin{frame}{卷积}
\begin{definition}
$f_1(t),f_2(t)$ 的\markdef{卷积}是指
\[\markatt{(f_1\ast f_2)(t)=\int_{-\infty}^{+\infty} f_1(\tau)f_2(t-\tau)\diff \tau.}\]
\end{definition}
\onslide<+->
容易验证卷积满足如下性质:
\begin{itemize}
\item $f_1\ast f_2=f_2\ast f_1,\ (f_1\ast f_2)\ast f_3=f_1\ast(f_2\ast f_3)$;
\item $f_1\ast(f_2+f_3)=f_1\ast f_2+f_1\ast f_3$;
\item $f\ast\delta=f$;
\item $(f_1\ast f_2)'=f_1'\ast f_2=f_1\ast f_2'$.
\end{itemize}
\end{frame}


\begin{frame}{例题: 计算卷积}
\begin{example}
设 $f_1(t)=u(t),f_2(t)=e^{-t}u(t)$.
求 $f_1\ast f_2$.
\end{example}
\begin{solution}
\vspace{-\baselineskip}
\[(f_1\ast f_2)(t)=\int_{-\infty}^{+\infty} f_2(\tau)f_1(t-\tau)\diff \tau=\int_0^{+\infty} e^{-\tau}u(t-\tau)\diff \tau.\]
\onslide<+->
当 $t<0$ 时, $(f_1\ast f_2)(t)=0$.
\onslide<+->
当 $t\ge0$ 时, 
\[(f_1\ast f_2)(t)=\int_0^t e^{-\tau}\diff \tau=1-e^{-t}.\]
\onslide<+->
故 $(f_1\ast f_2)(t)=(1-e^{-t})u(t)$.
\end{solution}
\end{frame}


\begin{frame}{卷积定理}
\beqskip{0pt}
\begin{theorem}[卷积定理]
\vspace{-8pt}
\[\msf[f_1\ast f_2]=F_1\cdot F_2,\quad
\msf^{-1}[F_1\ast F_2]=\frac1{2\pi}f_1\cdot f_2.\]
\vspace{-8pt}
\end{theorem}
\begin{proofs}
\vspace{-\baselineskip}
\begin{flalign*}
&&\msf[f_1\ast f_2]&=\int_{-\infty}^{+\infty}\int_{-\infty}^{+\infty}f_1(\tau)f_2(t-\tau)\diff \tau \cdot e^{-j\omega t}\diff t&\\
&&&\visible<+->{=\int_{-\infty}^{+\infty}\int_{-\infty}^{+\infty}f_1(\tau)e^{-j\omega \tau}\cdot f_2(t-\tau)e^{-j\omega (t-\tau)}\diff t\diff \tau}&\\
&&&\visible<+->{=\int_{-\infty}^{+\infty}\int_{-\infty}^{+\infty}f_1(\tau)e^{-j\omega \tau}\cdot f_2(t)e^{-j\omega t}\diff t\diff \tau}&\\
&&&\visible<+->{=\int_{-\infty}^{+\infty}f_1(\tau)e^{-j\omega \tau}\diff\tau\int_{-\infty}^{+\infty}f_2(t)e^{-j\omega t}\diff t}&\\
&&&\visible<+->{=\msf[f_1]\msf[f_2].}\mqed
\end{flalign*}
\end{proofs}
\endgroup
\end{frame}


\begin{frame}{使用傅里叶变换解微积分方程}
\begin{center}
\begin{tikzpicture}[node distance=40pt]
\node[draw,fill=alercolor!25] (p1){微分方程或积分方程};
\node[draw,right=110pt of p1] (p2){象函数的代数方程};
\node[draw,fill=alercolor!25,below=of p1] (p3){原象函数(方程的解)};
\node[draw,below=of p2] (p4){象函数};
\draw[cstnarrow] (p1)--node[above]{傅里叶变换 $\msf$}(p2);
\draw[cstnarrow] (p4)--node[below]{傅里叶逆变换 $\msf^{-1}$}(p3);
\draw[cstnarrow] (p2)--(p4);
\draw[cstnarrow,alecolor] (p1)--(p3);
\end{tikzpicture}
\end{center}
\end{frame}


\begin{frame}{例题: 使用傅里叶变换解微积分方程}
\beqskip{0pt}
\begin{example}
解方程 $x'(t)-\displaystyle\int_{-\infty}^t x(\tau)\diff \tau=2\delta(t)$.
\end{example}
\begin{solution}
设 $\msf[x]=X$, 则两边同时作傅里叶变换得到
\onslide<+->
\[j\omega X(\omega)-\frac1{j\omega}X(\omega)=2,\]
\onslide<+->
\[X(\omega)=-\frac{2j\omega}{1+\omega^2}=\frac1{1+j\omega}-\frac1{1-j\omega},\]
\onslide<+->
\begin{align*}
x(t)=\msf^{-1}\left[\frac1{1+j\omega}-\frac1{1-j\omega}\right]
&\visible<+->{=\begin{cases}
0-e^t=-e^t,&t<0,\\
0,&t=0,\\
e^{-t}-0=e^{-t},&t>0.
\end{cases}}
\end{align*}
\end{solution}
\vspace{-1pt}
\endgroup
\end{frame}


\begin{frame}{例题: 使用傅里叶变换解微积分方程}
\beqskip{8pt}
\begin{example}
解方程 $x''(t)-x(t)=0$.
\end{example}
\begin{solution}
设 $\msf[x]=X$,
\onslide<+->
则
\[\msf[x''(t)-x(t)]=[(j\omega)^2-1]X(\omega)=0,\]%
\onslide<+->
\vspace{-\baselineskip}
\[X(\omega)=0,\quad x(t)=\msf^{-1}[X(\omega)]=0.\]
\end{solution}
\onslide<+->
显然这是不对的, 该方程的解实际上是 $x(t)=C_1e^t+C_2e^{-t}$.
\onslide<+->
原因在于使用傅里叶变换要求函数是绝对可积的, 而非零的 $C_1e^t+C_2e^{-t}$ 并不满足该条件.
\onslide<+->
我们需要一个对函数限制更少的积分变换来解决此类方程, 例如拉普拉斯变换.
\endgroup
\end{frame}

