\section{傅里叶变换的性质和应用}


\begin{frame}{傅里叶变换的性质}
\onslide<+->
我们不可能也没必要每次都对需要变换的函数从定义出发计算傅里叶变换.
\onslide<+->
通过研究傅里叶变换的性质, 结合常见函数的傅里叶变换, 我们可以得到很多情形的傅里叶变换.

\begin{block}{线性性质}
\[\msf[\alpha f+\beta g]=\alpha F+\beta G,\quad
\msf^{-1}[\alpha F+\beta G]=\alpha f+\beta g.\]
\end{block}
\vspace{-5pt}

\begin{block}{位移性质}
\[\msf[f(t-t_0)]=e^{-j\omega t_0}F(\omega),\quad
\msf^{-1}[F(\omega-\omega_0)]=e^{j\omega_0 t}f(t).\]
\vspace{-\baselineskip}
\end{block}
\onslide<+->
位移性质通过变量替换容易证明.
\onslide<+->
由此可得
\[\aboxeq{\msf[\delta(t-t_0)]=e^{-j\omega t_0},\quad
\msf^{-1}[\delta(\omega-\omega_0)]=\dfrac1{2\pi}e^{j\omega_0 t}}.\]
\end{frame}


\begin{frame}{傅里叶变换的性质}
\begin{block}{微分性质}
\[\msf[f'(t)]=j\omega F(\omega),\quad
\msf^{-1}[F'(\omega)]=-jtf(t),\]
\onslide<+->
\vspace{-\baselineskip}
\[\msf[f^{(k)}(t)]=(j\omega)^k F(\omega),\quad
\msf^{-1}[F^{(k)}(\omega)]=(-jt)^kf(t).\]
\end{block}
\onslide<+->
这里, 被变换的函数都要求在 $\infty$ 处趋于 $0$, 下同. 
\onslide<+->
这由
\[\msf[f']=\pair{f'(t),e^{-j\omega t}}=-\pair{f(t),(e^{-j\omega t})'}=j\omega F(\omega)\]
可得.

\begin{block}{积分性质}
\[\msf\left[\int_{-\infty}^t f(\tau)\diff\tau\right]=\frac1{j\omega}F(\omega).\]
\end{block}
\end{frame}


\begin{frame}{傅里叶变换的性质}
由微分性质可得
\begin{block}{乘多项式性质}
\[\msf[tf(t)]=jF'(\omega),\quad
\msf^{-1}[\omega F(\omega)]=-jf'(t),\]
\onslide<+->
\[\msf[t^kf(t)]=j^kF^{(k)}(\omega),\quad
\msf^{-1}[\omega^kF(\omega)]=(-j)^kf^{(k)}(t).\]
\end{block}
\onslide<+->
由变量替换易得
\begin{block}{相似性质}
\[\msf[f(at)]=\frac1{|a|}F\left(\frac\omega a\right),\quad
\msf^{-1}[F(a\omega)]=\frac1{|a|}f\left(\frac t a\right).\]
\end{block}
\end{frame}


\begin{frame}{典型例题: 计算傅里叶变换}
\begin{example}
求 $\msf[t^k e^{-\beta t}u(t)],\beta>0$.
\end{example}
\begin{solution}
\vspace{-\baselineskip}
\begin{align*}
\msf[e^{-\beta t}u(t)]&=\frac{1}{\beta+j\omega},\\
\visible<+->{\msf[t^ke^{-\beta t}u(t)]}&\visible<.->{=j^k\left(\frac1{\beta+j\omega}\right)^{(k)}}
	\visible<+->{=\frac{k!}{(\beta+j\omega)^{k+1}}.}
\end{align*}
\end{solution}
\end{frame}


\begin{frame}{典型例题: 计算傅里叶变换}
\begin{example}
求 $\sin{\omega_0 t}$ 的傅里叶变换.
\end{example}
\begin{solution}
由于 $\msf[1]=2\pi\delta(\omega)$,
\onslide<+->
因此 $\msf[e^{j\omega_0t}]=2\pi\delta(\omega-\omega_0)$,
\onslide<+->
\begin{align*}
\aboxeq{\msf[\sin{\omega_0 t}]}&=\frac1{2j}\left[\msf[e^{j\omega_0t}]-\msf[e^{-j\omega_0t}]\right]&\\
&\visible<+->{=\frac1{2j}[2\pi\delta(\omega-\omega_0)-2\pi\delta(\omega+\omega_0)]}&\\
&\visible<+->{\aboxeq{=j\pi[\delta(\omega+\omega_0)-\delta(\omega-\omega_0)]}.}
\end{align*}
\end{solution}
\end{frame}


\begin{frame}{典型例题: 计算傅里叶变换}
\begin{exercise}
求 $\cos{\omega_0 t}$ 的傅里叶变换.
\end{exercise}
\begin{answer}
\abox{$\msf[\cos{\omega_0 t}]=\pi[\delta(\omega+\omega_0)+\delta(\omega-\omega_0)]$}.
\end{answer}
\end{frame}


\begin{frame}{卷积}
\begin{definition}
$f_1(t),f_2(t)$ 的\emph{卷积}是指
\[\alert{(f_1\ast f_2)(t)=\int_{-\infty}^{+\infty} f_1(\tau)f_2(t-\tau)\diff \tau.}\]
\end{definition}
\onslide<+->
容易验证卷积满足如下性质:
\begin{itemize}
\item $f_1\ast f_2=f_2\ast f_1,\ (f_1\ast f_2)\ast f_3=f_1\ast(f_2\ast f_3)$;
\item $f_1\ast(f_2+f_3)=f_1\ast f_2+f_1\ast f_3$;
\item $f\ast\delta=f$;
\item $(f_1\ast f_2)'=f_1'\ast f_2=f_1\ast f_2'$.
\end{itemize}
\end{frame}


\begin{frame}{例题: 计算卷积}
\begin{example}
设 $f_1(t)=u(t),f_2(t)=e^{-t}u(t)$.
求 $f_1\ast f_2$.
\end{example}
\begin{solution}
\vspace{-\baselineskip}
\[(f_1\ast f_2)(t)=\int_{-\infty}^{+\infty} f_2(\tau)f_1(t-\tau)\diff \tau=\int_0^{+\infty} e^{-\tau}u(t-\tau)\diff \tau.\]
\onslide<+->
当 $t<0$ 时, $(f_1\ast f_2)(t)=0$.
\onslide<+->
当 $t\ge0$ 时, 
\[(f_1\ast f_2)(t)=\int_0^t e^{-\tau}\diff \tau=1-e^{-t}.\]
\onslide<+->
故 $(f_1\ast f_2)(t)=(1-e^{-t})u(t)$.
\end{solution}
\end{frame}


\begin{frame}{卷积定理}
\beqskip{0pt}
\begin{block}{卷积定理}
\[\msf[f_1\ast f_2]=F_1\cdot F_2,\quad
\msf^{-1}[F_1\ast F_2]=\frac1{2\pi}f_1\cdot f_2.\]
\vspace{-8pt}
\end{block}
\begin{proofs}
\begin{align*}
\msf[f_1\ast f_2]&=\int_{-\infty}^{+\infty}\int_{-\infty}^{+\infty}f_1(\tau)f_2(t-\tau)\diff \tau \cdot e^{-j\omega t}\diff t\\
&\visible<+->{=\int_{-\infty}^{+\infty}\int_{-\infty}^{+\infty}f_1(\tau)e^{-j\omega \tau}\cdot f_2(t-\tau)e^{-j\omega (t-\tau)}\diff t\diff \tau}\\
&\visible<+->{=\int_{-\infty}^{+\infty}\int_{-\infty}^{+\infty}f_1(\tau)e^{-j\omega \tau}\cdot f_2(t)e^{-j\omega t}\diff t\diff \tau}\\
&\visible<+->{=\int_{-\infty}^{+\infty}f_1(\tau)e^{-j\omega \tau}\diff\tau\int_{-\infty}^{+\infty}f_2(t)e^{-j\omega t}\diff t}\\
&\visible<+->{=\msf[f_1]\msf[f_2].}\hspace{180pt}\mqed
\end{align*}
\end{proofs}
\endgroup
\end{frame}


\begin{frame}{卷积的应用: 奇怪的积分规律}
	\begin{example}
		使用计算机进行数值计算可以发现
		\begin{align*}
			I_1=\int_{-\infty}^{+\infty}\frac{\sin \omega}{\omega}\diff\omega&=3.1415926535897\cdots\\
			I_3=\int_{-\infty}^{+\infty}\frac{\sin \omega}{\omega}\cdot \frac{\sin 3\omega}{3\omega}\diff\omega&=3.1415926535897\cdots\\
			I_5=\int_{-\infty}^{+\infty}\frac{\sin \omega}{\omega}\cdot \frac{\sin 3\omega}{3\omega}\cdot
			\frac{\sin 5\omega}{5\omega}\diff\omega&=3.1415926535897\cdots\\
			I_{13}=\int_{-\infty}^{+\infty}\frac{\sin \omega}{\omega}\cdot \frac{\sin 3\omega}{3\omega}\cdots
			\frac{\sin 13\omega}{13\omega}\diff\omega&=3.1415926535897\cdots\\
			I_{15}=\int_{-\infty}^{+\infty}\frac{\sin \omega}{\omega}\cdot \frac{\sin 3\omega}{3\omega}\cdots
			\frac{\sin 15\omega}{1\omega}\diff\omega&=3.1415926535\alert{435\cdots}
		\end{align*}
	\end{example}
\end{frame}


\begin{frame}{卷积的应用: 奇怪的积分规律}
	\begin{solution}
		设 $F(\omega)=\dfrac{\sin\omega}{\omega}$, 则
		$\displaystyle\frac{I_k}{2\pi}=\frac1{2\pi}\int_{-\infty}^{+\infty}F(\omega)\cdots F(k\omega)\diff\omega$ 是被积函数的傅里叶逆变换在 $0$ 的取值.
		我们之前计算过
		\[f(t)=\begin{cases}
			1/2, & |t|<1,\\
			1/4, & |t|=1,\\
			0, & |t|>1
		\end{cases}\]
		的傅里叶变换是 $F(\omega)$, 所以 $F(m\omega)$ 的傅里叶逆变换为 $\dfrac1mf(\dfrac tm)$, 
		被积函数的傅里叶逆变换为
		$\displaystyle h_k(t)=f(t)\ast \frac13f(\frac t3)\ast\cdots\ast \frac1 kf(\frac tk)$.
	\end{solution}
	\end{frame}


\begin{frame}{卷积的应用: 奇怪的积分规律}
	\begin{solutionc}
		注意到 $\frac1mf(\frac tm)$ 在整个实轴上的积分为 $1$, 一个函数 $g(t)$ 和它做卷积之后在 $t$ 处的值相当于 $g(t)$ 在 $[t-\frac1m,t+\frac1m]$ 上取平均值.
		由此可知, 
		\begin{itemize}
			\item $f(t)\ast \frac13f(\frac13)$ 在 $|t|<1-\frac13$ 上取值为 $1/2$;
			\item $f(t)\ast \frac13f(\frac13)\ast \frac15f(\frac15)$ 在 $|t|<1-\frac13-\frac15$ 上取值为 $1/2$;
			\item 最终 $h_k(t)$ 在 $|t|<1-\frac13-\cdots-\frac1k$ 上取值为 $1/2$.
			\item 而 $\frac13+\cdots+\frac1{13}\approx 0.955$, $\frac13+\cdots+\frac1{15}\approx 1.02$, 所以
			\[I_1=I_3=\cdots=I_{13}=2\pi h_{13}(0)=\pi,\]
			\[I_{15}=2\pi h_{15}(0)=\frac{467 807 924 713 440 738 696 537 864 469}{467 807 924 720 320 453 655 260 875 000}\pi.\]
		\end{itemize}
	\end{solutionc}
\end{frame}


\begin{frame}{使用傅里叶变换解微积分方程}
\begin{center}
\begin{tikzpicture}[node distance=40pt]
\node[cstnodeg] (p1){微分方程或积分方程};
\node[cstnodeb,right=110pt of p1] (p2){象函数的代数方程};
\node[cstnodeg,below=of p1] (p3){原象函数(方程的解)};
\node[cstnodeb,below=of p2] (p4){象函数};
\draw[cstnarrow,dcolorb] (p1)--node[above]{傅里叶变换 $\msf$}(p2);
\draw[cstnarrow,dcolorb] (p4)--node[below]{傅里叶逆变换 $\msf^{-1}$}(p3);
\draw[cstnarrow,dcolorb] (p2)--(p4);
\draw[cstnarrow,dcolorc] (p1)--(p3);
\end{tikzpicture}
\end{center}
\end{frame}


\begin{frame}{例题: 使用傅里叶变换解微积分方程}
\beqskip{0pt}
\begin{example}
解方程 $x'(t)-\displaystyle\int_{-\infty}^t x(\tau)\diff \tau=2\delta(t)$.
\end{example}
\begin{solution}
设 $\msf[x]=X$, 则两边同时作傅里叶变换得到
\onslide<+->
\[j\omega X(\omega)-\frac1{j\omega}X(\omega)=2,\]
\onslide<+->
\[X(\omega)=-\frac{2j\omega}{1+\omega^2}=\frac1{1+j\omega}-\frac1{1-j\omega},\]
\onslide<+->
\begin{align*}
x(t)=\msf^{-1}\left[\frac1{1+j\omega}-\frac1{1-j\omega}\right]
&\visible<+->{=\begin{cases}
0-e^t=-e^t,&t<0,\\
0,&t=0,\\
e^{-t}-0=e^{-t},&t>0.
\end{cases}}
\end{align*}
\end{solution}
\vspace{-1pt}
\endgroup
\end{frame}


\begin{frame}{例题: 使用傅里叶变换解微积分方程}
\beqskip{8pt}
\begin{example}
解方程 $x''(t)-x(t)=0$.
\end{example}
\begin{solution}
设 $\msf[x]=X$,
\onslide<+->
则
\[\msf[x''(t)-x(t)]=[(j\omega)^2-1]X(\omega)=0,\]%
\onslide<+->
\vspace{-\baselineskip}
\[X(\omega)=0,\quad x(t)=\msf^{-1}[X(\omega)]=0.\]
\end{solution}
\onslide<+->
显然这是不对的, 该方程的解实际上是 $x(t)=C_1e^t+C_2e^{-t}$.
\onslide<+->
原因在于使用傅里叶变换要求函数是绝对可积的, 而非零的 $C_1e^t+C_2e^{-t}$ 并不满足该条件.
\onslide<+->
我们需要一个对函数限制更少的积分变换来解决此类方程, 例如拉普拉斯变换.
\endgroup
\end{frame}


% {
% \homework
% \begin{frame}[<*>]{作业}
% 	\begin{homeworks}
% 		\item(2020年A卷) 下列不是傅里叶变换对的是\fillbrace.
% 		\xx{$\delta(t),1$}%
% 			{$e^{j\omega_0t},2\pi\delta(\omega-\omega_0)$}%
% 			{$\sin \omega_0t, \pi[\delta(\omega-\omega_0)+\delta(\omega+\omega_0)]$}%
% 			{$1,2\pi\delta(\omega)$}
% 		\item(2020年B卷) $F(t)=\delta(t-t_0)$ 的傅里叶变换为\fillblank{}.
% 		\item(2021年A卷) $F(\omega)=\delta(\omega+2)$ 的傅里叶逆变换为\fillblank{}.
% 		\item(2022年A卷) 函数 $\sin t+j\cos t$ 的傅里叶变换为\fillblank{}.
% 	\end{homeworks}
% \end{frame}
% }
