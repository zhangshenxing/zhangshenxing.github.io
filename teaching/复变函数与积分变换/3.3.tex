\section{原函数和不定积分}


\begin{frame}{原函数的存在性}
\onslide<+->
设 $f(z)$ 在单连通域 $D$ 内解析, $C$ 是 $D$ 内一条起于 $z_0$ 终于 $z$ 的曲线.
\onslide<+->
由柯西-古萨基本定理可知, 积分 $\displaystyle\int_Cf(\zeta)\diff \zeta$ 与路径无关, 只与 $z_0,z$ 有关.
\onslide<+->
因此我们也将其记为 $\displaystyle\int_{z_0}^zf(\zeta)\diff\zeta$.

\onslide<+->
对于任意固定的 $z_0\in D$, 函数
\[F(z)=\int_{z_0}^zf(\zeta)\diff\zeta\]
定义了一个单值函数.

\onslide<+->
\begin{theorem}
$F(z)$ 是 $D$ 内的解析函数, 且 $F'(z)=f(z)$.
\end{theorem}
\end{frame}


\begin{frame}{原函数的存在性}
\onslide<+->
\begin{proofs}
\begin{center}
\begin{tikzpicture}[overlay,xshift=3.8cm,yshift=-1cm]
\filldraw[cstcurve,dcolora,rounded corners=0.5cm,cstfill] (-2.5,-0.8) rectangle (2,1);
\draw[cstcurve,dcolorb] (0,0) circle(0.7);
\draw[cstcurve,dcolorc] (-2,0)to [bend left](0,0);
\draw[cstcurve,dcolorc,visible on=<2->] (0,0)--(0.4,0.4);
\fill[cstdot,dcolora] (-2,0) circle;
\fill[cstdot,dcolora] (0,0) circle;
\fill[cstdot,dcolora,visible on=<2->] (0.4,0.4) circle;
\draw
  (-2,-0.3) node[dcolorb] {$z_0$}
  (0,-0.3) node[dcolorb] {$z$}
  (1.1,0.7) node[dcolorb,visible on=<2->] {$z+\Delta z$};
\end{tikzpicture}
\end{center}

以 $z$ 为中心作一包含在 $D$ 内的圆 $K$,

\onslide<+->{取 $|\Delta z|$ 小于 $K$ 的半径.}
\onslide<+->{那么
\[F(z+\Delta z)-F(z)=\int_{z_0}^{z+\Delta z}f(\zeta)\diff\zeta-\int_{z_0}^zf(\zeta)\diff\zeta
\visible<+->{=\int_z^{z+\Delta z}f(\zeta)\diff\zeta.}\]}

\onslide<+->{容易知道
\[\int_z^{z+\Delta z}f(z)\diff\zeta=f(z)\int_z^{z+\Delta z}\diff\zeta=f(z)\Delta z.\]}
\onslide<+->{我们需要比较上述两个积分, 其中 $z$ 到 $z+\Delta z$ 取直线.}
\end{proofs}
\end{frame}


\begin{frame}{原函数的存在性}
\onslide<+->
\begin{proofe}
由于 $f(z)$ 解析, 因此连续.
\onslide<+->{$\forall\varepsilon>0,\exists\delta>0$ 使得当 $|\zeta-z|<\delta$ 时, $z$ 落在 $K$ 中且 $|f(\zeta)-f(z)|<\varepsilon$.}
\onslide<+->{当 $|\Delta z|<\delta$ 时, 由长大不等式
\begin{align*}
\abs{\frac{F(z+\Delta z)-F(z)}{\Delta z}-f(z)}
&\visible<+->{=\abs{\int_z^{z+\Delta z}\frac{f(\zeta)-f(z)}{\Delta z}\diff \zeta}}\\
&\visible<+->{\le\frac{\varepsilon}{|\Delta z|}\cdot|\Delta z|=\varepsilon.}
\end{align*}}
\onslide<+->{由于 $\varepsilon$ 是任意的, 因此
\[f(z)=\lim_{\Delta z\to 0}\frac{F(z+\Delta z)-F(z)}{\Delta z}=F'(z).\qedhere\]}
\end{proofe}
\end{frame}


\begin{frame}{原函数的存在性}
\onslide<+->
\begin{alertblock}{牛顿-莱布尼兹定理}
设 $f(z)$ 在单连通区域 $D$ 上解析, $z_1$ 至 $z_2$ 的积分路径落在 $D$ 内, 则
\[\int_{z_1}^{z_2}f(z)\diff z=F(z_1)-F(z_2),\quad\text{其中}\quad F'(z)=f(z).\]
\end{alertblock}

\onslide<+->
如果 $D$ 上的解析函数 $G(z)$ 满足 $G'(z)=f(z)$, 则称 $G(z)$ 是 $f(z)$ 的一个\emph{原函数}.
\onslide<+->
由于导函数为 $0$ 的解析函数只能是常值函数,
\onslide<+->
因此 $\displaystyle G(z)=\int_{z_0}^zf(z)\diff z+C$.
\onslide<+->
我们称之为 $f(z)$ 的\emph{不定积分}, 记为 \emph{$\displaystyle\int f(z)\diff z$}.

\onslide<+->
复变函数和实变函数的牛顿-莱布尼兹定理的差异在哪呢?
\onslide<+->
复变情形要求是\alert{单连通区域上解析函数}, 实变情形要求是\alert{闭区间上连续函数}.
\end{frame}


\begin{frame}{典型例题: 利用原函数求积分}
\onslide<+->
\begin{example}
求 $\displaystyle\int_{z_0}^{z_1}z\diff z$.
\end{example}
\onslide<+->
\begin{solution}
由于 $f(z)=z$ 处处解析,
\onslide<+->{且 $\displaystyle\int z\diff z=\frac12 z^2+C$,}
\onslide<+->{因此
\[\int_{z_0}^{z_1}z\diff z=\frac12z^2\big|_{z_0}^{z_1}=\frac12(z_1^2-z_0^2).\]}
\end{solution}
\onslide<+->
因此之前的例子中 $\displaystyle\int_0^{3+4i}z\diff z=-\frac72+12i$, 而无论从 $0$ 到 $3+4i$ 的路径如何.
\end{frame}


\begin{frame}{典型例题: 利用原函数求积分}
\onslide<+->
\begin{example}
求 $\displaystyle\int_0^{\pi i}z\cos z^2\diff z$.
\end{example}
\onslide<+->
\begin{solution}
由于 $f(z)=z\cos z^2$ 处处解析,
\onslide<+->{且
\[\int z\cos z^2\diff z=\frac12\int \cos z^2\diff z^2=\frac12\sin z^2+C,\]}
\onslide<+->{因此
\[\int_0^{\pi i}z\cos z^2\diff z=\frac12\sin z^2\big|_0^{\pi i}=-\frac12\sin \pi^2.\]}
\end{solution}
\onslide<+->
这里我们使用了\alert{凑微分法}.
\end{frame}


\begin{frame}{典型例题: 利用原函数求积分}
\onslide<+->
\begin{example}
求 $\displaystyle\int_0^i z\cos z\diff z$.
\end{example}
\onslide<+->
\begin{solution}
由于 $f(z)=z\cos z$ 处处解析,
\onslide<+->{且
\begin{align*}
\int z\cos z\diff z&
=\int z\diff(\sin z)=z\sin z-\int \sin z\diff z\\
&\visible<+->{=z\sin z+\cos z+C,}
\end{align*}}
\onslide<+->{因此
\vspace{-1.5\baselineskip}
\begin{align*}
\int_0^i z\cos z\diff z&=(z\sin z+\cos z)\big|_0^i\\
&\visible<+->{=i\sin i+\cos i-1=e^{-1}-1.}
\end{align*}}
\vspace{-1.5\baselineskip}
\end{solution}
\onslide<+->
这里我们使用了\alert{分部积分法}.
\end{frame}


\begin{frame}{典型例题: 利用原函数求积分}
\onslide<+->
\begin{example}
求 $\displaystyle\int_1^{1+i} z e^z\diff z$.
\end{example}
\onslide<+->
\begin{solution}
由于 $f(z)=ze^z$ 处处解析,
\onslide<+->{且
\[\int z e^z\diff z=\int z\diff e^z=ze^z-\int e^z\diff z=(z-1)e^z+c,\]}
\onslide<+->{因此
\vspace{-\baselineskip}
\begin{align*}
\int_1^{1+i} z e^z\diff z&=(z-1)e^z\big|_1^{1+i}\\
&\visible<+->{=ie^{1+i}=e(-\sin 1+i\cos 1).}
\end{align*}}
\vspace{-\baselineskip}
\end{solution}
\end{frame}


\begin{frame}{典型例题: 利用原函数求积分}
\onslide<+->
\begin{exercise}
求 $\displaystyle\int_0^1 z\sin z\diff z$.
\end{exercise}
\onslide<+->

\begin{answer}
$\sin 1-\cos 1$.
\end{answer}
\end{frame}


\begin{frame}{典型例题: 利用原函数求积分}
\onslide<+->
\begin{example}
设 $C$ 为沿着 $|z|=1$ 从 $1$ 到 $i$ 的逆时针圆弧, 求 $\displaystyle\int_C\frac{\ln(z+1)}{z+1}\diff z$.
\end{example}
\onslide<+->
\begin{solution}
函数 $f(z)=\dfrac{\ln(z+1)}{z+1}$ 在 $\Re z\le -1$ 外的单连通区域解析.
\vspace{-0.5\baselineskip}
\onslide<+->{
  \[\int\frac{\ln(z+1)}{z+1}\diff z
=\int\ln(z+1)\diff[\ln(z+1)]=\frac12\ln^2(z+1)+c.\]}
\onslide<+->{因此
\vspace{-1.5\baselineskip}
\begin{align*}
&\peq\int_C\frac{\ln(z+1)}{z+1}\diff z=\frac12\ln^2(z+1)\big|_1^i
\visible<+->{=\frac12\left[\ln^2(1+i)-\ln^22\right]}\\
&\visible<+->{=\frac12\left[\left(\ln\sqrt2+\frac\pi4i\right)^2-\ln^22\right]
=-\frac{\pi^2}{32}-\frac38\ln^22+\frac{\pi\ln2}{8}i.}
\end{align*}}
\vspace{-1.1\baselineskip}
\end{solution}
\end{frame}


\begin{frame}{典型例题: 利用原函数求积分}
\onslide<+->
\begin{example}
求 $\displaystyle\int_C(2z^2+8z+1)\diff z$, 其中 $C$ 是连接 $0$ 到 $2\pi a$ 的摆线

$\displaystyle\begin{cases}
x=a(\theta-\sin\theta),& \\ y=a(1-\cos\theta),
\end{cases} 0\le \theta\le 2\pi.$
\rule{0pt}{1cm}
\begin{tikzpicture}[overlay,xshift=0.7cm,yshift=-0.6cm]
\draw[cstaxis,visible on=<2->](0,0)--(5,0);
\draw[cstaxis,visible on=<2->](0,0)--(0,1.8);
\draw[cstcurve,dcolora,smooth,domain=0:360,visible on=<9->] plot ({0.7*(pi/180*\x-sin(\x))}, {0.7*(1-cos(\x))});
\draw[cstcurve,dcolorb,visible on=<2>] (0,0.7) circle (0.7);
\fill[cstdot,dcolora,visible on=<2->] (0,0) circle;
\draw[cstcurve,dcolorb,visible on=<3>] ({0.7*(pi/180*60)},0.7) circle (0.7);
\fill[cstdot,dcolora,visible on=<3->] ({0.7*(pi/180*60-sin(60))},{0.7*(1-cos(60))}) circle;
\draw[cstcurve,dcolorb,visible on=<4>] ({0.7*(pi/180*120)},0.7) circle (0.7);
\fill[cstdot,dcolora,visible on=<4->] ({0.7*(pi/180*120-sin(120))},{0.7*(1-cos(120))}) circle;
\draw[cstcurve,dcolorb,visible on=<5>] ({0.7*(pi/180*180)},0.7) circle (0.7);
\fill[cstdot,dcolora,visible on=<5->] ({0.7*(pi/180*180-sin(180))},{0.7*(1-cos(180))}) circle;
\draw[cstcurve,dcolorb,visible on=<6>] ({0.7*(pi/180*240)},0.7) circle (0.7);
\fill[cstdot,dcolora,visible on=<6->] ({0.7*(pi/180*240-sin(240))},{0.7*(1-cos(240))}) circle;
\draw[cstcurve,dcolorb,visible on=<7>] ({0.7*(pi/180*300)},0.7) circle (0.7);
\fill[cstdot,dcolora,visible on=<7->] ({0.7*(pi/180*300-sin(300))},{0.7*(1-cos(300))}) circle;
\draw[cstcurve,dcolorb,visible on=<8>] ({0.7*(pi/180*360)},0.7) circle (0.7);
\fill[cstdot,dcolora,visible on=<8->] ({0.7*(pi/180*360-sin(360))},{0.7*(1-cos(360))}) circle;
\end{tikzpicture}
\end{example}
\onslide<10->{
\begin{solution}
由于 $f(z)=2z^2+8z+1$ 处处解析,
\onslide<11->{因此
\begin{align*}
&\peq\int_C(2z^2+8z+1)\diff z=\int_0^{2\pi a}(2z^2+8z+1)\diff z\\
&\visible<+->{=\left(\frac23z^3+4z^2+z\right)\bigg|_0^{2\pi a}=\frac{16}3\pi^3a^3+16\pi^2a^2+2\pi a.}
\end{align*}}
\vspace{-\baselineskip}
\end{solution}}
\end{frame}


