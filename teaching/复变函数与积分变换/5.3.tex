\section{留数在定积分的应用*}


\begin{frame}{形如 $\displaystyle\int_0^{2\pi} R(\cos\theta,\sin\theta)\diff\theta$ 的积分}
\onslide<+->
本节中我们将对若干种在实变中难以计算的定积分和广义积分使用复变函数和留数的技巧进行计算.

\onslide<+->
考虑 $\displaystyle\int_0^{2\pi} R(\cos\theta,\sin\theta)\diff\theta$, 其中 $R$ 是一个有理函数.
\onslide<+->
令 $z=e^{i\theta}$, 则 $\diff z=iz\diff\theta$,
\onslide<+->
\[\cos\theta=\frac12\left(z+\frac1z\right)=\frac{z^2+1}{2z},\quad
\sin\theta=\frac1{2i}\left(z-\frac1z\right)=\frac{z^2-1}{2iz},\]
\onslide<+->
\[\markatt{\int_0^{2\pi} R(\cos\theta,\sin\theta)\diff\theta=\oint_{|z|=1}R\left(\frac{z^2+1}{2z},\frac{z^2-1}{2iz}\right)\frac1{iz}\diff z.}\]
\onslide<+->
由于被积函数是一个有理函数, 它的积分可以由 $|z|<1$ 内奇点留数得到.
\end{frame}


\begin{frame}{例题: 第一类积分}
\beqskip{2pt}
\begin{example}
求 $\displaystyle\int_0^{2\pi}\frac{\sin^2\theta}{5-3\cos\theta}\diff\theta$.
\end{example}

\begin{solution}
令 $z=e^{i\theta}$, 则 $\diff z=iz\diff\theta$,
\onslide<+->
\[\cos\theta=\frac12\left(z+\frac1z\right)=\frac{z^2+1}{2z},\quad
\sin\theta=\frac1{2i}\left(z-\frac1z\right)=\frac{z^2-1}{2iz},\]
\onslide<+->
\begin{align*}
\int_0^{2\pi}\frac{\sin^2\theta}{5-3\cos\theta}\diff\theta
&=\oint_{|z|=1}\frac{(z^2-1)^2}{-4z^2}\cdot\frac1{5-3\frac{z^2+1}{2z}}\cdot\frac{\diff z}{iz}\\
&=-\frac i6\oint_{|z|=1}\frac{(z^2-1)^2}{z^2(z-3)(z-\frac13)}\diff z.
\end{align*}
\end{solution}
\endgroup
\end{frame}


\begin{frame}{例题: 留数在定积分上的应用}
\begin{solution}
设 $f(z)=\dfrac{(z^2-1)^2}{z^2(z-3)(z-\frac13)}$,
\onslide<+->
则
\[\Res[f(z),0]=\frac{10}3,\quad
\Res[f(z),\frac13]=-\frac83,\]
\onslide<+->
\begin{align*}
\int_0^{2\pi}\frac{\sin^2\theta}{5-3\cos\theta}\diff\theta
&=-\frac i6\cdot 2\pi i\Bigl[\Res[f(z),0]+\Res[f(z),\frac13]\Bigr]\\
&=\frac{2\pi}9.
\end{align*}
\end{solution}
\end{frame}


\begin{frame}{形如 $\displaystyle\int_{-\infty}^{+\infty}R(x)\diff x$ 的积分}
\onslide<+->
考虑 $\displaystyle\int_{-\infty}^{+\infty}R(x)\diff x$, 其中 $R(x)$ 是一个有理函数, 分母比分子至少高 $2$ 次, 且分母没有实根.
\onslide<+->
我们先考虑 $\displaystyle\int_{-r}^rR(x)\diff x$.
\onslide<+->
设 $f(z)=R(z),C=C_r+[-r,r]$ 如下图所示, 使得上半平面内 $f(z)$ 的奇点均在 $C$ 内,
\onslide<+->
则
\[2\pi i\sum_{\Im a>0}\Res[f(z),a]=\oint_Cf(z)\diff z=\int_{-r}^rR(x)\diff x+\int_{C_r}f(z)\diff z.\]
\onslide<3->
\begin{center}
\begin{tikzpicture}
\draw[cstaxis] (-2,0)--(2,0);
\draw[cstaxis] (0,-0.2)--(0,2);
\draw[cstcurve,dcolora] (-1.5,0) arc(180:0:1.5);
\draw[cstcurve,dcolora,cstarrowfrom] (-1.2,0.9) arc(135:130:1.5);
\draw[cstcurve,dcolorb] (-1.5,0)--(1.5,0);
\draw[cstcurve,dcolorb,cstarrowto] (-1.5,0)--(-0.5,0);
\draw
  (-1.5,-0.3) node[dcolorb] {$-r$}
  (1.5,-0.3) node[dcolorb] {$r$}
  (1.3,1.2) node[dcolora] {$C_r$};
\end{tikzpicture}
\end{center}
\end{frame}


\begin{frame}{形如 $\displaystyle\int_{-\infty}^{+\infty}R(x)\diff x$ 的积分}
\onslide<+->
由于 $P(x)$ 分母次数比分子至少高 $2$ 次,
\onslide<+->
当 $r\to+\infty$ 时,
\[\abs{\int_{C_r}f(z)\diff z}\le \pi r\max_{|z|=r}|f(z)|
=\pi \max_{|z|=r}|zf(z)|\to 0.\]
\onslide<+->
故
\[\markatt{\int_{-\infty}^{+\infty}R(x)\diff x=2\pi i\sum_{\Im a>0}\Res[R(z),a].}\]
\end{frame}


\begin{frame}{例题: 留数在定积分上的应用}
\beqskip{3pt}
\begin{example}
求 $\displaystyle\int_{-\infty}^{+\infty}\frac{\diff x}{(x^2+a^2)^3},a>0$.
\end{example}

\begin{solution}
$f(z)=\dfrac1{(z^2+a^2)^3}$ 在上半平面内的奇点为 $ai$.
\onslide<+->
\begin{align*}
\Res[f(z),ai]&=\frac1{2!}\lim_{z\to ai}\left[\frac1{(z+ai)^3}\right]''\\
&=\frac12\lim_{z\to ai}\frac{12}{(z+ai)^5}=\frac{3}{16a^5i},
\end{align*}
\onslide<+->
故
\[\int_{-\infty}^{+\infty}\frac{\diff x}{(x^2+a^2)^3}
=2\pi i\Res[f(z),ai]=\frac{3\pi}{8a^5}.\]
\end{solution}
\endgroup
\end{frame}


\begin{frame}{形如 $\displaystyle\int_{-\infty}^{+\infty}R(x)\cos{\lambda x}\diff x$,
$\displaystyle\int_{-\infty}^{+\infty}R(x)\sin{\lambda x}\diff x$ 的积分}
\onslide<+->
考虑 $\displaystyle\int_{-\infty}^{+\infty}R(x)\cos{\lambda x}\diff x$,
$\displaystyle\int_{-\infty}^{+\infty}R(x)\sin{\lambda x}\diff x$, 其中 $R(x)$ 是一个有理函数, 分母比分子至少高 $2$ 次, 且分母没有实根.
\onslide<+->
和前一种情形类似, 我们有
\[\markatt{\int_{-\infty}^{+\infty}R(x)e^{i\lambda x}\diff x
=2\pi i\sum_{\Im a>0}\Res[R(z)e^{i\lambda z},a],}\]
\onslide<+->
因此所求积分分别为它的实部和虚部.
\end{frame}


\begin{frame}{例题: 留数在定积分上的应用}
\beqskip{0pt}
\begin{example}
求 $\displaystyle\int_{-\infty}^{+\infty}\frac{\cos x\diff x}{(x^2+a^2)^2}, a>0$.
\end{example}
\begin{solution}
$f(z)=\dfrac{e^{iz}}{(z^2+a^2)^2}$ 在上半平面内的奇点为 $ai$,
\onslide<+->
\[\Res[f(z),ai]=\lim_{z\to ai}\left[\frac{e^{iz}}{(z+ai)^2}\right]'=-\frac{e^{-a}(a+1)i}{4a^3}.\]
\onslide<+->
故
\[\int_{-\infty}^{+\infty}\frac{e^{ix}\diff x}{(x^2+a^2)^2}=2\pi i \Res[f(z),ai]=\frac{\pi e^{-a}(a+1)}{2a^3},\]
\onslide<+->
\[\int_{-\infty}^{+\infty}\frac{\cos x\diff x}{(x^2+a^2)^2}=\frac{\pi e^{-a}(a+1)}{2a^3}.\]
\end{solution}
\endgroup
\end{frame}

%
%\begin{frame}{例题: 留数在定积分上的应用}
%\beqskip{5pt}
%\onslide<+->
%最后我们再来看两个例子.
%
%\begin{example}
%求菲涅尔积分
%\[I_1=\int_0^{+\infty}\cos x^2\diff x,\quad
%I_2=\int_0^{+\infty}\sin x^2\diff x.\]
%\end{example}
%\begin{solution}
%设 $f(z)=\exp(iz^2)$, $C$ 为下图所示闭路.
%\onslide<+->
%\vspace{-2pt}
%\begin{center}
%\begin{tikzpicture}
%\draw[cstaxis] (0,0)--(2.8,0);
%\draw[cstaxis] (0,0)--(0,2);
%\draw[cstcurve,dcolora] (2,0)coordinate (A)--(0,0)coordinate (B)--(2,2)coordinate (C) pic [draw,dcolorb,"$\frac\pi4$",angle eccentricity=1.5] {angle};
%\draw[cstcurve,dcolora] (2,0)--(2,2);
%\draw[cstcurve,dcolora,cstarrowto] (0,0)--(1.2,0);
%\draw[cstcurve,dcolora,cstarrowto] (2,0)--(2,1.2);
%\draw[cstcurve,dcolora,cstarrowto] (2,2)--(0.8,0.8);
%\draw
%  (-0.2,0.2) node {$O$}
%  (2.3,0.3) node {$R$}
%  (2.7,2) node {$R+Ri$}
%  (2.4,1) node[dcolora] {$C_1$}
%  (0.7,1.3) node[dcolora] {$C_2$};
%\end{tikzpicture}
%\end{center}
%\vspace{-2pt}
%\end{solution}
%\endgroup
%\end{frame}
%
%
%\begin{frame}{例题: 留数在定积分上的应用}
%\begin{solutionc}
%由于 $f(z)$ 处处解析, 因此
%\[\int_0^Re^{ix^2}\diff x+\int_{C_1}f(z)\diff z+\int_{C_2}f(z)\diff z=\oint_Cf(z)\diff z=0.\]
%\onslide<+->
%由于 $C_1:z=R+yi,f(z)=\exp\bigl[i(R+yi)^2\bigr]$,
%\onslide<+->
%因此
%\begin{align*}
%\abs{\int_{C_1}f(z)\diff z}&\le\int_0^R \abs{\exp\bigl[i(R+yi)^2\bigr]}\diff y\\
%&\visible<+->{=\int_0^R\exp(-2Ry)\diff y}\\
%&\visible<+->{=\frac{1-e^{-2R^2}}{2R}\to0 \quad(R\to+\infty).}
%\end{align*}
%\end{solutionc}
%\end{frame}
%
%
%\begin{frame}{例题: 留数在定积分上的应用}
%\begin{solutionc}
%由于 $C_2:z=\dfrac{1+i}{\sqrt 2}r,0\le r\le\sqrt2R,f(z)=\exp(-r^2)$,
%\onslide<+->
%因此
%\[\int_{C_2}f(z)\diff z=-\frac{1+i}{\sqrt 2}\int_0^{\sqrt 2R}\exp(-r^2)\diff r.\]
%\onslide<+->
%令 $R\to+\infty$ 可得
%\[I_1+iI_2=\frac{1+i}{\sqrt 2}\int_0^{+\infty}\exp(-r^2)\diff r=\frac{1+i}{\sqrt2}\cdot\frac{\sqrt\pi}2.\]
%\onslide<+->
%故
%\[I_1=I_2=\frac{\sqrt{2\pi}}4.\]
%\end{solutionc}
%\end{frame}


\begin{frame}{例题: 留数在定积分上的应用}
最后我们再来看一个例子.
\beqskip{2pt}
\begin{example}
求积分 $I=\displaystyle\int_0^{+\infty}\frac{x^p}{x(x+1)}\diff x,0<p<1$.
\end{example}
\begin{solution}
\indent
\[I=\int_0^{+\infty}\frac{x^p}{x(x+1)}\diff x\eqtxt{令 $x=e^t$}\int_{-\infty}^{+\infty}\frac{e^{pt}}{e^t+1}\diff t.\]
\onslide<+->
考虑 $f(z)=\dfrac{e^{pz}}{e^z+1}$ 在如下闭路 $C$ 上的积分.
\vspace{-\baselineskip}
\begin{center}
\begin{tikzpicture}
\draw[cstaxis] (-2,0)--(2,0);
\draw[cstaxis] (0,0)--(0,1.3);
\draw[cstcurve,dcolorb] (-1.5,0) rectangle (1.5,1);
\draw[cstcurve,dcolora] (-1.5,0)--(-1.5,1);
\draw[cstcurve,dcolora] (1.5,0)--(1.5,1);
\draw[cstcurve,dcolorb,cstarrowto] (-1.5,0)--(-0.5,0);
\draw[cstcurve,dcolorb,cstarrowto] (0,1)--(-0.5,1);
\draw[cstcurve,dcolora,cstarrowto] (1.5,0)--(1.5,0.7);
\draw[cstcurve,dcolora,cstarrowto] (-1.5,1)--(-1.5,0.3);
\draw
  (1.8,0.3) node {$R$}
  (-1.9,0.3) node {$-R$}
  (0.4,0.7) node {$2\pi i$}
  (1.2,0.5) node[dcolora] {$C_1$}
  (-1.1,0.5) node[dcolora] {$C_2$}
  (-0.6,0.8) node[dcolorb] {$l$};
\end{tikzpicture}
\end{center}
\vspace{-0.5\baselineskip}
\end{solution}
\endgroup
\end{frame}


\begin{frame}{例题: 留数在定积分上的应用}
\begin{solutionc}
\indent
由于 $l:z=t+2\pi i,-R\le t\le R$,
\onslide<+->
因此
\[\int_l f(z)\diff z
=\int_R^{-R}\frac{e^{2p\pi i}\cdot e^{pt}}{e^t+1}\diff t
=-e^{2p\pi i}\int_{-R}^Rf(t)\diff t.\]

\onslide<+->
由于 $C_1:z=R+it,0\le t\le 2\pi$,
\onslide<+->
因此
\[\abs{\int_{C_1}f(z)\diff z}\le \frac{e^{(p+1)R}}{e^R-1}\cdot 2\pi\to 0\quad(R\to+\infty).\]
\onslide<+->
同理
\[\abs{\int_{C_2}f(z)\diff z}\le \frac{e^{-(p+1)R}}{1-e^{-R}}\cdot 2\pi\to 0\quad(R\to+\infty).\]
\end{solutionc}
\end{frame}


\begin{frame}{例题: 留数在定积分上的应用}
\begin{solutionc}
\indent
由于
\[\Res[f(z),\pi i]
=\frac{e^{pz}}{(e^z+1)'}\bigg|_{z=\pi i}=-e^{p\pi i},\]
\onslide<+->
因此
\begin{align*}
&\left(\int_{-R}^R+\int_l+\int_{C_1}+\int_{C_2}\right)f(z)\diff z\\
=&\oint_Cf(z)\diff z=2\pi i\Res[f(z),\pi i]=-2\pi ie^{p\pi i},
\end{align*}
\onslide<+->
令 $R\to+\infty$,
\onslide<+->
则
\begin{align*}
&(1-e^{2p\pi i})I=-2\pi ie^{p\pi i},\quad
\visible<+->{I=\frac{2\pi i}{e^{p\pi i}-e^{-p\pi i}}=\frac{\pi}{\sin p\pi}.}
\end{align*}
\vspace{-\baselineskip}
\end{solutionc}
\end{frame}

