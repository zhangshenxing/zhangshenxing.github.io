\section{曲线和区域}


\begin{frame}{典型例题: 复数方程表平面图形}
\onslide<+->
\onslide<+->
很多的平面图形能用复数形式的方程来表示, 这种表示方程有些时候会显得更加直观和易于理解.
\onslide<+->
由于 $x=\dfrac{z+\ov z}{2i},y=\dfrac{z+\ov z}2$, 因此很容易将 $x,y$ 的方程和 $z$ 的方程相互转化.

\begin{example}
$|z+i|=2$.
\onslide<+->
该方程表示与 $-i$ 的距离为 $2$ 的点全体,即圆心为 $-i$ 半径为 $2$ 的圆.
\onslide<+->
设 $z=x+yi$, 则方程可以化为 $x^2+(y+1)^2=4$.
\vspace{-8pt}
\begin{center}
\begin{tikzpicture}
\draw[cstaxis] (-1.5,0)--(1.5,0);
\draw[cstaxis] (0,-1.8)--(0,1);
\draw[cstcurve,alecolor] (0,-0.5) circle(1);
\fill[cstdot,notcolor] (0,-0.5) circle;
\draw
  (-0.4,-0.5) node[notcolor] {$-i$};
\end{tikzpicture}
\end{center}
\vspace{-8pt}
\onslide<+->
一般的圆方程为 $|z-z_0|=R$, 其中 $z_0$ 是圆心, $R$ 是半径.
\end{example}
\end{frame}


\begin{frame}{典型例题: 复数方程表平面图形}
\begin{example}
$|z-2i|=|z+2|$.
\onslide<+->
该方程表示与 $2i$ 和 $-2$ 的距离相等的点, 即二者连线的垂直平分线.
\onslide<+->
两边同时平方化简可得 $z+i\ov z=0$ 或 $x+y=0$.

\begin{center}
\begin{tikzpicture}
\draw[cstaxis] (-1.5,0)--(1.5,0);
\draw[cstaxis] (0,-1.5)--(0,1.5);
\draw[cstcurve,alecolor] (-1.2,1.2)--(1.2,-1.2);
\fill[cstdot,notcolor] (0,1) circle;
\fill[cstdot,notcolor] (-1,0) circle;
\draw
  (-1,-0.3) node[notcolor] {$-2$}
  (0.4,1) node[notcolor] {$2i$};
\end{tikzpicture}
\end{center}
\end{example}
\end{frame}


\begin{frame}{典型例题: 复数方程表平面图形}
\begin{example}
\begin{itemize}
\item (3) $\Im(i+\ov z)=4$.
\onslide<+->
设 $z=x+yi$, 则 $\Im(i+\ov z)=1-y=4$, 因此 $y=-3$.
\item (4) $|z-z_1|+|z-z_2|=2a$.
\onslide<+->
该方程表示以 $z_1,z_2$ 为焦点, $a$ 为长半轴的椭圆.
\item (5) $|z-z_1|-|z-z_2|=2a$.
\onslide<+->
该方程表示以 $z_1,z_2$ 为焦点, $a$ 为实半轴的双曲线的一支.
\end{itemize}
\end{example}

\begin{exercise}
$z^2+\ov z^2=1$ 和 $z^2-\ov z^2=i$ 表示什么图形?
\end{exercise}
\begin{answer}
双曲线 $x^2-y^2=\dfrac12$ 和双曲线 $xy=\dfrac14$.
\end{answer}
\end{frame}


\begin{frame}{邻域}
\onslide<+->
在高等数学中, 为了引入极限的概念, 需要考虑点的邻域.
\onslide<+->
类似地, 在复变函数中, 称开圆盘
\[U(z_0,\delta)=\set{z:|z-z_0|<\delta}\]
为 $z_0$ 的一个 \markdef{$\delta$-邻域},
\onslide<+->
称去心开圆盘
\[\stackrel{\circ}U(z_0,\delta)={z:0<|z-z_0|<\delta}\]
为 $z_0$ 的一个\markdef{去心 $\delta$-邻域}.

\onslide<2->
\begin{center}
\begin{tikzpicture}
\filldraw[cstcurve,alecolor,cstfill] (0,0) circle(1);
\draw[cstcurve,cstarrowto,notcolor] (0,0)--(0.8,0.6);
\fill[cstdot,notcolor] (0,0) circle;
\filldraw[cstcurve,alecolor,cstfill,visible on=<3->] (3,0) circle(1);
\draw[cstcurve,visible on=<3->,cstarrowto,notcolor] (3,0)--(3.8,0.6);
\filldraw[cstdote,visible on=<3->] (3,0) circle;
\draw
  (-0.3,0) node[alecolor,notcolor] {$z_0$}
  (2.7,0) node[alecolor,visible on=<3->,notcolor] {$z_0$}
  (0.5,0.1) node[alecolor,notcolor] {$\delta$}
  (3.5,0.1) node[alecolor,visible on=<3->,notcolor] {$\delta$};
\end{tikzpicture}
\end{center}
\end{frame}

\begin{frame}{内部、外部、边界}
\onslide<+->
设 $G$ 是复平面的一个子集, $z_0\in\BC$.
\onslide<+->
它们的位置关系有三种可能:
\begin{enumerate}
\item 如果存在 $z_0$ 的一个邻域 $U$ 完全包含在 $G$ 中, 则称 $z_0$ 是 $G$ 的一个\markdef{内点}.
\item 如果存在 $z_0$ 的一个邻域 $U$ 完全不包含在 $G$ 中, 则称 $z_0$ 是 $G$ 的一个\markdef{外点}.
\item 如果 $z_0$ 的任何一个邻域 $U$, 都有属于和不属于 $G$ 的点, 则称 $z_0$ 是 $G$ 的一个\markdef{边界点}.
\end{enumerate}
\onslide<+->
显然内点都属于 $G$, 外点都不属于 $G$, 而边界点则都有可能.
\onslide<+->
这类比于区间的端点和区间的关系.

\onslide<1->
\begin{center}
\begin{tikzpicture}
\filldraw[cstcurve,alecolor,cstfill,smooth] plot coordinates {(0,-0.8) (1,-0.8) (1.5,0) (1,0.6) (0,0.8) (-1,0.7) (-1.2,0) (-1,-0.6) (0,-0.8)};
\draw[visible on=<3->,cstcurve,notcolor] (-0.5,0) circle (0.5);
\draw[visible on=<5->,cstcurve,defcolor] (1.43,0) circle (0.5);
\draw[visible on=<4->,cstcurve,black] (3,0) circle (0.5);
\fill[visible on=<3->,cstdot,notcolor] (-0.5,0) circle;
\fill[visible on=<5->,cstdot,defcolor] (1.5,0) circle;
\fill[visible on=<4->,cstdot,black] (3,0) circle;
\draw
  (0.5,0) node[alecolor] {$G$}
  (-0.5,0.2) node[visible on=<3->,notcolor] {$z_0$}
  (1.75,0) node[visible on=<5->,defcolor] {$z_0$}
  (3.3,0) node[visible on=<4->] {$z_0$};
\end{tikzpicture}
\end{center}
\end{frame}


\begin{frame}{开集和闭集}\onslide<+->
如果 $G$ 的所有点都是内点, 也就是说, $G$ 的边界点都不属于它, 称 $G$ 是一个\markdef{开集}.
\onslide<+->
例如
\[|z-z_0|<R,\quad 1<\Re z<3,\quad\frac\pi4<\arg z<\dfrac{3\pi}4\] 都是开集.
\onslide<+->
如果 $G$ 的所有边界点都属于 $G$, 称 $G$ 是一个\markdef{闭集}.
\onslide<+->
这等价于它的补集是开集.

\onslide<+->
直观上看: 开集往往由 $>,<$ 的不等式给出, 闭集往往由 $\ge,\le$ 的不等式给出.
\onslide<+->
不过注意这并不是绝对的.

\onslide<+->
如果 $D$ 可以被包含在某个开圆盘 $U(0,\delta)$ 中, 则称它是\markdef{有界}的.
\onslide<+->
否则称它是\markdef{无界}的.
\end{frame}


\begin{frame}{区域和闭区域}
\begin{definition}
如果开集 $D$ 的任意两个点之间都可以用一条完全包含在 $D$ 中的折线连接起来, 则称 $D$ 是一个\markdef{区域}.
\onslide<+->
也就是说, 区域是连通的开集.
\end{definition}
\onslide<+->
观察下侧的图案, 淡蓝色部分是一个区域.
\onslide<+->
红色的线条和点是它的边界.
\onslide<+->
区域和它的边界一起构成了\markdef{闭区域}, 记作 $\ov D$.
\onslide<+->
它是一个闭集.

\onslide<3->
\begin{center}
\begin{tikzpicture}
\filldraw[cstcurve,alecolor,cstfill,smooth] plot coordinates {(0,-1.2) (1.5,-1.2) (2.3,0) (1.5,0.9) (0,1.2) (-1.5,1) (-1.8,0) (-1.5,-0.9) (0,-1.2)};
\filldraw[cstcurve,alecolor,fill=white,smooth] plot coordinates {(-1.4,0) (-1,0.6) (-0.6,0) (-1.2,-0.5) (-1.4,0)};
\filldraw[cstcurve,alecolor,fill=white] (0.5,0.3) circle (0.3);
\fill[cstdot,alecolor] (1.5,0) circle;
\fill[cstdot,alecolor] (1.6,-0.5) circle;
\draw[cstcurve,alecolor,smooth] plot coordinates {(1,0.5) (1.2,0.3) (1.2,-0.3) (1.4,0.5)};
\draw[cstcurve,defcolor] (-1,0.8)--(-0.2,0.5)--(0.2,-0.5)--(1,-0.8);
\draw
  (-1.2,0.8) node[defcolor] {$z_1$}
  (1.3,-0.8) node[defcolor] {$z_2$};
\end{tikzpicture}
\end{center}
\end{frame}


\begin{frame}{常见区域}
\onslide<+->
复平面上的区域大多由复数的实部、虚部、模和辐角的不等式所确定.
\onslide<10->{这些区域对应的闭区域是什么?}

\onslide<2->
\begin{center}
\begin{tikzpicture}
\draw[cstaxis](-1.5,0.3)--(1.5,0.3);
\draw[cstaxis](0,-0.6)--(0,1.4);
\fill[cstfille,pattern color=notcolor] (-1.2,0.3) rectangle (1.2,1.1);
\fill[cstfille,pattern color=alecolor,visible on=<3->] (-1.2,0.3) rectangle (1.2,-0.5);
\draw[cstaxis,visible on=<4->](2.5,0)--(5.5,0);
\draw[cstaxis,visible on=<4->](4,-1.1)--(4,1.3);
\fill[cstfille,pattern color=notcolor,visible on=<4->] (2.8,-1) rectangle (4,1);
\fill[cstfille,pattern color=alecolor,visible on=<5->] (4,1) rectangle (5.2,-1);
\draw[cstaxis,visible on=<6->](6.5,0)--(8.5,0);
\draw[cstaxis,visible on=<6->](7,-1)--(7,1.2);
\fill[cstfille,pattern color=notcolor,visible on=<6->] (7.2,-0.9) rectangle (8,0.9);
\draw
  (-0.05,1.5) node[notcolor] {上半平面 $\Im z>0$}
  (-0.05,-0.9) node[alecolor,visible on=<3->] {下半平面 $\Im z<0$}
  (3.1,1.6) node[notcolor,align=center,visible on=<4->] {左半平面\\$\Re z<0$}
  (4.9,1.6) node[alecolor,align=center,visible on=<5->] {右半平面\\$\Re z>0$}
  (7.7,1.7) node[notcolor,align=center,visible on=<6->] {竖直带状区域\\$x_1<\Re z<x_2$};
\end{tikzpicture}
\end{center}
\onslide<7->
\begin{center}
\begin{tikzpicture}
\draw[cstaxis](-1.5,1.2)--(1.5,1.2);
\draw[cstaxis](0,1)--(0,2.2);
\fill[cstfille,pattern color=alecolor] (-1.2,1.3) rectangle (1.2,2);
\draw[cstaxis,visible on=<8->](2,0)--(4,0);
\draw[cstaxis,visible on=<8->](2.5,-0.2)--(2.5,1.3);
\fill[cstfille,pattern color=notcolor,visible on=<8->] (2.5,0)--(3.25,1.299) arc(60:10:1.5)--cycle;
\filldraw[cstcurve,alecolor,cstfill,visible on=<9->] (6.2,1) circle (1);
\filldraw[cstcurve,alecolor,cstfill,fill=white,visible on=<9->] (6.2,1) circle (0.5);
\draw[cstaxis,visible on=<9->](4.9,1)--(7.5,1);
\draw[cstaxis,visible on=<9->](6.2,-0.2)--(6.2,2.4);
\draw
  (0,0.5) node[alecolor,align=center] {水平带状区域\\$y_1<\Im z<y_2$}
  (3.4,1.8) node[notcolor,align=center,visible on=<8->] {角状区域\\$\alpha_1<\arg z<\alpha_2$}
  (8.2,2) node[alecolor,align=center,visible on=<9->] {圆环域\\$r<|z|<R$};
\end{tikzpicture}
\end{center}
\end{frame}


\begin{frame}{连续区间、简单曲线和闭路}
\onslide<+->
设 $x(t),y(t),t\in[a,b]$ 是两个连续函数,
\onslide<+->
则参变量方程
$\begin{cases}x=x(t),&\\y=y(t),&\end{cases}t\in[a,b]$ 定义了一条\markdef{连续曲线}.
\onslide<+->
这也等价于 $C:z=z(t)=x(t)+iy(t),t\in[a,b]$.

\onslide<+->
如果除了两个端点有可能重叠外, 其它情形不会出现重叠的点, 则称 $C$ 是\markdef{简单曲线}.
\onslide<+->
如果还满足两个端点重叠, 即 $z(a)=z(b)$, 则称 $C$ 是\markdef{简单闭曲线}, 也简称为\markdef{闭路}.

\onslide<2->
\begin{center}
\begin{tikzpicture}
\draw[cstaxis](-0.2,0)--(9.5,0);
\draw[cstaxis](0,-0.1)--(0,2.5);
\draw[cstcurve,alecolor,smooth] plot coordinates {(0.2,0.9) (1,1.6) (2,0.6) (3,0.9)};
\draw[cstcurve,notcolor,smooth,visible on=<3->] plot coordinates {(4,0.5) (4.6,0.9) (5.3,2) (4.7,2.5) (3.7,2) (4.7,0.9) (5.7,0.5)};
\draw[cstcurve,defcolor,smooth,smooth,visible on=<4->] plot coordinates {(7,1) (7.5,0.4) (8,0.2) (8.5,0.4) (9,1) (8.5,1.6) (8,1.8) (7.5,1.7) (7,1)};
\draw
  (0.5,0.7) node[alecolor] {$z(a)$}
  (3,1.2) node[alecolor] {$z(b)$};
\end{tikzpicture}
\end{center}
\end{frame}


\begin{frame}{闭路的内部和外部}
\onslide<+->
闭路 $C$ 把复平面划分成了两个区域, 一个有界一个无界.
\onslide<+->
分别称这两个区域是 $C$ 的\markdef{内部}和\markdef{外部}.
\onslide<+->
$C$ 是它们的公共边界.

\onslide<+->
这件事情的严格证明是十分困难的.

\begin{center}
\begin{tikzpicture}
\fill[cstfille] (-2,-2) rectangle (1.5,1.5);
\filldraw[cstcurve,alecolor,smooth,cstfill] plot coordinates {(-1.5,0) (-1,-0.5) (0,-1) (0.7,-1) (0.9,0) (0,0.8) (-1,0.8) (-1.5,0)};
\end{tikzpicture}
\end{center}
\end{frame}


\begin{frame}{单连通域和多连通域}
\onslide<+->
在前面所说的几个区域的例子中, 我们在区域中画一条闭路.
\onslide<+->
除了圆环域之外, 闭路的内部仍然包含在这个区域内.

\begin{definition}
如果区域 $D$ 中的任一闭路的内部都包含在 $D$ 中, 则称 $D$ 是\markdef{单连通域}.
\onslide<+->
否则称之为\markdef{多连通域}.
\end{definition}

\onslide<+->
单连通域内的任一闭路可以连续地变形成一个点.
\begin{center}
\begin{tikzpicture}
\filldraw[cstcurve,alecolor,smooth,cstfill] plot coordinates {(-2.2,0) (-1.5,-0.7) (0,-1.5) (1,-1.5) (1.4,0) (0,1.2) (-1.5,1.2) (-2.2,0)};
\draw[cstcurve,defcolor,smooth] plot coordinates {(-1.6,0) (-1.2,-0.5) (-0.3,-1) (0,-0.5) (-0.3,0.2) (-1.2,0.7) (-1.6,0)};
\filldraw[cstcurve,alecolor,smooth,fill=white] plot coordinates {(-1.4,0) (-1,-0.3) (-0.8,0) (-1,0.3) (-1.4,0)};
\filldraw[cstcurve,alecolor,smooth,fill=white] plot coordinates {(-0.8,-0.4) (-0.4,-0.7) (-0.2,-0.4) (-0.4,-0.1) (-0.8,-0.4)};
\draw[cstcurve,alecolor,smooth] plot coordinates {(-0.7,1) (-0.2,0.9) (0.1,0.7)};
\draw[cstcurve,alecolor](-0.4,1.2)--(0.1,0.7);
\fill[cstdot,alecolor] (0.7,0) circle;
\fill[cstdot,alecolor] (1,-0.6) circle;
\end{tikzpicture}
\end{center}
\end{frame}


\begin{frame}{例题: 区域的特性}
\begin{example}
$\Re(z^2)<1$.

\onslide<+->
设 $z=x+yi$, 则 $\Re(z^2)=x^2-y^2<1$.
\onslide<+->
这是无界的单连通域.
\end{example}

\begin{center}
\begin{tikzpicture}
\fill[cstfille] (-1.414,-1) rectangle (1.414,1);
\filldraw[cstcurve,alecolor,domain=-45:45,smooth,fill=white] plot ({sec(\x)},{tan(\x)});
\filldraw[cstcurve,alecolor,domain=-45:45,smooth,fill=white] plot ({-sec(\x)},{tan(\x)});
\draw[cstaxis] (-2.5,0)--(2.5,0);
\draw[cstaxis] (0,-1.5)--(0,1.5);
\end{tikzpicture}
\end{center}
\end{frame}


\begin{frame}{例题: 区域的特性}
\begin{example}
$|\arg z|<\dfrac\pi3$.

\onslide<+->
即角状区域 $-\dfrac\pi3<\arg z<\dfrac\pi3$.
\onslide<+->
这是无界的单连通域.
\end{example}

\onslide<+->
\begin{center}
\begin{tikzpicture}
\fill[cstfille] (0,0)--(1,1.732) arc(60:-60:2)--cycle;
\draw[cstcurve,alecolor] (0,0)--(1.2,2.078);
\draw[cstcurve,alecolor] (0,0)--(1.2,-2.078);
\draw[cstaxis] (-0.5,0)--(2.5,0);
\draw[cstaxis] (0,-2.5)--(0,2.5);
\end{tikzpicture}
\end{center}
\end{frame}


\begin{frame}{例题: 区域的特性}
\begin{example}
$\abs{\dfrac1z}<3$.

\onslide<+->
即 $|z|>\dfrac13$.
\onslide<+->
这是无界的单连通域.
\end{example}
\onslide<+->
\begin{center}
\begin{tikzpicture}
\fill[cstfille] (-1.6,-1.6) rectangle (1.6,1.6);
\filldraw[cstcurve,alecolor,fill=white] (0,0) circle (0.5);
\draw[cstaxis] (-2,0)--(2,0);
\draw[cstaxis] (0,-2)--(0,2);
\end{tikzpicture}
\end{center}
\end{frame}


\begin{frame}{例题: 区域的特性}
\begin{example}
$|z+1|+|z-1|<4$.

\onslide<+->
表示一个椭圆的内部. 
\onslide<+->
这是有界的单连通域.
\end{example}
\vspace{-5pt}
\onslide<+->
\begin{center}
\begin{tikzpicture}
\filldraw[cstcurve,alecolor,cstfill] (0,0) circle (1 and {0.5*sqrt(3)});
\draw[cstaxis] (-1.5,0)--(1.5,0);
\draw[cstaxis] (0,-1.1)--(0,1.1);
\end{tikzpicture}
\end{center}
\vspace{-5pt}

\begin{think}
$|z+1|+|z-1|\ge 1$ 表示什么区域?
\end{think}
\vspace{-5pt}
\begin{answer}
整个复平面.
\end{answer}
\end{frame}



