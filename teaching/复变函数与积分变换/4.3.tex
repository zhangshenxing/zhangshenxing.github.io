\section{泰勒级数}


\begin{frame}{泰勒级数}
\onslide<+->
上一节中我们已经知道, 幂级数在它的收敛域内的和函数是一个解析函数.
\onslide<+->
反过来, 解析函数是不是也一定可以在一点展开成幂级数呢? 也就是说是否存在\emph{泰勒级数}展开?

\onslide<+->
在实变函数中我们知道, 一个函数即使在一点附近无限次可导, 它的泰勒级数也未必收敛到原函数.
\onslide<+->
例如
\[f(x)=\begin{cases}
e^{-x^{-2}},&x\neq 0,\\
0,&x=0.\end{cases}\]
\onslide<+->
它处处可导, 但是它在 $0$ 处的各阶导数都是 $0$.
\onslide<+->
因此它的泰勒级数是 $0$, 余项恒为 $f(x)$.
\onslide<+->
所以它的麦克劳林级数除 $0$ 外均不收敛到原函数.
\end{frame}


\begin{frame}{泰勒级数}
\onslide<+->
而即使是泰勒级数能收敛到原函数的情形, 它成立的区间也很难从函数本身读出.
\onslide<+->
例如
\[\dfrac1{1+x}=1-x+x^2-x^3+\cdots,\quad|x|<1.\]
\onslide<+->
这可以从 $x=-1$ 是奇点看出.
\onslide<+->
而
\[\dfrac1{1+x^2}=1-x^2+x^4-x^6+\cdots,\quad|x|<1\]
却并没有奇点.
\onslide<+->
为什么它的麦克劳林级数成立的开区间也是 $(-1,1)$?
\onslide<+->
这个问题在本节可以得到回答.
\end{frame}


\begin{frame}{泰勒展开}
\onslide<+->
设函数 $f(z)$ 在区域 $D$ 解析, $z_0\in D$.
\onslide<+->
设 $|z-z_0|$ 小于 $z_0$ 到 $D$ 边界的距离 $d$,
则存在 $|z-z_0|<r<d$.
\onslide<+->
设 $K:|\zeta-z_0|=r$, 则 $K$ 和它的内部包含在 $D$ 中.
\onslide<+->
由于 $\abs{\dfrac{z-z_0}{\zeta-z_0}}<1$, 因此
\[\frac1{\zeta-z}=\frac1{\zeta-z_0}\cdot\frac1{1-\dfrac{z-z_0}{\zeta-z_0}}=\sum_{n=0}^\infty\frac{(z-z_0)^n}{(\zeta-z_0)^{n+1}}.\]

\onslide<1->
\begin{center}
\begin{tikzpicture}
\filldraw[cstcurve,dcolora,smooth,cstfill] plot coordinates {(-2.25,0) (-1.5,-0.75) (0,-1.5) (1.05,-1.5) (1.35,0) (0,1.2) (-1.5,1.2) (-2.25,0)};
\draw[cstdash,dcolorc,visible on=<2->] (0,0) circle (1);
\draw[cstcurve,dcolorb,visible on=<3->] (0,0) circle (0.8);
\fill[cstdot,dcolora] (0,0) circle;
\fill[cstdot,dcolorc,visible on=<2->] (0.4,0.4) circle;
\fill[cstdot,visible on=<3->] (0.6,-0.5) circle;
\draw[cstcurve,cstarrowto,dcolora,visible on=<3->] (0,0)--(-0.48,-0.64);
\draw
  (-0.25,0) node[dcolora] {$z_0$}
  (0.2,0.4) node[dcolorc,visible on=<2->] {$z$}
  (0,-0.5) node[dcolora,visible on=<3->] {$r$}
  (0.4,-0.3) node[visible on=<3->] {$\zeta$};
\end{tikzpicture}
\end{center}
\end{frame}


\begin{frame}{泰勒展开}
\onslide<+->
故
\begin{align*}
f(z)&=\frac1{2\pi i}\oint_K \frac{f(\zeta)}{\zeta-z}\diff\zeta
\visible<+->{=\frac1{2\pi i}\oint_K f(\zeta)\sum_{n=0}^\infty\frac{(z-z_0)^n}{(\zeta-z_0)^{n+1}}\diff\zeta}\\
&\visible<+->{=
\sum_{n=0}^{N-1}\left[\frac1{2\pi i}\oint_K\frac{f(\zeta)\diff\zeta}{(\zeta-z_0)^{n+1}}\right](z-z_0)^n+R_N(z),}\\
&\visible<+->{=
\sum_{n=0}^{N-1}\frac{f^{(n)}(z_0)}{n!}(z-z_0)^n+R_N(z),}
\end{align*}
\onslide<.->
其中
\[R_N(z)=\frac1{2\pi i}\oint_Kf(\zeta)\left[\sum_{n=N}^\infty\frac{(z-z_0)^n}{(\zeta-z_0)^{n+1}}\right]\diff\zeta.\]
\end{frame}


\begin{frame}{泰勒展开}
\onslide<+->
由于 $f(\zeta)$ 在 $D\supseteq K$ 上解析, 从而在 $K$ 上连续且有界.
\onslide<+->
设 $|f(\zeta)|\le M,\zeta\in K$,
\onslide<+->
那么
\begin{align*}
|R_N(z)|&\le\frac M{2\pi}\oint_K\abs{\sum_{n=N}^\infty\frac{(z-z_0)^n}{(\zeta-z_0)^{n+1}}}\diff s\\
&\visible<+->{\le\frac M{2\pi}\oint_K\sum_{n=N}^\infty\abs{\frac1{\zeta-z}\cdot\left(\frac{z-z_0}{\zeta-z_0}\right)^N}\diff s}\\
&\visible<+->{\le\frac M{2\pi}\cdot\frac1{r-|z-z_0|}\cdot\abs{\frac{z-z_0}{\zeta-z_0}}^N\cdot 2\pi r\to 0\quad (N\to\infty).}
\end{align*}
\onslide<+->
故
\[\alert{f(z)=\sum_{n=0}^\infty\frac{f^{(n)}(z_0)}{n!}(z-z_0)^n},\quad
|z-z_0|<d.\]
\end{frame}


\begin{frame}{泰勒展开的成立范围}
\onslide<+->
由于幂级数在收敛半径内的和函数是解析的, 因此解析函数的泰勒展开成立的圆域不包含奇点.
\onslide<+->
由此可知, 解析函数在 $z_0$ 处\alert{泰勒展开成立的圆域的最大半径是 $z_0$ 到最近奇点的距离}.

\onslide<+->
需要注意的是, 泰勒级数的收敛半径是有可能比这个半径更大的.
\onslide<+->
而且泰勒展开等式也可能在这个圆域之外的点成立.
\onslide<+->
例如 
\[f(z)=\begin{cases}
e^z,&z\neq 1;\\ 0,&z=1
\end{cases}\]
的麦克劳林展开为
\[f(z)=\sum_{n=0}^\infty \frac{z^n}{n!},\quad |z|<1.\]
\end{frame}


\begin{frame}{幂级数展开的唯一性}
\onslide<+->
现在我们来看 $f(z)=\dfrac1{1+z^2}$.
\onslide<+->
它的奇点为 $\pm i$, 所以它的麦克劳林展开成立的半径是 $1$.
\onslide<+->
这就解释了为什么函数 $f(x)=\dfrac1{1+x^2}$ 的麦克劳林展开成立的开区间是 $(-1,1)$.

\onslide<+->
若 $f(z)$ 在 $z_0$ 附近展开为 $\suml_{n=0}^\infty c_n(z-z_0)^n$,
\onslide<+->
则由幂级数的逐项求导性质可知
\[f^{(n)}(z_0)=\sum_{k=n}^\infty \frac{k!c_k}{(k-n)!}(z-z_0)^{k-n}\Big|_{z=z_0}=n!c_n.\]
\onslide<+->
所以\alert{解析函数的幂级数展开是唯一的}.
\onslide<+->
因此解析函数的泰勒展开不仅可以\emph{直接求出各阶导数得到}, 也可以\emph{利用幂级数的运算法则得到}.
\end{frame}


\begin{frame}{典型例题: 泰勒展开的计算}
\beqskip{4pt}
\begin{example}
由于 $(e^z)^{(n)}(0)=e^z|_{z=0}=1$,
\onslide<+->
因此
\[\emph{e^z=1+z+\frac{z^2}{2!}+\frac{z^3}{3!}+\cdots=\sum_{n=0}^\infty\frac{z^n}{n!},\quad\forall z.}\]
\end{example}

\begin{example}
由于
\[(\cos z)^{(n)}=\cos\left(z+\dfrac{n\pi}2\right),\]
\vspace{-0.8\baselineskip}
\onslide<+->
\[(\cos z)^{(2n+1)}(0)=0,\quad (\cos z)^{(2n)}(0)=(-1)^n,\]
\onslide<+->
因此
\[\emph{\cos z=1-\frac{z^2}{2!}+\frac{z^4}{4!}-\frac{z^6}{6!}+\cdots=\sum_{n=0}^\infty(-1)^n\frac{z^{2n}}{(2n)!},\quad\forall z.}\]
\end{example}
\endgroup
\end{frame}


\begin{frame}{典型例题: 泰勒展开的计算}
\begin{example}
由 $e^z$ 的泰勒展开可得
\begin{align*}
\sin z&=\frac{e^{iz}-e^{-iz}}{2i}
\visible<+->{=\sum_{n=0}^\infty\frac{(iz)^n-(-iz)^n}{2i\cdot n!}}\\
&\visible<+->{\emph{=z-\frac{z^3}{3!}+\frac{z^5}{5!}-\cdots}}
\visible<+->{\emph{=\sum_{n=0}^\infty(-1)^n\frac{z^{2n+1}}{(2n+1)!},\quad\forall z.}}
\end{align*}
\end{example}
\end{frame}


\begin{frame}{典型例题: 泰勒展开的计算}
\begin{example}
函数 $f(z)=(1+z)^\alpha$ 的主值为 $\exp\bigl[\alpha\ln(1+z)\bigr]$.
\onslide<+->
它在去掉射线 $z=x\le -1$ 的区域内解析.
\onslide<+->
由于
\begin{align*}
f^{(n)}(0)&=\alpha(\alpha-1)\cdots(\alpha-n+1)\exp\bigl[(\alpha-n)\ln(1+z)\bigr]\Big|_{z=0}\\
&\visible<+->{=\alpha(\alpha-1)\cdots(\alpha-n+1).}
\end{align*}
\onslide<+->
因此
\begin{align*}
\emph{(1+z)^\alpha}&\emph{=1+\alpha z+\frac{\alpha(\alpha-1)}2z^2+\frac{\alpha(\alpha-1)(\alpha-2)}{3!}z^3+\cdots}\\
&\emph{=\sum_{n=0}^\infty\frac{\alpha(\alpha-1)\cdots(\alpha-n+1)}{n!}z^n,\quad |z|<1.}
\end{align*}
\vspace{-\baselineskip}
\end{example}
\end{frame}


\begin{frame}{典型例题: 泰勒展开的计算}
\beqskip{0pt}
\begin{example}
将 $\dfrac1{(1+z)^2}$ 展开成 $z$ 的幂级数.
\end{example}
\begin{solution}
由于 $\dfrac1{(1+z)^2}$ 的奇点为 $z=-1$, 因此它在 $|z|<1$ 内解析.
\onslide<+->
由于
\[\frac1{1+z}=1-z+z^2-z^3+\cdots=\sum_{n=0}^\infty (-1)^nz^n,\]
\onslide<+->
因此
\begin{align*}
\frac1{(1+z)^2}&=-\left(\frac1{1+z}\right)'\visible<+->{=-\sum_{n=1}^\infty(-1)^n nz^{n-1}}\\
&\visible<+->{=\sum_{n=0}^\infty(-1)^n (n+1)z^n,\quad |z|<1.}
\end{align*}
\end{solution}
\endgroup
\end{frame}


\begin{frame}{典型例题: 泰勒展开的计算}
\beqskip{3pt}
\begin{example}
将对数函数的主值 $\ln(1+z)$ 展开成 $z$ 的幂级数.
\end{example}
\begin{solution}
由于 $\ln(1+z)$ 在去掉射线 $z=x\le-1$ 的区域内解析,
\onslide<+->
因此它在 $|z|<1$ 内解析.
\onslide<+->
此时
\[[\ln(1+z)]'=\frac1{1+z}=\sum_{n=0}^\infty(-1)^nz^n,\quad|z|<1\]
\onslide<+->
逐项积分得到
\begin{align*}
\ln(1+z)&=\int_0^z\frac1{1+\zeta}\diff\zeta
	\visible<+->{=\int_0^z\sum_{n=0}^\infty(-1)^n\zeta^n\diff\zeta}\\
&\visible<+->{=\sum_{n=0}^\infty\frac{(-1)^nz^{n+1}}{n+1}}
	\visible<+->{=\sum_{n=1}^\infty\frac{(-1)^{n+1}z^n}{n},\quad|z|<1.}
\end{align*}
\end{solution}
\endgroup
\end{frame}


\begin{frame}{典型例题: 泰勒展开的计算}
\begin{example}
将 $\dfrac1{3z-2}$ 展开成 $z$ 的幂级数.
\end{example}
\begin{solution}
由于 $\dfrac1{3z-2}$ 的奇点为 $z=\dfrac23$, 因此它在 $|z|<\dfrac23$ 内解析.
\onslide<+->
此时
\begin{align*}
\frac1{3z-2}&=-\frac12\cdot\frac1{1-\dfrac{3z}2}
	\visible<+->{=-\frac12\sum_{n=0}^\infty\left(\frac{3z}2\right)^n}\\
&\visible<+->{=-\sum_{n=0}^\infty\frac{3^n}{2^{n+1}}z^n,\quad|z|<\frac23.}
\end{align*}
\end{solution}
\end{frame}


\begin{frame}{典型例题: 泰勒展开的计算}
\begin{example}
将 $\dfrac{e^z}{1+z}$ 展开成 $z$ 的幂级数.
\end{example}
\begin{solution}
由于 $\dfrac{e^z}{1+z}$ 的奇点为 $-1$, 因此它在 $|z|<1$ 内解析.
\onslide<+->
此时
\[e^z=\sum_{n=0}^\infty\frac1{n!}z^n,\quad
\frac1{1+z}=\sum_{n=0}^\infty(-1)^nz^n,\]
\onslide<+->
\[\dfrac{e^z}{1+z}=\sum_{n=0}^\infty\left[\sum_{k=0}^n\frac{(-1)^{n-k}}{k!}\right]z^n
=1+\frac12z^2-\frac13z^3+\cdots,\quad|z|<1.\]
\end{solution}
\end{frame}


\begin{frame}{典型例题: 泰勒展开的计算}
\begin{exercise}
将 $\cos^2z$ 展开成 $z$ 的幂级数.
\end{exercise}
\vspace{-0.3\baselineskip}
\begin{answer}
\vspace{-\baselineskip}
\begin{align*}
\cos^2z&=\frac12(1+\cos{2z})=\frac12\left[1+\sum_{n=0}^\infty(-1)^n\frac{(2z)^{2n}}{(2n)!}\right]\\
&=1+\sum_{n=1}^\infty(-1)^n\frac{2^{2n-1}}{(2n)!}z^{2n},\quad\forall z.
\end{align*}
\end{answer}
\end{frame}


\begin{frame}{特殊函数的麦克劳林展开}
\begin{thinking}
奇函数和偶函数的麦克劳林展开有什么特点?
\end{thinking}
\begin{answer}
奇函数(偶函数)的麦克劳林展开只有奇数次项(偶数次项).

\onslide<+->
如果解析函数 $f(z)$ 满足 $f(\zeta z)=\zeta^k f(z)$, 其中 $\zeta=\exp\dfrac{2\pi i}m$ 是 $m$ 次单位根,
\onslide<+->
则两边同时对 $z$ 求导得到 $\zeta f'(\zeta z)=\zeta^k f'(z)$,
\onslide<+->
归纳可知
\[f^{(n)}(\zeta z)=\zeta^{k-n}f^{(n)}(z),
\quad\visible<+->{f^{(n)}(0)=\zeta^{k-n}f^{(n)}(0).}\]
\onslide<+->
因此当 $n-k$ 不是 $m$ 的倍数时, $f^{(n)}(0)=0$.
\onslide<+->
故 $f(z)$ 的麦克劳林展开只有 $ml+k$ 次项, $l\in\BZ$.
\end{answer}
\end{frame}


% {
% \homework
% \begin{frame}[<*>]{作业}
%   \begin{homeworks}
%     \item(2021年A卷) 如果 $f(z)=\dfrac{e^{1/z}}{z+1}$ 的泰勒级数为 $\displaystyle\sum_{n=0}^\infty c_n(z-i)^n$, 则该级数的收敛半径为\fillblank{}.
%   \end{homeworks}
% \end{frame}
% }
