\section{复变函数}

\subsection{复变函数的定义}
\begin{frame}{复变函数的定义}
\onslide<+->
\emph{复变函数}就是复数集合 $G\subseteq C$ 上的一个映射 $f:G\to \BC$.
\onslide<+->
换言之, 对于每一个 $z\in G$, 有一个唯一确定的复数 $w=f(z)$ 与之对应.
\onslide<+->
例如 $\Re z,\Im z,\arg z,|z|,z^n$ 都是复变函数.

\onslide<+->
$f$ 的\emph{定义域}是指 $G$, \emph{值域}是指 $\set{w=f(z):z\in G}$.

\onslide<+->
如果 $z_1\neq z_2\implies f(z_1)\neq f(z_2)$, 则称 $f$ 是\emph{单射}.
\end{frame}


\begin{frame}{多值复变函数}
\onslide<+->
不过在复变函数中, 我们常常会遇到\emph{多值的复变函数}, 也就是说一个 $z\in G$ 可能有多个 $w$ 与之对应.
\onslide<+->
例如 $\Arg z,\sqrt[n]z$ 等.
\onslide<+->
如果对每一个定义域范围内的 $z$, 选取固定的一个 $f(z)$ 的值, 则我们得到了这个多值函数的一个\emph{单值分支}.

\onslide<+->
在考虑多值的情况下, 复变函数总有反函数:
\onslide<+->
对于任意点 $w\in $, 存在一个或多个 $z\in G$ 使得 $w=f(z)$.
\onslide<+->
这样 $w$ 到 $z$ 的就定义了 $f$ 的反函数 $f^{-1}$.
\onslide<+->
如果 $f$ 和 $f^{-1}$ 都是单值的, 则称 $f$ 是\emph{一一对应}.
\onslide<+->
若无特别声明, \alert{复变函数总是指单值的复变函数}.
\end{frame}


\begin{frame}{与实变函数的关系}
\onslide<+->
\alert{每一个复变函数 $w=f(z)=u+iv$ 等价于给了两个二元实变函数}
\[\alert{u=u(x,y),\qquad v=v(x,y).}\]
\onslide<+->
例如
\[w=z^2=(x^2-y^2)+i\cdot 2xy,\]
\[u(x,y)=x^2-y^2,\quad v(x,y)=2xy.\]
\onslide<+->
其实我们也可以把 $f(z)$ 看成一个二元实变量复值函数.
\end{frame}


\subsection{映照}
\begin{frame}{映照}
\onslide<+->
在实变函数中, 我们常常用函数图像直观来理解和研究函数.
\onslide<+->
在复变函数中, 我们可以用两个复平面($z$ 复平面和 $w$ 复平面)之间的映射(称之为\emph{映照})来表示这种对应关系.
\onslide<+->
\begin{center}
\begin{tikzpicture}
\draw[cstaxis] (-4.5,0)--(-0.5,0);
\draw[cstaxis] (-2.5,-1.5)--(-2.5,1.5);
\draw[cstaxis] (0.5,0)--(4.5,0);
\draw[cstaxis] (2.5,-1.5)--(2.5,1.5);
\draw[cstcurve,dcolora,smooth] plot coordinates {(-4,0) (-4.2,-0.4) (-2.8,-0.9) (-2,-0.7) (-1.6,0) (-1.4,1) (-2.8,1.2) (-3.2,1) (-4,0)};
\draw[cstcurve,smooth,dcolorb] plot coordinates {(1.2,0) (2,-0.5) (2.5,-0.8) (3,-0.5) (3.5,0) (3.8,0.9) (3.3,1.2) (2,0.8) (1.2,0)};
\draw[cstdash,smooth,dcolorc,cstarrowto] plot coordinates {(-2,0.8) (0.4,0.6) (2.8,0.7)};
\draw[cstdash,smooth,dcolorc,cstarrowto] plot coordinates {(-2.3,0.5) (0,-0.2) (2.2,-0.3)};
\draw[cstdash,smooth,dcolorc,cstarrowto] plot coordinates {(-2.8,0.3) (0,-0.6) (2.2,-0.3)};
\fill[cstdot,dcolora] (-2,0.8) circle;
\fill[cstdot,dcolora] (-2.3,0.5) circle;
\fill[cstdot,dcolora] (-2.8,0.3) circle;
\fill[cstdot,dcolorb] (2.8,0.7) circle;
\fill[cstdot,dcolorb] (2.2,-0.3) circle;
\draw
  (-1.5,-1) node[dcolora] {$z$ 复平面}
  (-2.7,-0.2) node {$0$}
  (-0.5,-0.2) node {$x$}
  (-2.7,1.5) node {$y$}
  (3.5,-1) node[dcolorb] {$w$ 复平面}
  (2.3,0.2) node {$0$}
  (4.5,-0.2) node {$u$}
  (2.3,1.5) node {$v$};
\end{tikzpicture}
\end{center}
\end{frame}


\begin{frame}{例题: 映照}
\onslide<+->
\begin{example}
函数 $w=\ov z$.
\onslide<+->{如果把 $z$ 复平面和 $w$ 复平面重叠放置, 则这个映照对应的是关于 $z$ 轴的翻转变换.}
\onslide<+->{它把任一区域映成和它全等的区域.}
\end{example}
\onslide<+->
\begin{example}
函数 $w=az$.
\onslide<+->{设 $a=re^{i\theta}$, 则这个映照对应的是一个旋转映照(逆时针旋转 $\theta$)和一个相似映照(放大为 $r$ 倍)的复合.}
\onslide<+->{它把任一区域映成和它相似的区域.}
\end{example}
\end{frame}


\begin{frame}{例题: 映照}
\onslide<+->
\begin{example}
函数 $w=z^2$.
\onslide<+->{这个映照把 $z$ 的辐角增大一倍, 因此它会把角形区域变换为角形区域, 并将夹角放大一倍.}
\onslide<+->{
\begin{center}
\begin{tikzpicture}
\draw[cstaxis] (-4.5,0)--(-0.5,0);
\draw[cstaxis] (-2.5,-1)--(-2.5,2);
\draw[cstaxis] (0.5,0)--(4.5,0);
\draw[cstaxis] (2.5,-1)--(2.5,2);
\draw[cstdash,smooth,dcolorc,cstarrowto] plot coordinates {(-2.5,1) (-2,0.6) (-1,1.3) (1.5,0)};
\draw[cstdash,smooth,dcolorc,cstarrowto] plot coordinates {(-1.7,1.2) (0,1.7) (1.7,1.92)};
\draw[cstdash,smooth,dcolorc,cstarrowto] plot coordinates {(-3.1,-0.3) (-1,-1.5) (1,-1.5) (2.77,0.36)};
\fill[cstdot,fill=dcolora] (-2.5,1) circle;
\fill[cstdot,fill=dcolora] (-1.7,1.2) circle;
\fill[cstdot,fill=dcolora] (-3.1,-0.3) circle;
\fill[cstdot,fill=dcolorb] (1.5,0) circle;
\fill[cstdot,fill=dcolorb] (1.7,1.92) circle;
\fill[cstdot,fill=dcolorb] (2.77,0.36) circle;
\fill[cstfille,pattern color=dcolora,visible on=<4->] (-2.5,0)--(-1.31,0.913) arc(37.5:7.5:1.5)--cycle;
\fill[cstfille,pattern color=dcolorb,visible on=<4->] (2.5,0)--(2.966,1.74) arc(75:15:1.8)--cycle;
\draw[cstcurve,dcolora,visible on=<4->] (-2.5,0) --(-1.15,1.035);
\draw[cstcurve,dcolora,visible on=<4->] (-2.5,0) --(-0.814,0.222);
\draw[cstcurve,dcolorb,visible on=<4->] (2.5,0) --(3.018,1.93);
\draw[cstcurve,dcolorb,visible on=<4->] (2.5,0) --(4.43,0.518);
\draw
  (-2.7,-0.2) node {$0$}
  (-0.5,-0.2) node {$x$}
  (-2.7,1.8) node {$y$}
  (2.3,0.2) node {$0$}
  (4.5,-0.2) node {$u$}
  (2.3,1.8) node {$v$};
\end{tikzpicture}
\end{center}}
\onslide<5->{这个映射对应两个实变函数 $u=x^2-y^2,\quad v=2xy$.}
\end{example}
\end{frame}


\begin{frame}{例题: 映照}
\onslide<+->
\begin{example}[续]
因此它把 $z$ 平面上两族分别以直线 $y=\pm x$ 和坐标轴为渐近线的等轴双曲线
\[x^2-y^2=c_1,\quad 2xy=c_2\]
\onslide<+->{分别映射为 $w$ 平面上的两族平行直线
\[u=c_1,\quad v=c_2.\]}

\vspace{-1.7\baselineskip}
\onslide<+->{
\begin{center}
\begin{tikzpicture}
\draw[cstaxis] (-4.5,0)--(-0.5,0);
\draw[cstaxis] (-2.5,-1.6)--(-2.5,1.6);
\draw[cstaxis] (0.5,0)--(4.5,0);
\draw[cstaxis] (2.5,-1.6)--(2.5,1.6);
\draw[cstcurve,dcolorb] (-3.7,-1.2)--(-1.3,1.2);
\draw[cstcurve,dcolorb] (-3.7,1.2)--(-1.3,-1.2);
\draw[cstcurve,dcolorb,smooth,domain=-35:35]
  plot ({sec(\x)-2.5},{tan(\x)})
  plot ({-sec(\x)-2.5},{tan(\x)})
  plot ({tan(\x)-2.5},{sec(\x)})
  plot ({tan(\x)-2.5},{-sec(\x)});
\draw[cstcurve,dcolorb,smooth,domain=-46:46]
  plot ({(0.8*sec(\x)-2.5)},{0.8*tan(\x)})
  plot ({(-0.8*sec(\x)-2.5)},{0.8*tan(\x)})
  plot ({0.8*tan(\x)-2.5},{0.8*sec(\x)})
  plot ({0.8*tan(\x)-2.5},{0.8*-sec(\x)});
\draw[cstcurve,dcolorb,smooth,domain=-57:57]
  plot ({(0.6*sec(\x)-2.5)},{0.6*tan(\x)})
  plot ({(-0.6*sec(\x)-2.5)},{0.6*tan(\x)})
  plot ({0.6*tan(\x)-2.5},{0.6*sec(\x)})
  plot ({0.6*tan(\x)-2.5},{0.6*-sec(\x)});
\draw[cstcurve,dcolorb,smooth,domain=-68:68]
  plot ({(0.4*sec(\x)-2.5)},{0.4*tan(\x)})
  plot ({(-0.4*sec(\x)-2.5)},{0.4*tan(\x)})
  plot ({0.4*tan(\x)-2.5},{0.4*sec(\x)})
  plot ({0.4*tan(\x)-2.5},{0.4*-sec(\x)});

\draw[cstcurve,dcolora,visible on=<4->] (-4,0)--(-1,0);
\draw[cstcurve,dcolora,visible on=<4->] (-2.5,-1.3)--(-2.5,1.3);
\draw[cstcurve,dcolora,visible on=<4->,smooth,domain=-34:34]
  plot ({0.8*(sec(\x)+tan(\x))-2.5},{0.8*(sec(\x)-tan(\x))})
  plot ({0.8*(sec(\x)+tan(\x))-2.5},{-0.8*(sec(\x)-tan(\x))})
  plot ({-0.8*(sec(\x)+tan(\x))-2.5},{0.8*(sec(\x)-tan(\x))})
  plot ({-0.8*(sec(\x)+tan(\x))-2.5},{-0.8*(sec(\x)-tan(\x))});
\draw[cstcurve,dcolora,visible on=<4->,smooth,domain=-45:45]
  plot ({0.6*(sec(\x)+tan(\x))-2.5},{0.6*(sec(\x)-tan(\x))})
  plot ({0.6*(sec(\x)+tan(\x))-2.5},{-0.6*(sec(\x)-tan(\x))})
  plot ({-0.6*(sec(\x)+tan(\x))-2.5},{0.6*(sec(\x)-tan(\x))})
  plot ({-0.6*(sec(\x)+tan(\x))-2.5},{-0.6*(sec(\x)-tan(\x))});
\draw[cstcurve,dcolora,visible on=<4->,smooth,domain=-57:57]
  plot ({0.4*(sec(\x)+tan(\x))-2.5},{0.4*(sec(\x)-tan(\x))})
  plot ({0.4*(sec(\x)+tan(\x))-2.5},{-0.4*(sec(\x)-tan(\x))})
  plot ({-0.4*(sec(\x)+tan(\x))-2.5},{0.4*(sec(\x)-tan(\x))})
  plot ({-0.4*(sec(\x)+tan(\x))-2.5},{-0.4*(sec(\x)-tan(\x))});
\draw[cstcurve,dcolora,visible on=<4->,smooth,domain=-71:71]
  plot ({0.2*(sec(\x)+tan(\x))-2.5},{0.2*(sec(\x)-tan(\x))})
  plot ({0.2*(sec(\x)+tan(\x))-2.5},{-0.2*(sec(\x)-tan(\x))})
  plot ({-0.2*(sec(\x)+tan(\x))-2.5},{0.2*(sec(\x)-tan(\x))})
  plot ({-0.2*(sec(\x)+tan(\x))-2.5},{-0.2*(sec(\x)-tan(\x))});

\draw[cstcurve,dcolorb] (1.3,-1.3)--(1.3,1.3);
\draw[cstcurve,dcolorb] (1.6,-1.3)--(1.6,1.3);
\draw[cstcurve,dcolorb] (1.9,-1.3)--(1.9,1.3);
\draw[cstcurve,dcolorb] (2.2,-1.3)--(2.2,1.3);
\draw[cstcurve,dcolorb] (2.5,-1.3)--(2.5,1.3);
\draw[cstcurve,dcolorb] (2.8,-1.3)--(2.8,1.3);
\draw[cstcurve,dcolorb] (3.1,-1.3)--(3.1,1.3);
\draw[cstcurve,dcolorb] (3.4,-1.3)--(3.4,1.3);
\draw[cstcurve,dcolorb] (3.7,-1.3)--(3.7,1.3);

\draw[cstcurve,dcolora,visible on=<4->] (1.2,-1.2)--(3.8,-1.2);
\draw[cstcurve,dcolora,visible on=<4->] (1.2,-0.9)--(3.8,-0.9);
\draw[cstcurve,dcolora,visible on=<4->] (1.2,-0.6)--(3.8,-0.6);
\draw[cstcurve,dcolora,visible on=<4->] (1.2,-0.3)--(3.8,-0.3);
\draw[cstcurve,dcolora,visible on=<4->] (1.2,0)--(3.8,0);
\draw[cstcurve,dcolora,visible on=<4->] (1.2,0.3)--(3.8,0.3);
\draw[cstcurve,dcolora,visible on=<4->] (1.2,0.6)--(3.8,0.6);
\draw[cstcurve,dcolora,visible on=<4->] (1.2,0.9)--(3.8,0.9);
\draw[cstcurve,dcolora,visible on=<4->] (1.2,1.2)--(3.8,1.2);
\end{tikzpicture}
\end{center}}
\vspace{-\baselineskip}
\end{example}
\end{frame}


\begin{frame}{例题: 映照的像}
\onslide<+->
\begin{example}
求下列集合在映照 $w=z^2$ 下的像.

\enumnum1 线段 $0<|z|<2,\arg z=\dfrac\pi2$.
\end{example}
\onslide<+->
\begin{solution}
设 $z=re^{\frac{\pi i}2}=ir$, 则 $w=z^2=-r^2$.
\onslide<+->{因此它的像还是一条线段 $0<|w|<4,\arg w=-\pi$.}
\onslide<+->{
\begin{center}
\begin{tikzpicture}
\draw[cstaxis] (-4.5,0)--(-0.5,0);
\draw[cstaxis] (-2.5,-0.2)--(-2.5,1.6);
\draw[cstaxis] (0.5,0)--(4.5,0);
\draw[cstaxis] (2.5,-0.2)--(2.5,1.6);
\draw[cstcurve,dcolora] (-2.5,0)--(-2.5,1);
\filldraw[cstdote,draw=dcolora] (-2.5,0) circle;
\filldraw[cstdote,draw=dcolora] (-2.5,1) circle;
\draw[cstcurve,dcolorb] (0.5,0)--(2.5,0);
\filldraw[cstdote,draw=dcolorb] (0.5,0) circle;
\filldraw[cstdote,draw=dcolorb] (2.5,0) circle;
\draw[cstdash,cstarrowto,dcolorc] (-2.2,0.5) to[bend left] (1.5,0.5);
\draw
  (-0.5,-0.2) node {$x$}
  (-2.7,1.3) node {$y$}
  (4.5,-0.2) node {$u$}
  (2.3,1.3) node {$v$};
\end{tikzpicture}
\end{center}}
\end{solution}
\end{frame}


\begin{frame}{例题: 映照的像}
\onslide<+->
\begin{example}
求下列集合在映照 $w=z^2$ 下的像.

\enumnum2 双曲线 $x^2-y^2=4$.
\end{example}
\onslide<+->
\begin{solution*}
由于 \[w=u+iv=z^2=(x^2-y^2)+2xyi.\]
\onslide<+->{因此 $u=x^2-y^2=4,v=2xy$.}

\onslide<+->{对于任意 $v\in\BR$, 存在 $z=x+yi\in\BC$ 使得 $z^2=4+vi$, 且 $x^2-y^2=4$.}
\onslide<+->{因此这条双曲线的像是一条直线 $\Re w=4$.}
\end{solution*}
\end{frame}


\begin{frame}{例题: 映照的像}
\onslide<+->
\begin{example}
求下列集合在映照 $w=z^2$ 下的像.

\enumnum3 扇形区域 $0<\arg z<\dfrac\pi4,0<|z|<2$.
\end{example}
\onslide<+->
\begin{solution}
\onslide<+->{设 $z=re^{i\theta}$, 则 $w=r^2e^{2i\theta}$.}
\onslide<+->{因此它的像是扇形区域 $0<\arg w<\dfrac\pi2,0<|w|<4$.}
\end{solution}
\end{frame}


\begin{frame}{例题: 映照的像}
\onslide<+->
\begin{example}
求圆周 $|z|=2$ 在映照 $w=z+\dfrac1z$ 下的像.
\end{example}
\onslide<+->
\begin{solution}
设 $z=x+yi$, 则
\[w=z+\frac1z=z+\frac{\ov z}4=\frac54x+\frac34yi=u+vi,\]
\vspace{-0.2\baselineskip}
\onslide<+->{
\[x=\frac45u,\quad y=\frac43v,\quad \left(\frac45u\right)^2+\left(\frac43v\right)^2=4,\]}
\vspace{-0.2\baselineskip}
\onslide<+->{
\[\left(\frac{2u}5\right)^2+\left(\frac{2v}3\right)^2=1.\]}
\vspace{-0.2\baselineskip}
\end{solution}
\end{frame}

