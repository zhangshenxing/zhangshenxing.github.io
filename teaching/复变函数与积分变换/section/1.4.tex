\section{曲线和区域}

\subsection{复数表平面曲线}
\begin{frame}{典型例题: 复数方程表平面图形}
	\onslide<+->
	很多的平面图形能用复数形式的方程来表示, 这种表示方程有些时候会显得更加直观和易于理解.

	\onslide<+->
	\begin{example}
		\enumnum1 $|z+i|=2$.
		\onslide<+->{该方程表示与 $-i$ 的距离为 $2$ 的点全体, 即圆心为 $-i$ 半径为 $2$ 的圆.}

		\onslide<+->{一般的圆方程为 $|z-z_0|=R$, 其中 $z_0$ 是圆心, $R$ 是半径.}
		\onslide<3->{
		\begin{center}
			\begin{tikzpicture}
				\draw[cstaxis] (-1.5,0)--(1.5,0);
				\draw[cstaxis] (0,-1.8)--(0,1);
				\draw[cstcurve,dcolora] (0,-0.5) circle(1);
				\fill[cstdot,dcolorb] (0,-0.5) circle;
				\draw
				(-0.4,-0.5) node[dcolorb] {$-i$};
			\end{tikzpicture}
		\end{center}}
	\end{example}
\end{frame}


\begin{frame}{典型例题: 复数方程表平面图形}
	\onslide<+->
	\begin{example}[续]
		\enumnum2 $|z-2i|=|z+2|$.
		\onslide<+->{该方程表示与 $2i$ 和 $-2$ 的距离相等的点, 即二者连线的垂直平分线.
		}\onslide<+->{两边同时平方化简可得 $x+y=0$.}
		\onslide<2->{
		\begin{center}
			\begin{tikzpicture}
				\draw[cstaxis] (-1.5,0)--(1.5,0);
				\draw[cstaxis] (0,-1.5)--(0,1.5);
				\draw[cstcurve,dcolora] (-1.2,1.2)--(1.2,-1.2);
				\fill[cstdot,dcolorb] (0,1) circle;
				\fill[cstdot,dcolorb] (-1,0) circle;
				\draw
					(-1,-0.3) node[dcolorb] {$-2$}
					(0.4,1) node[dcolorb] {$2i$};
			\end{tikzpicture}
		\end{center}}
	\end{example}
\end{frame}


\begin{frame}{典型例题: 复数方程表平面图形}
	\onslide<+->
	\begin{example}[续]
		\enumnum3 $\Im(i+\ov z)=4$.
		\onslide<+->{设 $z=x+yi$, 则 $\Im(i+\ov z)=1-y=4$, 因此 $y=-3$.}

		\onslide<+->{\enumnum4 $|z-z_1|+|z-z_2|=2a$.
		}\onslide<+->{该方程表示以 $z_1,z_2$ 为焦点, $a$ 为长半轴的椭圆.}

		\onslide<+->{\enumnum5 $|z-z_1|-|z-z_2|=2a$.
		}\onslide<+->{该方程表示以 $z_1,z_2$ 为焦点, $a$ 为实半轴的双曲线的一支.}
	\end{example}

	\onslide<+->
	\begin{exercise}
		$z^2+\ov z^2=1$ 和 $z^2-\ov z^2=i$ 分别表示什么图形?
	\end{exercise}

	\onslide<+->
	\begin{answer}
		双曲线 $x^2-y^2=\dfrac12$ 和双曲线 $xy=\dfrac14$.
	\end{answer}
\end{frame}


\subsection{区域的定义}
\begin{frame}{邻域}
	\onslide<+->
	为了引入极限的概念, 我们需要考虑点的邻域.
	\onslide<+->
	类比于高等数学中的邻域和去心邻域, 我们在复变函数中, 称开圆盘
	\[U(z_0,\delta)=\set{z:|z-z_0|<\delta}\]
	为 $z_0$ 的一个 \emph{$\delta$-邻域},
	\onslide<+->
	称去心开圆盘
	\[\Uc(z_0,\delta)=\set{z:0<|z-z_0|<\delta}\]
	为 $z_0$ 的一个\emph{去心 $\delta$-邻域}.

	\onslide<2->
	\begin{center}
		\begin{tikzpicture}
			\filldraw[cstcurve,dcolorb,cstfill] (0,0) circle(1);
			\draw[cstcurve,cstarrowto,dcolora] (0,0)--(0.8,0.6);
			\fill[cstdot,dcolora] (0,0) circle;
			\filldraw[cstcurve,dcolorb,cstfill,visible on=<3->] (3,0) circle(1);
			\draw[cstcurve,visible on=<3->,cstarrowto,dcolora] (3,0)--(3.8,0.6);
			\filldraw[cstdote,visible on=<3->,draw=dcolora] (3,0) circle;
			\draw
			(-0.3,0) node[dcolora] {$z_0$}
			(2.7,0) node[visible on=<3->,dcolora] {$z_0$}
			(0.5,0.1) node[dcolora] {$\delta$}
			(3.5,0.1) node[visible on=<3->,dcolora] {$\delta$};
		\end{tikzpicture}
	\end{center}
\end{frame}


\begin{frame}{内部、外部、边界}
	\onslide<+->
	设 $G$ 是复平面的一个子集, $z_0\in\BC$.
	\onslide<+->
	它们的位置关系有三种可能:
	\begin{enumerate}
		\item 如果存在 $z_0$ 的一个邻域 $U$ 完全包含在 $G$ 中, 则称 $z_0$ 是 $G$ 的一个\emph{内点}.
		\item 如果存在 $z_0$ 的一个邻域 $U$ 完全不包含在 $G$ 中, 则称 $z_0$ 是 $G$ 的一个\emph{外点}.
		\item 如果 $z_0$ 的任何一个邻域 $U$, 都有属于和不属于 $G$ 的点, 则称 $z_0$ 是 $G$ 的一个\emph{边界点}.
	\end{enumerate}
	\onslide<+->
	显然内点都属于 $G$, 外点都不属于 $G$, 而边界点则都有可能.
	\onslide<+->
	这类比于区间的端点和区间的关系.

	\onslide<1->
	\begin{center}
		\begin{tikzpicture}
			\filldraw[cstcurve,dcolora,cstfill,smooth] plot coordinates {(0,-0.8) (1,-0.8) (1.5,0) (1,0.6) (0,0.8) (-1,0.7) (-1.2,0) (-1,-0.6) (0,-0.8)};
			\draw[visible on=<3->,cstcurve,dcolorb] (-0.5,0) circle (0.5);
			\draw[visible on=<5->,cstcurve,dcolorc] (1.43,0) circle (0.5);
			\draw[visible on=<4->,cstcurve,black] (3,0) circle (0.5);
			\fill[visible on=<3->,cstdot,dcolorb] (-0.5,0) circle;
			\fill[visible on=<5->,cstdot,dcolorc] (1.5,0) circle;
			\fill[visible on=<4->,cstdot,black] (3,0) circle;
			\draw
			(0.5,0) node[dcolora] {$G$}
			(-0.5,0.2) node[visible on=<3->,dcolorb] {$z_0$}
			(1.75,0) node[visible on=<5->,dcolorc] {$z_0$}
			(3.3,0) node[visible on=<4->] {$z_0$};
		\end{tikzpicture}
	\end{center}
\end{frame}


\begin{frame}{开集和闭集}
	\onslide<+->
	如果 $G$ 的所有点都是内点, 也就是说, $G$ 的边界点都不属于它, 称 $G$ 是一个\emph{开集}.
	\onslide<+->
	例如
	\[|z-z_0|<R,\quad 1<\Re z<3,\quad\frac\pi4<\arg z<\dfrac{3\pi}4\] 都是开集.
	\onslide<+->
	如果 $G$ 的所有边界点都属于 $G$, 称 $G$ 是一个\emph{闭集}.
	\onslide<+->
	这等价于它的补集是开集.

	\onslide<+->
	直观上看: 开集往往由 $>,<$ 的不等式给出, 闭集往往由 $\ge,\le$ 的不等式给出.
	\onslide<+->
	不过注意这并不是绝对的.

	\onslide<+->
	如果 $D$ 可以被包含在某个开圆盘 $U(0,R)$ 中, 则称它是\emph{有界}的.
	\onslide<+->
	否则称它是\emph{无界}的.
\end{frame}


\begin{frame}{区域和闭区域}
	\onslide<+->
	\begin{definition}
		如果开集 $D$ 的任意两个点之间都可以用一条完全包含在 $D$ 中的折线连接起来, 则称 $D$ 是一个\emph{区域}.
		\onslide<+->{也就是说, 区域是连通的开集.}
	\end{definition}

	\onslide<+->
	观察下侧的图案, 青色部分是一个区域(不包含红色部分).
	\onslide<+->
	红色的线条和点是它的边界.
	\onslide<+->
	区域和它的边界一起构成了\emph{闭区域}, 记作 $\ov D$.
	\onslide<+->
	它是一个闭集.

	\onslide<3->
	\begin{center}
		\begin{tikzpicture}
			\filldraw[cstcurve,dcolora,cstfill,smooth] plot coordinates {(0,-1.2) (1.5,-1.2) (2.3,0) (1.5,0.9) (0,1.2) (-1.5,1) (-1.8,0) (-1.5,-0.9) (0,-1.2)};
			\filldraw[cstcurve,dcolora,fill=white,smooth] plot coordinates {(-1.4,0) (-1,0.6) (-0.6,0) (-1.2,-0.5) (-1.4,0)};
			\filldraw[cstcurve,dcolora,fill=white] (0.5,0.3) circle (0.3);
			\fill[cstdot,dcolora] (1.5,0) circle;
			\fill[cstdot,dcolora] (1.6,-0.5) circle;
			\draw[cstcurve,dcolora,smooth] plot coordinates {(1,0.5) (1.2,0.3) (1.2,-0.3) (1.4,0.5)};
			\draw[cstcurve,dcolorc] (-1,0.8)--(-0.2,0.5)--(0.2,-0.5)--(1,-0.8);
			\draw
			(-1.2,0.8) node[dcolorc] {$z_1$}
			(1.3,-0.8) node[dcolorc] {$z_2$};
		\end{tikzpicture}
	\end{center}
\end{frame}


\begin{frame}{常见区域}
	\onslide<+->
	复平面上的区域大多由复数的实部、虚部、模和辐角的不等式所确定.
	\onslide<10->{这些区域对应的闭区域是什么?}

	\onslide<2->
	\begin{center}
		\begin{tikzpicture}
			\draw[cstaxis](-1.5,0.3)--(1.5,0.3);
			\draw[cstaxis](0,-0.6)--(0,1.4);
			\fill[cstfille,pattern color=dcolorb] (-1.2,0.3) rectangle (1.2,1.1);
			\fill[cstfille,pattern color=dcolora,visible on=<3->] (-1.2,0.3) rectangle (1.2,-0.5);
			\draw[cstaxis,visible on=<4->](2.5,0)--(5.5,0);
			\draw[cstaxis,visible on=<4->](4,-1.1)--(4,1.3);
			\fill[cstfille,pattern color=dcolorb,visible on=<4->] (2.8,-1) rectangle (4,1);
			\fill[cstfille,pattern color=dcolora,visible on=<5->] (4,1) rectangle (5.2,-1);
			\draw[cstaxis,visible on=<6->](6.5,0)--(8.5,0);
			\draw[cstaxis,visible on=<6->](7,-1)--(7,1.2);
			\fill[cstfille,pattern color=dcolorb,visible on=<6->] (7.2,-0.9) rectangle (8,0.9);
			\draw
			(-0.05,1.5) node[dcolorb] {上半平面 $\Im z>0$}
			(-0.05,-0.9) node[dcolora,visible on=<3->] {下半平面 $\Im z<0$}
			(3.1,1.6) node[dcolorb,align=center,visible on=<4->] {左半平面\\$\Re z<0$}
			(4.9,1.6) node[dcolora,align=center,visible on=<5->] {右半平面\\$\Re z>0$}
			(7.7,1.7) node[dcolorb,align=center,visible on=<6->] {竖直带状区域\\$x_1<\Re z<x_2$};
		\end{tikzpicture}
	\end{center}
	\onslide<7->
	\begin{center}
		\begin{tikzpicture}
			\draw[cstaxis](-1.5,1.2)--(1.5,1.2);
			\draw[cstaxis](0,1)--(0,2.2);
			\fill[cstfille,pattern color=dcolora] (-1.2,1.3) rectangle (1.2,2);
			\draw[cstaxis,visible on=<8->](2,0)--(4,0);
			\draw[cstaxis,visible on=<8->](2.5,-0.2)--(2.5,1.3);
			\fill[cstfille,pattern color=dcolorb,visible on=<8->] (2.5,0)--(3.25,1.299) arc(60:10:1.5)--cycle;
			\filldraw[cstcurve,dcolora,cstfill,visible on=<9->] (6.2,1) circle (1);
			\filldraw[cstcurve,dcolora,cstfill,fill=white,visible on=<9->] (6.2,1) circle (0.5);
			\draw[cstaxis,visible on=<9->](4.9,1)--(7.5,1);
			\draw[cstaxis,visible on=<9->](6.2,-0.2)--(6.2,2.4);
			\draw
			(0,0.5) node[dcolora,align=center] {水平带状区域\\$y_1<\Im z<y_2$}
			(3.4,1.8) node[dcolorb,align=center,visible on=<8->] {角状区域\\$\alpha_1<\arg z<\alpha_2$}
			(8.2,2) node[dcolora,align=center,visible on=<9->] {圆环域\\$r<|z|<R$};
		\end{tikzpicture}
	\end{center}
\end{frame}


\subsection{区域的特性}
\begin{frame}{简单闭曲线(闭路)}
	\onslide<+->
	设 $x(t),y(t),t\in[a,b]$ 是两个连续函数,
	\onslide<+->
	则参变量方程
	$\begin{cases}x=x(t),& \\y=y(t),&\end{cases}t\in[a,b]$ 定义了一条\emph{连续曲线}.
	\onslide<+->
	这也等价于 $C:z=z(t)=x(t)+iy(t),t\in[a,b]$.

	\onslide<+->
	如果除了两个端点有可能重叠外, 其它情形不会出现重叠的点, 则称 $C$ 是\emph{简单曲线}.
	\onslide<+->
	如果还满足两个端点重叠, 即 $z(a)=z(b)$, 则称 $C$ 是\emph{简单闭曲线}, 也简称为\emph{闭路}.

	\onslide<2->
	\begin{center}
		\begin{tikzpicture}
			\draw[cstaxis](-0.3,0)--(9.5,0);
			\draw[cstaxis](0,-0.3)--(0,2.5);
			\draw[cstcurve,dcolora,smooth] plot coordinates {(0.2,0.9) (1,1.6) (2,0.6) (3,0.9)};
			\draw[cstcurve,dcolorb,smooth,visible on=<4->] plot coordinates {(4,0.5) (4.6,0.9) (5.3,2) (4.7,2.5) (3.7,2) (4.7,0.9) (5.7,0.5)};
			\draw[cstcurve,dcolorc,smooth,smooth,visible on=<5->] plot coordinates {(7,1) (7.5,0.4) (8,0.2) (8.5,0.4) (9,1) (8.5,1.6) (8,1.8) (7.5,1.7) (7,1)};
			\draw
			(0.5,0.7) node[dcolora] {$z(a)$}
			(3,1.2) node[dcolora] {$z(b)$};
		\end{tikzpicture}
	\end{center}
\end{frame}


\begin{frame}{闭路的内部和外部}
	\onslide<+->
	闭路 $C$ 把复平面划分成了两个区域, 一个有界一个无界.
	\onslide<+->
	分别称这两个区域是 $C$ 的\emph{内部}和\emph{外部}.
	\onslide<+->
	$C$ 是它们的公共边界.

	\onslide<+->
	这件事情的严格证明是十分困难的 (Veblen 1905).

	\onslide<1->
	\begin{center}
		\begin{tikzpicture}
			\fill[cstfille] (-2,-2) rectangle (1.5,1.5);
			\filldraw[cstcurve,dcolora,smooth,cstfill] plot coordinates {(-1.5,0) (-1,-0.5) (0,-1) (0.7,-1) (0.9,0) (0,0.8) (-1,0.8) (-1.5,0)};
		\end{tikzpicture}
	\end{center}
\end{frame}


\begin{frame}{单连通域和多连通域}
	\onslide<+->
	在前面所说的几个区域的例子中, 我们在区域中画一条闭路.
	\onslide<+->
	除了圆环域之外, 闭路的内部仍然包含在这个区域内.

	\onslide<+->
		\begin{definition}
			如果区域 $D$ 中的任一闭路的内部都包含在 $D$ 中, 则称 $D$ 是\emph{单连通域}.
			否则称之为\emph{多连通域}.
	\end{definition}

	\onslide<+->
	单连通域内的任一闭路可以连续地变形成一个点.
	\onslide<+->
	\begin{center}
		\begin{tikzpicture}
			\filldraw[cstcurve,dcolora,smooth,cstfill] plot coordinates {(-2.2,0) (-1.5,-0.7) (0,-1.5) (1,-1.5) (1.4,0) (0,1.2) (-1.5,1.2) (-2.2,0)};
			\draw[cstcurve,dcolorc,smooth] plot coordinates {(-1.6,0) (-1.2,-0.5) (-0.3,-1) (0,-0.5) (-0.3,0.2) (-1.2,0.7) (-1.6,0)};
			\filldraw[cstcurve,dcolora,smooth,fill=white] plot coordinates {(-1.4,0) (-1,-0.3) (-0.8,0) (-1,0.3) (-1.4,0)};
			\filldraw[cstcurve,dcolora,smooth,fill=white] plot coordinates {(-0.8,-0.4) (-0.4,-0.7) (-0.2,-0.4) (-0.4,-0.1) (-0.8,-0.4)};
			\draw[cstcurve,dcolora,smooth] plot coordinates {(-0.7,1) (-0.2,0.9) (0.1,0.7)};
			\draw[cstcurve,dcolora](-0.4,1.2)--(0.1,0.7);
			\fill[cstdot,dcolora] (0.7,0) circle;
			\fill[cstdot,dcolora] (1,-0.6) circle;
		\end{tikzpicture}
	\end{center}
\end{frame}


\begin{frame}{例题: 区域的特性}
	\onslide<+->
	\begin{example}
		\enumnum1 $\Re(z^2)<1$.

		\onslide<+->{设 $z=x+yi$, 则 $\Re(z^2)=x^2-y^2<1$.
		}\onslide<+->{这是无界的单连通域.}
	\end{example}

	\onslide<2->
	\begin{center}
		\begin{tikzpicture}[framed]
			\fill[cstfille] (-1.414,-1) rectangle (1.414,1);
			\filldraw[cstcurve,dcolora,domain=-45:45,smooth,fill=white] plot ({sec(\x)},{tan(\x)});
			\filldraw[cstcurve,dcolora,domain=-45:45,smooth,fill=white] plot ({-sec(\x)},{tan(\x)});
			\draw[cstaxis] (-2.5,0)--(2.5,0);
			\draw[cstaxis] (0,-1.5)--(0,1.5);
		\end{tikzpicture}
	\end{center}
\end{frame}


\begin{frame}{例题: 区域的特性}
	\onslide<+->
	\begin{example}[续]
		\enumnum2 $|\arg z|<\dfrac\pi3$ (不含原点).
		\onslide<+->{即角状区域 $-\dfrac\pi3<\arg z<\dfrac\pi3$.
		}\onslide<+->{这是无界的单连通域.}
	\end{example}

	\onslide<2->
	\begin{center}
		\begin{tikzpicture}[framed]
			\fill[cstfille] (0,0)--(1,1.732) arc(60:-60:2)--cycle;
			\draw[cstcurve,dcolora] (0,0)--(1.2,2.078);
			\draw[cstcurve,dcolora] (0,0)--(1.2,-2.078);
			\draw[cstaxis] (-0.5,0)--(2.5,0);
			\draw[cstaxis] (0,-2.5)--(0,2.5);
		\end{tikzpicture}
	\end{center}
\end{frame}


\begin{frame}{例题: 区域的特性}
	\onslide<+->
	\begin{example}[续]
		\enumnum3 $\abs{\dfrac1z}\le3$.
		\onslide<+->{即 $|z|\ge\dfrac13$.
		}\onslide<+->{这是无界的多连通\emph{闭区域}.}
	\end{example}

	\onslide<2->
	\begin{center}
		\begin{tikzpicture}[framed]
			\fill[cstfille1] (-1.6,-1.6) rectangle (1.6,1.6);
			\filldraw[cstcurve,dcolora,fill=white] (0,0) circle (0.5);
			\draw[cstaxis] (-2,0)--(2,0);
			\draw[cstaxis] (0,-2)--(0,2);
		\end{tikzpicture}
	\end{center}
\end{frame}


\begin{frame}{例题: 区域的特性}
	\onslide<+->
	\begin{example}[续]
		\enumnum4 $|z+1|+|z-1|<4$.

		\onslide<+->{表示一个椭圆的内部.
		}\onslide<+->{这是有界的单连通域.}
	\end{example}

	\onslide<+->
	\begin{center}
		\begin{tikzpicture}[framed]
			\filldraw[cstcurve,dcolora,cstfill] (0,0) circle (1 and {0.5*sqrt(3)});
			\draw[cstaxis] (-1.5,0)--(1.5,0);
			\draw[cstaxis] (0,-1.1)--(0,1.1);
		\end{tikzpicture}
	\end{center}

	\onslide<+->
	\begin{thinking}
		$|z+1|+|z-1|\ge 1$ 表示什么集合?
	\end{thinking}

	\onslide<+->
	\begin{answer}
		整个复平面.
	\end{answer}
\end{frame}

