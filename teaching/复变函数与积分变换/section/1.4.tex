\section{曲线和区域}


\subsection{复数表平面曲线}


\begin{frame}{例题: 复数方程表圆}
	\onslide<+->
	很多的平面图形能用复数形式的方程来表示.
	\onslide<+->
	这种表示方程有些时候会显得更加直观和易于理解.
	\onslide<+->
	\begin{example}[sidepic,righthand width=3.3cm,indent]
		$|z+2\ii|=2$.
		\onslide<+->{%
			该方程表示与 $-2\ii $ 的距离为 $2$ 的点全体, 即圆心为 $-2\ii $ 半径为 $2$ 的圆.
		}
		
		\onslide<+->{%
			一般的圆方程为 $|z-z_0|=R$, 其中 $z_0$ 是圆心, $R$ 是半径.
		}
		\tcblower
		\onslide<4->{\begin{tikzpicture}
			\draw[cstaxis] (-1.5,0)--(1.5,0);
			\draw[cstaxis] (0,-2.5)--(0,.5);
			\coordinate (A) at (0,-1);
			\fill[cstdot] (A) circle
				node[left] {$-2\ii $};
			\draw[cstcurve,main] (A) circle (1);
		\end{tikzpicture}}
	\end{example}
\end{frame}


\begin{frame}{例题: 复数方程表直线}
	\onslide<+->
	\begin{example}[sidepic,righthand width=3.3cm]
		$|z-4\ii|=|z-2|$.
		\onslide<+->{%
			该方程表示与 $4\ii$ 和 $2$ 的距离相等的点, 即二者连线的垂直平分线.
		}\onslide<+->{%
			两边同时平方化简可得 $x-2y+3=0$.
		}\onslide<+->{%
			该方程也可以表达为
			\[
				(1+2\ii)z+(1-2\ii)\ov z+6=0.
			\]
		}
		\tcblower
		\onslide<2->{
      \begin{tikzpicture}[
        declare function={
          f(\y)=2*\y-1.5;
        }
      ]
        \def\u{.65}
        \draw[cstaxis] (-.7,0)--(1.7,0);
        \draw[cstaxis] (0,-.5)--(0,1.9);
        \coordinate (A) at (\u,0);
        \coordinate (B) at (0,{2*\u});
        \draw[cstdash,fourth] (A)--(B);
        \draw[cstcurve,main] ({f(.2)*\u},{.2*\u})--({f(1.8)*\u},{1.8*\u});
        \begin{scope}[cstdot]
          \fill (A) circle node[below] {$2$};
          \fill (B) circle node[right] {$4\ii$};
        \end{scope}
      \end{tikzpicture}}
	\end{example}
\end{frame}


\begin{frame}{例题: 复数方程表椭圆和双曲线}
	\onslide<+->
	\begin{example}
		$|z-z_1|+|z-z_2|=2a$.
		\begin{itemize}
			\item 当 $2a>|z_1-z_2|$ 时, 该方程表示以 $z_1,z_2$ 为焦点, $a$ 为长半轴的椭圆;
			\item 当 $2a=|z_1-z_2|$ 时, 该方程表示连接 $z_1,z_2$ 的线段;
			\item 当 $2a<|z_1-z_2|$ 时, 该方程表示空集.
		\end{itemize}
	\end{example}
	\onslide<+->
	\begin{example}
		$|z-z_1|-|z-z_2|=2a$.
		\begin{itemize}
			\item 当 $2a<|z_1-z_2|$ 时, 该方程表示以 $z_1,z_2$ 为焦点, $a$ 为实半轴的双曲线的一支;
			\item 当 $2a=|z_1-z_2|$ 时, 该方程表示以 $z_2$ 为起点, 与 $z_2,z_1$ 连线反向的射线;
			\item 当 $2a>|z_1-z_2|$ 时, 该方程表示空集.	
		\end{itemize}
	\end{example}
\end{frame}


\begin{frame}{例题: 复数方程表平面图形}
	\onslide<+->
	\begin{exercise}[nearnext]
		$z^2+\ov z^2=1$ 和 $z^2-\ov z^2=\ii$ 分别表示什么图形?
	\end{exercise}

	\onslide<+->
	\begin{answer}[nearprev]
		双曲线 $x^2-y^2=\dfrac12$ 和双曲线 $xy=\dfrac14$.
	\end{answer}
\end{frame}


\begin{frame}{连续曲线、闭路}
	\onslide<+->
	设 $x(t),y(t),t\in[a,b]$ 是两个连续函数.
	\onslide<+->
	参变量方程 $\begin{cases}
		x=x(t),& \\y=y(t),&
	\end{cases}t\in[a,b]$ 定义了一条\emph{连续曲线}.
	\onslide<+->
	这也等价于 $C:z=z(t)=x(t)+\ii y(t),t\in[a,b]$.
	\begin{itemize}
		\item 若除了两个端点有可能重叠外, 其它情形不会出现重叠的点, 则称 $C$ 是\emph{简单曲线}.
		\item 若连续曲线 $C$ 满足两个端点重叠, 即 $z(a) = z(b)$, 则称 $C$ 是闭合曲线.
		\item 称闭合的简单曲线为\emph{简单闭曲线}或\emph{闭路}.
	\end{itemize}
	\onslide<3->
	\begin{center}
		\begin{tikzpicture}
			\draw[cstaxis](-.3,.5)--(9.5,.5);
			\draw[cstaxis](0,.2)--(0,2.5);
			\coordinate (A) at (1.2,1.4);
			\coordinate (B) at (4,1.4);
			\draw[cstcurve,fourth,smooth] plot coordinates {(A) (2,2.1) (3,1.1) (B)};
			\fill[cstdot] (A) circle node[left] {$z(a)$};
			\fill[cstdot] (B) circle node[above] {$z(b)$};
			\draw[cstcurve,second,smooth,visible on=<4->] plot coordinates {(4.8,.7) (5.76,1.25) (6.31,2) (5.77,2.71) (4.77,2.45) (4.78,1.55) (6.2,.7) };
			\draw[cstcurve,main,smooth,visible on=<5->] plot coordinates {(9.02,1.5) (8.9,1.98) (8.33,2.27) (7.69,2.18) (7.07,1.95) (6.94,1.5) (7.11,1.04) (7.68,.75) (8.34,.82) (8.93,1.08) (9.02,1.5)};
		\end{tikzpicture}
	\end{center}
\end{frame}


\begin{frame}{例题: 连续曲线}
	\onslide<+->
	\begin{example}[near]
		圆 $\abs{z-z_0}=R$ 的参数方程: $z=z_0+R\ee^{\ii\theta},\quad\theta\in[0,2\pi]$.
	\end{example}
	\onslide<+->
	\begin{example}[sidepic,righthand width=150pt,near]
		直线段:
		\[
			z(t)=z_0+(z_1-z_0)t,\quad t\in[0,1],
		\]
		其中 $z_0,z_1$ 为两个端点.
		它是简单曲线.
		\tcblower
		\begin{center}
			\begin{tikzpicture}
				\draw[cstaxis] (-1.3,.3)--(1.7,.3);
				\draw[cstaxis] (-.3,-1)--(-.3,1.6);
				\coordinate (A) at (-1,-.5);
				\coordinate (B) at (1,.5);
				\draw[cstcurve,main] (A)--(B);
				\begin{scope}[cstdot,fourth]
					\fill (A) circle node[below] {$z(0)$};
					\fill (B) circle node[above] {$z(1)$};
				\end{scope}
			\end{tikzpicture}
		\end{center}
	\end{example}
	\onslide<+->
	\begin{example}[sidepic,righthand width=150pt,near]
		正弦函数曲线段
		\[
			z(t)=\sin t,\quad t\in[0,2\pi]
		\]
		是简单曲线.
		\tcblower
		\begin{center}
			\begin{tikzpicture}
				\draw[cstcurve,main,domain=0:360,smooth] plot ({\x*pi/360},{sin(\x)*.5});
				\draw[cstaxis] (0,-1)--(0,1);
				\draw[cstaxis] (-.5,0)--(3.5,0);
				\coordinate (C) at (0,0);
				\coordinate (D) at ({pi},0);
				\begin{scope}[cstdot,fourth]
					\fill (C) circle node[below right] {$z(0)$};
					\fill (D) circle node[above] {$z(2\pi)$};
				\end{scope}
			\end{tikzpicture}
		\end{center}
	\end{example}
\end{frame}


\begin{frame}{例题: 连续曲线}
	\onslide<+->
	\begin{example}[sidepic,righthand width=150pt]
		椭圆 $\abs{z-\sqrt5}+\abs{z+\sqrt5}=6$ 的参数方程: 
		\[
			z=3\cos\theta+2\ii\sin\theta,\quad \theta\in[0,2\pi].
		\]
		\tcblower
		\begin{center}
			\begin{tikzpicture}
				\coordinate (E) at (-.74,0);
				\coordinate (F) at (.74,0);
				\draw[cstcurve,main] (0,0) circle(1.25 and 1);
				\draw[cstaxis] (-1.5,0)--(1.5,0);
				\draw[cstaxis] (0,-1.3)--(0,1.3);
				\draw[cstdash,fourth] (E)--(.74,.73)--(F);
				\begin{scope}[cstdot,fourth]
					\fill (E) circle;
					\fill (F) circle;
				\end{scope}
			\end{tikzpicture}
		\end{center}
	\end{example}
	\onslide<+->
	\begin{example}[sidepic,righthand width=150pt]
		双纽线 $\abs{z^2-1}=1$ 是闭合曲线, 但不是闭路.
		\tcblower
		\begin{center}
			\begin{tikzpicture}
				\draw[cstaxis] (-1.8,0)--(1.8,0);
				\draw[cstaxis] (0,-1.3)--(0,1.3);
				\draw[cstcurve,main,domain=-45:45,smooth] plot ({sqrt(2*cos(2*\x))*cos(\x)},{sqrt(2*cos(2*\x))*sin(\x)});
				\draw[cstcurve,main,domain=-45:45,smooth] plot ({-sqrt(2*cos(2*\x))*cos(\x)},{sqrt(2*cos(2*\x))*sin(\x)});
			\end{tikzpicture}
		\end{center}
	\end{example}
\end{frame}


\subsection{区域和闭区域}


\begin{frame}{邻域}
	\onslide<+->
	为了引入极限的概念, 我们需要考虑点的邻域.
	\onslide<+->
	类比于高等数学中的邻域和去心邻域, 我们在复变函数中, 称开圆盘
		\[
			U(z_0,\delta)=\{z:|z-z_0|<\delta\}
		\]
		为 $z_0$ 的一个 \emph{$\delta$-邻域}.
	\onslide<+->
	称去心开圆盘
		\[
			\Uc(z_0,\delta)=\{z:0<|z-z_0|<\delta\}
		\]
		为 $z_0$ 的一个\emph{去心 $\delta$-邻域}.
	\onslide<2->
	\begin{center}
		\begin{tikzpicture}
			\begin{scope}
				\coordinate (A) at (-1,0);
				\coordinate (B) at (2,0);
				\filldraw[cstcurve,main,cstfill1] (A) circle (1);
				\filldraw[cstcurve,main,cstfill1,visible on=<3->] (B) circle (1);
				\draw[cstra,fourth,thick] (A)--++(.8,.6)
					node[midway,above left] {$\delta$};
				\draw[cstra,fourth,thick,visible on=<3->] (B)--++(.8,.6)
					node[midway,above left] {$\delta$};
			\end{scope}
			\fill[cstdot] (A) circle node[left] {$z_0$};
			\filldraw[cstdote,fourth,visible on=<3->] (B) circle node[left] {$z_0$};
		\end{tikzpicture}
	\end{center}
\end{frame}


\begin{frame}{内部、外部、边界}
	\onslide<+->
	设 $G$ 是复平面的一个子集, $z_0\in\BC$.
	\onslide<+->
	它们的位置关系有三种可能:
		\begin{enumerate}
			\item 若存在 $z_0$ 的一个邻域 $U$ 完全包含在 $G$ 中, 则称 $z_0$ 是 $G$ 的一个\emph{内点}.
			\item 若存在 $z_0$ 的一个邻域 $U$ 完全不包含在 $G$ 中, 则称 $z_0$ 是 $G$ 的一个\emph{外点}.
			\item 若 $z_0$ 既不是内点也不是外点, 则称 $z_0$ 是 $G$ 的一个\emph{边界点}.
		\end{enumerate}
	\onslide<+->
	内点都属于 $G$, 外点都不属于 $G$, 而边界点则都有可能.
	\onslide<+->
	这类比于区间的端点和区间的关系.
	\onslide<1->
	\begin{center}
		\begin{tikzpicture}[scale=.9]
			\filldraw[cstcurve,main,cstfill1,smooth] plot coordinates {(2,0) (1.83,.9) (.64,1.46) (-.63,1.6) (-1.66,1.01) (-2.35,0) (-1.81,-1.06) (-.73,-1.68) (.74,-1.57) (1.82,-.91) (2,0)};
			\begin{scope}[visible on=<3->]
				\coordinate (A) at (-.7,0);
				\draw[cstcurve,second] (A) circle (.5) node[above] {$z_0$};
				\fill[cstdot,second] (A) circle;
			\end{scope}
			\begin{scope}[visible on=<5->]
				\coordinate (B) at (2,0);
				\draw[cstcurve,third] (B) circle (.5) node[right] {$z_0$};
				\fill[cstdot,third] (B) circle;
			\end{scope}
			\begin{scope}[visible on=<4->]
				\coordinate (C) at (4,0);
				\draw[cstcurve,fourth] (C) circle (.5) node[above] {$z_0$};
				\fill[cstdot,fourth] (C) circle;
			\end{scope}
			\draw (.5,0) node[main] {$G$};
		\end{tikzpicture}
	\end{center}
\end{frame}


\begin{frame}{开集和闭集、有界和无界}
	\onslide<+->
	\begin{definition}[nearnext]
		\begin{enumerate}
			\item 若 $G$ 的所有点都是内点, 也就是说, $G$ 的边界点都不属于它, 称 $G$ 是一个\emph{开集}.
			\item 若 $G$ 的所有边界点都属于 $G$, 称 $G$ 是一个\emph{闭集}.
		\end{enumerate}
	\end{definition}
	\onslide<+->
	$G$ 是一个闭集当且仅当它的补集是开集.

	\onslide<+->
	例如
	\[
		\abs{z-z_0}<R,\quad 1<\Re z<3,\quad\frac\pi4<\arg z<\dfrac{3\pi}4
	\]
	都是开集.
	\onslide<+->
	直观上看: 开集往往由 $>,<$ 的不等式给出, 闭集往往由 $\ge,\le$ 的不等式给出. 
	\onslide<+->
	不过注意这并不是绝对的.

	\onslide<+->
	若 $D$ 可以被包含在某个开圆盘 $U(0,R)$ 中, 则称它是\emph{有界}的.
	\onslide<+->
	否则称它是\emph{无界}的.
\end{frame}


\begin{frame}{区域和闭区域}
	\onslide<+->
	\begin{definition}
		若开集 $D$ 的任意两个点之间都可以用一条完全包含在 $D$ 中的折线连接起来, 则称 $D$ 是一个\emph{区域}.
		\onslide<+->{%
			也就是说, 区域是连通的开集.
		}
	\end{definition}
	\onslide<+->
	\begin{twopart}[indent]{132pt}
		观察右侧图案, 阴影部分(不包含线条部分)中任意两点可用折线连接, 因此它是一个区域.
		\onslide<+->
		这些线条和点构成了它的边界.

		\onslide<+->
		区域和它的边界一起构成了\emph{闭区域}, 记作 $\ov D$.
		\onslide<+->
		它是一个闭集.
		\tcblower
		\onslide<3->
		\begin{tikzpicture}[scale=.8]
			\filldraw[cstcurve,main,cstfill1,smooth] plot coordinates {(2.81,0) (2.37,1.03) (.91,1.7) (-.8,1.48) (-2.29,1.05) (-2.89,0) (-2.24,-1.03) (-.92,-1.64) (.81,-1.65) (2.38,-.93) (2.81,0)};
			\filldraw[cstcurve,main,fill=white,smooth] plot coordinates {(-.86,-.3) (-1.16,.31) (-1.62,.1) (-1.68,-.69) (-1.17,-.91) (-.86,-.3)};
			\filldraw[cstcurve,main,fill=white] (.5,.3) circle (.3);
			\fill[cstdot,main] (1.5,0) circle;
			\fill[cstdot,main] (1.6,-.5) circle;
			\draw[cstcurve,second,main] plot coordinates {(1,.5) (1.2,.3) (1.2,-.3) (1.4,.5)};
			\coordinate [label=left:\textcolor{second}{$z_1$}] (A) at (-1,.8);
			\coordinate [label=below:\textcolor{second}{$z_2$}] (B) at (1,-.8);
			\draw[cstcurve,second] (A)--(-.2,.5)--(.2,-.5)--(B);
		\end{tikzpicture}
	\end{twopart}
	\onslide<+->
	注意数学中边界的概念与日常所说的边界是两码事.
	\onslide<+->
	例如区域 $|z|>1$ 的边界是 $|z|=1$, 其闭区域是 $|z|\ge 1$.
\end{frame}


\begin{frame}{常见区域}\small
	\onslide<+->
	很多区域可以由复数的实部、虚部、模和辐角的不等式所确定.
	\onslide<7->{%
		下方区域对应的闭区域是什么?
	}
	\vspace{-.5\baselineskip}
	\onslide<2->
	\begin{figure}[hbpt]
		\begin{minipage}{.24\textwidth}
			\centering
			\begin{tikzpicture}[scale=.7]
				\draw[cstaxis](-1.5,0)--(1.5,0);
				\draw[cstaxis](0,-1.5)--(0,1.5);
				\fill[cstfille1] (-1.2,0) rectangle (1.2,.8);
				\draw (0,-1.5) node[below,align=center] {上半平面\\$\Im z>0$};
			\end{tikzpicture}
		\end{minipage}
		\begin{minipage}{.24\textwidth}
			\centering
			\begin{tikzpicture}[scale=.7]
				\draw[cstaxis](-1.5,0)--(1.5,0);
				\draw[cstaxis](0,-1.5)--(0,1.5);
				\fill[cstfille1] (-1.2,0) rectangle (1.2,-.8);
				\draw (0,-1.5) node[below,align=center] {下半平面\\$\Im z<0$};
			\end{tikzpicture}
		\end{minipage}
		\begin{minipage}{.24\textwidth}
			\centering
			\begin{tikzpicture}[scale=.7,visible on=<3->]
				\draw[cstaxis](-1.5,0)--(1.5,0);
				\draw[cstaxis](0,-1.5)--(0,1.5);
				\fill[cstfille1] (-1.2,-1) rectangle (0,1);
				\draw (0,-1.5) node[below,align=center] {左半平面\\$\Re z<0$};
			\end{tikzpicture}
		\end{minipage}
		\begin{minipage}{.24\textwidth}
			\centering
			\begin{tikzpicture}[scale=.7,visible on=<3->]
				\draw[cstaxis](-1.5,0)--(1.5,0);
				\draw[cstaxis](0,-1.5)--(0,1.5);
				\fill[cstfille1] (0,1) rectangle (1.2,-1);
				\draw (0,-1.5) node[below,align=center] {右半平面\\$\Re z>0$};
			\end{tikzpicture}
		\end{minipage}
	\end{figure}
	\vspace{-.5\baselineskip}	
	\begin{figure}[hbpt]
		\begin{minipage}{.24\textwidth}
			\centering
			\begin{tikzpicture}[scale=.7,visible on=<4->]
				\draw[cstaxis](-1.5,0)--(1.5,0);
				\draw[cstaxis](0,-1.5)--(0,1.5);
				\fill[cstfille1] (-.6,-1) rectangle (.2,1);
				\draw[cstcurve,main] (-.6,-1)--(-.6,1);
				\draw[cstcurve,main] (.2,-1)--(.2,1);
				\draw (0,-1.5) node[below,align=center] {竖直带状区域\\$x_1<\Re z<x_2$};
			\end{tikzpicture}
		\end{minipage}
		\begin{minipage}{.24\textwidth}
			\centering
			\begin{tikzpicture}[scale=.7,visible on=<4->]
				\draw[cstaxis](-1.5,0)--(1.5,0);
				\draw[cstaxis](0,-1.5)--(0,1.5);
				\fill[cstfille1] (-1,-.4) rectangle (1,.4);
				\draw[cstcurve,main] (-1,-.4)--(1,-.4);
				\draw[cstcurve,main] (-1,.4)--(1,.4);
				\draw (0,-1.5) node[below,align=center] {水平带状区域\\$y_1<\Im z<y_2$};
			\end{tikzpicture}
		\end{minipage}
		\begin{minipage}{.24\textwidth}
			\centering
			\begin{tikzpicture}[scale=.7,visible on=<5->]
				\draw[cstaxis](-.5,0)--(2.5,0);
				\draw[cstaxis](0,-.5)--(0,2.5);
				\coordinate (A) at (0,0);
				\coordinate (B) at ({2.2*cos(60)},{2.2*sin(60)});
				\coordinate (C) at ({2.2*cos(10)},{2.2*sin(10)});
				\fill[cstfille1] (A)--(B) arc(60:10:2.2)--cycle;
				\draw[cstcurve,main] (C)--(A)--(B);
				\draw (1,-.5) node[below,align=center] {角状区域\\$\alpha_1<\arg z<\alpha_2$};
			\end{tikzpicture}
		\end{minipage}
		\begin{minipage}{.24\textwidth}
			\centering
			\begin{tikzpicture}[scale=.7,visible on=<6->]
				\filldraw[cstcurve,main,cstfill1] (0,0) circle (1.2);
				\filldraw[cstcurve,main,fill=white] (0,0) circle (.6);
				\draw[cstaxis](-1.5,0)--(1.5,0);
				\draw[cstaxis](0,-1.5)--(0,1.5);
				\draw (0,-1.5) node[below,align=center] {圆环域\\$r<|z|<R$};
			\end{tikzpicture}
		\end{minipage}
	\end{figure}
\end{frame}


\subsection{区域的特性}


\begin{frame}{闭路的内部和外部}
	\onslide<+->
	闭路 $C$ 把复平面划分成了两个区域, 一个有界一个无界.
	\onslide<+->
	分别称这两个区域是 $C$ 的\emph{内部}和\emph{外部}.
	\onslide<+->
	$C$ 是它们的公共边界.
	\onslide<1->
	\begin{center}
		\begin{tikzpicture}
			\fill[cstfille1] (-2,-1.5) rectangle (2,1.5);
			\filldraw[cstcurve,main,cstfill1,smooth] plot coordinates {(1.18,0) (.93,.64) (.33,1.06) (-.32,1.08) (-.83,.59) (-1.15,0) (-.83,-.68) (-.36,-1) (.36,-1.08) (.83,-.69) (1.18,0)};
		\end{tikzpicture}
	\end{center}
\end{frame}


\begin{frame}{单连通区域和多连通区域}
	\onslide<+->
	在前面所说的几个常见区域的例子中, 我们在区域中画一条闭路.
	\onslide<+->
	除了圆环域之外, 闭路的内部仍然包含在这个区域内.
	\onslide<+->
	\begin{definition}
		若区域 $D$ 中的任一闭路的内部都包含在 $D$ 中, 则称 $D$ 是\emph{单连通区域}.
		否则称之为\emph{多连通区域}.
	\end{definition}
	\onslide<+->
	单连通区域内的任一闭路可以``连续地变形''成一个点.
	\onslide<+->
	这也等价于: 设 $\ell_0,\ell_1$ 是从 $A$ 到 $B$ 的两条连续曲线, 则 $\ell_0$ 可以连续地变形为 $\ell_1$ 且保持端点不动.
	\onslide<4->
	\bigdel
	\begin{center}
		\begin{tikzpicture}[scale=.7]
			\filldraw[cstcurve,main,cstfill1,smooth] plot coordinates {(2.81,0) (2.37,1.03) (.91,1.7) (-.8,1.48) (-2.29,1.05) (-2.89,0) (-2.24,-1.03) (-.92,-1.64) (.81,-1.65) (2.38,-.93) (2.81,0)};
			\filldraw[cstcurve,main,fill=white,smooth] plot coordinates {(-.86,-.3) (-1.16,.31) (-1.62,.1) (-1.68,-.69) (-1.17,-.91) (-.86,-.3)};
			\filldraw[cstcurve,main,fill=white] (.5,.3) circle (.3);
			\fill[cstdot,main] (1.5,0) circle;
			\fill[cstdot,main] (1.6,-.5) circle;
			\draw[cstcurve,main] plot coordinates {(1,.5) (1.2,.3) (1.2,-.3) (1.4,.5)};
			\draw[cstcurve,second,smooth,visible on=<5->,shift={(.1,.2)}] plot coordinates {(1.94,-.2) (1.79,.41) (1.23,.77) (.58,.81) (.04,.35) (-.3,-.2) (.03,-.81) (.53,-1.19) (1.23,-1.08) (1.76,-.82) (1.94,-.2)};
		\end{tikzpicture}
	\end{center}
\end{frame}


\begin{frame}{例题: 区域的特性}
	\onslide<+->
	\begin{example}[sidepic,righthand width=3.3cm]
		$\Re(z^2)<1$.
		\onslide<+->{%
			设 $z=x+y\ii$, 则 $\Re(z^2)=x^2-y^2<1$.
		}\onslide<+->{%
			这是无界的单连通区域.
		}
		\tcblower
		\onslide<2->{%
		\begin{center}
			\begin{tikzpicture}[scale=.7]
				\def\t{{atan(1.5)}}
				\fill[cstfille1,domain=-\t:\t,smooth] plot ({sec(\x)},{tan(\x)})
				-- plot[domain=\t:-\t] ({-sec(\x)},{tan(\x)})
				--cycle;
				\draw[cstcurve,main,domain=-\t:\t,smooth] plot ({sec(\x)},{tan(\x)});
				\draw[cstcurve,main,domain=-\t:\t,smooth] plot ({-sec(\x)},{tan(\x)});
				\draw[cstaxis] (-1.8,0)--(1.8,0);
				\draw[cstaxis] (0,-1.5)--(0,1.5);
			\end{tikzpicture}
		\end{center}}
	\end{example}
	\onslide<+->
	\begin{example}[sidepic,righthand width=3.3cm]
		$\arg z\neq \pi$. 
		\onslide<+->{%
			即角状区域 $-\pi<\arg z<\pi$.
		}\onslide<+->{%
			这是无界的单连通区域.
		}
		\tcblower
		\onslide<5->{%
		\begin{center}
			\begin{tikzpicture}[scale=.8]
				\fill[cstfille1] (0,0) circle (1.2);
				\draw[cstaxis] (0,0)--(1.5,0);
				\draw[cstaxis] (0,-1.5)--(0,1.5);
				\draw[cstdash,main] (-1.5,0)--(0,0);
			\end{tikzpicture}
		\end{center}}
	\end{example}
\end{frame}


\begin{frame}{例题: 区域的特性}
	\onslide<+->
	\begin{example}[near,sidepic,righthand width=3.3cm]
		$\Bigabs{\dfrac1z}\le3$.
		\onslide<+->{%
			即 $|z|\ge\dfrac13$.
		}\onslide<+->{%
			这是无界的多连通\alertn{闭区域}.
		}
		\tcblower
		\onslide<2->{%
		\begin{center}
			\begin{tikzpicture}[scale=.8]
				\fill[cstfille1] (-1.2,-1.2) rectangle (1.2,1.2);
				\filldraw[cstcurve,main,fill=white] (0,0) circle (.5);
				\draw[cstaxis] (-1.5,0)--(1.5,0);
				\draw[cstaxis] (0,-1.5)--(0,1.5);
			\end{tikzpicture}
		\end{center}}
	\end{example}
	\onslide<+->
	\begin{example}[near,sidepic,righthand width=3.3cm]
		$|z+1|+|z-1|<4$.
		\onslide<+->{%
			表示一个椭圆的内部.
		}\onslide<+->{%
			这是有界的单连通区域.
		}
		
		\tcblower
		\onslide<5->{%
		\begin{center}
			\begin{tikzpicture}[scale=.9]
				\filldraw[cstdash,main,cstfill1] (0,0) circle (1 and {0.5*sqrt(3)});
				\draw[cstaxis] (-1.5,0)--(1.5,0);
				\draw[cstaxis] (0,-1.5)--(0,1.5);
			\end{tikzpicture}
		\end{center}}
	\end{example}
	\onslide<+->
	\begin{exercise}[near]
		$|z+1|+|z-1|\ge 1$ 表示什么集合?
		\onslide<+->{\alertn{整个复平面.}}
	\end{exercise}
\end{frame}

