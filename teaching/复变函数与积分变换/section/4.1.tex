\section{复数项级数}

\subsection{复数项级数}
\begin{frame}{复数项级数}
	\onslide<+->
	复数域上的级数与实数域上的级数并无本质差别.

	\onslide<+->
	\begin{definition}
		\begin{itemize}
			\item 设 $\{z_n\}_{n\ge1}$ 是复数列. 表达式 $\suml_{n=1}^\infty z_n$ 称为复数项\emph{无穷级数}.
			\item 称 $s_n:=z_1+z_2+\cdots+z_n$ 为该级数的\emph{部分和}.
			\item 如果部分和数列 $\set{s_n}_{n\ge 1}$ 极限存在, 则称 $\suml_{n=1}^\infty z_n$ \emph{收敛}, 并记 $\suml_{n=1}^\infty z_n=\liml_{n\to\infty}s_n$ 为它的\emph{和}. 否则称该级数\emph{发散}.
		\end{itemize}
	\end{definition}

	\onslide<+->
	如果 $\suml_{n=1}^\infty z_n=A$ 收敛, 则 $z_n=s_n-s_{n-1}\to A-A=0$.
	\onslide<+->
	因此 \alert{$z_n\to0$ 是 $\suml_{n=1}^\infty z_n$ 收敛的必要条件}.
\end{frame}


\begin{frame}{复数项级数敛散性的判定}
	\onslide<+->
	\begin{theorem}
		$\suml_{n=1}^\infty z_n=a+bi$ 当且仅当 $\suml_{n=1}^\infty x_n=a,\suml_{n=1}^\infty y_n=b$.
	\end{theorem}

	\onslide<+->
	\begin{proof}
		设部分和
	\[
		\sigma_n=x_1+x_2+\cdots+x_n,\quad
			\tau_n=y_1+y_2+\cdots+y_n.
	\]
		\onslide<+->{则
			\[s_n=z_1+z_2+\cdots+z_n=\sigma_n+\ii\tau_n.
	\]
		}\onslide<+->{由复数列的敛散性判定条件可知
			\[\lim_{n\to\infty}s_n=a+bi\iff	\lim_{n\to\infty}\sigma_n=a,\quad \lim_{n\to\infty}\tau_n=b.
	\]
		}\onslide<+->{于是命题得证.\qedhere}
	\end{proof}
\end{frame}


\begin{frame}{复数项级数敛散性的判定}
	\beqskip{3pt}
	\onslide<+->
	\begin{theorem}
		如果实数项级数 $\suml_{n=1}^\infty|z_n|$ 收敛, 则 $\suml_{n=1}^\infty z_n$ 也收敛, 且 $\abs{\suml_{n=1}^\infty z_n}\le\suml_{n=1}^\infty |z_n|$.
	\end{theorem}
	\onslide<+->
	\begin{proof}[indent]
		因为 $|x_n|,|y_n|\le|z_n|$, 由比较判别法可知实数项级数 $\suml_{n=1}^\infty x_n$, $\suml_{n=1}^\infty y_n$ 绝对收敛, 从而收敛.
	\onslide<+->{%
		故 $\suml_{n=1}^\infty z_n$ 也收敛.}

	\onslide<+->{%
		由三角不等式可知
			$\displaystyle\abs{\sum_{k=1}^n z_k}\le \sum_{k=1}^n|z_k|$.
	}\onslide<+->{%
		两边同时取极限即得级数的不等式关系
			\[\abs{\sum_{n=1}^\infty z_n}=\abs{\lim_{n\to\infty}\sum_{k=1}^n z_k}=
			\lim_{n\to\infty}\abs{\sum_{k=1}^n z_k}\le\lim_{n\to\infty}\sum_{k=1}^n|z_k|=\sum_{n=1}^\infty |z_n|,
	\]
	}\onslide<+->{%
		其中第二个等式是因为绝对值函数 $|z|$ 连续.\qedhere
	}
	\end{proof}
	\endgroup
\end{frame}

\subsection{绝对收敛和条件收敛}

\begin{frame}{绝对收敛和条件收敛}
	\onslide<+->
	\begin{definition}
			\begin{enumerate}
			\item 如果级数 $\suml_{n=1}^\infty |z_n|$ 收敛, 则称 $\suml_{n=1}^\infty z_n$ \emph{绝对收敛}.
			\item 称收敛但不绝对收敛的级数\emph{条件收敛}.
		\end{enumerate}
	\end{definition}
	\onslide<+->
	绝对收敛的复级数各项可以任意重排次序而不改变其绝对收敛性, 且不改变其和.
	\onslide<+->
	一般的级数重排有限项不改变其敛散性与和, 但如果重排无限项则可能会改变其敛散性与和.

	\onslide<+->
	\begin{theorem}
		$\suml_{n=1}^\infty z_n$ 绝对收敛当且仅当它的实部和虚部级数都绝对收敛.
	\end{theorem}
	\onslide<+->
	\begin{proof}
		必要性由前一定理的证明已经知道,
	\onslide<+->{%
		充分性由 $|z_n|\le|x_n|+|y_n|$ 加上正项级数可重排得到.\qedhere
	}
	\end{proof}
\end{frame}


\begin{frame}{绝对收敛和条件收敛的判定}
	\vspace{\baselineskip}
	\begin{center}
		\renewcommand\arraystretch{1.4}
		\begin{tabular}{cccc} 
			&\alertn{发散}&\alertn{条件收敛}&\alertn{绝对收敛}\\
			\emphn{发散}&发散&发散&发散\\ 
			\emphn{条件收敛}&发散&条件收敛&条件收敛\\ 
			\emphn{绝对收敛}&发散&条件收敛&绝对收敛
		\end{tabular}
		\begin{tikzpicture}[overlay,yshift=14.6mm,fourth]
			\draw[cstcurve] (-7.21,-.63)--(-.12,-.63);
			\draw[cstcurve] (-5.2,0)--(-5.2,-2.67);
		\end{tikzpicture}
	\end{center}
	\begin{tikzpicture}[overlay]
		\draw[decorate,decoration={brace,amplitude=8},thick,main] (3.6,0.6)--(3.6,2.6);
		\draw[decorate,decoration={brace,amplitude=8},thick,second] (5.7,3.3)--(10.7,3.3);
		\draw (8.2,3.8) node[second] {实部级数}
		(3.0,1.6) node[align=center,main] {虚\\部\\级\\数};
	\end{tikzpicture}
	\bigdel
	\onslide<+->
	\begin{thinking}
		什么时候 $\abs{\suml_{n=1}^\infty z_n}=\suml_{n=1}^\infty|z_n|$?
		\onslide<+->{\alertn{当且仅当非零的 $z_n$ 的辐角全都相同时成立.}}
	\end{thinking}
\end{frame}


\begin{frame}{典型例题: 判断级数的敛散性}
	\onslide<+->
	\begin{example}
		级数 $\displaystyle\sum_{n=1}^\infty\frac{1+\ii^n}n$ 发散、条件收敛、还是绝对收敛?
	\end{example}

	\onslide<+->
	\begin{solution}
		由于实部级数
	\[
		\sum_{n=1}^\infty x_n=
		1+\frac13+\frac24+\frac15+\frac17+\frac28+\cdots>\sum_{n=1}^\infty\frac1{2n-1}
	\]
		发散, 所以该级数发散.
	\end{solution}

	\onslide<+->
	它的虚部级数是一个交错级数, 从而是条件收敛的.
\end{frame}


\begin{frame}{典型例题: 复数项级数敛散性}
	\onslide<+->
	\begin{example}
		级数 $\displaystyle\sum_{n=1}^\infty\dfrac{i^n}n$ 发散、条件收敛、还是绝对收敛?
	\end{example}

	\onslide<+->
	\begin{solution}
		因为它的实部和虚部级数
	\[
		\sum_{n=1}^\infty x_n=-\half +\frac14+\frac16-\frac18+\cdots
	\]
	\onslide<+->{
	\[
		\sum_{n=1}^\infty y_n=1-\frac13+\frac15-\frac17+\cdots
	\]
		均条件收敛,
	}\onslide<+->{所以原级数条件收敛.}
	\end{solution}
\end{frame}


\begin{frame}{典型例题: 判断级数的敛散性}
	\onslide<+->
	\begin{exercise}
		级数 $\displaystyle\sum_{n=1}^\infty\left[\frac{(-1)^n}n+\frac i{2^n}\right]$ 发散、条件收敛、还是绝对收敛?
	\end{exercise}

	\onslide<+->
	\begin{answer}
		实部级数条件收敛, 虚部级数绝对收敛, 所以该级数条件收敛.
	\end{answer}
\end{frame}


\begin{frame}{级数敛散性判别法}
	由正项级数的判别法可以得到:
	\onslide<+->
	\begin{enumerate}
		\item \alert{达朗贝尔判别法(比值法)}: $\lambda=\displaystyle\lim_{n\to\infty}\abs{\frac{z_{n+1}}{z_n}}$ (假设存在);
		\item 柯西判别法(根式法): $\lambda=\displaystyle\lim_{n\to\infty}\sqrt[n]{\abs{z_n}}$ (假设存在);
		\item 柯西-阿达马判别法: $\lambda=\displaystyle\ov{\lim_{n\to\infty}}\sqrt[n]{\abs{z_n}}$ (子数列中极限的最大值).
	\end{enumerate}

	\begin{itemize}
		\item 当 $\lambda<1$ 时, $\suml_{n=0}^\infty z_n$ 绝对收敛.
		\item 当 $\lambda>1$ 时, $\suml_{n=0}^\infty z_n$ 发散.
		\item 当 $\lambda=1$ 时, 无法使用该方法判断敛散性.
	\end{itemize}
	\onslide<+->
	其证明是通过将该级数与相应的等比级数做比较得到的.
\end{frame}


\begin{frame}{典型例题: 判断级数的敛散性}
	\onslide<+->
	\begin{example}
		级数 $\displaystyle\sum_{n=0}^\infty\frac{(8\ii)^n}{n!}$ 发散、条件收敛、还是绝对收敛?
	\end{example}

	\onslide<+->
	\begin{solution}
		因为 $\displaystyle\lim_{n\to\infty}\abs{\frac{z_{n+1}}{z_n}}=\lim_{n\to\infty}\dfrac{8}{n+1}=0$, 所以该级数绝对收敛.
	\end{solution}

	\onslide<+->
	实际上, 它的实部和虚部级数分别为
	\[1-\frac{8^2}{2!}+\frac{8^4}{4!}-\cdots=\cos 8,\quad
	8-\frac{8^3}{3!}+\frac{8^5}{5!}-\cdots=\sin 8,
	\]
	\onslide<+->
	因此
	\[\sum_{n=0}^\infty\frac{(8\ii)^n}{n!}=\cos 8+\ii\sin 8=\ee^{8\ii}.
	\]
\end{frame}


