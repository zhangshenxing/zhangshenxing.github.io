\section{柯西积分公式}

\subsection{柯西积分公式}

\begin{frame}{柯西积分公式}
	\onslide<+->
	柯西-古萨定理是解析函数理论的基础, 但在很多情形下它由柯西积分公式表现.
	\onslide<+->
	\begin{theorem*}[][柯西积分公式]
		设
		\begin{itemize}[<*>]
			\item 函数 $f(z)$ 在(复合)闭路 $C$ 及其内部(围成的区域) $D$ 解析,
			\item $z_0\in D$,
		\end{itemize}
		\onslide<+->{%
			则
			\[
				f(z_0)=\frac1{2\pi\ii}\oint_C\frac{f(z)}{z-z_0}\d z.
			\]
		}
	\end{theorem*}
	\onslide<+->
	若 $z_0\notin \ov D$, 由柯西-古萨定理, 右侧的积分是 $0$.
\end{frame}


\begin{frame}{柯西积分公式: 注记}
	\onslide<+->
	解析函数可以用一个积分
	\[
		f(z)=\frac1{2\pi\ii}\oint_C\frac{f(\zeta)}{\zeta-z}\d\zeta,\quad z\in D
	\]
	来表示, 这是研究解析函数理论的强有力工具.

	\onslide<+->
	解析函数在闭路 $C$ 内部的取值完全由它在 $C$ 上的值所确定. 这也是解析函数的特征之一.
	\onslide<+->
	特别地, 解析函数在圆心处的值等于它在圆周上的平均值.
	\onslide<+->
	设 $z=z_0+R\ee^{\ii\theta}$, 则 $\d z=\ii R\ee^{\ii\theta}\d\theta$,
	\onslide<+->
	\[
		f(z_0)=\frac1{2\pi\ii}\oint_C\frac{f(z)}{z-z_0}\d z
		=\frac1{2\pi}\int_0^{2\pi}f(z_0+R\ee^{\ii\theta})\d\theta.
	\]
\end{frame}


\begin{frame}{柯西积分公式证明\noexer}
	\onslide<+->
	\begin{proof}
		由连续性可知, $\forall\varepsilon>0,\exists\delta>0$ 使得当 $\abs{z-z_0}\le\delta$ 时, $z\in D$ 且 $\abs{f(z)-f(z_0)}<\varepsilon$.
		\onslide<+->{%
			设 $\Gamma:\abs{z-z_0}=\delta$,
		}\onslide<+->{%
			则
			\begin{align*}
				\biggabs{\oint_C\frac{f(z)}{z-z_0}\d z-2\pi\ii f(z_0)}
				&\xeq{\text{复合闭路定理}}\biggabs{\oint_\Gamma\frac{f(z)}{z-z_0}\d z-2\pi\ii f(z_0)}\\
				&\visible<+->{=\biggabs{\oint_\Gamma\frac{f(z)}{z-z_0}\d z-\oint_\Gamma\frac{f(z_0)}{z-z_0}\d z}
				=\biggabs{\oint_\Gamma\frac{f(z)-f(z_0)}{z-z_0}\d z}}\\
				&\visible<+->{\le\frac\varepsilon \delta\cdot 2\pi \delta
				=2\pi \varepsilon.}
			\end{align*}
		}\onslide<+->{%
			由 $\varepsilon$ 的任意性可知 
			\[
				\doint_C\frac{f(z)}{z-z_0}\d z=2\pi\ii f(z_0).\qedhere
			\]
		}
	\end{proof}
\end{frame}


\begin{frame}{典型例题: 柯西积分公式的应用}
	\onslide<+->
	可以看出, 当被积函数分子解析而分母形如 $z-z_0$ 时, 绕闭路的积分可以使用柯西积分公式计算.
	\onslide<+->
	\begin{example}[nearnext]
		求 $\doint_{\abs{z}=4}\frac{\sin z}z\d z$.
	\end{example}
	\onslide<+->
	\begin{solution}[nearprev]
		函数 $\sin z$ 处处解析.
		\onslide<+->{%
			取 $f(z)=\sin z, z_0=0$ 并应用柯西积分公式得
			\[
				\oint_{\abs{z}=4}\frac{\sin z}z\d z
				=2\pi\ii \sin z|_{z=0}=0.
			\]
		}\bigdel
	\end{solution}
\end{frame}


\begin{frame}{典型例题: 柯西积分公式的应用}
	\onslide<+->
	\begin{example}[nearnext]
		求 $\doint_{\abs{z}=2}\frac{\ee^z}{z-1}\d z$.
	\end{example}
	\onslide<+->
	\begin{solution}[nearprev]
		由于函数 $\ee^z$ 处处解析,
		\onslide<+->{
			取 $f(z)=\ee^z, z_0=1$ 并应用柯西积分公式得
			\[
				\oint_{\abs{z}=2}\frac{\ee^z}{z-1}\d z
				=2\pi\ii \ee^z|_{z=1}=2\pi \ee\ii.
			\]
		}\bigdel
	\end{solution}
	\onslide<+->
	\begin{exercise}
		求 $\doint_{\abs{z}=2\pi}\frac{\cos z}{z-\pi}\d z=$\fillblankframe{$-2\pi\ii$}.
	\end{exercise}
\end{frame}


\begin{frame}{典型例题: 柯西积分公式的应用}
	\onslide<+->
	\begin{example}[nearnext]
		设 $f(z)=\doint_{\abs{\zeta}=\sqrt3}\frac{3\zeta^2+7\zeta+1}{\zeta-z}\d \zeta$, 求 $f'(1+\ii)$.
	\end{example}
	\onslide<+->
	\begin{solution}[nearprev]
		当 $\abs{z}<\sqrt3$ 时,由柯西积分公式得
		\[
			f(z)=\oint_{\abs{\zeta}=\sqrt3}\frac{3\zeta^2+7\zeta+1}{\zeta-z}\d \zeta
			\visible<+->{=2\pi\ii(3\zeta^2+7\zeta+1)|_{\zeta=z}=2\pi\ii(3z^2+7z+1).}
		\]
		\onslide<+->{%
			因此 $f'(z)=2\pi\ii(6z+7)$,
		}\onslide<+->{
			\[
				f'(1+\ii)=2\pi\ii(13+6\ii)=-12\pi+26\pi\ii.
			\]
		}\bigdel
	\end{solution}
	\onslide<+->
	注意当 $\abs{z}>\sqrt3$ 时, $f(z)\equiv0$.
\end{frame}


\begin{frame}{典型例题: 柯西积分公式的应用}
	\onslide<+->
	\begin{example}[nearnext]
		求 $\doint_{\abs{z}=3}\frac{\ee^z}{z(z^2-1)}\d z$.
	\end{example}
	\onslide<+->
	\begin{solution}[nearprev,sidepic,righthand width=4.1cm]
		被积函数的奇点为 $0,\pm1$.
		\onslide<+->{%
			设 $C_1$, $C_2$, $C_3$ 分别为绕 $0$, $1$, $-1$ 的分离圆周.
		}\onslide<+->{%
			由复合闭路定理和柯西积分公式
			\begin{align*}
				&\oint_{\abs{z}=3}\frac{\ee^z}{z(z^2-1)}\d z
				=\oint_{C_1+C_2+C_3}\frac{\ee^z}{z(z^2-1)}\d z\\
				\visible<+->{=}&\visible<.->{2\pi\ii\Bigg(\frac{\ee^z}{z^2-1}\bigg|_{z=0}+\frac{\ee^z}{z(z+1)}\bigg|_{z=1}+\frac{\ee^z}{z(z-1)}\bigg|_{z=-1}\Bigg)}\\
				\visible<+->{=}&\visible<.->{2\pi\ii\Bigg(-1+\frac \ee 2+\frac{\ee^{-1}}2\Bigg)=\pi\ii(\ee+\ee^{-1}-2).}
			\end{align*}
		}\bigdel
		\tcblower
		\begin{tikzpicture}[scale=.8]
			\draw[cstaxis] (-2.5,0)--(2.5,0);
			\draw[cstaxis] (0,-2.5)--(0,2.5);
			\draw[cstcurve,main] (0,0) circle (1.8);
			\draw[cstcurve,main,visible on=<3->] (1.2,0) circle(0.5);
			\draw[cstcurve,main,visible on=<3->] (-1.2,0) circle(0.5);
			\draw[cstcurve,main,visible on=<3->] (0,0) circle(0.5);
			\fill[cstdot,second,visible on=<2->] (0,0) circle;
			\fill[cstdot,second,visible on=<2->] (1.2,0) circle;
			\fill[cstdot,second,visible on=<2->] (-1.2,0) circle;
			\draw
				(1.6,1.6) node[main] {$C$}
				(-0.4,0.8) node[second,visible on=<3->] {$C_1$}
				(1.2,-.8) node[second,visible on=<3->] {$C_2$}
				(-1.2,-.8) node[second,visible on=<3->] {$C_3$};
		\end{tikzpicture}
	\end{solution}
\end{frame}


\subsection{高阶导数的柯西积分公式}

\begin{frame}{高阶导数的柯西积分公式}
	\onslide<+->
	解析函数可以由它的积分所表示.
	\onslide<+->
	不仅如此, 通过积分表示, 还可以说明\alert{解析函数是任意阶可导的}.
	\onslide<+->
	\begin{theorem*}[][柯西积分公式]
		设函数 $f(z)$ 在闭路或复合闭路 $C$ 及其内部 $D$ 解析, 则对任意 $z_0\in D$,
	\[
		f^{(n)}(z_0)=\frac{n!}{2\pi\ii}\oint_C\frac{f(z)}{(z-z_0)^{n+1}}\d z.
	\]
	\end{theorem*}
	\onslide<+->
	假如 $f(z)$ 有泰勒展开
	\[
		f(z)=f(z_0)+f'(z_0)(z-z_0)+\cdots+\frac{f^{(n)}(z_0)}{n!}(z-z_0)^n+\cdots
	\]
	\onslide<+->
	那么由 $\doint_C \frac{\d z}{(z-z_0)^n}$ 的性质可知上述公式右侧应当为 $f^{(n)}(z_0)$.
\end{frame}


\begin{frame}{高阶导数的柯西积分公式证明\noexer}
	\onslide<+->
	\begin{solution}[][证明]
		先证明 $n=1$ 的情形.
		\onslide<+->{%
			设 $\delta$ 为 $z_0$ 到 $C$ 的最短距离.
		}\onslide<+->{%
			当 $\abs{h}<\delta$ 时, $z_0+h\in D$.
		}\onslide<+->{%
			由柯西积分公式,
			\[
				f(z_0)=\frac1{2\pi\ii}\oint_C\frac{f(z)}{z-z_0}\d z,\quad 
				f(z_0+h)=\frac1{2\pi\ii}\oint_C\frac{f(z)}{z-z_0-h}\d z.
			\]
		}\onslide<+->{%
			两式相减得到
			\[
				\frac{f(z_0+h)-f(z_0)}h=\frac1{2\pi\ii}\alertn{\oint_C\frac{f(z)}{(z-z_0)(z-z_0-h)}\d z}.
			\]
		}\onslide<+->{%
			当 $h\to 0$ 时, 左边的极限是 $f'(z_0)$.
		}\onslide<+->{%
			因此我们只需要证明右边的极限等于 
			\[
				\frac1{2\pi\ii}\alertn{\oint_C\frac{f(z)}{(z-z_0)^2}\d z}.
			\]
		}\bigdel
	\end{solution}
\end{frame}


\begin{frame}{高阶导数的柯西积分公式证明\noexer}
	\onslide<+->
	\begin{proof}[indent][]
		\[
			\text{二者之差}=\frac1{2\pi\ii}\oint_C\frac{h f(z)}{(z-z_0)^2(z-z_0-h)}\d z.
		\]
		\onslide<+->{%
			由于 $f(z)$ 在 $C$ 上连续, 故存在 $M$ 使得 $\abs{f(z)}\le M$.
		}\onslide<+->{%
			注意到 $z\in C$, $\abs{z-z_0}\ge \delta$, $\abs{z-z_0-h}\ge\delta-\abs{h}$.
		}\onslide<+->{%
			由长大不等式,
			\[
				\biggabs{\oint_C\frac{h f(z)}{(z-z_0)^2(z-z_0-h)}\d z}\le\frac{M\abs{h}}{\delta^2(\delta-\abs{h})}\cdot L,
			\]
		其中 $L$ 是闭路 $C$ 的长度.
		}\onslide<+->{%
			当 $h\to0$ 时, 它的极限为 $0$, 因此 $n=1$ 情形得证.%
		}

		\onslide<+->{%
			对于一般的 $n$, 我们通过归纳法将 $f^{(n)}(z_0)$ 和 $f^{(n)}(z_0+h)$ 表达为积分形式.
		}\onslide<+->{%
			比较 $\dfrac{f^{(n)}(z_0+h)-f^{(n)}(z_0)}h$ 与积分公式右侧之差, 并利用长大不等式证明 $h\to 0$ 时, 差趋于零.
		}\onslide<+->{%
			具体过程见教材.\qedhere
		}
	\end{proof}
\end{frame}


\begin{frame}{典型例题: 使用高阶导数的柯西积分公式计算积分}
	\onslide<+->
	\alert{柯西积分公式不是用来计算高阶导数的, 而是用高阶导数来计算积分的.}
\onslide<+->
	\begin{example}[nearnext]
		求 $\doint_{\abs{z}=2}\frac{\cos(\pi z)}{(z-1)^5}\d z.$
	\end{example}
	\onslide<+->
	\begin{solution}[nearprev]
		由于 $\cos(\pi z)$ 处处解析,
		\onslide<+->{%
			因此由柯西积分公式,
			\[
				\oint_{\abs{z}=2}\frac{\cos(\pi z)}{(z-1)^5}\d z
				=\frac{2\pi\ii}{4!}\cos(\pi z)^{(4)}\big|_{z=1}
				\visible<+->{=\frac{2\pi\ii}{24}\cdot \pi^4\cos \pi=-\frac{\pi^5 \ii}{12}.}
			\]
		}\bigdel
	\end{solution}
\end{frame}


\begin{frame}{典型例题: 使用高阶导数的柯西积分公式计算积分}\small
	\onslide<+->
	\begin{example}[near]
		求 $\doint_{\abs{z}=2}\frac{\ee^z}{(z^2+1)^2}\d z.$
	\end{example}
	\onslide<+->
	\begin{solution}[near]
		被积函数在 $\abs{z}<2$ 的奇点为 $z=\pm \ii$.
		\onslide<+->{%
			取 $C_1,C_2$ 为以 $\ii,-\ii $ 为圆心的分离圆周.
		}\onslide<+->{%
			\[
				 \oint_{C_1}\frac{\ee^z}{(z^2+1)^2}\d z
				=\frac{2\pi\ii}{1}\biggl(\frac{\ee^z}{(z+\ii)^2}\biggr)'\Big|_{z=\ii}
				\visible<+->{=2\pi\ii\biggl(\frac{\ee^z}{(z+\ii)^2}-\frac{2\ee^z}{(z+\ii)^3}\biggr)\Big|_{z=\ii}
				=\frac{(1-\ii )\ee^\ii\pi}2.}
	\]
		}\onslide<+->{%
			类似地, $\doint_{C_2}\frac{\ee^z}{(z^2+1)^2}\d z=\frac{-(1+\ii)\ee^{-\ii }\pi}2$.
		}\onslide<+->{%
			故
			\begin{align*}
				 \oint_{\abs{z}=2}\frac{\ee^z}{(z^2+1)^2}\d z
			 &=\biggl(\oint_{C_1}+\oint_{C_2}\biggr)\frac{\ee^z}{(z^2+1)^2}\d z\\
			 &=\frac{(1-\ii )\ee^\ii\pi}2+\frac{-(1+\ii)\ee^{-\ii }\pi}2
			  =\pi\ii(\sin1-\cos1).
			\end{align*}
		}\bigdel
	\end{solution}
\end{frame}


\begin{frame}{典型例题: 使用高阶导数的柯西积分公式计算积分}
	\onslide<+->
	\begin{example}[nearnext]
		求 $\doint_{\abs{z}=1}z^n\ee^z\d z$, 其中 $n$ 是整数.
	\end{example}
	\onslide<+->
	\begin{solution}[nearprev]
		\begin{enumerate}
			\item 当 $n\ge 0$ 时, $z^n\ee^z$ 处处解析.
			\onslide<+->{%
				由柯西-古萨定理, 
				\[
					\oint_{\abs{z}=1}z^n\ee^z\d z=0.
				\]
			}
			\item 当 $n\le-1$ 时, $\ee^z$ 处处解析.
			\onslide<+->{%
				由柯西积分公式,
				\[
					\oint_{\abs{z}=1}z^n\ee^z\d z
					=\frac{2\pi\ii}{(-n-1)!}(\ee^z)^{(-n-1)}\big|_{z=0}
					=\frac{2\pi\ii}{(-n-1)!}.
				\]
			}
		\end{enumerate}
		\bigdel\meddel
	\end{solution}
\end{frame}


\begin{frame}{典型例题: 使用高阶导数的柯西积分公式计算积分}\small
	\beqskip{0pt}
	\onslide<+->
	\begin{example}[near]
		求 $\doint_{\abs{z-3}=2}\frac1{(z-2)^2z^3}\d z$ 和 $\doint_{\abs{z-1}=3}\frac1{(z-2)^2z^3}\d z$.
	\end{example}
	\onslide<+->
	\begin{solution}[near]
		\begin{enumerate}
			\item $\dfrac1{(z-2)^2z^3}$ 在 $\abs{z-3}<2$ 的奇点为 $z=2$.
			\onslide<+->{%
				由柯西积分公式,
				\[
					\oint_{\abs{z-3}=2}\frac1{(z-2)^2z^3}\d z
					=\frac{2\pi\ii}{1!}\Bigl(\frac1{z^3}\Bigr)'\bigg|_{z=2}
					=-\frac{3\pi\ii}8.
				\]
			}
			\item $\dfrac1{(z-2)^2z^3}$ 在 $\abs{z-1}<3$ 的奇点为 $z=0,2$.
			\onslide<+->{%
				取 $C_1,C_2$ 为以 $0,2$ 为圆心的分离圆周.
			}\onslide<+->{
				\begin{align*}
					&\oint_{\abs{z-1}=3}\frac1{(z-2)^2z^3}\d z=\oint_{C_1}\frac1{(z-2)^2z^3}\d z+\oint_{C_2}\frac1{(z-2)^2z^3}\d z\\
					\visible<+->{=}&\visible<.->{\frac{2\pi\ii}{2!}\biggl(\frac1{(z-2)^2}\biggr)''\Big|_{z=0}+\frac{2\pi\ii}{1!}\Bigl(\frac1{z^3}\Bigr)'\Big|_{z=2}=0.}
				\end{align*}
			}
		\end{enumerate}
		\bigdel
	\end{solution}
	\endgroup
\end{frame}


\begin{frame}{莫累拉定理}
	\beqskip{2pt}
	\onslide<+->
	\begin{exercise}
		$\doint_{\abs{z-2\ii}=3}\frac1{z^2(z-\ii )}\d z=$\fillblankframe{$0$}.
	\end{exercise}
	\onslide<+->
	\begin{example}[nearnext][莫累拉定理]
		设 $f(z)$ 在单连通区域 $D$ 内连续, 且对于 $D$ 中任意闭路 $C$ 都有 $\doint_Cf(z)\d z=0$, 则 $f(z)$ 在 $D$ 内解析.
	\end{example}
	\onslide<+->
	\begin{proof}[nearprev]
		由题设可知 $f(z)$ 的积分与路径无关.
		\onslide<+->{%
			固定 $z_0\in D$, 则
			\[
				F(z)=\int_{z_0}^zf(z)\d z
			\]
			定义了 $D$ 内的一个函数.
		}\onslide<+->{%
			类似于原函数的证明可知 $F'(z)=f(z)$.
		}\onslide<+->{%
			故 $f(z)$ 作为解析函数 $F(z)$ 的导数也是解析的.\qedhere
		}
	\end{proof}
	\endgroup
\end{frame}


\begin{frame}{解析函数与实函数的差异}
	\onslide<+->
	高阶柯西积分公式说明解析函数的导数与实函数的导数有何不同?
	\onslide<+->
	高阶柯西积分公式说明, 函数 $f(z)$ 只要在区域 $D$ 中处处可导, 它就一定无限次可导, 并且各阶导数仍然在 $D$ 中解析.
	\onslide<+->
	\alert{这一点与实变量函数有本质的区别.}

	\onslide<+->
	同时我们也可以看出, 若一个二元实函数 $u(x,y)$ 是一个解析函数的实部或虚部, 则 $u$ 也是具有任意阶偏导数.
	\onslide<+->
	这便引出了调和函数的概念.
\end{frame}


