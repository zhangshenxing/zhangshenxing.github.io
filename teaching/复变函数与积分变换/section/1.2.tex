\section{复数的三角与指数形式}

\subsection{复数的模和辐角}
\begin{frame}{复数的极坐标形式}
	\onslide<+->
	由平面的极坐标表示, 我们可以得到复数的另一种表示方式.
	\onslide<+->
	以 $0$ 为极点, 正实轴为极轴, 逆时针为极角方向可以自然定义出复平面上的极坐标系.
	\onslide<+->
	\begin{center}
		\begin{tikzpicture}
			\draw[cstaxis] (-0.4,0)--(2.5,0);
			\draw[cstaxis] (0,-0.4)--(0,2);
			\draw[cstdash] (1.6,0)--(1.6,1.2);
			\draw[cstdash] (0,1.2)--(1.6,1.2);
			\fill[cstdot,dcolora] (1.6,1.2) circle;
			\draw[decorate,decoration={brace,amplitude=5},dcolora,visible on =<8->] (1.6,0)--(0,0);
			\draw[decorate,decoration={brace,amplitude=5},dcolorb,visible on =<8->] (0,0)--(0,1.2);
			\draw
			(1.6,1.4) node[dcolora] {$z=x+yi$}
			(-0.2,-0.2) node {$0$}
			(0.7,0.8) node[dcolorc,visible on =<5->] {$r$}
			(6,1) node[dcolorc,visible on =<5->] {$r=\sqrt{x^2+y^2}$}
			(6,0.3) node[dcolorc,visible on =<6->] {$\theta=\arctan\dfrac yx$ 或 $\arctan\dfrac yx\pm\pi$}
			(1,-0.3) node[dcolora,visible on =<8->] {$x=r\cos\theta$}
			(-1.2,0.6) node[dcolorb,visible on =<8->] {$y=r\sin\theta$};
			\coordinate (A) at (3,0);
			\coordinate (B) at (0,0);
			\coordinate (C) at (1.6,1.2);
			\draw[cstcurve,dcolorc,cstarrowto,visible on =<5->] (B)--(C);
			\draw[dcolorc,thick,visible on =<6->,cstarrowto] pic [draw, "$\theta$", angle eccentricity=1.3, angle radius=0.8cm] {angle};
		\end{tikzpicture}
	\end{center}

	\onslide<+->
	\begin{definition}
		\begin{itemize}
			\item 称 $r$ 为 $z$ 的\emph{模}, 记为 \emph{$|z|=r$}.
			\item 称 $\theta$ 为 $z$ 的\emph{辐角}, 记为 \emph{$\Arg z=\theta$}.
			\onslide<+->{\alert{$0$ 的辐角没有意义}.}
		\end{itemize}
	\end{definition}
\end{frame}


\begin{frame}{主辐角}
	\onslide<+->
	任意 $z\neq 0$ 的辐角有无穷多个.
	\onslide<+->
	我们固定选择其中位于 $(-\pi,\pi]$ 的那个, 并称之为\emph{主辐角}或辐角主值, 记作 $\emph{\arg z}$.
	\begin{tikzpicture}[overlay,xshift=5cm,yshift=-3cm]
		\draw[cstaxis,visible on=<3->](-2.5,0)->(2.5,0); 
		\draw[cstaxis,visible on=<3->](0,-2)->(0,2);
		\draw[cstaxis,visible on=<3>,dcolorb,cstarrow1from] (-1,0) arc(180:-180:1);
		\filldraw[cstdote,visible on=<3>,draw=dcolora] (-1,-0.05) circle;
		\fill[cstdot,visible on=<4->,dcolora] (1,0) circle;
		\fill[cstdot,visible on=<4->,dcolora] (0.8,0.8) circle;
		\fill[cstdot,visible on=<4->,dcolora] (1.0,-0.8) circle;
		\fill[cstdot,visible on=<5->,dcolorc] (-1.0,0.6) circle;
		\fill[cstdot,visible on=<5->,dcolorc] (-0.7,0) circle;
		\fill[cstdot,visible on=<6->,dcolorb] (-0.9,-0.5) circle;
		\fill[cstdot,visible on=<7->] (0,0.5) circle;
		\fill[cstdot,visible on=<7->] (0,-0.5) circle;
		\draw
		(1.1,0.2) node[dcolora,visible on=<4->] {$0$}
		(1.0,1.1) node[dcolora,visible on=<4->] {$\arctan\dfrac yx$}
		(1.1,-1.2) node[dcolora,visible on=<4->] {$\arctan\dfrac yx$}
		(-1.2,1.2) node[dcolorc,visible on=<5->] {$\arctan\dfrac yx\colorbox{yellow}{$+\pi$}$}
		(-0.6,-0.2) node[dcolorc,visible on=<5->] {$\pi$}
		(-1.2,-1.1) node[dcolorb,visible on=<6->] {$\arctan\dfrac yx\colorbox{yellow}{$-\pi$}$}
		(-0.2,0.5) node[black,visible on=<7->] {$\dfrac\pi2$}
		(0.4,-0.5) node[black,visible on=<7->] {$-\dfrac\pi2$};
	\end{tikzpicture}
	\onslide<+->
	\begin{flalign*}
		&\colorbox{yellow}{$\displaystyle\arg z=\begin{cases}
		\visible<4->{\alert{\arctan\dfrac yx,}}&\visible<4->{\alert{x>0;}}\vspace{1ex}\\
		\visible<5->{\color{dcolorc}{\arctan\dfrac yx+\pi,}}&\visible<5->{\color{dcolorc}{x<0,y\ge0;}}\vspace{1ex}\\
		\visible<6->{\emph{\arctan\dfrac yx-\pi,}}&\visible<6->{\emph{x<0,y<0;}}\\
		\visible<7->{\dfrac\pi2,}&\visible<7->{x=0,y>0;}\\
		\visible<7->{-\dfrac\pi2,}&\visible<7->{x=0,y<0.}
		\end{cases}$}&
	\end{flalign*}
	\onslide<8->
	那么 \emph{$\Arg z=\arg z+2k\pi, k\in\BZ$}.
\end{frame}


\begin{frame}{复数模的性质}
	\onslide<+->
	复数的模满足如下性质:
	\onslide<+->
	\begin{block}{模的性质汇总}
	\begin{itemize}
		\item $z\ov z=|z|^2=|\ov z|^2$;
		\item $\abs{\Re z},\abs{\Im z}\le |z|\le\abs{\Re z}+\abs{\Im z}$;
		\item $\big||z_1|-|z_2|\big|\le|z_1\pm z_2|\le|z_1|+|z_2|$;
		\item $|z_1+z_2+\cdots+z_n|\le|z_1|+|z_2|+\cdots+|z_n|$.
	\end{itemize}
	\end{block}

	\begin{center}
		\begin{figure}[h]
			\begin{subfigure}{0.35\textwidth}
			\centering\onslide<4->
			\begin{tikzpicture}[framed]
				\draw[cstaxis,visible on=<4->] (-1,0)--(3,0);
				\draw[cstaxis,visible on=<4->] (0,-1)--(0,2);
				\draw[cstcurve,dcolorc,visible on=<4->] (1.6,0)--(0,0);
				\draw[cstcurve,dcolorb,visible on=<4->] (1.6,0)--(1.6,1.2);
				\draw[cstcurve,dcolora,visible on=<4->] (1.6,1.2)--(0,0);
				\fill[cstdot,dcolora,visible on=<4->] (1.6,1.2) circle;
				\draw[thick,visible on=<4->] (1.6,0.2)--(1.4,0.2)--(1.4,0);
				\draw[decorate,decoration={brace,amplitude=5},dcolorc,visible on=<4->] (1.6,0)--(0,0);
				\draw[decorate,decoration={brace,amplitude=5},dcolorb,visible on=<4->] (1.6,1.2)--(1.6,0);
				\draw 
					(0.6,0.9) node[dcolora,visible on=<4->] {$|z|$}
					(0.8,-0.4) node[dcolorc,visible on=<4->] {$\abs{\Re z}$}
					(2.3,0.6) node[dcolorb,visible on=<4->] {$\abs{\Im z}$};
			\end{tikzpicture}
			\end{subfigure}
			\begin{subfigure}{0.55\textwidth}
			\centering\onslide<5->
			\begin{tikzpicture}[framed]
				\draw[cstaxis,visible on=<5->] (-2,0)--(3,0);
				\draw[cstaxis,visible on=<5->] (0,-1.6)--(0,1.3);
				\draw[cstcurve,cstarrowto,dcolora,visible on=<5->] (0,0)--(1.6,-0.4);
				\draw[decorate,decoration={brace,amplitude=5},dcolora,visible on=<5->] (1.6,-0.4)--(0,0);
				\draw[cstcurve,cstarrowto,dcolorb,visible on=<5->] (0,0)--(0.8,1.2);
				\draw[decorate,decoration={brace,amplitude=5,aspect=0.35},dcolorb,visible on=<5->] (2.4,0.8)--(1.6,-0.4);
				\draw[cstcurve,cstarrowto,dcolorc,visible on=<5->] (0,0)--(2.4,0.8);
				\draw[decorate,decoration={brace,amplitude=5,aspect=0.65},dcolorc,visible on=<5->] (0,0)--(2.4,0.8);
				\draw[cstcurve,cstarrowto,dcolorc,visible on=<5->] (0,0)--(0.8,-1.6);
				\draw[decorate,decoration={brace,amplitude=5,aspect=0.65},dcolorc,visible on=<5->] (0.8,-1.6)--(0,0);
				\draw[dcolorc,cstarrowto,visible on=<5->] (0.116,-0.636)--(-0.7,-0.4);
				\draw[cstdash,dcolorb,visible on=<5->] (2.4,0.8)--(0.8,-1.6);
				\draw[cstdash,dcolora,visible on=<5->] (0.8,-1.6)--(-0.8,-1.2);
				\draw[cstdash,dcolorb,visible on=<5->] (-0.8,-1.2)--(0,0);
				\draw 
					(1.8,-0.6) node[dcolora,visible on=<5->] {\small $z_1$}
					(0.6,1.3) node[dcolorb,visible on=<5->] {\small $z_2$}
					(0.75,-0.55) node[dcolora,visible on=<5->] {\small $|z_1|$}
					(2.6,0.25) node[dcolorb,visible on=<5->] {\small $|z_2|$}
					(1.4,0.8) node[dcolorc,rotate=20,visible on=<5->] {\small $|z_1+z_2|$}
					(-1.5,-0.4) node[dcolorc,visible on=<5->] {\small $|z_1-z_2|$};
			\end{tikzpicture}
			\end{subfigure}
		\end{figure}
	\end{center}
\end{frame}


\begin{frame}{例题:共轭复数解决模的等式}
	\onslide<+->
	\begin{example}
		证明 \enumnum1 $|z_1z_2|=|z_1\ov{z_2}|=|z_1|\cdot|z_2|$;

		\enumnum2 $|z_1+z_2|^2=|z_1|^2+|z_2|^2+2\Re(z_1\ov{z_2})$.
	\end{example}

	\onslide<+->
	\begin{proof}
		\enumnum1 因为
		\[|z_1z_2|^2=z_1z_2\cdot\ov{z_1}\ov{z_2}
		=z_1z_2\ov{z_1}\ov{z_2}=|z_1|^2\cdot|z_2|^2,\]
		\onslide<+->{所以 $|z_1z_2|=|z_1|\cdot|z_2|$.
		}\onslide<+->{因此 $|z_1\ov{z_2}|=|z_1|\cdot|\ov{z_2}|=|z_1|\cdot|z_2|$.}

		\onslide<+->{\enumnum2 因为
		\begin{align*}
			\text{左边}&=(z_1+z_2)(\ov{z_1}+\ov{z_2})
			\visible<+->{=z_1\ov{z_1}+z_2\ov{z_2}+z_1\ov{z_2}+\ov{z_1}z_2,}\\
			\text{右边}&=z_1\ov{z_1}+z_2\ov{z_2}+z_1\ov{z_2}+\ov{z_1\ov{z_2}},
		\end{align*}}
		而 $\ov{z_1\ov{z_2}}=\ov{z_1}z_2$, 所以两侧相等.
	\end{proof}
\end{frame}


\subsection{复数的三角形式和指数形式}
\begin{frame}{复数的三角形式和指数形式}
	\onslide<+->
	由 $x=r\cos\theta,y=r\sin\theta$ 可得复数的\emph{三角形式}
	\[\alert{z=r(\cos\theta+i\sin\theta)}.\]
	\onslide<+->
	定义 $e^{i\theta}=\exp(i\theta):=\cos\theta+i\sin\theta$ (欧拉恒等式),
	\onslide<+->
	则我们得到复数的\emph{指数形式}
	\[\alert{z=re^{i\theta}=r\exp(i\theta)}.\]
	\onslide<+->
	这两种形式的等价的, 指数形式可以认为是三角形式的一种缩写方式.

	\onslide<+->
	求复数的三角/指数形式的\emph{关键在于计算模和辐角}.
\end{frame}


\begin{frame}{典型例题: 求复数的三角/指数形式}
	\onslide<+->
	\begin{example}
		将 $z=-\sqrt{12}-2i$ 化成三角形式和指数形式.
	\end{example}

	\onslide<+->
	\begin{solution}
		$r=|z|=\sqrt{12+4}=4$.
		\onslide<+->{由于 $z$ 在第三象限,
		}\onslide<+->{因此
			\[\arg z=\arctan\frac{-2}{-\sqrt{12}}-\pi=\frac\pi6-\pi=-\frac{5\pi}6.\]
		}\onslide<+->{故
			\[z=4\left[\cos\left(-\frac{5\pi}6\right)+i\sin\left(-
		\frac{5\pi}6\right)\right]=4e^{-\frac{5\pi i}6}.\]}
	\end{solution}
\end{frame}


\begin{frame}{典型例题: 求复数的三角/指数形式}
	\onslide<+->
	\begin{example}
		将 $z=\sin\dfrac\pi5+i\cos\dfrac\pi5$ 化成三角形式和指数形式.
	\end{example}

	\onslide
	\begin{solution}
		\[|z|=1,\quad \onslide<+->{
			\arg z=\arctan\frac{\cos(\pi/5)}{\sin(\pi/5)}=\arctan\cot\frac\pi 5=\frac\pi2-\frac\pi5=\frac{3\pi}{10}.
		}\]
		\onslide<+->{因此 $z=\displaystyle\cos\frac{3\pi}{10}+i\sin\frac{3\pi}{10}=e^{\frac{3\pi i}{10}}$.}
	\end{solution}
	\onslide<+->
		\begin{proofblock}{另解}
		\[
		z=\sin\frac\pi5+i\cos\frac\pi5
		\visible<+->{=\cos\left(\frac\pi2-\frac\pi5\right)+i\sin\left(\frac\pi2-\frac\pi5\right)}
		\visible<+->{=\cos\frac{3\pi}{10}+i\sin\frac{3\pi}{10}=e^{\frac{3\pi i}{10}}.}
		\]
	\end{proofblock}
\end{frame}


\begin{frame}{典型例题: 求复数的三角/指数形式}
	\onslide<+->
	求复数的三角或指数形式时, 我们只需要任取一个辐角就可以了, 不要求必须是主辐角.

	\onslide<+->
	\begin{exercise}
		将 $z=\sqrt 3-3i$ 化成三角形式和指数形式.
	\end{exercise}

	\onslide<+->
	\begin{answer}
		$\displaystyle z=2\sqrt3\left(\cos\frac{-\pi}3+i\sin\frac{-\pi}3\right)
		=2\sqrt3e^{-\frac{\pi i}3}$, 写成 $\dfrac{5\pi}3$ 也可以.
	\end{answer}
\end{frame}


\begin{frame}{模为 $1$ 的复数}
	\onslide<+->
	两个模相等的复数之和的三角/指数形式形式较为简单.
	\onslide<+->
	\[e^{i\theta}+e^{i\varphi}=2\cos\frac{\theta-\varphi}2e^{\frac{\theta+\varphi}2i}.\]
	\onslide<+->
	\begin{center}
			\begin{tikzpicture}[framed]
			\draw[cstaxis] (-1.5,0)--(2.5,0);
			\draw[cstaxis] (0,-0.6)--(0,2.5);
			\draw[cstcurve,cstarrowto,dcolora] (0,0)--(-0.8,1.72);
			\draw[cstdash] (-0.8,1.72)--(1,2.32)--(1.8,0.6)--(0.5,1.16);
			\draw[thick] (0.7,1.074)--(0.786,1.274)--(0.586,1.36);
			\coordinate (A) at (2,0);
			\coordinate (B) at (0,0);
			\coordinate (C) at (1.8,0.6);
			\draw[thick,dcolorb] pic [draw, "$\varphi$", angle eccentricity=1.4, angle radius=0.7cm] {angle};
			\coordinate (A) at (1.8,0.6);
			\coordinate (C) at (1,2.32);
			\draw[cstcurve,dcolora,cstarrowto] (B)--(A);
			\draw[cstcurve,dcolorb,cstarrowto] (B)--(C);
			\draw[thick,dcolorb] pic [draw, "$\frac{\theta-\varphi}2$", angle eccentricity=1.7] {angle};
			\draw
			(-0.2,-0.3) node {$0$}
			(2.1,0.7) node[dcolora] {$e^{i\varphi}$}
			(-1.1,1.7) node[dcolora] {$e^{i\theta}$};
		\end{tikzpicture}
	\end{center}
	\onslide<+->
	\begin{example}
		如果 $|z|=1,\arg z=\theta$, 则 $z+1=2\cos\dfrac\theta2 e^{\frac{\theta i}2}$.
	\end{example}
\end{frame}
