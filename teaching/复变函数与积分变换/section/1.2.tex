\section{复数的三角形式与指数形式}


\subsection{复数的模和辐角}


\begin{frame}{复数的极坐标形式}
	\onslide<+->
	由平面的极坐标表示, 我们可以得到复数的另一种表示方式.
	\onslide<+->
	以 $0$ 为极点, 正实轴为极轴, 逆时针为极角方向可以定义出复平面上的极坐标系.
	\onslide<4->{
	\begin{definition}
		\begin{enumerate}
			\item 称 $r$ 为 $z$ 的\emph{模}, 记为 \emph{$\abs{z}=r$}.
			\item<5-> 称 $\theta$ 为 $z$ 的\emph{辐角}, 记为 \emph{$\Arg z=\theta$}.
			\onslide<6->{约定 \alert{$0$ 的辐角没有定义}.}
		\end{enumerate}
	\end{definition}}
	\onslide<2->
	\begin{twopart}[nearprev]{185pt}
		\begin{center}
			\begin{tikzpicture}[scale=.8]
				\coordinate [label=below left:$0$] (O) at (0,0);
				\coordinate [label=above:\textcolor{second}{$z=x+y\ii$}] (Z) at (3,2);
				\coordinate (X) at (3,0);
				\coordinate (Y) at (0,2);
				\draw[decorate,decoration={brace,amplitude=5},main,cstfill1] (X)--(O) node[midway,below=2mm] {$x$};
				\draw[decorate,decoration={brace,amplitude=5},main,cstfill1] (O)--(Y) node[midway,left=2mm] {$y$};
				\draw[third,thick,cstra] pic [cstfill3,draw=third, "$\theta$", angle eccentricity=1.3, angle radius=0.8cm] {angle=X--O--Z};
				\draw[cstaxis] (-.5,0)--(4,0);
				\draw[cstaxis] (0,-.5)--(0,3);
				\draw[cstcurve,third,cstra] (O)--(Z) node[midway,above,third] {$r$};
				\draw[cstdash] (X)--(Z)--(Y);
				\fill[cstdot,second] (Z) circle;
			\end{tikzpicture}
		\end{center}
		\tcblower
		\begin{center}
			\onslide<3->{
			\[
				x=r\cos\theta,\qquad y=r\sin\theta,
			\]
			\[
				r=\sqrt{x^2+y^2}.
			\]}
		\end{center}
	\end{twopart}
\end{frame}


\begin{frame}{辐角主值}
	\onslide<+->
	任意非零复数 $z$ 都有无穷多个辐角.
	\onslide<+->
	称其中位于 $(-\pi,\pi]$ 的那个辐角为\emph{辐角主值}或\emph{主辐角}, 记作 $\emphm{\arg z}$.
	\onslide<+->
	那么
	\[
		\emphm{\Arg z=\arg z+2k\pi, k\in\BZ}.
	\]
	\onslide<8->
	注意 \alert{$\arg \ov z=-\arg z$ 未必成立}, 当且仅当 $z$ 不是负实数和 $0$ 时该等式成立.
	\onslide<2->
	\begin{twopart}[near]{203pt}
		\smalldel
		\onslide<4->{
		\[
			\arg z=\begin{cases}
				\visible<4->{\emphn{\arctan\dfrac yx,}}&\visible<4->{\emphn{x>0;}}\medskip\\
				\visible<5->{\alertn{\arctan\dfrac yx+\pi,}}&\visible<5->{\alertn{x<0,y\ge0;}}\medskip\\
				\visible<6->{\color{third}{\arctan\dfrac yx-\pi,}}&\visible<6->{\color{third}{x<0,y<0;}}\\
				\visible<7->{\color{fourth}{\dfrac\pi2,}}&
				\visible<7->{\color{fourth}{x=0,y>0;}}\\
				\visible<7->{\color{fourth}{-\dfrac\pi2,}}&
				\visible<7->{\color{fourth}{x=0,y<0.}}
			\end{cases}
		\]}\bigdel
		\tcblower\smalldel
		\begin{center}
			\begin{tikzpicture}
				\draw[cstaxis](-2.5,0)->(2.5,0); 
				\draw[cstaxis](0,-2)->(0,2);
				\draw[cstaxis,main,cstwla] (-1.5,0) arc(180:-180:1.5);
				\filldraw[cstdote,draw=main] (-1.5,-.07) circle;
				\begin{scope}[visible on=<4->]
					\coordinate [label=above:\textcolor{main}{$0$}] (A) at (1.7,0);
					\fill[cstdot,main] (A) circle;
					\coordinate [label=above right:\textcolor{main}{$\arctan\dfrac yx$}] (B) at (.9,.9);
					\fill[cstdot,main] (B) circle;
					\coordinate [label=right:\textcolor{main}{$\arctan\dfrac yx$}] (C) at (1.4,-.9);
					\fill[cstdot,main] (C) circle;
				\end{scope}
				\begin{scope}[visible on=<5->]
					\coordinate [label=above left:\textcolor{second}{$\arctan\dfrac yx+\pi$}] (D) at (-1.1,.8);
					\fill[cstdot,second] (D) circle;
					\coordinate [label=below:\textcolor{second}{$\pi$}] (E) at (-.9,0);
					\fill[cstdot,second] (E) circle;
				\end{scope}
				\begin{scope}[visible on=<6->]
					\coordinate [label=left:\textcolor{third}{$\arctan\dfrac yx-\pi$}] (F) at (-1.6,-.7);
					\fill[cstdot,third] (F) circle;
				\end{scope}
				\begin{scope}[visible on=<7->]
					\coordinate [label=left:\textcolor{fourth}{$\dfrac\pi2$}] (G) at (0,.5);
					\fill[cstdot,fourth] (G) circle;
					\coordinate [label=right:\textcolor{fourth}{$-\dfrac\pi2$}] (H) at (0,-.6);
					\fill[cstdot,fourth] (H) circle;
				\end{scope}
			\end{tikzpicture}
		\end{center}\bigdel
	\end{twopart}
\end{frame}


\begin{frame}{复数模的性质}\small
	\onslide<+->
	复数的模满足如下性质
	\begin{twopart}{173pt}
		\begin{center}
			\begin{tikzpicture}[visible on=<3->,scale=.75]
				\draw[cstaxis] (-4,0)--(4.3,0);
				\draw[cstaxis] (0,-3)--(0,3);
				\coordinate (O) at (0,0);
				\coordinate (Z) at (-2.5,1.5);
				\coordinate (R) at (-2.5,0);
				\draw[decorate,decoration={brace,amplitude=5},main,cstfill1] (O)--(R) node[midway,below=2mm,main] {$\abs{\Re z}$};
				\draw[decorate,decoration={brace,amplitude=5},main,cstfill1] (R)--(Z) node[midway,left=2mm,main] {$\abs{\Im z}$};
				\draw[cstcurve,second] (O)--(Z) node[midway,above,second] {$\abs{z}$};
				\draw[cstcurve,main] (Z)--(R)--(O);
				\draw[thick] (R) ++(0,.3)--++(.3,0)--++(0,-.3);
		
				\begin{scope}[visible on=<4->]
					\coordinate [label=below right:\textcolor{main}{$z_1$}] (Z1) at (2.8,-.4);
					\coordinate (Z2) at (1.2,2);
					\coordinate [label=above right:\textcolor{main}{$z_1+z_2$}] (P) at ($(Z1)+(Z2)$);
					\coordinate [label=below:\textcolor{third}{$z_1-z_2$}] (M) at ($(Z1)-(Z2)$);
					\draw[decorate,decoration={brace,amplitude=5},main] (Z1)--(O) node[midway,below,sloped] {$\abs{z_1}$};
					\draw[decorate,decoration={brace,amplitude=5},main] (P)--(Z1) node[midway,below,sloped] {$\abs{z_2}$};
					\draw[decorate,decoration={brace,amplitude=5},second] (O)--(P) node[midway,above,sloped] {$\abs{z_1+z_2}$};
					\draw[decorate,decoration={brace,amplitude=5},main] (Z1)--(M) node[midway,below,sloped] {$\abs{z_2}$};
					\draw[decorate,decoration={brace,amplitude=5},third] (M)--(O) node[midway,below,sloped] {$\abs{z_1-z_2}$};
					\begin{scope}[cstcurve,cstra]
						\draw[main] (O)--(Z1);
						\draw[main] (Z1)--(P);
						\draw[second] (O)--(P);
						\draw[third] (O)--(M);
						\draw[main] (Z1)--(M);
					\end{scope}
				\end{scope}
				\begin{scope}[visible on=<5->]
					\coordinate (A) at (2.7,2.4);
					\draw[decorate,decoration={brace,amplitude=5},main] (A)--(P) node[midway,above,sloped] {$\abs{z_3}$};
					\draw[decorate,decoration={brace,amplitude=5},fourth] (O)--(A) node[midway,above=2mm,sloped] {$\abs{z_1+z_2+z_3}$};
					\begin{scope}[cstcurve,cstra]
						\draw[main] (P)--(A);
						\draw[fourth] (O)--(A);
					\end{scope}
				\end{scope}
			\end{tikzpicture}
		\end{center}
		\tcblower
		\begin{enumerate}\bf
			\item $z\ov z=\abs{z}^2=\abs{\ov z}^2$;
			\item $\abs{\Re z},\abs{\Im z}\le \abs{z}\le\abs{\Re z}+\abs{\Im z}$;
			\item $\bigabs{\abs{z_1}-\abs{z_2}}\le\abs{z_1\pm z_2}\le\abs{z_1}+\abs{z_2}$;
			\item $\abs{z_1+z_2+\cdots+z_n}\le\abs{z_1}+\abs{z_2}+\cdots+\abs{z_n}$.
		\end{enumerate}
	\end{twopart}
\end{frame}


\begin{frame}{例题:共轭复数解决模的等式}
	\beqskip{4pt}
	\onslide<1->{
	\begin{example}[nearnext]
		证明
		\begin{subexample}[2]
			\item $\abs{z_1z_2}=\abs{z_1\ov{z_2}}=\abs{z_1}\cdot\abs{z_2}$;
			\item $\abs{z_1+z_2}^2=\abs{z_1}^2+\abs{z_2}^2+2\Re(z_1\ov{z_2})$.
		\end{subexample}
	\end{example}}
	\onslide<+->
	\begin{proof}[nearprev]
		\begin{enumerate}
			\item 因为
			\[
				\abs{z_1z_2}^2=z_1z_2\cdot\ov{z_1}\ov{z_2}
				\onslide<+->{=z_1\ov{z_1}\cdot z_2\ov{z_2}}
				\onslide<+->{=\abs{z_1}^2\cdot\abs{z_2}^2,}
			\]
			\onslide<+->{%
				所以 $\abs{z_1z_2}=\abs{z_1}\cdot\abs{z_2}$.
			}\onslide<+->{%
				因此 $\abs{z_1\ov{z_2}}=\abs{z_1}\cdot\abs{\ov{z_2}}=\abs{z_1}\cdot\abs{z_2}$.
			}
			\item 因为
			\begin{align*}
				\text{左边}&=(z_1+z_2)(\ov{z_1}+\ov{z_2})
				=z_1\ov{z_1}+z_2\ov{z_2}+z_1\ov{z_2}+\ov{z_1}z_2,\\
				\onslide<+->{\text{右边}}&\onslide<.->{=z_1\ov{z_1}+z_2\ov{z_2}+z_1\ov{z_2}+\ov{z_1\ov{z_2}},}
			\end{align*}
			\onslide<+->{%
				而 $\ov{z_1\ov{z_2}}=\ov{z_1}z_2$, 所以两侧相等.\qedhere
			}
		\end{enumerate}
	\end{proof}
	\endgroup
\end{frame}


\subsection{复数的三角和指数形式}


\begin{frame}{复数的三角和指数形式}
	\onslide<+->
	由 $x=r\cos\theta,y=r\sin\theta$ 可得
	\onslide<+->
	\begin{definition*}[][复数的三角形式]
	\[
		z=r(\cos\theta+\ii\sin\theta).
	\]
	\end{definition*}
	\onslide<+->
	定义 \alert{$\ee^{\ii\theta}=\exp(\ii\theta):=\cos\theta+\ii\sin\theta$} (欧拉恒等式).
	\onslide<+->
	那么我们得到
	\begin{definition*}[][复数的指数形式]
	\[
		z=r\ee^{\ii\theta}=r\exp(\ii\theta).
	\]
	\end{definition*}
	\onslide<+->
	这两种形式的等价的, 指数形式可以认为是三角形式的一种缩写方式.
	\onslide<+->
	求复数的三角和指数形式的\alert{关键在于计算模和辐角}.
\end{frame}


\begin{frame}{例题: 求复数的三角和指数形式}
	\onslide<+->
	\begin{example}[nearnext]
		将 $z=-\sqrt{12}-2\ii$ 化成三角形式和指数形式.
	\end{example}
	\onslide<+->
	\begin{solution}[nearprev]
		$r=\abs{z}=\sqrt{12+4}=4$.
		\onslide<+->{%
			由于 $z$ 在第三象限,
		}\onslide<+->{%
			因此
			\[
				\arg z=\arctan\frac{-2}{-\sqrt{12}}-\pi
				\onslide<+->{=\frac\pi6-\pi=-\frac{5\pi}6.}
			\]
		}\onslide<+->{%
			故
			\[
				z=4\biggl(\cos\Bigl(-\frac{5\pi}6\Bigr)+\ii\sin\Bigl(-
				\frac{5\pi}6\Bigr)\biggr)
				=4\ee^{-\frac{5\pi\ii}6}.
			\]
		}
		\meddel
	\end{solution}
\end{frame}


\begin{frame}{例题: 求复数的三角和指数形式}
	\beqskip{10pt}
	\onslide<+->
	\begin{example}[nearnext]
		将 $z=\sin\dfrac\pi5+\ii\cos\dfrac\pi5$ 化成三角形式和指数形式.
	\end{example}
	\onslide<+->
	\begin{solution}[nearprev]
		$r=\abs{z}=1$.
		\onslide<+->{%
			由于 $z$ 在第一象限, 因此
			\[
				\arg z
				=\arctan\frac{\cos(\pi/5)}{\sin(\pi/5)}
				\onslide<+->{=\arctan\cot\frac\pi 5}
				\onslide<+->{=\frac\pi2-\frac\pi5
				=\frac{3\pi}{10}.}
			\]
		}\onslide<+->{%
			\[
				z=\cos\frac{3\pi}{10}+\ii\sin\frac{3\pi}{10}=\ee^{\frac{3\pi\ii}{10}}.
			\]
		}\bigdel
	\end{solution}
	\onslide<+->
	求复数的三角或指数形式时, 只需取一个辐角就可以了, 不要求必须是辐角主值.
	\endgroup
\end{frame}


\begin{frame}{例题: 求复数的三角和指数形式}
	\onslide<+->
	\begin{solution}[][另解]
		\[
			z=\sin\frac\pi5+\ii\cos\frac\pi5
			\visible<+->{=\cos\Bigl(\frac\pi2-\frac\pi5\Bigr)+\ii\sin\Bigl(\frac\pi2-\frac\pi5\Bigr)}
			\visible<+->{=\cos\frac{3\pi}{10}+\ii\sin\frac{3\pi}{10}=\ee^{\frac{3\pi\ii}{10}}.}
		\]
		\bigdel
	\end{solution}
	\onslide<+->
	\begin{exercise}[nearnext]
		将 $z=\sqrt 3-3\ii$ 化成三角形式和指数形式.
	\end{exercise}
	\onslide<+->
	\begin{answer}[nearprev]
		$\displaystyle z=2\sqrt3\biggl(\cos\Bigl(-\frac{\pi}3\Bigr)+\ii\sin\Bigl(-\frac{\pi}3\Bigr)\biggr)
		=2\sqrt3 \ee^{-\frac{\pi\ii}3}$, 写成 $\dfrac{5\pi}3$ 也可以.
	\end{answer}
\end{frame}


\begin{frame}{模为 $1$ 的复数}
	\onslide<+->
	两个模相等的复数之和的三角和指数形式形式较为简单.
	\onslide<+->
	\begin{twopart}{180pt}
		\[
			\ee^{\ii\theta}+\ee^{\ii\varphi}
			=2\cos\frac{\theta-\varphi}2\ee^{\frac{\theta+\varphi}2\ii}.
		\]
		\tcblower
		\begin{center}
			\begin{tikzpicture}[scale=.75]
				\coordinate [label=below left:$0$] (O) at (0,0);
				\coordinate [label=right:\textcolor{main}{$\ee^{\ii\varphi}$}] (Z1) at ({3*cos(18)},{3*sin(18)});
				\coordinate [label=left:\textcolor{main}{$\ee^{\ii\theta}$}] (Z2) at ({3*cos(130)},{3*sin(130)});
				\coordinate [label=above right:\textcolor{second}{$\ee^{\ii\theta}+\ee^{\ii\varphi}$}] (P) at ($(Z1)+(Z2)$);
				\coordinate (M) at ($0.5*(P)$);
				\coordinate (X) at (2,0);
				\draw[thick,main] pic [cstfill1, draw=main,"$\varphi$", angle eccentricity=1.4, angle radius=0.7cm] {angle=X--O--Z1};
				\draw[thick,second] pic [cstfill2, draw=second, "$\frac{\theta-\varphi}2$", angle eccentricity=1.7] {angle=Z1--O--P};
				\draw[cstaxis] (-3,0)--(3,0);
				\draw[cstaxis] (0,-.4)--(0,3.5);
				\draw[cstcurve,cstra,main] (O)--(Z1);
				\draw[cstcurve,cstra,main] (O)--(Z2);
				\draw[cstcurve,cstra,second] (O)--(P);
				\draw[cstdash] (Z2)--(Z1)--(P)--(Z2);
				\draw[thick] (M)--++({.3*cos(16)},{-.3*sin(16)})--++({.3*sin(16)},{.3*cos(16)})--++({-.3*cos(16)},{.3*sin(16)});
			\end{tikzpicture}
		\end{center}
	\end{twopart}
	\onslide<+->
	\begin{example}
		若 $\abs{z}=1,\arg z=\theta$, 则 $z+1=2\cos\dfrac\theta2 \ee^{\frac{\theta \ii}2}$.
	\end{example}
\end{frame}
