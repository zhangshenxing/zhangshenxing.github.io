\section{柯西-古萨基本定理和复合闭路定理}

\subsection{柯西-古萨基本定理}

\begin{frame}{积分路径无关与闭路积分}
	\onslide<+->
	观察下方的两条曲线 $C_1,C_2$.
	\onslide<+->
	设 $C=C_1^-+C_2$.
	\onslide<+->
	可以看出
	\[\int_{C_1}f(z)\diff z=\int_{C_2}f(z)\diff z\iff\]
	\[\oint_Cf(z)\diff z=\int_{C_2}f(z)\diff z-\int_{C_1}f(z)\diff z=0.\]
	\onslide<+->
	所以 $f(z)$ 的积分只与起点终点有关 $\iff f(z)$ 绕任意闭路的积分为零.

	\onslide<1->
	\begin{center}
		\begin{tikzpicture}[framed]
			\draw
				(-0.2,-0.2) node[dcolorb] {$z_0$}
				(2.2,2.2) node[dcolorb] {$z$}
				(0.2,1.8) node[dcolora] {$C_1$}
				(1.8,0.3) node[dcolorc] {$C_2$};
			\draw[cstcurve,dcolorc] (2,2) arc(0:-90:2);
			\draw[cstcurve,dcolorc,cstarrow1to] ({2*cos(-50)},{2+2*sin(-50)}) arc(-50:-45:2);
			\draw[cstcurve,dcolora] (0,0) arc(180:90:2);
			\draw[cstcurve,dcolora,cstarrow1to] ({2+2*cos(140)},{2*sin(140)}) arc(140:135:2);
			\draw[cstcurve,dcolorb,cstarrow1from,visible on=<2->] ({1+0.3*cos(135)},{1+0.3*sin(135)}) arc(135:-110:0.3);
		\end{tikzpicture}
	\end{center}
\end{frame}


\begin{frame}{积分路径无关的函数特点}
	\onslide<+->
	上一节中我们计算了 $f(z)=z,\Re z,\dfrac1{z-z_0}$ 的积分.
	\onslide<+->
	其中
	\begin{itemize}
		\item $f(z)=z$ 处处解析, 积分只与起点终点有关 (闭路积分为零);
		\item $f(z)=\dfrac1{z-z_0}$ 有奇点 $z_0$, 沿绕 $z_0$ 闭路的积分非零;
		\item $f(z)=\Re z$ 处处不解析, 积分与路径有关 (闭路积分非零).
	\end{itemize}
	\onslide<+->
	由此可见函数沿闭路积分为零,
	\onslide<+->
	与函数在闭路内部是否解析有关.
\end{frame}


\begin{frame}{柯西-古萨基本定理: 推导}
	\onslide<+->
	设 $C$ 是一条闭路, $D$ 是其内部区域.
	\onslide<+->
	设 \emph{$f(z)$ 在闭区域 $\ov D=D\cup C$ 上解析},
	\onslide<+->
	即存在区域 $B\supseteq\ov D$ 使得 $f(z)$ 在 $B$ 上解析.

	\onslide<+->
	为了简便假设 $f'(z)$ 连续,
	\onslide<+->
	则
	\[\oint_Cf(z)\diff z=\oint_C(u\diff x-v\diff y)
	+i\oint_C(v\diff x+u\diff y).\]
	\onslide<+->
	由格林公式和C-R方程可知
	\[\oint_Cf(z)\diff z=-\iint_D(v_x+u_y)\diff x\diff y
	+i\iint_D(u_x-v_y)\diff x\diff y=0.\]
\end{frame}


\begin{frame}{柯西-古萨基本定理}
	\onslide<+->
	\begin{alertblock}{柯西-古萨基本定理}
	设 $f(z)$ 在闭路 $C$ 上连续, $C$ 内部解析, 则 $\displaystyle\oint_Cf(z)\diff z=0$.
	\end{alertblock}

	\onslide<+->
	\begin{corollary}
	设 $f(z)$ 在单连通域 $D$ 内解析, $C$ 是 $D$ 内一条闭合曲线(可以不是闭路), 则 $\displaystyle\oint_Cf(z)\diff z=0$.
	\end{corollary}

	\onslide<+->
	这是因为即使不是简单曲线也可以拆分为一些简单曲线.
	\onslide<4->
	\begin{center}
		\begin{tikzpicture}[framed]
			\draw[cstcurve,dcolora,smooth,domain=-45:-15] plot({3*sqrt(cos(2*\x))*cos(\x)},{3*sqrt(cos(2*\x))*sin(\x)});
			\draw[cstcurve,dcolora,smooth,domain=-20:45,cstarrow1from] plot({3*sqrt(cos(2*\x))*cos(\x)},{3*sqrt(cos(2*\x))*sin(\x)});
			\draw[cstcurve,dcolora,smooth,domain=-45:-15] plot({-2*sqrt(cos(2*\x))*cos(\x)},{2*sqrt(cos(2*\x))*sin(\x)});
			\draw[cstcurve,dcolora,smooth,domain=-20:45,cstarrow1from] plot({-2*sqrt(cos(2*\x))*cos(\x)},{2*sqrt(cos(2*\x))*sin(\x)});
			\draw[cstcurve,dcolorb,cstarrow1from,ultra thick] (1.05,0.35) arc (135:-135:0.5);
			\draw[cstcurve,dcolorc,cstarrow1from,ultra thick] (-0.75,-0.25) arc (315:45:0.3535);
		\end{tikzpicture}
	\end{center}
\end{frame}


\begin{frame}[<*>]{典型例题: 柯西-古萨基本定理计算积分}
	\onslide<+->
	\begin{example}
		求 $\displaystyle\oint_{|z|=1}\frac1{2z-3}\diff z$.
	\end{example}

	\onslide<+->
	\begin{solution}
		由于 $\dfrac1{2z-3}$ 在 $|z|\le 1$ 上解析,
		\onslide<+->{因此由柯西-古萨基本定理
			\[\oint_{|z|=1}\frac1{2z-3}\diff z=0.\]}
	\end{solution}

	\onslide<+->
	\begin{exercise}
		求 $\displaystyle\oint_{|z-2|=1}\frac1{z^2+z}\diff z=$\fillblank{\visible<+->{$0$}}.
	\end{exercise}
\end{frame}


\begin{frame}{例题: 柯西-古萨基本定理计算积分}
	\onslide<+->
	\begin{example}
		求 $\displaystyle\oint_C\frac1{z(z^2+1)}\diff z$, 其中 $C:|z-i|=\half $.
	\end{example}

	\onslide<+->
	\begin{solution}
		注意到 $\dfrac1{z(z^2+1)}=\dfrac1z-\dfrac12\left(\dfrac1{z+i}+\dfrac1{z-i}\right)$.
		\onslide<+->{由于 $\dfrac1z,\dfrac1{z+i}$ 在 $|z-i|\le\dfrac12$ 上解析,
		}\onslide<+->{因此由柯西-古萨基本定理
			\vspace{-0.5\baselineskip}
			\[\oint_C\frac1z\diff z
			=\oint_C\frac1{z+i}\diff z=0,\]
		}\onslide<+->{
			\vspace{-0.5\baselineskip}
			\[\oint_C\frac1{z(z^2+1)}\diff z
			=-\half\oint_C\frac1{z-i}\diff z=-\pi i.\]}
	\end{solution}
\end{frame}


\subsection{复合闭路定理}

\begin{frame}{多连通域边界与复合闭路}
	\onslide<+->
	设 $C_0,C_1,\dots,C_n$ 是 $n+1$ 条简单闭曲线, $C_1,\dots,C_n$ 每一条都包含在其它闭路的外部, 而且它们都包含在 $C_0$ 的内部.
	\onslide<+->
	这样它们围成了一个多连通区域 $D$, 它的边界称为一个\emph{复合闭路} \[C=C_0+C_1^-+\cdots+C_n^-.\]
	\onslide<+->
	沿着 $C$ 前进的点, $D$ 总在它的左侧,所以这就是它的正方向.
	\onslide<1->
	\begin{center}
		\begin{tikzpicture}[framed]
			\fill[cstcurve,cstfill,visible on=<2->] (0,0) circle(2.5 and 1.5);
			\fill[cstcurve,white,visible on=<2->] (1,0) circle(0.6);
			\fill[cstcurve,white,visible on=<2->] (-1,0) circle(0.6);
			\draw[cstcurve,dcolora] (0,0) circle(2.5 and 1.5);
			\draw[cstcurve,dcolora] (1,0) circle(0.6);
			\draw[cstcurve,dcolora] (-1,0) circle(0.6);
			\draw[cstcurve,dcolora,domain=85:90,cstarrow1to,visible on=<2->] plot ({2.5*cos(\x)}, {1.5*sin(\x)});
			\draw[cstcurve,dcolora,domain=135:140,cstarrow1from,visible on=<2->] plot ({1+0.6*cos(\x)}, {0.6*sin(\x)});
			\draw[cstcurve,dcolora,domain=135:140,cstarrow1from,visible on=<2->] plot ({-1+0.6*cos(\x)}, {0.6*sin(\x)});
			\draw
				(2,1.3) node[dcolora] {$C_0$}
				(-1,0.85) node[dcolora] {$C_1^{\visible<2->-}$}
				(1,0.85) node[dcolora] {$C_2^{\visible<2->-}$};
		\end{tikzpicture}
	\end{center}
\end{frame}


\begin{frame}{复合闭路定理}
	\onslide<+->
	\begin{alertblock}{复合闭路定理}
		设 $f(z)$ 在复合闭路 $C=C_0+C_1^-+\cdots+C_n^-$ 及其所围成的多连通区域内解析, 则
		\[\oint_{C_0}f(z)\diff z=
		\oint_{C_1}f(z)\diff z+\cdots+\oint_{C_n}f(z)\diff z.\]
	\end{alertblock}
\end{frame}


\begin{frame}{复合闭路定理}
	\onslide<+->
	\begin{proof}
		以曲线 $\gamma_1,\gamma_2,\dots,\gamma_{n+1}$ 把 $C_0,C_1,\dots,C_n$ 连接起来, 则它们把区域 $D$ 分成了两个单连通域 $D_1,D_2$.
		\onslide<+->{对 $D_1$ 和 $D_2$ 的边界应用柯西积分定理并相加, 则 $\gamma_i$ 对应的部分正好相互抵消,
		}\onslide<+->{因此
			\[\oint_{C_0}f(z)\diff z-
		\oint_{C_1}f(z)\diff z-\cdots-\oint_{C_n}f(z)\diff z=0.\]
		}\onslide<+->{于是定理得证.\qedhere}
		\vspace{-\baselineskip}
		\onslide<1->
		\begin{center}
			\begin{tikzpicture}[framed]
				\fill[cstcurve,cstfill] (0,0) circle(2.5 and 1.5);
				\fill[cstcurve,white] (1,0) circle(0.6);
				\fill[cstcurve,white] (-1,0) circle(0.6);

				\draw[cstcurve,dcolora,domain=0:90,cstarrow1to] plot ({2.5*cos(\x)}, {1.5*sin(\x)});
				\draw[cstcurve,dcolora,domain=85:180] plot ({2.5*cos(\x)}, {1.5*sin(\x)});
				\draw[cstcurve,dcolorb,domain=180:270,cstarrow1to] plot ({2.5*cos(\x)}, {1.5*sin(\x)});
				\draw[cstcurve,dcolorb,domain=265:360] plot ({2.5*cos(\x)}, {1.5*sin(\x)});

				\draw[cstcurve,dcolora,domain=0:140] plot ({1+0.6*cos(\x)}, {0.6*sin(\x)});
				\draw[cstcurve,dcolora,domain=135:180,cstarrow1from] plot ({1+0.6*cos(\x)}, {0.6*sin(\x)});
				\draw[cstcurve,dcolorb,domain=180:320] plot ({1+0.6*cos(\x)}, {0.6*sin(\x)});
				\draw[cstcurve,dcolorb,domain=315:360,cstarrow1from] plot ({1+0.6*cos(\x)}, {0.6*sin(\x)});

				\draw[cstcurve,dcolora,domain=0:140] plot ({-1+0.6*cos(\x)}, {0.6*sin(\x)});
				\draw[cstcurve,dcolora,domain=135:180,cstarrow1from] plot ({-1+0.6*cos(\x)}, {0.6*sin(\x)});
				\draw[cstcurve,dcolorb,domain=180:320] plot ({-1+0.6*cos(\x)}, {0.6*sin(\x)});
				\draw[cstcurve,dcolorb,domain=315:360,cstarrow1from] plot ({-1+0.6*cos(\x)}, {0.6*sin(\x)});

				\draw
					(2,1.3) node[dcolora] {$C_0$}
					(-1,0.85) node[dcolora] {$C_1^-$}
					(1,0.85) node[dcolora] {$C_2^-$}
					(-2,-0.3) node {$\gamma_1$}
					(0,-0.3) node {$\gamma_2$}
					(2,-0.3) node {$\gamma_3$};
				\draw[cstcurve] (-2.5,0)--(-1.6,0);
				\draw[cstcurve] (-0.4,0)--(0.4,0);
				\draw[cstcurve] (1.6,0)--(2.5,0);
				\draw[cstcurve,thick,dcolora,visible on=<2->] (-2.5,0.025)--(-1.6,0.025);
				\draw[cstcurve,thick,dcolorb,visible on=<2->] (-2.5,-0.025)--(-1.6,-0.025);
				\draw[cstcurve,thick,dcolora,visible on=<2->] (-0.4,0.025)--(0.4,0.025);
				\draw[cstcurve,thick,dcolorb,visible on=<2->] (-0.4,-0.025)--(0.4,-0.025);
				\draw[cstcurve,thick,dcolora,visible on=<2->] (1.6,0.025)--(2.5,0.025);
				\draw[cstcurve,thick,dcolorb,visible on=<2->] (1.6,-0.025)--(2.5,-0.025);
			\end{tikzpicture}
		\end{center}
		\vspace{-\baselineskip}
	\end{proof}
\end{frame}


\begin{frame}{例题: 复合闭路定理的应用}
	\onslide<+->
	\begin{example}
		证明对于任意闭路 $C$, $\displaystyle\int_C(z-a)^n\diff z=0$, $n\neq -1$ 为整数.
	\end{example}

	\onslide<+->
	\begin{proof*}
		\onslide<+->{如果 $a$ 不在 $C$ 的内部,
		}\onslide<+->{则 $(z-a)^n$ 在 $C$ 及其内部解析.
		}\onslide<+->{由柯西积分定理, $\displaystyle\int_C(z-a)^n\diff z=0$.}

		\onslide<+->{如果 $a$ 在 $C$ 的内部, 则在 $C$ 的内部取一个以 $a$ 为圆心的圆周 $C_1$.
		}\onslide<+->{由复合闭路定理以及上一节的结论
			\[\int_C(z-a)^n\diff z=\int_{C_1}(z-a)^n\diff z=0.\qedhere\]}
		\vspace{-\baselineskip}
	\end{proof*}
\end{frame}


\begin{frame}{例题: 复合闭路定理的应用}
	\onslide<+->
	同理, 由复合闭路定理和上一节的结论可知当 $a$ 在 $C$ 的内部且 $n=-1$ 时积分为 $2\pi i$.
	\onslide<+->
	\begin{alertblock@}
		当 $a$ 在 $C$ 的内部时,
		\[\oint_C\frac{\diff z}{(z-a)^{n+1}}=\begin{cases}2\pi i,&n=0;\\0,&n\neq 0.\end{cases}\]
	\end{alertblock@}
\end{frame}


\begin{frame}{例题: 复合闭路定理的应用}
	\onslide<+->
	\begin{example}
		求 $\displaystyle\int_\Gamma\frac{2z-1}{z^2-z}\diff z$, 其中 $\Gamma$ 是由 $2\pm i,-2\pm i$ 形成的矩形闭路.
		\vspace{-\baselineskip}
		\begin{center}
			\begin{tikzpicture}[framed]
				\draw[cstaxis] (-2.3,0)--(2.3,0);
				\draw[cstaxis] (0,-1)--(0,1);
				\draw[cstcurve,cstarrow1to,dcolora] (-1,-0.7)--(-0.5,-0.7);
				\draw[cstcurve,cstarrow1to,dcolorb] (-0.4,0) arc(180:225:0.4);
				\draw[cstcurve,cstarrow1to,dcolorb] (0.6,0) arc(180:225:0.4);
				\draw[cstcurve,dcolorb,visible on=<3->] (0,0) circle (0.4);
				\draw[cstcurve,dcolorb,visible on=<3->] (1,0) circle (0.4);
				\fill[cstdot,dcolorc,visible on=<2->] (1,0) circle;
				\fill[cstdot,dcolorc,visible on=<2->] (0,0) circle;
				\draw[cstcurve,dcolora] (-2,-0.7) rectangle (2,0.7);
				\draw
					(-0.6,0.4) node[dcolorb,visible on=<3->] {$C_1$}
					(1.6,0.4) node[dcolorb,visible on=<3->] {$C_2$}
					(-1,-0.5) node[dcolora] {$\Gamma$};
			\end{tikzpicture}
			\vspace{-\baselineskip}
		\end{center}
	\end{example}

	\onslide<+->
	\begin{solution}
		函数 $\dfrac{2z-1}{z^2-z}$ 在 $\Gamma$ 内有两个奇点 $z=0,1$.
		\onslide<+->{设 $C_1,C_2$ 如图所示,
		}\onslide<+->{由复合闭路定理
			\begin{align*}
				&\peq\oint_\Gamma\frac{2z-1}{z^2-z}\diff z
				=\oint_{C_1}\frac{2z-1}{z^2-z}\diff z+\oint_{C_2}\frac{2z-1}{z^2-z}\diff z\\
				&\visible<+->{=\oint_{C_1}\frac1z\diff z+\oint_{C_1}\frac1{z-1}\diff z
				+\oint_{C_2}\frac1z\diff z+\oint_{C_2}\frac1{z-1}\diff z}\\
				&\visible<+->{=2\pi i+0+0+2\pi i=4\pi i.}
			\end{align*}}
		\vspace{-2\baselineskip}
	\end{solution}
\end{frame}


\begin{frame}{例题: 复合闭路定理的应用}
	\onslide<+->
	\begin{example}
		求 $\displaystyle\int_\Gamma\frac{e^z}z\diff z$, 其中 $\Gamma=C_1+C_2^-$, $C_1:|z|=2, C_2:|z|=1$.
		\vspace{-\baselineskip}
		\begin{center}
			\begin{tikzpicture}[framed]
				\filldraw[cstcurve,dcolora,cstfill] (0,0) circle (1.5);
				\draw[cstcurve,dcolora,cstarrow1from] (-1.06,1.06) arc(135:90:1.5);
				\filldraw[cstcurve,dcolorb,fill=white] (0,0) circle (0.75);
				\draw[cstcurve,dcolorb,cstarrow1to] (-0.75,0) arc(180:130:0.75);
				\draw
					(-1.6,0.8) node[dcolora] {$C_1$}
					(0.8,-0.7) node[dcolorb] {$C_2^-$};
				\draw[cstaxis] (-1.8,0)--(1.8,0);
				\draw[cstaxis] (0,-1.8)--(0,1.8);
			\end{tikzpicture}
		\end{center}
		\vspace{-\baselineskip}
	\end{example}

	\onslide<+->
	\begin{solution}
		函数 $\dfrac{e^z}z$ 在 $C_1,C_2$ 围城的圆环域内解析.%
		\onslide<+->{由复合闭路定理可知 $\displaystyle\int_\Gamma\frac{e^z}z\diff z=0$.}
	\end{solution}
\end{frame}


