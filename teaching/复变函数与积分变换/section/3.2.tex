\section{柯西-古萨定理和复合闭路定理}

\subsection{柯西-古萨定理}

\begin{frame}{积分路径无关与闭路积分}
	\onslide<+->
	观察下方的两条曲线 $C_1,C_2$.
	\onslide<+->
	设 $C=C_1^-+C_2$.
	\onslide<+->
	可以看出
	\[
		\int_{C_1}f(z)\d z=\int_{C_2}f(z)\d z\iff
		\oint_Cf(z)\d z=\int_{C_2}f(z)\d z-\int_{C_1}f(z)\d z=0.
	\]
	\onslide<+->
	所以 $f(z)$ 的积分只与起点终点有关 $\iff f(z)$ 绕任意闭路的积分为零.

	\onslide<+->
	这里, 若闭合曲线 $C$ 不是闭路(有自相交的点), 也可以拆分为一些闭路的并.
	\onslide<1->
	\begin{center}
		\begin{tikzpicture}
			\coordinate (O);
			\coordinate (A) at (2,2);
			\coordinate (C) at ({2+2*cos(145)},{2*sin(145)});
			\coordinate (D) at ({2*cos(-45)},{2+2*sin(-45)});
			\draw (O) node[below left,second] {$z_0$};
			\draw (A) node[above right,second] {$z$};
			\draw (C) node[above left,main] {$C_1$};
			\draw (D) node[below right,third] {$C_2$};
			\begin{scope}[cstcurve]
				\draw[third] (A) arc(0:-90:2);
				\draw[third,cstwra] ({2*cos(-50)},{2+2*sin(-50)}) arc(-50:-45:2);
				\draw[main] (O) arc(180:90:2);
				\draw[main,cstwra] ({2+2*cos(140)},{2*sin(140)}) arc(140:135:2);
				\draw[second,cstwla,visible on=<2->] ({1+0.3*cos(135)},{1+0.3*sin(135)}) arc(135:-110:0.3);
			\end{scope}
		\end{tikzpicture}
	\end{center}
\end{frame}


\begin{frame}{积分路径无关的函数特点}
	\onslide<+->
	上一节中我们计算了 $f(z)=z,\Re z,\dfrac1{z-z_0}$ 的积分.
	\onslide<+->
	其中
	\begin{itemize}
		\item $f(z)=z$ 处处解析, 积分只与起点终点有关 (闭路积分为零);
		\item $f(z)=\dfrac1{z-z_0}$ 有奇点 $z_0$, 沿绕 $z_0$ 闭路的积分非零;
		\item $f(z)=\Re z$ 处处不解析, 积分与路径有关 (闭路积分可能非零).
	\end{itemize}
	\onslide<+->
	由此可见函数沿闭路积分为零,
	\onslide<+->
	与函数在闭路内部是否解析有关.
\end{frame}


\begin{frame}{柯西-古萨定理: 推导}
	\onslide<+->
	设 $C$ 是一条闭路, $D$ 是其内部区域.
	\onslide<+->
	设 \emph{$f(z)$ 在闭区域 $\ov D=D\cup C$ 上解析},
	\onslide<+->
	即存在区域 $B\supseteq\ov D$ 使得 $f(z)$ 在 $B$ 上解析.

	\onslide<+->
	为了简便假设 $f'(z)$ 连续,
	\onslide<+->
	则
	\begin{align*}
		\oint_Cf(z)\d z&
		=\oint_C(u\d x-v\d y)+\ii\oint_C(v\d x+u\d y)\\&
		\onslide<6->{\xeq{\text{格林公式}}-\iint_D(v_x+u_y)\d x\d y+\ii\iint_D(u_x-v_y)\d x\d y}\\&
		\onslide<7->{\xeq{\text{C-R方程}}0.}
	\end{align*}
	\onslide<8->
	也可以从
	\[
		\oint_Cf(z)\d z
		=-\iint_D\frac{\partial f}{\partial \ov z}\d z\d \ov z=0
	\]
	看出.
\end{frame}


\begin{frame}{柯西-古萨定理}
	\onslide<+->
	\begin{theorem*}[][柯西-古萨定理]
	设 $f(z)$ 在闭路 $C$ 上连续, $C$ 内部解析, 则 $\doint_Cf(z)\d z=0$.
	\end{theorem*}

	\onslide<+->
	\begin{corollary}
	设 $f(z)$ 在\alert{单连通区域} $D$ 内解析, $C$ 是 $D$ 内一条闭合曲线, 则 $\doint_Cf(z)\d z=0$.
	\end{corollary}
	\onslide<+->
	这里的闭合曲线可以不是闭路.
\end{frame}


\begin{frame}{例题: 柯西-古萨定理计算积分}
	\onslide<+->
	\begin{example}[nearnext]
		求 $\doint_{\abs{z}=1}\frac1{2z-3}\d z$.
	\end{example}
	\onslide<+->
	\begin{solution}[nearprev]
		由于 $\dfrac1{2z-3}$ 在 $\abs{z}\le 1$ 上解析,
		\onslide<+->{%
			因此由柯西-古萨定理 $\doint_{\abs{z}=1}\frac1{2z-3}\d z=0$.
		}
	\end{solution}
	\onslide<+->
	\begin{exercise}
		\begin{enumerate}
			\item $\doint_{\abs{z-2}=1}\frac1{z^2+z}\d z=$\fillblankframe{$0$}.
			\item $\doint_{\abs{z}=2}\dfrac{\sin z}{\abs{z}}\d z=$\fillblankframe{$0$}.
		\end{enumerate}
	\end{exercise}
\end{frame}


\begin{frame}{例题: 柯西-古萨定理计算积分}
	\beqskip{8pt}
	\onslide<+->
	\begin{example}[near]
		求 $\doint_C\frac1{z(z^2+1)}\d z$, 其中 $C:\abs{z-\ii}=\dfrac12$.
	\end{example}
	\onslide<+->
	\begin{solution}[near]
		注意到
		\[
			\dfrac1{z(z^2+1)}=\dfrac1z-\dfrac12\Bigl(\dfrac1{z+\ii}+\dfrac1{z-\ii }\Bigr).
		\]
		\onslide<+->{%
			由于 $\dfrac1z,\dfrac1{z+\ii}$ 在 $\abs{z-\ii}\le\dfrac12$ 上解析,
		}\onslide<+->{%
			因此由柯西-古萨定理
			\[
				\oint_C\frac1z\d z=\oint_C\frac1{z+\ii}\d z=0,
			\]
		}\onslide<+->{%
			\[
				\oint_C\frac1{z(z^2+1)}\d z
				=-\half\oint_C\frac1{z-\ii }\d z=-\pi\ii.
			\]
		}\bigdel
	\end{solution}
	\endgroup
\end{frame}


\subsection{复合闭路定理和连续变形定理}

\begin{frame}{多连通区域边界与复合闭路}
	\onslide<+->
	设 $C_0,C_1,\dots,C_n$ 是 $n+1$ 条闭路, $C_1,\dots,C_n$ 每一条都包含在其它闭路的外部, 而且它们都包含在 $C_0$ 的内部.
	\onslide<+->
	这样它们围成了一个多连通区域 $D$, 它的边界称为一个\emph{复合闭路}
	\[
		C=C_0+C_1^-+\cdots+C_n^-.
	\]
	\onslide<+->
	沿着 $C$ 前进的点, $D$ 总在它的左侧,所以这就是它的正方向.
	\onslide<1->
	\begin{center}
		\begin{tikzpicture}
			\fill[cstfill,visible on=<2->,rounded corners=15mm] (-2.5,-1.5) rectangle (2.5,1.5);
			\draw[cstcurve,main,rounded corners=15mm] (-2.5,-1.5) rectangle (2.5,1.5);
			\fill[cstcurve,white,visible on=<2->] (1,0) circle(0.6);
			\fill[cstcurve,white,visible on=<2->] (-1,0) circle(0.6);
			\draw[cstcurve,main] (1,0) circle(0.6);
			\draw[cstcurve,main] (-1,0) circle(0.6);
			\draw[cstcurve,main,domain=85:90,cstwra,visible on=<2->] plot (.3,1.5)--(0,1.5);
			\draw[cstcurve,main,domain=135:140,cstwla,visible on=<2->] plot ({1+0.6*cos(\x)}, {0.6*sin(\x)});
			\draw[cstcurve,main,domain=135:140,cstwla,visible on=<2->] plot ({-1+0.6*cos(\x)}, {0.6*sin(\x)});
			\draw
				(2.2,1.3) node[main] {$C_0$}
				(-1,0.85) node[main] {$C_1^{\visible<2->-}$}
				(1,0.85) node[main] {$C_2^{\visible<2->-}$};
		\end{tikzpicture}
	\end{center}
\end{frame}


\begin{frame}{复合闭路定理}
	\onslide<+->
	\begin{theorem*}[][复合闭路定理]
		设 $f(z)$ 在复合闭路 $C=C_0+C_1^-+\cdots+C_n^-$ 及其所围成的多连通区域内解析, 则
		\[
			\oint_{C_0}f(z)\d z=
			\oint_{C_1}f(z)\d z+\cdots+\oint_{C_n}f(z)\d z.
		\]
	\end{theorem*}
	\onslide<+->
	事实上, 复合闭路定理和柯西-古萨定理可以看成一个定理的两种情形: 
	\onslide<+->
	设 $C$ 是一个闭路或复合闭路, 若 $f(z)$ 在 $C$ 及其围成的区域(单连通或多连通)内解析, 则 $\doint_Cf(z)\d z=0$.
\end{frame}



\begin{frame}{连续变形定理}
	\onslide<+->
	在实际应用中, 若被积函数 $f(z)$ 在(复合)闭路 $C$ 的内部有有限多个奇点 $z_1,\dots,z_k$.
	\onslide<+->
	那么我们可以在 $C$ 内部(围成的区域)构造闭路 $C_1,\dots,C_k$, 使得每个 $C_j$ 内部只包含一个奇点 $z_j$.
	\onslide<+->
	这样, 内部含多个奇点的情形就可以化成内部只含一个奇点的情形. 最后将这些闭路上的积分相加即可.
	\onslide<1->
	\begin{center}
		\begin{tikzpicture}
			\draw[cstcurve,main,rounded corners=10mm] (-2.2,-1.2) rectangle (2.2,1.2);
			\draw[cstcurve,second,visible on=<2->] (1,0) circle(0.4);
			\draw[cstcurve,second,visible on=<2->] (0,0) circle(0.4);
			\draw[cstcurve,second,visible on=<2->] (-1,0) circle(0.4);
			\fill[cstdot,second] (1,0) circle;
			\fill[cstdot,second] (0,0) circle;
			\fill[cstdot,second] (-1,0) circle;
			\draw
				(-1,-0.25) node[second] {$z_1$}
				(0,-0.25) node[second] {$z_2$}
				(1,-0.25) node[second] {$z_3$}
				(2.2,1.1) node[main] {$C$}
				(-1,0.65) node[second,visible on=<2->] {$C_1$}
				(0,0.65) node[second,visible on=<2->] {$C_2$}
				(1,0.65) node[second,visible on=<2->] {$C_3$};
		\end{tikzpicture}
	\end{center}
\end{frame}


\begin{frame}{连续变形定理}
	\onslide<+->
	此外, 从复合闭路定理还可以看出, 在计算积分 $\doint_C f(z)\d z$ 时, $C$ 的具体形状无关紧要, 只要其内部奇点不变, $C$ 可以任意变形.
	\onslide<+->
	因为我们总可以选择一个包含这些奇点的闭路 $C'$, 使得 $C'$ 包含在 $C$ 及其变形后的闭路内部. 这样它们的积分自然都和 $C'$ 上的积分相同.
	\onslide<+->
	这里即使 $C$ 是复合闭路也是可以自由变形的.
	\onslide<1->
	\begin{center}
		\begin{tikzpicture}
			\draw[cstcurve,main,rounded corners=10mm] (-2.2,-1.2) rectangle (2.2,1.2);
			\draw[cstcurve,third] (0,0) circle(1.2 and 2.2);
			\draw[cstcurve,second,visible on=<2->] (0,0) circle(1);
			\fill[cstdot,second] (.7,0.2) circle;
			\fill[cstdot,second] (0,-0.4) circle;
			\fill[cstdot,second] (-.5,0.1) circle;
			\draw
				(2,1.4) node[main] {$C_1$}
				(-1.1,1.8) node[third] {$C_2$}
				(-.4,0.65) node[second,visible on=<2->] {$C'$};
		\end{tikzpicture}
	\end{center}
\end{frame}


\begin{frame}{复合闭路定理}
	\onslide<+->
	\begin{proof}
		以曲线 $\gamma_1,\gamma_2,\dots,\gamma_{n+1}$ 把 $C_0,C_1,\dots,C_n$ 连接起来, 则它们把区域 $D$ 分成了两个单连通区域 $D_1,D_2$.
		\onslide<+->{%
			对 $D_1$ 和 $D_2$ 的边界应用柯西-古萨定理并相加, 则 $\gamma_i$ 对应的部分正好相互抵消,
		}\onslide<+->{%
			因此
			\[
				\oint_{C_0}f(z)\d z-\oint_{C_1}f(z)\d z-\cdots-\oint_{C_n}f(z)\d z=0.
			\]
		}\onslide<+->{%
			于是定理得证.\qedhere
		}\bigdel
		\onslide<1->
		\begin{center}
			\begin{tikzpicture}[scale=.9]
				\fill[cstcurve,cstfill] (0,0) circle(2.5 and 1.5);
				\fill[cstcurve,white] (1,0) circle(0.6);
				\fill[cstcurve,white] (-1,0) circle(0.6);

				\draw[cstcurve,main,domain=0:90,cstwra] plot ({2.5*cos(\x)}, {1.5*sin(\x)});
				\draw[cstcurve,main,domain=85:180] plot ({2.5*cos(\x)}, {1.5*sin(\x)});
				\draw[cstcurve,second,domain=180:270,cstwra] plot ({2.5*cos(\x)}, {1.5*sin(\x)});
				\draw[cstcurve,second,domain=265:360] plot ({2.5*cos(\x)}, {1.5*sin(\x)});

				\draw[cstcurve,main,domain=0:140] plot ({1+0.6*cos(\x)}, {0.6*sin(\x)});
				\draw[cstcurve,main,domain=135:180,cstwla] plot ({1+0.6*cos(\x)}, {0.6*sin(\x)});
				\draw[cstcurve,second,domain=180:320] plot ({1+0.6*cos(\x)}, {0.6*sin(\x)});
				\draw[cstcurve,second,domain=315:360,cstwla] plot ({1+0.6*cos(\x)}, {0.6*sin(\x)});

				\draw[cstcurve,main,domain=0:140] plot ({-1+0.6*cos(\x)}, {0.6*sin(\x)});
				\draw[cstcurve,main,domain=135:180,cstwla] plot ({-1+0.6*cos(\x)}, {0.6*sin(\x)});
				\draw[cstcurve,second,domain=180:320] plot ({-1+0.6*cos(\x)}, {0.6*sin(\x)});
				\draw[cstcurve,second,domain=315:360,cstwla] plot ({-1+0.6*cos(\x)}, {0.6*sin(\x)});

				\draw
					(2,1.3) node[main] {$C_0$}
					(-1,0.85) node[main] {$C_1^-$}
					(1,0.85) node[main] {$C_2^-$}
					(-2,-0.3) node {$\gamma_1$}
					(0,-0.3) node {$\gamma_2$}
					(2,-0.3) node {$\gamma_3$};
				\draw[cstcurve] (-2.5,0)--(-1.6,0);
				\draw[cstcurve] (-0.4,0)--(0.4,0);
				\draw[cstcurve] (1.6,0)--(2.5,0);
				\draw[cstcurve,thick,main,visible on=<2->] (-2.5,0.025)--(-1.6,0.025);
				\draw[cstcurve,thick,second,visible on=<2->] (-2.5,-0.025)--(-1.6,-0.025);
				\draw[cstcurve,thick,main,visible on=<2->] (-0.4,0.025)--(0.4,0.025);
				\draw[cstcurve,thick,second,visible on=<2->] (-0.4,-0.025)--(0.4,-0.025);
				\draw[cstcurve,thick,main,visible on=<2->] (1.6,0.025)--(2.5,0.025);
				\draw[cstcurve,thick,second,visible on=<2->] (1.6,-0.025)--(2.5,-0.025);
			\end{tikzpicture}
		\end{center}
		\bigdel
	\end{proof}
\end{frame}


\begin{frame}{例题: 复合闭路定理的应用}
	\onslide<+->
	\begin{example}[nearnext]
		证明对于任意闭路 $C$, $\dint_C(z-a)^n\d z=0$, $n\neq -1$ 为整数.
	\end{example}
	\onslide<+->
	\begin{proof}[nearprev]
		\begin{enumerate}
			\item 若 $a$ 不在 $C$ 的内部, 则 $(z-a)^n$ 在 $C$ 及其内部解析.
			\onslide<+->{%
				由柯西-古萨定理,
				\[
					\dint_C(z-a)^n\d z=0.
				\]
			}
			\item 若 $a$ 在 $C$ 的内部, 则在 $C$ 的内部取一个以 $a$ 为圆心的圆周 $C_1$.
			\onslide<+->{%
				由复合闭路定理以及上一节的结论
				\[
					\int_C(z-a)^n\d z=\int_{C_1}(z-a)^n\d z=0.\qedhere
				\]
			}
			\bigdel
		\end{enumerate}
	\end{proof}
\end{frame}


\begin{frame}{例题: 复合闭路定理的应用}
	\onslide<+->
	同理, 由复合闭路定理和上一节的结论可知当 $a$ 在 $C$ 的内部且 $n=-1$ 时积分为 $2\pi\ii$.
	\onslide<+->
	\begin{theorem}
		当 $a$ 在 $C$ 的内部时,
	\[
		\oint_C\frac{\d z}{(z-a)^{n+1}}=\begin{cases}
			2\pi\ii,&n=0;\\
			0,&n\neq 0.
		\end{cases}
	\]
	\end{theorem}
\end{frame}


\begin{frame}{例题: 复合闭路定理的应用}
	\beqskip{4pt}
	\onslide<+->
	\begin{example}[nearnext,sidepic,righthand width=144pt]
		求 $\dint_\Gamma\frac{2z-1}{z^2-z}\d z$, 其中 $\Gamma$ 是由 $2\pm \ii,-2\pm \ii$ 形成的矩形闭路.
		\tcblower
		\begin{center}
			\begin{tikzpicture}
				\draw[cstaxis] (-2.5,0)--(2.5,0);
				\draw[cstaxis] (0,-1)--(0,1);
				\draw[cstcurve,cstwra,main] (-1,-0.7)--(-0.5,-0.7);
				\draw[cstcurve,cstwra,second,visible on=<3->] (-0.4,0) arc(180:225:0.4);
				\draw[cstcurve,cstwra,second,visible on=<3->] (0.6,0) arc(180:225:0.4);
				\draw[cstcurve,second,visible on=<3->] (0,0) circle (0.4);
				\draw[cstcurve,second,visible on=<3->] (1,0) circle (0.4);
				\fill[cstdot,third,visible on=<2->] (1,0) circle;
				\fill[cstdot,third,visible on=<2->] (0,0) circle;
				\draw[cstcurve,main] (-2,-0.7) rectangle (2,0.7);
				\draw
					(-0.6,0.4) node[second,visible on=<3->] {$C_1$}
					(1.6,0.4) node[second,visible on=<3->] {$C_2$}
					(-1,-0.5) node[main] {$\Gamma$};
			\end{tikzpicture}
		\end{center}
	\end{example}
	\onslide<+->
	\begin{solution}[nearprev]
		函数 $\dfrac{2z-1}{z^2-z}$ 在 $\Gamma$ 内有两个奇点 $z=0,1$.
		\onslide<+->{%
			设 $C_1,C_2$ 如图所示,
		}\onslide<+->{%
			由复合闭路定理
			\begin{align*}
				&\oint_\Gamma\frac{2z-1}{z^2-z}\d z
				=\oint_{C_1}\frac{2z-1}{z^2-z}\d z+\oint_{C_2}\frac{2z-1}{z^2-z}\d z\\
				\visible<+->{=}&\visible<.->{\oint_{C_1}\frac1z\d z+\oint_{C_1}\frac1{z-1}\d z
				+\oint_{C_2}\frac1z\d z+\oint_{C_2}\frac1{z-1}\d z}\\
				\visible<+->{=}&\visible<.->{2\pi\ii+0+0+2\pi\ii=4\pi\ii.}
			\end{align*}
		}\bigdel
	\end{solution}
	\endgroup
\end{frame}


\begin{frame}{例题: 复合闭路定理的应用}
	\onslide<+->
	\begin{example}[nearnext,sidepic,righthand width=94pt]
		求 $\dint_\Gamma\frac{\ee^z}z\d z$, 其中 $\Gamma=C_1+C_2^-$, $C_1:\abs{z}=2, C_2:\abs{z}=1$.
		\tcblower
		\begin{center}
			\begin{tikzpicture}[scale=.8]
				\filldraw[cstcurve,main,cstfill] (0,0) circle (1.5);
				\draw[cstcurve,main,cstwla] (-1.06,1.06) arc(135:90:1.5);
				\filldraw[cstcurve,second,fill=white] (0,0) circle (0.75);
				\draw[cstcurve,second,cstwra] (-0.75,0) arc(180:130:0.75);
				\draw
					(-1.6,0.8) node[main] {$C_1$}
					(0.8,-0.7) node[second] {$C_2^-$};
				\draw[cstaxis] (-1.8,0)--(1.8,0);
				\draw[cstaxis] (0,-1.8)--(0,1.8);
			\end{tikzpicture}
		\end{center}
		\bigdel
	\end{example}
	\onslide<+->
	\begin{solution}[nearprev]
		函数 $\dfrac{\ee^z}z$ 在 $C_1,C_2$ 围城的圆环域内解析.
		\onslide<+->{由复合闭路定理可知 $\dint_\Gamma\frac{\ee^z}z\d z=0$.}
	\end{solution}
\end{frame}


\begin{frame}{例题: 有理函数绕闭路积分}
	\onslide<+->
	\begin{example}
		设 $f(z)=\dfrac{p(z)}{q(z)}$ 是一个有理函数, 其中 $p,q$ 的次数分别是 $m,n$.
		证明: 若 $f(z)$ 的所有奇点都在闭路 $C$ 的内部, 则
		\[
			\oint_C f(z)\d z=\begin{cases}
				0,&\text{若}\ n-m\ge 2,\\
				2\pi\ii a/b,&\text{若}\ n-m=1,
			\end{cases}
		\]
		其中 $a,b$ 分别是 $p(z),q(z)$ 的最高次项系数.
	\end{example}
\end{frame}


\begin{frame}{例题: 有理函数绕闭路积分}
	\beqskip{8pt}
	\onslide<+->
	\begin{proof}[near]
		设 $C_R:\abs{z}=R$.
		\onslide<+->{%
			注意到
			\[
				\lim_{z\ra\infty} zf(z)=\begin{cases}
					0,&\text{若}\ n-m\ge 2,\\
					a/b,&\text{若}\ n-m=1,
				\end{cases}
			\]
		}\onslide<+->{%
			于是由大圆弧引理可知
			\[
				\lim_{R\ra+\infty}\oint_{C_R} f(z)\d z=\begin{cases}
					0,&\text{若}\ n-m\ge 2,\\
					2\pi\ii a/b,&\text{若}\ n-m=1.
				\end{cases}
			\]
		}\onslide<+->{%
			由连续变形定理可知, 当 $R$ 充分大使得 $f(z)$ 的所有奇点都在 $C_R$ 的内部时,
			\[
				\oint_C f(z)\d z=\oint_{C_R} f(z)\d z
			\]
			恒成立, 由此命题得证.\qedhere
		}
	\end{proof}
	\endgroup
\end{frame}


\begin{frame}{例题: 有理函数绕闭路积分}
	\onslide<+->
	注意闭路 $C$ 内部必须包含 $f(z)$ 的所有奇点上述结论方可成立.
	\onslide<+->
	若 $m\ge n$, 则我们可将 $f(z)$ 写成一个多项式和上述形式有理函数之和.
	\onslide<+->
	\begin{exercise}
		$\doint_{\abs{z}=2}\dfrac{z^2}{(2z+1)(z^2+1)}=$\fillblankframe{$\pi\ii$}.
	\end{exercise}
\end{frame}