\begin{frame}{课程安排}
	\onslide<+->
	本课程共$10$周$40$课时, 自2023年9月12日至2023年11月16日.
	
	\onslide<+->
\noindent
	\emph{课程QQ群}:(入群答案\textbf{1400261B})
	\begin{itemize}[<*>]
		\item 003班(机器人、自动化) \emph{\textbf{871140152}}
		\item 004班(力学、电气) \emph{\textbf{871141886}}
	\end{itemize}
	\onslide<+->
	\emph{教材}:
		\begin{tikzpicture}[overlay,xshift=8.5cm,yshift=-1cm]
			\draw
			(0,0) node {\includegraphics[height=3.5cm]{../image/book1.jpg}}
			(3.5,0) node {\includegraphics[height=3.5cm]{../image/book2.png}};
		\end{tikzpicture}
	\begin{itemize}[<*>]
		\item 西交高数教研室《复变函数》
		\item 张元林《积分变换》
	\end{itemize}
	\onslide<+->
	\emph{成绩构成}:
	\begin{itemize}
		\item \alert{作业 15\%}, 每章交一次
		\item \alert{课堂测验 25\%}, 一共3次,  取最高的两次
		\item \alert{期末报告 10\%}
		\item \alert{期末考试 50\%}, 至少45分才计算总评
	\end{itemize}
\end{frame}


\begin{frame}{复变函数的应用}
	\onslide<+->
	复变函数的应用非常广泛, 它包括:
	\begin{itemize}
		\item \emph{数学}中的代数、数论、几何、分析、动力系统……
		\item \emph{物理学}中流体力学、材料力学、电磁学、光学、量子力学……
		\item \emph{信息学}、\emph{电子学}、\emph{电气工程}……
	\end{itemize}
	\onslide<+->
	可以说复变函数应用之广, 在大学数学课程中仅次于高等数学和线性代数. 
\end{frame}


\begin{frame}{课程内容}
	\begin{center}
		\begin{tikzpicture}[
			small mindmap,
			every node/.style={concept, circular drop shadow,execute at begin node=\hskip0pt},
			root concept/.append style={concept color=black, fill=white, line width=1ex, text=black},
			font=\scriptsize, text=white,
			childa/.style={concept color=red,faded/.style={concept color=red!50}},
			childb/.style={concept color=blue,faded/.style={concept color=blue!50}},
			childc/.style={concept color=orange,faded/.style={concept color=orange!50}},
			childd/.style={concept color=green!50!black,faded/.style={concept color=green!50!black!50}},
			grow cyclic,
			level 1/.append style={level distance=2.8cm,font=\small},
			level 2/.append style={level distance=2.5cm,sibling angle=45,font=\scriptsize}]
			\node [root concept] {复变函数与积分变换} % root
			child [grow=-20, level distance=3cm, childa] { node {复数的概念}
				child[grow=205, level distance=2.5cm] { node {复数的基本要素} }
				child[grow=-30] { node {复数的运算法则和性质} }
				child[grow=15, level distance=2.5cm] { node {复数列、级数、函数、极限、积分} }
			}
			child [grow=20, level distance=3cm, childb] { node {复数的解析理论}
				child[grow=10,level distance=2.5cm] { node {刻画: C-R方程} }
				child[grow=50,level distance=2.8cm] { node {性质: 柯西-古萨基本定理} }
				child[grow=115,level distance=2.5cm] { node {性质: 复闭路定理} }
				child[grow=150,level distance=3cm] { node {应用: 柯西积分公式} }
			}
			child [grow=160, level distance=3cm, childc] { node {复数的级数理论}
				child[grow=70,level distance=2.5cm] { node {幂级数与泰勒展开} }
				child[grow=120,level distance=2.7cm] { node {洛朗展开} }
				child[grow=160,level distance=2.5cm] { node {孤立奇点的留数} }
			}
			child [grow=200, level distance=3.3cm, childd] { node {积分变换}
				child[grow=160,level distance=2.3cm] { node {傅里叶变换} }
				child[grow=210,level distance=2.5cm] { node {拉普拉斯变换} }
				child[grow=-32,level distance=2.3cm] { node {积分变换的应用} }
			};
		\end{tikzpicture}
	\end{center}
\end{frame}

\begin{frame}{课程学习方法}
	\begin{center}
		\begin{tikzpicture}[node distance=25pt]
			\node[cstnodeg,align=center] (1) at (0,2)  {\emph{课前}\\预习课本};
			\node[cstnodeg,align=center] (2) at (3,0)  {\emph{课上}\\认真听课\\记好笔记};
			\node[cstnodeg,align=center] (3) at (0,-2) {\emph{课后}\\过一遍教材\\与课上知识点};
			\node[cstnodeg,align=center] (4) at (-3,0) {\emph{作业}\\检测学\\习效果};
			\draw[cstnarrow,dcolorc] (1.east) to[bend left] (2.north);
			\draw[cstnarrow,dcolorc] (2.south) to[bend left] (3.east);
			\draw[cstnarrow,dcolorc] (3.west) to[bend left] (4.south);
			\draw[cstnarrow,dcolorc] (4.north) to[bend left] (1.west);
		\end{tikzpicture}
	\end{center}
\end{frame}

