\begin{frame}[<*>]{课程信息}
\begin{itemize}
	\item 课时:
		\begin{itemize}
			\item $10$周$40$课时;
			\item 2024-09-09 $\sim$ 2024-11-14
		\end{itemize}
	\item 课程QQ群(入群答案 1400261B)
		\begin{itemize}
			\item 008班(生医、通信工程) \alert{\textbf{973042840}}
			\item 009班(车辆创新实验、集成) \alert{\textbf{973041550}}
		\end{itemize}
	\item 教材: 西交高数教研室《复变函数》,张元林《积分变换》
\end{itemize}
\begin{center}
	\begin{figure}
		\includegraphics[height=32mm]{../image/book1.jpg}
		\includegraphics[height=32mm]{../image/book2.png}
	\end{figure}
\end{center}
\end{frame}


\begin{frame}{成绩构成}
	\vspace{-\baselineskip}
	\begin{center}
	\begin{tikzpicture}

		\begin{scope}
			\begin{scope}[xshift=.2mm,yshift=1.7mm,scale=2]
				\filldraw[cstcurve,main,cstfill1] (0,0)--(0,1) arc (90:144:1) -- cycle;
			\end{scope}
			\draw[main] (-.7,1.5)--(-1.6,2.5)--(-3,2.5);
			\filldraw[main,cstfill1] (-.7,1.5) circle(.1);
			\node at (-5,1.7) [text width=53mm]  (1){
				\begin{main*}{作业 15分}
					作业每次会提前发布, 每章交一次.
					\alert{作业不允许迟交}.
					没带的请当天联系助教补交, 迟一天交 $-50\%$ 当次作业分, 迟两天或以上0分.
					请假需提前交给我请假条.
				\end{main*}
			};
		\end{scope}

		\begin{scope}[visible on=<2->]
			\begin{scope}[xshift=1.5mm,yshift=1.7mm,scale=2]
				\filldraw[cstcurve,second,cstfill2] (0,0)--(1,0) arc (0:90:1) -- cycle;
			\end{scope}
			\draw[second] (.9,1.5)--(1.6,2.5)--(3,2.5);
			\filldraw[second,cstfill2] (.9,1.5) circle(.1);
			\node at (4.7,1.8) [text width=43mm] (2){{\begin{second*}{课堂测验 25分}
				\onslide<2->{%
					课堂测验共3次, 取最高的两次平均. 测验范围和时间会提前通知. \alert{测验时在教室内作答,否则按未考处理}. 
				}
			\end{second*}}};
		\end{scope}

		\begin{scope}[visible on=<3->]
			\begin{scope}[xshift=1.8mm,yshift=.5mm,scale=1.95]
				\filldraw[cstcurve,third,cstfill3] (0,0)--({cos{36}},-sin{36}) arc (-36:0:1) -- cycle;
			\end{scope}
			\draw[third] (1.5,-.4)--(2.2,-1.1)--(3,-1.1);
			\filldraw[third,cstfill3] (1.5,-.4) circle(.1);
			\node at (4.7,-1.5) [text width=43mm] (3){\begin{third*}{期末报告 10分}
				\onslide<3->{期末之前会告知主题. 请交手写纸质版, 并自行留存电子版本以免意外丢失. }
			\end{third*}};
		\end{scope}

		\begin{scope}[visible on=<4->]
			\begin{scope}[scale=2.1]
				\filldraw[cstcurve,fourth,cstfill4] (0,0)--({cos{144}},sin{144}) arc (144:324:1) -- cycle;
			\end{scope}
			\draw[fourth] (-.7,-1)--(-1.6,-1.7)--(-3,-1.7);
			\filldraw[fourth,cstfill4] (-.7,-1) circle(.1);
			\node at (-5,-1.7) [text width=53mm] (4){\begin{fourth*}{期末考试 50分}
				\onslide<4->{期末卷面需要达到45分才计算总评分数, 45分以下直接不及格.}
			\end{fourth*}};
		\end{scope}
	\end{tikzpicture}
\end{center}
\end{frame}



\begin{frame}{复变函数的应用}
	\onslide<+->
	复变函数的应用非常广泛, 它包括:
	\begin{itemize}
		\item \alert{数学}中的代数、数论、几何、分析、动力系统……
		\item \alert{物理学}中流体力学、材料力学、电磁学、光学、量子力学……
		\item \alert{信息学}、\alert{电子学}、\alert{电气工程}……
	\end{itemize}
	\onslide<+->
	可以说复变函数应用之广, 在大学数学课程中仅次于高等数学和线性代数. 
\end{frame}


\begin{frame}{课程内容}
	\begin{center}
		\begin{tikzpicture}[
			small mindmap,
			every node/.style={concept, circular drop shadow,execute at begin node=\hskip0pt},
			root concept/.append style={concept color=black, fill=white, line width=1ex, text=black},
			font=\scriptsize, text=white,
			childa/.style={concept color=main,faded/.style={concept color=main!50}},
			childb/.style={concept color=second,faded/.style={concept color=second!50}},
			childc/.style={concept color=third,faded/.style={concept color=third!50}},
			childd/.style={concept color=fourth,faded/.style={concept color=fourth!50}},
			grow cyclic,
			level 1/.append style={level distance=2.8cm,font=\small},
			level 2/.append style={level distance=2.5cm,sibling angle=45,font=\scriptsize}]
			\node [root concept] {复变函数与积分变换} % root
			child [grow=-20, level distance=3cm, childa] { node {复数的概念}
				child[grow=205, level distance=2.5cm] { node {复数的基本要素} }
				child[grow=-25] { node {复数的运算法则和性质} }
				child[grow=20, level distance=2.5cm] { node {复数列、级数、函数、极限、积分} }
			}
			child [grow=20, level distance=3cm, childb] { node {复数的解析理论}
				child[grow=15,level distance=2.5cm] { node {刻画: C-R方程} }
				child[grow=55,level distance=2.5cm] { node {性质: 柯西-古萨基本定理} }
				child[grow=115,level distance=2.5cm] { node {性质: 复闭路定理} }
				child[grow=150,level distance=3cm] { node {应用: 柯西积分公式} }
			}
			child [grow=160, level distance=3cm, childc] { node {复数的级数理论}
				child[grow=70,level distance=2.5cm] { node {幂级数与泰勒展开} }
				child[grow=120,level distance=2.7cm] { node {洛朗展开} }
				child[grow=160,level distance=2.5cm] { node {孤立奇点的留数} }
			}
			child [grow=200, level distance=3.3cm, childd] { node {积分变换}
				child[grow=160,level distance=2.3cm] { node {傅里叶变换} }
				child[grow=210,level distance=2.5cm] { node {拉普拉斯变换} }
				child[grow=-32,level distance=2.3cm] { node {积分变换的应用} }
			};
		\end{tikzpicture}
	\end{center}
\end{frame}

\begin{frame}{课程学习方法}
	\begin{center}
		\begin{tikzpicture}[node distance=25pt]
			\node[cstnode1,align=center] (1) at (0,2)  {\alert{课前}\\预习课本};
			\node[cstnode1,align=center] (2) at (3,0)  {\alert{课上}\\认真听课\\记好笔记};
			\node[cstnode1,align=center] (3) at (0,-2) {\alert{课后}\\过一遍教材\\与课上知识点};
			\node[cstnode1,align=center] (4) at (-3,0) {\alert{作业}\\检测学\\习效果};
			\draw[cstnarrow,main] (1.east) to[bend left] (2.north);
			\draw[cstnarrow,main] (2.south) to[bend left] (3.east);
			\draw[cstnarrow,main] (3.west) to[bend left] (4.south);
			\draw[cstnarrow,main] (4.north) to[bend left] (1.west);
		\end{tikzpicture}
	\end{center}
\end{frame}

