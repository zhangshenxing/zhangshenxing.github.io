\section{复变函数}

\subsection{复变函数的定义}


\begin{frame}{复变函数的定义}
	\onslide<+->
	所谓的\emph{映射}, 就是两个集合之间的一种对应 $f:A\to B$, 使得对于每一个 $a\in A$, 有一个唯一确定的 $b=f(a)$ 与之对应.
	\begin{enumerate}
		\item 当 $A$ 和 $B$ 都是实数集合的子集时, 它就是一个实变函数.
		\item 当 $A$ 和 $B$ 都是复数集合的子集时, 它就是一个\emph{复变函数}.
	\end{enumerate}
	\onslide<+->
	\begin{example}[nearnext]
		$f(z)=\Re z,\arg z,\abs{z}$, $z^n$ ($n$ 为整数), $\dfrac{z+1}{z^2+1}$ 都是复变函数.
	\end{example}
	\onslide<+->
	\begin{definition}
		称 $A$ 为 函数 $f$ 的\emph{定义域},
		\onslide<+->{%
			称 $\{w=f(z)\mid z\in A\}$ 为它的\emph{值域}.
		}
	\end{definition}
	\onslide<+->
	上述函数的定义域和值域分别是什么?
\end{frame}


\begin{frame}{多值复变函数}
	\onslide<+->
	在复变函数理论中, 我们常常会遇到\emph{多值的复变函数}, 也就是说一个 $z\in A$ 可能有多个 $w$ 与之对应.
	\onslide<+->
	例如 $\Arg z,\sqrt[n]z$ 等.

	\onslide<+->
	为了方便研究, 我们常常需要对每一个 $z$, 选取固定的一个 $f(z)$ 的值.
	\onslide<+->
	这样我们得到了这个多值函数的一个\emph{单值分支}.
	\onslide<+->
	\begin{example}
		$\arg z$ 是无穷多值函数 $\Arg z$ 的一个单值分支.
	\end{example}
	\onslide<+->
	在考虑多值的情况下, 复变函数总有反函数.
	\onslide<+->
	若 $f$ 和 $f^{-1}$ 都是单值的, 则称 $f$ 是\emph{一一对应}.
	\onslide<+->
	\begin{example}
		$f(z)=z^n$ 的反函数就是 $f^{-1}(w)=\sqrt[n]{w}$.
		\onslide<+->{%
			当 $n=\pm1$ 时, $f$ 是一一对应.
		}
	\end{example}
	\onslide<+->
	若无特别声明, 本课程中\alert{复变函数总是指单值的复变函数}.
\end{frame}


\subsection{复平面的变换}


\begin{frame}{变换}
	\onslide<+->
	大部分复变函数的图像无法在三维空间中表示出来.
	\onslide<+->
	为了直观理解和研究, 我们用两个复平面($z$ 复平面和 $w$ 复平面)之间的变换(也叫映射、映照)来表示这种对应关系.
	\onslide<+->
	注意到 $w$ 的实部和虚部可以看作 $z$ 的实部和虚部的函数, 即
		\[
			w=u+\ii v=u(x,y)+\ii v(x,y)
		\]
		的实部和虚部是两个二元实变函数.
	\onslide<+->
	\begin{center}
		\begin{tikzpicture}
			\begin{scope}[xshift=-25mm]
				\draw[cstaxis] (-2,0)--(2,0);
				\draw[cstaxis] (0,-1.5)--(0,1.5);
				\draw
					(2,0) node[above] {$x$}
					(0,1.5) node[left] {$y$}
					(0,-1.5) node[below,main] {$z$ 复平面};
				\draw[cstcurve,main,smooth] plot coordinates {(-1.5,0) (-1.7,-.4) (-.3,-.9) (.5,-.7) (.9,0) (1.1,1) (-.3,1.2) (-.7,1) (-1.5,0)};
				\coordinate (a) at (.5,.8);
				\coordinate (b) at (.5,.5);
				\coordinate (c) at (-.3,.3);
			\end{scope}
			\begin{scope}[xshift=25mm]
				\draw[cstaxis] (-2,0)--(2,0);
				\draw[cstaxis] (0,-1.5)--(0,1.5);
				\draw
					(2,0) node[above] {$u$}
					(0,1.5) node[left] {$v$}
					(0,-1.5) node[below,second] {$w$ 复平面};
				\draw[cstcurve,smooth,second] plot coordinates {(-1.3,0) (-.5,-.5) (0,-.8) (.5,-.5) (1,0) (1.3,.9) (.8,1.2) (-.5,.8) (-1.3,0)};
				\coordinate (A) at (.3,.7) circle;
				\coordinate (B) at (-.3,-.3) circle;
			\end{scope}
			\draw[cstdash,smooth,third,cstra] (a) to [bend left=25] (A);
			\draw[cstdash,smooth,third,cstra] (b)to [bend right=15] (B);
			\draw[cstdash,smooth,third,cstra] (c) to [bend right=25] (B);
			\fill[cstdot,main] (a) circle;
			\fill[cstdot,main] (b) circle;
			\fill[cstdot,main] (c) circle;
			\fill[cstdot,second] (A) circle;
			\fill[cstdot,second] (B) circle;
		\end{tikzpicture}
	\end{center}
\end{frame}


\begin{frame}{例题: 翻转变换}
	\onslide<+->
	\begin{example}[sidepic,righthand width=5cm]
		函数 $w=\ov z$.
		\onslide<+->{%
			若把 $z$ 复平面和 $w$ 复平面重叠放置, 则这个变换是关于 $z$ 轴的翻转变换.
		}\onslide<+->{%
			它把任一区域映成和它全等的区域, 且 $u=x,v=-y$.
		}
		\tcblower
		\onslide<3->{%
		\begin{tikzpicture}
			\begin{scope}[scale=.8]
				\draw[cstcurve,cstnra,third] (-2.5,1) to[bend right=30] (-2.5,-1);
				\draw[third] (-3.3,0) node[rotate=90] {$w=\ov z$};
				\draw[cstaxis] (-2.5,0)--(2,0) node[above] {$x,u$};
				\draw[cstaxis] (-2,-2)--(-2,2) node[right] {$y,v$};
				\draw (0,-2) node[below] {$z,w$ 平面重叠};
				\begin{scope}[fourth,shift={(0,.7)}]
					\draw[cstcurve,smooth cycle] plot coordinates {(-1.5,0) (-1.7,-.4) (-.3,-.9) (.5,-.7) (.9,0) (1.1,1) (-.3,1.2) (-.7,1)};
					\coordinate (a) at (-1.2,-.3);
					\coordinate (b) at (.6,.9);
					\coordinate (c) at ($(a)!.2!(b)$);
					\coordinate (d) at ($(a)!.5!(b)$);
					\coordinate (e) at ($(a)!.8!(b)$);
					\draw[cstcurve] (a)--(b);
				\end{scope}
				\begin{scope}[second,shift={(0,-.7)}]
					\draw[cstcurve,smooth cycle] plot coordinates {(-1.5,0) (-1.7,.4) (-.3,.9) (.5,.7) (.9,0) (1.1,-1) (-.3,-1.2) (-.7,-1)};
					\coordinate (A) at (-1.2,.3);
					\coordinate (B) at (.6,-.9);
					\coordinate (C) at ($(A)!.2!(B)$);
					\coordinate (D) at ($(A)!.5!(B)$);
					\coordinate (E) at ($(A)!.8!(B)$);
					\draw[cstcurve] (A)--(B);
				\end{scope}
				\begin{scope}[cstdash,cstra,third]
					\draw (c) to (C);
					\draw (d) to (D);
					\draw (e) to (E);
				\end{scope}
			\end{scope}
			\begin{scope}[cstdot,fourth]
				\fill (c) circle;
				\fill (d) circle;
				\fill (e) circle;
			\end{scope}
			\begin{scope}[cstdot,second]
				\fill (C) circle;
				\fill (D) circle;
				\fill (E) circle;
			\end{scope}
		\end{tikzpicture}}
	\end{example}
\end{frame}


\begin{frame}{例题: 旋转和相似变换}
	\onslide<+->
	\begin{example}[near]
		函数 $w=az$.
		\onslide<+->{%
			设 $a=r\ee^{\ii\theta}$, 则这个变换是一个旋转变换(逆时针旋转 $\theta$)和一个相似变换(放大为 $r$ 倍)的复合.
		}\onslide<+->{%
			它把任一区域映成和它相似的区域.
		}
		\tcblower
		\onslide<3->{%
		\begin{center}
			\begin{tikzpicture}[scale=.8]
				\draw[cstcurve,cstnra,third] (-1,0)-- node[above=2pt] {$w=az$} (1,0);
				\begin{scope}[shift={(-4,0)}]
					\draw[cstaxis] (-2,0)--(2,0) node[above] {$x$};
					\draw[cstaxis] (0,-2)--(0,2) node[right] {$y$};
					\draw[cstcurve,fourth,smooth cycle] plot coordinates {(-1.5,0) (-1.7,-.4) (-.3,-.9) (.5,-.7) (.9,0) (1.1,1) (-.3,1.2) (-.7,1)};
					\coordinate (a) at (-1.2,-.3);
					\coordinate (b) at (.6,.9);
					\coordinate (c) at ($(a)!.2!(b)$);
					\coordinate (d) at ($(a)!.5!(b)$);
					\coordinate (e) at ($(a)!.8!(b)$);
					\draw[cstcurve,fourth] (a)--(b);
				\end{scope}
				\begin{scope}[shift={(4,0)}]
					\draw[cstaxis] (-2,0)--(2,0) node[above] {$u$};
					\draw[cstaxis] (0,-2)--(0,2) node[right] {$v$};
					\draw[cstcurve,second,smooth cycle,scale=.8,rotate=90] plot coordinates {(-1.5,0) (-1.7,-.4) (-.3,-.9) (.5,-.7) (.9,0) (1.1,1) (-.3,1.2) (-.7,1)};
					\coordinate (A) at (.24,-.96);
					\coordinate (B) at (-.72,.48);
					\coordinate (C) at ($(A)!.2!(B)$);
					\coordinate (D) at ($(A)!.5!(B)$);
					\coordinate (E) at ($(A)!.8!(B)$);
					\draw[cstcurve,second] (A)--(B);
				\end{scope}
				\begin{scope}[cstdash,cstra,third]
					\draw (c) to[bend right=25] (C);
					\draw (d) to[bend right=10] (D);
					\draw (e) to[bend left=20] (E);
				\end{scope}
				\begin{scope}[cstdot,fourth]
					\fill (c) circle;
					\fill (d) circle;
					\fill (e) circle;
				\end{scope}
				\begin{scope}[cstdot,second]
					\fill (C) circle;
					\fill (D) circle;
					\fill (E) circle;
				\end{scope}
			\end{tikzpicture}
		\end{center}}
	\end{example}
\end{frame}


\begin{frame}{例题: 平方变换}
	\onslide<+->
	\begin{example}[near]
		函数 $w=z^2$.
		\onslide<+->{%
			这个变换把 $z$ 的辐角增大一倍, 因此它会把角形区域变换为角形区域, 并将夹角放大一倍.
		}
		\tcblower
		\onslide<3->{%
		\begin{center}
			\begin{tikzpicture}
				\begin{scope}[xshift=-25mm]
					\draw[cstaxis] (-2,0)--(2,0);
					\draw[cstaxis] (0,-1.5)--(0,1.5);
					\draw
						(2,0) node[above] {$x$}
						(0,1.5) node[left] {$y$};
					\fill[cstfille1] (0,0)--({1.5*cos(37.5)},{1.5*sin(37.5)}) arc(37.5:7.5:1.5)--cycle;
					\draw[cstcurve,main] (0,0)--({1.5*cos(37.5)},{1.5*sin(37.5)});
					\draw[cstcurve,main] (0,0)--({1.5*cos(7.5)},{1.5*sin(7.5)});
					\coordinate (a) at (0,1);
					\coordinate (b) at (.8,1.2);
					\coordinate (c) at (-.6,-.3);
				\end{scope}
				\begin{scope}[xshift=25mm]
					\draw[cstaxis] (-2,0)--(2,0);
					\draw[cstaxis] (0,-1.5)--(0,1.5);
					\draw
						(2,0) node[above] {$u$}
						(0,1.5) node[left] {$v$};
					\fill[cstfille2] (0,0)--({1.8*cos(75)},{1.8*sin(75)}) arc(75:15:1.8)--cycle;
					\draw[cstcurve,second] (0,0)--({1.8*cos(75)},{1.8*sin(75)});
					\draw[cstcurve,second] (0,0)--({1.8*cos(15)},{1.8*sin(15)});
					\coordinate (A) at (-1,0);
					\coordinate (B) at (-.8,1.92);
					\coordinate (C) at (.27,.36);
				\end{scope}
				\draw[cstdash,smooth,third,cstra] (a) to[bend left=10] (A);
				\draw[cstdash,smooth,third,cstra] (b) to[bend left=20] (B);
				\draw[cstdash,smooth,third,cstra] (c) to[bend right=25] (C);
				\fill[cstdot,fill=main] (a) circle;
				\fill[cstdot,fill=main] (b) circle;
				\fill[cstdot,fill=main] (c) circle;
				\fill[cstdot,fill=second] (A) circle;
				\fill[cstdot,fill=second] (B) circle;
				\fill[cstdot,fill=second] (C) circle;
			\end{tikzpicture}
		\end{center}}
	\end{example}
\end{frame}


\begin{frame}{例题: 平方变换}
	\onslide<+->
	\begin{example*}[near][]%
		由于 $u=x^2-y^2,v=2xy$.
		\onslide<+->{%
			因此它把 $z$ 复平面上以直线 $y=\pm x$ 为渐近线的等轴双曲线 $x^2-y^2=c_1$ 变换为 $w$ 复平面上的直线 $u=c_1$,
		}\onslide<+->{%
			把 $z$ 复平面上以坐标轴为渐近线的等轴双曲线 $
			2xy=c_2$ 变换为 $w$ 复平面上的直线 $v=c_2$.
		}
		\tcblower
		\onslide<3->{%
		\begin{center}
			\begin{tikzpicture}[scale=.9]
				\begin{scope}[xshift=-25mm]
					\draw[cstaxis] (-2,0)--(2,0);
					\draw[cstaxis] (0,-1.5)--(0,1.5);
					\begin{scope}[cstcurve,main,smooth]
						\draw (-1.2,-1.2)--(1.2,1.2);
						\draw (-1.2,1.2)--(1.2,-1.2);
						\draw[domain=-35:35]
							plot ({sec(\x)},{tan(\x)})
							plot ({-sec(\x)},{tan(\x)})
							plot ({tan(\x)},{sec(\x)})
							plot ({tan(\x)},{-sec(\x)});
						\draw[domain=-46:46]
							plot ({(.8*sec(\x))},{0.8*tan(\x)})
							plot ({(-.8*sec(\x))},{0.8*tan(\x)})
							plot ({0.8*tan(\x)},{0.8*sec(\x)})
							plot ({0.8*tan(\x)},{0.8*-sec(\x)});
						\draw[domain=-57:57]
							plot ({(.6*sec(\x))},{0.6*tan(\x)})
							plot ({(-.6*sec(\x))},{0.6*tan(\x)})
							plot ({0.6*tan(\x)},{0.6*sec(\x)})
							plot ({0.6*tan(\x)},{0.6*-sec(\x)});
						\draw[domain=-68:68]
							plot ({(.4*sec(\x))},{0.4*tan(\x)})
							plot ({(-.4*sec(\x))},{0.4*tan(\x)})
							plot ({0.4*tan(\x)},{0.4*sec(\x)})
							plot ({0.4*tan(\x)},{0.4*-sec(\x)});
					\end{scope}
					\begin{scope}[cstcurve,second,smooth,rotate=45,visible on=<4->]
						\draw (-1.2,-1.2)--(1.2,1.2);
						\draw (-1.2,1.2)--(1.2,-1.2);
						\draw[domain=-35:35]
							plot ({sec(\x)},{tan(\x)})
							plot ({-sec(\x)},{tan(\x)})
							plot ({tan(\x)},{sec(\x)})
							plot ({tan(\x)},{-sec(\x)});
						\draw[domain=-46:46]
							plot ({(.8*sec(\x))},{0.8*tan(\x)})
							plot ({(-.8*sec(\x))},{0.8*tan(\x)})
							plot ({0.8*tan(\x)},{0.8*sec(\x)})
							plot ({0.8*tan(\x)},{0.8*-sec(\x)});
						\draw[domain=-57:57]
							plot ({(.6*sec(\x))},{0.6*tan(\x)})
							plot ({(-.6*sec(\x))},{0.6*tan(\x)})
							plot ({0.6*tan(\x)},{0.6*sec(\x)})
							plot ({0.6*tan(\x)},{0.6*-sec(\x)});
						\draw[domain=-68:68]
							plot ({(.4*sec(\x))},{0.4*tan(\x)})
							plot ({(-.4*sec(\x))},{0.4*tan(\x)})
							plot ({0.4*tan(\x)},{0.4*sec(\x)})
							plot ({0.4*tan(\x)},{0.4*-sec(\x)});
					\end{scope}
				\end{scope}
				\begin{scope}[xshift=25mm,visible on=<3->]
					\draw[cstaxis] (-2,0)--(2,0);
					\draw[cstaxis] (0,-1.5)--(0,1.5);
					\begin{scope}[cstcurve,second,visible on=<4->]
						\draw (-1.3,-1.2)--(1.3,-1.2);
						\draw (-1.3,-0.9)--(1.3,-.9);
						\draw (-1.3,-0.6)--(1.3,-.6);
						\draw (-1.3,-0.3)--(1.3,-.3);
						\draw (-1.3,0)--(1.3,0);
						\draw (-1.3,0.3)--(1.3,.3);
						\draw (-1.3,0.6)--(1.3,.6);
						\draw (-1.3,0.9)--(1.3,.9);
						\draw (-1.3,1.2)--(1.3,1.2);
					\end{scope}
					\begin{scope}[cstcurve,main,rotate=90]
						\draw (-1.3,-1.2)--(1.3,-1.2);
						\draw (-1.3,-0.9)--(1.3,-.9);
						\draw (-1.3,-0.6)--(1.3,-.6);
						\draw (-1.3,-0.3)--(1.3,-.3);
						\draw (-1.3,0)--(1.3,0);
						\draw (-1.3,0.3)--(1.3,.3);
						\draw (-1.3,0.6)--(1.3,.6);
						\draw (-1.3,0.9)--(1.3,.9);
						\draw (-1.3,1.2)--(1.3,1.2);
					\end{scope}
				\end{scope}
			\end{tikzpicture}
		\end{center}}
		\bigdel
	\end{example*}
\end{frame}


\begin{frame}{例题: 变换的像}
	\onslide<+->
	\begin{example}[near]
		求线段 $0<\abs{z}<3,\arg z=\dfrac\pi4$ 在变换 $w=z^2$ 下的像.
	\end{example}
	\onslide<+->
	\begin{solution}[nearprev]
		设 $z=r\ee^{\frac{\pi\ii}4}$, 则 $w=z^2=r^2\ee^{\frac{\pi\ii}2}=\ii r^2$.
		\onslide<+->{%
			因此它的像是连接 $0$ 和 $9\ii$ 的线段:
			\[
				0<\abs{w}<9,\quad \arg w=\frac\pi2.
			\]
		}\bigdel\smalldel
	\end{solution}
	\onslide<+->
	\begin{example}[nearnext]
		求双曲线 $x^2-y^2=2$ 在变换 $w=z^2$ 下的像.
	\end{example}
	\onslide<+->
	\begin{solution}[nearprev]
		由于
		\[
			w=u+\ii v=z^2=(x^2-y^2)+2xy\ii.
		\]
		\onslide<+->{%
			因此 $u=x^2-y^2=2$.
		}
		\onslide<+->{%
			由于任意 $2+\ii v$ 均存在平方根, 因此所求的像就是直线 $\Re w=2$.
		}
	\end{solution}
\end{frame}


\begin{frame}{例题: 变换的像}
	\onslide<+->
	\begin{example}[nearnext]
		求扇形区域 $0<\arg z<\dfrac\pi3,0<\abs{z}<2$ 在变换 $w=z^2$ 下的像.
	\end{example}
	\onslide<+->
	\begin{solution}[nearprev]
		设 $z=r\ee^{\ii\theta}$, 则 $w=r^2\ee^{2\ii\theta}$.
		\onslide<+->{%
			因此它的像是扇形区域
			\[
				0<\arg w<\frac{2\pi}3,\quad 0<\abs{w}<4.
			\]
		}\bigdel
	\end{solution}
\end{frame}


\begin{frame}{例题: 变换的像}
	\onslide<+->
	\begin{example}[nearnext]
		求圆周 $\abs{z}=2$ 在映射 $w=\dfrac{z+1}{z-1}$ 下的像.
	\end{example}
	\onslide<+->
	\begin{solution}[nearprev]
		不难看出 $z=\dfrac{w+1}{w-1}$.
		\onslide<+->{%
			由 $\biggabs{\dfrac{w+1}{w-1}}=2$ 可知 $\abs{w+1}=2\abs{w-1}$.
		}\onslide<+->{%
			从而
			\[
				w\ov w+w+\ov w+1=4w\ov w-4w-4\ov w+4.
			\]
		}\onslide<+->{%
			\[
				w\ov w-\dfrac53 w-\dfrac53\ov w+1=0.
			\]
		}\onslide<+->{%
			于是 $\biggabs{w-\dfrac53}^2=\dfrac{16}9$, 即 $\biggabs{w-\dfrac53}=\dfrac43$, 它是一个圆周.
		}\meddel
	\end{solution}
\end{frame}


\begin{frame}{例题: 变换的像}
	\onslide<+->
	形如
		\[
			f(z)=\frac{az+b}{cz+d}
		\]
		的映射叫作\emph{分式线性映射}, 其中 $ad\neq bc$.
	\onslide<+->
	它总把直线和圆映成直线或圆.
	\onslide<+->
	\begin{center}
		\begin{tikzpicture}[scale=.4]
			\draw[cstcurve,cstnra,third] (-1.5,0)-- node[above] {$w=1/z$} (1.5,0);
			\begin{scope}[shift={(-7,0)}]
				\begin{scope}[cstcurve,fourth]
					\filldraw[fill=fourth!15] (-2,0) circle (2);
					\filldraw[fill=fourth!35] (-1,0) circle (1);
					\filldraw[fill=fourth!55] (-.5,0) circle (.5);
					\filldraw[fill=fourth!75] (-.3333,0) circle (.3333);
				\end{scope}
				\begin{scope}[cstcurve,main]
					\filldraw[fill=main!15] (2,0) circle (2);
					\filldraw[fill=main!35] (1,0) circle (1);
					\filldraw[fill=main!55] (.5,0) circle (.5);
					\filldraw[fill=main!75] (.3333,0) circle (.3333);
				\end{scope}
				\draw[cstaxis] (-4.8,0)--(4.8,0);
				\draw[cstaxis] (0,-3)--(0,3);
			\end{scope}
			\begin{scope}[shift={(7,0)}]
				\begin{scope}[cstcurve,fourth]
					\fill[fourth!75] (-4,-3) rectangle (-3,3);
					\fill[fourth!55] (-3,-3) rectangle (-2,3);
					\fill[fourth!35] (-2,-3) rectangle (-1,3);
					\fill[fourth!15] (-1,-3) rectangle (-.5,3);
				\end{scope}
				\begin{scope}[cstcurve,main]
					\fill[main!15] (.5,-3) rectangle (1,3);
					\fill[main!35] (1,-3) rectangle (2,3);
					\fill[main!55] (2,-3) rectangle (3,3);
					\fill[main!75] (3,-3) rectangle (4,3);
				\end{scope}
				\begin{scope}[cstcurve,fourth]
					\draw (-.5,-3)--(-.5,3);
					\draw (-1,-3)--(-1,3);
					\draw (-2,-3)--(-2,3);
					\draw (-3,-3)--(-3,3);
				\end{scope}
				\begin{scope}[cstcurve,main]
					\draw (3,-3)--(3,3);
					\draw (2,-3)--(2,3);
					\draw (1,-3)--(1,3);
					\draw (.5,-3)--(.5,3);
				\end{scope}
				\draw[cstaxis] (-4.8,0)--(4.8,0);
				\draw[cstaxis] (0,-3)--(0,3);
			\end{scope}
		\end{tikzpicture}
	\end{center}
	\onslide<+->
	\begin{exercise}
		直线在分式线性映射 $w=\dfrac{az+b}{cz+d}$ 下的像一定经过复数\fillblankframe[3em][3mm]{$\dfrac ac$}.
	\end{exercise}
\end{frame}

