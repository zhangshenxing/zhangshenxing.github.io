\section{复变函数}

\subsection{复变函数的定义}
\begin{frame}{复变函数的定义}
	\onslide<+->
	所谓的映射, 就是两个集合之间的一种对应 $f:A\to B$, 使得对于每一个 $a\in A$, 有一个唯一确定的 $b=f(a)$ 与之对应.
	\begin{itemize}
		\item 当 $A$ 和 $B$ 都是实数集合的子集时, 它就是一个实变函数.
		\item 当 $A$ 和 $B$ 都是复数集合的子集时, 它就是一个\emph{复变函数}.
	\end{itemize}

	\onslide<+->
	\begin{example}
		$f(z)=\Re z,\arg z,|z|,z^n,\dfrac{z+1}{z^2+1}$ 都是复变函数.
	\end{example}

	\onslide<+->
	\begin{definition}
		\begin{itemize}
			\item 称 $A$ 为 函数 $f$ 的\emph{定义域}.
			\item 称 $\set{w=f(z):z\in A}$ 为它的\emph{值域}.
		\end{itemize} 
	\end{definition}
	\onslide<+->
	上述函数的定义域和值域分别是什么?
\end{frame}


\begin{frame}{多值复变函数}
	\onslide<+->
	在复变函数理论中, 我们常常会遇到\emph{多值的复变函数}, 也就是说一个 $z\in G$ 可能有多个 $w$ 与之对应.
	\onslide<+->
	例如 $\Arg z,\sqrt[n]z$ 等.
	\onslide<+->
	为了方便研究, 我们常常需要对每一个 $z$, 选取固定的一个 $f(z)$ 的值.
	\onslide<+->
	这样我们得到了这个多值函数的一个\emph{单值分支}.
	\onslide<+->
	\begin{example}
		$\arg z$ 是无穷多值函数 $\Arg z$ 的一个单值分支.
	\end{example}

	\onslide<+->
	在考虑多值的情况下, 复变函数总有反函数.
	\onslide<+->
	如果 $f$ 和 $f^{-1}$ 都是单值的, 则称 $f$ 是\emph{一一对应}.
	\onslide<+->
	\begin{example}
		$f(z)=z^n$ 的反函数就是 $f^{-1}(w)=\sqrt[n]{w}$.
		\onslide<+->{当 $n=\pm1$ 时, $f$ 是一一对应.}
	\end{example}

	\onslide<+->
	若无特别声明, \alert{复变函数总是指单值的复变函数}.
\end{frame}


\subsection{映照}
\begin{frame}{映照}
	\onslide<+->
	大部分复变函数的图像无法在三维空间中表示出来.
	\onslide<+->
	为了直观理解和研究, 我们用两个复平面($z$ 复平面和 $w$ 复平面)之间的\emph{映照}来表示这种对应关系,
	\onslide<+->
	其中 
	\[w=u+iv=u(x,y)+iv(x,y)\]
	的实部和虚部是两个二元实变函数.
	\onslide<+->
	\begin{center}
		\begin{tikzpicture}
			\draw[cstaxis] (-4.5,0)--(-0.5,0);
			\draw[cstaxis] (-2.5,-1.5)--(-2.5,1.5);
			\draw[cstaxis] (0.5,0)--(4.5,0);
			\draw[cstaxis] (2.5,-1.5)--(2.5,1.5);
			\draw[cstcurve,dcolora,smooth] plot coordinates {(-4,0) (-4.2,-0.4) (-2.8,-0.9) (-2,-0.7) (-1.6,0) (-1.4,1) (-2.8,1.2) (-3.2,1) (-4,0)};
			\draw[cstcurve,smooth,dcolorb] plot coordinates {(1.2,0) (2,-0.5) (2.5,-0.8) (3,-0.5) (3.5,0) (3.8,0.9) (3.3,1.2) (2,0.8) (1.2,0)};
			\draw[cstdash,smooth,dcolorc,cstarrowto] (-2,0.8) to [bend left=25] (2.8,0.7);
			\draw[cstdash,smooth,dcolorc,cstarrowto] (-2.3,0.5)to [bend right=15] (2.2,-0.3);
			\draw[cstdash,smooth,dcolorc,cstarrowto] (-2.8,0.3) to [bend right=25] (2.2,-0.3);
			\fill[cstdot,dcolora] (-2,0.8) circle;
			\fill[cstdot,dcolora] (-2.3,0.5) circle;
			\fill[cstdot,dcolora] (-2.8,0.3) circle;
			\fill[cstdot,dcolorb] (2.8,0.7) circle;
			\fill[cstdot,dcolorb] (2.2,-0.3) circle;
			\draw
			(-1.5,-1) node[dcolora] {$z$ 复平面}
			(-2.7,-0.2) node {$0$}
			(-0.5,-0.2) node {$x$}
			(-2.7,1.5) node {$y$}
			(3.5,-1) node[dcolorb] {$w$ 复平面}
			(2.3,0.2) node {$0$}
			(4.5,-0.2) node {$u$}
			(2.3,1.5) node {$v$};
		\end{tikzpicture}
	\end{center}
\end{frame}


\begin{frame}{例题: 映照}
	\onslide<+->
	\begin{example}
		函数 $w=\ov z$.
		\onslide<+->{如果把 $z$ 复平面和 $w$ 复平面重叠放置, 则这个映照对应的是关于 $z$ 轴的翻转变换.
		}\onslide<+->{它把任一区域映成和它全等的区域, 且 $u=x,v=-y$.}
	\end{example}

	\onslide<+->
	\begin{center}
		\begin{tikzpicture}
			\draw[cstaxis] (-4.5,0)--(-0.5,0);
			\draw[cstaxis] (-2.5,-1.5)--(-2.5,1.5);
			\draw[cstaxis] (0.5,0)--(4.5,0);
			\draw[cstaxis] (2.5,-1.5)--(2.5,1.5);
			\draw[cstcurve,dcolora,smooth] plot coordinates {(-4,0) (-4.2,-0.4) (-2.8,-0.9) (-2,-0.7) (-1.6,0) (-1.4,1) (-2.8,1.2) (-3.2,1) (-4,0)};
			\draw[cstcurve,dcolorb,smooth] plot coordinates {(1,0) (0.8,0.4) (2.2,0.9) (3,0.7) (3.4,0) (3.6,-1) (2.2,-1.2) (1.8,-1) (1,0)};
			\draw[cstdash,smooth,dcolorc,cstarrowto] (-2,0.8) to[bend left=45] (3,-0.8);
			\draw[cstdash,smooth,dcolorc,cstarrowto] (-2.8,0.3) to[bend right=25] (2.2,-0.3);
			\draw[cstdash,smooth,dcolorc,cstarrowto] (-3.6,-0.2) to[bend right=15] (1.4,0.2);
			\draw[cstcurve,dcolora] (-1.84,0.9)--(-3.76,-0.3);
			\draw[cstcurve,dcolorb] (3.16,-0.9)--(1.24,0.3);
			\fill[cstdot,dcolora] (-2,0.8) circle;
			\fill[cstdot,dcolora] (-2.8,0.3) circle;
			\fill[cstdot,dcolora] (-3.6,-0.2) circle;
			\fill[cstdot,dcolorb] (3,-0.8) circle;
			\fill[cstdot,dcolorb] (2.2,-0.3) circle;
			\fill[cstdot,dcolorb] (1.4,0.2) circle;
			\draw
			(-0.5,0.2) node {$x$}
			(-2.7,1.5) node {$y$}
			(4.5,-0.2) node {$u$}
			(2.3,1.5) node {$v$};
		\end{tikzpicture}
	\end{center}
\end{frame}


\begin{frame}{例题: 映照}
	\onslide<+->
	\begin{example}
		函数 $w=az$.
		\onslide<+->{设 $a=re^{i\theta}$, 则这个映照对应的是一个旋转映照(逆时针旋转 $\theta$)和一个相似映照(放大为 $r$ 倍)的复合.
		}\onslide<+->{它把任一区域映成和它相似的区域.}
	\end{example}
	\onslide<+->
	\begin{center}
		\begin{tikzpicture}
			\draw[cstaxis] (-4.5,0)--(-0.5,0);
			\draw[cstaxis] (-2.5,-1.5)--(-2.5,1.5);
			\draw[cstaxis] (0.5,0)--(4.5,0);
			\draw[cstaxis] (2.5,-1.5)--(2.5,1.5);
			\draw[cstcurve,dcolora,smooth] plot coordinates {(-4,0) (-4.2,-0.4) (-2.8,-0.9) (-2,-0.7) (-1.6,0) (-1.4,1) (-2.8,1.2) (-3.2,1) (-4,0)};
			\draw[cstcurve,dcolorb,smooth] plot coordinates {(2.5,-0.9) (2.74,-1.02) (3.04,-0.18) (2.92,0.3) (2.5,0.54) (1.9,0.66) (1.78,-0.18) (1.9,-0.42) (2.5,-0.9)};
			\draw[cstdash,smooth,dcolorc,cstarrowto] (-2,0.8) to[bend left=20] (2.02,0.3);
			\draw[cstdash,smooth,dcolorc,cstarrowto] (-2.8,0.3) to[bend right=10] (2.32,-0.18);
			\draw[cstdash,smooth,dcolorc,cstarrowto] (-3.6,-0.2) to[bend right=25] (2.62,-0.66);
			\draw[cstcurve,dcolora] (-1.84,0.9)--(-3.76,-0.3);
			\draw[cstcurve,dcolorb] (1.96,0.396)--(2.68,-0.756);
			\fill[cstdot,dcolora] (-2,0.8) circle;
			\fill[cstdot,dcolora] (-2.8,0.3) circle;
			\fill[cstdot,dcolora] (-3.6,-0.2) circle;
			\fill[cstdot,dcolorb] (2.02,0.3) circle;
			\fill[cstdot,dcolorb] (2.32,-0.18) circle;
			\fill[cstdot,dcolorb] (2.62,-0.66) circle;
			\draw
			(-0.5,0.2) node {$x$}
			(-2.7,1.5) node {$y$}
			(4.5,-0.2) node {$u$}
			(2.3,1.5) node {$v$};
		\end{tikzpicture}
	\end{center}
\end{frame}


\begin{frame}{例题: 映照}
	\onslide<+->
	\begin{example}
		函数 $w=z^2$.
		\onslide<+->{这个映照把 $z$ 的辐角增大一倍, 因此它会把角形区域变换为角形区域, 并将夹角放大一倍.
		}\onslide<+->{
		\begin{center}
			\begin{tikzpicture}
				\draw[cstaxis] (-4.5,0)--(-0.5,0);
				\draw[cstaxis] (-2.5,-1)--(-2.5,2);
				\draw[cstaxis] (0.5,0)--(4.5,0);
				\draw[cstaxis] (2.5,-1)--(2.5,2);
				\draw[cstdash,smooth,dcolorc,cstarrowto] (-2.5,1) to[bend left=10] (1.5,0);
				\draw[cstdash,smooth,dcolorc,cstarrowto] (-1.7,1.2) to[bend left=20] (1.7,1.92);
				\draw[cstdash,smooth,dcolorc,cstarrowto] (-3.1,-0.3) to[bend right=25] (2.77,0.36);
				\fill[cstdot,fill=dcolora] (-2.5,1) circle;
				\fill[cstdot,fill=dcolora] (-1.7,1.2) circle;
				\fill[cstdot,fill=dcolora] (-3.1,-0.3) circle;
				\fill[cstdot,fill=dcolorb] (1.5,0) circle;
				\fill[cstdot,fill=dcolorb] (1.7,1.92) circle;
				\fill[cstdot,fill=dcolorb] (2.77,0.36) circle;
				\fill[cstfille,pattern color=dcolora,visible on=<4->] (-2.5,0)--(-1.31,0.913) arc(37.5:7.5:1.5)--cycle;
				\fill[cstfille,pattern color=dcolorb,visible on=<4->] (2.5,0)--(2.966,1.74) arc(75:15:1.8)--cycle;
				\draw[cstcurve,dcolora,visible on=<4->] (-2.5,0) --(-1.15,1.035);
				\draw[cstcurve,dcolora,visible on=<4->] (-2.5,0) --(-0.814,0.222);
				\draw[cstcurve,dcolorb,visible on=<4->] (2.5,0) --(3.018,1.93);
				\draw[cstcurve,dcolorb,visible on=<4->] (2.5,0) --(4.43,0.518);
				\draw
				(-2.7,-0.2) node {$0$}
				(-0.5,-0.2) node {$x$}
				(-2.7,1.8) node {$y$}
				(2.3,0.2) node {$0$}
				(4.5,-0.2) node {$u$}
				(2.3,1.8) node {$v$};
			\end{tikzpicture}
		\end{center}}
	\end{example}
\end{frame}


\begin{frame}{例题: 映照}
	\onslide<+->
	\begin{example}[续]
		由于 $u=x^2-y^2,v=2xy$.
		\onslide<+->{因此它把 $z$ 复平面上两族分别以直线 $y=\pm x$ 和坐标轴为渐近线的等轴双曲线 $x^2-y^2=c_1,2xy=c_2$
		}\onslide<+->{分别映射为 $w$ 复平面上的两族平行直线 $u=c_1,v=c_2$.}
		\onslide<+->{
		\begin{center}
			\begin{tikzpicture}
				\draw[cstaxis] (-4.5,0)--(-0.5,0);
				\draw[cstaxis] (-2.5,-1.6)--(-2.5,1.6);
				\draw[cstaxis] (0.5,0)--(4.5,0);
				\draw[cstaxis] (2.5,-1.6)--(2.5,1.6);
				\draw[cstcurve,dcolorb] (-3.7,-1.2)--(-1.3,1.2);
				\draw[cstcurve,dcolorb] (-3.7,1.2)--(-1.3,-1.2);
				\draw[cstcurve,dcolorb,smooth,domain=-35:35]
					plot ({sec(\x)-2.5},{tan(\x)})
					plot ({-sec(\x)-2.5},{tan(\x)})
					plot ({tan(\x)-2.5},{sec(\x)})
					plot ({tan(\x)-2.5},{-sec(\x)});
				\draw[cstcurve,dcolorb,smooth,domain=-46:46]
					plot ({(0.8*sec(\x)-2.5)},{0.8*tan(\x)})
					plot ({(-0.8*sec(\x)-2.5)},{0.8*tan(\x)})
					plot ({0.8*tan(\x)-2.5},{0.8*sec(\x)})
					plot ({0.8*tan(\x)-2.5},{0.8*-sec(\x)});
				\draw[cstcurve,dcolorb,smooth,domain=-57:57]
					plot ({(0.6*sec(\x)-2.5)},{0.6*tan(\x)})
					plot ({(-0.6*sec(\x)-2.5)},{0.6*tan(\x)})
					plot ({0.6*tan(\x)-2.5},{0.6*sec(\x)})
					plot ({0.6*tan(\x)-2.5},{0.6*-sec(\x)});
				\draw[cstcurve,dcolorb,smooth,domain=-68:68]
					plot ({(0.4*sec(\x)-2.5)},{0.4*tan(\x)})
					plot ({(-0.4*sec(\x)-2.5)},{0.4*tan(\x)})
					plot ({0.4*tan(\x)-2.5},{0.4*sec(\x)})
					plot ({0.4*tan(\x)-2.5},{0.4*-sec(\x)});

				\draw[cstcurve,dcolora,visible on=<4->] (-4,0)--(-1,0);
				\draw[cstcurve,dcolora,visible on=<4->] (-2.5,-1.3)--(-2.5,1.3);
				\draw[cstcurve,dcolora,visible on=<4->,smooth,domain=-34:34]
					plot ({0.8*(sec(\x)+tan(\x))-2.5},{0.8*(sec(\x)-tan(\x))})
					plot ({0.8*(sec(\x)+tan(\x))-2.5},{-0.8*(sec(\x)-tan(\x))})
					plot ({-0.8*(sec(\x)+tan(\x))-2.5},{0.8*(sec(\x)-tan(\x))})
					plot ({-0.8*(sec(\x)+tan(\x))-2.5},{-0.8*(sec(\x)-tan(\x))});
				\draw[cstcurve,dcolora,visible on=<4->,smooth,domain=-45:45]
					plot ({0.6*(sec(\x)+tan(\x))-2.5},{0.6*(sec(\x)-tan(\x))})
					plot ({0.6*(sec(\x)+tan(\x))-2.5},{-0.6*(sec(\x)-tan(\x))})
					plot ({-0.6*(sec(\x)+tan(\x))-2.5},{0.6*(sec(\x)-tan(\x))})
					plot ({-0.6*(sec(\x)+tan(\x))-2.5},{-0.6*(sec(\x)-tan(\x))});
				\draw[cstcurve,dcolora,visible on=<4->,smooth,domain=-57:57]
					plot ({0.4*(sec(\x)+tan(\x))-2.5},{0.4*(sec(\x)-tan(\x))})
					plot ({0.4*(sec(\x)+tan(\x))-2.5},{-0.4*(sec(\x)-tan(\x))})
					plot ({-0.4*(sec(\x)+tan(\x))-2.5},{0.4*(sec(\x)-tan(\x))})
					plot ({-0.4*(sec(\x)+tan(\x))-2.5},{-0.4*(sec(\x)-tan(\x))});
				\draw[cstcurve,dcolora,visible on=<4->,smooth,domain=-71:71]
					plot ({0.2*(sec(\x)+tan(\x))-2.5},{0.2*(sec(\x)-tan(\x))})
					plot ({0.2*(sec(\x)+tan(\x))-2.5},{-0.2*(sec(\x)-tan(\x))})
					plot ({-0.2*(sec(\x)+tan(\x))-2.5},{0.2*(sec(\x)-tan(\x))})
					plot ({-0.2*(sec(\x)+tan(\x))-2.5},{-0.2*(sec(\x)-tan(\x))});

				\draw[cstcurve,dcolorb] (1.3,-1.3)--(1.3,1.3);
				\draw[cstcurve,dcolorb] (1.6,-1.3)--(1.6,1.3);
				\draw[cstcurve,dcolorb] (1.9,-1.3)--(1.9,1.3);
				\draw[cstcurve,dcolorb] (2.2,-1.3)--(2.2,1.3);
				\draw[cstcurve,dcolorb] (2.5,-1.3)--(2.5,1.3);
				\draw[cstcurve,dcolorb] (2.8,-1.3)--(2.8,1.3);
				\draw[cstcurve,dcolorb] (3.1,-1.3)--(3.1,1.3);
				\draw[cstcurve,dcolorb] (3.4,-1.3)--(3.4,1.3);
				\draw[cstcurve,dcolorb] (3.7,-1.3)--(3.7,1.3);

				\draw[cstcurve,dcolora,visible on=<4->] (1.2,-1.2)--(3.8,-1.2);
				\draw[cstcurve,dcolora,visible on=<4->] (1.2,-0.9)--(3.8,-0.9);
				\draw[cstcurve,dcolora,visible on=<4->] (1.2,-0.6)--(3.8,-0.6);
				\draw[cstcurve,dcolora,visible on=<4->] (1.2,-0.3)--(3.8,-0.3);
				\draw[cstcurve,dcolora,visible on=<4->] (1.2,0)--(3.8,0);
				\draw[cstcurve,dcolora,visible on=<4->] (1.2,0.3)--(3.8,0.3);
				\draw[cstcurve,dcolora,visible on=<4->] (1.2,0.6)--(3.8,0.6);
				\draw[cstcurve,dcolora,visible on=<4->] (1.2,0.9)--(3.8,0.9);
				\draw[cstcurve,dcolora,visible on=<4->] (1.2,1.2)--(3.8,1.2);
			\end{tikzpicture}
		\end{center}}
	\end{example}
\end{frame}


\begin{frame}{例题: 映照的像}
	\onslide<+->
	\begin{example}
		求下列集合在映照 $w=z^2$ 下的像.

		\enumnum1 线段 $0<|z|<2,\arg z=\dfrac\pi2$.
	\end{example}

	\onslide<+->
	\begin{solution}
		设 $z=re^{\frac{\pi i}2}=ir$, 则 $w=z^2=-r^2$.
		\onslide<+->{因此它的像还是一条线段 $0<|w|<4,\arg w=-\pi$.
		}\onslide<+->{
		\begin{center}
			\begin{tikzpicture}
				\draw[cstaxis] (-4.5,0)--(-0.5,0);
				\draw[cstaxis] (-2.5,-0.2)--(-2.5,1.6);
				\draw[cstaxis] (0.5,0)--(4.5,0);
				\draw[cstaxis] (2.5,-0.2)--(2.5,1.6);
				\draw[cstcurve,dcolora] (-2.5,0)--(-2.5,1);
				\filldraw[cstdote,draw=dcolora] (-2.5,0) circle;
				\filldraw[cstdote,draw=dcolora] (-2.5,1) circle;
				\draw[cstcurve,dcolorb] (0.5,0)--(2.5,0);
				\filldraw[cstdote,draw=dcolorb] (0.5,0) circle;
				\filldraw[cstdote,draw=dcolorb] (2.5,0) circle;
				\draw[cstdash,cstarrowto,dcolorc] (-2.2,0.5) to[bend left] (1.5,0.5);
				\draw
				(-0.5,-0.2) node {$x$}
				(-2.7,1.3) node {$y$}
				(4.5,-0.2) node {$u$}
				(2.3,1.3) node {$v$};
			\end{tikzpicture}
		\end{center}}
	\end{solution}
\end{frame}


\begin{frame}{例题: 映照的像}
	\onslide<+->
	\begin{example}
		求下列集合在映照 $w=z^2$ 下的像.

		\enumnum2 双曲线 $x^2-y^2=4$.
	\end{example}

	\onslide<+->
	\begin{solution*}
		由于 \[w=u+iv=z^2=(x^2-y^2)+2xyi.\]
		\onslide<+->{因此 $u=x^2-y^2=4,v=2xy$.}

		\onslide<+->{可以说明当 $u=4$ 时, 对任意 $v$, $u+iv$ 都是该双曲线上某一点的像.
		}\onslide<+->{所以这条双曲线的像是直线 $\Re w=4$.}
	\end{solution*}
\end{frame}


\begin{frame}{例题: 映照的像}
	\onslide<+->
	\begin{example}
		求下列集合在映照 $w=z^2$ 下的像.

		\enumnum3 扇形区域 $0<\arg z<\dfrac\pi4,0<|z|<2$.
	\end{example}

	\onslide<+->
	\begin{solution}
		设 $z=re^{i\theta}$, 则 $w=r^2e^{2i\theta}$.
		\onslide<+->{因此它的像是扇形区域 $0<\arg w<\dfrac\pi2,0<|w|<4$.}
	\end{solution}
\end{frame}


\begin{frame}{例题: 映照的像}
	\onslide<+->
	\begin{example}
		求圆周 $|z|=2$ 在映照 $w=\dfrac{z+1}{z-1}$ 下的像.
	\end{example}

	\onslide<+->
	\begin{solution}
		由于 $z=\dfrac{w+1}{w-1}$, $\abs{\dfrac{w+1}{w-1}}=2$,
		\onslide<+->{因此
		\[|w+1|=2|w-1|,\quad w\ov w+w+\ov w+1=4w\ov w-4w-4\ov w+4,\]}
		\onslide<+->{
			\[w\ov w-\frac53 w-\frac53\ov w+1=0,\quad \abs{w-\frac53}^2=\dfrac{16}9,\]
		}\onslide<+->{即 $\abs{w-\dfrac53}=\dfrac43$, 是一个圆周.}
	\end{solution}
\end{frame}

