\section{傅里叶变换}

\subsection{傅里叶级数}

\begin{frame}{周期函数的傅里叶级数展开}
	\onslide<+->
	为了引入傅里叶变换, 我们回顾下傅里叶级数展开.

	\onslide<+->
	设 $f(t)$ 是定义在 $(-\infty,+\infty)$ 上周期为 $T$ 的可积实变函数.
	\onslide<+->
	我们知道 $\cos n\omega t$ 和 $\sin n\omega t$ 周期也是 $T$, 其中 $\omega=\dfrac{2\pi}T$.
	\onslide<+->
	如果 $f(t)$ 在 $\left[-\dfrac T2,\dfrac T2\right]$ 上满足\emph{狄利克雷条件}:
	\begin{itemize}
		\item 间断点只有有限多个, 且均为第一类间断点;
		\item 只有有限个极值点,
	\end{itemize}
	\onslide<+->
	则我们有\emph{傅里叶级数}展开:
	\[f(t)=\frac{a_0}2+\sum_{n=1}^\infty \left(a_n\cos n\omega t+b_n \sin n\omega t\right).\]
	\onslide<+->
	当 $t$ 是间断点时, 傅里叶级数的左侧需改为 $\dfrac{f(t+)+f(t-)}2$.
\end{frame}


\begin{frame}{傅里叶级数的复指数形式}
	\onslide<+->
	我们来将其改写为复指数形式.
	\onslide<+->
	物理中为了与电流 $i$ 区分, 通常用 $j$ 来表示虚数单位.
	\onslide<+->
	由
	\[\cos x=\frac{e^{jx}+e^{-jx}}2,\quad \sin x=\frac{e^{jx}-e^{-jx}}{2j}\]
	\onslide<+->
	可知 $f(t)$ 的傅里叶级数可以表示为函数 $e^{jn\omega t}$ 的线性组合.
	\onslide<+->
	设 $f(t)=\suml_{n=-\infty}^{+\infty} c_n e^{jn\omega t}$, 则
	\[c_n=\frac 1T \int_{-\frac T2}^{\frac T2} f(t)e^{-jn\omega t} \diff t.\]
	\onslide<+->
	于是我们得到周期函数\emph{傅里叶级数的复指数形式}:
	\[f(t)=\frac 1T\sum_{n=-\infty}^{+\infty}\left[\int_{-\frac T2}^{\frac T2}f(\tau)e^{-jn\omega\tau} \diff\tau\right] e^{jn\omega t}.\]
\end{frame}


\begin{frame}{从傅里叶级数到傅里叶积分公式}
	\onslide<+->
	对于一般的函数 $f(t)$, 它未必是周期的.
	\onslide<+->
	我们考虑它在 $\left[-\dfrac T2,\dfrac T2\right]$ 上的限制, 并向两边扩展成一个周期函数 $f_T(t)$.
	\onslide<+->
	设
	\[\omega_n=n\omega,\quad \Delta\omega_n=\omega_n-\omega_{n-1}=\omega,\]
	则
	\begin{align*}
		f(t)&=\lim_{T\to +\infty}f_T(t)\\
		&\visible<+->{=\lim_{T\to+\infty} \frac 1T \sum_{n=-\infty}^{+\infty} \left[\int_{-\frac T2}^{\frac T2}f(\tau)e^{-j\omega_n\tau}\diff\tau\right] e^{j\omega_n t}}\\
		&\visible<+->{=\frac1{2\pi}\lim_{\Delta\omega_n\to 0}\sum_{n=-\infty}^{+\infty}\emph{\left[\int_{-\frac T2}^{\frac T2}f(\tau)e^{-j\omega_n\tau} \diff \tau\right] e^{j\omega_n t}}\Delta\omega_n}\\
		&\visible<+->{=\frac1{2\pi}\int_{-\infty}^{+\infty}\emph{\left[\int_{-\infty}^{+\infty}f(\tau) e^{-j\omega\tau} \diff \tau\right]e^{j\omega t}}\diff\omega.}
	\end{align*}
\end{frame}


\subsection{傅里叶积分与傅里叶变换}

\begin{frame}{傅里叶积分定理}
	\onslide<+->
	\begin{main}{傅里叶积分定理}
			设函数 $f(t)$ 满足
			\begin{itemize}
				\item 在 $(-\infty,+\infty)$ 上\emph{绝对可积};
				\item 在任一有限区间上满足狄利克雷条件.
			\end{itemize}
		\onslide<+->{那么
			\[\alert{f(t)=\frac1{2\pi} \int_{-\infty}^{+\infty}F(\omega) e^{j\omega t} \diff\omega,\qquad
			F(\omega)=\int_{-\infty}^{+\infty}f(t) e^{-j\omega t} \diff t.}\]
		}\onslide<+->{对于 $f(t)$ 的间断点左边需要改成 $\frac{f(t+)+f(t-)}2$.
		}
	\end{main}
	\onslide<+->
	\begin{center}
		\begin{tikzpicture}[node distance=100pt]
			\node[cstnode2](a){原象函数 $f(t)$};
			\node[cstnode2,right=of a](b){象函数 $F(\omega)$};
			\draw[cstnra, main] (a.9) -- node[above]{傅里叶变换 $\msf$} (b.170);
			\draw[cstnra, third] (b.190) -- node[below]{傅里叶逆变换 $\msf^{-1}$} (a.-9);
		\end{tikzpicture}
	\end{center}
\end{frame}


\begin{frame}{傅里叶积分公式的三角形式\noexer}
	\onslide<+->
	傅里叶积分公式有一些变化形式.
	\onslide<+->
	例如:
	\begin{align*}
		&f(t)=\frac1{2\pi} \int_{-\infty}^{+\infty}\left[\int_{-\infty}^{+\infty}f(\tau) e^{-j\omega\tau} \diff\tau\right] e^{j\omega t} \diff\omega\\
		\visible<+->{=}&\visible<.->{\frac1{2\pi} \int_{-\infty}^{+\infty}\int_{-\infty}^{+\infty}f(\tau) e^{j\omega(t-\tau)}\diff\tau \diff\omega}\\
		\visible<+->{=}&\visible<.->{\frac1{2\pi} \int_{-\infty}^{+\infty}\Bigl[
			\resizebox{!}{5mm}{
				$\displaystyle\underbrace{\int_{-\infty}^{+\infty}f(\tau)\cos \omega(t-\tau) \diff\tau}_{\text{\small$\omega$ 的偶函数}}
				+j\underbrace{\int_{-\infty}^{+\infty}f(\tau)\sin \omega(t-\tau) \diff\tau}_{\text{\small$\omega$ 的奇函数}}$
			}\Bigr] \diff\omega}\\
		&\visible<+->{\alert{=\frac1\pi\int_0^{+\infty}\left[\int_{-\infty}^{+\infty}f(\tau) \cos \omega(t-\tau)\diff\tau\right]\diff\omega.}}
	\end{align*}
	\onslide<+->
	此即\emph{傅里叶积分公式的三角形式}.
\end{frame}


\begin{frame}{傅里叶正弦/余弦积分公式\noexer}
	\onslide<+->
	对上式再次展开得到:
	\[f(t)=\frac1\pi\int_0^{+\infty}\left[\int_{-\infty}^{+\infty}f(\tau)(\cos\omega t\cos\omega\tau+\sin\omega t\sin \omega\tau)\diff\tau\right]\diff\omega.\]

	\onslide<+->
	若 $f(t)$ 是奇函数, $f(\tau)\cos\omega\tau$ 是 $\tau$ 的奇函数, $f(\tau)\sin \omega\tau$ 是 $\tau$ 的偶函数.
	\onslide<+->
	从而得到\emph{傅里叶正弦积分公式}:
	\[\alert{f(t)=\frac2\pi\int_0^{+\infty}\left[\int_0^{+\infty}f(\tau)\sin \omega\tau\diff\tau\right]\sin\omega t\diff\omega.}\]

	\onslide<+->
	类似地, 若 $f(t)$ 是偶函数, 有\emph{傅里叶余弦积分公式}:
	\[\alert{f(t)=\frac2\pi\int_0^{+\infty}\left[\int_0^{+\infty}f(\tau)\cos\omega\tau\diff\tau\right]\cos\omega t\diff\omega.}\]
\end{frame}


\begin{frame}{例: 求傅里叶变换}
	\onslide<+->
	\begin{example}
		求函数 $f(t)=
			\begin{cases}
				1, & |t|\le 1,\\
				0, & |t|>1
			\end{cases}$
		的傅里叶变换.
	\end{example}

	\onslide<+->
	\begin{solution}
		\vspace{-\baselineskip}
		\begin{align*}
			F(\omega)&=\msf[f(t)]=\int_{-\infty}^{+\infty}f(t)e^{-j\omega t}\diff t\\
			&\visible<+->{=\int_{-1}^1(\cos\omega t-j\sin\omega t)\diff t}\visible<+->{=\frac{2\sin \omega}{\omega}.}
		\end{align*}
	\end{solution}
\end{frame}


\begin{frame}{例: 求傅里叶变换}
	\onslide<+->
	由傅里叶积分公式
	\begin{align*}
		f(t)&=\msf^{-1}[F(\omega)]=\frac1{2\pi}\int_{-\infty}^{+\infty}F(\omega)e^{j\omega t}\diff\omega\\
		&\visible<+->{=\frac1{2\pi}\int_{-\infty}^{+\infty}\frac{2\sin\omega}{\omega}(\cos\omega t+j\sin\omega t)\diff \omega}\\
		&\visible<+->{=\frac2\pi\int_0^{+\infty}\frac{\sin\omega\cos\omega t}{\omega}\diff \omega.}
	\end{align*}
	\onslide<+->
	当 $t=\pm1$ 时, 左侧应替换为 $\bigl(f(t+)+f(t-)\bigr)/2=1/2 $.
	\onslide<+->
	由此可得
	\[\int_0^{+\infty}\frac{\sin \omega\cos\omega t}\omega\diff\omega=\begin{cases}
		\pi/2,&|t|<1,\\
		\pi/4,&|t|=1,\\
		0,&|t|>1.
	\end{cases}\]
	\onslide<+->
	特别地, 可以得到狄利克雷积分
	$\displaystyle\int_0^{+\infty}\frac{\sin\omega}\omega\diff\omega=\frac\pi2$.
\end{frame}


\begin{frame}{例: 求傅里叶变换}
	\onslide<+->
	\begin{example}
		求函数 $f(t)=\dfrac{\sin t}{t}$ 的傅里叶变换.
	\end{example}

	\onslide<+->
	\begin{solution}
		根据前面的例子可知
		\vspace{-\baselineskip}
		\begin{align*}
			F(\omega)&=\msf[f(t)]=\int_{-\infty}^{+\infty}f(t)e^{-j\omega t}\diff t\\
			&\visible<+->{=2\int_0^{+\infty}\frac{\sin t\cos\omega t}{t}\diff t}
			\visible<+->{=\begin{cases}
				\pi,&|\omega|<1,\\
				\pi/2,&|\omega|=1,\\
				0,&|\omega|>1.
				\end{cases}}
		\end{align*}
	\end{solution}
\end{frame}


\begin{frame}{例: 求傅里叶变换}
	\onslide<+->
	\begin{example}
		求函数 $f(t)=
			\begin{cases}
				1,&t\in(0,1),\\
				-1,&t\in(-1,0),\\
				0,&\text{其它情形}
			\end{cases}$
		的傅里叶变换.
	\end{example}

	\onslide<+->
	\begin{solution}
		\vspace{-\baselineskip}
		\begin{align*}
			F(\omega)&=\msf[f(t)]=\int_{-\infty}^{+\infty}f(t)e^{-j\omega t}\diff t\\
			&\visible<+->{=\left(\int_0^1-\int_{-1}^0\right)(\cos\omega t-j\sin\omega t)\diff t}
			\visible<+->{=-\frac{2j(1-\cos\omega)}\omega.}&\visible<.->{}
		\end{align*}
	\end{solution}

	\onslide<+->
	类似可得
	$\displaystyle\int_0^{+\infty}\frac{(1-\cos\omega)\sin\omega t}\omega\diff\omega=
		\begin{cases}
			\pi/2,&0<t<1,\\
			\pi/4,&t=1,\\
			0,&t>1.
		\end{cases}$
\end{frame}


\begin{frame}{例: 求傅里叶变换}
	\onslide<+->
	\begin{example}
		求\emph{指数衰减函数} $f(t)=
			\begin{cases}
				0,&t<0,\\
				e^{-\beta t},&t\ge 0
			\end{cases}$ 的傅里叶变换, $\beta>0$.
	\end{example}

	\onslide<+->
	\begin{solution}
		\vspace{-\baselineskip}
		\begin{align*}
			F(\omega)&=\msf[f(t)]=\int_{-\infty}^{+\infty}f(t)e^{-j\omega t}\diff t\\
			&\visible<+->{=\int_0^{+\infty}e^{-\beta t}e^{-j\omega t}\diff t}
			\visible<+->{=\int_0^{+\infty}e^{-(\beta+j\omega)t}\diff t}
			\visible<+->{\alert{=\frac1{\beta+j\omega}}.}&\visible<.->{}
		\end{align*}
	\end{solution}

	\onslide<+->
	类似可得
	$\displaystyle\int_0^{+\infty}\frac{\beta\cos\omega t+\omega\sin\omega t}{\beta^2+\omega^2}\diff\omega=
		\begin{cases}
			0,&t<0,\\
			\pi/2,&t=0,\\
			\pi e^{-\beta t},&t>0.
		\end{cases}$
\end{frame}


\begin{frame}{例: 求傅里叶变换}
	\onslide<+->
	\begin{example}
		求\emph{钟形脉冲函数} $f(t)=e^{-\beta t^2}$ 的傅里叶变换和积分表达式, $\beta>0$.
	\end{example}

	\onslide<+->
	\begin{solution}
		\vspace{-\baselineskip}
		\begin{align*}
			F(\omega)&=\msf[f(t)]=\int_{-\infty}^{+\infty}f(t)e^{-j\omega t}\diff t\\
			&\visible<+->{=\int_{-\infty}^{+\infty}e^{-\beta t^2}e^{-j\omega t}\diff t}\\
			&\visible<+->{=e^{-\frac{\omega^2}{4\beta}}\int_{-\infty}^{+\infty}\exp\left[-\beta\left(t+\frac{j\omega}{2\beta}\right)^2\right]\diff t}
			\visible<+->{\alert{=\sqrt{\frac\pi\beta}e^{-\frac{\omega^2}{4\beta}}.}}&\visible<.->{}
		\end{align*}
	\end{solution}

	\onslide<+->
	类似可得
	$\displaystyle\int_0^{+\infty}e^{-\frac{\omega^2}{4\beta}}\cos\omega t\diff\omega=\sqrt{\pi\beta}e^{-\beta t^2}$.
\end{frame}


\subsection{狄拉克 \texorpdfstring{$\delta$}{δ} 函数}

\begin{frame}{广义函数}
	\onslide<+->
	傅里叶变换存在的条件是比较苛刻的.
	\onslide<+->
	例如常值函数 $f(t)=1$ 在 $(-\infty,+\infty)$ 上不是可积的, 所以它没有傅里叶变换, 这很影响我们使用傅里叶变换.
	\onslide<+->
	为此我们引入广义函数的概念.

	\onslide<+->
	设 $\msc$ 是一些函数形成的线性空间, 例如全体绝对可积函数, 或者全体光滑函数之类的. 
	\onslide<+->
	从一个函数 $\lambda(t)$ 出发, 可以定义一个线性映射 $\msc\to \BR$:
	\[\pair{\lambda,f}:=\int_{-\infty}^{+\infty}\lambda(t)f(t)\diff t.\]
	\onslide<+->
	这个线性映射基本上确定了 $\lambda(t)$ 本身(至多可数个点处不同).

	\onslide<+->
	\emph{广义函数}就是指一个线性映射 $\msc\to \BR$.
	\onslide<+->
	为了和普通函数类比, 通常也将广义函数表为上述积分形式(并不是真的积分):
	\[\int_{-\infty}^{+\infty}\lambda(t)f(t)\diff t.\]
	这里的 $\lambda(t)$ 并不表示一个真正的函数.
\end{frame}


\begin{frame}{狄拉克 $\delta$ 函数}
	\onslide<+->
	\begin{definition}
		\emph{$\delta$ 函数}是指广义函数
		\[\pair{\delta,f}=\int_{-\infty}^{+\infty}\delta(t)f(t)\diff t=f(0).\]
	\end{definition}

	\onslide<6->
	\begin{tikzpicture}[overlay]\begin{tikzpicture}[xshift=-190mm,yshift=10mm]
		\draw[cstaxis](8.2,0)--(9.8,0);
		\draw[cstaxis](9,-0.6)--(9,2);
		\draw
			(9.8,-0.25) node {$t$}
			(8.7,1.6) node {$1$}
			(9.5,1.6) node[main] {$\delta(t)$}
			(8.8,-0.2) node {$O$};
		\draw[main,cstcurve,cstra](9,0)--(9,1.5);
	\end{tikzpicture}\end{tikzpicture}
	\vspace{-5\baselineskip}

	\onslide<+->
	设 $\delta_\varepsilon(t)=\begin{cases}
	1/\varepsilon,&0\le t\le \varepsilon,\\
	0,&\text{其它情形,}
	\end{cases}$
	\onslide<+->
	则对于连续函数 $f(t)$,
	\[\pair{\delta_\varepsilon,f}=\frac1\varepsilon\int_0^\varepsilon f(t)\diff t=f(\xi),\quad \xi\in(0,\varepsilon).\]
	\onslide<+->
	当 $\varepsilon\to0$ 时, 右侧就趋于 $f(0)$.
	\onslide<+->
	因此 $\delta$ 可以看成 $\delta_\varepsilon$ 的某种极限.
	\onslide<+->
	基于此, 我们通常用长度为 $1$ 的有向线段来表示它.
\end{frame}


\begin{frame}{狄拉克 $\delta$ 函数的性质}
	\onslide<+->
	对于广义函数 $\lambda$, 我们可以形式地定义 $\lambda(at),\lambda'$:
	\[\int_{-\infty}^{+\infty}\lambda(at)f(t)\diff t
	=\int_{-\infty}^{+\infty}\lambda(t)\cdot\frac1{|a|}f\bigl(\frac ta\bigr)\diff t,\]
	\[\int_{-\infty}^{+\infty}\lambda'(t)f(t)\diff t
	=-\int_{-\infty}^{+\infty}\lambda(t)f'(t)\diff t.\]
	\onslide<+->
	由此可知
	\begin{itemize}
		\item $\pair{\delta^{(n)},f}=(-1)^nf^{(n)}(0)$, 其中 $f(t)$ 是光滑函数.
		\item $\delta(at)=\dfrac1{|a|}\delta(t)$. 特别地 $\delta(t)=\delta(-t)$.
		\item $u'(t)=\delta(t)$, 其中 $u(t)=\begin{cases}1,&t\ge0,\\0,&t<0\end{cases}$ 是\emph{单位阶跃函数}.
	\end{itemize}
\end{frame}


\begin{frame}{Dirac $\delta$ 函数的傅里叶变换和逆变换}
	\onslide<+->
	根据 $\delta$ 函数的定义可知
	\[\msf[\delta(t)]=\int_{-\infty}^{+\infty}\delta(t)e^{-j\omega t}\diff t=1.\]
	\onslide<+->
	同理可得其傅里叶逆变换.
	\onslide<+->
	因此我们得到:
	\begin{second@}
		\[\msf[\delta(t)]=1,\qquad
		\msf^{-1}[\delta(\omega)]=\dfrac1{2\pi}.\]
	\end{second@}
\end{frame}


\begin{frame}{例: 求傅里叶变换}
	\onslide<+->
	\begin{example}
		证明 \alert{$\msf[u(t)]=\dfrac1{j\omega}+\pi\delta(\omega)$}.
	\end{example}

	\onslide<+->
	\begin{proof}
			\[\msf^{-1}\left[\frac1{j\omega}\right]
			=\frac1{2\pi}\int_{-\infty}^{+\infty}\frac{e^{j\omega t}}{j\omega} \diff\omega
			=\frac1\pi\int_0^{+\infty}\frac{\sin\omega t}{\omega} \diff\omega.\]
		\onslide<+->{由
			$\displaystyle\int_0^{+\infty}\frac{\sin\omega}\omega \diff\omega=\dfrac\pi2$
			可知
			$\displaystyle\int_0^{+\infty}\frac{\sin\omega t}\omega \diff\omega=\frac\pi2\sgn(t)$.
		}\onslide<+->{故
			\[\msf^{-1} \left[\frac1{j\omega}+\pi\delta(\omega)\right]
			=\half\sgn(t)+\half =u(t)\quad (t\neq 0).\qedhere\]
		}
	\end{proof}
\end{frame}


\subsection{傅里叶变换的性质}


\begin{frame}{傅里叶变换的性质: 线性性质}
	\onslide<+->
	我们不可能也没必要每次都对需要变换的函数从定义出发计算傅里叶变换.
	\onslide<+->
	通过研究傅里叶变换的性质, 结合常见函数的傅里叶变换, 我们可以得到很多情形的傅里叶变换.

	\onslide<+->
	\begin{main}{线性性质}
		\[\msf[\alpha f+\beta g]=\alpha F+\beta G,\quad
		\msf^{-1}[\alpha F+\beta G]=\alpha f+\beta g.\]
	\end{main}

	\onslide<+->
	\begin{main}{位移性质}
		\[\msf[f(t-t_0)]=e^{-j\omega t_0}F(\omega),\quad
		\msf^{-1}[F(\omega-\omega_0)]=e^{j\omega_0 t}f(t).\]
		\vspace{-\baselineskip}
	\end{main}
\end{frame}


\begin{frame}{傅里叶变换的性质: 位移性质}
	\onslide<+->
	\begin{proof}
		\begin{align*}
			\msf[f(t-t_0)]&=\int_{-\infty}^{+\infty}f(t-t_0)e^{-j\omega t}\diff t\\
			&=\int_{-\infty}^{+\infty}f(t)e^{-j\omega (t+t_0)}\diff t=e^{-j\omega t_0}F(\omega).
		\end{align*}
		\onslide<+->{逆变换情形类似可得.\qedhere}
	\end{proof}

	\onslide<+->
	由此可得
	\[\msf[\delta(t-t_0)]=e^{-j\omega t_0},\quad
	\msf^{-1}[\delta(\omega-\omega_0)]=\dfrac1{2\pi}e^{j\omega_0 t}.\]
\end{frame}


\begin{frame}{傅里叶变换的性质: 微分性质}
	\onslide<+->
	\begin{main}{微分性质}
		\[\msf[f'(t)]=j\omega F(\omega),\quad
		\msf^{-1}[F'(\omega)]=-jtf(t),\]
		\vspace{-\baselineskip}
		\[\msf[f^{(k)}(t)]=(j\omega)^k F(\omega),\quad
		\msf^{-1}[F^{(k)}(\omega)]=(-jt)^kf(t).\]
		这里, 被变换的函数要求在 $\infty$ 处趋于 $0$.
	\end{main}

	\onslide<+->
	\begin{proof}
		\begin{align*}
			\msf[f']&=\int_{-\infty}^{+\infty}f'(t)e^{-j\omega t}\diff t\\
			&=-\int_{-\infty}^{+\infty}f(t)(e^{-j\omega t})'\diff t=j\omega F(\omega)
		\end{align*}
		\onslide<+->{逆变换情形类似可得. 一般的 $k$ 归纳可得.\qedhere}
	\end{proof}
\end{frame}


\begin{frame}{傅里叶变换的性质}
	\onslide<+->
	由微分性质可得
	\begin{main}{乘多项式性质}
		\[\msf[tf(t)]=jF'(\omega),\quad
		\msf^{-1}[\omega F(\omega)]=-jf'(t),\]
		\[\msf[t^kf(t)]=j^kF^{(k)}(\omega),\quad
		\msf^{-1}[\omega^kF(\omega)]=(-j)^kf^{(k)}(t).\]
	\end{main}

	\onslide<+->
	\begin{main}{积分性质}
		\[\msf\left[\int_{-\infty}^t f(\tau)\diff\tau\right]=\frac1{j\omega}F(\omega).\]
	\end{main}

	\onslide<+->
	由变量替换易得
	\begin{main}{相似性质}
		\[\msf[f(at)]=\frac1{|a|}F\left(\frac\omega a\right),\quad
		\msf^{-1}[F(a\omega)]=\frac1{|a|}f\left(\frac t a\right).\]
	\end{main}
\end{frame}


\begin{frame}{典型例题: 计算傅里叶变换}
	\onslide<+->
	\begin{example}
		求 $\msf[t^k e^{-\beta t}u(t)],\beta>0$.
	\end{example}

	\onslide<+->
	\begin{solution}
			由于
			\[\msf[e^{-\beta t}u(t)]=\frac{1}{\beta+j\omega},\]
		\onslide<+->{因此
			\begin{align*}
				\msf[t^ke^{-\beta t}u(t)]&=j^k\left(\frac1{\beta+j\omega}\right)^{(k)}
				=\frac{k!}{(\beta+j\omega)^{k+1}}.
			\end{align*}
		}
		\vspace{-\baselineskip}
	\end{solution}
\end{frame}


\begin{frame}{典型例题: 计算傅里叶变换}
	\onslide<+->
	\begin{example}
		求 $\sin{\omega_0 t}$ 的傅里叶变换.
	\end{example}

	\onslide<+->
	\begin{solution}
			由于 $\msf[1]=2\pi\delta(\omega)$,
		\onslide<+->{因此 $\msf[e^{j\omega_0t}]=2\pi\delta(\omega-\omega_0)$,
		}
		\onslide<+->{\begin{align*}
				\msf[\sin{\omega_0 t}]&=\frac1{2j}\left[\msf[e^{j\omega_0t}]-\msf[e^{-j\omega_0t}]\right]&\\
				&\visible<+->{=\frac1{2j}[2\pi\delta(\omega-\omega_0)-2\pi\delta(\omega+\omega_0)]}&\\
				&\visible<+->{=j\pi[\delta(\omega+\omega_0)-\delta(\omega-\omega_0)].}
			\end{align*}
		}
		\vspace{-1.3\baselineskip}
	\end{solution}

	\onslide<+->
	\begin{exercise}
		$\msf[\cos{\omega_0 t}]=$\fillblank[5cm][1mm]{\visible<+->{$\pi[\delta(\omega+\omega_0)+\delta(\omega-\omega_0)]$}}.
	\end{exercise}
\end{frame}


\begin{frame}{常见傅里叶变换汇总}
	\onslide<+->
	\begin{second}{常见傅里叶变换汇总 I}
		\[\msf[\delta(t)]=1,\quad \msf[\delta(t-t_0)]=e^{-j\omega t_0},\]
		\[\msf[1]=2\pi\delta(\omega),\quad \msf[e^{j\omega_0 t}]=2\pi\delta(\omega-\omega_0).\]
		\begin{align*}
		\msf[\sin \omega_0t]&=j\pi[\delta(\omega+\omega_0)-\delta(\omega-\omega_0)],\\
		\msf[\cos{\omega_0 t}]&=\pi[\delta(\omega+\omega_0)+\delta(\omega-\omega_0)].
		\end{align*}
	\end{second}

	\onslide<+->
	\begin{main}{常见傅里叶变换汇总 II}
		\[\msf[u(t)e^{-\beta t}]=\dfrac1{\beta+j\omega},\quad
		\msf[e^{-\beta t^2}]=\sqrt{\dfrac\pi\beta}e^{-\omega^2/(4\beta)},\]
		\[\msf[u(t)]=\dfrac1{j\omega}+\pi\delta(\omega).\]
	\end{main}
\end{frame}


\subsection{卷积*}
\begin{frame}{卷积\noexer}
	\onslide<+->
	\begin{definition}
		$f_1(t),f_2(t)$ 的\emph{卷积}是指
		\[\alert{(f_1\ast f_2)(t)=\int_{-\infty}^{+\infty} f_1(\tau)f_2(t-\tau)\diff \tau.}\]
	\end{definition}

	\onslide<+->
	容易验证卷积满足如下性质:
	\begin{itemize}
		\item $f_1\ast f_2=f_2\ast f_1,\ (f_1\ast f_2)\ast f_3=f_1\ast(f_2\ast f_3)$;
		\item $f_1\ast(f_2+f_3)=f_1\ast f_2+f_1\ast f_3$;
		\item $f\ast\delta=f$;
		\item $(f_1\ast f_2)'=f_1'\ast f_2=f_1\ast f_2'$.
	\end{itemize}
\end{frame}


\begin{frame}{例: 计算卷积\noexer}
	\onslide<+->
	\begin{example}
		设 $f_1(t)=u(t),f_2(t)=e^{-t}u(t)$. 求 $f_1\ast f_2$.
	\end{example}

	\onslide<+->
	\begin{solution}
			\[(f_1\ast f_2)(t)=\int_{-\infty}^{+\infty} f_2(\tau)f_1(t-\tau)\diff \tau=\int_0^{+\infty} e^{-\tau}u(t-\tau)\diff \tau.\]
		\onslide<+->{当 $t<0$ 时, $(f_1\ast f_2)(t)=0$.
		}\onslide<+->{当 $t\ge0$ 时, 
			\[(f_1\ast f_2)(t)=\int_0^t e^{-\tau}\diff \tau=1-e^{-t}.\]
		}\onslide<+->{故 $(f_1\ast f_2)(t)=(1-e^{-t})u(t)$.
		}
	\end{solution}
\end{frame}


\begin{frame}{卷积定理\noexer}
	\onslide<+->
	\begin{main}{卷积定理}
	\[\msf[f_1\ast f_2]=F_1\cdot F_2,\quad
	\msf^{-1}[F_1\ast F_2]=\frac1{2\pi}f_1\cdot f_2.\]
	\vspace{-\baselineskip}
	\end{main}

	\onslide<+->
	\begin{proof}
		\vspace{-\baselineskip}
		\begin{align*}
			\msf[f_1\ast f_2]&=\int_{-\infty}^{+\infty}\int_{-\infty}^{+\infty}f_1(\tau)f_2(t-\tau)\diff \tau \cdot e^{-j\omega t}\diff t\\
			&\visible<+->{=\int_{-\infty}^{+\infty}\int_{-\infty}^{+\infty}f_1(\tau)e^{-j\omega \tau}\cdot f_2(t-\tau)e^{-j\omega (t-\tau)}\diff t\diff \tau}\\
			&\visible<+->{=\int_{-\infty}^{+\infty}\int_{-\infty}^{+\infty}f_1(\tau)e^{-j\omega \tau}\cdot f_2(t)e^{-j\omega t}\diff t\diff \tau}\\
			&\visible<+->{=\int_{-\infty}^{+\infty}f_1(\tau)e^{-j\omega \tau}\diff\tau\int_{-\infty}^{+\infty}f_2(t)e^{-j\omega t}\diff t}\\
			&\visible<+->{=\msf[f_1]\msf[f_2].\qedhere}
		\end{align*}
	\end{proof}
\end{frame}


\begin{frame}{卷积的应用\noexer}
	\onslide<+->
	\begin{example}
		求 $\displaystyle	I=\int_{-\infty}^{+\infty}\frac{\sin \omega}{\omega}\cdot \frac{\sin (\omega/3)}{\omega/3}\diff\omega$.
	\end{example}

	\onslide<+->
	\begin{solution}
			设 $F(\omega)=\dfrac{\sin\omega}{\omega},G(\omega)=\dfrac{\sin(\omega/3)}{\omega/3}$,%
		\onslide<+->{则
			\[\msf^{-1}[FG]=\frac1{2\pi}\int_{-\infty}^{+\infty}F(\omega)G(\omega)e^{j\omega t}\diff\omega,\]
		}\onslide<+->{
			\[\msf^{-1}[FG](0)=\frac1{2\pi}\int_{-\infty}^{+\infty}F(\omega)G(\omega)\diff\omega=\frac1{2\pi}I.\]
		}
	\end{solution}
\end{frame}


\begin{frame}{卷积的应用\noexer}
	\onslide<+->
	\begin{solution}[续解]
			我们之前计算过
			\[\msf^{-1}[F(\omega)]=f(t)=\begin{cases}
				1/2, & |t|<1,\\
				1/4, & |t|=1,\\
				0, & |t|>1.
			\end{cases}\]
		\onslide<+->{所以 $\msf^{-1}[G(\omega)]=g(t)=3f(3t)$,
		}\onslide<+->{
			\begin{align*}
				\msf^{-1}[FG](0)&=(f\ast g)(0)\\
				&=\int_{-\infty}^{+\infty}f(-t)g(t)\diff t=\half\int_{-1}^1 g(t)\diff t=\half ,
			\end{align*}
		}\onslide<+->{故 $I=2\pi\msf^{-1}[FG](0)=\pi$.}
	\end{solution}
\end{frame}


\subsection{傅里叶变换的应用*}
\begin{frame}{使用傅里叶变换解微积分方程\noexer}
	\begin{center}
		\begin{tikzpicture}[node distance=40pt]
			\node[cstnode2] (a){微分方程或积分方程};
			\node[cstnode1,right=110pt of a] (b){象函数的代数方程};
			\node[cstnode2,below=of a] (c){原象函数(方程的解)};
			\node[cstnode1,below=of b] (d){象函数};
			\draw[cstnra,second] (a)--node[above]{傅里叶变换 $\msf$}(b);
			\draw[cstnra,second] (d)--node[below]{傅里叶逆变换 $\msf^{-1}$}(c);
			\draw[cstnra,second] (b)--(d);
			\draw[cstnra,third] (a)--(c);
		\end{tikzpicture}
	\end{center}
\end{frame}


\begin{frame}{例: 使用傅里叶变换解微积分方程\noexer}
	\beqskip{1pt}
	\onslide<+->
	\begin{example}
		解方程 $y'(t)-\displaystyle\int_{-\infty}^t y(\tau)\diff \tau=2\delta(t)$.
	\end{example}

	\onslide<+->
	\begin{solution}
			设 $\msf[y]=Y$.
		\onslide<+->{两边同时作傅里叶变换得到
			\[j\omega Y(\omega)-\frac1{j\omega}Y(\omega)=2,\]
		}\onslide<+->{
			\[Y(\omega)=-\frac{2j\omega}{1+\omega^2}=\frac1{1+j\omega}-\frac1{1-j\omega},\]
		}\onslide<+->{
			\begin{align*}
				y(t)=\msf^{-1}\left[\frac1{1+j\omega}-\frac1{1-j\omega}\right]
				&\visible<+->{=\begin{cases}
				0-e^t=-e^t,&t<0,\\
				0,&t=0,\\
				e^{-t}-0=e^{-t},&t>0.
				\end{cases}}
			\end{align*}
		}
	\end{solution}
	\endgroup
\end{frame}


\begin{frame}{例: 使用傅里叶变换解微积分方程\noexer}
	\onslide<+->
	\begin{example}
		解方程 $y''(t)-y(t)=0$.
	\end{example}

	\onslide<+->
	\begin{solution}
			设 $\msf[y]=Y$,
		\onslide<+->{则
			\[\msf[y''(t)-y(t)]=[(j\omega)^2-1]Y(\omega)=0,\]
		}
		\vspace{-\baselineskip}
		\onslide<+->{\[Y(\omega)=0,\quad y(t)=\msf^{-1}[Y(\omega)]=0.\]}
		\vspace{-\baselineskip}
	\end{solution}

	\onslide<+->
	显然这是不对的, 该方程的解应该是 $y(t)=C_1e^t+C_2e^{-t}$.

	\onslide<+->
	原因在于使用傅里叶变换要求函数是绝对可积的, 而 $e^t,e^{-t}$ 并不满足该条件.
	\onslide<+->
	我们需要一个对函数限制更少的积分变换来解决此类方程, 例如拉普拉斯变换.
\end{frame}

