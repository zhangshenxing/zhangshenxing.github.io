\section{傅里叶变换}

\subsection{积分变换的引入}

\begin{frame}{积分变换的引入}
	\onslide<+->
	在学习指数和对数的时候, 我们了解到利用对数可以将乘除、幂次转化为加减、乘除.
	\onslide<+->
	\begin{example}[nearnext]
		计算 $12345\times 67890$.
	\end{example}
	\onslide<+->
	\begin{solution}[nearprev]
		通过查对数表得到
		\[
			\ln 12345\approx 9.4210,\qquad\ln 67890\approx 11.1256.
		\]
		\onslide<+->{%
			将二者相加并通过反查对数表得到原值
			\[
				12345\times 67890\approx \exp(20.5466)\approx 8.3806\times 10^8.
			\]
		}\bigdel
	\end{solution}
\end{frame}


\begin{frame}{积分变换的引入}
	\onslide<+->
	而对于函数而言, 我们常常要解函数的方程.
	\onslide<+->
	\begin{example}[nearnext]
		解微分方程
		\[
			\begin{cases}
				y''+y=t,&\\
				y(0)=y'(0)=0.&
			\end{cases}
		\]
	\end{example}
	\onslide<+->
	\begin{solution}[nearprev,indent]
		我们希望能找到一种函数的\emph{变换 $\msl$}, 使得它可以把函数的微分和积分变成代数运算, 计算之后通过\emph{反变换 $\msl^{-1}$} 求得原来的解.

		\onslide<+->{%
			这个变换最常见的就是我们将要介绍的\emph{傅里叶变换}和\emph{拉普拉斯变换}.
		}
	\end{solution}
\end{frame}


\subsection{傅里叶级数}

\begin{frame}{周期函数的傅里叶级数展开}
	\onslide<+->
	为了引入傅里叶变换, 我们回顾下傅里叶级数.

	\onslide<+->
	考虑定义在 $(-\infty,+\infty)$ 上周期为 $T>0$ 的函数 $f(t)$.
	\onslide<+->
	例如
	\[
		1,\sin{\omega t},\cos{\omega t},
		\sin{2\omega t},\cos{2\omega t},
		\sin{3\omega t},\cos{3\omega t},\dots
	\]
	的周期都是 $T$, 其中 $\omega=\dfrac{2\pi}T$.
	\onslide<+->
	类似于线性组合的概念, 我们希望将 $f$ 表达为上述函数的线性叠加.
	\onslide<+->
	若 $f(t)$ 在 $\left[-\dfrac T2,\dfrac T2\right]$ 上满足\emph{狄利克雷条件}:
	\begin{itemize}
		\item 间断点只有有限多个, 且均为第一类间断点;
		\item 只有有限个极值点,
	\end{itemize}
	\onslide<+->
	则我们有\emph{傅里叶级数}展开:
	\[
		f(t)=\frac{a_0}2+\sumf1 \Bigl(a_n\cos n\omega t+b_n \sin n\omega t\Bigr).
	\]
	\onslide<+->
	当 $t$ 是间断点时, 傅里叶级数的左侧需改为 $\dfrac{f(t+)+f(t-)}2$.
\end{frame}


\begin{frame}{傅里叶级数的复指数形式}
	\onslide<+->
	我们来将其改写为复指数形式.
	\onslide<+->
	由
	\[
		\cos x=\frac{\ee^{\ii x}+\ee^{-\ii x}}2,\quad \sin x=\frac{\ee^{\ii x}-\ee^{-\ii x}}{2\ii }
	\]
	\onslide<+->
	可知 $f(t)$ 的傅里叶级数可以表示为函数 $\ee^{\ii n\omega t}$ 的线性叠加
	\[
		f(t)=\sumff c_n\ee^{\ii n\omega t}.
	\]
	\onslide<+->
	现在我们来计算这个线性叠加的系数.

	\onslide<+->
	对于定义在 $(-\infty,+\infty)$ 上周期为 $T>0$ 的\alert{复值}函数 $f,g$, 定义内积
	\[
		(f,g):=\frac1T\int_{-\frac T2}^{\frac T2} f(t)\ov g(t)\d t.
	\]
\end{frame}


\begin{frame}{傅里叶级数的复指数形式}
	\onslide<+->
	那么
	\[
		(\ee^{\ii m\omega t},\ee^{\ii n\omega t})
		=\frac1T\int_{-\frac T2}^{\frac T2} \ee^{\ii (m-n)\omega t}\d t=\begin{cases}
			1,&m=n\\
			0,&m\neq n.
		\end{cases}
	\]
	\onslide<+->
	所以
	\[
		\cdots,\ee^{-2\ii \omega t},\ee^{-\ii \omega t},1,\ee^{\ii \omega t},\ee^{2\ii \omega t},\cdots
	\]
	是一组标准正交基.
	\onslide<+->
	于是
	\[
		c_n=(f,\ee^{\ii n\omega t})=\frac 1T\int_{-\frac T2}^{\frac T2}f(t)\ee^{-\ii n\omega t} \d t,
	\]
	\onslide<+->
	我们得到周期函数\emph{傅里叶级数的复指数形式}:
	\[
		f(t)=\frac 1T\sumff\left[\int_{-\frac T2}^{\frac T2}f(\tau)\ee^{-\ii n\omega\tau} \d\tau\right] \ee^{\ii n\omega t}.
	\]
\end{frame}


\begin{frame}{帕塞瓦尔恒等式}
	\onslide<+->
	我们可以利用傅里叶级数计算一些级数的和.
	\onslide<+->
	由
	\[
		(f,f)
		=\sum_{m=-\infty}^{+\infty}\sumff 
			c_m\ov{c_n}\bigl(\ee^{\ii m\omega t},\ee^{\ii n\omega t}\bigr)
		=\sumff \abs{c_n}^2
	\]
	\onslide<+->
	可得 (本章中这些用于计算级数和积分的定理会在考试时按需提供):
	\begin{theorem*}[][帕塞瓦尔恒等式]
		对于周期为 $T$ 的函数 $f(t)$, 我们有
		\[
			 \frac1{T}\intT \abs{f(t)}^2\d t
			=\sumff \abs{c_n}^2,
		\]
		其中 $c_n$ 是它的傅里叶系数, 即 $f(t)=\sumff c_n\ee^{\ii n\omega t}$.
	\end{theorem*}
\end{frame}


\begin{frame}{例题: 帕塞瓦尔恒等式的应用}\small
	\onslide<+->
	\begin{example}[near]
		计算 $\sumf1 \dfrac1{n^2}$.
	\end{example}
	\onslide<+->
	\begin{solution}[near,sidepic,righthand width=3.2cm]
		设 $f(t)$ 是一个周期为 $2\pi$ 的函数, 且当 $t\in[-\pi,\pi)$ 时 $f(t)=t$.
		\onslide<+->{%
			当 $n\neq0$ 时,
			\[
				c_n=\frac1{2\pi}\int_{-\pi}^\pi t\ee^{-\ii nt}\d t
				=\frac1{2\pi}\Bigl(\frac{\ii t}n+\frac1{n^2}\Bigr)\ee^{-\ii nt}\Big|_{-\pi}^\pi
				=\frac{(-1)^n\ii}n.
			\]
		}\onslide<+->{%
			\[
				c_0=\frac1{2\pi}\int_{-\pi}^\pi t\d t
				=\frac1{2\pi}\cdot\frac{t^2}2\Big|_{-\pi}^\pi
				=0.
			\]
		}\onslide<+->{%
			因此
			$\displaystyle
				\sum_{n\neq 0}\frac1{n^2}
				=(f,f)
				=\frac1{2\pi}\int_{-\pi}^{\pi}t^2\d t
				=\frac{\pi^2}3,
				\qquad
				\sumf1 \frac1{n^2}=\frac{\pi^2}6.
			$
		}
		\tcblower
		\begin{tikzpicture}[scale=.4]
			\draw[cstaxis] (-4,0)--(4,0);
			\draw[cstaxis] (0,-2)--(0,2);
			\draw[cstcurve,main] (-3,-1)--(-1,1);
			\draw[cstcurve,main] (-1,-1)--(1,1);
			\draw[cstcurve,main] (1,-1)--(3,1);
			\draw[cstdash] (-3,-1)--(-3,0);
			\draw[cstdash] (-1,-1)--(-1,1);
			\draw[cstdash] (1,-1)--(1,1);
			\draw[cstdash] (3,0)--(3,1);
			\draw (0,0) node[below right] {$0$};
			\draw (2,0) node[below right] {$2\pi$};
			\draw (-2,0) node[above left] {$-2\pi$};
		\end{tikzpicture}
	\end{solution}
\end{frame}


\subsection{傅里叶变换}

\begin{frame}{从傅里叶级数到傅里叶积分公式}
	\onslide<+->
	对于一般的函数 $f(t)$, 它未必是周期的.
	\onslide<+->
	此时它无法像前面的情形一样, 表达成可数多个函数
	\[
		\cdots,\ee^{-2\ii \omega t},\ee^{-\ii \omega t},1,\ee^{\ii \omega t},\ee^{2\ii \omega t},\cdots
	\]
	的线性叠加,
	\onslide<+->
	而是\alert{所有的 $\ee^{\ii \omega t},\omega\in(-\infty,+\infty)$ 的叠加}.
	\onslide<+->
	这种叠加的``系数''应当是无穷小方可, 而求和应当改为积分.
	\onslide<+->
	所以, 若记 $\ee^{\ii \omega t}$ 的``系数''为函数 $\dfrac1{2\pi}F(\omega)$, 则
	\[
		f(t)=\frac1{2\pi}\intff  F(\omega) \ee^{\ii \omega t}\d t.
	\]
\end{frame}


\begin{frame}{从傅里叶级数到傅里叶积分公式}
	\onslide<+->
	我们来从傅里叶级数形式地推导出函数 $F(\omega)$.
	\onslide<+->
	考虑 $f(t)$ 它在 $\left[-\dfrac T2,\dfrac T2\right]$ 上的限制, 并向两边扩展成一个周期函数 $f_T(t)$.
	\onslide<+->
	设
	\[
		\omega_n=n\omega,\quad \delt\omega_n=\omega_n-\omega_{n-1}=\omega,
	\]
	则
	\begin{align*}
		f(t)&=\lim_{T\to +\infty}f_T(t)\\
		&\visible<+->{=\lim_{T\to+\infty} \frac 1T \sumff \left[\int_{-\frac T2}^{\frac T2}f(\tau)\ee^{-\ii \omega_n\tau}\d\tau\right] \ee^{\ii \omega_n t}}\\
		&\visible<+->{=\frac1{2\pi}\lim_{\delt\omega_n\to 0}\sumff\emphn{\left[\int_{-\frac T2}^{\frac T2}f(\tau)\ee^{-\ii \omega_n\tau} \d \tau\right] \ee^{\ii \omega_n t}}\delt\omega_n}\\
		&\visible<+->{=\frac1{2\pi}\intff \emphn{\left[\intff f(\tau) \ee^{-\ii \omega\tau} \d \tau\right]\ee^{\ii \omega t}}\d\omega.}
	\end{align*}
\end{frame}



\begin{frame}{傅里叶积分定理}
	\onslide<+->
	\begin{theorem*}[][傅里叶积分定理]
		若 $f(t)$ 在 $(-\infty,+\infty)$ 上\emph{绝对可积}, 且在任一有限区间上满足狄利克雷条件, 则
		\[
			\alertm{f(t)=\frac1{2\pi} \intff F(\omega) \ee^{\ii \omega t} \d\omega,\qquad
			F(\omega)=\intff f(t) \ee^{-\ii \omega t} \d t.}
		\]
		\onslide<+->{%
		对于 $f(t)$ 的间断点左边需要改成 $\frac{f(t+)+f(t-)}2$.
		}
	\end{theorem*}
	\onslide<+->
	\begin{center}
		\begin{tikzpicture}[node distance=100pt]
			\node[cstnode2](a){原象函数 $f(t)$};
			\node[cstnode2,right=of a](b){象函数 $F(\omega)$};
			\draw[cstnra, main] (a.9) -- node[above]{傅里叶变换 $\msf$} (b.170);
			\draw[cstnra, third] (b.190) -- node[below]{傅里叶逆变换 $\msf^{-1}$} (a.-9);
		\end{tikzpicture}
	\end{center}

	\onslide<+->
	若 $f(t)$ 表示随时间 $t$ 变化的函数, 那么 $F(\omega)$ 表示的是频率 $\omega$ 的函数, 所以傅里叶变换是时域到频域的转换.
\end{frame}


\begin{frame}{傅里叶积分公式的三角形式\noexer}
	\onslide<+->
	傅里叶积分公式有一些变化形式.
	\onslide<+->
	例如:
	\begin{align*}
		f(t)&=\frac1{2\pi} \intff \left[\intff f(\tau) \ee^{-\ii \omega\tau} \d\tau\right] \ee^{\ii \omega t} \d\omega\\
		&\visible<+->{=\frac1{2\pi} \intff \intff f(\tau) \ee^{\ii \omega(t-\tau)}\d\tau \d\omega}\\
		&\visible<+->{=\frac1{2\pi} \intff \Bigl[
				\underbrace{\intff f(\tau)\cos \omega(t-\tau) \d\tau}_{\text{\small$\omega$ 的偶函数}}
				+\ii \underbrace{\intff f(\tau)\sin \omega(t-\tau) \d\tau}_{\text{\small$\omega$ 的奇函数}}
			\Bigr] \d\omega}\\
		&\visible<+->{=\alertn{\frac1\pi\int_0^{+\infty}\left[\intff f(\tau) \cos \omega(t-\tau)\d\tau\right]\d\omega.}}
	\end{align*}
	\onslide<+->
	此即\emph{傅里叶积分公式的三角形式}.
\end{frame}


\begin{frame}{傅里叶正弦/余弦积分公式\noexer}
	\onslide<+->
	若 $f(t)$ 是偶函数, 则 $f(t)\cos{\omega t}$ 是偶函数, $f(t)\sin{\omega t}$ 是奇函数,
	\onslide<+->
	\[
		F(\omega)=\intff f(t)\ee^{-\ii \omega t}\d t
	=2\int_0^{+\infty} f(t)\cos{\omega t}\d t
	\]
	也是偶函数,
	\onslide<+->
	从而得到\emph{傅里叶余弦积分公式}:
	\[
		\alertn{f(t)=\frac2\pi\int_0^{+\infty}\left[\int_0^{+\infty}f(\tau)\cos\omega\tau\d\tau\right]\cos\omega t\d\omega.}
	\]

	\onslide<+->
	类似地, 若 $f(t)$ 是奇函数, 则 $\displaystyle F(\omega)=2\ii \int_0^{+\infty} f(t)\sin{\omega t}\d t$ 也是奇函数, 且有\emph{傅里叶正弦积分公式}:
	\[
		\alertn{f(t)=\frac2\pi\int_0^{+\infty}\left[\int_0^{+\infty}f(\tau)\sin \omega\tau\d\tau\right]\sin\omega t\d\omega.}
	\]
\end{frame}


\begin{frame}{例题: 求傅里叶变换}
	\onslide<+->
	\begin{example}[nearnext]
		求矩形脉冲函数 $f(t)=
			\begin{cases}
				1/2, & \abs{t}\le 1,\\
				0, & \abs{t}>1
			\end{cases}$
		的傅里叶变换.
	\end{example}
	\onslide<+->
	\begin{solution}[nearprev]
		由于 $f(t)$ 是偶函数, 因此
		\begin{align*}
			F(\omega)&=\msf[f(t)]
			=\intff f(t)\ee^{-\ii \omega t}\d t
			=\intff f(t)\cos{\omega t}\d t\\
			&\visible<+->{=\frac12\int_{-1}^1\cos{\omega t}\d t}
				\visible<+->{=\frac{\sin \omega}{\omega}.}
		\end{align*}
	\end{solution}
	\onslide<+->
	它的傅里叶变换为 $\sinc$ 函数 $\sinc(x)=\dfrac{\sin x}x$.
\end{frame}


\begin{frame}{例题: 求傅里叶变换}
	\onslide<+->
	由傅里叶积分公式
	\begin{align*}
		f(t)&=\msf^{-1}[F(\omega)]=\frac1{2\pi}\intff F(\omega)\ee^{\ii \omega t}\d\omega\\
		&\visible<+->{=\frac1{2\pi}\intff \frac{\sin\omega}{\omega}(\cos\omega t+\ii \sin\omega t)\d \omega}
		\visible<+->{=\frac1\pi\int_0^{+\infty}\frac{\sin\omega\cos\omega t}{\omega}\d \omega.}
	\end{align*}
	\onslide<+->
	当 $t=\pm1$ 时, 左侧应替换为 $\frac{f(t+)+f(t-)}2=\frac14$.
	\onslide<+->
	由此可得
	\[
		\int_0^{+\infty}\frac{\sin \omega\cos\omega t}\omega\d\omega=\begin{cases}
		\pi/2,&\abs{t}<1,\\
		\pi/4,&\abs{t}=1,\\
		0,&\abs{t}>1.
	\end{cases}
	\]
	\onslide<+->
	特别地, 可以得到狄利克雷积分
	$\dint_0^{+\infty}\frac{\sin\omega}\omega\d\omega=\frac\pi2$.
\end{frame}


\begin{frame}{例题: 求傅里叶变换}
	\onslide<+->
	\begin{center}
		\begin{tikzpicture}
			\begin{scope}
				\draw[cstaxis](-2,0)--(2,0);
				\draw[cstaxis](0,-1.5)--(0,2);
				\draw[cstdash](-1,0)--(-1,1);
				\draw[cstdash](1,0)--(1,1);
				\draw[main,cstcurve](-1,1)--(1,1);
			\end{scope}
			\begin{scope}[xshift=6cm]
				\draw[cstaxis](-2.5,0)--(2.5,0);
				\draw[cstaxis](0,-1.5)--(0,2);
				\draw[cstcurve,second,domain=1:700,smooth] plot ({\x*0.003}, {1.5*sin(\x)*180/\x/pi});
				\draw[cstcurve,second,domain=1:700,smooth] plot ({-\x*0.003}, {1.5*sin(\x)*180/\x/pi});
			\end{scope}
		\end{tikzpicture}
	\end{center}
	\onslide<+->
	我们将 $f(t),F(\omega)$ 称为傅里叶变换对.
	\onslide<+->
	不难发现 $F(t),2\pi f(-\omega)$ 也是傅里叶变换对 (不连续点处值需要修改).
	\onslide<+->
	故
	\[
		\msf[\sinc(t)]=\begin{cases}
		\pi,&\abs{\omega}<1,\\
		\pi/2,&\abs{\omega}=1,\\
		0,&\abs{\omega}>1.
		\end{cases}.
	\]
\end{frame}


\begin{frame}{例题: 求傅里叶变换}
	\onslide<+->
	\begin{example}[nearnext]
		求函数 $f(t)=
			\begin{cases}
				1,&t\in(0,1),\\
				-1,&t\in(-1,0),\\
				0,&\text{其它情形}
			\end{cases}$
		的傅里叶变换.
	\end{example}

	\onslide<+->
	\begin{solution}[nearprev]
		由于 $f(t)$ 是奇函数, 因此
		\begin{align*}
			F(\omega)&=\msf[f(t)]
			=\intff f(t)\ee^{-\ii \omega t}\d t
			=2\ii \int_0^{+\infty}f(t)\sin{\omega t}\d t\\
			&\visible<+->{=2\ii \int_0^1\sin{\omega t}\d t}
			\visible<+->{=-\frac{2\ii (1-\cos\omega)}\omega.}
		\end{align*}
	\end{solution}
\end{frame}


\begin{frame}{例题: 求傅里叶变换}
	\onslide<+->
	\begin{center}
		\begin{tikzpicture}
			\begin{scope}
				\draw[cstaxis](-2,0)--(2,0);
				\draw[cstaxis](0,-1.5)--(0,2);
				\draw[cstdash](-1,0)--(-1,-1);
				\draw[cstdash](1,0)--(1,1);
				\draw[main,cstcurve](-1,-1)--(0,-1);
				\draw[main,cstcurve](0,1)--(1,1);
			\end{scope}
			\begin{scope}[xshift=6cm]
				\draw[cstaxis](-2.5,0)--(2.5,0);
				\draw[cstaxis](0,-1.5)--(0,2);
				\draw[cstcurve,second,domain=1:540,smooth] plot ({\x*pi/800}, {-1.5*(1-cos(\x))*180/\x/pi});
				\draw[cstcurve,second,domain=1:540,smooth] plot ({-\x*pi/800}, {1.5*(1-cos(\x))*180/\x/pi});
			\end{scope}
		\end{tikzpicture}
	\end{center}
	\onslide<+->
	类似可得
	$\dint_0^{+\infty}\frac{(1-\cos\omega)\sin\omega t}\omega\d\omega=
		\begin{cases}
			\pi/2,&0<t<1,\\
			\pi/4,&t=1,\\
			0,&t>1.
		\end{cases}$
\end{frame}


\begin{frame}{例题: 求傅里叶变换}
	\onslide<+->
	\begin{example}[nearnext]
		求\emph{指数衰减函数} $f(t)=
			\begin{cases}
				0,&t<0,\\
				\ee^{-\beta t},&t\ge 0
			\end{cases}$ 的傅里叶变换, $\beta>0$.
	\end{example}
	\onslide<+->
	\begin{solution}[nearprev]
		\bigdel
		\begin{align*}
			F(\omega)&=\msf[f(t)]=\intff f(t)\ee^{-\ii \omega t}\d t\\
			&\visible<+->{=\int_0^{+\infty}\ee^{-\beta t}\ee^{-\ii \omega t}\d t}
			\visible<+->{=\int_0^{+\infty}\ee^{-(\beta+\ii \omega)t}\d t}
			\visible<+->{=\alertn{\frac1{\beta+\ii \omega}}.}&\visible<.->{}
		\end{align*}
	\end{solution}

	\onslide<+->
	类似可得
	$\dint_0^{+\infty}\frac{\beta\cos\omega t+\omega\sin\omega t}{\beta^2+\omega^2}\d\omega=
		\begin{cases}
			0,&t<0,\\
			\pi/2,&t=0,\\
			\pi \ee^{-\beta t},&t>0.
		\end{cases}$
\end{frame}


\begin{frame}{例题: 求傅里叶变换}
	\onslide<+->
	\begin{example}[nearnext]
		求\emph{钟形脉冲函数} $f(t)=\ee^{-\beta t^2}$ 的傅里叶变换和积分表达式, $\beta>0$.
	\end{example}
	\onslide<+->
	\begin{solution}[nearprev]
		\bigdel
		\begin{align*}
			F(\omega)&=\msf[f(t)]=\intff f(t)\ee^{-\ii \omega t}\d t\\
			&\visible<+->{=\intff \ee^{-\beta t^2}\ee^{-\ii \omega t}\d t}\\
			&\visible<+->{=\ee^{-\frac{\omega^2}{4\beta}}\intff \exp\left[-\beta\Bigl(t+\frac{\ii \omega}{2\beta}\Bigr)^2\right]\d t}
			\visible<+->{=\alertn{\sqrt{\frac\pi\beta}\ee^{-\frac{\omega^2}{4\beta}}.}}
		\end{align*}
	\end{solution}

	\onslide<+->
	类似可得
	$\dint_0^{+\infty}\ee^{-\frac{\omega^2}{4\beta}}\cos\omega t\d\omega=\sqrt{\pi\beta}\ee^{-\beta t^2}$.
\end{frame}


\subsection{狄拉克 \texorpdfstring{$\dirac$}{δ} 函数}

\begin{frame}{广义函数\noexer}
	\onslide<+->
	傅里叶变换存在的条件是比较苛刻的.
	\onslide<+->
	例如常值函数 $f(t)=1$ 在 $(-\infty,+\infty)$ 上不是可积的, 所以它没有傅里叶变换, 这很影响我们使用傅里叶变换.
	\onslide<+->
	为此我们引入广义函数的概念.

	\onslide<+->
	设 $\mss$ 是由一些``非常好''的函数(光滑速降函数)形成的线性空间.
	\onslide<+->
	从一个(局部可积缓增)函数 $\lambda(t)$ 出发, 我们可以定义一个线性映射 $\mss\to \BR$:
	\[
		\pair{\lambda,f}:=\intff \lambda(t)f(t)\d t.
	\]
	\onslide<+->
	这个线性映射基本上确定了 $\lambda(t)$ 本身(至多可数个点处不同).

	\onslide<+->
	\emph{广义函数(分布)}就是指一个线性映射 $\mss\to \BR$.
	\onslide<+->
	为了和普通函数类比, 也可将广义函数表为上述积分形式(并不是真的积分):
	\[
		\intff \lambda(t)f(t)\d t.
	\]
	这里的 $\lambda(t)$ 并不表示一个真正的函数.
\end{frame}


\begin{frame}{狄拉克 $\dirac$ 函数}
	\onslide<+->
	\begin{definition}[sidepic,righthand width=2cm]
		\emph{狄拉克 $\dirac$ 函数}是指广义函数
		\[
			\pair{\dirac,f}=\intff \dirac(t)f(t)\d t=f(0).
		\]
		\tcblower
		\begin{tikzpicture}
			\draw[cstaxis](8.2,0)--(9.8,0);
			\draw[cstaxis](9,-0.6)--(9,2);
			\draw
				(9.8,-0.25) node {$t$}
				(8.7,1.6) node {$1$}
				(9.5,1.6) node[main] {$\dirac(t)$}
				(8.8,-0.2) node {$O$};
			\draw[main,cstcurve,cstra](9,0)--(9,1.5);
		\end{tikzpicture}
	\end{definition}
	\onslide<+->
	设 $\dirac_\varepsilon(t)=\begin{cases}
		1/\varepsilon,&0\le t\le \varepsilon,\\
		0,&\text{其它情形,}
	\end{cases}$
	\onslide<+->
	则
	\[
		\pair{\dirac_\varepsilon,f}=\frac1\varepsilon\int_0^\varepsilon f(t)\d t=f(\xi),\quad \xi\in(0,\varepsilon).
	\]
	\onslide<+->
	当 $\varepsilon\to0$ 时, 右侧就趋于 $f(0)$.
	\onslide<+->
	因此 $\dirac$ 可以看成 $\dirac_\varepsilon$ 的``弱极限''.
	\onslide<+->
	基于此, 我们通常用长度为 $1$ 的有向线段来表示它.
\end{frame}


\begin{frame}{狄拉克 $\dirac$ 函数的性质}
	\onslide<+->
	对于广义函数 $\lambda$, 我们可以形式地定义 $\lambda(at),\lambda'$:
	\[
		\intff \lambda(at)f(t)\d t
	=\intff \lambda(t)\cdot\frac1{\abs{a}}f\bigl(\frac ta\bigr)\d t,
	\]
	\[
		\intff \lambda'(t)f(t)\d t
	=-\intff \lambda(t)f'(t)\d t.
	\]
	\onslide<+->
	由此可知
	\begin{itemize}
		\item $\pair{\dirac^{(n)},f}=(-1)^nf^{(n)}(0)$, 其中 $f(t)$ 是光滑函数.
		\item $\dirac(at)=\dfrac1{\abs{a}}\dirac(t)$. 特别地 $\dirac(t)=\dirac(-t)$.
		\item $u'(t)=\dirac(t)$, 其中 $u(t)=\begin{cases}1,&t\ge0,\\0,&t<0\end{cases}$ 是\emph{单位阶跃函数}.
	\end{itemize}
\end{frame}


\begin{frame}{Dirac $\dirac$ 函数的傅里叶变换和逆变换}
	\onslide<+->
	根据 $\dirac$ 函数的定义可知
	\[
		\msf[\dirac(t)]=\intff \dirac(t)\ee^{-\ii \omega t}\d t=1.
	\]
	\onslide<+->
	同理可得其傅里叶逆变换.
	\onslide<+->
	因此我们得到:
	\begin{theorem*}[][]%
	\[
		\msf[\dirac(t)]=1,\qquad
		\msf^{-1}[\dirac(\omega)]=\dfrac1{2\pi}.
	\]
	\end{theorem*}
\end{frame}


\begin{frame}{例题: 求傅里叶变换}
	\onslide<+->
	\begin{example}[nearnext]
		证明 \alert{$\msf[u(t)]=\dfrac1{\ii \omega}+\pi\dirac(\omega)$}.
	\end{example}
	\onslide<+->
	\begin{proof}[nearprev]
		\[
			\msf^{-1}\left[\frac1{\ii \omega}\right]
				=\frac1{2\pi}\intff \frac{\ee^{\ii \omega t}}{\ii \omega} \d\omega
				=\frac1\pi\int_0^{+\infty}\frac{\sin\omega t}{\omega} \d\omega.
		\]
		\onslide<+->{%
			由
			$\dint_0^{+\infty}\frac{\sin\omega}\omega \d\omega=\dfrac\pi2$
			可知
			$\dint_0^{+\infty}\frac{\sin\omega t}\omega \d\omega=\frac\pi2\sgn(t)$.
		}\onslide<+->{%
			故
			\[
				\msf^{-1} \left[\frac1{\ii \omega}+\pi\dirac(\omega)\right]
				=\half\sgn(t)+\half =u(t)\quad (t\neq 0).\qedhere
			\]
		}
	\end{proof}
\end{frame}


\subsection{傅里叶变换的性质}


\begin{frame}{傅里叶变换的性质: 线性性质}
	\onslide<+->
	我们不可能也没必要每次都对需要变换的函数从定义出发计算傅里叶变换.
	\onslide<+->
	通过研究傅里叶变换的性质, 结合常见函数的傅里叶变换, 我们可以得到很多情形的傅里叶变换.
	\onslide<+->
	\begin{theorem*}[][线性性质]
		\[
			\msf[\alpha f+\beta g]=\alpha F+\beta G,\quad
			\msf^{-1}[\alpha F+\beta G]=\alpha f+\beta g.
		\]
	\end{theorem*}
	\onslide<+->
	\begin{theorem*}[][位移性质]
		\[
			\msf[f(t-t_0)]=\ee^{-\ii \omega t_0}F(\omega),\quad
			\msf^{-1}[F(\omega-\omega_0)]=\ee^{\ii \omega_0 t}f(t).
		\]
	\end{theorem*}
\end{frame}


\begin{frame}{傅里叶变换的性质: 位移性质}
	\onslide<+->
	\begin{proof}
		\begin{align*}
			\msf[f(t-t_0)]&=\intff f(t-t_0)\ee^{-\ii \omega t}\d t\\
			&=\intff f(t)\ee^{-\ii \omega (t+t_0)}\d t=\ee^{-\ii \omega t_0}F(\omega).
		\end{align*}
		\onslide<+->{%
			逆变换情形类似可得.\qedhere
		}
	\end{proof}
	\onslide<+->
	由此可得
	\[
		\alertn{\msf[\dirac(t-t_0)]=\ee^{-\ii \omega t_0},\quad
		\msf^{-1}[\dirac(\omega-\omega_0)]=\dfrac1{2\pi}\ee^{\ii \omega_0 t}}.
	\]
\end{frame}


\begin{frame}{典型例题: 计算傅里叶变换}
	\onslide<+->
	\begin{example}[nearnext]
		求 $\sin{\omega_0 t}$ 的傅里叶变换.
	\end{example}
	\onslide<+->
	\begin{solution}[nearprev]
		由于 $\msf[1]=2\pi\dirac(\omega)$,
		\onslide<+->{%
			因此 $\msf[\ee^{\ii \omega_0t}]=2\pi\dirac(\omega-\omega_0)$,
		}\onslide<+->{%
			\begin{align*}
				\msf[\sin{\omega_0 t}]&=\frac1{2\ii }\left[\msf[\ee^{\ii \omega_0t}]-\msf[\ee^{-\ii \omega_0t}]\right]&\\
				&\visible<+->{=\frac1{2\ii }[2\pi\dirac(\omega-\omega_0)-2\pi\dirac(\omega+\omega_0)]}&\\
				&\visible<+->{=\ii \pi[\dirac(\omega+\omega_0)-\dirac(\omega-\omega_0)].}
			\end{align*}
		}\bigdel
	\end{solution}

	\onslide<+->
	\begin{exercise}
		$\msf[\cos{\omega_0 t}]=$\fillblankframe[5cm][1mm]{$\pi[\dirac(\omega+\omega_0)+\dirac(\omega-\omega_0)]$}.
	\end{exercise}
\end{frame}


\begin{frame}{常见傅里叶变换汇总}
	\onslide<+->
	\begin{theorem*}[][与 $\dirac$ 函数有关的傅里叶变换汇总]
		\begin{enumerate}
			\item $\msf[\dirac(t)]=1$, $\msf[\dirac(t-t_0)]=\ee^{-\ii \omega t_0}$;
			\item $\msf[1]=2\pi\dirac(\omega)$, $\msf[\ee^{\ii \omega_0 t}]=2\pi\dirac(\omega-\omega_0)$;
			\item $\msf[\sin \omega_0t]=\ii \pi[\dirac(\omega+\omega_0)-\dirac(\omega-\omega_0)]$;
			\item $\msf[\cos{\omega_0 t}]=\pi[\dirac(\omega+\omega_0)+\dirac(\omega-\omega_0)]$;\smallskip
			\item $\msf[u(t)]=\dfrac1{\ii \omega}+\pi\dirac(\omega)$.
		\end{enumerate}
	\end{theorem*}
\end{frame}


\begin{frame}{傅里叶变换的性质: 微分性质}\small
	\onslide<+->
	\begin{theorem*}[near][微分性质]
		\[
			\msf[f'(t)]=\ii \omega F(\omega),\quad
			\msf^{-1}[F'(\omega)]=-\ii tf(t),
		\]
		\[
			\msf[f^{(k)}(t)]=(\ii \omega)^k F(\omega),\quad
			\msf^{-1}[F^{(k)}(\omega)]=(-\ii t)^kf(t).
		\]
		这里, 被变换的函数要求在 $\infty$ 处趋于 $0$.
	\end{theorem*}
	\onslide<+->
	\begin{proof}[near]
		\[
			\msf[f']=\intff f'(t)\ee^{-\ii \omega t}\d t
			=-\intff f(t)(\ee^{-\ii \omega t})'\d t=\ii \omega F(\omega)
		\]
		\onslide<+->{%
			逆变换情形类似可得. 一般的 $k$ 归纳可得.\qedhere
		}
	\end{proof}
	\onslide<+->
	\begin{theorem*}[near][乘多项式性质]
	\[
		\msf[tf(t)]=\ii F'(\omega),\quad
		\msf^{-1}[\omega F(\omega)]=-\ii f'(t),
	\]
	\[
		\msf[t^kf(t)]=\ii ^kF^{(k)}(\omega),\quad
		\msf^{-1}[\omega^kF(\omega)]=(-\ii )^kf^{(k)}(t).
	\]
	\end{theorem*}
\end{frame}


\begin{frame}{傅里叶变换的性质}
	\onslide<+->
	\begin{theorem*}[][积分性质]
	\[
		\msf\left[\int_{-\infty}^t f(\tau)\d\tau\right]=\frac1{\ii \omega}F(\omega)+\pi \delta(\omega)\intff f(t).
	\]
	\end{theorem*}

	\onslide<+->
	由变量替换易得
	\begin{theorem*}[][相似性质]
	\[
		\msf[f(at)]=\frac1{\abs{a}}F\Bigl(\frac\omega a\Bigr),\quad
		\msf^{-1}[F(a\omega)]=\frac1{\abs{a}}f\Bigl(\frac t a\Bigr).
	\]
	\end{theorem*}
\end{frame}


\begin{frame}{典型例题: 计算傅里叶变换}
	\onslide<+->
	\begin{example}[nearnext]
		求 $\msf[t^k \ee^{-\beta t}u(t)],\beta>0$.
	\end{example}
	\onslide<+->
	\begin{solution}[nearprev]
			由于
			\[
		\msf[\ee^{-\beta t}u(t)]=\frac{1}{\beta+\ii \omega},
	\]
		\onslide<+->{因此
			\begin{align*}
				\msf[t^k\ee^{-\beta t}u(t)]&=\ii ^k\Bigl(\frac1{\beta+\ii \omega}\Bigr)^{(k)}
				=\frac{k!}{(\beta+\ii \omega)^{k+1}}.
			\end{align*}
		}
		\bigdel
	\end{solution}
	\onslide<+->
	由此可得任意有理函数(分子次数小于分母)的傅里叶(逆)变换.
\end{frame}


\begin{frame}{例题: 周期函数的傅里叶变换}
	\onslide<+->
	\begin{example}[nearnext]
		设 $f(t)$ 定义在 $\left[-\dfrac T2,\dfrac T2\right]$ 上,
		求 $g(t)=\sumff f(t+nT)$ 的傅里叶变换.
	\end{example}
	\onslide<+->
	\begin{solution}[nearprev]
		$g(t)$ 是一个周期为 $T$ 的函数, 设 $\omega_0=\dfrac{2\pi}T$.
		\onslide<+->{%
			我们先计算它的傅里叶级数展开
			\[
				c_n=\frac1T\int_{-\frac T2}^{\frac T2} f(t)\ee^{-\ii n\omega_0 t}\d t=\frac 1TF(n\omega_0),\quad
				\onslide<+->{f(t)=\frac1T\sumff F(n\omega_0)\ee^{\ii n\omega_0t}.}
			\]
		}\onslide<+->{%
			因此
			\[
					G(\omega)=\msf[f(t)]=\frac{2\pi}T\sumff F(n\omega_0)\dirac(n-n\omega_0).
			\]
		}
	\end{solution}
\end{frame}


\begin{frame}{例题: 周期函数的傅里叶变换}
	\onslide<+->
	函数 $f(t)$ 的周期扩展 $g(t)$ 的傅里叶变换是一系列脉冲函数的组合, 而各个脉冲的强度正是由 $F(\omega)$ 所决定.
	\onslide<+->
	例如矩形脉冲函数作周期扩展后得到通断比为 $\dfrac12$ 的方波信号, 它的傅里叶变换就是一系列脉冲函数的组合, 各个脉冲函数的强度为脉冲函数所在时间的 $\sinc$ 函数值的 $\dfrac{2\pi}T$ 倍.
	\onslide<+->
	\begin{center}
		\begin{tikzpicture}[yscale=1.8]
			\begin{scope}[xscale=.4]
				\draw[cstaxis](-6,0)--(6,0);
				\draw[cstaxis](0,-.5)--(0,2);
				\draw[cstdash](-1,0)--(-1,1);
				\draw[cstdash](1,0)--(1,1);
				\draw[cstdash](-3,0)--(-3,1);
				\draw[cstdash](3,0)--(3,1);
				\draw[main,cstcurve](-1,1)--(1,1);
				\draw[main,cstcurve](-5,1)--(-3,1);
				\draw[main,cstcurve](3,1)--(5,1);
			\end{scope}
			\begin{scope}[xshift=7cm,xscale=.3]
				\draw[cstaxis](-10,0)--(10,0);
				\draw[cstaxis](0,-.5)--(0,2);
				\draw[domain=1:800,smooth] plot ({\x/90}, {sin(\x)*90/\x});
				\draw[domain=1:800,smooth] plot ({-\x/90}, {sin(\x)*90/\x});
				\draw[cstcurve,cstra,second] (0,0)--(0,{pi/2});
				\draw[cstcurve,cstra,second] (1,0)--(1,1);
				\draw[cstcurve,cstra,second] (-1,0)--(-1,1);
				\draw[cstcurve,cstra,second] (3,0)--(3,-.333);
				\draw[cstcurve,cstra,second] (-3,0)--(-3,-.333);
				\draw[cstcurve,cstra,second] (5,0)--(5,.2);
				\draw[cstcurve,cstra,second] (-5,0)--(-5,.2);
			\end{scope}
		\end{tikzpicture}
	\end{center}
\end{frame}


\subsection{傅里叶变换的应用}


\begin{frame}{乘积定理}
	\onslide<+->
	\begin{theorem*}[][乘积定理]
		若 $\msf[f(t)]=F(\omega),\msf[g(t)]=G(\omega)$, 则
		\[
			\intff f(t)\ov{g(t)}\d t
			=\frac1{2\pi}\intff F(\omega)\ov{G(\omega)}\d \omega.
		\]
		特别地,
		\[
			\intff \abs{f(t)}^2\d t
			=\frac1{2\pi}\intff \abs{F(\omega)}^2\d \omega.
		\]
	\end{theorem*}
	\onslide<+->
  此即 $(F,G)=2\pi(f,g)$. 
	\onslide<+->
	这表明 $\dfrac1{\sqrt{2\pi}}\msf$ 是正交变换.
\end{frame}


\begin{frame}{例题: 乘积定理的应用}\small
	\onslide<+->
	\begin{example}[near]
		计算 $\intff \frac{\sin^4 x}{x^2}\d x$.
	\end{example}
	\onslide<+->
	\begin{solution}[near]
		由于
		\[
			f(t)=\begin{cases}
				1, &t\in(0,1),\\
				-1, &t\in(-1,0),\\
				0,&\text{其它情形.}
			\end{cases}
		\]
		的傅里叶变换为 $F(\omega)=-2\ii\dfrac{1-\cos \omega}{\omega}$.
		\onslide<+->{%
			因此
			\begin{align*}
				\intff \frac{\sin^4 x}{x^2}\d x&
				=\frac12\intff \Bigl(\frac{1-\cos 2x}{2x}\Bigr)^2\d (2x)
				=\frac18\intff \abs{F(\omega)}^2\d \omega\\&
				=\frac18\cdot2\pi\intff \abs{f(t)}^2\d t
				=\frac\pi 4\int_{-1}^1 \d t
				=\frac\pi2.
			\end{align*}
		}\bigdel
	\end{solution}
\end{frame}


\begin{frame}{泊松求和公式}\small
	\beqskip{0pt}
	\onslide<+->
	\begin{theorem}[near]
		$\sumff f(n)=\sumff F(2\pi n)$,
		间断点处需修改为左右极限平均值.
	\end{theorem}
	\onslide<+->
	\begin{proof}[near]
		考虑 $g(t)=\sumff f(t+n)$ 的傅里叶展开:
		\onslide<+->{%
			\begin{align*}
				&\int_0^1 g(t)\ee^{-2\pi \ii nt}\d t
				=\sum_{m=-\infty}^{+\infty}\int_0^1f(t+m)\ee^{-2\pi \ii nt}\d t\\
				\onslide<+->{=}&\onslide<.->{\sum_{m=-\infty}^{+\infty}\int_m^{m+1}f(t)\ee^{-2\pi \ii nt}\d t
				=\intff f(t)\ee^{-2\pi \ii nt}\d t=F(2\pi n).}
			\end{align*}
		}\onslide<+->{%
			因此 $g(t)=\sumff F(2\pi n)\ee^{2\pi \ii  n t}$.
		}\onslide<+->{%
			令 $t=0$ 即得.\qedhere
		}\meddel
	\end{proof}
	\endgroup
\end{frame}


\begin{frame}{例题: 泊松求和公式的应用}
	\onslide<+->
	\begin{example}
		为了计算 $\sumff\dfrac1{n^2+1}$, 
		\onslide<+->{%
			考虑
			\[
				F(\omega)=\frac1{2\pi +\ii \omega}+\frac1{2\pi -\ii \omega}=\frac{4\pi }{4\pi^2+\omega^2}
			\]
			的傅里叶逆变换
		}\onslide<+->{%
			\[
				f(t)=\msf^{-1}[F(\omega)]=\ee^{-2\pi t\sgn(t)}.
			\]
		}\onslide<+->{%
			于是
			\[
				\sumff\frac1{n^2+1}
				=\pi\sumff F(2\pi n)
				\onslide<+->{=\pi\sumff f(n)
				=\pi\Bigl(1+2\sumf1 \ee^{-2\pi n}\Bigr)
				=\frac{\pi}{\tanh\pi}.}
			\]
		}
	\end{example}
\end{frame}


\begin{frame}{卷积}
	\onslide<+->
	\begin{definition}
		$f_1(t),f_2(t)$ 的\emph{卷积}是指
	\[
		\alertn{(f_1\ast f_2)(t)=\intff  f_1(\tau)f_2(t-\tau)\d \tau.}
	\]
	\end{definition}
	\onslide<+->
	卷积满足如下重要性质:
	\onslide<+->
	\begin{theorem*}[][卷积定理]
	\[
		\msf[f_1\ast f_2]=F_1\cdot F_2,\quad
	\msf^{-1}[F_1\ast F_2]=\frac1{2\pi}f_1\cdot f_2.
	\]
	\bigdel
	\end{theorem*}
\end{frame}


\begin{frame}{卷积的性质}\small
	\onslide<+->
	\begin{proof}[nearprev]
		\bigdel
		\begin{align*}
			\msf[f_1\ast f_2]&=\intff \intff f_1(\tau)f_2(t-\tau)\d \tau \cdot \ee^{-\ii \omega t}\d t\\
			&\visible<+->{=\intff \intff f_1(\tau)\ee^{-\ii \omega \tau}\cdot f_2(t-\tau)\ee^{-\ii \omega (t-\tau)}\d t\d \tau}\\
			&\visible<+->{=\intff \intff f_1(\tau)\ee^{-\ii \omega \tau}\cdot f_2(t)\ee^{-\ii \omega t}\d t\d \tau}\\
			&\visible<+->{=\intff f_1(\tau)\ee^{-\ii \omega \tau}\d\tau\intff f_2(t)\ee^{-\ii \omega t}\d t}\visible<+->{=\msf[f_1]\msf[f_2].\qedhere}
		\end{align*}
	\end{proof}
	\onslide<+->
	由函数的乘法性质可知卷积满足如下性质:
	\begin{itemize}\bf
		\item $f_1\ast f_2=f_2\ast f_1,\ (f_1\ast f_2)\ast f_3=f_1\ast(f_2\ast f_3)$;
		\item $f_1\ast(f_2+f_3)=f_1\ast f_2+f_1\ast f_3$;
		\item $f\ast\dirac=f$;
		\item $(f_1\ast f_2)'=f_1'\ast f_2=f_1\ast f_2'$.
	\end{itemize}
\end{frame}


\begin{frame}{例题: 计算卷积}
	\onslide<+->
	\begin{example}[nearnext]
		设 $f_1(t)=u(t),f_2(t)=\ee^{-t}u(t)$. 求 $f_1\ast f_2$.
	\end{example}
	\onslide<+->
	\begin{solution}[nearprev]
			\[
		(f_1\ast f_2)(t)=\intff  f_2(\tau)f_1(t-\tau)\d \tau=\int_0^{+\infty} \ee^{-\tau}u(t-\tau)\d \tau.
	\]
		\onslide<+->{当 $t<0$ 时, $(f_1\ast f_2)(t)=0$.
		}\onslide<+->{当 $t\ge0$ 时, 
			\[
		(f_1\ast f_2)(t)=\int_0^t \ee^{-\tau}\d \tau=1-\ee^{-t}.
	\]
		}\onslide<+->{故 $(f_1\ast f_2)(t)=(1-\ee^{-t})u(t)$.
		}
	\end{solution}
\end{frame}


\begin{frame}{例题: 卷积的应用\noexer}
	\onslide<+->
	\begin{example}[nearnext]
		求 $\displaystyle	I=\intff \frac{\sin \omega}{\omega}\cdot \frac{\sin (\omega/3)}{\omega/3}\d\omega$.
	\end{example}
	\onslide<+->
	\begin{solution}[nearprev]
			设 $F(\omega)=\dfrac{\sin\omega}{\omega},G(\omega)=\dfrac{\sin(\omega/3)}{\omega/3}$,%
		\onslide<+->{则
			\[
		\msf^{-1}[FG]=\frac1{2\pi}\intff F(\omega)G(\omega)\ee^{\ii \omega t}\d\omega,
	\]
		}\onslide<+->{
			\[
		\msf^{-1}[FG](0)=\frac1{2\pi}\intff F(\omega)G(\omega)\d\omega=\frac1{2\pi}I.
	\]
		}
	\end{solution}
\end{frame}


\begin{frame}{例题: 卷积的应用}
	\onslide<+->
	\begin{solution}[][]
		我们之前计算过
		\[
			\msf^{-1}[F(\omega)]=f(t)=\begin{cases}
				1/2, & \abs{t}<1,\\
				1/4, & \abs{t}=1,\\
				0, & \abs{t}>1.
			\end{cases}
		\]
		\onslide<+->{%
			所以 $\msf^{-1}[G(\omega)]=g(t)=3f(3t)$,
		}\onslide<+->{%
			\begin{align*}
				\msf^{-1}[FG](0)&=(f\ast g)(0)\\
				&=\intff f(-t)g(t)\d t=\half\int_{-1}^1 g(t)\d t=\half ,
			\end{align*}
		}\onslide<+->{%
			故 $I=2\pi\msf^{-1}[FG](0)=\pi$.
		}
	\end{solution}
\end{frame}


\begin{frame}{使用傅里叶变换解微积分方程}
	\begin{center}
		\begin{tikzpicture}[node distance=40pt]
			\node[cstnode2] (a){微分方程或积分方程};
			\node[cstnode1,right=110pt of a] (b){象函数的代数方程};
			\node[cstnode2,below=of a] (c){原象函数(方程的解)};
			\node[cstnode1,below=of b] (d){象函数};
			\draw[cstnra,second] (a)--node[above]{傅里叶变换 $\msf$}(b);
			\draw[cstnra,second] (d)--node[below]{傅里叶逆变换 $\msf^{-1}$}(c);
			\draw[cstnra,second] (b)--(d);
			\draw[cstnra,third] (a)--(c);
		\end{tikzpicture}
	\end{center}
\end{frame}


\begin{frame}{例题: 使用傅里叶变换解微积分方程}
	\beqskip{1pt}
	\onslide<+->
	\begin{example}[nearnext]
		解方程 $y'(t)-\dint_{-\infty}^t y(\tau)\d \tau=2\dirac(t)$.
	\end{example}
	\onslide<+->
	\begin{solution}[nearprev]
		设 $\msf[y]=Y$.
		\onslide<+->{%
			两边同时作傅里叶变换得到
			\[
				\ii \omega Y(\omega)-\frac1{\ii \omega}Y(\omega)=2,
			\]
		}\onslide<+->{%
			\[
				Y(\omega)=-\frac{2\ii \omega}{1+\omega^2}=\frac1{1+\ii \omega}-\frac1{1-\ii \omega},
			\]
		}\onslide<+->{%
			\begin{align*}
				y(t)=\msf^{-1}\left[\frac1{1+\ii \omega}-\frac1{1-\ii \omega}\right]
				&\visible<+->{=\begin{cases}
				0-\ee^t=-\ee^t,&t<0,\\
				0,&t=0,\\
				\ee^{-t}-0=\ee^{-t},&t>0.
				\end{cases}}
			\end{align*}
		}
	\end{solution}
	\endgroup
\end{frame}


\begin{frame}{例题: 使用傅里叶变换解微积分方程}
	\onslide<+->
	\begin{example}[nearnext]
		解方程 $y''(t)-y(t)=0$.
	\end{example}
	\onslide<+->
	\begin{solution}[nearprev]
		设 $\msf[y]=Y$,
		\onslide<+->{%
			则
			\[
				\msf[y''(t)-y(t)]=[(\ii \omega)^2-1]Y(\omega)=0,
			\]
		}\onslide<+->{%
			\[
				Y(\omega)=0,\quad y(t)=\msf^{-1}[Y(\omega)]=0.
			\]
		}\bigdel
	\end{solution}
	\onslide<+->
	显然这是不对的, 该方程的解应该是 $y(t)=C_1\ee^t+C_2\ee^{-t}$.

	\onslide<+->
	原因在于使用傅里叶变换要求函数是绝对可积的, 而 $\ee^t,\ee^{-t}$ 并不满足该条件.
	\onslide<+->
	我们需要一个对函数限制更少的积分变换来解决此类方程, 例如拉普拉斯变换.
\end{frame}

