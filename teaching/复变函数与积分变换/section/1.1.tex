\section{复数及其代数运算}

\subsection{复数的产生}

\begin{frame}{复数的引入\noexer}
	\begin{itemize}
		\item 复数起源于多项式方程的求根问题.
		\item 考虑一元二次方程 $x^2+bx+c=0$.
		\item 配方可得 $\Bigl(x+\dfrac b2\Bigr)^2=\dfrac{b^2-4c}4$.
		\item 于是得到求根公式 $x=\dfrac{-b\pm\sqrt\Delta}2$, 其中 $\Delta=b^2-4c$.
		\begin{enumerate}
			\item 当 $\Delta>0$ 时, 有两个不同的实根;
			\item 当 $\Delta=0$ 时, 有一个二重的实根;
			\item 当 $\Delta<0$ 时, 无实根.
		\end{enumerate}
		\item 可以看出, 在一元二次方程中, 我们可以舍去包含\alert{负数开方}的解.
		\item 然而在一元三次方程中, 即便只考虑实数根也会不可避免地引入负数开方.
	\end{itemize}
\end{frame}


\begin{frame}{三次方程的根\noexer}
	\onslide<+->
	\begin{example}[nearnext]
		解方程 $x^3+6x-20=0$.
	\end{example}
	\onslide<+->
	\begin{solution}[nearprev,sidepic,righthand width=3.2cm]
		\begin{itemize*}
			\item 设 $x=u+v$,
			\item 那么 $u^3+v^3+3uv(u+v)+6(u+v)-20=0$.
			\item 我们希望 $u^3+v^3=20$, $uv=-2$.
			\item 那么 $u^3,v^3$ 满足一元二次方程 $X^2-20X-8=0$.
			\item 解得 $u^3=10\pm\sqrt{108}=(1\pm\sqrt3)^3$.
			\item 所以 $u=1\pm\sqrt3$, $v=1\mp\sqrt 3$.
			\item \centering $x=u+v=2$.
		\end{itemize*}
		\tcblower
		\onslide<+->{
		\begin{tikzpicture}
			\begin{scope}[xscale=.78]
				\draw[cstaxis] (-2,0)--(2,0);
				\draw[cstaxis] (0,-2)--(0,2);
				\clip (-2,-2) rectangle (2,2);
				\begin{scope}[xscale=.3,yscale=.04]
					\draw[cstcurve,main,domain=-3:4,smooth] plot (\x,{\x*\x*\x+6*\x-20});
					\coordinate (A) at (2,0);
					\coordinate (B) at (0,-20);
				\end{scope}
				\draw[inner sep=2pt]
					(A) node[below right] {$2$}
					(B) node[above left] {$-20$};
			\end{scope}
			\begin{scope}[cstdot]
				\fill (A) circle;
				\fill (B) circle;
			\end{scope}
		\end{tikzpicture}}
	\end{solution}
\end{frame}


\begin{frame}{三次方程的根\noexer}
	\beqskip{0pt}
	\onslide<+->
	\begin{example}[near]
		解方程 $x^3-7x+6=0$.
	\end{example}
	\onslide<+->
	\begin{solution}[near,sidepic,righthand width=2.7cm]
		\begin{itemize*}
			\item 类似地 $x=u+v$, 其中 $u^3+v^3=-6$, $uv=7/3$.
			\item 于是 $u^3,v^3$ 满足一元二次方程 $X^2+6X+343/27=0$.
			\item 该方程无实数解, 我们可以强行解得 $u^3=-3+\dfrac{10}9\sqrt{-3}$.
			\item \centering $u=\dfrac{3+2\sqrt{-3}}3,\dfrac{-9+\sqrt{-3}}6,\dfrac{3-5\sqrt{-3}}6$,
			\item \centering $v=\dfrac{3-2\sqrt{-3}}3,\dfrac{-9-\sqrt{-3}}6,\dfrac{3+5\sqrt{-3}}6$,
			\item \centering $x=u+v=2,-3,1$.
		\end{itemize*}
		\tcblower
		\onslide<+->{%
		\begin{tikzpicture}
			\begin{scope}[xscale=.7]
				\def\a{-3}
				\def\b{1}
				\def\c{2}
				\draw[cstaxis] (-2,0)--(2,0);
				\draw[cstaxis] (0,-2)--(0,2);
				\clip (-2,-2) rectangle (2,2);
				\begin{scope}[xscale=.45,yscale=.1]
					\draw[cstcurve,main,domain=-4:4,smooth] plot (\x,{(\x-\a)*(\x-\b)*(\x-\c)});
					\coordinate (A) at (\a,0);
					\coordinate (B) at (\b,0);
					\coordinate (C) at (\c,0);
				\end{scope}
				\draw[inner sep=2pt]
					(A) node[below right] {$-3$}
					(B) node[below left] {$1$}
					(C) node[below right] {$2$};
			\end{scope}
			\begin{scope}[cstdot]
				\fill (A) circle;
				\fill (B) circle;
				\fill (C) circle;
			\end{scope}
		\end{tikzpicture}}
	\end{solution}
	\endgroup
\end{frame}


\begin{frame}{三次方程的根\noexer}
	\begin{itemize}
		\item 一般地, 方程 $x^3+px+q=0$ 的解为($p=0$ 情形较简单, 这里不考虑)
		\[
			x=u-\frac p{3u},\quad u^3=-\frac q2+\sqrt{\Delta},\quad \Delta=\frac{q^2}4+\frac{p^3}{27}.
		\]
		\item 通过分析函数图像的极值点可以知道:
	\end{itemize}
	\begin{center}
		\begin{tikzpicture}[visible on=<3->]
			\begin{scope}[scale=.65,
				declare function={
					f(\x)=\x*\x*\x-3*\x;
				}
			]
				\def\a{2.5}
				\draw[cstaxis] (-2,0)--(2,0);
				\draw[cstaxis] (0,-2) node[below] {$\Delta>0$, 有 $1$ 个根}--(0,2);
				\clip (-2,-2) rectangle (2,2);
				\begin{scope}[xscale=.35,yscale=.1]
					\draw[cstcurve,main,domain=-3:4,smooth] plot (\x,{f(\x)-f(\a)});
					\coordinate (A) at (\a,0);
				\end{scope}
			\end{scope}
			\fill[cstdot] (A) circle;
			\begin{scope}[visible on=<4->,xshift=5cm]
				\begin{scope}[scale=.65]
					\def\a{-2}
					\def\b{1}
					\draw[cstaxis] (-2,0)--(2,0);
					\draw[cstaxis] (0,-2) node[below,align=center] {$\Delta=0$, 有 $2$ 个根\\$x=-\sqrt[3]{4q},\half\sqrt[3]{4q}$ ($2$重).}--(0,2);
					\clip (-2,-2) rectangle (2,2);
					\begin{scope}[xscale=.35,yscale=.1]
						\draw[cstcurve,main,domain=-4:3,smooth] plot ({\x},{(\x-\a)*(\x-\b)*(\x-\b)});
						\coordinate (A) at (\a,0);
						\coordinate (B) at (\b,0);
					\end{scope}
				\end{scope}
				\begin{scope}[cstdot]
					\fill (A) circle;
					\fill (B) circle;
				\end{scope}
			\end{scope}
			\begin{scope}[visible on=<5->,xshift=10cm]
				\begin{scope}[scale=.65]
					\def\a{-3}
					\def\b{.5}
					\def\c{2.5}
					\draw[cstaxis] (-2,0)--(2,0);
					\draw[cstaxis] (0,-2) node[below] {$\Delta<0$, 有 $3$ 个根}--(0,2);
					\clip (-2,-2) rectangle (2,2);
					\begin{scope}[xscale=.3,yscale=.05]
						\draw[cstcurve,main,domain=-5:5,smooth] plot ({\x},{(\x-\a)*(\x-\b)*(\x-\c)});
						\coordinate (A) at (\a,0);
						\coordinate (B) at (\b,0);
						\coordinate (C) at (\c,0);
					\end{scope}
				\end{scope}
				\begin{scope}[cstdot]
					\fill (A) circle;
					\fill (B) circle;
					\fill (C) circle;
				\end{scope}
			\end{scope}
		\end{tikzpicture}
	\end{center}
\end{frame}


\begin{frame}{三次方程的根\noexer}
	\begin{itemize}
		\item 由此可见, 若想使用求根公式, 就\alert{必须接受负数开方}.
		\item 那么为什么当 $\Delta<0$ 时, 从求根公式一定能得到 $3$ 个实根呢?
		\item 这个问题在我们学习了第一章的内容之后可以得到回答.
		\item 尽管在十六世纪, 人们已经得到了三次方程的求根公式, 然而对其中出现的虚数, 却是难以接受.
		\item 莱布尼兹曾说: {\color{third}\itshape 圣灵在分析的奇观中找到了超凡的显示, 这就是那个理想世界的端兆, 那个介于存在与不存在之间的两栖物, 那个我们称之为虚的 $-1$ 的平方根。}
		\item 不过, 现在我们可以用更为现代和严格的语言来引入复数.
	\end{itemize}
\end{frame}

\subsection{复数的概念}

\begin{frame}{复数的定义}
	\begin{itemize}
		\item 现在我们来正式介绍复数的概念.
		\item 为了避免记号 $\sqrt{-1}$ 带来的歧义, 我们先引入抽象符号 $\ii$, 再通过定义它的运算来构造复数.
	\end{itemize}
	\onslide<+->
	\begin{definition}
		固定一个记号 $\ii$, \emph{复数}就是形如 $z=x+y\ii$ 的元素, 其中 $x,y$ 均是实数, 且不同的 $(x,y)$ 对应不同的复数.
	\end{definition}
	\begin{itemize}
		\item 实数 $x$ 可以自然地看成复数 $x+0\ii$.
	\end{itemize}
\end{frame}


\begin{frame}{复平面}
	\begin{itemize}
		\item 回忆全体实数、有理数、整数、自然数构成的集合分别记作 $\BR,\BQ,\BZ,\BN$.
		\item 将\emph{全体复数记作 $\BC$}.
		\item 那么 $\BC$ 自然构成一个二维实线性空间, 且 $\{1,\ii\}$ 是一组基. 
		\item 因此它和平面上的点可以建立一一对应, 并将建立起这种对应的平面称为\emph{复平面}.
	\end{itemize}
	\onslide<+->
	\begin{center}
		\begin{tikzpicture}
			\begin{scope}
				\draw[cstaxis] (-.5,0)--(3,0);
				\draw[cstaxis] (0,-.5)--(0,2.5);
				\coordinate [label=below left:$0$] (O) at (0,0);
				\coordinate [label=above:\textcolor{second}{$z=x+y\ii$}] (A) at (2,1.5);
				\coordinate (B) at (2,0);
				\coordinate (C) at (0,1.5);
				\draw[cstdash] (B)--(A)--(C);
				\fill[cstdot,second] (A) circle;
				\draw[third,Latex-Latex,line width=.5mm] (2.8,1)--(4,1) node[midway,below,third] {一一对应};
			\end{scope}
			\begin{scope}[xshift=5cm]
				\coordinate [label=below left:$O$] (O) at (0,0);
				\coordinate [label=above:\textcolor{second}{$Z(x,y)$}] (A) at (2,1.5);
				\coordinate (B) at (2,0);
				\coordinate (C) at (0,1.5);
				\draw[cstdash] (B)--(A)--(C);
				\fill[cstdot,second] (A) circle;
				\draw[decorate,decoration={brace,amplitude=5},main,cstfill1] (O)--(B) node[midway,above=2mm] {$x$};
				\draw[decorate,decoration={brace,amplitude=5},main,cstfill1] (C)--(O) node[midway,right=2mm] {$y$};
				\draw[third,Latex-Latex,line width=.5mm] (2.8,1)--(4,1) node[midway,below,third] {一一对应};
				\draw[cstaxis] (-.5,0)--(3,0);
				\draw[cstaxis] (0,-.5)--(0,2.5);
			\end{scope}
			\begin{scope}[xshift=10cm]
				\draw[cstaxis] (-.5,0)--(3,0);
				\draw[cstaxis] (0,-.5)--(0,2.5);
				\coordinate [label=below left:$O$] (O) at (0,0);
				\coordinate [label=above:\textcolor{second}{$\overrightarrow{OZ}=(x,y)$}] (A) at (2,1.5);
				\draw[cstcurve,cstra,second] (O)--(A);
			\end{scope}
		\end{tikzpicture}
	\end{center}
\end{frame}


\begin{frame}{实部和虚部, 虚数和纯虚数}
	\onslide<+->
	\begin{itemize}
		\item $x,y$ 轴分别对应复平面的\emph{实轴}和\alert{虚轴}.
		\item 称 $z=x+y\ii$ 中 $x=\Re z$ 为 $z$ 的\emph{实部}; $y=\Im z$ 为 $z$ 的\alert{虚部}.
		\item 当虚部 $\Im z=0$ 时, $z$ 为实数, 它落在实轴上.
		\item 不是实数的复数是\textcolor{third}{\bf 虚数}.
		\item 当实部 $\Re z=0$ 且 \alert{$z\neq0$} 时, $z$ 为\alert{纯虚数}, 它落在虚轴上.
	\end{itemize}
	\onslide<1->
	\begin{figure}[hbpt]
		\centering
		\begin{minipage}{.48\textwidth}
			\raggedleft
			\begin{tikzpicture}
				\coordinate [label=below left:$0$] (O) at (0,0);
				\coordinate (B) at (2,0);
				\coordinate (C) at (0,1.5);
				\draw[cstaxis] (-.5,0)--(3,0);
				\draw[cstaxis] (0,-.5)--(0,2.5);
				\begin{scope}[visible on=<3->]
					\draw[decorate,decoration={brace,amplitude=5},main,cstfill1] (B)--(O) node[midway,below=1.5mm] {$\Re z$};
					\draw[decorate,decoration={brace,amplitude=5},second,cstfill2] (C)--(O) node[midway,right=1.5mm] {$\Im z$};
					\coordinate [label=above:\textcolor{third}{$z=x+y\ii$}] (A) at (2,1.5);
					\draw[cstdash] (B)--(A)--(C);
					\fill[cstdot,third] (A) circle;
				\end{scope}
				\begin{scope}[visible on=<2->]
					\coordinate [label=above:\textcolor{main}{实轴}] (R) at (3,0);
					\coordinate [label=right:\textcolor{second}{虚轴}] (I) at (0,2.5);
					\draw[cstaxis,main] (-.5,0)--(R);
					\draw[cstaxis,second] (0,-.5)--(I);
				\end{scope}
				\draw[main,->,thick,visible on=<4->] (-2,.2)-|(.6,0);
				\draw[second,->,thick,visible on=<6->] (-2,1.3)--(0,1.3);
				\draw 
					(-2,.1) node[cstnode,draw=main,text=main,visible on=<4->] {实数}
					(-2,1.2) node[align=center,cstnode,draw=second,text=second,visible on=<6->] {纯虚数\\不含原点};
			\end{tikzpicture}
		\end{minipage}
		\begin{minipage}{.48\textwidth}
			\centering
			\begin{tikzpicture}
				\filldraw[cstcurve,cstfill] (.8,0) circle (2.6 and 2);
				\coordinate (R) at (0,-.8);
				\filldraw[cstcurve,main,fill=white,visible on=<4->] (R) circle (1.2 and .7);
				\coordinate (I) at (0,.8);
				\draw (R) node[align=center,main,visible on=<4->] {实数 \\$0,1,\sqrt2,\pi,\ee$};
				\draw (I) node[align=center,second,visible on=<6->] {纯虚数 \\$\ii,-\ii ,\pi\ii$};
				\draw[cstcurve,second,visible on=<6->] (I) circle (1.2 and .7);
				\draw 
					(3.7,0) node[align=center] {全\\体\\复\\数}
					(2,0) node[align=center,third,visible on=<5->] {虚数 \\$\ii,\pi\ii,\frac{-1+\sqrt 3 \ii}2$};
			\end{tikzpicture}
		\end{minipage}
	\end{figure}
\end{frame}


\begin{frame}{例题:判断实数和纯虚数}
	\onslide<+->
	\begin{example}[nearnext]
		实数 $x$ 取何值时, $z=(x^2+3x-4)+(x^2+5x-6)\ii$ 是:
		\begin{subexample}[2]
			\item 实数;
			\item 纯虚数.
		\end{subexample}
	\end{example}
	\onslide<+->
	\begin{solution}[nearprev]
		\begin{enumerate}
			\item $\Im z=x^2+5x-6=0$, 即 $x=1$ 或 $-6$.
			\item $\Re z=x^2+3x-4=0$, 即 $x=1$ 或 $-4$.
				\onslide<+->{%
					但同时要求 $\Im z=x^2+5x-6\neq 0$, 因此 $x\neq 1$.
				}\onslide<+->{%
					故 $x=-4$.
				}
		\end{enumerate}
	\end{solution}
	\onslide<+->
	\begin{exercise}
		若 $x^2(1+\ii)-x(5+4\ii)+4+3\ii$ 是纯虚数, 则实数 $x=$\fillblankframe{$4$}.
	\end{exercise}
\end{frame}


\subsection{复数的代数运算}


\begin{frame}{复数的加法与减法}
	\begin{itemize}
		\item 设 $z_1=x_1+y_1\ii,z_2=x_2+y_2\ii$.
		\item 定义复数的\emph{加法}和\emph{减法}:
		\[
			z_1+z_2=(x_1+x_2)+(y_1+y_2)\ii,\quad
			z_1-z_2=(x_1-x_2)+(y_1-y_2)\ii.
		\]
		\item 复数的加减法与其对应的向量 $\overrightarrow{OZ}$ 的加减法是一致的.
	\end{itemize}
	\onslide<1->
	\begin{center}
		\begin{tikzpicture}[scale=.7]
			\draw[cstaxis] (-2,0)--(4,0);
			\draw[cstaxis] (0,-3)--(0,2.5);
			\coordinate (O) at (0,0);
			\coordinate [label=right:\textcolor{main}{$z_1$}] (Z1) at (2.5,-1);
			\coordinate [label=above:\textcolor{main}{$z_2$}] (Z2) at (1.5,2);
			\begin{scope}[visible on=<3->]
				\coordinate [label=above right:\textcolor{second}{$z_1+z_2$}] (P) at ($(Z1)+(Z2)$);
				\draw[cstcurve,cstra,second] (O)--(P);
				\draw[cstdash] (Z2)--(P)--(Z1);
			\end{scope}
			\begin{scope}[visible on=<4->]
				\coordinate [label=below:\textcolor{third}{$z_1-z_2$}] (M) at ($(Z1)-(Z2)$);
				\coordinate [label=left:{$-z_2$}] (neg) at ($(O)-(Z2)$);
				\draw[cstcurve,cstra,third] (O)--(M);
				\draw[cstdash,cstra] (O)--(neg);
				\draw[cstdash] (Z1)--(M)--(neg);
			\end{scope}
			\draw[cstcurve,cstra,main] (O)--(Z1);
			\draw[cstcurve,cstra,main] (O)--(Z2);
		\end{tikzpicture}
	\end{center}
\end{frame}


\begin{frame}{复数的乘除法}
	\begin{itemize}
		\item \alert{规定 $\ii\cdot \ii=-1$}.
		\item 定义复数的\emph{乘法}:
		\begin{align*}
			z_1\cdot z_2&
			=(x_1+y_1\ii)(x_2+y_2\ii)
			=x_1\cdot x_2+x_1\cdot y_2\ii+y_1\ii\cdot x_2+y_1\ii\cdot y_2\ii\\
			&=(x_1x_2-y_1y_2)+(x_1y_2+x_2y_1)\ii.
		\end{align*}
		\item 此时加法/乘法交换律, 结合律以及乘法分配律均成立.
		\item 待定系数可得复数的\emph{除法}定义为:
		\[
			\frac{z_1}{z_2}
			=\frac{(x_1+y_1\ii)(x_2-y_2\ii)}{x_2^2+y_2^2}
			=\frac{x_1x_2+y_1y_2}{x_2^2+y_2^2}+\frac{x_2y_1-x_1y_2}{x_2^2+y_2^2}\ii.
		\]
		\item 对于正整数 $n$, 定义 $z$ 的 \emph{$n$ 次幂}为 $n$ 个 $z$ 相乘.
		\item 当 $z\neq 0$ 时, 还可以定义 $z^0=1,z^{-n}=\dfrac1{z^n}$.
	\end{itemize}
\end{frame}


\begin{frame}{例: 单位根}
	\onslide<+->
	\begin{example}
		\begin{enumerate}
			\item $\ii^2=-1,\ii^3=-\ii ,\ii^4=1$.
			\onslide<+->{%
			一般地, 对于整数 $n$, 
			\[
				\ii^{4n}=1,\quad \ii^{4n+1}=i,\quad
				\ii^{4n+2}=-1,\quad \ii^{4n+3}=-\ii.
			\]
			}
			\vspace{-\baselineskip}
			\item 令 $\omega=\dfrac{-1+\sqrt 3\ii}2$, 则 $\omega^2=\dfrac{-1-\sqrt3\ii}2,\omega^3=1$.
			\item 令 $z=1+\ii$, \onslide<+->{则
			\[
				z^2=2\ii,\quad z^3=-2+2\ii,\quad z^4=-4,\quad z^8=16=2^4.
			\]}
		\end{enumerate}
		\bigdel\bigdel
	\end{example}
	\begin{itemize}
		\item 将满足 $z^n=1$ 的复数 $z$ 称为 \emph{$n$ 次单位根}.
		\item 那么 $1,\ii,-1,-\ii $ 是 $4$ 次单位根, $1,\omega,\omega^2$ 是 $3$ 次单位根, $-\omega$ 是 $6$ 次单位根.
	\end{itemize}
\end{frame}


\begin{frame}{例: 代数式的计算}
	\begin{itemize}
		\item 实数情形的等差数列求和公式、等比数列求和公式、二项式展开、平方差公式等在复数情形也成立.
	\end{itemize}
	\onslide<+->
	\begin{example}[nearnext]
		化简 $1+\ii+\ii^2+\dots+\ii^{1000}$.
	\end{example}
	\onslide<+->
	\begin{solution}[nearprev]
		根据等比数列求和公式, $1+\ii+\ii^2+\dots+\ii^{1000}
			=\dfrac{\ii^{1001}-1}{\ii-1}
			\visible<+->{=\dfrac{\ii-1}{\ii-1}=1.}$
	\end{solution}
	\onslide<+->
	\begin{exercise}
		化简 $\Bigl(\dfrac{1+\ii}{1-\ii}\Bigr)^{2026}$=\fillblankframe{$-1$}.
	\end{exercise}
\end{frame}


\subsection{共轭复数}


\begin{frame}{共轭复数的定义}
	\onslide<+->
	\begin{definition}
		称 $z$ 在复平面关于实轴的对称点为它的\emph{共轭复数 $\ov z$}.
		换言之, $\ov{x+y\ii}=x-y\ii$.
	\end{definition}
	\onslide<+->
	\begin{exercise}
		$z$ 关于虚轴的对称点是\fillblankframe{$-\ov z$}.
	\end{exercise}
\end{frame}

\begin{frame}{共轭复数的性质}
	\begin{itemize}
		\item 从定义出发, 不难验证共轭复数满足如下性质:
		\begin{enumerate}\bf
			\item $z$ 是 $\ov z$ 的共轭复数.\hfill\alert{共轭是一种对合}
			\item $\ov{z_1\pm z_2}=\ov{z_1}\pm\ov{z_2},\ 
			\ov{z_1\cdot z_2}=\ov{z_1}\cdot\ov{z_2},\ 
			\ov{\Bigl(\dfrac{z_1}{z_2}\Bigr)}=\dfrac{~\ov{z_1}~}{~\ov{z_2}~}$.
			\hfill \alert{共轭复数和四则运算交换}
			\item $z\ov{z}=(\Re z)^2+(\Im z)^2$.
			\item $z+\ov z=2\Re z,\ z-\ov z=2\ii\Im z$.
			\hfill \alert{$x,y$ 和 $z,\ov z$ 可相互表示}
			\item $z=\ov z\iff z$ 是实数; $z=-\ov z\iff z$ 是纯虚数或 $z=0$.\hfill\alert{判断实数和纯虚数}
		\end{enumerate}
		\item 这些性质意味着使用共轭复数进行计算和证明,往往比直接使用 $x,y$ 表达的形式更简单.
	\end{itemize}
\end{frame}


\begin{frame}{例题:共轭复数证明等式}
	\onslide<+->
	\begin{example}
		证明 $z_1\cdot\ov{z_2}-\ov{z_1}\cdot z_2=2\ii\Im(z_1\cdot\ov{z_2})$.
	\end{example}
	\onslide<+->
	我们可以设 $z_1=x_1+y_1\ii,z_2=x_2+y_2\ii$, 然后代入等式两边化简并比较实部和虚部得到.
	\onslide<+->
	但我们利用共轭复数可以更简单地证明它.
	\onslide<+->
	\begin{proof}
		\begin{itemize*}
			\item 由于 $\ov{z_1\cdot\ov{z_2}}=\ov{z_1}\cdot\ov{\ov{z_2}}=\ov{z_1}\cdot z_2$, 
			\item 因此
			\[
				z_1\cdot\ov{z_2}-\ov{z_1}\cdot z_2
				=z_1\cdot\ov{z_2}-\ov{z_1\cdot\ov{z_2}}
				=2\ii\Im(z_1\cdot\ov{z_2}).\qedhere
			\]
		\end{itemize*}
	\end{proof}
\end{frame}


\begin{frame}{例题:共轭复数判断实数}
	\onslide<+->
	\begin{example}[nearnext]
		设 $z=x+y\ii$ 是虚数.
		证明: $x^2+y^2=1$ 当且仅当 $z+\dfrac 1z$ 是实数.
	\end{example}
	\onslide<+->
	\begin{proof}[nearprev]
		\begin{itemize*}
			\item $z+\dfrac 1z$ 是实数等价于
				$z+\dfrac 1z=\ov{\Bigl(z+\dfrac 1z\Bigr)}=\ov z+\dfrac1{~\ov z~}$,
			\item 等价于
			\[
				z-\ov z=\frac1{~\ov z~}-\frac1z=\frac{z-\ov z}{z\ov z},\quad (z-\ov z)(z\ov z-1)=0.
			\]
			\item 由 $z$ 是虚数可知 $z\neq \ov z$.
			\item 故上述等式等价于 $z\ov z=1$, 即 $x^2+y^2=1$.\qedhere
		\end{itemize*}
	\end{proof}
\end{frame}


\begin{frame}{例: 复数的代数计算}
	\onslide<+->
	由于 $z\ov z$ 是一个实数,
	\onslide<+->
	因此在做复数的除法运算时, 可以利用下式将其转化为乘法:
	\[
		\dfrac{z_1}{z_2}=\dfrac{z_1\ov{z_2}}{z_2\ov{z_2}}=\dfrac{z_1\ov{z_2}}{x_2^2+y_2^2}.
	\]
	\bigdel
	\onslide<+->
	\begin{example}[nearnext]
		$z=-\dfrac1\ii-\dfrac{3\ii}{1-\ii }$, 求 $\Re z,\Im z$ 以及 $z\ov z$.
	\end{example}
	\onslide<+->
	\begin{solution}[nearprev]
		\[
			z=-\frac1\ii-\frac{3\ii}{1-\ii }
			\onslide<+->{=\ii-\frac{3\ii-3}2=\frac32-\half \ii,}
		\]
		\onslide<+->{%
		\[
			\Re z=\frac32,\quad\Im z=-\half ,\quad
			z\ov z=\Bigl(\frac32\Bigr)^2+\Bigl(-\half\Bigr)^2=\frac52.
		\]
		}
		\bigdel
	\end{solution}
\end{frame}


\begin{frame}{例: 复数的代数计算}
	\onslide<+->
	\begin{example}[nearnext]
		设 $z_1=5-5\ii,z_2=-3+4\ii$, 求 $\ov{\Bigl(\dfrac{z_1}{z_2}\Bigr)}$.
	\end{example}
	\onslide<+->
	\begin{solution}[nearprev]
		\begin{align*}
			\frac{z_1}{z_2}&=\frac{5-5\ii}{-3+4\ii}
			\onslide<+->{=\frac{(5-5\ii)(-3-4\ii)}{(-3)^2+4^2}}\\
			&\onslide<+->{=\frac{(-15-20)+(-20+15)\ii}{25}}
			\onslide<+->{=-\frac75-\frac15\ii,}
		\end{align*}
		\onslide<+->{%
			因此 $\ov{\Bigl(\dfrac{z_1}{z_2}\Bigr)}=-\dfrac75+\dfrac15\ii$.
		}
	\end{solution}
\end{frame}



% \begin{frame}{复数域\noexer}
% 	\begin{itemize}
% 		\item 复数全体构成一个\emph{域}.
% 		\item 所谓的域, 是指带有如下内容和性质的集合
% 		\begin{itemize}\bf
% 			\item 包含 $0,1$, 且有四则运算;
% 			\item 满足加法结合/交换律, 乘法结合/交换/分配律;
% 			\item 对任意 $a$, $a+0=a\times 1=a$.
% 		\end{itemize}
% 		\item 有理数全体 $\BQ$, 实数全体 $\BR$ 也构成域, 它们是 $\BC$ 的子域.
% 		\item 与有理数域和实数域有着本质不同的是, 复数域是\emph{代数闭域}:
% 		\item 对于任何次数 $n\ge 1$ 的复系数多项式
% 		\[
% 			p(z)=z^n+c_{n-1}z^{n-1}+\cdots+c_1z+c_0,
% 		\]
% 		都存在复数 $z_0$ 使得 $p(z_0)=0$.
% 		\item 由此不难知道, 复系数多项式可以因式分解成一次多项式的乘积.
% 		\item 我们会在第五章证明该结论.
% 	\end{itemize}
% \end{frame}


% \begin{frame}{复数域不是有序域\noexer}
% 	\begin{itemize}
% 		\item \onslide<+->
% 	\end{itemize}
% 	在 $\BQ,\BR$ 上可以定义出一个好的大小关系,
% 	\onslide<+->
% 	换言之它们是有序域, 即存在一个满足下述性质的 $>$:
% 	\begin{itemize}\bf
% 		\item 若 $a\neq b$, 则要么 $a>b$, 要么 $b>a$;
% 		\item 若 $a>b$, 则对于任意 $c$, $a+c>b+c$;
% 		\item 若 $a>b,c>0$, 则 $ac>bc$.
% 	\end{itemize}
% 	\onslide<+->
% 	而 \alert{$\BC$ 却不是有序域}.
% 	\onslide<+->
% 	若 $\ii>0$, 则
% 	\[
% 		-1=\ii\cdot \ii>0,\quad -\ii =-1\cdot \ii>0.
% 	\]
% 	\onslide<+->
% 	于是 $0>\ii$, 矛盾! 同理 $\ii<0$ 也不可能.
% \end{frame}
