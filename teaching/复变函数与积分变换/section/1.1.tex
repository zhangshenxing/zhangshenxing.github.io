\section{复数及其代数运算}

\subsection{复数的概念}

\begin{frame}{复数的定义}
	\onslide<+->
	现在我们来正式介绍复数的概念.
	\onslide<+->
	为了避免记号 $\sqrt{-1}$ 带来的歧义, 我们先引入抽象符号 $i$, 再通过定义它的运算来构造复数.
	\onslide<+->
	\begin{definition}[复数]
		固定一个记号 $i$, \emph{复数}就是形如 $z=x+yi$ 的元素, 其中 $x,y$ 均是实数, 且不同的 $(x,y)$ 对应不同的复数.
	\end{definition}
	\onslide<+->
	于是复数全体构成一个二维实线性空间, $\{1,i\}$ 是一组基. 而且实数 $x$ 可以自然地看成复数 $x+0i$.
	\onslide<+->
	将\emph{全体复数记作 $\BC$}, 全体实数记作 $\BR$, 则 $\BC=\BR+\BR i$, $\BR\subseteq \BC$.
\end{frame}


\begin{frame}{复平面}
	\onslide<+->
	由此, $\BC$ 和平面上的点可以建立一一对应, 并将建立起这种对应的平面称为\emph{复平面}.
	\onslide<+->
	\begin{center}
		\begin{tikzpicture}
			\begin{scope}
				\draw[cstaxis] (-.5,0)--(3,0);
				\draw[cstaxis] (0,-.5)--(0,2.5);
				\coordinate [label=below left:$0$] (O) at (0,0);
				\coordinate [label=above:\textcolor{second}{$z=x+yi$}] (A) at (2,1.5);
				\coordinate (B) at (2,0);
				\coordinate (C) at (0,1.5);
				\draw[cstdash] (B)--(A)--(C);
				\fill[cstdot,second] (A) circle;
				\draw[third,Latex-Latex,line width=.5mm] (2.8,1)--(4,1) node[midway,below,third] {一一对应};
			\end{scope}
			\begin{scope}[xshift=5cm]
				\coordinate [label=below left:$O$] (O) at (0,0);
				\coordinate [label=above:\textcolor{second}{$Z(x,y)$}] (A) at (2,1.5);
				\coordinate (B) at (2,0);
				\coordinate (C) at (0,1.5);
				\draw[cstdash] (B)--(A)--(C);
				\fill[cstdot,second] (A) circle;
				\draw[decorate,decoration={brace,amplitude=5},main,cstfill1] (O)--(B) node[midway,above=2mm] {$x$};
				\draw[decorate,decoration={brace,amplitude=5},main,cstfill1] (C)--(O) node[midway,right=2mm] {$y$};
				\draw[third,Latex-Latex,line width=.5mm] (2.8,1)--(4,1) node[midway,below,third] {一一对应};
				\draw[cstaxis] (-.5,0)--(3,0);
				\draw[cstaxis] (0,-.5)--(0,2.5);
			\end{scope}
			\begin{scope}[xshift=10cm]
				\draw[cstaxis] (-.5,0)--(3,0);
				\draw[cstaxis] (0,-.5)--(0,2.5);
				\coordinate [label=below left:$O$] (O) at (0,0);
				\coordinate [label=above:\textcolor{second}{$\overrightarrow{OZ}=(x,y)$}] (A) at (2,1.5);
				\draw[cstcurve,cstra,second] (O)--(A);
			\end{scope}
		\end{tikzpicture}
	\end{center}
\end{frame}


\begin{frame}{实部和虚部, 虚数和纯虚数}
	\onslide<+->
	当 $y=0$ 时, $z=x$ 就是一个实数.
	\onslide<+->
	它对应复平面上的点就是直角坐标系的 $x$ 轴上的点.
	\onslide<+->
	因此我们称 $x$ 轴为\emph{实轴}.
	\onslide<+->
	相应地, 称 $y$ 轴为\alert{虚轴}.
	\onslide<+->
	称 $z=x+yi$ 在实轴和虚轴的投影为它的\emph{实部 $\Re z=x$} 和\alert{虚部 $\Im z=y$}.

	\onslide<+->
	当 $\Im z=0$ 时, $z$ 是实数.
	\onslide<+->
	不是实数的复数是\textcolor{third}{虚数}.
	\onslide<+->
	当 $\Re z=0$ 且 $z\neq 0$ 时, 称 $z$ 是\alert{纯虚数}.
	\onslide<1->
	\begin{figure}[hbpt]
		\centering
		\begin{minipage}{.48\textwidth}
			\raggedleft
			\begin{tikzpicture}
				\coordinate [label=below left:$0$] (O) at (0,0);
				\draw[cstaxis] (-.5,0)--(3,0);
				\draw[cstaxis] (0,-.5)--(0,2.5);
				\begin{scope}[visible on=<5->]
					\draw[decorate,decoration={brace,amplitude=5},main,cstfill1] (B)--(O) node[midway,below=1.5mm] {$\Re z$};
					\draw[decorate,decoration={brace,amplitude=5},second,cstfill2] (C)--(O) node[midway,right=1.5mm] {$\Im z$};
					\coordinate [label=above:\textcolor{third}{$z=x+yi$}] (A) at (2,1.5);
					\draw[cstdash] (B)--(A)--(C);
					\fill[cstdot,third] (A) circle;
				\end{scope}
				\begin{scope}[visible on=<3->]
					\coordinate [label=above:\textcolor{main}{实轴}] (R) at (3,0);
				\end{scope}
				\begin{scope}[visible on=<4->]
					\coordinate [label=right:\textcolor{second}{虚轴}] (I) at (0,2.5);
				\end{scope}
				\coordinate (B) at (2,0);
				\coordinate (C) at (0,1.5);
				\draw[cstaxis,main,visible on=<2->] (-.5,0)--(R);
				\draw[main,->,thick,visible on=<2->] (-2,.2)-|(.6,0);
				\draw[cstaxis,second,visible on=<4->] (0,-.5)--(I);
				\draw[second,->,thick,visible on=<8->] (-2,1.3)--(0,1.3);
				\draw 
					(-2,.1) node[cstnode,draw=main,text=main,visible on=<2->] {实数}
					(-2,1.2) node[align=center,cstnode,draw=second,text=second,visible on=<8->] {纯虚数\\不含原点};
			\end{tikzpicture}
		\end{minipage}
		\begin{minipage}{.48\textwidth}
			\centering
			\begin{tikzpicture}
				\filldraw[cstcurve,cstfill] (.8,0) circle (2.6 and 2);
				\coordinate (R) at (0,-.8);
				\filldraw[cstcurve,main,fill=white,visible on=<2->] (R) circle (1.2 and .7);
				\coordinate (I) at (0,.8);
				\draw (R) node[align=center,main,visible on=<2->] {实数 \\$0,1,\sqrt2,\pi,e$};
				\draw (I) node[align=center,second,visible on=<8->] {纯虚数 \\$i,-i,\pi i$};
				\draw[cstcurve,second,visible on=<8->] (I) circle (1.2 and .7);
				\draw 
					(3.7,0) node[align=center] {全\\体\\复\\数}
					(2,0) node[align=center,third,visible on=<7->] {虚数 \\$i,\pi i,\frac{-1+\sqrt 3 i}2$};
			\end{tikzpicture}
		\end{minipage}
	\end{figure}
\end{frame}


\begin{frame}{例题:判断实数和纯虚数}
	\onslide<+->
	\begin{example}
		实数 $x$ 取何值时, $z=(x^2-3x-4)+(x^2-5x-6)i$ 是:
		\begin{enumerate}
			\item 实数;
			\item 纯虚数.
		\end{enumerate}
	\end{example}
	\onslide<+->
	\begin{solution}
		\begin{enumerate}
			\item $\Im z=x^2-5x-6=0$, 即 $x=-1$ 或 $6$.
			\item $\Re z=x^2-3x-4=0$, 即 $x=-1$ 或 $4$.
				\onslide<+->{%
					但同时要求 $\Im z=x^2-5x-6\neq 0$, 因此 $x\neq -1$.
				}\onslide<+->{%
					故 $x=4$.
				}
		\end{enumerate}
	\end{solution}
	\onslide<+->
	\begin{exercise}
		若 $x^2(1+i)+x(5+4i)+4+3i$ 是纯虚数, 则实数 $x=$\fillblank{\visible<+->{$-4$}}.
	\end{exercise}
\end{frame}


\subsection{复数的代数运算}

\begin{frame}{复数的加法与减法}
	\onslide<+->
	设 $z_1=x_1+y_1i,z_2=x_2+y_2i$.
	\onslide<+->
	由 $\BC$ 是二维实线性空间可得复数的加法和减法:
	\onslide<+->
	\[
		z_1+z_2=(x_1+x_2)+(y_1+y_2)i,\quad
		\visible<+->{z_1-z_2=(x_1-x_2)+(y_1-y_2)i.}
	\]
	\onslide<+->
	复数的加减法与其对应的向量 $\overrightarrow{OZ}$ 的加减法是一致的.
	\onslide<1->
	\begin{center}
		\begin{tikzpicture}[scale=.8]
			\draw[cstaxis] (-2,0)--(4,0);
			\draw[cstaxis] (0,-3)--(0,2.5);
			\coordinate (O) at (0,0);
			\coordinate [label=right:\textcolor{main}{$z_1$}] (Z1) at (2.5,-1);
			\coordinate [label=above:\textcolor{main}{$z_2$}] (Z2) at (1.5,2);
			\begin{scope}[visible on=<3->]
				\coordinate [label=above right:\textcolor{second}{$z_1+z_2$}] (P) at ($(Z1)+(Z2)$);
				\draw[cstcurve,cstra,second] (O)--(P);
				\draw[cstdash] (Z2)--(P)--(Z1);
			\end{scope}
			\begin{scope}[visible on=<4->]
				\coordinate [label=below:\textcolor{third}{$z_1-z_2$}] (M) at ($(Z1)-(Z2)$);
				\coordinate [label=left:{$-z_2$}] (neg) at ($(O)-(Z2)$);
				\draw[cstcurve,cstra,third] (O)--(M);
				\draw[cstdash,cstra] (O)--(neg);
				\draw[cstdash] (Z1)--(M)--(neg);
			\end{scope}
			\draw[cstcurve,cstra,main] (O)--(Z1);
			\draw[cstcurve,cstra,main] (O)--(Z2);
		\end{tikzpicture}
	\end{center}
\end{frame}


\begin{frame}{复数的乘除法}
	\onslide<+->
	\alert{规定 $i\cdot i=-1$} 并要求实数与复数的乘法和标量乘法(数乘)一致.
	\onslide<+->
	我们希望 $\BC$ 上的运算满足乘法分配律,
	\onslide<+->
	则
	\begin{align*}
		z_1\cdot z_2&=x_1\cdot x_2+x_1\cdot y_2i+y_1i\cdot x_2+y_1i\cdot y_2i\\
		&=(x_1x_2-y_1y_2)+(x_1y_2+x_2y_1)i,
	\end{align*}
	\onslide<+->
	由此可得 $z\neq0$ 时,
	\[\frac1{z}=\frac{x-yi}{x^2+y^2},\]
	\onslide<+->
	从而
	\[\frac{z_1}{z_2}=\frac{x_1x_2+y_1y_2}{x_2^2+y_2^2}+\frac{x_2y_1-x_1y_2}{x_2^2+y_2^2}i.\]

	\onslide<+->
	对于正整数 $n$, 定义 $z$ 的 \emph{$n$ 次幂}为 $n$ 个 $z$ 相乘.

	\onslide<+->
	当 $z\neq 0$ 时, 还可以定义 $z^0=1,z^{-n}=\dfrac1{z^n}$.
\end{frame}


\begin{frame}{例题: 单位根}
	\onslide<+->
	\begin{example}
		\begin{enumerate}
			\item $i^2=-1,i^3=-i,i^4=1$.
			\onslide<+->{%
				一般地, 对于整数 $n$, 
				\[i^{4n}=1,\quad i^{4n+1}=i,\quad i^{4n+2}=-1,\quad i^{4n+3}=-i.\]
			}
			\vspace{-\baselineskip}
			\item 令 $\omega=\dfrac{-1+\sqrt 3i}2$, 则 $\omega^2=\dfrac{-1-\sqrt3i}2,\omega^3=1$.
			\item 令 $z=1+i$, \onslide<+->{则
			\[z^2=2i,\quad z^3=-2+2i,\quad z^4=-4,\quad z^8=16=2^4.\]}
		\end{enumerate}
		\vspace{-\baselineskip}
	\end{example}
	\onslide<+->
	将满足 $z^n=1$ 的复数 $z$ 称为 \emph{$n$ 次单位根}.
	\onslide<+->
	那么 $1,i,-1,-i$ 是 $4$ 次单位根, $1,\omega,\omega^2$ 是 $3$ 次单位根.
\end{frame}


\begin{frame}{例题: 单位根}
	\onslide<+->
	\begin{example}
		化简 $1+i+i^2+i^3+i^4$.
	\end{example}
	\onslide<+->
	\begin{solution}
		根据等比数列求和公式,
		\[1+i+i^2+i^3+i^4=\frac{i^5-1}{i-1}
		\visible<+->{=\frac{i-1}{i-1}=1.}\]
	\end{solution}
	\onslide<+->
	\begin{exercise}[2020年A卷]
		化简 $\left(\dfrac{1-i}{1+i}\right)^{2020}$=\fillblank{\visible<+->{$1$}}.
	\end{exercise}
\end{frame}


\begin{frame}{复数域\noexer}
	\onslide<+->
	复数全体构成一个\emph{域}.
	\onslide<+->
	所谓的域, 是指带有如下内容和性质的集合
	\begin{itemize}
		\item 包含 $0,1$, 且有四则运算;
		\item 满足加法结合/交换律, 乘法结合/交换/分配律;
		\item 对任意 $a$, $a+0=a\times 1=a$.
	\end{itemize}
	\onslide<+->
	有理数全体 $\BQ$, 实数全体 $\BR$ 也构成域, 它们是 $\BC$ 的子域.
	\onslide<+->
	与有理数域和实数域有着本质不同的是, 复数域是\emph{代数闭域}:
	\onslide<+->
	对于任何次数 $n\ge 1$ 的复系数多项式
		\[p(z)=z^n+c_{n-1}z^{n-1}+\cdots+c_1z+c_0,\]
	都存在复数 $z_0$ 使得 $p(z_0)=0$.
	\onslide<+->
	由此不难知道, 复系数多项式可以因式分解成一次多项式的乘积.
	\onslide<+->
	我们会在第五章证明该结论.
\end{frame}


\begin{frame}{复数域不是有序域\noexer}
	\onslide<+->
	在 $\BQ,\BR$ 上可以定义出一个好的大小关系,
	\onslide<+->
	换言之它们是有序域, 即存在一个满足下述性质的 $>$:
	\begin{itemize}
		\item 若 $a\neq b$, 则要么 $a>b$, 要么 $b>a$;
		\item 若 $a>b$, 则对于任意 $c$, $a+c>b+c$;
		\item 若 $a>b,c>0$, 则 $ac>bc$.
	\end{itemize}
	\onslide<+->
	而 \alert{$\BC$ 却不是有序域}.
	\onslide<+->
	如果 $i>0$, 则
		\[-1=i\cdot i>0,\quad -i=-1\cdot i>0.\]
	\onslide<+->
	于是 $0>i$, 矛盾! 同理 $i<0$ 也不可能.
\end{frame}


\subsection{共轭复数}

\begin{frame}{共轭复数}
	\onslide<+->
	\begin{definition}[共轭复数]
		称 $z$ 在复平面关于实轴的对称点为它的\emph{共轭复数 $\ov z$}.
	换言之, $\ov{x+yi}=x-yi$.
	\end{definition}
	\onslide<+->
	从定义出发, 不难验证共轭复数满足如下性质:
	\begin{enumerate}
		\item $z$ 是 $\ov z$ 的共轭复数.
		\item $\ov{z_1\pm z_2}=\ov{z_1}\pm\ov{z_2},\ 
		\ov{z_1\cdot z_2}=\ov{z_1}\cdot\ov{z_2},\ 
		\ov{z_1/z_2}=\ov{z_1}/\ov{z_2}$.
		\item $z\ov{z}=(\Re z)^2+(\Im z)^2$.
		\item $z+\ov z=2\Re z,\ z-\ov z=2i\Im z$.
		\item $z=\ov z\iff z$ 是实数; $z=-\ov z\iff z$ 是纯虚数或 $z=0$.
	\end{enumerate}
	\onslide<+->
	\enumnum4表明了 $x,y$ 可以用 $z,\ov z$ 表出.
	\onslide<+->
	\enumnum2表明共轭复数和四则运算交换.
	\onslide<+->
	这意味着使用共轭复数进行计算和证明,往往比直接使用 $x,y$ 表达的形式更简单.
\end{frame}


\begin{frame}{例题:共轭复数证明等式}
	\onslide<+->
	\begin{exercise}
		$z$ 关于虚轴的对称点是\fillblank{\visible<+->{$-\ov z$}}.
	\end{exercise}
	\onslide<+->
	\begin{example}
		证明 $z_1\cdot\ov{z_2}+\ov{z_1}\cdot z_2=2\Re(z_1\cdot\ov{z_2})$.
	\end{example}
	\onslide<+->
	我们可以设 $z_1=x_1+y_1i,z_2=x_2+y_2i$, 然后代入等式两边化简并比较实部和虚部得到.
	\onslide<+->
	但我们利用共轭复数可以更简单地证明它.
	\onslide<+->
	\begin{proof}
		\onslide<+->{%
			由于 $\ov{z_1\cdot\ov{z_2}}=\ov{z_1}\cdot\ov{\ov{z_2}}=\ov{z_1}\cdot z_2$, 
		}\onslide<+->{%
			因此
			\[z_1\cdot\ov{z_2}+\ov{z_1}\cdot z_2
				=z_1\cdot\ov{z_2}+\ov{z_1\cdot\ov{z_2}}
				=2\Re(z_1\cdot\ov{z_2}).\qedhere\]
		}
		\vspace{-\baselineskip}
	\end{proof}
\end{frame}


\begin{frame}{例题:共轭复数判断实数}
	\onslide<+->
	\begin{example}
		设 $z=x+yi$ 且 $y\neq 0,\pm1$. 证明: $x^2+y^2=1$ 当且仅当 $\dfrac z{1+z^2}$ 是实数.
	\end{example}
	\onslide<+->
	\begin{proof}
		$\dfrac z{1+z^2}$ 是实数当且仅当
		\[\frac z{1+z^2}=\ov{\left(\frac z{1+z^2}\right)}=\frac{\ov z}{1+{\ov z}^2},\]
		\onslide<+->{%
			即
			\[z(1+{\ov z}^2)=\ov z(1+z^2),\quad (z-\ov z)(z\ov z-1)=0.\]
		}\onslide<+->{%
			由 $y\neq0$ 可知 $z\neq \ov z$.
		}\onslide<+->{%
			故上述等式等价于 $z\ov z=1$, 即 $x^2+y^2=1$.\qedhere
		}
	\end{proof}
\end{frame}

\begin{frame}{例题: 复数的代数计算}
	\onslide<+->
	由于 $z\ov z$ 是一个实数,
	\onslide<+->
	因此在做复数的除法运算时, 可以利用下式将其转化为乘法:
		\[\dfrac{z_1}{z_2}=\dfrac{z_1\ov{z_2}}{z_2\ov{z_2}}=\dfrac{z_1\ov{z_2}}{x_2^2+y_2^2}.\]
	\vspace{-.5\baselineskip}
	\onslide<+->
	\begin{example}
		$z=-\dfrac1i-\dfrac{3i}{1-i}$, 求 $\Re z,\Im z$ 以及 $z\ov z$.
	\end{example}
	\onslide<+->
	\begin{solution}
		\[z=-\frac1i-\frac{3i}{1-i}
		\onslide<+->{=i-\frac{3i-3}2=\frac32-\half i,}\]
		\onslide<+->{%
			因此
				\[\Re z=\frac32,\quad\Im z=-\half ,\quad
				z\ov z=\left(\frac32\right)^2+\left(-\half\right)^2=\frac52.\]
		}\vspace{-\baselineskip}
	\end{solution}
\end{frame}


\begin{frame}{例题: 复数的代数计算}
	\onslide<+->
	\begin{example}
		设 $z_1=5-5i,z_2=-3+4i$, 求 $\ov{\left(\dfrac{z_1}{z_2}\right)}$.
	\end{example}
	\onslide<+->
	\begin{solution}
		\begin{align*}
			\frac{z_1}{z_2}&=\frac{5-5i}{-3+4i}
			\onslide<+->{=\frac{(5-5i)(-3-4i)}{(-3)^2+4^2}}\\
			&\onslide<+->{=\frac{(-15-20)+(-20+15)i}{25}}
			\onslide<+->{=-\frac75-\frac15i,}
		\end{align*}
		\onslide<+->{%
			因此 $\ov{\left(\dfrac{z_1}{z_2}\right)}=-\dfrac75+\dfrac15i$.
		}
	\end{solution}
\end{frame}

