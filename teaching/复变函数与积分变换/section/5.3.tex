\section{留数在定积分的应用}

\subsection{正弦余弦的有理函数的积分}

\begin{frame}{形如 \raisebox{2pt}{\resizet{0.85}{$\dint_0^{2\pi}$}}$R(\cos\theta,\sin\theta)\diff\theta$ 的积分\noexer}
	\onslide<+->
	本节中我们将对若干种在实变中难以计算的定积分和广义积分使用复变函数和留数的技巧进行计算.
	\onslide<+->
	本节内容不作考试要求.

	\onslide<+->
	考虑 $\dint_0^{2\pi} R(\cos\theta,\sin\theta)\diff\theta$, 其中 $R$ 是一个有理函数.
	\onslide<+->
	令 $z=\ee^{\ii\theta}$, 则 $\diff z=iz\diff\theta$,
	\onslide<+->
	\[\cos\theta=\half\Bigl(z+\frac1z\Bigr)=\frac{z^2+1}{2z},\quad
	\sin\theta=\frac1{2\ii}\Bigl(z-\frac1z\Bigr)=\frac{z^2-1}{2\iiz},
	\]
	\onslide<+->
	\[\alertn{\int_0^{2\pi} R(\cos\theta,\sin\theta)\diff\theta=\oint_{|z|=1}R\Bigl(\frac{z^2+1}{2z},\frac{z^2-1}{2\iiz}\Bigr)\frac1{iz}\diff z.}
	\]
	\onslide<+->
	由于被积函数是一个有理函数, 它的积分可以由 $|z|<1$ 内奇点留数得到.
\end{frame}


\begin{frame}{例题: 第一类积分\noexer}\small
	\beqskip{1pt}
	\onslide<+->
	\begin{example}
		求 $\dint_0^{2\pi}\frac{\sin^2\theta}{5-3\cos\theta}\diff\theta$.
	\end{example}

	\onslide<+->
	\begin{solution}
		令 $z=\ee^{\ii\theta}$, 则 $\diff z=iz\diff\theta$,
		\onslide<+->{
			\[\cos\theta=\half\Bigl(z+\frac1z\Bigr)=\frac{z^2+1}{2z},\qquad
			\sin\theta=\frac1{2\ii}\Bigl(z-\frac1z\Bigr)=\frac{z^2-1}{2\iiz},
	\]
		}\onslide<+->{
			\[
				\int_0^{2\pi}\frac{\sin^2\theta}{5-3\cos\theta}\diff\theta
				=\oint_{|z|=1}\frac{(z^2-1)^2}{-4z^2}\cdot\frac1{5-3\dfrac{z^2+1}{2z}}\cdot\frac{\diff z}{iz}
				=-\frac i6\oint_{|z|=1}\frac{(z^2-1)^2}{z^2(z-3)(z-\dfrac13)}\diff z.
	\]
		}\onslide<+->{%
			设 $f(z)=\dfrac{(z^2-1)^2}{z^2(z-3)(z-\frac13)}$,
		}\onslide<+->{%
			则
			$\Res[f(z),0]=\dfrac{10}3, \Res[f(z),\dfrac13]=-\dfrac83$,
		}
		\onslide<+->{
			\[
				\int_0^{2\pi}\frac{\sin^2\theta}{5-3\cos\theta}\diff\theta
				=-\frac i6\cdot 2\pi\ii\Bigl[\Res[f(z),0]+\Res[f(z),\frac13]\Bigr]
				=\frac{2\pi}9.
	\]
		}
		\vspace{-.5\baselineskip}
	\end{solution}
	\endgroup
\end{frame}

\subsection{有理函数的广义积分}

\begin{frame}{形如 \raisebox{2pt}{\resizet{0.85}{$\dint_{-\infty}^{+\infty}$}}$R(x)\diff x$ 的积分\noexer}
	\onslide<+->
	考虑 $\dint_{-\infty}^{+\infty}R(x)\diff x$, 其中 $R(x)$ 是一个有理函数, 分母比分子至少高 $2$ 次, 且分母没有实根.
	\onslide<+->
	我们先考虑 $\dint_{-r}^rR(x)\diff x$.
	\onslide<+->
	设 $f(z)=R(z),C=C_r+[-r,r]$ 如下图所示, 使得上半平面内 $f(z)$ 的奇点均在 $C$ 内,
	\onslide<+->
	则
	\[2\pi\ii\sum_{\Im a>0}\Res[f(z),a]=\oint_Cf(z)\diff z=\int_{-r}^rR(x)\diff x+\int_{C_r}f(z)\diff z.
	\]
	\onslide<3->
	\begin{center}
		\begin{tikzpicture}
			\draw[cstaxis] (-2,0)--(2,0);
			\draw[cstaxis] (0,-0.2)--(0,2);
			\draw[cstcurve,main] (-1.5,0) arc(180:0:1.5);
			\draw[cstcurve,main,cstla] (-1.2,0.9) arc(135:130:1.5);
			\draw[cstcurve,second] (-1.5,0)--(1.5,0);
			\draw[cstcurve,second,cstra] (-1.5,0)--(-0.5,0);
			\draw
				(-1.5,-0.3) node[second] {$-r$}
				(1.5,-0.3) node[second] {$r$}
				(1.3,1.2) node[main] {$C_r$};
		\end{tikzpicture}
	\end{center}
\end{frame}


\begin{frame}{形如 \raisebox{2pt}{\resizet{0.85}{$\dint_{-\infty}^{+\infty}$}}$R(x)\diff x$ 的积分\noexer}
	\onslide<+->
	由于 $P(x)$ 分母次数比分子至少高 $2$ 次,
	\onslide<+->
	当 $r\to+\infty$ 时,
	\[\abs{\int_{C_r}f(z)\diff z}\le \pi r\max_{|z|=r}|f(z)|
	=\pi \max_{|z|=r}|zf(z)|\to 0.
	\]
	\onslide<+->
	故
	\[\alertn{\int_{-\infty}^{+\infty}R(x)\diff x=2\pi\ii\sum_{\Im a>0}\Res[R(z),a].}
	\]
	\onslide<+->
	这里我们考虑的广义积分都是指柯西主值
	\[P.V.\int_{-\infty}^{+\infty}f(x)\diff x=\lim_{a\to+\infty}\int_{-a}^af(x)\diff t.
	\]
\end{frame}


\begin{frame}{例题: 留数在定积分上的应用\noexer}
	\onslide<+->
	\begin{example}
		求 $\dint_{-\infty}^{+\infty}\frac{\diff x}{(x^2+a^2)^3},a>0$.
	\end{example}

	\onslide<+->
	\begin{solution}
		$f(z)=\dfrac1{(z^2+a^2)^3}$ 在上半平面内的奇点为 $ai$.
		\bigdel
		\onslide<+->{
			\begin{align*}
			\Res[f(z),ai]&=\frac1{2!}\lim_{z\to ai}\left[\frac1{(z+ai)^3}\right]''\\
			&=\half\lim_{z\to ai}\frac{12}{(z+ai)^5}=\frac{3}{16a^5\ii},
			\end{align*}
		}\onslide<+->{故
			\[\int_{-\infty}^{+\infty}\frac{\diff x}{(x^2+a^2)^3}
		=2\pi\ii\Res[f(z),ai]=\frac{3\pi}{8a^5}.
	\]
		}
		\vspace{-.5\baselineskip}
	\end{solution}
\end{frame}


\subsection{有理函数与三角函数之积的广义积分}

\begin{frame}{形如 \raisebox{2pt}{\resizet{0.85}{$\dint_{-\infty}^{+\infty}$}}$R(x)\cos{\lambda x}\diff x$,
	\raisebox{2pt}{\resizet{0.85}{$\dint_{-\infty}^{+\infty}$}}$R(x)\sin{\lambda x}\diff x$ 的积分\noexer}
	\onslide<+->
	考虑 $\dint_{-\infty}^{+\infty}R(x)\cos{\lambda x}\diff x$,
	$\dint_{-\infty}^{+\infty}R(x)\sin{\lambda x}\diff x$, 其中 $R(x)$ 是一个有理函数, 分母比分子至少高 $2$ 次, 且分母没有实根.
	\onslide<+->
	和前一种情形类似, 我们有
	\[\alertn{\int_{-\infty}^{+\infty}R(x)\ee^{\ii\lambda x}\diff x
	=2\pi\ii\sum_{\Im a>0}\Res[R(z)\ee^{\ii\lambda z},a],}
	\]
	\onslide<+->
	因此所求积分分别为它的实部和虚部.
\end{frame}


\begin{frame}{例题: 留数在定积分上的应用\noexer}
	\beqskip{2pt}
	\onslide<+->
	\begin{example}
		求 $\dint_{-\infty}^{+\infty}\frac{\cos x\diff x}{(x^2+a^2)^2}, a>0$.
	\end{example}

	\onslide<+->
	\begin{solution}
		$f(z)=\dfrac{\ee^{iz}}{(z^2+a^2)^2}$ 在上半平面内的奇点为 $ai$,
		\onslide<+->{
			\[\Res[f(z),ai]=\lim_{z\to ai}\left[\frac{\ee^{iz}}{(z+ai)^2}\right]'=-\frac{\ee^{-a}(a+1)\ii}{4a^3}.
	\]
		}\onslide<+->{故
			$\displaystyle\qquad \int_{-\infty}^{+\infty}\frac{\ee^{ix}\diff x}{(x^2+a^2)^2}=2\pi\ii \Res[f(z),ai]=\frac{\pi \ee^{-a}(a+1)}{2a^3}$,
		}\onslide<+->{
			\[\int_{-\infty}^{+\infty}\frac{\cos x\diff x}{(x^2+a^2)^2}=\frac{\pi \ee^{-a}(a+1)}{2a^3}.
	\]
		}
	\end{solution}
	\endgroup
\end{frame}


\subsection{其它例子}

\begin{frame}{例题: 留数在定积分上的应用\noexer}
	\onslide<+->
	最后我们再来看一个例子.

	\onslide<+->
	\begin{example}
		求积分 $I=\dint_0^{+\infty}\frac{x^p}{x(x+1)}\diff x,0<p<1$.
	\end{example}

	\onslide<+->
	\begin{solution}
	\[
		I=\int_0^{+\infty}\frac{x^p}{x(x+1)}\diff x\xto{\text{令}\ x=\ee^t}\int_{-\infty}^{+\infty}\frac{\ee^{pt}}{\ee^t+1}\diff t.
	\]
		\onslide<+->{考虑 $f(z)=\dfrac{\ee^{pz}}{\ee^z+1}$ 在如下闭路 $C$ 上的积分.
		\begin{center}
			\begin{tikzpicture}[scale=.8]
				\draw[cstaxis] (-3,0)--(3,0);
				\draw[cstaxis] (0,0)--(0,1.7);
				\draw[cstcurve,second] (-1.5,0) rectangle (1.5,1.2);
				\draw[cstcurve,main] (-1.5,0)--(-1.5,1.2);
				\draw[cstcurve,main] (1.5,0)--(1.5,1.2);
				\draw[cstcurve,second,cstra] (-1.5,0)--(-0.5,0);
				\draw[cstcurve,second,cstra] (0,1.2)--(-0.5,1.2);
				\draw[cstcurve,main,cstra] (1.5,0)--(1.5,0.9);
				\draw[cstcurve,main,cstra] (-1.5,1.2)--(-1.5,0.3);
				\draw
					(1.8,0.3) node {$R$}
					(-2,0.3) node {$-R$}
					(0.4,0.8) node {$2\pi\ii$}
					(1.2,0.4) node[main] {$C_1$}
					(-1.1,0.5) node[main] {$C_2$}
					(-0.4,0.8) node[second] {$l$};
			\end{tikzpicture}
		\end{center}}
	\end{solution}
\end{frame}


\begin{frame}{例题: 留数在定积分上的应用\noexer}
	\onslide<+->
	\begin{solution}[续解]
			由于 $l:z=t+2\pi\ii,-R\le t\le R$,
		\onslide<+->{因此
			\[\int_l f(z)\diff z
			=\int_R^{-R}\frac{\ee^{2p\pi\ii}\cdot \ee^{pt}}{\ee^t+1}\diff t
			=-\ee^{2p\pi\ii}\int_{-R}^Rf(t)\diff t.
	\]
		}\onslide<+->{由于 $C_1:z=R+\iit,0\le t\le 2\pi$,
		}\onslide<+->{因此
			\[\abs{\int_{C_1}f(z)\diff z}\le \frac{\ee^{pR}}{\ee^R-1}\cdot 2\pi\to 0\quad(R\to+\infty).
	\]
		}\onslide<+->{同理
			\[\abs{\int_{C_2}f(z)\diff z}\le \frac{\ee^{-pR}}{1-\ee^{-R}}\cdot 2\pi\to 0\quad(R\to+\infty).
	\]
		}
	\end{solution}
\end{frame}


\begin{frame}{例题: 留数在定积分上的应用\noexer}
	\onslide<+->
	\begin{solution}[续解]
		由于
	\[
		\Res[f(z),\pi\ii]
		=\frac{\ee^{pz}}{(\ee^z+1)'}\bigg|_{z=\pi\ii}=-\ee^{p\pi\ii},
	\]
		\onslide<+->{因此
			\begin{align*}
			&\Bigl(\int_{-R}^R+\int_l+\int_{C_1}+\int_{C_2}\Bigr)f(z)\diff z\\
			=&\oint_Cf(z)\diff z=2\pi\ii\Res[f(z),\pi\ii]=-2\pi\ii\ee^{p\pi\ii},
			\end{align*}
		}\onslide<+->{令 $R\to+\infty$,
		}\onslide<+->{则
			\begin{align*}
			&(1-\ee^{2p\pi\ii})I=-2\pi\ii\ee^{p\pi\ii},\quad
			\visible<+->{I=\frac{2\pi\ii}{\ee^{p\pi\ii}-\ee^{-p\pi\ii}}=\frac{\pi}{\sin p\pi}.}
			\end{align*}
		}
		\bigdel
	\end{solution}
\end{frame}

