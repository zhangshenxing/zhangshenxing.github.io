\section{幂级数}


\subsection{幂级数及其收敛圆}

\begin{frame}{函数项级数与幂级数}
	\onslide<+->
	复变函数级数与实变量函数级数也是类似的.
	\onslide<+->
	\begin{definition}[near]
		\begin{enumerate}
			\item 设 $\{f_n(z)\}_{n\ge 1}$ 是一个复变函数列, 其中每一项都在区域 $D$ 上有定义.
			表达式 $\sumf1 f_n(z)$ 称为\emph{复变函数项级数}.
			\item 对于 $z_0\in D$, 若级数 $\sumf1 f_n(z_0)$ 收敛, 则称 \emph{$\sumf1 f_n(z)$ 在 $z_0$ 处收敛}, 相应级数的值称为它的\emph{和}.
			\item 若 $\sumf1 f_n(z)$ 在 $D$ 上处处收敛, 则它的和是一个函数, 称为\emph{和函数}.
			\item 称形如 $\sumf0 c_n(z-a)^n$ 的函数项级数为\emph{幂级数}.
		\end{enumerate}\meddel
	\end{definition}
\end{frame}


\begin{frame}{阿贝尔定理}
	\onslide<+->
	我们只需要考虑 $a=0$ 情形的幂级数, 因为二者的收敛范围与和函数只是差一个平移.
	\onslide<+->
	\begin{theorem*}[][阿贝尔定理]
		\begin{enumerate}
			\item 若 $\sumf0 c_nz^n$ 在 $z_0\neq 0$ 处收敛, 那么对任意 $\abs{z}<\abs{z_0}$ 的 $z$, 该级数必绝对收敛.
			\item 若 $\sumf0 c_nz^n$ 在 $z_0\neq 0$ 处发散, 那么对任意 $\abs{z}>\abs{z_0}$ 的 $z$, 该级数必发散.
		\end{enumerate}
	\end{theorem*}
\end{frame}


\begin{frame}{阿贝尔定理}
	\onslide<+->
	\begin{proof}
		\begin{enumerate}
			\item 因为级数收敛, 所以 $\liml_{n\to\infty}c_n z_0^n=0$.
			\onslide<+->{%
				故存在 $M$ 使得 $\abs{c_nz_0^n}<M$.
			}\onslide<+->{%
				对于 $\abs{z}<\abs{z_0}$,
				\[
					\sumf0\abs{c_nz^n}
					=\sumf0\abs{c_nz_0^n}\cdot\Bigabs{\frac z{z_0}}^n
					\visible<+->{\le M\sumf0\Bigabs{\frac z{z_0}}^n
					=\frac{M}{1-\Bigabs{\dfrac z{z_0}}}.}
				\]
			}\onslide<+->{%
				所以级数在 $z$ 处绝对收敛.
			}
			\item 是\enumnum1的逆否命题.\qedhere
		\end{enumerate}
	\end{proof}
\end{frame}


\begin{frame}{幂级数的收敛半径}
	\onslide<+->
	设 $R$ 是实幂级数 $\sumf0\abs{c_n}x^n$ 的收敛半径.
	\begin{itemize}
		\item 若 $R=+\infty$, 由阿贝尔定理可知 $\sumf0 c_nz^n$ 处处绝对收敛.
		\item 若 $0<R<+\infty$, 那么 $\sumf0 c_nz^n$ 在 $\abs{z}<R$ 上绝对收敛, 在 $\abs{z}>R$ 上发散.
		\item 若 $R=0$, 那么 $\sumf0 c_nz^n$ 仅在 $z=0$ 处收敛, 对任意 $z\neq 0$ 都发散.
	\end{itemize}
	\onslide<+->
	我们称 $R$ 为该幂级数的\emph{收敛半径}.
	\onslide<+->
	\begin{center}
		\begin{tikzpicture}
			\filldraw[cstcurve,main,cstfill] (0,0) circle (1.2);
			\fill[cstdot] (0,0) circle;
			\draw[cstcurve,cstra] (0,0)--(0.96,0.72);
			\draw[cstcurve,main,cstra] (-0.96,0.72)--(-2,0.72);
			\draw
				(0.6,0.1) node {$R$}
				(0,-0.4) node[second] {绝对收敛}
				(2.5,-0.4) node[second] {发散}
				(-3,0.72) node[main] {都有可能};
		\end{tikzpicture}
	\end{center}
\end{frame}


\begin{frame}{例题: 收敛半径的计算}
	\onslide<+->
	\begin{example}[nearnext]
		求幂级数 $\sumf0 z^n=1+z+z^2+\cdots$ 的收敛半径与和函数.
	\end{example}
	\onslide<+->
	\begin{solution}[nearprev]
		若幂级数收敛, 则由 $z^n\to0$ 可知 $\abs{z}<1$.
		\onslide<+->{当 $\abs{z}<1$ 时, 和函数为
			\[\lim_{n\to\infty}s_n=\lim_{n\to\infty}\frac{1-z^{n+1}}{1-z}=\frac1{1-z}.
	\]
		}\onslide<+->{因此收敛半径为 $1$.}
	\end{solution}
\end{frame}


\subsection{收敛半径的计算}
\begin{frame}{收敛半径的计算}
	\onslide<+->
	由正项级数的相应判别法容易得到公式 $\alertm{R=\dfrac1r}$, 其中
	\begin{enumerate}
		\item \alert{达朗贝尔公式(比值法)}: $\alertm{r=\displaystyle\lim_{n\to\infty}\Bigabs{\frac{c_{n+1}}{c_n}}}$ (假设存在);
		\item 柯西公式(根式法): $r=\displaystyle\lim_{n\to\infty}\sqrt[n]{\abs{c_n}}$ (假设存在);
		\item 柯西-阿达马公式: $r=\displaystyle\ov{\lim_{n\to\infty}} \sqrt[n]{\abs{c_n}}$.
	\end{enumerate}
	\onslide<+->
	若 $r=0$ 或 $+\infty$, 则 $R=+\infty$ 或 $0$.
\end{frame}


\begin{frame}{典型例题: 收敛半径的计算}
	\onslide<+->
	\begin{example}[nearnext]
		求幂级数 $\sumf1\dfrac{(z-1)^n}n$ 的收敛半径, 并讨论 $z=0,2$ 处的敛散性.
	\end{example}
	\onslide<+->
	\begin{solution}[nearprev,indent]
		由
		\[
			\lim_{n\to\infty}\Bigabs{\frac{c_{n+1}}{c_n}}
			=\lim_{n\to\infty}\frac n{n+1}=1
		\]
		可知收敛半径为 $1$.

		\onslide<+->{当 $z=2$ 时, $\displaystyle\sumf1\frac{(z-1)^n}n=\sumf1\frac1n$ 发散.}

		\onslide<+->{当 $z=0$ 时, $\displaystyle\sumf1\frac{(z-1)^n}n=\sumf1\frac{(-1)^n}n$ 收敛.}
	\end{solution}
\end{frame}


\begin{frame}{典型例题: 收敛半径的计算}
	\onslide<+->
	\begin{example}[nearnext]
		求幂级数 $\displaystyle\sumf1\frac{z^n}{n(n+1)}$ 的收敛半径, 并讨论收敛圆周上的情形.
	\end{example}
	\onslide<+->
	\begin{solution}[nearprev,indent]
		由
		\[
			\lim_{n\to\infty}\Bigabs{\frac{c_{n+1}}{c_n}}
			=\lim_{n\to\infty}\frac {n(n+1)}{(n+1)(n+2)}=1
		\]
		可知收敛半径为 $1$.

		\onslide<+->{%
		当 $\abs{z}=1$ 时, $\displaystyle\sumf1\biggabs{\frac{z^n}{n(n+1)}}=\sumf1\frac1{n(n+1)}=1$ 收敛.
		}\onslide<+->{%
		因此该幂级数在收敛圆周上处处绝对收敛.
		}
	\end{solution}

	\onslide<+->
	事实上, \alert{收敛圆周上既可能处处收敛, 也可能处处发散, 也可能既有收敛的点也有发散的点}.
\end{frame}


\begin{frame}{典型例题: 收敛半径的计算}
	\beqskip{6pt}
	\onslide<+->
	\begin{example}[nearnext]
		求幂级数 $\sumf0\cos(\ii n)z^n$ 的收敛半径.
	\end{example}
	\onslide<+->
	\begin{solution}[nearprev]
		我们有 $c_n=\cos(\ii n)=\dfrac{\ee^n+\ee^{-n}}2$.
		\onslide<+->{%
			由
			\[
				\lim_{n\to\infty}\Bigabs{\frac{c_{n+1}}{c_n}}=\lim_{n\to\infty}\frac{\ee^{n+1}+\ee^{-n-1}}{\ee^n+\ee^{-n}}=\ee\lim_{n\to\infty}\frac{1+\ee^{-2n-2}}{1+\ee^{-2n}}=\ee
			\]
			可知收敛半径为 $1/\ee$.
		}
	\end{solution}
	\onslide<+->
	\begin{exercise}
		幂级数 $\sumf0(1+\ii)^nz^n$ 的收敛半径为\fillblankframe[2cm]{$\sqrt2/2$}.
	\end{exercise}
	\endgroup
\end{frame}


% \begin{frame}{典型例题: 收敛半径的计算\noexer}
% \onslide<+->
% \begin{example}[nearnext]
% 求幂级数 $\displaystyle\sumf1\frac{z^n}{n^p}$ 的收敛半径并讨论在收敛圆周上的情形, 其中 $p\in\BR$.
% \end{example}
% \onslide<+->
% \begin{solution}[nearprev]
% 由 $\displaystyle\lim_{n\to\infty}\abs{\frac{c_{n+1}}{c_n}}=\lim_{n\to\infty}\Bigl(\frac n{n+1}\Bigr)^p=1$ 可知收敛半径为 $1$.
% \onslide<+->{设 $\abs{z}=1$.
% \begin{itemize}
% \item 若 $p>1$, $\displaystyle\sumf1\abs{\frac{z^n}{n^p}}=\sumf1\frac1{n^p}$ 收敛,
% \onslide<+->{原级数在收敛圆周上处处绝对收敛.}
% \item 若 $p\le 0$, $\abs{\dfrac{z^n}{n^p}}=\dfrac1{n^p}\not\to0$,
% \onslide<+->{原级数在收敛圆周上处处发散.}
% \end{itemize}}
% \end{solution}
% \end{frame}


% \begin{frame}{典型例题: 收敛半径的计算\noexer}
% \onslide<+->
% 回忆\emph{狄利克雷判别法}: 若 $\{a_n\}_{n\ge 1}$ 部分和有界, 实数项数列 $\{b_n\}_{n\ge 1}$ 单调趋于 $0$, 则 $\sumf1 a_nb_n$ 收敛.

% \onslide<+->
% \begin{solution}[][]
% \begin{itemize}
% \item 若 $0<p\le1$, $\displaystyle\sumf1\frac1{n^p}$ 发散, 
% \onslide<+->{而在收敛圆周上其它点 $z\neq1$ 处,
% \[\abs{z+z^2+\cdots+z^n}=\abs{\frac{z(1-z^n)}{1-z}}
% \le\frac{2}{\abs{1-z}}
	% \]
% 有界, 数列 $\{n^{-p}\}_{n\ge 1}$ 单调趋于 $0$,}
% \onslide<+->{因此 $\displaystyle\sumf1\frac{z^n}{n^p}$ 收敛.}
% \onslide<+->{故该级数在 $z=1$ 发散, 在收敛圆周上其它点收敛.}
% \end{itemize}
% \end{solution}
% \end{frame}


\subsection{幂级数的运算性质}
\begin{frame}{幂级数的有理运算}
	\onslide<+->
	\begin{theorem}
		设幂级数
	\[
		f(z)=\sumf0 a_nz^n,\abs{z}<R_1,\quad
		g(z)=\sumf0 b_nz^n,\abs{z}<R_2.
	\]
		\onslide<+->{那么当 $\abs{z}<R=\min\{R_1,R_2\}$ 时,
	\[
		(f\pm g)(z)=\sumf0 (a_n\pm b_n)z^n,\quad
		(fg)(z)=\sumf0\Bigl(\sum_{k=0}^na_kb_{n-k}\Bigr)z^n.\]}
	\end{theorem}

	\onslide<+->
	当 $f,g$ 的收敛半径相同时, $f\pm g$ 或 $fg$ 的收敛半径可以比 $f,g$ 的大.
\end{frame}


% \begin{frame}{幂级数的代换运算}
% 	\onslide<+->
% 	\begin{theorem}
% 		设幂级数
% 	\[
% 		f(z)=\sumf0 a_nz^n,\abs{z}<R,
% 	\]
% 		设函数 $\varphi(z)$ 在 $D$ 上满足 $\abs{\varphi(z)}<R$, 
% 		\onslide<+->{%
% 		那么当 $z\in D$ 时,
% 	\[
% 		f\bigl(\varphi(z)\bigr)\sumf0 a_n\bigl(\varphi(z)\bigr)^n.
% 	\]
% 		}
% 	\end{theorem}
% \end{frame}


\begin{frame}{幂级数的解析性质}
	\onslide<+->
	\begin{theorem}
		设幂级数 $\sumf0 c_nz^n$ 的收敛半径为 $R$, 则在 $\abs{z}<R$ 上:
		\begin{enumerate}
			\item 它的和函数 $f(z)=\sumf0 c_nz^n$ 解析,
			\item $f'(z)=\sumf1 nc_nz^{n-1}$,
			\item $\dint_0^zf(\zeta)\d \zeta=\sumf0 \frac{c_n}{n+1}z^{n+1}$.
		\end{enumerate}
	\end{theorem}

	\onslide<+->
	也就是说, \alert{在收敛圆内, 幂级数的和函数解析, 且可以逐项求导, 逐项积分}.

	\onslide<+->
	由于和函数在 $\abs{z}>R$ 上没有定义, 因此和函数在 $\abs{z}=R$ 上不可能解析.
\end{frame}


\begin{frame}{例题: 幂级数展开}
	\onslide<+->
	\begin{example}[nearnext]
		把函数 $\dfrac1{z-b}$ 表成形如 $\sumf0 c_n(z-a)^n$ 的幂级数, 其中 $a\neq b$.
	\end{example}
	\onslide<+->
	\begin{solution}[nearprev]
		\[
			\frac1{z-b}=\frac1{(z-a)-(b-a)}
			\visible<+->{=\frac1{a-b}\cdot\frac1{1-\dfrac{z-a}{b-a}}.}
		\]
		\onslide<+->{%
			当 $\abs{z-a\abs{<}b-a}$ 时,
		}\onslide<+->{%
			$\displaystyle\frac1{z-b}=\frac1{a-b}\sumf0\Bigl(\frac{z-a}{b-a}\Bigr)^n$,
		}\onslide<+->{%
			即
			\[
				\frac1{z-b}=-\sumf0\frac{(z-a)^n}{(b-a)^{n+1}},\quad\abs{z-a\abs{<}b-a}.
			\]
		}\bigdel
	\end{solution}
\end{frame}


\begin{frame}{典型例题: 幂级数的收敛半径与和函数}
	\onslide<+->
	\begin{example}[near]
		求幂级数 $\sumf1(2^n-1)z^{n-1}$ 的收敛半径与和函数.
	\end{example}
	\onslide<+->
	\begin{solution}[nearprev]
		由
		\[
			\lim_{n\to\infty}\Bigabs{\frac{c_{n+1}}{c_n}}
			=\lim_{n\to\infty}\frac{2^{n+2}-1}{2^{n+1}-1}=2
		\]
		可知收敛半径为 $\dfrac12$.
		\onslide<+->{%
			当 $\abs{z}<\dfrac12$ 时, $\abs{2z}<1$.
		}\onslide<+->{%
			从而
			\begin{align*}
				\sumf1(2^n-1)z^{n-1}&
				=\sumf1 2^n z^{n-1}-\sumf1 z^{n-1}\\
				&\visible<+->{=\frac2{1-2z}-\frac1{1-z}
				=\frac1{(1-2z)(1-z)}.}
			\end{align*}
		}\bigdel
	\end{solution}
\end{frame}


\begin{frame}{典型例题: 幂级数的收敛半径与和函数}
	\onslide<+->
	\begin{example}[nearnext]
		求幂级数 $\sumf0(n+1)z^n$ 的收敛半径与和函数.
	\end{example}
	\onslide<+->
	\begin{solution}[nearprev]
		由 $\displaystyle\lim_{n\to\infty}\Bigabs{\frac{c_{n+1}}{c_n}}=\lim_{n\to\infty}\frac{n+2}{n+1}=1$ 可知收敛半径为 $1$.
		\onslide<+->{%
			当 $\abs{z}<1$ 时,
			\[
				\sumf0 z^{n+1}
				=\frac z{1-z}
				=-1-\frac1{z-1},
			\]
		}\onslide<+->{%
			因此
			\[
				\sumf0 (n+1)z^n
				=\Bigl(-\frac1{z-1}\Bigr)'
				=\frac1{(z-1)^2},
					\quad \abs{z}<1.
			\]
		}\bigdel
	\end{solution}
\end{frame}


\begin{frame}{特定形式系数幂级数\noexer}
	\onslide<+->
	通过对
	\[
		1+\lambda z+\lambda^2 z^2+\cdots=\dfrac1{1-\lambda z}
	\]
	两边求 $k$ 阶导数可得
	\[
		\sum_{n=0}^{\infty}(n+k)\cdots(n+2)(n+1)\lambda^n z^n
		=\frac{k!}{(1-\lambda z)^{k+1}}.
	\]
	\onslide<+->
	因此若 $p(n)$ 是次数为 $m-1$ 的多项式, 那么
	\[
		\sumf0 p(n)\lambda^n z^n
		=\frac{P(z)}{(1-\lambda z)^{m}},
	\]
	其中 $P$ 是多项式.
\end{frame}


\begin{frame}{特定形式系数幂级数\noexer}
	\beqskip{4pt}
	\onslide<+->
	一般地, 若幂级数的系数形如
	\[
		c_n=p_1(n)\lambda_1^n+\cdots+p_k(n)\lambda_k^n,
	\]
	\onslide<+->
	则和函数一定是形如
	\[
		\sum_{n=0}^{\infty}c_nz^n
		=\frac{P(z)}{(1-\lambda_1z)^{m_1}\cdots(1-\lambda_kz)^{m_k}}
	\]
	的有理函数,	其中 $m_j=\deg p_j+1$.
	\onslide<+->
	反过来这样的分式展开成幂级数的系数也一定有上述形式, 至多有有限多项例外.
	\onslide<+->
	这可以帮助我们进行计算的验证.
	\onslide<+->
	\begin{exercise}[nearnext]
		求幂级数 $\sumf1\dfrac{z^n}n$ 的收敛半径与和函数.
	\end{exercise}
	\onslide<+->
	\begin{answer}[nearprev]
		收敛半径为 $1$, 和函数为 $-\ln(1-z)$.
	\end{answer}
	\endgroup
\end{frame}


\begin{frame}{例题: 函数项级数的积分}
	\onslide<+->
	\begin{example}[nearnext]
		求 $\doint_{\abs{z}=\half }\Bigl(\sumf{-1} z^n\Bigr)\d z$.
	\end{example}
	\onslide<+->
	\begin{solution}[nearprev]
		由于 $\sumf0 z^n$ 在 $\abs{z}<1$ 收敛,
		\onslide<+->{它的和函数解析.
		}\onslide<+->{因此
			\begin{align*}
			\oint_{\abs{z}=\half }\Bigl(\sumf{-1} z^n\Bigr)\d z
			&=\oint_{\abs{z}=\half }\frac1z\d z+\oint_{\abs{z}=\half }\Bigl(\sumf0 z^n\Bigr)\d z\\
			&\visible<+->{=2\pi\ii+0=2\pi\ii.}
		\end{align*}}
		\bigdel
	\end{solution}
\end{frame}

