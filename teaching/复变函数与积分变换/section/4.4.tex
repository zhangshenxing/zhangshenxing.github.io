\section{洛朗级数}

\subsection{双边幂级数}

\begin{frame}{双边幂级数}
	\onslide<+->
	如果解析函数 $f(z)$ 在 $z_0$ 处解析, 那么在 $z_0$ 处可以展开成泰勒级数.
	\onslide<+->
	如果 $f(z)$ 在 $z_0$ 处不解析呢?
	\onslide<+->
	此时 $f(z)$ 一定不能展开成 $z-z_0$ 的幂级数,
	\onslide<+->
	然而它却可能可以展开为\emph{双边幂级数}
	\onslide<+->
	\[\sum_{n=-\infty}^\infty c_n(z-z_0)^n=\alert{
	\underbrace{\sum_{n=1}^\infty c_{-n}(z-z_0)^{-n}}_{\text{\normalsize 负幂次部分}}}+
	\emph{\underbrace{\sum_{n=0}^\infty c_n(z-z_0)^n}_{\text{\normalsize 非负幂次部分}}}.\]
	\onslide<+->
	例如
	\[\frac1{z^2(1-z)}=\frac1{z^2}+\frac1z+1+z+z^2+\cdots,\quad 0<|z|<1.\]
\end{frame}


\begin{frame}{双边幂级数的敛散性}
	\onslide<+->
	为了保证双边幂级数的收敛范围有一个好的性质以便于我们使用, 我们对它的敛散性作如下定义:

	\onslide<+->
	\begin{definition}
		如果双边幂级数的非负幂次部分和负幂次部分作为函数项级数都收敛, 则我们称这个双边幂级数\emph{收敛}.
		否则我们称之为\emph{发散}.
	\end{definition}

	\onslide<+->
	注意双边幂级数的敛散性不能像幂级数那样通过部分和形成的数列的极限来定义,
	\onslide<+->
	因为使用不同的部分和选取方式会影响到极限的数值.
\end{frame}


\begin{frame}{双边幂级数的收敛域}
	\onslide<+->
	设 $\suml_{n=0}^\infty c_n(z-z_0)^n$ 的收敛半径为 $R_2$, 则它在 $|z-z_0|<R_2$ 内收敛, 在 $|z-z_0|>R_2$ 内发散.

	\onslide<+->
	对于负幂次部分, 令 $\zeta=\frac1{z-z_0}$, 那么负幂次部分是 $\zeta$ 的一个幂级数 $\suml_{n=1}^\infty c_{-n}\zeta^n$.
	\onslide<+->
	设该幂级数的收敛半径为 $R$, 则它在 $|\zeta|<R$ 内收敛, 在 $|\zeta|>R$ 内发散.
	\onslide<+->
	设 $R_1:=\frac1R$, 则 $\suml_{n=1}^\infty c_{-n}(z-z_0)^{-n}$ 在 $|z-z_0|>R_1$ 内收敛, 在 $|z-z_0|<R_1$ 内发散.

	\begin{enumerate}
		\item 如果 $R_1>R_2$, 则该双边幂级数处处不收敛.
		\item 如果 $R_1=R_2$, 则该双边幂级数只在圆周 $|z-z_0|=R_1$ 上可能有收敛的点.
		\onslide<+->
		此时没有收敛域.
		\item 如果 $R_1<R_2$, 则该双边幂级数在 $R_1<|z-z_0|<R_2$ 内收敛, 在 $|z-z_0|<R_1$ 或 $>R_2$ 内发散, 在圆周 $|z-z_0|=R_1$ 或 $R_2$ 上既可能发散也可能收敛.
	\end{enumerate}
\end{frame}


\begin{frame}{双边幂级数的收敛域}
	\onslide<+->
	因此\alert{双边幂级数的收敛域为圆环域 $R_1<|z-z_0|<R_2$}.

	\onslide<+->
	当 $R_1=0$ 或 $R_2=+\infty$ 时, 圆环域的形状会有所不同.
	\onslide<+->
	\begin{center}
		\begin{tikzpicture}
			\filldraw[cstcurve,draw=dcolora,cstfill] (0,0) circle (1.5);
			\filldraw[cstdote,draw=dcolora] (0,0) circle;
			\draw[dcolorb,cstarrowto,thick] (0.048,0.036)--(1.2,0.9);
			\draw
				(0.7,0.2) node[dcolorb] {$R_2$}
				(-0.3,0) node[dcolora] {$z_0$}
				(0,-2) node {$0<|z-z_0|<R_2$};

			\fill[cstfille,visible on=<+->] (2.5,-1.5) rectangle (5.5,1.5);
			\filldraw[cstcurve,fill=white,draw=dcolora,visible on=<.->] (4,0) circle (1); 
			\fill[cstdot,dcolorb,visible on=<.->] (4,0) circle;
			\draw[dcolorb,cstarrowto,thick,visible on=<.->] (4,0)--(4.6,-0.8);
			\draw
				(4.1,-0.6) node[dcolorb,visible on=<.->] {$R_1$}
				(3.7,0) node[dcolora,visible on=<.->] {$z_0$}
				(4,-2) node[visible on=<.->] {$R_1<|z-z_0|<+\infty$};

			\fill[cstfille,visible on=<+->] (6.5,-1.5) rectangle (9.5,1.5);
			\filldraw[cstdote,draw=dcolora,visible on=<.->] (8,0) circle;
			\draw
				(7.7,0) node[dcolora,visible on=<.->] {$z_0$}
				(8,-2) node[visible on=<.->] {$0<|z-z_0|<+\infty$};
		\end{tikzpicture}
	\end{center}

	\onslide<+->
	双边幂级数的非负幂次部分和负幂次部分在收敛圆环域内都收敛,
	\onslide<+->
	因此它们的和函数都解析($\zeta=\dfrac1{z-z_0}$ 关于 $z$ 解析), 且可以逐项求导、逐项积分.
	\onslide<+->
	从而\emph{双边幂级数的和函数也是解析的, 且可以逐项求导、逐项积分}.
\end{frame}


\begin{frame}{例: 双边幂级数的收敛域}
	\onslide<+->
	\begin{example}
		求双边幂级数 $\displaystyle\sum_{n=1}^\infty\frac{2^n}{z^n}+\sum_{n=0}^\infty\frac{z^n}{(2+i)^n}$ 的收敛域与和函数.
	\end{example}

	\onslide<+->
	\begin{solution}
		非负幂次部分收敛域为 $|z|<|2+i|=\sqrt5$, 负幂次部分收敛域为 $|z|>|2|=2$.
		\onslide<+->{因此该双边幂级数的收敛域为 $2<|z|<\sqrt5$.
		}\onslide<+->{此时
			\[\sum_{n=1}^\infty\frac{2^n}{z^n}+\sum_{n=0}^\infty\frac{z^n}{(2+i)^n}
			=\frac{\dfrac 2z}{1-\dfrac 2z}+\frac1{1-\dfrac z{2+i}}
			=\frac{-iz}{(z-2)(z-2-i)}.\]}
	\end{solution}
\end{frame}


\subsection{洛朗展开的形式}

\begin{frame}{洛朗级数}
	\onslide<+->
	反过来, 在圆环域内解析的函数也一定能展开为双边幂级数, 被称为\alert{洛朗级数}.

	\onslide<+->
	例如 $f(z)=\dfrac1{z(1-z)}$ 在 $z=0,1$ 以外解析.
	\onslide<+->
	在圆环域 $0<|z|<1$ 内,
	\[f(z)=\frac1z+\frac1{1-z}=\frac1z+1+z+z^2+z^3+\cdots\]
	\onslide<+->
	在圆环域 $1<|z|<+\infty$ 内,
	\[f(z)=\frac1z-\frac1z\cdot\frac1{1-\dfrac1z}=-\frac1{z^2}-\frac1{z^3}-\frac1{z^4}-\cdots\]
\end{frame}


\begin{frame}{洛朗级数的形式}
	\onslide<+->
	现在我们来证明洛朗级数的存在性并得到洛朗展开式.
	\onslide<+->
	设 $f(z)$ 在圆环域 $R_1<|z-z_0|<R_2$ 内解析.
	\onslide<+->
	设
	\[\alert{K_1:|z-z_0|=r},\quad \emph{K_2:|z-z_0|=R},\quad R_1<r<R<R_2.\]
	是该圆环域内的两个圆周. 
	\onslide<+->
	对于 $r<|z-z_0|<R$, 由柯西积分公式,
	\[f(z)=\frac1{2\pi i}
	\emph{\oint_{K_2}\frac{f(\zeta)}{\zeta-z}\diff\zeta}
	-\frac1{2\pi i}\alert{\oint_{K_1}\frac{f(\zeta)}{\zeta-z}\diff\zeta}.\]
	\onslide<2->
	\begin{center}
		\begin{tikzpicture}
			\filldraw[cstcurve,dcolorb,cstfill] (0,0) circle (1.5);
			\filldraw[cstcurve,dcolora,fill=white] (0,0) circle (1);
			\fill[cstdot,dcolora,visible on=<4->] (-1,0) circle;
			\fill[cstdot,dcolorb,visible on=<4->] (-1.48,-0.3) circle;
			\fill[cstdot,black,visible on=<4->] (0,1.25) circle;
			\draw[dcolorb,cstarrowto,thick,visible on=<3->] (0,0)--(1.2,0.9);
			\draw[dcolora,cstarrowto,thick,visible on=<3->] (0,0)--(0.6,-0.8);
			\draw
				(-1.25,-0.15) node[visible on=<4->] {$\zeta$}
				(0.5,0.1) node[dcolorb,visible on=<3->] {$R$}
				(0.1,-0.5) node[dcolora,visible on=<3->] {$r$}
				(-0.3,0) node {$z_0$}
				(0.8,-1) node[dcolora,visible on=<3->] {$K_1$}
				(1.4,1.2) node[dcolorb,visible on=<3->] {$K_2$}
				(-0.3,1.25) node[visible on=<4->] {$z$};
			\fill[cstdot,black] (0,0) circle;
		\end{tikzpicture}
	\end{center}
\end{frame}


\begin{frame}{洛朗级数}
	\onslide<+->
	和泰勒级数的推导类似,
	\[\frac1{2\pi i}\emph{\oint_{K_2}\frac{f(\zeta)}{\zeta-z}\diff\zeta}
	=\sum_{n=0}^\infty\left[\frac1{2\pi i}\oint_{K_2}\frac{f(\zeta)\diff\zeta}{(\zeta-z_0)^{n+1}}\right](z-z_0)^n\]
	可以表达为幂级数的形式.
	\onslide<+->
	对于 $\zeta\in K_1$, 由 $\abs{\dfrac{\zeta-z_0}{z-z_0}}<1$ 可得
	\onslide<+->
	\[-\frac1{\zeta-z}=\frac1{z-z_0}\cdot\frac1{1-\dfrac{\zeta-z_0}{z-z_0}}=\sum_{n=1}^\infty\frac{(z-z_0)^{-n}}{(\zeta-z_0)^{-n+1}},\]
	\onslide<+->
	\[-\frac1{2\pi i}\alert{\oint_{K_1}\frac{f(\zeta)}{\zeta-z}\diff\zeta}=\frac1{2\pi i}\oint_{K_1}f(\zeta)\sum_{n=1}^\infty\frac{(z-z_0)^{-n}}{(\zeta-z_0)^{-n+1}}\diff\zeta.\]
\end{frame}


\begin{frame}{洛朗级数的形式}
	\onslide<+->
	令
	\begin{align*}
		R_N(z)&=\frac1{2\pi i}\oint_{K_1}f(\zeta)\sum_{n=N}^\infty\frac{(z-z_0)^{-n}}{(\zeta-z_0)^{-n+1}}\diff\zeta\\
		&=\frac1{2\pi i}\oint_{K_1}\frac{f(\zeta)}{z-\zeta}\cdot\left(\frac{\zeta-z_0}{z-z_0}\right)^{N-1}\diff\zeta.
	\end{align*}
	\onslide<+->
	由于 $f(\zeta)$ 在 $D\supseteq K_1$ 上解析, 从而在 $K_1$ 上连续且有界.
	\onslide<+->
	设 $|f(\zeta)|\le M,\zeta\in K_1$,
	\onslide<+->
	那么
	\begin{align*}
	|R_N(z)|&\le\frac 1{2\pi}\oint_{K_1}\abs{\frac{f(\zeta)}{z-\zeta}\cdot\left(\frac{\zeta-z_0}{z-z_0}\right)^{N-1}}\diff s\\
	&\visible<+->{\le\frac 1{2\pi}\cdot\frac M{|z-z_0|-r}\cdot\abs{\frac{\zeta-z_0}{z-z_0}}^{N-1}\cdot 2\pi r\to 0\quad (N\to\infty).}
	\end{align*}
\end{frame}


\begin{frame}{洛朗级数的形式}
	\onslide<+->
	故
	\begin{align*}
	f(z)&=\sum_{n=0}^\infty\left[\frac1{2\pi i}\oint_{K_2}\frac{f(\zeta)\diff\zeta}{(\zeta-z_0)^{n+1}}\right](z-z_0)^n\\
	&\qquad+\sum_{n=1}^\infty \left[\frac1{2\pi i}\oint_{K_1}\frac{f(\zeta)\diff\zeta}{(\zeta-z_0)^{-n+1}}\right](z-z_0)^{-n},
	\end{align*}
	其中 $r<|z-z_0|<R$.
	\onslide<+->
	由复合闭路定理, $K_1,K_2$ 可以换成任意一条在圆环域内绕 $z_0$ 的闭路 $C$.
	\onslide<+->
	从而我们得到 \emph{$f(z)$ 在以 $z_0$ 为圆心的圆环域的洛朗展开}
	\[\alert{f(z)=\sum_{n=-\infty}^\infty\left[\emph{\frac1{2\pi i}\oint_C\frac{f(\zeta)\diff\zeta}{(\zeta-z_0)^{n+1}}}\right](z-z_0)^n},\]
	其中 $R_1<|z-z_0|<R_2$.
\end{frame}


\subsection{洛朗展开的计算方法}

\begin{frame}{洛朗展开的唯一性}
	\onslide<+->
	我们称 $f(z)$ 洛朗展开的非负幂次部分为它的\emph{解析部分}, 负幂次部分为它的\emph{主要部分}.

	\onslide<+->
	设在圆环域 $R_1<|z-z_0|<R_2$ 内的解析函数 $f(z)$ 可以表达为双边幂级数
	\[f(z)=\sum_{n=-\infty}^\infty c_n(z-z_0)^n,\]
	\onslide<+->
	则逐项积分得到
	\[\oint_C\frac{f(\zeta)\diff\zeta}{(\zeta-z_0)^{n+1}}=\sum_{k=-\infty}^\infty c_k\oint_C(\zeta-z_0)^{k-n-1}\diff\zeta=2\pi i c_n.\]
	\onslide<+->
	因此 $f(z)$ 在圆环域内的\alert{双边幂级数展开是唯一的, 它就是洛朗级数}.
\end{frame}


\begin{frame}{典型例题: 求洛朗级数}
	\onslide<+->
	\begin{example}
		将 $f(z)=\dfrac{e^z-1}{z^2}$ 展开为以 $0$ 为中心的洛朗级数.
	\end{example}

	\onslide<+->
	由洛朗级数的唯一性, 我们可以从 $e^z$ 的泰勒展开通过代数运算来得到洛朗级数.
	\onslide<+->
	这种做法比直接计算积分更简便.
	\onslide<+->
	因此我们一般\alert{不用直接法}, 而是\abox{用双边幂级数的} \abox{代数、求导、求积分运算}来得到洛朗级数.

	\onslide<+->
	\begin{solution}
		\[\frac{e^z-1}{z^2}=\frac1{z^2}\left(z+\frac{z^2}{2!}+\frac{z^3}{3!}+\cdots\right)
		\visible<+->{=\frac1z+\sum_{n=0}^\infty \frac1{(n+2)!}z^n,}\]
		\onslide<+->{其中 $0<|z|<+\infty$.}
	\end{solution}
\end{frame}


\begin{frame}{典型例题: 求洛朗展开}
	\onslide<+->
	\begin{example}
		在下列圆环域中把 $f(z)=\dfrac1{(z-1)(z-2)}$ 展开为洛朗级数.

		\onslide<+->{\enumnum1 $0<|z|<1$, \enumnum2 $1<|z|<2$, \enumnum3 $2<|z|<+\infty$.}
	\end{example}

	\onslide<+->
	\begin{solution}
		由于 $f(z)$ 的奇点为 $z=1,2$, 因此在这些圆环域内 $f(z)$ 都可以展开为洛朗级数.
		\onslide<+->{注意到
			\[f(z)=\frac1{z-2}-\frac1{z-1},\]
			因此我们可以根据 $|z|$ 的范围来将其展开成等比级数.}
	\end{solution}
\end{frame}


\begin{frame}{典型例题: 求洛朗展开}
	\onslide<+->
	\begin{solutionc}
		\enumnum1 由于 $|z|<1,\abs{\dfrac z2}<1$,
		\onslide<+->{因此
			\begin{align*}
				f(z)&=-\frac1{2-z}+\frac1{1-z}
				\visible<+->{=-\half\cdot\frac1{1-\dfrac z2}+\frac1{1-z}}\\
				&\visible<+->{=-\half\sum_{n=0}^\infty\left(\frac z2\right)^n+\sum_{n=0}^\infty z^n}
				\visible<+->{=\sum_{n=0}^\infty\left(1-\frac1{2^{n+1}}\right)z^n}\\
				&\visible<+->{=\half +\frac34z+\frac78z^2+\cdots}
			\end{align*}}
	\end{solutionc}
\end{frame}


\begin{frame}{典型例题: 求洛朗展开}
	\onslide<+->
	\begin{solutionc}
		\enumnum2 由于 $\abs{\dfrac1z}<1,\abs{\dfrac z2}<1$, 
		\onslide<+->{因此
			\begin{align*}
				f(z)&=\frac1{1-z}-\frac1{2-z}
				\visible<+->{=-\frac1z\cdot\frac1{1-\dfrac1z}-\half\cdot\frac1{1-\dfrac z2}}\\
				&\visible<+->{=-\frac1z\sum_{n=0}^\infty \left(\frac1z\right)^n-\half\sum_{n=0}^\infty\left(\frac z2\right)^n}
				\visible<+->{=-\sum_{n=1}^{\infty}\frac1{z^n}-\sum_{n=0}^\infty\frac1{2^{n+1}}z^n}\\
				&\visible<+->{=\cdots-\frac1{z^2}-\frac1z-\half -\frac14z-\frac18z^2-\cdots}
			\end{align*}}
	\end{solutionc}
\end{frame}


\begin{frame}{典型例题: 求洛朗展开}
	\onslide<+->
	\begin{solutionc}
		\enumnum3 由于 $\abs{\dfrac1z}<1,\abs{\dfrac2z}<1$, \onslide<+->{因此
			\begin{align*}
				f(z)&=\frac1{1-z}-\frac1{2-z}
					\visible<+->{=-\frac1z\cdot\frac1{1-\dfrac1z}+\frac1z\cdot\frac1{1-\dfrac2z}}\\
				&\visible<+->{=-\frac1z\sum_{n=0}^\infty \left(\frac1z\right)^n+\frac1z\sum_{n=0}^\infty\left(\frac2z\right)^n}
					\visible<+->{=\sum_{n=0}^\infty\frac{2^n-1}{z^{n+1}}}\\
				&\visible<+->{=\frac1{z^2}+\frac3{z^3}+\frac7{z^4}+\cdots}
			\end{align*}}
	\end{solutionc}
\end{frame}


\begin{frame}{洛朗级数的特点\noexer}
	洛朗展开的一些特点可以帮助我们检验计算的正确性.
	\begin{itemize}
		\item  若 $f(z)$ 在 $|z-z_0|<R_2$ 内\emph{解析},
		\onslide<+->
		则 $f(z)$ 可以展开为泰勒级数.
		\onslide<+->
		由唯一性可知泰勒级数等于洛朗级数,
		\onslide<+->
		因此\emph{此时洛朗展开一定没有负幂次项}.
		\item 若 $f(z)$ 在圆周 $|z-z_0|=R_1,R_2>0$ 上\emph{有奇点}, 则在圆环域 $R_1<|z-z_0|<R_2$ 上的洛朗展开\emph{一定有无穷多负幂次和无穷多正幂次项}.
		\item 有理函数在 $0<|z-z_0|<r$ 洛朗展开\emph{最多只有有限多负幂次项}, 在 $R<|z-z_0|<+\infty$ 洛朗展开\emph{最多只有有限多正幂次项}.
	\end{itemize}
\end{frame}


\begin{frame}{洛朗级数的特点\noexer}
	\onslide<+->
	如果有理函数 $f(z)$ 在圆环域 $r<|z|<R$ 内解析, 那么它的洛朗展开一定形如
	\[f(z)=P(z)+\sum_{n\ge 0}a_n z^n+\sum_{n<0}b_n z^n,\]
	其中 $P(z)$ 只有有限多项, 
	\onslide<+->
	$a_n$ 是形如 $p(n)\lambda^{-n}$ 的线性组合, $|\lambda|\ge R$ 是奇点, $\deg p+1$ 是 $\lambda$ 在 $f(z)$ 出现的重数; 而 $b_n$ 则是 $|\lambda|\le r$ 的那个奇点对应的组合.

	\onslide<+->
	不仅如此, 在不同的圆环域不同圆环域上的洛朗展开{\itshape 形式地相减}, 系数会有\emph{共同的通项形式}.
\end{frame}


\begin{frame}{洛朗级数的特点\noexer}
	\onslide<+->
	例如在 $0<|z|<1$ 内,
	\[f(z)=\frac{120}{(z-1)(z^2-4)(z^2-9)}=\sum_{n\ge 0}\left(-5+\frac2{(-2)^{n+1}}+\frac6{2^{n+1}}-\frac1{(-3)^{n+1}}-\frac2{3^{n+1}}\right)z^n.\]
	\onslide<+->
	那么在 $2<|z|<3$ 内洛朗展开的系数就是上面的每个系数(不论正负 $n$)减去一个共同形式的项, 使得其非负幂次通项对应那些奇点 $|\lambda|\ge 3$, 而负幂次通项对应那些奇点 $|\lambda|\le2$.
	\onslide<+->
	故
	\[f(z)=\sum_{n\ge 0}\left(-\frac1{(-3)^{n+1}}-\frac2{3^{n+1}}\right)z^n+\sum_{n\le-1}\left(5-\frac2{(-2)^{n+1}}-\frac6{2^{n+1}}\right)z^n.\]
	\onslide<+->
	其证明可见\alert{\href{https://zhangshenxing.gitee.io/teaching/publications/袁志杰张神星2023 复变函数在不同圆环域内洛朗展开的联系.pdf}{复变函数在不同圆环域内洛朗展开的联系}}一文.
\end{frame}


\begin{frame}{典型例题: 求洛朗展开}
	\onslide<+->
	\begin{example}
		(2022年A卷) 将函数 $f(z)=\dfrac{z+1}{(z-1)^2}$ 在圆环域 $0<|z|<1$ 内展开成洛朗级数.
	\end{example}

	\onslide<+->
	\begin{solution}
		\[f(z)=\frac{z-1+2}{(z-1)^2}=\frac1{z-1}+\frac{2}{(z-1)^2}=-\frac1{1-z}+2\left(\frac1{1-z}\right)'\]
		\onslide<+->{因此当 $0<|z|<1$ 时,
			\[f(z)=-\sum_{n=0}^\infty z^n+2\left(\sum_{n=0}^\infty z^n\right)'=-\sum_{n=0}^\infty z^n+2\sum_{n=1}^\infty nz^{n-1}
			=\sum_{n=0}^\infty(2n+1)z^n.\]}
	\end{solution}
\end{frame}


\begin{frame}{典型例题: 求洛朗展开}
	\onslide<+->
	\begin{exercise}
		(2022年A卷) 将函数 $f(z)=\dfrac{z+1}{(z-1)^2}$ 在圆环域 $1<|z|<+\infty$ 内展开成洛朗级数.
	\end{exercise}

	\onslide<+->
	\begin{answer}
		\[f(z)=\sum_{n=1}^\infty \frac{2n-1}{z^n}.\]
	\end{answer}
	\onslide<+->
	利用幂函数的泰勒展开可以得到
	\[\frac1{(1-\lambda z)^k}=-\sum_{n<0}(n+k-1)\cdots(n+2)(n+1)z^n,\ |z|>\frac1{|\lambda|}.\]
	\onslide<+->
	实际上求和范围可以改成 $n\le -k$.
\end{frame}


\begin{frame}{例: 洛朗展开的应用}
	\onslide<+->
	注意到当 $n=-1$ 时, 洛朗级数的系数
	\[c_{-1}=\frac1{2\pi i}\oint_C f(\zeta)\diff\zeta,\]
	\onslide<+->
	因此洛朗展开可以用来帮助计算函数的积分,
	\onslide<+->
	这便引出了\alert{留数}的概念.
\end{frame}


\begin{frame}{例: 洛朗展开的应用}
	\beqskip{3pt}
	\onslide<+->
	\begin{example}
		求 $\displaystyle\oint_{|z|=3}\frac1{z(z+1)^2}\diff z$.
	\end{example}

	\onslide<+->
	\begin{solution}
		注意到闭路 $|z|=3$ 落在 $1<|z+1|<+\infty$ 内.
		\onslide<+->{我们在这个圆环域内求 $f(z)=\dfrac1{z(z+1)^2}$ 的洛朗展开.
		}\onslide<+->{
			\begin{align*}
			f(z)&=\frac1{z(z+1)^2}=\frac1{(z+1)^3}\cdot\frac1{1-\dfrac1{z+1}}\\
			&\visible<+->{=\frac1{(z+1)^3}\sum_{n=0}^\infty\frac1{(z+1)^n}}
			\visible<+->{=\sum_{n=0}^{\infty}\frac1{(z+1)^{n+3}}}
			\end{align*}
		}\onslide<+->{故
			$\displaystyle\oint_Cf(z)\diff z=2\pi i c_{-1}=0$.}
	\end{solution}
	% \vspace{-.\baselineskip}
	\endgroup
\end{frame}



\begin{frame}{例: 洛朗展开的应用}
	\onslide<+->
	\begin{example}
		求 $\displaystyle\oint_{|z|=1}\frac{z}{\sin z^2}\diff z$.
	\end{example}

	\onslide<+->
	\begin{solution}
		注意到闭路 $|z|=1$ 落在 $0<|z|<\sqrt \pi$ 内.
		\onslide<+->{我们在这个圆环域内求 $f(z)=\dfrac{z}{\sin z^2}$ 的洛朗展开.
		}\onslide<+->{
			\[
			f(z)=\frac{z}{\sin z^2}=
			\frac{z}{z^2-\dfrac{z^6}{3!}+\dfrac{z^{10}}{5!}+\cdots}
			\visible<+->{=\frac1z+\frac{z^3}6+\cdots}
			\]
		}\onslide<+->{故
			$\displaystyle\oint_Cf(z)\diff z=2\pi i c_{-1}=2\pi i$.}
	\end{solution}
\end{frame}
