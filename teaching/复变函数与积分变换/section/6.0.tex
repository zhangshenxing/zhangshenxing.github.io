
\subsection{积分变换的引入}

\begin{frame}{积分变换的引入}
	\onslide<+->
	在学习指数和对数的时候, 我们了解到利用对数可以将乘除、幂次转化为加减、乘除.
	\onslide<+->
	\begin{example}
		计算 $12345\times 67890$.
	\end{example}
	\onslide<+->
	\begin{solution}
		通过查对数表得到
	\[
			\ln 12345\approx 9.4210,\qquad\ln 67890\approx 11.1256.
	\]
		\onslide<+->{%
		将二者相加并通过反查对数表得到原值
	\[
			12345\times 67890\approx \exp(20.5466)\approx 8.3806\times 10^8.
	\]
		}\bigdel
	\end{solution}
\end{frame}


\begin{frame}{积分变换的引入}
	\onslide<+->
	而对于函数而言, 我们常常要解函数的方程.
	\onslide<+->
	\begin{example}
		解微分方程
	\[
		\begin{cases}
			y''+y=t,&\\
			y(0)=y'(0)=0.&
		\end{cases}
	\]
	\end{example}
	\onslide<+->
	\begin{solution}[indent]
			我们希望能找到一种函数的\emph{变换 $\msl$}, 使得它可以把函数的微分和积分变成代数运算, 计算之后通过\emph{反变换 $\msl^{-1}$} 求得原来的解.

		\onslide<+->{这个变换最常见的就是我们将要介绍的\emph{傅里叶变换}和\emph{拉普拉斯变换}.
		}
	\end{solution}
\end{frame}

