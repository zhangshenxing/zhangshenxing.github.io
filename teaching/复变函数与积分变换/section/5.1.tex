\section{留数}

\subsection{留数定理}

\begin{frame}{留数}
	\onslide<+->
	\begin{definition}
		设 $z_0$ 为 $f(z)$ 的孤立奇点, $f(z)$ 在它的某个去心邻域内的洛朗展开为
		\[
			f(z)=\cdots+\frac{c_{-1}}{z-z_0}+c_0+c_1(z-z_0)+\cdots.
		\]
		\onslide<+->{%
			称
			\[
				\alertm{\Res[f(z),z_0]:=c_{-1}=\frac1{2\pi\ii}\oint_Cf(z)\d z}
			\]
			为函数 \emph{$f(z)$ 在 $z_0$ 的留数}, 其中 $C$ 为该去心邻域中绕 $z_0$ 的一条闭路.
		}
	\end{definition}
	\onslide<+->
	可以看出, 知道留数之后可以用来计算积分.
\end{frame}


\begin{frame}{留数定理}
	\onslide<+->
	\begin{theorem*}[nearnext][复变函数积分计算方法 III: 留数定理]
		若 $f(z)$ 在闭路 $C$ 上解析, 在 $C$ 内部的奇点为 $z_1,z_2,\dots,z_n$, 则
		\[
			\oint_Cf(z)\d z=2\pi\ii\sum_{k=1}^n\Res[f(z),z_k].
		\]
	\end{theorem*}
	\onslide<+->
	\begin{proof}[nearprev,sidepic,righthand width=5cm]
		由复合闭路定理,
		\begin{align*}
			\doint_Cf(z)\d z&
			=\sum_{k=1}^n\oint_{C_k}f(z)\d z\\&
			=2\pi\ii\sum_{k=1}^n\Res[f(z),z_k].\qedhere
		\end{align*}
		\tcblower
		\begin{center}
			\begin{tikzpicture}
				\begin{scope}[scale=.7]
					\def\r{.6}
					\filldraw [
						cstcurve,
						main,
						cstfill1,
						decoration = {
							markings,
							mark = at position .28 with {
								\arrow{Straight Barb}
								\node[above]{$C$};
							}
						},
						postaction={decorate},
						domain=0:360,
						samples=500,
					] plot ({3*cos(\x)+.2*cos(2*\x)-.3*cos(3*\x)-.2}, {1.8*sin(\x)+.2*sin(2*\x)});
					\coordinate (Z1) at (-1.7,0.2);
					\coordinate (Z2) at (0,-.5);
					\coordinate (Z3) at (1.7,.3);
					\begin{scope}[cstcurve,second]
						\draw[
							decoration = {
								markings,
								mark = at position .15 with {
									\arrow[rotate=-7]{Straight Barb}
									\node[above right,shift={(-.1,-.1)}]{$C_1$};
								}
							},
							postaction={decorate}
						] (Z1) circle (\r);
						\draw[
							decoration = {
								markings,
								mark = at position .75 with {
									\arrow[rotate=-7]{Straight Barb}
									\node[below right,shift={(0,.1)}]{$C_2$};
								}
							},
							postaction={decorate}
						] (Z2) circle (\r);
						\draw[
							decoration = {
								markings,
								mark = at position .25 with {
									\arrow[rotate=-7]{Straight Barb}
									\node[above left]{$C_3$};
								}
							},
							postaction={decorate}
						] (Z3) circle (\r);
					\end{scope}
					\begin{scope}[third]
						\draw (Z1) node[below] {$z_1$};
						\draw (Z2) node[below] {$z_2$};
						\draw (Z3) node[below] {$z_3$};
					\end{scope}
				\end{scope}
				\begin{scope}[cstdot,third]
					\fill (Z1) circle;
					\fill (Z2) circle;
					\fill (Z3) circle;
				\end{scope}
			\end{tikzpicture}
		\end{center}
	\end{proof}
\end{frame}


\subsection{留数的计算方法}

\begin{frame}{可去奇点的留数}
	\onslide<+->
	若 $z_0$ 为 $f(z)$ 的可去奇点, 则显然 $\Res[f(z),z_0]=0$.
	\onslide<+->
	\begin{example}
		$f(z)=\dfrac{z^3(\ee^z-1)^2}{\sin z^4}$.
		\onslide<+->{%
			由于 $0$ 是分子的五阶零点, 分母的四阶零点, 因此 $z=0$ 是 $f(z)$ 的可去奇点.
		}\onslide<+->{%
			故
			\[
				\Res[f(z),0]=0.
			\]
		}
	\end{example}
\end{frame}


\begin{frame}{本性奇点的留数}
	\onslide<+->
	若 $z_0$ 为 $f(z)$ 的本性奇点, 一般只能从定义计算.
	\onslide<+->
	\begin{example}
		$f(z)=z^4\sin\dfrac1z$.
		\onslide<+->{%
			由于
			\[
				f(z)=z^4\sumf0(-1)^n\frac{z^{-2n-1}}{(2n+1)!}
				=z^3-\frac z{3!}+\frac1{5!z}+\cdots
			\]
		}\onslide<+->{%
			因此
			\[
				\Res[f(z),0]=\frac1{120}.
			\]
		}
	\end{example}
\end{frame}


\begin{frame}{极点的留数计算方法}
	\onslide<+->
	设 $z_0$ 为 $f(z)$ 的极点.
	\onslide<+->
	\begin{theorem*}[][留数计算公式 I]
		若 $z_0$ 是 $\le m$ 阶极点或可去奇点, 那么
		\[
			\Res[f(z),z_0]=\frac1{(m-1)!}\lim_{z\to z_0}\bigl[(z-z_0)^mf(z)\bigr]^{(m-1)}.
		\]
	\end{theorem*}
	\onslide<+->
	\begin{theorem*}[][留数计算公式 II]
		若 $z_0$ 是一阶极点或可去奇点, 那么
		\[
			\Res[f(z),z_0]=\liml_{z\to z_0}(z-z_0)f(z).
		\]
	\end{theorem*}
\end{frame}


\begin{frame}{极点的留数计算方法}
	\onslide<+->
	\begin{proof}
		设
		\begin{align*}
			f(z)&=c_{-m}(z-z_0)^{-m}+\cdots+c_{-1}(z-z_0)^{-1}+c_0+\cdots,\\
			g(z)&=c_{-m}+\cdots+c_{-1}(z-z_0)^{m-1}+c_0(z-z_0)^m+\cdots,
		\end{align*}
		\onslide<+->{%
			则 $g(z)=(z-z_0)^mf(z)$.
		}%
		\onslide<+->{%
			由泰勒展开系数与函数导数的关系可知
			\[
				\Res[f(z),z_0]=c_{-1}=\dfrac1{(m-1)!}g^{(m-1)}(z_0).\qedhere
			\]
		}
	\end{proof}
\end{frame}


\begin{frame}{典型例题: 留数的计算}
	\onslide<+->
	\begin{example}[nearnext]
		求 $\Res\left[\dfrac{\ee^z}{z^n},0\right]$.
	\end{example}
	\onslide<+->
	\begin{solution}[nearprev]
		显然 $0$ 是 $n$ 阶极点,
		\onslide<+->{%
			\begin{align*}
				\Res\left[\frac{\ee^z}{z^n},0\right]&=\frac1{(n-1)!}\lim_{z\to0}(\ee^z)^{(n-1)}\\
				&\visible<+->{=\frac1{(n-1)!}\lim_{z\to0}\ee^z=\frac1{(n-1)!}.}
			\end{align*}
		}
	\end{solution}
\end{frame}


\begin{frame}{典型例题: 留数的计算}
	\onslide<+->
	\begin{example}[near]
		求 $\Res\left[\dfrac{z-\sin z}{z^6},0\right]$.
	\end{example}
	\onslide<+->
	\begin{solution}[near]
		因为 $z=0$ 是 $z-\sin z$ 的三阶零点,
		\onslide<+->{%
			所以是 $\dfrac{z-\sin z}{z^6}$ 的三阶极点.
		}\onslide<+->{%
			若用公式
			\[
				\Res\left[\frac{z-\sin z}{z^6},0\right]
				=\frac1{2!}\lim_{z\to0}\Bigl(\frac{z-\sin z}{z^3}\Bigr)''
			\]
			计算会很繁琐.\meddel
		}\onslide<+->{%
			\[
				\Res\left[\frac{z-\sin z}{z^6},0\right]=\frac1{5!}\lim_{z\to0}(z-\sin z)^{(5)}
				\visible<+->{=\frac1{5!}\lim_{z\to0}(-\cos z)=-\frac1{120}.}
			\]
		}\bigdel\meddel
	\end{solution}
	\onslide<+->
	\begin{exercise}[near]
			求 $\Res\left[\dfrac{\ee^z-1}{z^5},0\right]=$\fillblankframe{$1/24$}.
	\end{exercise}
\end{frame}


\begin{frame}{留数的计算方法}
	\onslide<+->
	\begin{theorem*}[nearnext][留数计算公式 III]
		设 $P(z),Q(z)$ 在 $z_0$ 解析且 $z_0$ 是 $Q$ 的一阶零点, 则
		\[
			\Res\left[\frac{P(z)}{Q(z)},z_0\right]=\frac{P(z_0)}{Q'(z_0)}.
		\]
	\end{theorem*}
	\onslide<+->
	\begin{proof}[nearprev]
		不难看出 $z_0$ 是 $f(z)=\dfrac{P(z)}{Q(z)}$ 的一阶极点或可去奇点.
		\onslide<+->{%
			因此
			\begin{align*}
				&\Res[f(z),z_0]=\lim_{z\to z_0}(z-z_0)f(z)\\
				\visible<+->{=}&\visible<.->{\lim_{z\to z_0}\frac{P(z)}{\dfrac{Q(z)-Q(z_0)}{z-z_0}}}
				\visible<+->{=\frac{P(z_0)}{\liml_{z\to z_0}\dfrac{Q(z)-Q(z_0)}{z-z_0}}=\frac{P(z_0)}{Q'(z_0)}.\qedhere}
			\end{align*}
		}\bigdel
	\end{proof}
\end{frame}


\begin{frame}{典型例题: 留数的计算}
	\onslide<+->
	\begin{example}[nearnext]
		求 $\Res\left[\dfrac{z}{z^8-1},\dfrac{1+\ii}{\sqrt2}\right]$.
	\end{example}
	\onslide<+->
	\begin{solution}[nearprev]
		由于 $z=\dfrac{1+\ii}{\sqrt2}$ 是分母的 $1$ 阶零点,
		\onslide<+->{%
			因此
			\[
				\Res\left[\frac z{z^8-1},\frac{1+\ii}{\sqrt2}\right]
				=\frac z{(z^8-1)'}\Big|_{z=\frac{1+\ii}{\sqrt2}}
				=\frac z{8z^7}\Big|_{z=\frac{1+\ii}{\sqrt2}}
				=\frac \ii8.
			\]
		}
	\end{solution}
\end{frame}


\begin{frame}{例题: 留数的应用}
	\onslide<+->
	\begin{example}[nearnext]
		计算积分 $\doint_{\abs{z}=2}\frac{\ee^z}{z(z-1)^2}\d z$.
	\end{example}
	\onslide<+->
	\begin{solution}[nearprev]
		$f(z)=\dfrac{\ee^z}{z(z-1)^2}$ 在 $\abs{z}<2$ 内有奇点 $z=0,1$.
		\onslide<+->{%
			\begin{align*}
				\Res[f(z),0]&=\lim_{z\to0}\frac{\ee^z}{(z-1)^2}=1,\\
				\visible<+->{\Res[f(z),1]}&\visible<.->{=\lim_{z\to1}\Bigl(\frac{\ee^z}z\Bigr)'=\lim_{z\to1}\frac{\ee^z(z-1)}{z^2}=0,}
			\end{align*}
		}\onslide<+->{%
			\[
				\oint_{\abs{z}=2}\frac{\ee^z}{z(z-1)^2}\d z
				=2\pi\ii\bigl[\Res[f(z),0]+\Res[f(z),1]\bigr]
				=2\pi\ii.
			\]
		}\bigdel
	\end{solution}
\end{frame}


\subsection{在 \texorpdfstring{$\infty$}{∞} 的留数}

\begin{frame}{在 $\infty$ 的留数}
	\onslide<+->
	\begin{definition}
		设 $f(z)$ 在某个圆环域 $R<\abs{z}<+\infty$ 内解析, 且洛朗展开为
		\[
			f(z)=\cdots+c_{-1}z^{-1}+c_0+c_1z+\cdots
		\]
		称
		\[
			\emphm{\Res[f(z),\infty]:=-c_{-1}=\frac1{2\pi\ii}\oint_{C^-}f(z)\d z}
		\]
		为函数 \emph{$f(z)$ 在 $\infty$ 的留数}, 其中 $C$ 为该圆环域中绕 $0$ 的一条闭路.
	\end{definition}
\end{frame}


\begin{frame}{在 $\infty$ 的留数计算公式}
	\onslide<+->
	由于
	\[
		f\Bigl(\frac1z\Bigr)\frac1{z^2}=\cdots+\frac{c_1}{z^3}+\frac{c_0}{z^2}+\frac{c_{-1}}z+c_{-2}+\cdots
	\]
	\onslide<+->
	因此有
	\begin{theorem*}[][留数计算公式 IV]
		\[
			\Res[f(z),\infty]=-\Res\left[f\Bigl(\frac1z\Bigr)\frac1{z^2},0\right].
		\]
	\end{theorem*}
\end{frame}


\begin{frame}{留数之和为 $0$}
	\onslide<+->
	\begin{theorem}[nearnext]
		若 $f(z)$ 只有有限个奇点, 那么 $f(z)$ 在扩充复平面内各奇点处的留数之和为 $0$.
	\end{theorem}
	\onslide<+->
	\begin{proof}[nearprev]
		设闭路 $C$ 内部包含除 $\infty$ 外所有奇点 $z_1,\dots,z_n$.
		\onslide<+->{%
			由留数定理
			\[
				-\Res[f(z),\infty]
				=\frac1{2\pi\ii}\oint_C f(z)\d z
				=\sum_{k=1}^n\Res[f(z),z_k].
			\]
		}\onslide<+->{%
			故 $\suml_{k=1}^n\Res[f(z),z_k]+\Res[f(z),\infty]=0$.\qedhere
		}
	\end{proof}
\end{frame}


\begin{frame}{例题: 留数的应用}
	\onslide<+->
	\begin{example}
		求 $\doint_{\abs{z}=2}f(z)\d z$, 其中 $f(z)=\dfrac{\sin(1/z)}{(z+\ii)^{10}(z-1)(z-3)}$.
	\end{example}
	\onslide<+->
	$f(z)$ 在 $\abs{z}<2$ 内有奇点 $1,-\ii ,0$,
	\onslide<+->
	其中 $0$ 是本性奇点, 它的留数不易求得.
	而 $-\ii $ 是 $10$ 阶极点, 它的留数也难以计算.
	\onslide<+->
	因此我们将问题转化为计算闭路 $C$ 外部的奇点处留数.
	\onslide<+->
	\begin{solution}
		$f(z)$ 在 $\abs{z}>2$ 内只有奇点 $3,\infty$.
		\onslide<+->{%
			\[
				\Res[f(z),3]=\lim_{z\to3}(z-3)f(z)=\frac1{2(3+\ii)^{10}}\sin\frac13.
			\]
		}\bigdel
	\end{solution}
\end{frame}


\begin{frame}{例题: 留数的应用}
	\onslide<+->
	\begin{solution}[][]
		\begin{align*}
			\Res[f(z),\infty]&=-\Res\left[f\Bigl(\frac1z\Bigr)\frac1{z^2},0\right]\\
			&=-\Res\left[\frac{z^{10}\sin z}{(1+\ii z)^{10}(1-z)(1-3z)},0\right]=0.
		\end{align*}
		\onslide<+->{%
			\begin{align*}
				\oint_{\abs{z}=2}f(z)\d z
				&=2\pi\ii\bigl[\Res[f(z),-\ii ]+\Res[f(z),1]+\Res[f(z),0]\bigr]\\
				&\visible<+->{=-2\pi\ii\bigl[\Res[f(z),3]+\Res[f(z),\infty]\bigr]=-\frac{\pi\ii}{(3+\ii)^{10}}\sin\frac13.}
			\end{align*}
		}
	\end{solution}
\end{frame}


\begin{frame}{积分的计算方法汇总}
\begin{center}
\begin{tikzpicture}[node distance=25pt]
	\node at (1.6,6.3) [cstnode1] (integral) {定积分 $\dint_Cf(z)\d z$ 的计算};
	\node at (-2.5,4.9) [cstnode2,visible on=<2->,text width=1.7cm] (trans) {若 $f(z)$ 包含 $\abs{z},\ov z$, 可先尝试通过 $C$ 的方程将这些换成 $z$ 的表达式};
	\node at (0,4.5) [cstnode,visible on=<3->] (isclosed) {是闭路};
	\node at (4.17,0.5) [cstnode2,visible on=<4->] (end1) {求出 $f(z)$ 的原函数 $F(z)$ 得到 $\dint_Cf(z)\d z=F(b)-F(a)$};
	\node at (0,2.2) [cstnode,visible on=<4->] (isanalytic) {$f(z)$ 解析};
	\node at (4.5,2.2) [cstnode2,align=center, visible on=<5->] (end2) {设曲线方程为 $z(t)$, 则\\
	积分$\displaystyle=\int_a^bf(z)z'(t)\d t$\\
	可能需要分段计算};
	\node at (4.5,4.5) [cstnode,visible on=<6->] (issingle) {内(外)部只有孤立奇点};
	\node at (9,4.5) [cstnode,align=center, visible on=<8->] (ismany) {闭路内部和外部\\孤立奇点数量};
	\node at (9,6.3) [cstnode2,align=center, visible on=<9->] (end3) {闭路内奇点留数之和\\并乘 $2\pi\ii$};
	\node at (9,2.2) [cstnode2,align=center, visible on=<10->] (end4) {闭路外奇点留数之和\\并乘 $-2\pi\ii$};

	\draw[cstmra,fourth,visible on=<2->] (integral.175) -- (trans.58);
	\draw[cstmra,fourth,visible on=<2->] (trans.51)--(integral.185);
	\draw[cstmra,fourth,visible on=<3->] (integral.-160) -- (isclosed);
	\draw[cstmra,fourth,visible on=<4->] (isanalytic) -- node[left]{是} (end1.172);
	\draw[cstmra,fourth,visible on=<6->] (isclosed) -- node[above]{是} (issingle);
	\draw[cstmra,fourth,visible on=<8->] (issingle) -- node[above]{是} (ismany);
	\draw[cstmra,fourth,visible on=<4->] (isclosed) -- node[left]{否} (isanalytic);
	\draw[cstmra,fourth,visible on=<7->] (issingle) -- node[left]{否} (end2);
	\draw[cstmra,fourth,visible on=<5->] (isanalytic) -- node[above]{否} (end2);
	\draw[cstmra,fourth,visible on=<9->] (ismany) -- node[left]{闭路内易计算} (end3);
	\draw[cstmra,fourth,visible on=<10->] (ismany) -- node[left]{闭路外易计算} (end4);
\end{tikzpicture}
\end{center}
\end{frame}


% \begin{frame}{例题: 留数在有理函数分解中的应用}
% 	\onslide<+->
% 	在求有理函数的洛朗展开, 以及之后在求有理函数的拉普拉斯逆变换时, 我们需要将一个有理函数表达为分母只有一个零点的有理函数之和.
% 	\onslide<+->
% 	例如:
% 	\[\frac{z-3}{(z+1)(z-1)^2}=\frac1{z-1}-\frac1{(z-1)^2}-\frac1{z+1}.
% 	\]
% 	\onslide<+->
% 	我们将之前所使用的利用柯西积分公式的方法翻译为留数的语言.
% \end{frame}


% \begin{frame}{例题: 留数在有理函数分解中的应用}
% 	\onslide<+->
% 	\begin{solution}
% 			设 $\displaystyle f(z)=\frac{z-3}{(z+1)(z-1)^2}=\frac a{z-1}+\frac b{(z-1)^2}+\frac c{z+1}$,
% 		\onslide<+->{则
% 			\begin{align*}
% 				a&=\Res[f(z),1]=\Bigl(\frac{z-3}{z+1}\Bigr)'\Big|_{z=1}=\frac 4{(z+1)^2}\Big|_{z=1}=1,\\
% 				b&=\Res[(z-1)f(z),1]=\frac{z-3}{z+1}\Big|_{z=1}=-1,\\
% 				c&=\Res[f(z),-1]=\frac{z-3}{(z-1)^2}\Big|_{z=-1}=-1.
% 			\end{align*}
% 		}\onslide<+->{故 $\displaystyle f(z)=\frac1{z-1}-\frac1{(z-1)^2}-\frac1{z+1}$.
% 		}
% 	\end{solution}
% \end{frame}

