\section{解析函数与调和函数的关系}


\subsection{调和函数}

\begin{frame}{调和函数}
	\onslide<+->
	调和函数是一类重要的二元实变函数, 它和解析函数有着紧密的联系.
	\onslide<+->
	为了简便, 我们用 $u_{xx},u_{yy}$ 来表示二阶偏导数.
	\onslide<+->
	\begin{definition}
		若二元实变函数 $u(x,y)$ 在区域 $D$ 内有二阶连续偏导数, 且满足\emph{拉普拉斯方程}
		\[
			\alertn{\delt u:=u_{xx}+u_{yy}=0},
		\]
		则称 $u(x,y)$ 是 $D$ 内的\emph{调和函数}.
	\end{definition}
\end{frame}


\begin{frame}{解析函数与调和函数的联系}
	\onslide<+->
	\begin{theorem}[nearnext]
		区域 $D$ 内解析函数 $f(z)$ 的实部和虚部都是调和函数.
	\end{theorem}
	\onslide<+->
	\begin{proof}[nearprev]
		设 $f(z)=u(x,y)+\ii v(x,y)$, 则 $u,v$ 存在偏导数且
		\[
			f'(z)=u_x+\ii v_x=v_y-\ii u_y.
		\]
		\onslide<+->{%
			由于 $f(z)$ 任意阶可导, 因此 $u,v$ 存在任意阶偏导数.
		}\onslide<+->{%
			由C-R方程 $u_x=v_y,u_y=-v_x$
		}\onslide<+->{%
			可知
			\[
				\delt u=u_{xx}+u_{yy}=v_{yx}-v_{xy}=0,
			\]\bigdel
		}\onslide<+->{%
			\[
				\delt v=v_{xx}+v_{yy}=-u_{yx}+u_{xy}=0.\qedhere
			\]
		}\bigdel
	\end{proof}
\end{frame}


\subsection{共轭调和函数}
\begin{frame}{解析函数与调和函数的联系}
	\onslide<+->
	反过来, 调和函数是否一定是某个解析函数的实部或虚部呢?
	\onslide<+->
	对于单连通的情形, 答案是肯定的.

	\onslide<+->
	若 $u+\ii v$ 是区域 $D$ 内的解析函数, 则我们称 $v$ 是 $u$ 的\emph{共轭调和函数}.
	\onslide<+->
	换言之 $u_x=v_y,u_y=-v_x$.
	\onslide<+->
	显然 $-u$ 是 $v$ 的共轭调和函数.
	\onslide<+->
	\begin{theorem}
		设 $u(x,y)$ 是单连通区域 $D$ 内的调和函数, 则线积分
	\[
		v(x,y)=\int_{(x_0,y_0)}^{(x,y)}-u_y\d x+u_x\d y+C
	\]
		是 $u$ 的共轭调和函数.
	\end{theorem}
	\onslide<+->
	由此可知, 区域 $D$ 上的调和函数在 $z\in D$ 的一个邻域内是一解析函数的实部, 从而在该邻域内具有任意阶连续偏导数.
	\onslide<+->
	而 $z$ 的任取的, 因此\alert{调和函数总具有任意阶连续偏导数}.
\end{frame}


\begin{frame}{共轭调和函数的求法}
	\onslide<+->
	若 $D$ 是多连通区域, 则未必存在共轭调和函数.
	\onslide<+->
	例如 $\ln(x^2+y^2)$ 是复平面去掉原点上的调和函数, 但它并不是某个解析函数的实部.
	\onslide<+->
	事实上, 它是 $2\Ln z$ 的实部.

	\onslide<+->
	在实际计算中, 我们\emph{一般不用线积分}来得到共轭调和函数, 而是采用下述两种办法:
	\onslide<+->
	\begin{theorem*}[][偏积分法]
		通过 $v_y=u_x$ 解得 $v=\varphi(x,y)+\psi(x)$, 其中 $\psi(x)$ 待定.
		\onslide<+->{再代入 $u_y=-v_x$ 中解出 $\psi(x)$.}
	\end{theorem*}
	\onslide<+->
	\begin{theorem*}[][不定积分法]
		对 $f'(z)=u_x-\ii u_y=v_y+\ii v_x$ 求不定积分得到 $f(z)$.
	\end{theorem*}
\end{frame}


\begin{frame}{典型例题: 求共轭调和函数和相应的解析函数}
	\onslide<+->
	\begin{example}[nearnext]
		证明 $u(x,y)=y^3-3x^2y$ 是调和函数, 并求其共轭调和函数以及由它们构成的解析函数.
	\end{example}
	\onslide<+->
	\begin{solution}[nearprev,indent]
		由 $u_x=-6xy,u_y=3y^2-3x^2$ 可知
		\[
			u_{xx}+u_{yy}=-6y+6y=0,
		\]
		\onslide<+->{%
			故 $u$ 是调和函数.
		}
		
		\onslide<+->{%
			由 $v_y=u_x=-6xy$ 得 $v=-3xy^2+\psi(x)$.
		}
		
		\onslide<+->{%
			由 $v_x=-u_y=3x^2-3y^2$ 得 $\psi'(x)=3x^2$,
		}\onslide<+->{%
			$\psi(x)=x^3+C$.
		}
		
		\onslide<+->{%
			故 $v(x,y)=-3xy^2+x^3+C$,
		}\onslide<+->{%
			\[
				f(z)=u+\ii v=y^3-3x^2y+\ii(-3xy^2+x^3+C)
				\visible<+->{=\ii(x+\ii y)^3+\ii C=\ii(z^3+C), C\in\BR.}
			\]
		}\bigdel
	\end{solution}
\end{frame}


\begin{frame}{典型例题: 求共轭调和函数和相应的解析函数}
	\onslide<+->
  当解析函数 $f(z)$ 是 $x,y$ 的多项式形式时, $f(z)$ 本身一定是 $z$ 的多项式.
	\onslide<+->
  于是将 $x$ 全部换成 $z$, $y$ 全部换成 $0$ 就是 $f(z)$.
	\onslide<+->
  即使 $f(z)$ 不是 $x,y$ 的多项式形式, 这种替换方式很多时候也奏效.

	\onslide<+->
	在上例中我们也可由另一种方法计算得到:
	\onslide<+->
	\[
		f'(z)=u_x-\ii u_y=-6xy-\ii (3y^2-3x^2)=3\ii z^2.
	\]
	\onslide<+->
	因此 $f(z)=\ii z^3+C$.
\end{frame}


\begin{frame}{典型例题: 求共轭调和函数和相应的解析函数}
	\beqskip{6pt}
	\onslide<+->
	\begin{example}[nearnext]
		求解析函数 $f(z)$ 使得它的虚部为
		\[
			v(x,y)=\ee^x(y\cos y+x\sin y)+x+y.
		\]
	\end{example}
	\onslide<+->
	\begin{solution}[nearprev]
		由 $u_x=v_y=\ee^x(\cos y-y\sin y+x\cos y)+1$ 得
		\[
			u=\ee^x(x\cos y-y\sin y)+x+\psi(y).
		\]
		\onslide<+->{%
			由 $u_y=-v_x=-\ee^x(y\cos y+x\sin y+\sin y)-1$ 得
			\[
				\psi'(y)=-1,\quad\psi(y)=-y+C.
			\]
		}\onslide<+->{%
			故
			\begin{align*}
				f(z)&=u+\ii v
				=\ee^x(x\cos y-y\sin y)+x-y+C
				+\ii\bigl[\ee^x(y\cos y+x\sin y)+x+y\bigr]\\
				&\visible<+->{=z\ee^z+(1+\ii)z+C,\quad C\in\BR.}
			\end{align*}
		}\bigdel
	\end{solution}
	\endgroup
\end{frame}


\begin{frame}{典型例题: 求共轭调和函数和相应的解析函数}
	\onslide<+->
	也可由
	\begin{align*}
		f'(z)&=v_y+\ii v_x\\
		&=\ee^x(\cos y-y\sin y+x\cos y)+1
		+\ii\bigl(\ee^x(y\cos y+x\sin y+\sin y)+1\bigr)\\
		&\visible<+->{=(z+1)\ee^z+1+\ii.}
	\end{align*}
	\onslide<+->
	得 $f(z)=z\ee^z+(1+\ii)z+C$.
	\onslide<+->
	\begin{exercise}[nearnext]
		证明 $u(x,y)=x^3-6x^2y-3xy^2+2y^3$ 是调和函数并求它的共轭调和函数.
	\end{exercise}
	\onslide<+->
	\begin{answer}[nearprev]
		$v(x,y)=2x^3+3x^2y-6xy^2-y^3+C$.
	\end{answer}
	\onslide<+->
	显然 $u+\ii v=(1+2\ii)z^3+\ii C$.
\end{frame}

