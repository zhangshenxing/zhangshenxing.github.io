\section{解析函数的概念}

\subsection{可导函数}

\begin{frame}{复变函数的导数}
	\onslide<+->
	由于 $\BC$ 和 $\BR$ 一样是域, 因此我们可以像一元实变函数一样去定义复变函数的导数和微分.
	\onslide<+->
	\begin{definition}
		设 $w=f(z)$ 的定义域是区域 $D$, $z_0\in D$.
		\onslide<+->{%
			若极限
			\[
				\lim_{z\to z_0}\frac{f(z)-f(z_0)}{z-z_0}
				=\lim_{\Delta z\to 0}\frac{f(z_0+\Delta z)-f(z_0)}{\Delta z}
			\]
			存在, 则称 \emph{$f(z)$ 在 $z_0$ 可导}.
		}\onslide<+->{%
			这个极限值称为 \emph{$f(z)$ 在 $z_0$ 的导数}, 记作 $f'(z_0)$.
		}\onslide<+->{%
			若 $f(z)$ 在区域 $D$ 内处处可导, 称 \emph{$f(z)$ 在 $D$ 内可导}.
		}
	\end{definition}
\end{frame}


\begin{frame}{例题: 线性函数的不可导性}
	\onslide<+->
	\begin{example}[nearnext]
		函数 $f(z)=x+2y\ii$ 在哪些点处可导?
	\end{example}
	\onslide<+->
	\begin{solution}[nearprev]
		由定义可知
		\begin{align*}
			f'(z)&=\lim_{\Delta z\to 0}\frac{f(z+\Delta z)-f(z)}{\Delta z}\\
			&\visible<+->{=\lim_{\Delta z\to 0}\frac{(x+\Delta x)+2(y+\Delta y)\ii-(x+2y\ii)}{\Delta z}}\\
			&\visible<+->{=\lim_{\Delta z\to 0}\frac{\Delta x+2\Delta y\ii}{\Delta x+\Delta y\ii}.}
		\end{align*}
		\onslide<+->{%
			当 $\Delta x=0, \Delta y\to 0$ 时, 上式$\to2$;
		}\onslide<+->{%
			当 $\Delta y=0, \Delta x\to 0$ 时, 上式$\to1$.
		}\onslide<+->{%
			因此该极限不存在, $f(z)$ 处处不可导.
		}\meddel
	\end{solution}
\end{frame}


\begin{frame}{例题: 复变函数的导数}
	\onslide<+->
	\begin{exercise}[nearnext]
		函数 $f(z)=\ov z=x-y\ii$ 在哪些点处可导? 
	\end{exercise}
	\onslide<+->
	\begin{answer}[nearprev]
		处处不可导.
	\end{answer}
	\onslide<+->
	\begin{example}[nearnext]
		求 $f(z)=z^2$ 的导数.
	\end{example}
	\onslide<+->
	\begin{solution}[nearprev]
		\bigdel
		\[
			f'(z)=\lim_{\Delta z\to 0}\frac{f(z+\Delta z)-f(z)}{\Delta z}
			\visible<+->{=\lim_{\Delta z\to 0}\frac{(z+\Delta z)^2-z^2}{\Delta z}}
			\visible<+->{=\lim_{\Delta z\to 0}(2z+\Delta z)=2z.}
		\]\bigdel
	\end{solution}
\end{frame}


\begin{frame}{求导运算法则}
	\onslide<+->
	事实上, 和单变量实函数情形类似, 复变函数也有如下求导法则.
	\onslide<+->
	\begin{theorem}
		\begin{enumerate}
			\item $(c)'=0$, 其中 $c$ 为复常数;
			\item $(z^n)'=nz^{n-1}$, 其中 $n$ 为整数;
			\item $(f\pm g)'=f'\pm g',\quad (cf)'=cf'$;\hfill\alert{线性性质}
			\item $(fg)'=f'g+fg',\quad \Bigl(\dfrac fg\Bigr)'=\dfrac{f'g-fg'}{g^2}$;\hfill\alert{莱布尼兹法则}
			\item $\Bigl(f\bigl(g(z)\bigr)\Bigr)'=f'[g(z)]\cdot g'(z)$;\hfill\alert{复合函数求导}
			\item $g'(z)=\dfrac1{f'(w)}, g=f^{-1}, w=g(z)$.\hfill\alert{反函数求导}
		\end{enumerate}
	\end{theorem}
\end{frame}


\begin{frame}{四则运算和复合的可导性}
	\onslide<+->
	由上述求导法则, 不难知道:
	\onslide<+->
	\begin{theorem}
		\begin{enumerate}
			\item 在 $z_0$ 处可导的两个函数 $f(z)$, $g(z)$ 之和、差、积、商($g(z_0)\neq 0$) 仍然在 $z_0$ 处可导.
			\item 若函数 $g(z)$ 在 $z_0$ 处可导, 函数 $f(w)$ 在 $g(z_0)$ 处可导, 则 $f\bigl(g(z)\bigr)$ 在 $z_0$ 处可导.
		\end{enumerate}
	\end{theorem}
	\onslide<+->
	由此可知, 多项式函数处处可导, 有理函数在其定义域内处处可导, 且二者导数形式和单变量实函数情形类似.
\end{frame}


\begin{frame}{例题: 利用求导运算法则计算导数}
	\onslide<+->
	\begin{example}[nearnext]
		求 $f(z)=\dfrac{z^2+3}{z+1}$ 的导数.
	\end{example}
	\onslide<+->
	\begin{solution}[nearprev]
		由于
		\[
			f(z)=z-1+\frac4{z+1},
		\]
		\onslide<+->{%
			因此
			\[
				f'(z)=1-\frac4{(z+1)^2}.
			\]
		}
	\end{solution}
\end{frame}


\begin{frame}{可导蕴含连续}
	\onslide<+->
	\begin{theorem}[nearprev]
		若 $f(z)$ 在 $z_0$ 可导, 则 $f(z)$ 在 $z_0$ 连续.
	\end{theorem}
	\onslide<+->
	该定理的证明和单变量实函数情形完全相同.
	\onslide<+->
	\begin{proof}
		设 $\Delta w=f(z_0+\Delta z)-f(z_0)$,
		\onslide<+->{%
			则
			\[
				\lim_{\Delta z\to 0}\Delta w
				=\lim_{\Delta z\to 0}\frac{\Delta w}{\Delta z}\cdot\Delta z
				\visible<+->{=\lim_{\Delta z\to 0}\frac{\Delta w}{\Delta z}\cdot\lim_{\Delta z\to 0}\Delta z}
				\visible<+->{=f'(z_0)\cdot 0=0,}
			\]
			}\onslide<+->{%
				从而 $f(z)$ 在 $z_0$ 处连续.\qedhere
			}
	\end{proof}
\end{frame}


\subsection{可微函数}


\begin{frame}{复变函数的微分}
	\onslide<+->
	复变函数的微分也和单变量实函数情形类似.
	\onslide<+->
	\begin{definition}
		若存在常数 $A$ 使得函数 $w=f(z)$ 满足
		\[
			\Delta w=f(z_0+\Delta z)-f(z_0)=A\Delta z+o(\Delta z),
		\]
		其中 $o(\Delta z)$ 表示 $\Delta z$ 的高阶无穷小量,
		\onslide<+->{%
			则称 \emph{$f(z)$ 在 $z_0$ 处可微},
		}
		\onslide<+->{%
			称 $A\Delta z$ 为 \emph{$f(z)$ 在 $z_0$ 的微分}, 记作 $\diff w=A \Delta z$.
		}
	\end{definition}
	\onslide<+->
	和一元实变函数情形一样, 复变函数的可微和可导是等价的, 且 $\diff w=f'(z_0)\Delta z$, $\diff z=\Delta z$.
	\onslide<+->
	故
	\[
		\emphn{\diff w=f'(z_0)\diff z,f'(z_0)=\dfrac{\diff w}{\diff z}}.
	\]
\end{frame}


\subsection{解析函数}


\begin{frame}{解析、解析函数和奇点}
	\onslide<+->
	\begin{definition}
		\begin{enumerate}
			\item 若函数 $f(z)$ 在 $z_0$ 的一个邻域内处处可导, 则称 \emph{$f(z)$ 在 $z_0$ 解析}.
			\item 若 $f(z)$ 在区域 $D$ 内处处解析, 则称 $f(z)$ 在 $D$ 内解析, 或称 $f(z)$ 是 $D$ 内的一个\emph{解析函数}.
			\item 若 $f(z)$ 在 $z_0$ 不解析, 则称 $z_0$ 为 $f(z)$ 的一个\emph{奇点}.
		\end{enumerate}
	\end{definition}
	\onslide<+->
	\alert{在一点解析蕴含在这点可导, 反之未必.} 
	\onslide<+->
	无定义、不连续、不可导、可导但不解析, 都会导致奇点的产生.
	\onslide<+->
	不过, 若 $z_0$ 是 $f(z)$ 定义域的外点, 即存在 $z_0$ 的邻域与 $f(z)$ 定义域交集为空集, 这种情形不甚有趣, 因此我们不考虑这类奇点.

	\onslide<+->
	\alert{在区域 $D$ 内解析和在 $D$ 内可导是等价的.} 
	\onslide<+->
	这是因为任意 $z_0\in D$ 均存在一个包含在 $D$ 内的邻域.
\end{frame}


\begin{frame}{例题: 解析与可导的关系}
	\onslide<+->
	由于一个点的邻域也是一个开集, 因此若 $f(z)$ 在 $z_0$ 处解析, 则 $f(z)$ 在 $z_0$ 的一个邻域内处处可导, 从而在该邻域内解析. 
	\onslide<+->
	因此 \alert{$f(z)$ 解析点全体是一个开集}, 它是可导点集合的内点构成的集合.
	\onslide<+->
	\begin{exercise}[nearnext]
		函数 $f(z)$ 在点 $z_0$ 处解析是 $f(z)$ 在该点可导的\fillbraceframe{A}.
		\begin{exchoice}(2)
			() 充分条件
			() 必要条件
			() 充要条件
			() 既非充分也非必要条件
		\end{exchoice}
	\end{exercise}
	\onslide<+->
	\begin{answer}[nearprev]
		解析要求在 $z_0$ 的一个邻域内都可导才行.
	\end{answer}
\end{frame}


\begin{frame}{例题: 函数的解析性}
	\onslide<+->
	\begin{example}[nearnext]
		研究函数 $f(z)=|z|^2$ 的解析性.
	\end{example}
	\onslide<+->
	\begin{solution}[nearprev]
		注意到
		\[
			\frac{f(z+\Delta z)-f(z)}{\Delta z}
			=\frac{(z+\Delta z)(\ov z+\ov{\Delta z})-z\ov z}{\Delta z}
			=\ov z+\ov{\Delta z}+z\frac{\Delta x-\Delta y\ii}{\Delta x+\Delta y\ii}.
		\]
		\begin{itemize}
			\item 若 $z=0$, 则当 $\Delta z\to 0$ 时该极限为 $0$.
			\item 若 $z\neq0$, 则当 $\Delta y=0,\Delta x\to 0$ 时该极限为 $\ov z+z$;
		\end{itemize}
		\onslide<+->{%
			当 $\Delta x=0,\Delta y\to 0$ 时该极限为 $\ov z-z$. 
		}\onslide<+->{%
			因此此时极限不存在.
		}\onslide<+->{%
			故 $f(z)$ 仅在 $z=0$ 处可导, 从而处处不解析.
		}
	\end{solution}
\end{frame}


\begin{frame}{解析函数的四则运算和复合}
	\onslide<+->
	\begin{theorem}
		\begin{enumerate}
			\item 在 $z_0$ 处解析的两个函数 $f(z)$, $g(z)$ 之和、差、积、商($g(z_0)\neq 0$) 仍然在 $z_0$ 处解析.
			\item 若函数 $g(z)$ 在 $z_0$ 处解析, 函数 $f(w)$ 在 $g(z_0)$ 处解析, 则 $f\bigl(g(z)\bigr)$ 在 $z_0$ 处解析.
		\end{enumerate}
	\end{theorem}
	\onslide<+->
	\begin{theorem}
		\begin{enumerate}
			\item 在 $D$ 内解析的两个函数 $f(z)$, $g(z)$ 之和、差、积、商仍然在 $D$ (作商时需要去掉 $g(z)=0$ 的点) 内解析.
			\item 若函数 $g(z)$ 在 $D$ 内解析且像均落在 $D'$ 中, 函数 $f(w)$ 在 $D'$ 内解析, 则 $f\bigl(g(z)\bigr)$ 在 $D$ 内解析.
		\end{enumerate}
	\end{theorem}
	\onslide<+->
	由此可知, 多项式函数处处解析. 有理函数在其定义域内处处解析, 分母的零点是它的奇点.
\end{frame}


