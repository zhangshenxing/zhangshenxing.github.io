\section{解析函数的概念}

\subsection{可导函数}
\begin{frame}{复变函数的导数}
	\begin{itemize}
		\item 由于 $\BC$ 和 $\BR$ 一样是域, 因此我们可以像一元实变函数一样去定义复变函数的导数和微分.
	\end{itemize}
	\onslide<+->
	\begin{definition}[leftupper=0pt]
		\begin{itemize}
			\item 设 $w=f(z)$ 的定义域是区域 $D$, $z_0\in D$.
			\item 若极限
			\[
				\lim_{z\to z_0}\frac{f(z)-f(z_0)}{z-z_0}
				=\lim_{\Delta z\to 0}\frac{f(z_0+\Delta z)-f(z_0)}{\Delta z}
			\]
			存在, 则称 \emph{$f(z)$ 在 $z_0$ 可导}.
			这个极限值称为 \emph{$f(z)$ 在 $z_0$ 的导数}, 记作 $f'(z_0)$.
			\item 若 $f(z)$ 在区域 $D$ 内处处可导, 称 \emph{$f(z)$ 在 $D$ 内可导}.
		\end{itemize}
	\end{definition}
\end{frame}


\begin{frame}{例: 线性函数的不可导性}
	\onslide<+->
	\begin{example}[nearnext]
		函数 $f(z)=x+2y\ii$ 在哪些点处可导?
	\end{example}
	\onslide<+->
	\begin{solution}[nearprev]
		\begin{align*}
			f'(z)&=\lim_{\Delta z\to 0}\frac{f(z+\Delta z)-f(z)}{\Delta z}\\
			&\visible<+->{=\lim_{\Delta z\to 0}\frac{(x+\Delta x)+2(y+\Delta y)\ii-(x+2yi)}{\Delta z}}\\
			&\visible<+->{=\lim_{\Delta z\to 0}\frac{\Delta x+2\Delta y i}{\Delta x+\Delta yi}.}
		\end{align*}
		\onslide<+->{%
			当 $\Delta x=0, \Delta y\to 0$ 时, 上式$\to2$;
		}\onslide<+->{%
			当 $\Delta y=0, \Delta x\to 0$ 时, 上式$\to1$.
		}\onslide<+->{%
			因此该极限不存在, $f(z)$ 处处不可导.
		}
	\end{solution}
\end{frame}


\begin{frame}{例: 复变函数的导数}
	\onslide<+->
		\begin{exercise}
			函数 $f(z)=x-yi$ 在哪些点处可导? 
		\end{exercise}\onslide<+->
		\begin{answer}
			处处不可导.
		\end{answer}
	\onslide<+->
	\begin{example}
		求 $f(z)=z^2$ 的导数.
	\end{example}

	\onslide<+->
	\begin{solution}
	\[
		f'(z)=\lim_{\Delta z\to 0}\frac{f(z+\Delta z)-f(z)}{\Delta z}
		\visible<+->{=\lim_{\Delta z\to 0}\frac{(z+\Delta z)^2-z^2}{\Delta z}}
		\visible<+->{=\lim_{\Delta z\to 0}(2z+\Delta z)=2z.}
	\]
	\end{solution}
\end{frame}


\begin{frame}{求导运算法则}
	\onslide<+->
	和一元实变函数情形类似, 我们有如下求导法则:
	\onslide<+->
	\begin{theorem}
		\begin{itemize}
			\item $(c)'=0$, 其中 $c$ 为复常数;
			\item $(z^n)'=nz^{n-1}$, 其中 $n$ 为整数;
			\item $(f\pm g)'=f'\pm g',\quad (cf)'=cf'$;
			\item $(fg)'=f'g+fg',\quad \Bigl(\dfrac fg\Bigr)'=\dfrac{f'g-fg'}{g^2}$;
			\item $[f(g(z))]'=f'[g(z)]\cdot g'(z)$;
			\item $g'(z)=\dfrac1{f'(w)}, g=f^{-1}, w=g(z)$.
		\end{itemize}
	\end{theorem}
\end{frame}


\begin{frame}{可导蕴含连续}
	\onslide<+->
	\begin{theorem}
		若 $f(z)$ 在 $z_0$ 可导, 则 $f(z)$ 在 $z_0$ 连续.
	\end{theorem}
	\onslide<+->
	该定理的证明和实变量情形完全相同.
	\onslide<+->
	\begin{proof}
		设 \[\Delta w=f(z_0+\Delta z)-f(z_0),
	\]
		\onslide<+->{%
			则
			\begin{align*}
			\lim_{\Delta z\to 0}\Delta w&=\lim_{\Delta z\to 0}\frac{\Delta w}{\Delta z}\cdot\Delta z\\
			&\visible<+->{=\lim_{\Delta z\to 0}\frac{\Delta w}{\Delta z}\cdot\lim_{\Delta z\to 0}\Delta z}
			\visible<+->{=f'(z_0)\cdot 0=0.}\qedhere
			\end{align*}
		}
		\vspace{-\baselineskip}
	\end{proof}
\end{frame}


\subsection{可微函数}

% \alertn{f'(z_0)=\frac{\diff w}{\diff z}\Big|_{z=z_0}
% =\lim_{\Delta z\to 0}\frac{f(z_0+\Delta z)-f(z_0)}{\Delta z}.}
% \]
\begin{frame}{复变函数的微分}
	\onslide<+->
	复变函数的微分也和一元实变函数情形类似.

	\onslide<+->
	\begin{definition}
		若存在常数 $A$ 使得函数 $w=f(z)$ 满足
	\[
		\Delta w=f(z_0+\Delta z)-f(z_0)=A\Delta z+o(\Delta z),
	\]
		其中 $o(\Delta z)$ 表示 $\Delta z$ 的高阶无穷小量,
		\onslide<+->{%
			则称 \emph{$f(z)$ 在 $z_0$ 处可微},
		}\onslide<+->{%
			称 $A\Delta z$ 为 \emph{$f(z)$ 在 $z_0$ 的微分}, 记作 $\diff w=A \Delta z$.
		}
	\end{definition}

	\onslide<+->
	和一元实变函数情形一样, 复变函数的可微和可导是等价的, 且 $\diff w=f'(z_0)\Delta z, \diff z=\Delta z$.
	\onslide<+->
	故 \emph{$\diff w=f'(z_0)\diff z,f'(z_0)=\dfrac{\diff w}{\diff z}$}.
\end{frame}


\subsection{解析函数}

\begin{frame}{解析函数}
	\onslide<+->
	\begin{definition}
		\begin{itemize}
			\item 若函数 $f(z)$ 在 $z_0$ 的一个邻域内处处可导, 则称 \emph{$f(z)$ 在 $z_0$ 解析}.
			\item 若 $f(z)$ 在区域 $D$ 内处处解析, 则称 $f(z)$ 在 $D$ 内解析, 或称 $f(z)$ 是 $D$ 内的一个\emph{解析函数}.
			\item 若 $f(z)$ 在 $z_0$ 不解析, 则称 $z_0$ 为 $f(z)$ 的一个\emph{奇点}.
		\end{itemize}
	\end{definition}

	\onslide<+->
	无定义、不连续、不可导、可导但不解析, 都会导致奇点的产生.

	\onslide<+->
	由于区域 $D$ 是一个开集, 其中的任意 $z_0\in D$ 均存在一个包含在 $D$ 的邻域. 
	\onslide<+->
	所以 \alert{$f(z)$ 在 $D$ 内解析和在 $D$ 内可导是等价的}.

	\onslide<+->
	若 $f(z)$ 在 $z_0$ 解析, 则 $f(z)$ 在 $z_0$ 的一个邻域内处处可导, 从而在该邻域内解析.
	\onslide<+->
	因此 \alert{$f(z)$ 解析点全体是一个开集}.
\end{frame}


\begin{frame}{解析函数}
	\onslide<+->
	\begin{exercise}
		函数 $f(z)$ 在点 $z_0$ 处解析是 $f(z)$ 在该点可导的\fillbraceframe{A}.
		\begin{exchoice}(2)
			() 充分条件
			() 必要条件
			() 充要条件
			() 既非充分也非必要条件
		\end{exchoice}
	\end{exercise}

	\onslide<+->
	\begin{answer}
		解析要求在 $z_0$ 的一个邻域内都可导才行.
	\end{answer}
\end{frame}

\begin{frame}{解析函数}
	\onslide<+->
	\begin{example}
		研究函数 $f(z)=|z|^2$ 的解析性.
	\end{example}

	\onslide<+->
	\begin{solution}
		由于
	\[
			\frac{f(z+\Delta z)-f(z)}{\Delta z}
			=\frac{(z+\Delta z)(\ov z+\ov{\Delta z})-z\ov z}{\Delta z}
			=\ov z+\ov{\Delta z}+z\frac{\Delta x-\Delta yi}{\Delta x+\Delta yi},
	\]
		\begin{enumerate}
			\item 若 $z=0$, 则当 $\Delta z\to 0$ 时该极限为 $0$.
			\item 若 $z\neq0$, 则当 $\Delta y=0,\Delta x\to 0$ 时该极限为 $\ov z+z$;
			\onslide<+->{%
				当 $\Delta x=0,\Delta y\to 0$ 时该极限为 $\ov z-z$.
			}\onslide<+->{%
				因此此时极限不存在.
			}
		\end{enumerate}
		\onslide<+->{%
			故 $f(z)$ 仅在 $z=0$ 处可导, 从而处处不解析.
		}
	\end{solution}
\end{frame}

