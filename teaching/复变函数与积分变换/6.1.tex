\section{傅里叶积分和傅里叶变换}


\begin{frame}{傅里叶级数}
\onslide<+->
我们先考虑周期函数的傅里叶级数展开.

\onslide<+->
设 $f(t)$ 是定义在 $(-\infty,+\infty)$ 上周期为 $T$ 的可积实变函数.
\onslide<+->
我们知道 $\cos{n\omega t}$ 和 $\sin{n\omega t}$ 周期也是 $T$, 其中 $\omega=\dfrac{2\pi}T$.
\onslide<+->
如果 $f(t)$ 在 $\left[-\dfrac T2,\dfrac T2\right]$ 上满足\markdef{狄利克雷条件}:
\begin{itemize}
\item 间断点只有有限多个, 且均为第一类间断点;
\item 只有有限个极值点,
\end{itemize}
\onslide<+->
则我们有\markdef{傅里叶级数}展开:
\[f(t)=\frac{a_0}2+\sum_{n=1}^\infty \left(a_n\cos{n\omega t}+b_n  \sin{n\omega t}\right).\]
\onslide<+->
当 $t$ 是间断点时, 傅里叶级数的左侧需改为 $\dfrac{f(t+)+f(t-)}2$.
\end{frame}


\begin{frame}{傅里叶级数的复指数形式}
\onslide<+->
我们来将其改写为复指数形式.
\onslide<+->
物理中为了与电流 $i$ 区分, 通常用 $j$ 来表示虚数单位.
\onslide<+->
由
\[\cos x=\frac{e^{jx}+e^{-jx}}2,\quad \sin x=\frac{e^{jx}-e^{-jx}}{2j}\]
\onslide<+->
可知 $f(t)$ 的傅里叶级数可以表示为函数 $e^{jn\omega t}$ 的线性组合.
\onslide<+->
设 $f(t)=\suml_{n=-\infty}^{+\infty} c_n e^{jn\omega t}$, 则
\[c_n=\frac 1T \int_{-\frac T2}^{\frac T2} f(t)e^{-jn\omega t} \diff t.\]
\onslide<+->
于是我们得到周期函数\markdef{傅里叶级数的复指数形式}:
\[f(t)=\frac 1T\sum_{n=-\infty}^{+\infty}\left[\int_{-\frac T2}^{\frac T2}f(\tau)e^{-jn\omega\tau} \diff\tau\right] e^{jn\omega t}.\]
\end{frame}


\begin{frame}{从傅里叶级数到傅里叶积分公式}
\onslide<+->
对于一般的函数 $f(t)$, 它未必是周期的.
\onslide<+->
我们考虑它在 $\left[-\dfrac T2,\dfrac T2\right]$ 上的限制, 并向两边扩展成一个周期函数 $f_T(t)$.
\onslide<+->
设 $\omega_n=n\omega, \Delta\omega_n=\omega_n-\omega_{n-1}=\omega$, 则
\begin{align*}
f(t)&=\lim_{T\to +\infty}f_T(t)\\
&\visible<+->{=\lim_{T\to+\infty} \frac 1T \sum_{n=-\infty}^{+\infty} \left[\int_{-\frac T2}^{\frac T2}f(\tau)e^{-j\omega_n\tau}\diff\tau\right] e^{j\omega_n t}}\\
&\visible<+->{=\frac1{2\pi}\lim_{\Delta\omega_n\to 0}\sum_{n=-\infty}^{+\infty}\marknot{\left[\int_{-\frac T2}^{\frac T2}f(\tau)e^{-j\omega_n\tau} \diff \tau\right] e^{j\omega_n t}}\Delta\omega_n}\\
&\visible<+->{=\frac1{2\pi}\int_{-\infty}^{+\infty}\marknot{\left[\int_{-\infty}^{+\infty}f(\tau) e^{-j\omega\tau} \diff \tau\right]e^{j\omega t}}\diff\omega.}
\end{align*}
\end{frame}


\begin{frame}{傅里叶积分定理}
\beqskip{9pt}
\begin{theorem}[傅里叶积分定理]
设函数 $f(t)$ 满足
\begin{itemize}
\item 在 $(-\infty,+\infty)$ 上\marknot{绝对可积};
\item 在任一有限区间上满足狄利克雷条件.
\end{itemize}
\onslide<+->
那么
\[\markatt{f(t)=\frac1{2\pi} \int_{-\infty}^{+\infty}F(\omega) e^{j\omega t} \diff\omega,\qquad  F(\omega)=\int_{-\infty}^{+\infty}f(t) e^{-j\omega t} \diff t.}\]
\onslide<+->
对于 $f(t)$ 的间断点左边需要改成 $\dfrac{f(t+)+f(t-)}2$.
\end{theorem}
\onslide<+->
\begin{center}
\begin{tikzpicture}[node distance=-15pt]
  \node[draw](a){原象函数 $f(t)$};
  \node[draw,right=100pt of a]	(b){象函数 $F(\omega)$};
  \node[above=of a] (a1){\phantom{原象函数 $f(t)$}};
  \node[below=of a] (a2){\phantom{原象函数 $f(t)$}};
  \node[above=of b] (b1){\phantom{象函数 $F(\omega)$}};
  \node[below=of b] (b2){\phantom{象函数 $F(\omega)$}};
  \draw[cstnarrow, alecolor] (a1) -- node[above]{傅里叶变换 $\msf$} (b1);
  \draw[cstnarrow, defcolor]   (b2) -- node[below]{傅里叶逆变换 $\msf^{-1}$} (a2);
\end{tikzpicture}
\end{center}
\endgroup
\end{frame}


\begin{frame}{傅里叶积分公式的三角形式*}
\onslide<+->
我们来看一下\markdef{傅里叶积分公式的三角形式}:
\onslide<+->
\begin{align*}
f(t)&=\frac1{2\pi} \int_{-\infty}^{+\infty}\left[\int_{-\infty}^{+\infty}f(\tau) e^{-j\omega\tau} \diff\tau\right] e^{j\omega t} \diff\omega\\
&\visible<+->{=\frac1{2\pi} \int_{-\infty}^{+\infty}\int_{-\infty}^{+\infty}f(\tau) e^{j\omega(t-\tau)}\diff\tau \diff\omega}\\
&\visible<+->{=\frac1{2\pi} \int_{-\infty}^{+\infty}\int_{-\infty}^{+\infty}f(\tau)[\cos{\omega(t-\tau)}+j\sin{\omega(t-\tau)}]\diff\tau \diff\omega}\\
&\visible<+->{\markatt{=\frac1\pi\int_0^{+\infty}\left[\int_{-\infty}^{+\infty}f(\tau)\cos{\omega(t-\tau)}\diff\tau\right]\diff\omega.}}
\end{align*}
\onslide<+->
这里, 作为 $\omega$ 的函数, 带 $\cos$ 部分的积分是偶函数, 带 $\sin$ 部分的积分是奇函数.
\end{frame}


\begin{frame}{傅里叶正弦/余弦积分公式*}
\onslide<+->
对上式再次展开得到:
\[f(t)=\frac1\pi\int_0^{+\infty}\left[\int_{-\infty}^{+\infty}f(\tau)(\cos{\omega t}\cos{\omega\tau}+\sin{\omega t}\sin{\omega\tau})\diff\tau\right]\diff\omega.\]
\onslide<+->
若 $f(t)$ 是奇函数, $f(\tau)\cos{\omega\tau}$ 是奇函数, $f(\tau)\sin{\omega\tau}$ 是偶函数.
\onslide<+->
由此得到\markdef{傅里叶正弦变换/正弦积分公式}:
\[\markatt{f(t)=\frac2\pi\int_0^{+\infty}\left[\int_0^{+\infty}f(\tau)\sin{\omega\tau}\diff\tau\right]\sin{\omega t}\diff\omega.}\]
\onslide<+->
类似地, 若 $f(t)$ 是偶函数, 有\markdef{傅里叶余弦变换/余弦积分公式}:
\[\markatt{f(t)=\frac2\pi\int_0^{+\infty}\left[\int_0^{+\infty}f(\tau)\cos{\omega\tau}\diff\tau\right]\cos{\omega t}\diff\omega.}\]
\end{frame}


\begin{frame}{例题: 求傅里叶变换}
\begin{example}
求函数 $f(t)=
	\begin{cases}
		1, & |t|\le 1,\\
		0, & |t|>1
	\end{cases}$
的傅里叶变换.
\end{example}

\begin{solution}
\vspace{-\baselineskip}
\begin{align*}
F(\omega)&=\msf[f(t)]=\int_{-\infty}^{+\infty}f(t)e^{-j\omega t}\diff t\\
&\visible<+->{=\int_{-1}^1(\cos{\omega t}-j\sin{\omega t})\diff t}\visible<+->{=\frac{2\sin \omega}{\omega}.}&\visible<.->{}
\end{align*}
\end{solution}
\end{frame}


\begin{frame}{例题: 求傅里叶变换}
\onslide<+->
由傅里叶积分公式
\begin{align*}
f(t)&=\msf^{-1}[F(\omega)]=\frac1{2\pi}\int_{-\infty}^{+\infty}F(\omega)e^{j\omega t}\diff\omega\\
&\visible<+->{=\frac1{2\pi}\int_{-\infty}^{+\infty}\frac{2\sin\omega}{\omega}(\cos{\omega t}+j\sin{\omega t})\diff \omega}\\
&\visible<+->{=\frac2\pi\int_0^{+\infty}\frac{\sin\omega\cos{\omega t}}{\omega}\diff \omega.}
\end{align*}
\onslide<+->
当 $t=\pm1$ 时, 左侧应替换为 $\frac12[f(t+)+f(t-)]=\frac12$.
\onslide<+->
由此可得
\[\int_0^{+\infty}\frac{\sin \omega\cos{\omega t}}\omega\diff\omega=\begin{cases}
\pi/2,&|t|<1,\\
\pi/4,&|t|=1,\\
0,&|t|>1.
\end{cases}\]
\onslide<+->
特别地, 可以得到狄利克雷积分
$\displaystyle\int_0^{+\infty}\frac{\sin\omega}\omega\diff\omega=\frac\pi2$.
\end{frame}


\begin{frame}{例题: 求傅里叶变换}
\begin{example}
求函数 $f(t)=
	\begin{cases}
		1,&t\in(0,1),\\
		-1,&t\in(-1,0),\\
		0,&\text{其它情形}
	\end{cases}$ 的傅里叶变换.
\end{example}
\begin{solution}
\vspace{-\baselineskip}
\begin{align*}
F(\omega)&=\msf[f(t)]=\int_{-\infty}^{+\infty}f(t)e^{-j\omega t}\diff t\\
&\visible<+->{=\left(\int_0^1-\int_{-1}^0\right)(\cos{\omega t}-j\sin{\omega t})\diff t}\\
&\visible<+->{=-\frac{2j(1-\cos\omega)}\omega.}&\visible<.->{}
\end{align*}
\end{solution}
\end{frame}


\begin{frame}{例题: 求傅里叶变换}
\onslide<+->
由傅里叶积分公式
\begin{align*}
f(t)&=\msf^{-1}[F(\omega)]=\frac1{2\pi}\int_{-\infty}^{+\infty}F(\omega)e^{j\omega t}\diff\omega\\
&\visible<+->{=-\frac j\pi\int_{-\infty}^{+\infty}\frac{1-\cos\omega}\omega(\cos{\omega t}+j\sin{\omega t})\diff\omega}\\
&\visible<+->{=\frac2\pi\int_0^{+\infty}\frac{(1-\cos\omega)\sin{\omega t}}\omega\diff\omega\quad(t\neq\pm1),}
\end{align*}
\onslide<+->
由此可得
\[\int_0^{+\infty}\frac{(1-\cos\omega)\sin{\omega t}}\omega\diff\omega=
	\begin{cases}
		\pi/2,&0<t<1,\\
		\pi/4,&t=1,\\
		0,&t>1.	
	\end{cases}\]
\end{frame}


\begin{frame}{例题: 求傅里叶变换}
\begin{example}
求\markdef{指数衰减函数} $f(t)=
	\begin{cases}
		0,&t<0,\\
		e^{-\beta t},&t\ge 0
	\end{cases}$ 的傅里叶变换, $\beta>0$.
\end{example}
\begin{solution}
\vspace{-\baselineskip}
\begin{align*}
F(\omega)&=\msf[f(t)]=\int_{-\infty}^{+\infty}f(t)e^{-j\omega t}\diff t\\
&\visible<+->{=\int_0^{+\infty}e^{-\beta t}e^{-j\omega t}\diff t}
\visible<+->{=\int_0^{+\infty}e^{-(\beta+j\omega)t}\diff t}\\
&\visible<+->{\markatt{=\frac1{\beta+j\omega}}.}&\visible<.->{}
\end{align*}
\end{solution}
\end{frame}


\begin{frame}{例题: 求傅里叶变换}
\onslide<+->
由傅里叶积分公式
\begin{align*}
f(t)&=\msf^{-1}[F(\omega)]=\frac1{2\pi}\int_{-\infty}^{+\infty}F(\omega)e^{j\omega t}\diff\omega
\visible<+->{=\frac1{2\pi}\int_{-\infty}^{+\infty}\frac{e^{j\omega t}}{\beta+j\omega}\diff\omega}\\
&\visible<+->{=\frac1{2\pi}\int_{-\infty}^{+\infty}\frac{\beta\cos{\omega t}+\omega\sin{\omega t}}{\beta^2+\omega^2}\diff\omega}\\
&\visible<+->{=\frac1\pi\int_0^{+\infty}\frac{\beta\cos{\omega t}+\omega\sin{\omega t}}{\beta^2+\omega^2}\diff\omega.}
\end{align*}
\onslide<+->
由此可得
\[\int_0^{+\infty}\frac{\beta\cos{\omega t}+\omega\sin{\omega t}}{\beta^2+\omega^2}\diff\omega=
	\begin{cases}
		0,&t<0,\\
		\pi/2,&t=0,\\
		\pi e^{-\beta t},&t>0.
	\end{cases}\]
\end{frame}


\begin{frame}{例题: 求傅里叶变换}
\begin{example}
求\markdef{钟形脉冲函数} $f(t)=e^{-\beta t^2}$ 的傅里叶变换和积分表达式, $\beta>0$.
\end{example}
\begin{solution}
\vspace{-\baselineskip}
\begin{align*}
F(\omega)&=\msf[f(t)]=\int_{-\infty}^{+\infty}f(t)e^{-j\omega t}\diff t\\
&\visible<+->{=\int_{-\infty}^{+\infty}e^{-\beta t^2}e^{-j\omega t}\diff t}\\
&\visible<+->{=e^{-\frac{\omega^2}{4\beta}}\int_{-\infty}^{+\infty}\exp\left[-\beta\left(t+\frac{j\omega}{2\beta}\right)^2\right]\diff t}
\visible<+->{\markatt{=\sqrt{\frac\pi\beta}e^{-\frac{\omega^2}{4\beta}}.}}&\visible<.->{}
\end{align*}
\end{solution}
\end{frame}


\begin{frame}{例题: 求傅里叶变换}
\onslide<+->
由傅里叶积分公式
\begin{align*}
f(t)&=\msf^{-1}[F(\omega)]=\frac1{2\pi}\int_{-\infty}^{+\infty}F(\omega)e^{j\omega t}\diff\omega\\
&\visible<+->{=\frac1{2\pi}\int_{-\infty}^{+\infty}\sqrt{\frac\pi\beta}e^{-\frac{\omega^2}{4\beta}}e^{j\omega t}\diff\omega}\\
&\visible<+->{=\frac1{\sqrt{\pi\beta}}\int_0^{+\infty}e^{-\frac{\omega^2}{4\beta}}\cos{\omega t}\diff\omega.}
\end{align*}
\onslide<+->
从该傅里叶积分可以得到反常积分
\[\int_0^{+\infty}e^{-\frac{\omega^2}{4\beta}}\cos{\omega t}\diff\omega=\sqrt{\pi\beta}e^{-\beta t^2}.\]
\end{frame}


\begin{frame}{广义函数}
\onslide<+->
傅里叶变换存在的条件是比较苛刻的.
\onslide<+->
例如常值函数 $f(t)=1$ 在 $(-\infty,+\infty)$ 上不是可积的. 所以它没有傅里叶变换, 这很影响我们使用傅里叶变换.
\onslide<+->
为此我们需要引入广义函数的概念.

\onslide<+->
设 $\msc$ 是一些函数形成的线性空间, 例如全体绝对可积函数, 或者全体光滑函数之类的. 
\onslide<+->
从一个函数 $\lambda(t)$ 出发, 可以定义一个线性映射 $\msc\to \BR$:
\[\pair{\lambda,f}:=\int_{-\infty}^{+\infty}\lambda(t)f(t)\diff t.\]
\onslide<+->
这个线性映射基本上确定了 $\lambda(t)$ 本身(至多可数个点处不同).

\onslide<+->
\markdef{广义函数}是指一个线性映射 $\msc\to \BR$.
\onslide<+->
为了和普通函数类比, 通常也将广义函数表为上述积分形式:
\[\int_{-\infty}^{+\infty}\lambda(t)f(t)\diff t.\]
这里的 $\lambda(t)$ 并不表示一个函数.
\end{frame}


\begin{frame}{Dirac $\delta$ 函数}
\onslide<+->
\markdef{$\delta$ 函数}定义为线性映射
\[\pair{\delta,f}=\int_{-\infty}^{+\infty}\delta(t)f(t)\diff t=f(0).\]
\onslide<6->
\begin{tikzpicture}[overlay,xshift=2cm]
\draw[cstaxis](8,-1)--(10,-1);
\draw[cstaxis](9,-1.5)--(9,2);
\draw
	(9.8,-1.25) node {$t$}
	(8.7,0.9) node {$1$}
	(9.5,0.8) node[alecolor] {$\delta(t)$}
	(8.8,-1.2) node {$O$};
\draw[alecolor,cstcurve,cstarrowto](9,-1)--(9,1);
\end{tikzpicture}
\onslide<+->
设 $\delta_\varepsilon(t)=\begin{cases}
1/\varepsilon,&0\le t\le \varepsilon,\\
0,&\text{其它情形,}
\end{cases}$
\onslide<+->
则对于连续函数 $f(t)$,
\[\pair{\delta_\varepsilon,f}=\frac1\varepsilon\int_0^\varepsilon f(t)\diff t=f(\xi),\quad \xi\in(0,\varepsilon).\]
\onslide<+->
当 $\varepsilon\to0$ 时, 右侧就趋于 $f(0)$.
\onslide<+->
因此 $\delta$ 可以看成 $\delta_\varepsilon$ 的某种极限.
\onslide<+->
基于此, 我们通常用长度为 $1$ 的有向线段来表示它, 有时候也记做
\[\delta(t)=\begin{cases}
\infty,&t=0,\\
0,&t=0.
\end{cases}\]
\end{frame}


\begin{frame}{Dirac $\delta$ 函数的性质}
对于广义函数 $\lambda$, 我们可以形式地定义 $\lambda(at),\lambda'$.
\onslide<+->
\begin{itemize}
\item $\pair{\delta^{(n)},f}=(-1)^n\pair{\delta,f^{(n)}}=(-1)^nf^{(n)}(0)$, 其中 $f(t)$ 是光滑函数.
\item $\delta(at)=\frac1{|a|}\delta(t)$, 特别地 $\delta(t)=\delta(-t)$.
\item 设 $u(t)=\begin{cases}1,&t\ge0,\\0,&t<0\end{cases}$ 是\markdef{单位阶跃函数}(Heaviside 函数), 则
\[\int_{-\infty}^x\delta(t)\diff t=\int_{-\infty}^{+\infty}\textbf{1}_{(-\infty,x]}\delta(t)\diff t=u(x).\]
因此 $u'(t)=\delta(t)$.
\item $\markatt{\msf[\delta(t)]}=\pair{\delta(t),e^{-j\omega t}}\markatt{=1}$, $\markatt{\msf^{-1}[\delta(\omega)]}=\dfrac1{2\pi}\pair{\delta(\omega),e^{j\omega t}}\markatt{=\dfrac1{2\pi}}$.
\end{itemize}
\end{frame}


\begin{frame}{例题: 求傅里叶变换}
\begin{example}
证明 \markatt{$\msf[u(t)]=\dfrac1{j\omega}+\pi\delta(\omega)$}.
\end{example}
\begin{proof}
\[\msf^{-1}\left[\frac1{j\omega}\right]
=\frac1{2\pi}\int_{-\infty}^{+\infty}\frac{e^{j\omega t}}{j\omega} \diff\omega
=\frac1\pi\int_0^{+\infty}\frac{\sin{\omega t}}{\omega} \diff\omega.\]
\onslide<+->
由 $\displaystyle\int_0^{+\infty}\frac{\sin\omega}\omega \diff\omega=\dfrac\pi2$
可知 $\displaystyle\int_0^{+\infty}\frac{\sin{\omega t}}\omega \diff\omega=\frac\pi2\sgn(t)$.
\onslide<+->
故
\[\msf^{-1} \left[\frac1{j\omega}+\pi\delta(\omega)\right]
=\frac12\sgn(t)+\frac12=u(t)\ (t\neq 0).\qedhere\]
\end{proof}
\end{frame}
