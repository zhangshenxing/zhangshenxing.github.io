\subsection{复数的引入}

\begin{frame}{复数的引入}
\onslide<+->复数起源于多项式方程的求根问题.
\onslide<+->我们考虑一元二次方程 $x^2+bx+c=0$,
\onslide<+->配方可得
	\[\left(x+\frac b2\right)^2=\frac{b^2-4c}4.\]
\onslide<+->于是得到求根公式
	\[x=\frac{-b\pm\sqrt\Delta}2,\quad \Delta=b^2-4c.\]
	\vspace{-\baselineskip}
\begin{enumerate}
\item 当 $\Delta>0$ 时, 有两个不同的实根;
\item 当 $\Delta=0$ 时, 有一个二重的实根;
\item 当 $\Delta<0$ 时, 无实根.
\onslide<+->然而, 如果我们\emph{接受负数开方}的话, 此时仍然有两个根, 形式地计算可以发现它们满足原来的方程.
\end{enumerate}
\end{frame}

\begin{frame}{三次方程的根}
\onslide<+->现在我们来考虑一元三次方程.
\onslide<+->
\begin{example}
解方程 $x^3+6x-20=0$.
\end{example}
\onslide<+->
\begin{solution}
设 $x=u+v$, 则 $\quad u^3+v^3+3uv(u+v)+6(u+v)-20=0$.\par
\vspace{\abovedisplayskip}
\onslide<+->{我们希望 $\qquad\qquad u^3+v^3=20,\quad uv=-2$,}\par
\vspace{\abovedisplayskip}
\onslide<+->{则 $u^3,v^3$ 满足一元二次方程 $X^2-20X-8=0$.}
\onslide<+->{解得
\[u^3=10\pm\sqrt{108}\visible<+->{=(1\pm\sqrt3)^3.}\]}
\onslide<+->{所以 $\qquad u=1\pm\sqrt3,\quad v=1\mp\sqrt 3$,}
\onslide<+->{$\quad	x=u+v=2$.}
\end{solution}
\end{frame}


\begin{frame}{三次方程的根}
\onslide<+->
那么这个方程是不是真的只有 $x=2$ 这一个解呢?
\onslide<+->
这从它的函数图像可以看出.
\begin{center}
	\begin{tikzpicture}
	\filldraw[cstcurve,dcolora,domain=-3.5:4,smooth,fill=white] plot ({(\x)*0.2},{(\x*\x*\x+6*\x-20)*0.02});
	\draw[cstaxis] (-1.5,0)--(1.5,0);
	\draw[cstaxis] (0,-1.7)--(0,1.5);
	\fill[cstdot,dcolorb] (0,-0.4) circle;
	\draw
		(1.4,-0.7) node[dcolorb] {\small 拐点 $(0,-20)$};
	\end{tikzpicture}
\end{center}
\end{frame}


\begin{frame}{三次方程的根}
\onslide<+->
\begin{example}
解方程 $x^3-7x+6=0$.
\end{example}
\onslide<+->
\begin{solution*}
同样地我们有 $x=u+v$, 其中
	\[u^3+v^3=-6,\quad uv=\frac73.\]
\visible<+->{于是 $u^3,v^3$ 满足一元二次方程 $X^2+6X+\frac{343}{27}=0$.}
\visible<+->{然而这个方程没有实数解.}

\visible<+->{我们可以强行解得
	\[u^3=-3+\frac{10}9\sqrt{-3}.\]}
\vspace{-1.3\baselineskip}
\end{solution*}
\end{frame}


\begin{frame}{三次方程的根}
\onslide<+->
\begin{solutionc}
	\[u=\sqrt[3]{-3+\frac{10}9\sqrt{-3}}
	=\frac{3+2\sqrt{-3}}3,\frac{-9+\sqrt{-3}}6,\frac{3-5\sqrt{-3}}6,\]
\visible<+->{相应地
	\[v=\frac{3-2\sqrt{-3}}3,\frac{-9-\sqrt{-3}}6,\frac{3+5\sqrt{-3}}6,\]}
\visible<+->{\[x=u+v=2,-3,1.\]}
\vspace{-\baselineskip}
\end{solutionc}
\onslide<+->{所以我们从一条``\emph{错误的路径}''走到了正确的目的地?}
\end{frame}


\begin{frame}{三次方程的根}
\onslide<+->对于一般的三次方程 $x^3+px+q=0$ 而言, 类似可得:
\[x=u-\frac p{3u},\quad u^3=-\frac q2+\sqrt{\Delta},\quad \Delta=\frac{q^2}4+\frac{p^3}{27}.\]
\onslide<+->由于 $p=0$ 情形较为简单, 所以我们不考虑这种情形.
\onslide<+->通过分析函数图像的极值点可以知道:
\begin{enumerate}
\item 当 $\Delta>0$ 时, 有 $1$ 个实根.
\item 当 $\Delta=0$ 时, 有 $2$ 个实根 $x=-\sqrt[3]{4q},\frac12\sqrt[3]{4q}$ ($2$重).
\item 当 $\Delta<0$ 时, 有 $3$ 个实根.
\end{enumerate}

\begin{center}
	\begin{figure}[h]
		\begin{subfigure}{0.3\textwidth}
			\centering
			\begin{tikzpicture}
				\draw[cstcurve,dcolora,domain=-2.8:3.5,smooth,visible on=<4->] plot ({(\x)*0.2},{(\x*\x*\x+6*\x-20)*0.02});
				\draw[cstaxis,visible on=<4->] (-1.5,0)--(1.5,0);
				\draw[cstaxis,visible on=<4->] (0,-1.2)--(0,1.2);
			\end{tikzpicture}
		\end{subfigure}
		\begin{subfigure}{0.3\textwidth}
			\centering
			\begin{tikzpicture}
				\draw[cstcurve,dcolora,domain=-3.1:2.7,smooth,visible on=<5->] plot ({(\x)*0.25},{(\x*\x*\x-3*\x+2)*0.06});
				\draw[cstaxis,visible on=<5->] (-1.5,0)--(1.5,0);
				\draw[cstaxis,visible on=<5->] (0,-1.2)--(0,1.2);
			\end{tikzpicture}
		\end{subfigure}
		\begin{subfigure}{0.3\textwidth}
			\centering
			\begin{tikzpicture}
				\draw[cstcurve,dcolora,domain=-3.9:3.7,smooth,visible on=<6->] plot ({(\x)*0.2},{(\x*\x*\x-7*\x+1)*0.03});
				\draw[cstaxis,visible on=<6->] (-1.5,0)--(1.5,0);
				\draw[cstaxis,visible on=<6->] (0,-1.2)--(0,1.2);
			\end{tikzpicture}
		\end{subfigure}
	\end{figure}
\end{center}
\end{frame}


\begin{frame}{三次方程的根}
\onslide<+->所以我们想要使用求根公式的话, 就\emph{必须接受负数开方}.
\onslide<+->那么为什么当 $\Delta<0$ 时, 从求根公式一定能得到 $3$ 个实根呢?
\onslide<+->在学习了第一章的内容之后我们就可以回答这个问题了.

\onslide<+->尽管在十六世纪, 人们已经得到了三次方程的求根公式, 然而对其中出现的虚数, 却是难以接受.
\onslide<+->
\begin{quote@*}
圣灵在分析的奇观中找到了超凡的显示, 这就是那个理想世界的端兆, 那个介于存在与不存在之间的两栖物, 那个我们称之为虚的 $-1$ 的平方根。
\tcblower
莱布尼兹 (Leibniz)
\end{quote@*}
\end{frame}
