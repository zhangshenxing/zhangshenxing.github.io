\section{柯西积分公式}


\begin{frame}{柯西积分公式}
\onslide<+->
柯西积分定理是解析函数理论的基础, 但在很多情形下它由柯西积分公式表现.
\begin{block}{柯西积分公式}
设
\begin{itemize}[<*>]
\item 函数 $f(z)$ 在闭路或复合闭路 $C$ 及其内部 $D$ 解析,
\item $z_0\in D$,
\end{itemize}
\onslide<+->
则
\[f(z_0)=\frac1{2\pi i}\oint_C\frac{f(z)}{z-z_0}\diff z.\]
\end{block}
\onslide<+->
如果 $z_0\notin D$, 由柯西-古萨基本定理, 右侧的积分是 $0$.
\end{frame}


\begin{frame}{柯西积分公式: 注记}
\begin{enumerate}
\item 解析函数可以用一个积分
\[f(z)=\frac1{2\pi i}\oint_C\frac{f(\zeta)}{\zeta-z}\diff\zeta,\quad z\in D\]
来表示, 这是研究解析函数理论的强有力工具.
\item 求积分 $\displaystyle\oint_C g(z)\diff z$ 时, 如果 $g(z)$ 在 $C$ 内部只有一个奇点 $z_0$, 且 $g(z)(z-z_0)$ 解析, 那么我们就可以使用柯西积分公式来计算该积分.
\item 解析函数在闭路 $C$ 内部的取值完全由它在 $C$ 上的值所确定. 这也是解析函数的特征之一.

\onslide<+->
\indent
特别地, 解析函数在圆心处的值等于它在圆周上的平均值.
\onslide<+->
设 $z=z_0+Re^{i\theta}$, 则 $\diff z=iRe^{i\theta}\diff\theta$,
\onslide<+->
\[f(z_0)=\frac1{2\pi i}\oint_C\frac{f(z)}{z-z_0}\diff z=\frac1{2\pi}\int_0^{2\pi}f(z_0+Re^{i\theta})\diff\theta.\]
\end{enumerate}
\end{frame}


\begin{frame}{柯西积分公式: 证明}
\begin{proof}
由连续性可知, $\forall\varepsilon>0,\exists\delta>0$ 使得当 $|z-z_0|<\delta$ 时, $z\in D$ 且 $|f(z)-f(z_0)|<\varepsilon$.
\onslide<+->
设 $\Gamma:|z-z_0|=r<\delta$.
\onslide<+->
由复合闭路定理和长大不等式
\begin{align*}
&\peq\abs{\oint_C\frac{f(z)}{z-z_0}\diff z-2\pi i f(z_0)}
=\abs{\oint_\Gamma\frac{f(z)}{z-z_0}\diff z-2\pi i f(z_0)}\\
&\visible<+->{=\abs{\oint_\Gamma\frac{f(z)}{z-z_0}\diff z-\oint_\Gamma\frac{f(z_0)}{z-z_0}\diff z}
=\abs{\oint_\Gamma\frac{f(z)-f(z_0)}{z-z_0}\diff z}}\\
&\visible<+->{\le \frac\varepsilon r\cdot 2\pi r=2\pi \varepsilon.}
\end{align*}
\onslide<+->
由 $\varepsilon$ 的任意性可知 $\displaystyle\oint_C\frac{f(z)}{z-z_0}\diff z=2\pi i f(z_0)$.
\end{proof}
\end{frame}


\begin{frame}{典型例题: 柯西积分公式的应用}
\begin{example}
求 $\displaystyle\oint_{|z|=4}\frac{\sin z}z\diff z$.
\end{example}
\begin{solution}
由于函数 $\sin z$ 处处解析,
\onslide<+->
因此由柯西积分公式
\[\oint_{|z|=4}\frac{\sin z}z\diff z
=2\pi i \sin z|_{z=0}=0.\]
\end{solution}
\end{frame}


\begin{frame}[<*>]{例题: 柯西积分公式的应用}
\onslide<+->
\begin{example}
求 $\displaystyle\oint_{|z|=2}\frac{e^z}{z-1}\diff z$.
\end{example}

\onslide<+->
\begin{solution}
由于函数 $e^z$ 处处解析,
\onslide<+->
因此由柯西积分公式
\[\oint_{|z|=2}\frac{e^z}{z-1}\diff z
=2\pi i e^z|_{z=1}=2\pi ei.\]
\end{solution}

\onslide<+->
\begin{columns}
	\column{0.48\textwidth}
		\begin{exercise}
		求 $\displaystyle\oint_{|z|=2\pi}\frac{\cos z}{z-\pi}\diff z$.
		\end{exercise}\onslide<+->
	\column{0.48\textwidth}
		\begin{answer}
		$-2\pi i\vphantom{\displaystyle\oint_{|z|=2\pi}\frac{\cos z}{z-\pi}\diff z}$.
		\end{answer}
\end{columns}
\end{frame}


\begin{frame}{例题: 柯西积分公式的应用}
\beqskip{5pt}
\begin{example}
设 $f(z)=\displaystyle\oint_{|\zeta|=\sqrt3}\frac{3\zeta^2+7\zeta+1}{\zeta-z}\diff \zeta$, 求 $f'(1+i)$.
\end{example}
\begin{solution}
由柯西积分公式, 当 $|z|<\sqrt3$ 时,
\begin{align*}
f(z)&=\oint_{|\zeta|=\sqrt3}\frac{3\zeta^2+7\zeta+1}{\zeta-z}\diff \zeta\\
&\visible<+->{=2\pi i(3\zeta^2+7\zeta+1)|_{\zeta=z}=2\pi i(3z^2+7z+1).}
\end{align*}
\onslide<+->
因此
\vspace{-\baselineskip}
\[f'(z)=2\pi i(6z+7),\]
\vspace{-\baselineskip}
\onslide<+->
\[f'(1+i)=2\pi i(13+6i)=-12\pi+26\pi i.\]
\end{solution}
\onslide<+->
注意当 $|z|>\sqrt3$ 时, $f(z)\equiv0$.
\endgroup
\end{frame}


\begin{frame}{例题: 柯西积分公式的应用}
\begin{example}
求 $\displaystyle\oint_{|z|=3}\frac{e^z}{z(z^2-1)}\diff z$.
\end{example}
\begin{solution}
被积函数的奇点为 $0,\pm1$.
\onslide<+->
设 $C_1,C_2,C_3$ 分别为绕 $0,1,-1$ 的分离圆周.
\onslide<1->
\begin{center}
\begin{tikzpicture}
\draw[cstaxis] (-1.3,0)--(1.3,0);
\draw[cstaxis] (0,-1.1)--(0,1.25);
\draw[cstcurve,dcolorb] (0,0) circle (1);
\draw[cstcurve,dcolora,visible on=<3->] (0.5,0) circle(0.2);
\draw[cstcurve,dcolora,visible on=<3->] (-0.5,0) circle(0.2);
\draw[cstcurve,dcolora,visible on=<3->] (0,0) circle(0.2);
\fill[cstdot,dcolorb,visible on=<2->] (0,0) circle;
\fill[cstdot,dcolorb,visible on=<2->] (0.5,0) circle;
\fill[cstdot,dcolorb,visible on=<2->] (-0.5,0) circle;
\draw
  (1,0.8) node[dcolorb] {$C$}
  (-0.3,0.45) node[dcolora,visible on=<3->] {$C_1$}
  (0.5,-0.45) node[dcolora,visible on=<3->] {$C_2$}
  (-0.5,-0.45) node[dcolora,visible on=<3->] {$C_3$};
\end{tikzpicture}
\end{center}
\end{solution}
\end{frame}


\begin{frame}{例题: 柯西积分公式的应用}
\begin{solutionc}
由复合闭路定理和柯西积分公式
\begin{align*}
&\peq\oint_{|z|=3}\frac{e^z}{z(z^2-1)}\diff z
=\oint_{C_1+C_2+C_3}\frac{e^z}{z(z^2-1)}\diff z\\
&\visible<+->{=2\pi i\left[\frac{e^z}{z^2-1}\bigg|_{z=0}+\frac{e^z}{z(z+1)}\bigg|_{z=1}+\frac{e^z}{z(z-1)}\bigg|_{z=-1}\right]}\\
&\visible<+->{=2\pi i\left(-1+\frac e2+\frac{e^{-1}}2\right)=\pi i(e+e^{-1}-2).}
\end{align*}
\end{solutionc}
\end{frame}


\begin{frame}[<*>]{典型例题: 计算复变函数沿曲线的积分}
\onslide<+->
\begin{columns}
	\column{0.48\textwidth}
		\begin{thinking}
		对于闭路 $C$ 的外部, 是否有类似的柯西积分公式?
		\end{thinking}
		\onslide<+->
	\column{0.48\textwidth}
		\begin{tikzpicture}
		\fill[cstfille] (2.8,1.6) rectangle (-2.8,-1.6);
		\filldraw[cstcurve,dcolora,rounded corners=0.3cm,fill=white] (-0.6,0.6) rectangle (-1.3,-0.6);
		\draw[cstcurve,visible on=<4->,dcolorc] (0,0) circle (0.4);
		\draw[cstcurve,visible on=<4->,dcolorb] (0,0) circle (1.5);
		\fill[cstdot,dcolora] (0,0) circle;
		\draw
		  (0,0.2) node[dcolora] {$z_0$}
		  (-0.6,-0.8) node[dcolora] {$C$}
		  (0.7,-0.2) node[dcolorc,visible on=<4->] {$C_1$}
		  (0.9,0.8) node[dcolorb,visible on=<4->] {$C_2$};
		\end{tikzpicture}
\end{columns}
\onslide<+->
\begin{answer}
这时候我们要求 $f(\infty)=\lim\limits_{z\to\infty}f(z)$ 存在.
\onslide<+->
当 $z_0$ 在 $C$ 的外部时,
\[\oint_C\frac{f(z)}{z-z_0}\diff z
=\oint_{C_2}\frac{f(z)}{z-z_0}\diff z-\oint_{C_1}\frac{f(z)}{z-z_0}\diff z
=f(\infty)-f(z_0).\]
\onslide<+->
其中 $\displaystyle\oint_{C_2}\frac{f(z)}{z-z_0}\diff z=f(\infty)$ 可利用长大不等式证明.
\end{answer}
\end{frame}


\begin{frame}{高阶导数的柯西积分公式}
\onslide<+->
解析函数可以由它的积分所表示.
\onslide<+->
不仅如此, 通过积分表示, 还可以说明\alert{解析函数存在任意阶解析的导数}.

\begin{block}{柯西积分公式}
设函数 $f(z)$ 在闭路或复合闭路 $C$ 及其内部 $D$ 解析, 则对任意 $z_0\in D$,
\[f^{(n)}(z_0)=\frac{n!}{2\pi i}\oint_C\frac{f(z)}{(z-z_0)^{n+1}}\diff z.\]
\end{block}
\onslide<+->
其中右侧被积函数可以记忆成公式
\[f(z_0)=\frac{1}{2\pi i}\oint_C\frac{f(z)}{z-z_0}\diff z\]
右侧被积函数对 $z_0$ 求导 $n$ 次得到.
\end{frame}


\begin{frame}{高阶导数的柯西积分公式}
\beqskip{5pt}
\begin{proofs}
\indent
先证明 $n=1$ 的情形.
\onslide<+->
设 $\delta$ 为 $z_0$ 到 $C$ 的最短距离.
\onslide<+->
当 $|h|<\delta$ 时, $z_0+h\in D$.
\onslide<+->
由柯西积分公式,
\begin{align*}
f(z_0)&=\frac1{2\pi i}\oint_C\frac{f(z)}{z-z_0}\diff z,\\
f(z_0+h)&=\frac1{2\pi i}\oint_C\frac{f(z)}{z-z_0-h}\diff z.
\end{align*}
\onslide<+->
两式相减得到
\[\frac{f(z_0+h)-f(z_0)}h=\frac1{2\pi i}\emph{\oint_C\frac{f(z)}{(z-z_0)(z-z_0-h)}\diff z}.\]
\onslide<+->
当 $h\to 0$ 时, 左边的极限是 $f'(z_0)$.
\onslide<+->
因此我们只需要证明右边的极限等于 $\displaystyle\frac1{2\pi i}\emph{\oint_C\frac{f(z)}{(z-z_0)^2}\diff z}$.
\end{proofs}
\endgroup
\end{frame}


\begin{frame}{高阶导数的柯西积分公式}
\begin{proofe}
\indent
二者之差 $=\displaystyle\frac1{2\pi i}\oint_C\frac{h f(z)}{(z-z_0)^2(z-z_0-h)}\diff z$.
\onslide<+->
由于 $f(z)$ 在 $C$ 上连续, 故存在 $M$ 使得 $|f(z)|\le M$.
\onslide<+->
注意到 $z\in C$, $|z-z_0|\ge \delta$, $|z-z_0-h|\ge\delta-|h|$.
\onslide<+->
由长大不等式,
\[\abs{\oint_C\frac{h f(z)}{(z-z_0)^2(z-z_0-h)}\diff z}\le\frac{M|h|}{\delta^2(\delta-|h|)}\cdot L,\]
其中 $L$ 是闭路 $C$ 的长度.
\onslide<+->
当 $h\to0$ 时, 它的极限为 $0$, 因此 $n=1$ 情形得证.

\indent
\onslide<+->
对于一般的 $n$, 我们通过归纳法将 $f^{(n)}(z_0)$ 和 $f^{(n)}(z_0+h)$ 表达为积分形式.
\onslide<+->
然后利用长大不等式证明 $h\to 0$ 时, $\dfrac{f^{(n)}(z_0+h)-f^{(n)}(z_0)}h$ 趋于积分公式右侧.
\onslide<+->
具体过程省略.
\end{proofe}
\end{frame}


\begin{frame}{典型例题: 使用高阶导数的柯西积分公式计算积分}
\onslide<+->
\alert{柯西积分公式的作用不在于计算高阶导数, 而是用高阶导数来计算积分.}
\begin{example}
求 $\displaystyle\oint_{|z|=2}\frac{\cos(\pi z)}{(z-1)^5}\diff z.$
\end{example}
\begin{solution}
由于 $\cos(\pi z)$ 在 $|z|<2$ 处处解析,
\onslide<+->
因此由柯西积分公式,
\begin{align*}
\oint_{|z|=2}\frac{\cos(\pi z)}{(z-1)^5}\diff z
&=\frac{2\pi i}{4!}[\cos(\pi z)]^{(4)}\big|_{z=1}\\
&\visible<+->{=\frac{2\pi i}{24}\cdot \pi^4\cos \pi=-\frac{\pi^5 i}{12}.}
\end{align*}
\end{solution}
\end{frame}


\begin{frame}{典型例题: 使用高阶导数的柯西积分公式计算积分}
\begin{example}
求 $\displaystyle\oint_{|z|=2}\frac{e^z}{(z^2+1)^2}\diff z.$
\end{example}
\begin{solution}
$\dfrac{e^z}{(z^2+1)^2}$ 在 $|z|<2$ 的奇点为 $z=\pm i$.
\onslide<+->
取 $C_1,C_2$ 为以 $i,-i$ 为圆心的分离圆周.
\onslide<+->
由复合闭路定理,
\[\oint_{|z|=2}\frac{e^z}{(z^2+1)^2}\diff z
=\oint_{C_1}\frac{e^z}{(z^2+1)^2}\diff z
+\oint_{C_2}\frac{e^z}{(z^2+1)^2}\diff z.\]
\end{solution}
\end{frame}


\begin{frame}{典型例题: 使用高阶导数的柯西积分公式计算积分}
\begin{solutionc}
由柯西积分公式,
\begin{align*}
&\peq\oint_{C_1}\frac{e^z}{(z^2+1)^2}\diff z
=\frac{2\pi i}{1}\left[\frac{e^z}{(z+i)^2}\right]'\Big|_{z=i}\\
&\visible<+->{=2\pi i\left[\frac{e^z}{(z+i)^2}-\frac{2e^z}{(z+i)^3}\right]\Big|_{z=i}
=\frac{(1-i)e^i\pi}2.}
\end{align*}
\onslide<+->

类似地, $\displaystyle\oint_{C_2}\frac{e^z}{(z^2+1)^2}\diff z=\frac{-(1+i)e^{-i}\pi}2$.
\onslide<+->
故
\begin{align*}
\oint_{|z|=2}\frac{e^z}{(z^2+1)^2}\diff z
&=\frac{(1-i)e^i\pi}2+\frac{-(1+i)e^{-i}\pi}2\\
&=\pi i(\sin1-\cos1).
\end{align*}
\vspace{-20pt}
\end{solutionc}
\end{frame}


\begin{frame}{典型例题: 使用高阶导数的柯西积分公式计算积分}
\begin{example}
求 $\displaystyle\oint_{|z|=1}z^ne^z\diff z$, 其中 $n$ 是整数.
\end{example}
\begin{solution}
当 $n\ge 0$ 时, $z^ne^z$ 处处解析.
\onslide<+->
由柯西-古萨基本定理, 
\[\oint_{|z|=1}z^ne^z\diff z=0.\]

\onslide<+->
当 $n\le-1$ 时, $e^z$ 处处解析.
\onslide<+->
由柯西积分公式,
\[\oint_{|z|=1}z^ne^z\diff z
=\frac{2\pi i}{(-n-1)!}(e^z)^{(-n-1)}\big|_{z=0}
=\frac{2\pi i}{(-n-1)!}.\]
\end{solution}
\end{frame}


\begin{frame}{典型例题: 使用高阶导数的柯西积分公式计算积分}
\begin{example}
求 $\displaystyle\oint_{|z-3|=2}\frac1{(z-2)^2z^3}\diff z$ 和 $\displaystyle\oint_{|z-1|=2}\frac1{(z-2)^2z^3}\diff z$.
\end{example}
\begin{solution}
\enumnum1 $\dfrac1{(z-2)^2z^3}$ 在 $|z-3|<2$ 的奇点为 $z=2$.
\onslide<+->
由柯西积分公式,
\[\oint_{|z-3|=2}\frac1{(z-2)^2z^3}\diff z
	=\frac{2\pi i}{1!}\left(\frac1{z^3}\right)'\bigg|_{z=2}
	=-\frac{3\pi i}8.\]
\end{solution}
\end{frame}


\begin{frame}[<*>]{典型例题: 使用高阶导数的柯西积分公式计算积分}
\onslide<+->
\begin{solutionc}
\enumnum2 $\dfrac1{(z-2)^2z^3}$ 在 $|z-1|<3$ 的奇点为 $z=0,2$.
\onslide<+->
取 $C_1,C_2$ 分别为以 $0$ 和 $2$ 为圆心的分离圆周.
\onslide<+->
由复合闭路定理和柯西积分公式,
\onslide<+->
\begin{align*}
&\peq\oint_{|z-1|=3}\frac1{(z-2)^2z^3}\diff z=\oint_{C_1}\frac1{(z-2)^2z^3}\diff z+\oint_{C_2}\frac1{(z-2)^2z^3}\diff z\\
&\visible<+->{=\frac{2\pi i}{2!}\left[\frac1{(z-2)^2}\right]''\Big|_{z=0}+\frac{2\pi i}{1!}\left(\frac1{z^3}\right)'\Big|_{z=2}=0.}
\end{align*}
\end{solutionc}
\onslide<+->
\begin{columns}
	\column{0.48\textwidth}
		\begin{exercise}
		求 $\displaystyle\oint_{|z-2i|=3}\frac1{z^2(z-i)}\diff z$.
		\end{exercise}\onslide<+->
	\column{0.48\textwidth}
		\begin{answer}
		$0\vphantom{\displaystyle\oint_{|z-2i|=3}\frac1{z^2(z-i)}\diff z}$.
		\end{answer}
\end{columns}
\end{frame}


\begin{frame}{例题: 使用柯西积分公式证明莫累拉定理}
\begin{example}[莫累拉定理]
设 $f(z)$ 在单连通域 $D$ 内连续, 且对于 $D$ 中任意闭路 $C$ 都有 $\displaystyle\oint_Cf(z)\diff z=0$, 则 $f(z)$ 在 $D$ 内解析.
\end{example}
\onslide<+->
该定理可视作柯西-古萨基本定理的逆定理.
\begin{proof}
由题设可知 $f(z)$ 的积分与路径无关.
\onslide<+->
固定的 $z_0\in D$, 则
\[F(z)=\int_{z_0}^zf(z)\diff z\]
定义了 $D$ 内一个单值函数.
\onslide<+->
类似于原函数的证明可知 $F'(z)=f(z)$.
\onslide<+->
故 $f(z)$ 作为解析函数 $F(z)$ 的导数也是解析的.
\end{proof}
\end{frame}


\begin{frame}{解析函数与实函数的差异}
\onslide<+->
高阶柯西积分公式说明解析函数的导数与实函数的导数有何不同?
\onslide<+->
高阶柯西积分公式说明, 函数 $f(z)$ 只要在闭区域 $\ov D$ 中处处可导, 它就一定无限次可导, 并且各阶导数仍然在 $\ov D$ 中解析.
\onslide<+->
\alert{这一点与实变量函数有本质的区别.}

\onslide<+->
同时我们也可以看出, 如果一个二元实函数 $u(x,y)$ 是一个解析函数的实部或虚部, 则 $u$ 也是具有任意阶偏导数.
\onslide<+->
这便引出了调和函数的概念.
\end{frame}


