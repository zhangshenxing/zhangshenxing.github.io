\chapter{练习答案}


\begin{multicols}{2}

\setcounter{chapter}{0}

\begin{exerciseanswer}
  \exans $-4$.
  \exans $\pm(\sqrt3+\sqrt2\ii)$.
  \exans $1$.
  \exans 分别是 $-\ov z$ 和 $-z$.
  \exans $\dfrac{7-4\ii}5$.
  \exans 仅当 $z$ 不是负实数和 $0$ 时成立, 即 $\arg z\neq\cpi$.
  \exans $z_1,z_2,\cdots,z_n$ 中的非零元辐角都相等.
  \exans $\displaystyle z=2\sqrt3\Bigl(\cos\bigl(-\frac\cpi3\bigr)+\ii\sin\bigl(-\frac\cpi3\bigr)\Bigr)=2\sqrt3\ee^{-\frac{\cpi\ii}3}$, 辐角写成 $\dfrac{5\cpi}3$ 也可以.
  \exans $4+\ii$ 或 $2-\ii$.
  \exans $2^{48}$.
  \exans $\pm\dfrac{\sqrt3+\ii}2,\pm \ii,\pm\dfrac{\sqrt3-\ii}2$.
  \exans 双曲线 $x^2-y^2=\dfrac12$ 和双曲线 $xy=\dfrac14$.
  \begin{exansenum}
    \item 上半平面对应 $\Im z\ge0$.
    \item 下半平面对应 $\Im z\le0$.
    \item 左半平面对应 $\Re z\le0$.
    \item 右半平面对应 $\Re z\ge0$.
    \item 竖直带状区域对应 $x_1\le\Re z\le x_2$.
    \item 水平带状区域对应 $y_1\le\Im z\le y_2$.
    \item 角形区域对应 $\alpha_1\le \Arg z\le \alpha_2$ 以及原点.
    \item 圆域对应 $\abs{z}\le R$.
    \item 圆环域对应 $r\le\abs{z}\le R$.
  \end{exansenum}
  \exans C
  \begin{exansenum}
    \item $\Re z$ 的定义域为 $\BC$, 值域为 $\BR$.
    \item $\arg z$ 的定义域为 $\{z\in\BC\mid z\neq 0\}$, 值域为 $(-\cpi,\cpi]$.
    \item 当 $n>0$ 时, $z^n$ 的定义域为 $\BC$, 值域为 $\BC$.
    当 $n<0$ 时, $z^n$ 的定义域为 $\{z\in\BC\mid z\neq 0\}$, 值域为 $\{z\in\BC\mid z\neq 0\}$.
    \item $\dfrac{z+\ii}{z^2+1}$ 的定义域为 $\{z\in\BC\mid z\neq \pm i\}$, 值域为 $\BC$.
  \end{exansenum}
  \exans $-\dfrac dc$.
  \exans 收敛到 $0$.
\end{exerciseanswer}

\begin{exerciseanswer}
  \exans 处处不可导.
  \exans A
  \exans D
  \exans A
  \exans $\ln 2-\dfrac{2\cpi\ii}3$.
  \exans $\ln 3$.
\end{exerciseanswer}

\begin{exerciseanswer}
  \exans $-\dfrac12+\dfrac\ii2$.
  \begin{exanslist}(2)
    \item $0$.
    \item $0$.
  \end{exanslist}
  \exans $\cpi\ii$.
  \exans $\sin1-\cos 1$.
  \exans $-2\cpi\ii$.
  \exans $2\cpi\ii$.
  \exans $2uv+C,C\in\BR$.
\end{exerciseanswer}

\begin{exerciseanswer}
  \exans 当且仅当非零的 $z_n$ 的辐角全都相同时成立.
  \exans 条件收敛.
  \exans $\dfrac{\sqrt2}2$.
  \exans 收敛半径为 $1$, 和函数为 $-\ln(1-z)$.
  \exans $\sumf0(2^{n+1}-1)z^n,\quad \abs{z}<\dfrac12$.
  \exans $\sumf1 \dfrac{2n-1}{z^n}$.
  \exans $-1$ 是一阶极点, $1$ 是二阶极点.
  \exans $1$ 是二阶极点, $0$ 是一阶极点, $-1$ 是三阶极点.
  \exans $z=2k\cpi\ii $ 是一阶极点, $k\neq 0,\pm1$; $z=0$ 是四阶极点; $z=\pm 2\cpi\ii $ 是可去奇点; $z=\infty$ 不是孤立奇点.
\end{exerciseanswer}

\begin{exerciseanswer}
  \exans $\dfrac1{24}$.
  \exans $4\ii$.
  \exans $2\cpi\ii$.
  \exans $4$ 个.
\end{exerciseanswer}

\begin{exerciseanswer}
  \exans D
  \exans $w=\dfrac{4z}{z-1}$.
  \exans 先通过幂函数 $s=\sqrt z$ 将其映成上半圆域, 再通过分式线性变换 $t=\ii\dfrac{1-s}{1+s}$ 将其映成第一象限区域, 再通过幂函数 $w=t^2$ 将其映成 $\BH$.
  \exans 下半平面 $\Im w<0$.
\end{exerciseanswer}

\begin{exerciseanswer}
  \exans $-2\cpi\dirac(\omega+\omega_0)$.
  \exans $\ii\omega$.
  \exans $\dfrac{k}{s^2+4k^2}$.
  \exans $1-e^{-t}$.
\end{exerciseanswer}

\end{multicols}
