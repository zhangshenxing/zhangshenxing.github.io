\section{复数的乘除、方幂与方根}


\begin{frame}{复数的乘除与三角/指数表示}
\onslide<+->
三角形式和指数形式在进行复数的乘法、除法和幂次计算中非常方便.
\begin{theorem}
设
\[z_1=r_1(\cos\theta_1+i\sin\theta_1)=r_1e^{i\theta_1},\]
\[z_2=r_2(\cos\theta_2+i\sin\theta_2)=r_2e^{i\theta_1}\neq 0,\]
则
\[\markatt{z_1z_2=r_1r_2[\cos(\theta_1+\theta_2)+i\sin(\theta_1+\theta_2)]=r_1r_2e^{i(\theta_1+\theta_2)}},\]
\[\markatt{\frac{z_1}{z_2}=\frac{r_1}{r_2}[\cos(\theta_1-\theta_2)+i\sin(\theta_1-\theta_2)]=\frac{r_1}{r_2}e^{i(\theta_1-\theta_2)}}.\]
\end{theorem}
\end{frame}


\begin{frame}{复数的乘除与三角/指数表示}
\onslide<+->
换言之,
\[\markatt{|z_1z_2|=|z_1|\cdot|z_2|,\quad\abs{\frac{z_1}{z_2}}=\frac{|z_1|}{|z_2|},}\]
\onslide<+->
\[\markatt{\Arg(z_1z_2)=\Arg z_1+\Arg z_2,\quad
\Arg\left(\frac{z_1}{z_2}\right)=\Arg z_1-\Arg z_2.}\]
\onslide<+->
关于多值函数的等式的含义是指: 两边所能取到的值构成的集合相等.
\onslide<+->
例如此处关于辐角的等式的含义是:
\[\Arg(z_1z_2)=\set{\theta_1+\theta_2:\theta_1\in\Arg z_1,\theta_2\in\Arg z_2}.\]
\[\Arg\left(\frac{z_1}{z_2}\right)=\set{\theta_1-\theta_2:\theta_1\in\Arg z_1,\theta_2\in\Arg z_2}.\]
\end{frame}


\begin{frame}{复数的乘除与三角/指数表示}
\onslide<+->
注意上述等式中 $\Arg$ 不能换成 $\arg$, 也就是说
\[\arg(z_1z_2)=\arg z_1+\arg z_2,\quad
\arg\left(\frac{z_1}{z_2}\right)=\arg z_1-\arg z_2\]
\markatt{不一定成立}.
\onslide<+->
这是因为 $\arg z_1\pm\arg z_2$ 有可能不落在区间 $(-\pi,\pi]$ 上.
\onslide<+->
例如
\[(-1+i)(-1+i)=2i,\]
\vspace{-\baselineskip}
\onslide<+->
\[\arg(-1+i)+\arg(-1+i)=\frac{3\pi}4+\frac{3\pi}4=\frac{3\pi}2,\]
\vspace{-\baselineskip}
\onslide<+->
\[\arg(-2i)=-\frac\pi2.\]
\end{frame}


\begin{frame}{复数的乘除与三角/指数表示}
\begin{proof}
\vspace{-\baselineskip}	
\begin{align*}
z_1z_2&=r_1(\cos\theta_1+i\sin\theta_1)\cdot
r_2(\cos\theta_2+i\sin\theta_2)\\
&\visible<+->{=r_1r_2\bigl[(\cos\theta_1\cos\theta_2-\sin\theta_1\sin\theta_2)}\\
&\visible<.->{\qquad+i(\cos\theta_1\sin\theta_2+\sin\theta_1\cos\theta_2)\bigr]}\\
&\visible<+->{=r_1r_2\bigl[\cos(\theta_1+\theta_2)+i\sin(\theta_1+\theta_2)\bigr]}
\end{align*}
\onslide<+->
因此乘法情形得证.

\indent\onslide<+->
设 $\dfrac{z_1}{z_2}=re^{i\theta}$,
\onslide<+->
则由乘法情形可知
\vspace{-4pt}
\[rr_2=r_1,\quad \theta+\Arg z_2=\Arg z_1.\]
\onslide<+->
因此 $r=\dfrac{r_1}{r_2},\theta=\theta_1-\theta_2+2k\pi$, 其中 $k\in\BZ$.
\end{proof}
\end{frame}


\begin{frame}{乘积的几何意义}
\onslide<+->
从该定理可以看出, 乘以复数 $z=re^{i\theta}$ 可以理解为模放大为 $r$ 倍, 并按逆时针旋转角度 $\theta$.
\onslide<+->
\begin{center}
\begin{tikzpicture}
\draw[cstaxis] (-2,0)--(3,0);
\draw[cstaxis] (0,-0.4)--(0,4);
\coordinate (A) at (2,0);
\coordinate (B) at (0,0);
\coordinate (C) at (1.2,2.4);
\draw[cstcurve,dcolora,cstarrowto] (B)--(A);
\draw[cstcurve,dcolora,cstarrowto] (B)--(C);
\draw[cstcurve,dcolora] pic [draw, "$\theta$", angle eccentricity=1.4] {angle};
\coordinate (A) at (1.3,1.6);
\coordinate (C) at (-1.14,2.52);
\draw[cstcurve,dcolorb,cstarrowto,visible on =<3->] (B)--(A);
\draw[cstcurve,dcolorb,cstarrowto,visible on =<3->] (B)--(C);
\draw[cstcurve,dcolorb,visible on =<3->] pic [draw, "$\theta$", angle eccentricity=1.3, angle radius= 0.7cm] {angle};
\draw 
  (1.9,2.6) node[dcolora] {$z=re^{i\theta}$}
  (2,-0.3) node[dcolora] {$1$}
  (1.6,1.5) node[dcolorb,visible on =<3->] {$z_1$}
  (-1.1,2.7) node[dcolorb,visible on =<3->] {$zz_1$};
\end{tikzpicture}
\end{center}
\end{frame}


\begin{frame}{例题: 复数解决平面几何问题}
\begin{example}
已知正三角形的两个顶点为 $z_1=1$ 和 $z_2=2+i$, 求它的另一个顶点.
\end{example}
\begin{solution}
由于 $\overrightarrow{Z_1Z_3}$ 为 $\overrightarrow{Z_1Z_2}$ 顺时针或逆时针旋转 $\dfrac\pi3$,
\begin{center}
\begin{tikzpicture}
\draw[cstaxis] (-0.3,0)--(2.8,0);
\draw[cstaxis] (0,-0.5)--(0,1.4);
\draw[cstcurve,dcolora] (1,0)--(2,1)--(0.634,1.366)--cycle;
\draw[cstdash,dcolora] (1,0)--(2.366,-0.366)--(2,1);
\draw 
  (1.0,-0.2) node[dcolora] {$z_1$}
  (2.3,1) node[dcolora] {$z_2$}
  (0.4,1.2) node[dcolora] {$z_3$}
  (2.6,-0.3) node[dcolora] {$z_3'$};
\end{tikzpicture}
\end{center}
\end{solution}
\end{frame}


\begin{frame}{例题: 复数解决平面几何问题}
\begin{solutionc}
因此
\begin{align*}
z_3-z_1&=(z_2-z_1)\exp\left(\pm\frac{\pi i}3\right)
\visible<+->{=(1+i)\left(\frac12\pm\frac{\sqrt3}2i\right)}\\
&\visible<+->{=\frac{1-\sqrt3}2+\frac{1+\sqrt3}2i\ \text{或}\ \frac{1+\sqrt3}2+\frac{1-\sqrt3}2i,}
\end{align*}
\onslide<+->
\[z_3=\frac{3-\sqrt3}2+\frac{1+\sqrt3}2i\ \text{或}\ \frac{3+\sqrt3}2+\frac{1-\sqrt3}2i.\]
\end{solutionc}
\end{frame}


\begin{frame}{复数的乘幂}
\onslide<+->
设 $z=r(\cos\theta+i\sin\theta)=re^{i\theta}\neq0$.
\onslide<+->
根据复数三角形式的乘法和除法运算法则, 我们有
\[\eqatt{z^n=r^n(\cos{n\theta}+i\sin{n\theta})=r^ne^{in\theta}},\quad\forall n\in\BZ.\]
\onslide<+->
特别地, 当 $r=1$ 时, 我们得到\markdef{棣莫弗(De Moivre)公式}
\[\marknot{(\cos\theta+i\sin\theta)^n=\cos{n\theta}+i\sin{n\theta}.}\]
\end{frame}


\begin{frame}{复数的乘幂*}
\onslide<+->
对棣莫弗公式左侧进行二项式展开
\onslide<+->
\begin{align*}
\cos(n\theta)&=\Re\left[
\sum_{0\le k\le n}\mathrm{C}_n^k \cos^{n-k}\theta(i\sin\theta)^k\right]\\
&\visible<+->{=\sum_{0\le 2k\le n}\mathrm{C}_n^{2k} \cos^{n-2k}\theta(-\sin^2\theta)^k}\\
&\visible<+->{=\sum_{0\le k\le\frac n2}\mathrm{C}_n^{2k} \cos^{n-2k}\theta(\cos^2\theta-1)^k.}
\end{align*}
\onslide<+->
因此 $\cos{n\theta}$ 是 $\cos\theta$ 的多项式.
\onslide<+->
这个多项式
\[g_n(T)=\sum_{0\le k\le\frac n2}\mathrm{C}_n^{2k} T^{n-2k} (T^2-1)^k.\]
叫做切比雪夫多项式.
\onslide<+->
它在计算数学的逼近理论中有着重要作用.
\end{frame}


\begin{frame}{典型例题: 复数乘幂的计算}
\begin{example}
求 $(1+i)^n+(1-i)^n$.
\end{example}
\begin{solution}
由于
\[1+i=\sqrt2\left(\cos\frac\pi4+i\sin\frac\pi4\right),\quad
1-i=\sqrt2\left(\cos\frac\pi4-i\sin\frac\pi4\right),\]
\onslide<+->
因此
\vspace{-\baselineskip}
\begin{align*}
&\peq(1+i)^n+(1-i)^n\\
&=2^{\frac n2}\left(\cos\frac{n\pi}4+i\sin\frac{n\pi}4+\cos\frac{n\pi}4-i\sin\frac{n\pi}4\right)\\
&\visible<+->{=2^{\frac n2+1}\cos\frac{n\pi}4.}
\end{align*}
\vspace{-\baselineskip}
\end{solution}
\end{frame}


\begin{frame}{典型例题: 复数乘幂的计算}
\begin{exercise}
求 $(\sqrt3+i)^{2022}$.
\end{exercise}
\begin{answer}
$-2^{2022}$.
\end{answer}
\end{frame}


\begin{frame}{复数的方根}
\onslide<+->
我们利用复数方幂公式来计算复数 $z$ 的 \markdef{$n$ 次方根 $\sqrt[n]z$}.
\onslide<+->
设
\[w^n=z=r\exp(i\theta)\neq0,\quad w=\rho\exp(i\varphi),\]
\onslide<+->
则
\[w^n=\rho^n(\cos{n\varphi}+i\sin{n\varphi})=r(\cos\theta+i\sin\theta).\]
\onslide<+->
比较两边的模可知 $\rho^n=r,\rho=\sqrt[n]r$.

\onslide<+->
为了避免记号冲突, 当 $r$ 是正实数时, $\sqrt[n]r$ 默认表示 $r$ 的唯一的 $n$ 次正实根, 称之为\markdef{算术根}.
\onslide<+->
由于 $n\varphi$ 和 $\theta$ 的正弦和余弦均相等, 因此存在整数 $k$ 使得
\[n\varphi=\theta+2k\pi,\quad \varphi=\frac{\theta+2k\pi}n.\]
\end{frame}


\begin{frame}{复数的方根}
\onslide<+->
故
\begin{align*}
&\eqatt{w=w_k=\sqrt[n]r\exp\frac{(\theta+2k\pi)i}n}\\
&\hspace{42pt}\eqatt{=\sqrt[n]r\left(\cos\frac{\theta+2k\pi}n+i\sin\frac{\theta+2k\pi}n\right).}
\end{align*}
\onslide<+->
不难看出, $w_k=w_{k+n}$, 而 $w_0,w_1,\dots,w_{n-1}$ 两两不同.
\onslide<+->
因此只需取 \boxatt{$k=0,1,\dots,n-1$}.
\onslide<+->
故\marknot{任意一个非零复数的 $n$ 次方根有 $n$ 个值}.

\onslide<+->
这些根的模都相等, 且 $w_k$ 和 $w_{k+1}$ 辐角相差 $\dfrac{2\pi}n$.
\onslide<+->
因此\marknot{它们是以原点为中心, $\sqrt[n]r$ 为半径的圆的正接 $n$ 边形的顶点}.
\end{frame}


\begin{frame}{方幂和方根的辐角等式}
\onslide<+->
注意当 $|n|\ge 2$ 时, \markatt{$\Arg(z^n)=n\Arg z$ 不成立}.
\onslide<+->
这是因为
\[\Arg(z^n)=n\arg z+2k\pi i,\quad k\in\BZ,\]
\[n\Arg z=n\arg z+2nk\pi i,\quad k\in\BZ.\]

\onslide<+->
我们总有
\[\markatt{\Arg \sqrt[n]z=\dfrac1n\Arg z}=\dfrac{\arg z+2k\pi}n,\quad k\in\BZ.\]
\end{frame}

\begin{frame}{典型例题: 复数方根的计算}
\beqskip{6pt}
\begin{example}
求 $\sqrt[4]{1+i}$.
\end{example}
\begin{solution}
由于 $1+i=\sqrt2\exp\left(\dfrac{\pi i}4\right)$,
\onslide<+->
因此
\[\sqrt[4]{1+i}=\sqrt[8]2\exp\left[\frac{(\frac\pi4+2k\pi)i}4\right],\quad k=0,1,2,3.\]
\onslide<+->
所以该方根所有值为
\[w_0=\sqrt[8]2\exp\frac{\pi i}{16},\qquad
w_1=\sqrt[8]2\exp\frac{9\pi i}{16},\]
\[w_2=\sqrt[8]2\exp\frac{17\pi i}{16},\qquad
w_3=\sqrt[8]2\exp\frac{25\pi i}{16}.\]
\end{solution}
\endgroup
\end{frame}


\begin{frame}{典型例题: 复数方根的计算}
\onslide<+->
$w_0,w_1=iw_0,w_2=-w_0,w_3=-iw_0$ 形成了一个正方形.
\onslide<+->
\begin{center}
\begin{tikzpicture}
\draw[cstaxis] (-2,0)--(2,0);
\draw[cstaxis] (0,-1.8)--(0,1.8);
\draw[cstcurve,dcolora,cstarrowto] (0,0)--(1.5,0.3);
\draw[cstcurve,dcolora,cstarrowto] (0,0)--(-1.5,-0.3);
\draw[cstcurve,dcolora,cstarrowto] (0,0)--(-0.3,1.5);
\draw[cstcurve,dcolora,cstarrowto] (0,0)--(0.3,-1.5);
\draw[cstcurve,dcolorb] (1.5,0.3)--(-0.3,1.5)--(-1.5,-0.3)--(0.3,-1.5)--cycle;
\draw
  (-0.2,-0.3) node {$0$}
  (1.8,0.3) node[dcolora] {$w_0$}
  (-0.8,1.3) node[dcolora] {$w_1$}
  (-1.8,-0.3) node[dcolora] {$w_2$}
  (0.3,-1.7) node[dcolora] {$w_3$};
\end{tikzpicture}
\end{center}
\end{frame}


\begin{frame}{典型例题: 复数方根的计算}
\begin{exercise}
求 $\sqrt[6]{-1}$.
\end{exercise}
\begin{answer}
$\pm\dfrac{\sqrt3+i}2,\pm i,\pm\dfrac{\sqrt3-i}2$.
\end{answer}

\begin{think}
$i=\sqrt{-1}$ 吗?
\end{think}
\begin{answer}
$\sqrt{-1}$ 是多值的, 此时 $\sqrt{-1}=\pm i$.
\onslide<+->
除非给定单值分支 $\sqrt z=\sqrt{|z|}\exp\left(\frac{i\arg z}2\right)$, 否则不能说 $\sqrt{-1}=i$.
\end{answer}
\end{frame}


\begin{frame}{三次方程的求根问题*}
\onslide<+->
现在我们来看三次方程 $x^3-3px-2q=0$ 的根.
\onslide<+->
\[x=u+\frac pu,\quad
u^3=q+\sqrt{\Delta},\quad 
\Delta=q^2-p^3.\]
\begin{enumerate}
\item 如果 $\Delta>0$, 设 $\alpha=\sqrt[3]{q+\sqrt{\Delta}}$ 是算术根.
\onslide<+->
则
\[x=\alpha+\frac p\alpha,\quad \alpha\omega+\frac p\alpha \omega^2,\quad \alpha\omega^2+\frac p\alpha \omega.\]
\onslide<+->
容易证明后两个根都是虚数.
\item 如果 $\Delta<0$, 则 $p>0$.
\onslide<+->
设 $\sqrt[3]{q+\sqrt{\Delta}}=u_1,u_2,u_3$,
\onslide<+->
则 $u_i$ 都是虚数, 且
\[|u_i|^6=\abs{q+\sqrt{\Delta}}^2=p^3,\quad u_i\ov u_i=|u_i|^2=p.\]
\onslide<+->
于是我们得到 $3$ 个实根 $x=u_i+\overline{u_i}$.
\end{enumerate}
\end{frame}


