\section{幂级数}


\subsection{幂级数的收敛域}
\begin{frame}{函数项级数与幂级数}
	\onslide<+->
	复变函数级数与实变量函数级数也是类似的.
	
	\onslide<+->
	\begin{definition}
		\begin{itemize}
			\item 设 $\{f_n(z)\}_{n\ge 1}$ 是一个复变函数列, 其中每一项都在区域 $D$ 上有定义.
			表达式 $\suml_{n=1}^\infty f_n(z)$ 称为\emph{复变函数项级数}.
			\item 对于 $z_0\in D$, 如果级数 $\suml_{n=1}^\infty f_n(z_0)$ 收敛, 则称 \emph{$\suml_{n=1}^\infty f_n(z)$ 在 $z_0$ 处收敛}, 相应级数的值称为它的\emph{和}.
			\item 如果 $\suml_{n=1}^\infty f_n(z)$ 在 $D$ 上处处收敛, 则它的和是一个函数, 称为\emph{和函数}.
			\item 称形如 $\suml_{n=0}^\infty c_n(z-a)^n$ 的函数项级数为\emph{幂级数}.
		\end{itemize}
	\end{definition}

	\onslide<+->
	我们只需要考虑 $a=0$ 情形的幂级数, 因为二者的收敛范围与和函数只是差一个平移.
\end{frame}


\begin{frame}{阿贝尔定理}
	\beqskip{0pt}
	\vspace{-3pt}
	\onslide<+->
	\begin{block}{阿贝尔定理}
		\begin{enumerate}
			\item 如果 $\suml_{n=0}^\infty c_nz^n$ 在 $z_0\neq 0$ 处收敛, 那么对任意 $|z|<|z_0|$ 的 $z$, 该级数必绝对收敛.
			\item 如果 $\suml_{n=0}^\infty c_nz^n$ 在 $z_0\neq 0$ 处发散, 那么对任意 $|z|>|z_0|$ 的 $z$, 该级数必发散.
		\end{enumerate}
	\end{block}
	
	\onslide<+->
	\begin{proof}
		\onslide<+->{\enumnum1 因为级数收敛, 所以 $\lim\limits_{n\to\infty}c_n z_0^n=0$.
		}\onslide<+->{故存在 $M$ 使得 $|c_nz_0^n|<M$.
		}\onslide<+->{对于 $|z|<|z_0|$,
			\[\sum_{n=0}^\infty|c_nz^n|=\sum_{n=0}^\infty|c_nz_0^n|\cdot\abs{\frac z{z_0}}^n
			\visible<+->{\le M\sum_{n=0}^\infty\abs{\frac z{z_0}}^n
			=\frac{M}{1-\abs{\dfrac z{z_0}}}.}\]
		}\onslide<+->{所以级数在 $z$ 处绝对收敛.
		}\onslide<+->{\enumnum2是\enumnum1的逆否命题.\qedhere}
	\end{proof}
	\endgroup
\end{frame}


\begin{frame}{幂级数的收敛半径}
	\onslide<+->
	设 $R$ 是实幂级数 $\suml_{n=0}^\infty|c_n|x^n$ 的收敛半径.
	\begin{itemize}
		\item 如果 $R=+\infty$, 由阿贝尔定理可知 $\suml_{n=0}^\infty c_nz^n$ 处处绝对收敛.
		\item 如果 $0<R<+\infty$, 那么 $\suml_{n=0}^\infty c_nz^n$ 在 $|z|<R$ 上绝对收敛, 在 $|z|>R$ 上发散.
		\item 如果 $R=0$, 那么 $\suml_{n=0}^\infty c_nz^n$ 仅在 $z=0$ 处收敛, 对任意 $z\neq 0$ 都发散.
	\end{itemize}
	\onslide<+->
	我们称 $R$ 为该幂级数的\emph{收敛半径}.
	
	\onslide<+->
	\begin{center}
		\begin{tikzpicture}
			\filldraw[cstcurve,dcolora,cstfill] (0,0) circle (1.2);
			\fill[cstdot] (0,0) circle;
			\draw[cstcurve,cstarrowto] (0,0)--(0.96,0.72);
			\draw[cstcurve,cstarrowto] (-0.96,0.72)--(-2,0.72);
			\draw
				(0.6,0.1) node {$R$}
				(0,-0.4) node[dcolorb] {绝对收敛}
				(2.5,-0.4) node[dcolorb] {发散}
				(-3,0.72) node[dcolora] {都有可能};
		\end{tikzpicture}
	\end{center}
\end{frame}


\begin{frame}{例题: 收敛半径的计算}
	\onslide<+->
	\begin{example}
		求幂级数 $\suml_{n=0}^\infty z^n=1+z+z^2+\cdots$ 的收敛半径与和函数.
	\end{example}

	\onslide<+->
	\begin{solution}
		如果幂级数收敛, 则由 $z^n\to0$ 可知 $|z|<1$.
		\onslide<+->{当 $|z|<1$ 时, 和函数为
			\[\lim_{n\to\infty}s_n=\lim_{n\to\infty}\frac{1-z^{n+1}}{1-z}=\frac1{1-z}.\]
		}\onslide<+->{因此收敛半径为 $1$.}
	\end{solution}
\end{frame}


\subsection{收敛半径的计算}
\begin{frame}{收敛半径的计算}
	\onslide<+->
	由正项级数的相应判别法容易得到公式 $R=\dfrac1r$, 其中
	\begin{enumerate}
		\item \emph{达朗贝尔公式(比值法)}: \abox{$r=\displaystyle\lim_{n\to\infty}\abs{\frac{c_{n+1}}{c_n}}$} (假设存在);
		\item \emph{柯西公式(根式法)}: $r=\displaystyle\lim_{n\to\infty}\sqrt[n]{|c_n|}$ (假设存在);
		\item \emph{柯西-阿达马公式}: $r=\displaystyle\ov{\lim_{n\to\infty}} \sqrt[n]{|c_n|}$.
	\end{enumerate}
	\onslide<+->
	如果 $r=0$ 或 $+\infty$, 则 $R=+\infty$ 或 $0$.
\end{frame}


\begin{frame}{典型例题: 收敛半径的计算}
	\onslide<+->
	\begin{example}
		求幂级数 $\displaystyle\sum_{n=1}^\infty\frac{(z-1)^n}n$ 的收敛半径, 并讨论 $z=0,2$ 的情形.
	\end{example}

	\onslide<+->
	\begin{solution}
		由 $\displaystyle\lim_{n\to\infty}\abs{\frac{c_{n+1}}{c_n}}=\lim_{n\to\infty}\frac n{n+1}=1$ 可知收敛半径为 $1$.

		\onslide<+->{当 $z=2$ 时, $\displaystyle\sum_{n=1}^\infty\frac{(z-1)^n}n=\sum_{n=1}^\infty\frac1n$ 发散.}

		\onslide<+->{当 $z=0$ 时, $\displaystyle\sum_{n=1}^\infty\frac{(z-1)^n}n=\sum_{n=1}^\infty\frac{(-1)^n}n$ 收敛.}
	\end{solution}

	\onslide<+->
	事实上, \alert{收敛圆周上既可能处处收敛, 也可能处处发散, 也可能既有收敛的点也有发散的点}.
\end{frame}


\begin{frame}{典型例题: 收敛半径的计算}
	\beqskip{5pt}
	\onslide<+->
	\begin{example}
		求幂级数 $\suml_{n=0}^\infty\cos(in)z^n$ 的收敛半径.
	\end{example}

	\onslide<+->
	\begin{solution}
		我们有 $c_n=\cos(in)=\dfrac{e^n+e^{-n}}2$.
		\onslide<+->{由
			\[\lim_{n\to\infty}\abs{\frac{c_{n+1}}{c_n}}=\lim_{n\to\infty}\frac{e^{n+1}+e^{-n-1}}{e^n+e^{-n}}=e\lim_{n\to\infty}\frac{1+e^{-2n-2}}{1+e^{-2n}}=e\]
		可知收敛半径为 $\frac1e$.}
	\end{solution}

	\onslide<+->
	\begin{exercise}
		幂级数 $\suml_{n=0}^\infty(1+i)^nz^n$ 的收敛半径为\fillblank[2cm]{\visible<+->{$\sqrt2/2$}}.
	\end{exercise}
	\endgroup
\end{frame}


% \begin{frame}{典型例题: 收敛半径的计算*}
% \onslide<+->
% \begin{example}
% 求幂级数 $\displaystyle\sum_{n=1}^\infty\frac{z^n}{n^p}$ 的收敛半径并讨论在收敛圆周上的情形, 其中 $p\in\BR$.
% \end{example}
% \onslide<+->
% \begin{solution}
% 由 $\displaystyle\lim_{n\to\infty}\abs{\frac{c_{n+1}}{c_n}}=\lim_{n\to\infty}\left(\frac n{n+1}\right)^p=1$ 可知收敛半径为 $1$.
% \onslide<+->{设 $|z|=1$.
% \begin{itemize}
% \item 若 $p>1$, $\displaystyle\sum_{n=1}^\infty\abs{\frac{z^n}{n^p}}=\sum_{n=1}^\infty\frac1{n^p}$ 收敛,
% \onslide<+->{原级数在收敛圆周上处处绝对收敛.}
% \item 若 $p\le 0$, $\abs{\dfrac{z^n}{n^p}}=\dfrac1{n^p}\not\to0$,
% \onslide<+->{原级数在收敛圆周上处处发散.}
% \end{itemize}}
% \end{solution}
% \end{frame}


% \begin{frame}{典型例题: 收敛半径的计算*}
% \onslide<+->
% 回忆\emph{狄利克雷判别法}: 若 $\set{a_n}_{n\ge 1}$ 部分和有界, 实数项数列 $\set{b_n}_{n\ge 1}$ 单调趋于 $0$, 则 $\suml_{n=1}^\infty a_nb_n$ 收敛.

% \onslide<+->
% \begin{solutionc}
% \begin{itemize}
% \item 若 $0<p\le1$, $\displaystyle\sum_{n=1}^\infty\frac1{n^p}$ 发散, 
% \onslide<+->{而在收敛圆周上其它点 $z\neq1$ 处,
% \[|z+z^2+\cdots+z^n|=\abs{\frac{z(1-z^n)}{1-z}}
% \le\frac{2}{|1-z|}\]
% 有界, 数列 $\set{n^{-p}}_{n\ge 1}$ 单调趋于 $0$,}
% \onslide<+->{因此 $\displaystyle\sum_{n=1}^\infty\frac{z^n}{n^p}$ 收敛.}
% \onslide<+->{故该级数在 $z=1$ 发散, 在收敛圆周上其它点收敛.}
% \end{itemize}
% \end{solutionc}
% \end{frame}


\subsection{幂级数的运算性质}
\begin{frame}{幂级数的有理运算}
	\onslide<+->
	\begin{theorem}
		设幂级数
		\[f(z)=\sum_{n=0}^\infty a_nz^n,|z|<R_1,\quad
		g(z)=\sum_{n=0}^\infty b_nz^n,|z|<R_2.\]
		\onslide<+->{那么当 $|z|<R=\min\{R_1,R_2\}$ 时,
		\[(f\pm g)(z)=\sum_{n=0}^\infty (a_n\pm b_n)z^n,\quad
		(fg)(z)=\sum_{n=0}^\infty\left(\sum_{k=0}^na_kb_{n-k}\right)z^n.\]}
	\end{theorem}
	
	\onslide<+->
	当 $f,g$ 的收敛半径相同时, $f\pm g$ 或 $fg$ 的收敛半径可以比 $f,g$ 的大.
\end{frame}


\begin{frame}{幂级数的代换运算}
	\onslide<+->
	\begin{theorem}
		设幂级数
		\[f(z)=\sum_{n=0}^\infty a_nz^n,|z|<R,\]
		设函数 $\varphi(z)$ 在 $|z|<r$ 上解析且 $|\varphi(z)|<R$, 
		\onslide<+->{那么当 $|z|<r$ 时,
		\[f[\varphi(z)]=\sum_{n=0}^\infty a_n[\varphi(z)]^n.\]}
	\end{theorem}
\end{frame}


\begin{frame}{幂级数的解析性质}
	\onslide<+->
	\begin{theorem}
		设幂级数 $\suml_{n=0}^\infty c_nz^n$ 的收敛半径为 $R$, 则在 $|z|<R$ 上:
		\begin{enumerate}
			\item 它的和函数 $f(z)=\suml_{n=0}^\infty c_nz^n$ 解析,
			\item $f'(z)=\suml_{n=1}^\infty nc_nz^{n-1}$,
			\item $\displaystyle\int_0^zf(z)\diff z=\sum_{n=0}^\infty \frac{c_n}{n+1}z^{n+1}$.
		\end{enumerate}
	\end{theorem}

	\onslide<+->
	也就是说, \alert{在收敛圆内, 幂级数的和函数解析, 且可以逐项求导, 逐项积分}.
\end{frame}


\begin{frame}{幂级数的解析性质}
	\onslide<+->
	由于和函数在 $|z|>R$ 上没有定义, 因此我们不能谈和函数在 $|z|=R$ 上的解析性.

	\onslide<+->
	如果函数 $g(z)$ 在该幂级数收敛的点处和 $f(z)$ 均相同, 则 $g(z)$ \alert{一定在收敛圆周上有奇点}.
	\onslide<+->
	这是因为一旦 $g(z)$ 在收敛圆周上处处解析, 该和函数就可以在一个半径更大的圆域上作泰勒展开.
\end{frame}


\begin{frame}{例题: 幂级数展开}
	\onslide<+->
	\begin{example}
		把函数 $\dfrac1{z-b}$ 表成形如 $\suml_{n=0}^\infty c_n(z-a)^n$ 的幂级数, 其中 $a\neq b$.
	\end{example}
	
	\onslide<+->
	\begin{solution}
		\[\frac1{z-b}=\frac1{(z-a)-(b-a)}
		\visible<+->{=\frac1{a-b}\cdot\frac1{1-\dfrac{z-a}{b-a}}.}\]
		\onslide<+->{当 $|z-a|<|b-a|$ 时,
		}\onslide<+->{$\displaystyle\frac1{z-b}=\frac1{a-b}\sum_{n=0}^\infty\left(\frac{z-a}{b-a}\right)^n$,
		}\onslide<+->{即
		\[\frac1{z-b}=-\sum_{n=0}^\infty\frac{(z-a)^n}{(b-a)^{n+1}},\quad|z-a|<|b-a|.\]}
	\end{solution}
\end{frame}


\begin{frame}{典型例题: 幂级数的收敛半径与和函数}
	\onslide<+->
	\begin{example}
		求幂级数 $\suml_{n=1}^\infty(2^n-1)z^{n-1}$ 的收敛半径与和函数.
	\end{example}

	\onslide<+->
	\begin{solution}
		由 $\displaystyle\lim_{n\to\infty}\abs{\frac{c_{n+1}}{c_n}}=\lim_{n\to\infty}\frac{2^{n+1}-1}{2^n-1}=2$ 可知收敛半径为 $\dfrac12$.
		\onslide<+->{当 $|z|<\dfrac12$ 时, $|2z|<1$.}
		\onslide<+->{从而
		\begin{align*}
		\sum_{n=1}^\infty(2^n-1)z^{n-1}&=\sum_{n=1}^\infty 2^n z^{n-1}-\sum_{n=1}^\infty z^{n-1}\\
		&\visible<+->{=\frac2{1-2z}-\frac1{1-z}=\frac1{(1-2z)(1-z)}.}
		\end{align*}}
		\vspace{-\baselineskip}
	\end{solution}
\end{frame}


\begin{frame}{典型例题: 幂级数的收敛半径与和函数}
	\onslide<+->
	\begin{example}
		求幂级数 $\suml_{n=0}^\infty(n+1)z^n$ 的收敛半径与和函数.
	\end{example}
	
	\onslide<+->
	\begin{solution}
		由 $\displaystyle\lim_{n\to\infty}\abs{\frac{c_{n+1}}{c_n}}=\lim_{n\to\infty}\frac{n+1}n=1$ 可知收敛半径为 $1$.
	\onslide<+->{当 $|z|<1$ 时,
		\[\int_0^z\sum_{n=0}^\infty(n+1)z^n\diff z=\sum_{n=0}^\infty z^{n+1}=\frac z{1-z}=-1-\frac1{z-1},\]
	}\onslide<+->{因此
		\[\sum_{n=0}^\infty(n+1)z^n=\left(-\frac1{z-1}\right)'=\frac1{(z-1)^2},\quad |z|<1.\]}
	\vspace{-\baselineskip}
\end{solution}
\end{frame}


\begin{frame}{典型例题: 幂级数的收敛半径与和函数}
	\onslide<+->
	\begin{exercise}
		求幂级数 $\suml_{n=1}^\infty\dfrac{z^n}n$ 的收敛半径与和函数.
	\end{exercise}

	\onslide<+->
	\begin{answer}
		收敛半径为 $1$, 和函数为 $\ln(1-z)$.
	\end{answer}
\end{frame}


\begin{frame}{例题: 函数项级数的积分}
	\onslide<+->
	\begin{example}
		求 $\displaystyle\oint_{|z|=\frac12}\left(\sum_{n=-1}^\infty z^n\right)\diff z$.
	\end{example}
	
	\onslide<+->
	\begin{solution}
		由于 $\suml_{n=0}^\infty z^n$ 在 $|z|<1$ 收敛,
	\onslide<+->{它的和函数解析.
	}\onslide<+->{因此
		\begin{align*}
		\oint_{|z|=\frac12}\left(\sum_{n=-1}^\infty z^n\right)\diff z
		&=\oint_{|z|=\frac12}\frac1z\diff z+\oint_{|z|=\frac12}\left(\sum_{n=0}^\infty z^n\right)\diff z\\
		&\visible<+->{=2\pi i+0=2\pi i.}
	\end{align*}}
	\vspace{-\baselineskip}
	\end{solution}
\end{frame}

