\section{留数}


\begin{frame}{留数}
设 $z_0$ 为 $f(z)$ 的孤立奇点,
\onslide<+->
那么 $f(z)$ 在某个 $0<|z-z_0|<\delta$ 上可以展开为洛朗级数
\[f(z)=\cdots+\frac{c_{-1}}{z-z_0}+c_0+c_1(z-z_0)+\cdots\]
\onslide<+->
其中 $c_n=\displaystyle\frac1{2\pi i}\oint_C\frac{f(z)}{(z-z_0)^{n+1}}\diff z$, $C$ 为该圆环域中绕 $z_0$ 的一条闭路.

\onslide<+->
特别地,
\[\markatt{\Res[f(z),z_0]:=c_{-1}=\frac1{2\pi i}\oint_Cf(z)\diff z}\]
被称为函数 \markdef{$f(z)$ 在 $z_0$ 的留数}.
\onslide<+->
可以看出, 知道留数之后可以用来计算积分.
\end{frame}


\begin{frame}{留数定理}
\beqskip{0pt}
\begin{theorem}
如果 $f(z)$ 在区域 $D$ 内除 $z_1,z_2,\dots,z_n$ 外处处解析, $C$ 是 $D$ 内包含这些奇点的一条闭路, 那么
\[\oint_Cf(z)\diff z=2\pi i\sum_{k=1}^n\Res[f(z),z_k].\]
\end{theorem}
\begin{proof}
\begin{center}
\begin{tikzpicture}
\filldraw[cstcurve,rounded corners=0.4cm,alecolor,cstfill] (-3,-0.7) rectangle (3.4,0.7);
\draw[cstcurve,alecolor,cstarrow1to] (3.4,-0.2)--(3.4,0.1);
\filldraw[cstcurve,rounded corners=0.4cm,white,draw=notcolor] (-2.5,-0.5) rectangle (-1.5,0.5);
\filldraw[cstcurve,rounded corners=0.4cm,white,draw=notcolor] (-0.5,-0.5) rectangle (0.5,0.5);
\filldraw[cstcurve,rounded corners=0.4cm,white,draw=notcolor] (1.5,-0.5) rectangle (2.5,0.5);
\draw[cstcurve,notcolor,cstarrow1to] (-2.5,0.2)--(-2.5,-0.1);
\draw[cstcurve,notcolor,cstarrow1to] (-0.5,0.2)--(-0.5,-0.1);
\draw[cstcurve,notcolor,cstarrow1to] (1.5,0.2)--(1.5,-0.1);
\fill[cstdot] (-2,0) circle;
\fill[cstdot] (0,0) circle;
\fill[cstdot] (2,0) circle;
\draw
  (3.8,0) node[alecolor] {$C$}
  (-1.2,-0.2) node[notcolor] {$C_1$}
  (0.8,-0.2) node[notcolor] {$C_2$}
  (2.8,-0.2) node[notcolor] {$C_3$}
  (-2,-0.3) node {$z_1$}
  (0,-0.3) node {$z_2$}
  (2,-0.3) node {$z_3$};
\end{tikzpicture}
\end{center}
\onslide<+->
由复闭路定理可知,
\[\oint_Cf(z)\diff z=\sum_{k=1}^n\oint_{C_k}f(z)\diff z
=2\pi i\sum_{k=1}^n\Res[f(z),z_k].\qedhere\]
\end{proof}
\endgroup
\end{frame}


\begin{frame}{留数的计算方法}
\beqskip{9pt}
\begin{enumerate}
\item $z_0$ 为 $f(z)$ 的可去奇点, 显然 $\Res[f(z),z_0]=0$.
\item $z_0$ 为 $f(z)$ 的本性奇点, 一般只能从定义计算.
\item $z_0$ 为 $f(z)$ 的极点.
\end{enumerate}

\begin{conclusion}
如果 $z_0$ 是 $\le m$ 阶极点, 那么
\[\markatt{\Res[f(z),z_0]=\frac1{(m-1)!}\lim_{z\to z_0}\frac{\diff^{m-1}}{\diff z^{m-1}}[(z-z_0)^mf(z)].}\]
\end{conclusion}

\begin{conclusion}
如果 $z_0$ 是 $1$ 阶极点, 那么
\[\markatt{\Res[f(z),z_0]=\lim\limits_{z\to z_0}(z-z_0)f(z)}.\]
\end{conclusion}
\endgroup
\end{frame}


\begin{frame}{留数的计算方法}
\begin{proof}
由于 $g(z)=(z-z_0)^mf(z)$ 在 $z_0$ 附近解析,
\onslide<+->
且
\[g(z)=c_{-m}+\cdots+c_{-1}(z-z_0)^{m-1}+c_0(z-z_0)^m+\cdots\]
\onslide<+->
由幂级数系数与导数的关系可知 $c_{-1}=\dfrac1{(m-1)!}g^{(m-1)}(z_0)$.
\end{proof}

\begin{conclusion}
设 $P(z),Q(z)$ 在 $z_0$ 解析且 $P(z_0)\neq 0$, $Q(z_0)=0$, $Q'(z_0)\neq 0$.
则 $z_0$ 是 $f(z)$ 的 $1$ 阶极点且
\[\markatt{\Res[P/Q,z_0]=\frac{P(z_0)}{Q'(z_0)}.}\]
\end{conclusion}
\end{frame}


\begin{frame}{留数的计算方法}
\begin{proofs}
由于 $z_0$ 是 $Q(z)$ 的 $1$ 阶零点且 $P(z_0)\neq 0$, 因此 $z_0$ 是 $P/Q$ 的 $1$ 阶极点.
\onslide<+->
于是
\begin{align*}
\Res[f(z),z_0]&=\lim_{z\to z_0}\frac{(z-z_0)P(z)}{Q(z)}\\
&=\lim_{z\to z_0}\frac{P(z)}{Q(z)/(z-z_0)}
=\frac{P(z_0)}{Q'(z_0)}.\mqed
\end{align*}
\end{proofs}
\end{frame}


\begin{frame}{典型例题: 留数的计算}
\beqskip{5pt}
\begin{example}
求 $\Res\left[\dfrac{e^z}{z^n},0\right]$.
\end{example}
\begin{solution}
由于 $(e^z)(0)=1$, 因此 $0$ 是 $n$ 阶极点,
\onslide<+->
\begin{align*}
\Res\left[\frac{e^z}{z^n},0\right]
&=\frac1{(n-1)!}\lim_{z\to0}(e^z)^{(n-1)}\\
&\visible<+->{=\frac1{(n-1)!}\lim_{z\to0}e^z
=\frac1{(n-1)!}.}
\end{align*}
\onslide<+->
也可以直接从该函数的洛朗展开得到:
\[\frac{e^z}{z^n}=\sum_{k=0}^\infty\frac{z^{k-n}}{k!}
=\cdots+\frac1{(n-1)!z}+\cdots\qedhere\]
\end{solution}
\endgroup
\end{frame}


\begin{frame}{典型例题: 留数的计算}
\beqskip{5pt}
\begin{example}
求 $\Res\left[\dfrac{z-\sin z}{z^6},0\right]$.
\end{example}
\begin{solutions}
\indent
设 $P(z)=z-\sin z$, 则
\[P(0)=P'(0)=P''(0)=0,\quad P'''(0)\neq 0.\]
\onslide<+->
因此 $z=0$ 是 $P(z)$ 的三阶零点,
\onslide<+->
是 $\dfrac{z-\sin z}{z^6}$ 的 $3$ 阶极点.

\indent
\onslide<+->
如果用公式
\[\Res\left[\frac{z-\sin z}{z^6},0\right]
=\frac1{2!}\lim_{z\to0}\left(\frac{z-\sin z}{z^3}\right)''\]
计算会很繁琐.
\end{solutions}
\endgroup
\end{frame}


\begin{frame}{典型例题: 留数的计算}
\begin{solutione}
\vspace{-\baselineskip}
\begin{align*}
\Res\left[\frac{z-\sin z}{z^6},0\right]
&=\frac1{5!}\lim_{z\to0}(z-\sin z)^{(5)}\\
&\visible<+->{=\frac1{5!}\lim_{z\to0}(-\cos z)=-\frac1{120}.}
\end{align*}

\indent
\onslide<+->
也可以由 $\sin z$ 的泰勒展开得到
\[z-\sin z=\frac{z^3}6-\frac{z^5}{120}+\cdots,
\frac{z-\sin z}{z^6}=\frac1{6z^3}-\frac1{120z}+\cdots\]
\onslide<+->
\[\Res\left[\frac{z-\sin z}{z^6},0\right]=-\frac1{120}.\qedhere\]
\end{solutione}
\end{frame}


\begin{frame}{典型例题: 留数的计算}
\begin{exercise}
求 $\Res\left[\dfrac{e^z-1}{z^5},0\right]$.
\end{exercise}
\begin{answer}
$\dfrac1{24}$.
\end{answer}
\end{frame}


\begin{frame}{典型例题: 留数的计算}
\begin{example}
求 $\Res\left[\dfrac{z}{z^4-1},i\right]$.
\end{example}
\begin{solution}
$z=i$ 是 $z$ 的取值非零的点, 是 $z^4-1$ 的 $1$ 阶零点.
\onslide<+->
\[\Res\left[\frac{z}{z^4-1},i\right]
=\frac{z}{(z^4-1)'}\Big|_{z=i}
=\frac{z}{4z^3}\Big|_{z=i}=-\frac14.\qedhere\]
\end{solution}
\end{frame}


\begin{frame}{例题: 留数的应用}
\beqskip{0pt}
\vspace{-5pt}
\begin{example}
计算积分 $\displaystyle\oint_{|z|=2}\frac{e^z}{z(z-1)^2}\diff z$.
\end{example}
\vspace{-5pt}
\begin{solutions}
$f(z)=\dfrac{e^z}{z(z-1)^2}$ 在 $|z|<2$ 内有奇点 $z=0,1$.
\onslide<+->
\[\Res[f(z),0]=\lim_{z\to0}\frac{e^z}{(z-1)^2}=1,\]
\onslide<+->
\[\Res[f(z),1]=\lim_{z\to1}\left(\frac{e^z}z\right)'
=\lim_{z\to1}\frac{e^z(z-1)}{z^2}=0,\]
\onslide<+->
\begin{align*}
\oint_{|z|=2}\frac{e^z}{z(z-1)^2}\diff z
&=2\pi i\bigl[\Res[f(z),0]+\Res[f(z),1]\bigr]\\
&\visible<+->{=2\pi i(1+0)=2\pi i.\mqed}
\end{align*}
\end{solutions}
\endgroup
\end{frame}


\begin{frame}{在 $\infty$ 的留数*}
\onslide<+->
设 $\infty$ 为 $f(z)$ 的孤立奇点,
\onslide<+->
那么 $f(z)$ 在某个 $R<|z|<+\infty$ 上可以展开为洛朗级数
\[f(z)=\cdots+c_{-1}z^{-1}+c_0+c_1z+\cdots\]
\onslide<+->
其中 $c_n=\displaystyle\frac1{2\pi i}\oint_C\frac{f(z)}{z^{n+1}}\diff z$, $C$ 为该圆环域中绕 $0$ 的闭路.
\onslide<+->
称
\[\markatt{\Res[f(z),\infty]:=-c_{-1}=\frac1{2\pi i}\oint_{C^-}f(z)\diff z}\]
为函数 \markdef{$f(z)$ 在 $\infty$ 的留数}.
\onslide<+->
由于
\[f\left(\frac1z\right)\frac1{z^2}=\cdots+\frac{c_1}{z^3}+\frac{c_0}{z^2}+\frac{c_{-1}}z+c_{-2}+\cdots\]
\onslide<+->
因此
\[\markatt{\Res[f(z),\infty]=-\Res\left[f\left(\frac1z\right)\frac1{z^2},0\right].}\]
\end{frame}


\begin{frame}{留数之和为 $0$*}
\onslide<+->
需要注意的是, 和普通复数不同, \markatt{即便 $\infty$ 是可去奇点, 也不意味着 $\Res[f(z),\infty]=0$}.

\begin{theorem}
如果 $f(z)$ 在扩充复平面内只有有限个孤立奇点, 那么 $f(z)$ 在各\markatt{奇点处的留数之和为 $0$}.
\end{theorem}
\begin{proof}
设闭路 $C$ 内部包含除 $\infty$ 外所有奇点 $z_1,\dots,z_n$.
\onslide<+->
由留数定理
\[-2\pi i\Res[f(z),\infty]=\oint_C f(z)\diff z=2\pi i\sum_{k=1}^n\Res[f(z),z_k].\]
\onslide<+->
故 $\suml_{k=1}^n\Res[f(z),z_k]+\Res[f(z),\infty]=0$.
\end{proof}
\end{frame}


\begin{frame}{典型例题: 留数的应用*}
\begin{example}
求 $\displaystyle\oint_{|z|=2}\frac{z^3}{z^4-1}\diff z$.
\end{example}
\begin{solution}
$f(z)=\dfrac{z^3}{z^4-1}$ 在 $|z|>2$ 内只有奇点 $\infty$.
\onslide<+->
\[\Res[f(z),\infty]=-\Res\left[f\left(\frac1z\right)\frac1{z^2},0\right]
=-\Res\left[\frac1{z(1-z^4)},0\right]=-1.\]
\onslide<+->
因此
\[\oint_{|z|=2}\frac z{z^4-1}\diff z=-2\pi i\Res[f(z),\infty]=2\pi i.\qedhere\]
\end{solution}
\end{frame}


\begin{frame}{典型例题: 留数的应用*}
\begin{exercise}
求 $\displaystyle\oint_{|z|=2}\frac{z^4}{z^5+1}\diff z$.
\end{exercise}
\begin{answer}
$2\pi i$.
\end{answer}
\begin{example}
求 $\displaystyle\oint_{|z|=2}f(z)\diff z$, 其中 $f(z)=\dfrac1{(z+i)^{10}(z-1)(z-3)}$.
\end{example}
\end{frame}


\begin{frame}{典型例题: 留数的应用*}
\beqskip{7pt}
\begin{solutions}
$f(z)$ 在 $|z|>2$ 内只有奇点 $3,\infty$.
\onslide<+->
\begin{align*}
\Res[f(z),\infty]&=-\Res\left[f\left(\frac1z\right)\frac1{z^2},0\right]\\
&=-\Res\left[\frac{z^{10}}{(1+iz)^{10}(1-z)(1-3z)},0\right]=0.\end{align*}
\onslide<+->
\[\Res[f(z),3]=\lim_{z\to3}(z-3)f(z)=\frac1{2(3+i)^{10}}.\]
\onslide<+->
\vspace{-\baselineskip}
\begin{align*}
&\peq\oint_{|z|=2}f(z)\diff z=2\pi i\bigl[\Res[f(z),-i]+\Res[f(z),1]\bigr]\\
&\visible<+->{=-2\pi i\bigl[\Res[f(z),3]+\Res[f(z),\infty]\bigr]=-\frac{\pi i}{(3+i)^{10}}.\mqed}
\end{align*}
\end{solutions}
\endgroup
\end{frame}


\begin{frame}{积分的计算方法汇总}
\begin{center}
\begin{tikzpicture}[node distance=25pt]
  \node (integral){\vphantom{$\displaystyle\int$}};
  \node[draw,white,fill=strucolor,right=-40pt of integral]  (integral0){定积分 $\displaystyle\int_Cf(z)\diff z$ 的计算};
  \node[draw,below=35pt of integral, visible on=<2->]	(isclosed){是闭路};
  \node[draw,below=45pt of isclosed, visible on=<3->]	(isanalytic){$f(z)$ 解析};
  \node[draw,right=50pt of isclosed, visible on=<5->]	(issingle){只有孤立奇点};
  \node[draw,right=45pt of issingle,align=center, visible on=<7->]	(ismany){闭路内部和外部\\孤立奇点数量};
  \node[below=30pt of isanalytic]	(end1){\vphantom{$\displaystyle\int$}};
  \node[draw,fill=alercolor!25,right=-30pt of end1, visible on=<3->]	(end10){求出 $f(z)$ 的原函数 $F(z)$ 得到 $\displaystyle\int_Cf(z)\diff z=F(b)-F(a)$};
  \node[draw,fill=alercolor!25,below=of issingle,align=center, visible on=<4->]	(end2){设曲线方程为 $z(t)$, 则\\
积分$\displaystyle=\int_a^bf(z)z'(t)\diff t$\\
可能需要分段计算};
  \node[draw,fill=alercolor!25,above=of ismany,align=center, visible on=<8->]	(end3){用闭路内的孤立\\奇点的留数计算};
  \node[draw,fill=alercolor!25,below=of ismany,align=center, visible on=<9->]	(end4){用闭路外的孤立\\奇点的留数计算, \\最后取负号*};
  \draw[cstnarrow,visible on=<2->] (integral) -- (isclosed);
  \draw[cstnarrow,visible on=<3->] (isanalytic) -- node[left]{是} (end1);
  \draw[cstnarrow,visible on=<5->] (isclosed) -- node[above]{是} (issingle);
  \draw[cstnarrow,visible on=<7->] (issingle) -- node[above]{是} (ismany);
  \draw[cstnarrow,visible on=<3->] (isclosed) -- node[left]{否} (isanalytic);
  \draw[cstnarrow,visible on=<6->] (issingle) -- node[left]{否} (end2);
  \draw[cstnarrow,visible on=<4->] (isanalytic) -- node[above]{否} (end2);
  \draw[cstnarrow,visible on=<8->] (ismany) -- node[left]{闭路内的少} (end3);
  \draw[cstnarrow,visible on=<9->] (ismany) -- node[left]{闭路外的少} (end4);
\end{tikzpicture}
\end{center}
\end{frame}


