\section{留数}


\begin{frame}{留数}
设 $z_0$ 为 $f(z)$ 的孤立奇点,
\onslide<+->
那么 $f(z)$ 在某个 $0<|z-z_0|<\delta$ 上可以展开为洛朗级数
\[f(z)=\cdots+\frac{c_{-1}}{z-z_0}+c_0+c_1(z-z_0)+\cdots\]
\onslide<+->
其中 $c_n=\displaystyle\frac1{2\pi i}\oint_C\frac{f(z)}{(z-z_0)^{n+1}}\diff z$, $C$ 为该圆环域中绕 $z_0$ 的一条闭路.

\onslide<+->
特别地,
\[\alert{\Res[f(z),z_0]:=c_{-1}=\frac1{2\pi i}\oint_Cf(z)\diff z}\]
被称为函数 \emph{$f(z)$ 在 $z_0$ 的留数}.
\onslide<+->
可以看出, 知道留数之后可以用来计算积分.
\end{frame}


\begin{frame}{留数定理}
\beqskip{5pt}
\begin{block}{留数定理}
如果 $f(z)$ 在闭路 $C$ 上解析, 在 $C$ 内部的奇点为 $z_1,z_2,\dots,z_n$, 那么
\[\oint_Cf(z)\diff z=2\pi i\sum_{k=1}^n\Res[f(z),z_k].\]
\end{block}
\begin{proof}
\begin{center}
\begin{tikzpicture}
\filldraw[cstcurve,rounded corners=0.4cm,dcolora,cstfill] (-3,-0.7) rectangle (3.4,0.7);
\draw[cstcurve,dcolora,cstarrow1to] (3.4,-0.2)--(3.4,0.1);
\filldraw[cstcurve,rounded corners=0.4cm,white,draw=dcolorb] (-2.5,-0.5) rectangle (-1.5,0.5);
\filldraw[cstcurve,rounded corners=0.4cm,white,draw=dcolorb] (-0.5,-0.5) rectangle (0.5,0.5);
\filldraw[cstcurve,rounded corners=0.4cm,white,draw=dcolorb] (1.5,-0.5) rectangle (2.5,0.5);
\draw[cstcurve,dcolorb,cstarrow1to] (-2.5,0.2)--(-2.5,-0.1);
\draw[cstcurve,dcolorb,cstarrow1to] (-0.5,0.2)--(-0.5,-0.1);
\draw[cstcurve,dcolorb,cstarrow1to] (1.5,0.2)--(1.5,-0.1);
\fill[cstdot] (-2,0) circle;
\fill[cstdot] (0,0) circle;
\fill[cstdot] (2,0) circle;
\draw
  (3.8,0) node[dcolora] {$C$}
  (-1.2,-0.2) node[dcolorb] {$C_1$}
  (0.8,-0.2) node[dcolorb] {$C_2$}
  (2.8,-0.2) node[dcolorb] {$C_3$}
  (-2,-0.3) node {$z_1$}
  (0,-0.3) node {$z_2$}
  (2,-0.3) node {$z_3$};
\end{tikzpicture}
\end{center}
\onslide<+->
由复闭路定理可知,
\[\oint_Cf(z)\diff z=\sum_{k=1}^n\oint_{C_k}f(z)\diff z
=2\pi i\sum_{k=1}^n\Res[f(z),z_k].\qedhere\]
\end{proof}
\endgroup
\end{frame}


\begin{frame}{留数的计算方法}
\onslide<+->
若 $z_0$ 为 $f(z)$ 的可去奇点, 显然 $\Res[f(z),z_0]=0$.
\begin{example}
$f(z)=\dfrac{z^3(e^z-1)^2}{\sin z^4}$.
\onslide<+->
由于 $z=0$ 是 $f(z)$ 的可去奇点,
\onslide<+->
因此
\[\Res[f(z),0]=0.\]
\end{example}
\end{frame}


\begin{frame}{留数的计算方法}
\onslide<+->
若 $z_0$ 为 $f(z)$ 的本性奇点, 一般只能从定义计算.
\begin{example}
$f(z)=z^4\sin\dfrac1z$.
\onslide<+->
由于
\[f(z)=z^4\sum_{n=0}^\infty(-1)^n\frac{z^{-2n-1}}{(2n+1)!}
=z^3-\frac z{3!}+\frac1{5!z}+\cdots\]
\onslide<+->
因此
\[\Res[f(z),0]=\frac1{120}.\]
\end{example}
\end{frame}


\begin{frame}{留数的计算方法}
设 $z_0$ 为 $f(z)$ 的极点.
\begin{block}{极点留数计算公式 I}
如果 $z_0$ 是 $\le m$ 阶极点或可去奇点, 那么
\[\aboxeq{\Res[f(z),z_0]=\frac1{(m-1)!}\lim_{z\to z_0}\frac{\diff^{m-1}}{\diff z^{m-1}}[(z-z_0)^mf(z)]}.\]
\end{block}

\begin{block}{极点留数计算公式 II}
如果 $z_0$ 是 $1$ 阶极点或可去奇点, 那么
\[\aboxeq{\Res[f(z),z_0]=\lim\limits_{z\to z_0}(z-z_0)f(z)}.\]
\end{block}
\end{frame}


\begin{frame}{留数的计算方法}
\begin{proof}
设
\[f(z)=c_{-m}(z-z_0)^{-m}+\cdots+c_{-1}(z-z_0)^{-1}+c_0+\cdots.\]
\onslide<+->
令 $g(z)$ 为幂级数
\[c_{-m}+\cdots+c_{-1}(z-z_0)^{m-1}+c_0(z-z_0)^m+\cdots\]
的和函数, 则 $g(z)=(z-z_0)^mf(z)$.
\onslide<+->
由幂级数系数与导数的关系可知
\[\Res[f(z),z_0]=c_{-1}=\dfrac1{(m-1)!}g^{(m-1)}(z_0).\qedhere\]
\end{proof}
\end{frame}


\begin{frame}{典型例题: 留数的计算}
\begin{example}
求 $\Res\left[\dfrac{e^z}{z^n},0\right]$.
\end{example}
\begin{solution}
由于 $(e^z)(0)=1$, 因此 $0$ 是 $n$ 阶极点,
\onslide<+->
\begin{align*}
\Res\left[\frac{e^z}{z^n},0\right]&=\frac1{(n-1)!}\lim_{z\to0}(e^z)^{(n-1)}\\
&\visible<+->{=\frac1{(n-1)!}\lim_{z\to0}e^z=\frac1{(n-1)!}.}
\end{align*}
\end{solution}
\end{frame}


\begin{frame}{典型例题: 留数的计算}
\begin{example}
求 $\Res\left[\dfrac{z-\sin z}{z^6},0\right]$.
\end{example}
\begin{solution}
因为 $z=0$ 是 $z-\sin z$ 的三阶零点,
\onslide<+->
所以是 $\dfrac{z-\sin z}{z^6}$ 的 $3$ 阶极点.
\onslide<+->
如果用公式
\[\Res\left[\frac{z-\sin z}{z^6},0\right]
=\frac1{2!}\lim_{z\to0}\left(\frac{z-\sin z}{z^3}\right)''\]
计算会很繁琐.
\end{solution}
\end{frame}


\begin{frame}[<*>]{典型例题: 留数的计算}
\onslide<+->
\begin{solutionc}
\begin{align*}
\Res\left[\frac{z-\sin z}{z^6},0\right]&=\frac1{5!}\lim_{z\to0}(z-\sin z)^{(5)}\\
&\visible<+->{=\frac1{5!}\lim_{z\to0}(-\cos z)=-\frac1{120}.}
\end{align*}
\end{solutionc}
\onslide<+->
\begin{columns}
	\column{0.48\textwidth}
		\begin{exercise}
求 $\Res\left[\dfrac{e^z-1}{z^5},0\right]$.
		\end{exercise}\onslide<+->
	\column{0.48\textwidth}
		\begin{answer}
$\dfrac1{24}$.
		\end{answer}
\end{columns}
\end{frame}


\begin{frame}{留数的计算方法}
\begin{block}{极点留数计算公式 III}
设 $P(z),Q(z)$ 在 $z_0$ 解析且 $z_0$ 是 $Q$ 的 $1$ 阶零点, 则
\[\aboxeq{\Res\left[\frac{P(z)}{Q(z)},z_0\right]=\frac{P(z_0)}{Q'(z_0)}}.\]
\end{block}
\begin{proofs}
不难看出 $z_0$ 是 $f(z)=\dfrac{P(z)}{Q(z)}$ 的 $1$ 阶极点或可去奇点.
\onslide<+->
因此
\begin{flalign*}
&&&\peq\Res[f(z),z_0]=\lim_{z\to z_0}(z-z_0)f(z)&\\
&&&\visible<+->{=\lim_{z\to z_0}\frac{P(z)}{\dfrac{Q(z)-Q(z_0)}{z-z_0}}}
  \visible<+->{=\frac{P(z_0)}{\lim\limits_{z\to z_0}\dfrac{Q(z)-Q(z_0)}{z-z_0}}=\frac{P(z_0)}{Q'(z_0)}.}&\mqed
\end{flalign*}
\end{proofs}
\end{frame}


\begin{frame}{典型例题: 留数的计算}
\begin{example}
求 $\Res\left[\dfrac{z}{z^8-1},\dfrac{1+i}{\sqrt2}\right]$.
\end{example}
\begin{solution}
由于 $z=\dfrac{1+i}{\sqrt2}$ 是分母的 $1$ 阶零点,
\onslide<+->
因此
\[\Res\left[\frac z{z^8-1},\frac{1+i}{\sqrt2}\right]
=\frac z{(z^8-1)'}\Big|_{z=\frac{1+i}{\sqrt2}}
=\frac z{8z^7}\Big|_{z=\frac{1+i}{\sqrt2}}
=-\frac i8.\]
\end{solution}
\begin{example}
计算积分 $\displaystyle\oint_{|z|=2}\frac{e^z}{z(z-1)^2}\diff z$.
\end{example}
\end{frame}


\begin{frame}{例题: 留数的应用}
\begin{solution}
$f(z)=\dfrac{e^z}{z(z-1)^2}$ 在 $|z|<2$ 内有奇点 $z=0,1$.
\onslide<+->
\[\Res[f(z),0]=\lim_{z\to0}\frac{e^z}{(z-1)^2}=1,\]
\onslide<+->
\[\Res[f(z),1]=\lim_{z\to1}\left(\frac{e^z}z\right)'
=\lim_{z\to1}\frac{e^z(z-1)}{z^2}=0,\]
\onslide<+->
\[\oint_{|z|=2}\frac{e^z}{z(z-1)^2}\diff z
=2\pi i\bigl[\Res[f(z),0]+\Res[f(z),1]\bigr]
=2\pi i.\]
\end{solution}
\end{frame}


\begin{frame}{在 $\infty$ 的留数*}
\onslide<+->
设 $\infty$ 为 $f(z)$ 的孤立奇点,
\onslide<+->
那么 $f(z)$ 在某个 $R<|z|<+\infty$ 上可以展开为洛朗级数
\[f(z)=\cdots+c_{-1}z^{-1}+c_0+c_1z+\cdots\]
\onslide<+->
其中 $c_n=\displaystyle\frac1{2\pi i}\oint_C\frac{f(z)}{z^{n+1}}\diff z$, $C$ 为该圆环域中绕 $0$ 的闭路.
\onslide<+->
称
\[\alert{\Res[f(z),\infty]:=-c_{-1}=\frac1{2\pi i}\oint_{C^-}f(z)\diff z}\]
为函数 \emph{$f(z)$ 在 $\infty$ 的留数}.
\onslide<+->
由于
\[f\left(\frac1z\right)\frac1{z^2}=\cdots+\frac{c_1}{z^3}+\frac{c_0}{z^2}+\frac{c_{-1}}z+c_{-2}+\cdots\]
\onslide<+->
因此
\[\alert{\Res[f(z),\infty]=-\Res\left[f\left(\frac1z\right)\frac1{z^2},0\right].}\]
\end{frame}


\begin{frame}{留数之和为 $0$*}
\onslide<+->
需要注意的是, 和普通复数不同, \alert{即便 $\infty$ 是可去奇点, 也不意味着 $\Res[f(z),\infty]=0$}.

\begin{theorem}
如果 $f(z)$ 只有有限个奇点, 那么 $f(z)$ 在\alert{扩充复平面内各奇点处的留数之和为 $0$}.
\end{theorem}
\begin{proof}
设闭路 $C$ 内部包含除 $\infty$ 外所有奇点 $z_1,\dots,z_n$.
\onslide<+->
由留数定理
\[-2\pi i\Res[f(z),\infty]=\oint_C f(z)\diff z=2\pi i\sum_{k=1}^n\Res[f(z),z_k].\]
\onslide<+->
故 $\suml_{k=1}^n\Res[f(z),z_k]+\Res[f(z),\infty]=0$.
\end{proof}
\end{frame}

%
%\begin{frame}{典型例题: 留数的应用*}
%\begin{example}
%求 $\displaystyle\oint_{|z|=2}\frac{z^3}{z^4-1}\diff z$.
%\end{example}
%\begin{solution}
%$f(z)=\dfrac{z^3}{z^4-1}$ 在 $|z|>2$ 内只有奇点 $\infty$.
%\onslide<+->
%\[\Res[f(z),\infty]=-\Res\left[f\left(\frac1z\right)\frac1{z^2},0\right]
%=-\Res\left[\frac1{z(1-z^4)},0\right]=-1.\]
%\onslide<+->
%因此
%\[\oint_{|z|=2}\frac z{z^4-1}\diff z=-2\pi i\Res[f(z),\infty]=2\pi i.\]
%\end{solution}
%\begin{exercise}
%求 $\displaystyle\oint_{|z|=2}\frac{z^4}{z^5+1}\diff z$.
%\end{exercise}
%\begin{answer}
%$2\pi i$.
%\end{answer}
%\end{frame}


\begin{frame}{例题: 留数的应用*}
\begin{example}
求 $\displaystyle\oint_{|z|=2}f(z)\diff z$, 其中 $f(z)=\dfrac{\sin\frac1z}{(z+i)^{10}(z-1)(z-3)}$.
\end{example}
\begin{solution}
$f(z)$ 在 $|z|>2$ 内只有奇点 $3,\infty$.
\onslide<+->
\[\Res[f(z),3]=\lim_{z\to3}(z-3)f(z)=\frac1{2(3+i)^{10}}\sin\frac13.\]
\end{solution}
\end{frame}


\begin{frame}{例题: 留数的应用*}
\begin{solutionc}
\begin{align*}
\Res[f(z),\infty]&=-\Res\left[f\left(\frac1z\right)\frac1{z^2},0\right]\\
&=-\Res\left[\frac{z^{10}\sin z}{(1+iz)^{10}(1-z)(1-3z)},0\right]=0.
\end{align*}
\vspace{-\baselineskip}
\onslide<+->
\begin{align*}
&\peq\oint_{|z|=2}f(z)\diff z\\
&=2\pi i\bigl[\Res[f(z),-i]+\Res[f(z),1]+\Res[f(z),0]\bigr]\\
&\visible<+->{=-2\pi i\bigl[\Res[f(z),3]+\Res[f(z),\infty]\bigr]=-\frac{\pi i}{(3+i)^{10}}\sin\frac13.}
\end{align*}
\end{solutionc}
\end{frame}


\begin{frame}{积分的计算方法汇总}
\begin{center}
\begin{tikzpicture}[node distance=25pt]
  \node (integral){\vphantom{$\displaystyle\int$}};
  \node[cstnodeb,right=-40pt of integral]  (integral0){定积分 $\displaystyle\int_Cf(z)\diff z$ 的计算};
  \node[cstnode,below=35pt of integral, visible on=<2->]	(isclosed){是闭路};
  \node[cstnode,below=45pt of isclosed, visible on=<3->]	(isanalytic){$f(z)$ 解析};
  \node[cstnode,right=50pt of isclosed, visible on=<5->]	(issingle){只有孤立奇点};
  \node[cstnode,right=45pt of issingle,align=center, visible on=<7->]	(ismany){闭路内部和外部\\孤立奇点数量};
  \node[below=30pt of isanalytic]	(end1){\vphantom{$\displaystyle\int$}};
  \node[cstnodeg,right=-30pt of end1, visible on=<3->]	(end10){求出 $f(z)$ 的原函数 $F(z)$ 得到 $\displaystyle\int_Cf(z)\diff z=F(b)-F(a)$};
  \node[cstnodeg,below=of issingle,align=center, visible on=<4->]	(end2){设曲线方程为 $z(t)$, 则\\
积分$\displaystyle=\int_a^bf(z)z'(t)\diff t$\\
可能需要分段计算};
  \node[cstnodeg,above=of ismany,align=center, visible on=<8->]	(end3){用闭路内的孤立\\奇点的留数计算};
  \node[cstnodeg,below=of ismany,align=center, visible on=<9->]	(end4){用闭路外的孤立\\奇点的留数计算, \\最后取负号*};
  \draw[cstnarrow,dcolorc,visible on=<2->] (integral) -- (isclosed);
  \draw[cstnarrow,dcolorc,visible on=<3->] (isanalytic) -- node[left]{是} (end1);
  \draw[cstnarrow,dcolorc,visible on=<5->] (isclosed) -- node[above]{是} (issingle);
  \draw[cstnarrow,dcolorc,visible on=<7->] (issingle) -- node[above]{是} (ismany);
  \draw[cstnarrow,dcolorc,visible on=<3->] (isclosed) -- node[left]{否} (isanalytic);
  \draw[cstnarrow,dcolorc,visible on=<6->] (issingle) -- node[left]{否} (end2);
  \draw[cstnarrow,dcolorc,visible on=<4->] (isanalytic) -- node[above]{否} (end2);
  \draw[cstnarrow,dcolorc,visible on=<8->] (ismany) -- node[left]{闭路内的少} (end3);
  \draw[cstnarrow,dcolorc,visible on=<9->] (ismany) -- node[left]{闭路外的少} (end4);
\end{tikzpicture}
\end{center}
\end{frame}


\begin{frame}{例题: 留数在有理函数分解中的应用}
\onslide<+->
在求有理函数的洛朗展开, 以及之后在求有理函数的拉普拉斯逆变换时, 我们需要将一个有理函数表达为分母只有一个零点的有理函数之和.
\onslide<+->
例如:
\[\frac{z-3}{(z+1)(z-1)^2}=\frac1{z-1}-\frac1{(z-1)^2}-\frac1{z+1}.\]
\onslide<+->
我们可以用待定系数法计算, 不过使用留数会更为简便.
\end{frame}


\begin{frame}{例题: 留数在有理函数分解中的应用}
\onslide<+->
\begin{solution}
设
\[f(z)=\frac{z-3}{(z+1)(z-1)^2}=\frac a{z-1}+\frac b{(z-1)^2}+\frac c{z+1},\]
\onslide<+->
则
\[a=\Res[f(z),1]=\left(\frac{z-3}{z+1}\right)'\Big|_{z=1}
=\frac 4{(z+1)^2})\Big|_{z=1}=1,\]
\[b=\Res[(z-1)f(z),1]=\frac{z-3}{z+1}\Big|_{z=1}
=-1,\]
\[c=\Res[f(z),-1]=\frac{z-3}{(z-1)^2}\Big|_{z=-1}
=-1.\]
\end{solution}
\end{frame}

