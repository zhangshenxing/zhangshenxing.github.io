\section{极限和连续性}


\begin{frame}{数列极限}
\onslide<+->
复变函数的极限和连续性的定义和实函数情形是类似的.
\onslide<+->
我们先来看数列极限的定义.

\begin{definition}
\begin{itemize}
\item 设 $\{z_n\}_{n\ge 1}$ 是一个复数列.
\alert{如果 $\forall \varepsilon>0,\exists N$ 使得当 $n\ge N$ 时 $|z_n-z|<\varepsilon$}, 则称 $z$ 是\emph{数列 $\{z_n\}$ 的极限}, 记作 \emph{$\lim\limits_{n\to\infty}z_n=z$}.
\item 如果 \alert{$\forall X>0,\exists N$ 使得当 $n\ge N$ 时 $|z_n|>X$}, 则称 $\infty$ 是数列 $\{z_n\}$ 的极限, 记作 $\lim\limits_{n\to\infty}z_n=\infty$.
\end{itemize}
\end{definition}
\end{frame}


\begin{frame}{数列极限}
\onslide<+->
如果我们称
\[\stackrel{\circ}{U}(\infty,X)=\set{z\in\BC:|z|>X}\]
为 $\infty$ 的(去心)邻域.
\onslide<+->
那么 $\lim\limits_{n\to\infty}z_n=z\in\BC\cup\set\infty$ 可统一表述为:
\onslide<+->
对 $z$ 的任意邻域 $U$, $\exists N$ 使得当 $n\ge N$ 时 $z_n\in U$.
\onslide<1->
\begin{center}
\begin{tikzpicture}
\fill[cstfille] (0,0) circle (1.2);
\filldraw[cstcurve,dcolora,fill=white] (0,0) circle (0.7);
\draw[cstaxis] (-1.5,0)--(1.5,0);
\draw[cstaxis] (0,-1.5)--(0,1.5);
\end{tikzpicture}
\end{center}
\end{frame}


\begin{frame}{复球面}
\onslide<+->
那么有没有一种看法使得 $\infty$ 的邻域和普通复数的邻域没有差异呢?
\onslide<+->
我们将介绍复球面的概念, 它是复数的一种几何表示且自然包含无穷远点 $\infty$.
\onslide<+->
这种思想是在黎曼研究多值复变函数时引入的.

\onslide<+->
取一个与复平面相切于原点 $z=0$ 的球面.
\onslide<+->
过 $O$ 做垂直于复平面的直线, 并与球面相交于另一点 $N$, 称之为\emph{北极}.

\onslide<4->
\begin{center}
\begin{tikzpicture}
\fill[cstfill] (-3.65,-0.804)--(-1.85,0.804)--(3.65,0.804)--(1.85,-0.804)--cycle;
\filldraw[cstcurve,fill=black!10] (0,1) circle (1);
\draw[cstdash,dcolorc] (0,1) circle (1 and 0.3);
\draw[cstdash,black,visible on=<5->] (0,0)--(0,2);
\fill[cstdot,dcolorc] (0,1) circle;
\fill[cstdot,dcolorc,visible on=<5->] (0,2) circle;
\draw[cstaxis] (0,0)--(2.5,0);
\draw[cstaxis] (0,0)--(-0.8,-0.9);
\draw
  (0,2.3) node[dcolorc,visible on=<5->] {$N$};
\end{tikzpicture}
\end{center}
\end{frame}


\begin{frame}{复球面}
\onslide<+->
对于平面上的任意一点 $z$, 连接北极 $N$ 和 $z$ 的直线一定与球面相交于除 $N$ 以外的唯一一个点 $Z$.
\onslide<+->
反之, 球面上除了北极外的任意一点 $Z$, 直线 $NZ$ 一定与复平面相交于唯一一点.
\onslide<+->
这样, 球面上除北极外的所有点和全体复数建立了一一对应.

\onslide<1->
\begin{center}
\begin{tikzpicture}
\fill[cstfill] (-3.65,-0.804)--(-1.85,0.804)--(3.65,0.804)--(1.85,-0.804)--cycle;
\filldraw[cstcurve,fill=black!10] (0,1) circle (1);
\draw[cstdash] (0,1) circle (1 and 0.3);
\draw[cstdash,dcolorc] (0,0) circle (2 and 0.6);
\draw[cstdash,black] (0,0)--(0,2);
\draw[cstaxis] (0,0)--(2.5,0);
\draw[cstaxis] (0,0)--(-0.8,-0.9);
\draw[cstcurve,dcolora,cstarrowto] (0,2)--(1.65,-0.75);
\fill[cstdot,dcolora] (0.6,1) circle;
\draw[cstcurve,cstarrowto,dcolorb,visible on=<2->] (0,2)--(-1,0);
\fill[cstdot,dcolorb,visible on=<2->] (-0.7,0.6) circle;
\fill[cstdot,dcolorc] (0,2) circle;
\draw
  (0,2.3) node[dcolorc] {$N$}
  (-0.3,0.3) node[dcolorb,visible on=<2->] {$Z_2$}
  (-1.4,0) node[dcolorb,visible on=<2->] {$z_2$}
  (1.3,1) node[dcolora] {$Z_1$}
  (1.9,-0.75) node[dcolora] {$z_1$};
\end{tikzpicture}
\end{center}
\end{frame}


\begin{frame}{复球面: 无穷远点}
\onslide<+->
当 $|z|$ 越来越大时, 其对应球面上点也越来越接近 $N$.
\onslide<+->
如果我们在复平面上添加一个额外的"点"——\emph{无穷远点}, 记作 \emph{$\infty$}.
\onslide<+->
那么\emph{扩充复数集合 $\BC^*=\BC\cup\set\infty$} 就正好和球面上的点一一对应.
\onslide<+->
称这样的球面为\emph{复球面}, 称包含无穷远点的复平面为\emph{扩充复平面}(\emph{闭复平面}).
\onslide<1->
\begin{center}
\begin{tikzpicture}
\fill[cstfill] (-3.65,-0.804)--(-1.85,0.804)--(3.65,0.804)--(1.85,-0.804)--cycle;
\filldraw[cstcurve,fill=black!10] (0,1) circle (1);
\draw[cstdash] (0,1) circle (1 and 0.3);
\draw[cstdash,dcolorc] (0,0) circle (2 and 0.6);
\draw[cstdash] (0,0)--(0,2);
\draw[cstaxis] (0,0)--(2.5,0);
\draw[cstaxis] (0,0)--(-0.8,-0.9);
\draw[cstcurve,dcolora,cstarrowto] (0,2)--(1.65,-0.75);
\fill[cstdot,dcolora] (0.6,1) circle;
\draw[cstcurve,cstarrowto,dcolorb] (0,2)--(-1,0);
\fill[cstdot,dcolorb] (-0.7,0.6) circle;
\fill[cstdot,dcolorc] (0,2) circle;
\draw
  (0,2.3) node[dcolorc] {$N$}
  (-0.3,0.3) node[dcolorb] {$Z_2$}
  (-1.4,0) node[dcolorb] {$z_2$}
  (1.5,1) node[dcolora] {$Z_1$}
  (1.9,-0.75) node[dcolora] {$z_1$};
\end{tikzpicture}
\end{center}
\end{frame}


\begin{frame}{复球面: 与实数无穷的联系}
\onslide<+->
它和实数中 $\pm\infty$ 有什么联系呢?
\onslide<+->
选取上述图形的一个截面来看, 实轴可以和圆周去掉一点建立一一对应.
\onslide<+->
同样的, 当 $|x|$ 越来越大时, 其对应圆周上点也越来越接近 $N$.
\onslide<+->
所以实数中的 $\pm\infty$ 在复球面上或闭复平面上就是 $\infty$, 只是在实数时我们往往还关心它的符号, 所以区分正负.

\onslide<2->
\begin{center}
\begin{tikzpicture}
\filldraw[cstcurve,fill=black!10] (0,1) circle (1);
\draw[cstdash] (0,0)--(0,2);
\draw[cstaxis] (-2,0)--(2.5,0);
\draw[cstcurve,cstarrowto,dcolora] (0,2)--(2.2,0);
\fill[cstdot,dcolora] (1,1.1) circle;
\draw[cstcurve,cstarrowto,dcolorb] (0,2)--(-1,0);
\fill[cstdot,dcolorb] (-0.8,0.4) circle;
\fill[cstdot,dcolorc] (0,2) circle;
\draw
  (0,2.3) node[dcolorc] {$N$}
  (-0.4,0.5) node[dcolorb] {$X_2$}
  (-1,-0.2) node[dcolorb] {$x_2$}
  (1.4,1.2) node[dcolora] {$X_1$}
  (2.1,-0.2) node[dcolora] {$x_1$};
\end{tikzpicture}
\end{center}
\end{frame}


\begin{frame}{$\infty$ 的性质}
\onslide<+->
$\infty$ 的实部、虚部和辐角无意义, 规定 $|\infty|=+\infty$.
\onslide<+->
约定
\[z\pm \infty=\infty\pm z=\infty\quad (z\neq \infty),\]
\vspace{-\baselineskip}
\onslide<+->
\[z\cdot\infty=\infty\cdot z=\infty\quad (z\neq 0),\]
\vspace{-\baselineskip}
\onslide<+->
\[\frac z\infty=0\ (z\neq\infty),\quad
\frac\infty z=\infty\ (z\neq 0),\quad
\frac z0=\infty\ (z\neq 0).\]

\onslide<+->
根据开集的定义可知, 包含 $z$ 的任何一个开集均包含 $z$ 的一个邻域.
\onslide<+->
由此可知, 将极限定义中的 $\varepsilon$-邻域换成开邻域(包含 $z$ 的开集)并不影响极限的定义.
\onslide<+->
在复球面上的任意一点, 可以自然地定义 $z\in\BC^*$ 的开邻域.
\onslide<+->
它在上述对应下的像便是 $z$ 的一个开邻域.
\end{frame}


\begin{frame}{数列收敛的等价刻画}
\begin{theorem}
设 $z_n=x_n+y_ni,z=x+yi$, 则
\[\lim_{n\to\infty}z_n=z\iff \lim_{n\to\infty}x_n=x,\lim_{n\to\infty}y_n=y.\]
\end{theorem}

\begin{proof}
由三角不等式
\[|x_n-x|,|y_n-y|\le|z_n-z|\le|x_n-x|+|y_n-y|\]
易证.
\end{proof}
\end{frame}


\begin{frame}{例题: 数列的敛散性}
\begin{example}
设 $z_n=\left(1+\dfrac1n\right)\exp\left(\frac{\pi i}n\right)$. 数列 $\{z_n\}$ 是否收敛?
\end{example}
\begin{solution}
由于
\[x_n=\left(1+\frac1n\right)\cos\frac\pi n\to 1,\quad
y_n=\left(1+\frac1n\right)\sin\frac\pi n\to 0.\]
\onslide<+->
因此 $\{z_n\}$ 收敛且 $\lim\limits_{n\to\infty}z_n=1$.
\end{solution}
\end{frame}


\begin{frame}{函数的极限}
\begin{definition}
设函数 $f(z)$ 在点 $z_0$ 的某个去心邻域内有定义.
\onslide<+->
如果存在复数 $A$ 使得对 $A$ 的任意邻域 $U,\exists\delta>0$ 使得当 $z\in\stackrel\circ U(z_0,\delta)$ 时, 有 $f(z)\in U$, 则称 $A$ 为 \emph{$f(z)$ 当 $z\to z_0$ 时的极限}, 记为 \emph{$\lim\limits_{z\to z_0}f(z)=A$} 或 \emph{$f(z)\to A (z\to z_0)$}.
\end{definition}

\onslide<+->
对于 $z_0=\infty$ 或 $A=\infty$ 的情形, 也可以用上述定义统一描述.

\onslide<+->
通常我们说极限存在是不包括 $\lim f(z)=\infty$ 的情形的.
\end{frame}


\begin{frame}{与实函数极限之联系}
\onslide<+->
\emph{每一个复变函数 $w=f(z)=u+iv$ 等价于给了两个二元实变函数}
\[\emph{u=u(x,y),\qquad v=v(x,y).}\]
\onslide<+->
例如
\[w=z^2=(x^2-y^2)+i\cdot 2xy,\]
\[u(x,y)=x^2-y^2,\quad v(x,y)=2xy.\]
\onslide<+->
通过与二元实函数的极限对比可知, 复变函数的极限和二元实函数的极限定义是类似的.
\onslide<+->
$z\to z_0$ 可以是沿着任意一条曲线趋向于 $z_0$, 或者看成 $z$ 是在一个开圆盘内任意的点逐渐地靠拢 $z_0$.

\begin{theorem}
设 $f(z)=u(x,y)+iv(x,y),z_0=x_0+y_0i,A=u_0+v_0i$, 则
\[\lim_{z\to z_0}f(z)=A\iff
\lim_{\substack{x\to x_0\\y\to y_0}}u(x,y)=u_0,\quad
\lim_{\substack{x\to x_0\\y\to y_0}}v(x,y)=v_0.\]
\vspace{-0.5\baselineskip}
\end{theorem}

\begin{proof}
由三角不等式
\[|u-u_0|,|v-v_0|\le|z-z_0|\le|u-u_0|+|v-v_0|\]
易证.
\end{proof}
\end{frame}


\begin{frame}{极限的四则运算}
\onslide<+->
由此可知极限的四则运算法则对于复变函数也是成立的.
\begin{theorem}
设 $\lim\limits_{z\to z_0}f(z)=A,\lim\limits_{z\to z_0}g(z)=B$, 则
\begin{enumerate}
\item $\lim\limits_{z\to z_0}(f\pm g)(z)=A\pm B$;
\item $\lim\limits_{z\to z_0}(fg)(z)=AB$;
\item 当 $B\neq 0$ 时, $\lim\limits_{z\to z_0}\left(\dfrac fg\right)(z)=\dfrac AB$.
\end{enumerate}
\end{theorem}
\end{frame}


\begin{frame}{例题: 判断函数极限是否存在}
\begin{example}
证明当 $z\to0$ 时, 函数 $f(z)=\dfrac{\Re z}{|z|}$ 的极限不存在.
\end{example}
\begin{proof}
令 $z=x+yi$, 则 $f(z)=\dfrac x{\sqrt{x^2+y^2}}$.
\onslide<+->
因此
\[u(x,y)=\frac x{\sqrt{x^2+y^2}},\quad v(x,y)=0.\]

\onslide<+->
当 $z$ 沿着直线 $y=0$ 左右两侧趋向于 $0$ 时, 则 $u(x,y)\to\pm1$.
\onslide<+->
因此 $\lim\limits_{\substack{x\to 0\\y\to0}}u(x,y)$ 不存在,
\onslide<+->
从而 $\lim\limits_{z\to z_0}f(z)$ 不存在.
\end{proof}
\end{frame}


\begin{frame}{函数的连续性}
\begin{definition}
\begin{itemize}
\item 如果 $\lim\limits_{z\to z_0}f(z)=f(z_0)$, 则称 $f(z)$ 在 \emph{$z_0$ 处连续}.
\item 如果 $f(z)$ 在区域 $D$ 内处处连续, 则称 $f(z)$ 在 \emph{$D$ 内连续}.
\end{itemize}
\end{definition}

\begin{theorem}
函数 $f(z)=u(x,y)+iv(x,y)$ 在 $z_0=x_0+iy_0$ 处连续当且仅当 $u(x,y)$ 和 $v(x,y)$ 在 $(x_0,y_0)$ 处连续.
\end{theorem}

\onslide<+->
例如 $f(z)=\ln(x^2+y^2)+i(x^2-y^2)$.
\onslide<+->
$u(x,y)=\ln(x^2+y^2)$ 除原点外处处连续, $v(x,y)=x^2-y^2$ 处处连续.
\onslide<+->
因此 $f(z)$ 在 $z\neq0$ 处连续.
\end{frame}


\begin{frame}{连续函数的性质}
\begin{theorem}
\begin{itemize}
\item 在 $z_0$ 处连续的两个函数 $f(z),g(z)$ 之和、差、积、商($g(z_0)\neq 0$) 在 $z_0$ 处仍然连续.
\item 如果函数 $g(z)$ 在 $z_0$ 处连续, 函数 $f(w)$ 在 $g(z_0)$ 处连续, 则 $f(g(z))$ 在 $z_0$ 处连续.
\end{itemize}
\end{theorem}

\onslide<+->
显然 $f(z)=z$ 是处处连续的,
\onslide<+->
故多项式函数
\[P(z)=a_0+a_1z+a_2z^2+\cdots+a_nz^n\]
也处处连续,
\onslide<+->
有理函数 $\dfrac{P(z)}{Q(z)}$ 在 $Q(z)$ 的零点以外处处连续.

\onslide<+->
有时候我们会遇到在曲线上连续的函数, 它指的是当 $z$ 沿着该曲线趋向于 $z_0$ 时, $f(z)\to f(z_0)$.
\onslide<+->
对于闭合曲线或包含端点的曲线段, 其之上的连续函数 $f(z)$ 是有界的.
\end{frame}


\begin{frame}{例题: 函数连续性的判定}
\begin{example}
证明: 如果 $f(z)$ 在 $z_0$ 连续, 则 $\ov{f(z)}$ 在 $z_0$ 也连续.
\end{example}
\begin{proof}
\indent
设 $f(z)=u(x,y)+iv(x,y),z_0=x_0+iy_0$.
\onslide<+->
那么 $u(x,y),v(x,y)$ 在 $(x_0,y_0)$ 连续.
\onslide<+->
从而 $-v(x,y)$ 也在 $(x_0,y_0)$ 连续.
\onslide<+->
所以 $\ov{f(z)}=u(x,y)-iv(x,y)$ 在 $(x_0,y_0)$ 连续.

\indent
\onslide<+->
另一种看法是, 函数 $g(z)=\ov z=x-iy$ 处处连续,
\onslide<+->
从而 $g(f(z))=\ov{f(z)}$ 在 $z_0$ 处连续.
\end{proof}
\end{frame}


\begin{frame}{注记}
\onslide<+->
可以看出, 在极限和连续性上, 复变函数和两个二元实函数没有什么差别.
\onslide<+->
那么复变函数和多变量微积分的差异究竟是什么导致的呢?
\onslide<+->
归根到底就在于 $\BC$ 是一个域, 上面可以做除法.

\onslide<+->
这就导致了复变函数有\alert{导数}, 而不是像多变量实函数只有偏导数.
\onslide<+->
这种特性使得可导的复变函数具有整洁优美的性质, 我们将在下一章来逐步揭开它的神秘面纱.
\end{frame}

