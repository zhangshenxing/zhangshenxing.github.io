\begin{frame}{课程安排}
\onslide<+->
本课程共$10$周$40$课时, 自2022年9月30日至2022年11月3日. 
\onslide<+->
\emph{课程QQ群}:(入群答案\textbf{1400261B})
\onslide<+->
\begin{itemize}[<*>]
\item 009班(电信工) \emph{\textbf{476993411}}
\item 010班(光信息和智感工) \emph{\textbf{672903188}}
\end{itemize}
\onslide<+->
\emph{教材}:
\begin{tikzpicture}[overlay,xshift=6.5cm,yshift=-1cm]
\draw
	(0,0) node {\includegraphics[height=2.5cm]{misc/book1.jpg}}
	(2.5,0) node {\includegraphics[height=2.5cm]{misc/book2.png}};
\end{tikzpicture}
\begin{itemize}[<*>]
\item 西交高数教研室《复变函数》
\item 张元林《积分变换》
\end{itemize}
\onslide<+->
\emph{成绩构成}:
\begin{itemize}
\item \alert{作业 15\%}, 每章交一次
\item \alert{课堂测验 25\%}, 一共3次,  取最高的两次
\item \alert{期末报告 10\%}
\item \alert{期末考试 50\%}, 至少45分才计算总评
\end{itemize}
\end{frame}


\begin{frame}{复变函数的应用}
\onslide<+->
复变函数的应用非常广泛, 它包括:
\begin{itemize}
\item \emph{数学}中的代数、数论、几何、分析、动力系统……
\item \emph{物理学}中流体力学、材料力学、电磁学、光学、量子力学……
\item \emph{信息学}、\emph{电子学}、\emph{电气工程}……
\end{itemize}

\onslide<+->
可以说复变函数应用之广, 在大学数学课程中仅次于高等数学和线性代数. 
\end{frame}


\begin{frame}{课程内容和学习方法}
\onslide<+->
本课程主要研究下述问题:
\begin{itemize}
\item 复数的由来, 以及复数的基本要素. (\emph{\S~1.1})
\item 复数的运算法则和性质. (\emph{\S\S~1.1-1.3})
\item 复变量函数的定义, 以及与实变量函数的异同点. (\emph{\S\S~1.4-1.6}) 
\item 复变函数的解析性的刻画, 即复变函数的微分理论. (\emph{\S\S~2.1-3.3})
\item 复变函数的积分、级数和留数理论. (\emph{\S\S~3.4-5.3})
\item 积分变换及其应用. (\emph{\S\S~6.1-6.3})
\end{itemize}
\onslide<+->
\begin{center}
\begin{tikzpicture}[node distance=30pt]
\node[cstnodeg,align=center]            (1)  {课前\\预习};
\coordinate[below=of 1] (2);
\node[cstnodeg,align=center,right=of 2] (3)  {认真听课\\记好笔记};
\coordinate[above=of 3] (4);
\node[cstnodeg,align=center,right=of 4] (5)  {课后先阅读\\一遍教材};
\coordinate[below=of 5] (6);
\node[cstnodeg,align=center,right=of 6] (7)  {结合笔记\\汇总知识点};
\coordinate[above=of 7] (8);
\node[cstnodeg,align=center,right=of 8] (9)  {再用作业检\\测学习效果};
\draw[cstnarrow,dcolorc] (1.south) to [bend right] (3.west);
\draw[cstnarrow,dcolorc] (3.north) to [bend left] (5.west);
\draw[cstnarrow,dcolorc] (5.south) to [bend right] (7.west);
\draw[cstnarrow,dcolorc] (7.north) to [bend left] (9.west);
\end{tikzpicture}
\end{center}
\end{frame}

