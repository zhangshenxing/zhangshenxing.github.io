\section{复变函数积分的概念}

\subsection{复变函数积分的定义}

\begin{frame}{有向曲线}
	\onslide<+->
	设 $C$ 是平面上一条光滑或逐段光滑的连续曲线,
	\onslide<+->
	也就是说它的参数方程 $z=z(t),a\le t\le b$ 除去有限个点之外都有非零导数.

	\onslide<+->
	固定它的一个方向, 称为\emph{正方向}, 则我们得到一条\emph{有向曲线}.
	\onslide<+->
	和这条曲线方向相反的记作 $C^-$, 它的方向被称为该曲线\emph{负方向}.

	\onslide<+->
	对于闭路, 它的\alert{正方向总是指逆时针方向}, 负方向总是指顺时针方向.
	\onslide<+->
	以后我们不加说明的话\alert{默认是正方向}.

	\onslide<1->
	\begin{center}
		\begin{tikzpicture}
			\draw[cstaxis](-2.5,0)--(2.5,0);
			\draw[cstaxis](0,-0.4)--(0,2.5);
			\draw[cstcurve,dcolorb,smooth,domain=-35:125] plot ({1.4+1.3*cos(\x)}, {1.1+1.3*sin(\x)});
			\draw (3.4,0.3) node[dcolorb] {$A=z(a)$};
			\draw (0.8,1.9) node[dcolorb] {$B=z(b)$};
			\fill[cstdot,dcolorb] ({1.4+1.3*cos(-35)}, {1.1+1.3*sin(-35)}) circle;
			\fill[cstdot,dcolorb] ({1.4+1.3*cos(125)}, {1.1+1.3*sin(125)}) circle;
			\draw[cstcurve,dcolorb,smooth,domain=40:45,cstarrow1to,visible on=<3->] plot ({1.4+1.3*cos(\x)}, {1.1+1.3*sin(\x)});
			\draw[cstcurve,dcolora,smooth,domain=-65:30,visible on=<5->,cstarrow1to] plot ({-1.4+0.7*cos(\x)}, {1.4+0.7*sin(\x)});
			\draw[cstcurve,dcolora,smooth,domain=25:120,visible on=<5->,cstarrow1to] plot ({-1.4+0.7*cos(\x)}, {1.4+0.7*sin(\x)});
			\draw[cstcurve,dcolora,smooth,domain=115:210,visible on=<5->,cstarrow1to] plot ({-1.4+0.7*cos(\x)}, {1.4+0.7*sin(\x)});
			\draw[cstcurve,dcolora,smooth,domain=205:300,visible on=<5->,cstarrow1to] plot ({-1.4+0.7*cos(\x)}, {1.4+0.7*sin(\x)});
		\end{tikzpicture}
	\end{center}
\end{frame}


\begin{frame}{复变函数积分的定义}
	\onslide<+->
	所谓的复变函数积分, 本质上仍然是第二类曲线积分.
	\onslide<+->
	设复变函数 $w=f(z)=u(x,y)+iv(x,y)$ 定义在区域 $D$ 内, 有向曲线 $C$ 包含在 $D$ 中.
	\onslide<+->
	形式地展开
		\[f(z)\diff z=(u+iv)(\!\diff x+i\diff y)=(u\diff x-v\diff y)+i(u\diff y+v\diff x).\]
	\vspace{-\baselineskip}
	
	\onslide<+->
	\begin{definition}
		如果下述右侧两个线积分均存在, 则定义
		\[\int_C f(z)\diff z=\int_C(u\diff x-v\diff y)+i\int_C(v\diff x+u\diff y)\]
		为\emph{函数 $f(z)$ 沿曲线 $C$ 的积分}.
	\end{definition}
\end{frame}


\begin{frame}{复变函数积分的定义}
	\onslide<+->
	当然, 我们也可以像线积分那样通过分割来定义.
	\onslide<+->
	在曲线 $C$ 上依次选择分点 $z_0=A,z_1,\dots,z_n=B$.
	\onslide<+->
	然后在每一段弧上任取 $\zeta_k\in\warc{z_{k-1}z_k}$ 并作和式
	\[S_n=\sum_{k=1}^n f(\zeta_k)\Delta z_k,\quad \Delta z_k=z_k-z_{k-1}.\]
	\onslide<+->
	然后称 $n\to\infty$, 分割的弧长 $\ra 0$ 时 $S_n$ 的极限为复变函数积分.
	\onslide<+->
	这二者是等价的.

	\onslide<2->
	\begin{center}
		\begin{tikzpicture}
			\draw[cstcurve,dcolorc,smooth,domain=0:360] plot ({0.015*\x},{0.6*sin(\x)});
			\draw[cstcurve,dcolorc,smooth,domain=195:200,cstarrowto] plot ({0.015*\x},{0.6*sin(\x)});
			\fill[cstdot,dcolorc] (0,0) circle;
			\fill[cstdot,dcolorb,visible on=<3->]({0.015*32},{0.6*sin(32)}) circle;
			\fill[cstdot,dcolora,visible on=<2->]({0.015*64},{0.6*sin(64)}) circle;
			\fill[cstdot,dcolorb,visible on=<3->]({0.015*96},{0.6*sin(96)}) circle;
			\fill[cstdot,dcolora,visible on=<2->]({0.015*128},{0.6*sin(128)}) circle;
			\fill[cstdot,dcolorb,visible on=<3->]({0.015*160},{0.6*sin(160)}) circle;
			\fill[cstdot,dcolora,visible on=<2->]({0.015*232},{0.6*sin(232)}) circle;
			\fill[cstdot,dcolorb,visible on=<3->]({0.015*264},{0.6*sin(264)}) circle;
			\fill[cstdot,dcolora,visible on=<2->]({0.015*296},{0.6*sin(296)}) circle;
			\fill[cstdot,dcolorb,visible on=<3->]({0.015*328},{0.6*sin(328)}) circle;
			\fill[cstdot,dcolorc] ({0.015*360},0) circle;
			\draw
			({0.015*32},{0.6*sin(32)-0.3}) node[dcolorb,visible on=<3->] {$\zeta_1$}
			({0.015*64},{0.6*sin(64)+0.3}) node[dcolora,visible on=<2->] {$z_1$}
			({0.015*96},{0.6*sin(96)-0.3}) node[dcolorb,visible on=<3->] {$\zeta_2$}
			({0.015*128},{0.6*sin(128)+0.3}) node[dcolora,visible on=<2->] {$z_2$}
			({0.015*160},{0.6*sin(160)-0.3}) node[dcolorb,visible on=<3->] {$\zeta_3$}
			({0.015*232},{0.6*sin(232)+0.3}) node[dcolora,visible on=<2->] {$z_{n-2}$}
			({0.015*264},{0.6*sin(264)-0.3}) node[dcolorb,visible on=<3->] {$\zeta_{n-1}$}
			({0.015*296},{0.6*sin(296)+0.3}) node[dcolora,visible on=<2->] {$z_{n-1}$}
			({0.015*328},{0.6*sin(328)-0.3}) node[dcolorb,visible on=<3->] {$\zeta_n$}
			({0.015*180},{0.6*sin(180)+0.5}) node[dcolora,visible on=<2->] {$\ddots$}
			(-0.3,0) node[dcolorc] {$A$}
			({0.015*360+0.3},0) node[dcolorc] {$B$};
		\end{tikzpicture}
	\end{center}
\end{frame}


\begin{frame}{复变函数积分的存在性}
	\onslide<+->
	如果 $C$ 是闭曲线, 则该积分记为 \alert{$\displaystyle\oint_Cf(z)\diff z$}.
	\onslide<+->
	此时该积分不依赖端点的选取.

	\onslide<+->
	如果 $C$ 是实轴上的区间 $[a,b]$ 且 $f(z)=u(x)$, 
	\onslide<+->
	则
	\[\int_Cf(z)\diff z=\int_a^bf(z)\diff z=\int_a^b u(x)\diff x\]
	就是黎曼积分.

	\onslide<+->
	根据线积分的存在性条件可知:
	\onslide<+->
	\begin{theorem}
	如果 $f(z)$ 在 $D$ 内连续, $C$ 是光滑曲线, 则 $\displaystyle\int_Cf(z)\diff z$ 总存在.
	\end{theorem}
\end{frame}


\subsection{复变函数积分的计算法}

\begin{frame}{复变函数积分的计算法}
	\onslide<+->
	线积分中诸如变量替换等技巧可以照搬过来使用.
	\onslide<+->
	设
	\[C:z(t)=x(t)+iy(t),\quad a\le t\le b\]
	是一条光滑有向曲线, 且正方向为 $t$ 增加的方向.
	\onslide<+->
	则 $\diff z=z'(t)\diff t$,
	
	\onslide<+->
	\begin{alertblock}{复变函数积分的一般计算方法}
		\[\int_Cf(z)\diff z=\int_a^b f(z)z'(t)\diff t.\]
	\end{alertblock}
	\onslide<+->
	\noindent 如果 $C$ 的正方向是从 $z(b)$ 到 $z(a)$, 则需要交换右侧积分的上下限.

	\onslide<+->
	如果 $C$ 是逐段光滑的, 则相应的积分就是各段的积分之和.
	\onslide<+->
	以后我们\emph{只考虑逐段光滑曲线上的连续函数的积分}.
\end{frame}


\begin{frame}{典型例题: 计算复变函数沿曲线的积分}
	\onslide<+->
	\begin{example}
		求 $\displaystyle\int_Cz\diff z$, 其中 $C$ 是从原点到点 $3+4i$ 的直线段.
	\end{example}

	\onslide<+->
	\begin{solution}
		由于 $z=(3+4i)t,0\le t\le 1$,
		\onslide<+->{因此
		\begin{flalign*}
			\int_Cz\diff z&=\int_0^1(3+4i)t\cdot(3+4i)\diff t&\\
			&\visible<+->{=(3+4i)^2\int_0^1t\diff t}&\\
			&\visible<+->{=\frac12(3+4i)^2=-\frac72+12i.}&
		\end{flalign*}}
		\vspace{-\baselineskip}
	\end{solution}
	
	\onslide<1->
	\begin{tikzpicture}[overlay,xshift=7cm,yshift=1.3cm]
		\draw[cstaxis](0,0)--(3,0);
		\draw[cstaxis](0,0)--(0,2.5);
		\draw[cstcurve,dcolora,cstarrowto](0,0)--(1.5,2);
		\draw
			(2.8,2) node[dcolora,align=center,visible on=<2->] {$z=(3+4i)t$\\$0\le t\le 1$};
	\end{tikzpicture}
\end{frame}


\begin{frame}{典型例题: 计算复变函数沿曲线的积分}
	\onslide<+->
	\begin{example}
		求 $\displaystyle\int_Cz\diff z$, 其中 $C$ 是抛物线 $y=\dfrac49x^2$ 上从原点到点 $3+4i$ 的曲线段.
	\end{example}

	\onslide<+->
	\begin{solution}
		由于 $z=t+\dfrac49it^2,0\le t\le 3$,
		\onslide<+->{因此
		\begin{flalign*}
			\int_Cz\diff z&=\int_0^3\left(t+\frac{4}9it^2\right)\cdot\left(1+\frac89it\right)\diff t&\\
			&\visible<+->{=\int_0^3\left(t+\frac43it^2-\frac{32}{81}t^3\right)\diff t}&\\
			&\visible<+->{=\left(\frac12t^2+\frac49it^3-\frac8{81}t^4\right)\Big|_0^3}
			\visible<+->{=-\frac72+12i.}&
		\end{flalign*}}
		\vspace{-\baselineskip}
	\end{solution}

	\onslide<1->
	\begin{tikzpicture}[overlay,xshift=7cm,yshift=2cm]
		\draw[cstaxis](0,0)--(3,0);
		\draw[cstaxis](0,0)--(0,2.5);
		\draw[cstcurve,dcolora,domain=0:1.5,cstarrowto] plot({\x},{8*\x*\x/9});
		\draw
			(2.8,2) node[dcolora,align=center,visible on=<2->] {$z=t+\dfrac49it^2$\\[1mm]$0\le t\le 3$};
	\end{tikzpicture}
\end{frame}


\begin{frame}{典型例题: 计算复变函数沿曲线的积分}
	\onslide<+->
	\begin{example}
		求 $\displaystyle\int_C\Re z\diff z$, 其中 $C$ 是从原点到点 $1+i$ 的直线段.
	\end{example}

	\onslide<+->
	\begin{solution}
		由于 $z=(1+i)t,0\le t\le 1$,
		\onslide<+->{因此 $\Re z=t$,
		}\onslide<+->{
			\begin{flalign*}
				\int_C\Re z\diff z&=\int_0^1t\cdot(1+i)\diff t&\\
				&\visible<+->{=(1+i)\int_0^1t\diff t}&\\
				&\visible<+->{=\frac{1+i}2.}&
			\end{flalign*}}
		\vspace{-\baselineskip}
	\end{solution}

	\onslide<1->
	\begin{tikzpicture}[overlay,xshift=7.8cm,yshift=1cm]
		\draw[cstaxis](0,0)--(3,0);
		\draw[cstaxis](0,0)--(0,2.5);
		\draw[cstcurve,dcolora,cstarrowto](0,0)--(2,2);
		\draw
			(2.3,0.9) node[dcolora,align=center,visible on=<2->] {$z=(1+i)t$\\$0\le t\le 1$};
	\end{tikzpicture}
\end{frame}


\begin{frame}{典型例题: 计算复变函数沿曲线的积分}
	\onslide<+->
	\begin{example}
		求 $\displaystyle\int_C\Re z\diff z$, 其中 $C$ 是从原点到点 $i$ 再到 $1+i$ 的折线段.
	\end{example}
	\onslide<+->
	\begin{solution}
		\onslide<+->{第一段 $z=it,0\le t\le 1, \Re z=0$,}

		\onslide<+->{第二段 $z=t+i$, $0\le t\le 1$, $\Re z=t$.
		}\onslide<+->{因此
			\[\int_C\Re z\diff z=\int_0^1 t\diff t=\frac12.\]}
		\onslide<2->
		\vspace{-1.4\baselineskip}
		\begin{center}
			\begin{tikzpicture}
				\draw[cstaxis](0,0)--(2.5,0);
				\draw[cstaxis](0,0)--(0,2);
				\draw[cstcurve,cstarrowto,dcolora](0,0)--(0,1.5);
				\draw[cstcurve,cstarrowto,dcolorb](0,1.5)--(1.5,1.5);
				\draw
					(-1.1,1.2) node[dcolora,align=center,visible on=<3->] {$z=it$\\$0\le t\le 1$}
					(2.5,1.5) node[dcolorb,align=center,visible on=<3->] {$z=t+i$\\$0\le t\le 1$};
			\end{tikzpicture}
		\end{center}
		\vspace{-0.6\baselineskip}
	\end{solution}
\end{frame}


\begin{frame}{典型例题: 计算复变函数沿曲线的积分}
	\onslide<+->
	可以看出, 即便起点和终点相同, 沿不同路径 $f(z)=\Re z$ 的积分也可能不同.
	\onslide<+->
	而 $f(z)=z$ 的积分则只和起点和终点位置有关, 与路径无关.
	\onslide<+->
	原因在于 $f(z)=z$ 是处处解析的, 我们以后会详加解释.
	
	\onslide<+->
	\begin{exercise}
		求 $\displaystyle\int_C\Im z\diff z=$\fillblank[3cm][2mm]{\visible<+->{$-\dfrac12+\dfrac i2$}}, 其中 $C$ 是从原点沿 $y=x$ 到点 $1+i$ 再到 $i$ 的折线段.
		\begin{center}
			\begin{tikzpicture}
				\draw[cstaxis](0,0)--(2.5,0);
				\draw[cstaxis](0,0)--(0,2.5);
				\draw[cstcurve,cstarrowto,dcolora](0,0)--(2,2);
				\draw[cstcurve,cstarrowto,dcolorb](2,2)--(0,2);
			\end{tikzpicture}
		\end{center}
	\end{exercise}
\end{frame}


\begin{frame}{例题: 计算复变函数沿圆周的积分}
	\onslide<+->
	\begin{example}
		求 $\displaystyle\oint_{|z-z_0|=r}\frac{\diff z}{(z-z_0)^{n+1}}$, 其中 $n$ 为整数.
	\end{example}

	\onslide<+->
	\begin{solution}
		$C: |z-z_0|=r$ 的参数方程为 $z=z_0+re^{i\theta},0\le \theta\le 2\pi$.

		\onslide<+->{于是 $\diff z=ire^{i\theta}\diff \theta$,}
		\onslide<+->{
			\begin{flalign*}
				\oint_C\frac{\diff z}{(z-z_0)^{n+1}}
				=\int_0^{2\pi}i(re^{i\theta})^{-n}\diff\theta
				\visible<+->{=ir^{-n}\int_0^{2\pi}e^{-in\theta}\diff\theta.}&&
			\end{flalign*}
		}\onslide<+->{当 $n=0$ 时, 该积分值为 $2\pi i$.}
		
		\vspace{5mm}
		\onslide<+->{当 $n\neq 0$ 时, 该积分值 $=\dfrac{ir^{-n}}{-in}e^{-in\theta}\big|_0^{2\pi}=0$.}
	\end{solution}
	\onslide<2->
	\begin{tikzpicture}[overlay,xshift=9.5cm,yshift=1.6cm]
		\draw[cstcurve,dcolorb] ({1.3*cos(120)},{1.3*sin(120)})coordinate (C) -- (0,0)coordinate (B) --(1.3,0)coordinate (A) pic [draw,dcolorb,"$\theta$",angle eccentricity=1.7,angle radius=3mm] {angle};
		\fill[cstdot,dcolorb](0,0) circle;
		\draw[cstcurve,dcolora,cstarrow1from]({1.3*cos(45)},{1.3*sin(45)}) arc (45:-320:1.3);
		\draw
			(-0.5,0.5) node[dcolorb] {$r$}
			(0,-0.3) node[dcolorb] {$z_0$};
	\end{tikzpicture}
\end{frame}


\begin{frame}{例题: 计算复变函数沿圆周的积分}
	\onslide<+->
	所以
	\begin{alertblock@}
		\[\oint_{|z-z_0|=r}\frac{\diff z}{(z-z_0)^{n+1}}=\begin{cases}2\pi i,&n=0;\\0,&n\neq 0.\end{cases}\]
	\end{alertblock@}

	\onslide<+->
	这个积分以后经常用到, 它的特点是积分值与圆周的圆心和半径都无关.
\end{frame}


\begin{frame}{积分的性质}
	\onslide<+->
	\begin{theorem}
		\begin{enumerate}
			\item $\displaystyle\int_Cf(z)\diff z=-\displaystyle\int_{C^-}f(z)\diff z$.
			\item $\displaystyle\int_Ckf(z)\diff z=k\displaystyle\int_Cf(z)\diff z$.
			\item $\displaystyle\int_C[f(z)\pm g(z)]\diff z
			=\displaystyle\int_Cf(z)\diff z\pm\displaystyle\int_Cg(z)\diff z$.
				\begin{tikzpicture}[overlay,xshift=0cm,yshift=1.2cm]
					\draw[decorate,thick,decoration={brace,amplitude=5},dcolorb] (0,1.3)--(0,-1.3);
					\draw (1.2,0) node[dcolorb] {线性性质};
				\end{tikzpicture}
			\item (\emph{长大不等式}) 设 $C$ 的长度为 $L$, $f(z)$ 在 $C$ 上满足 $|f(z)|\le M$, 则
			\[\alert{\abs{\int_Cf(z)\diff z}\le\int_C|f(z)|\diff s\le ML}.\]
		\end{enumerate}
	\end{theorem}
\end{frame}


\begin{frame}{积分的性质}
	\onslide<+->
	\begin{proof}
		我们来证明下\enumnum4.
		\onslide<+->{由
			\[\abs{\sum_{k=1}^n f(\zeta_k)\Delta z_k}
			\le\sum_{k=1}^n|f(\zeta_k)\Delta z_k|
			\le\sum_{k=1}^n|f(\zeta_k)|\Delta s_k
			\le M\sum_{k=1}^n\Delta s_k\]
		}\onslide<+->{可知
			\[\abs{\int_Cf(z)\diff z}\le\int_C|f(z)|\diff s\le ML.\qedhere\]}
	\end{proof}

	\onslide<+->
	长大不等式常常用于证明等式: 估算一个积分和一个具体的数值之差不超过任意给定的 $\varepsilon$, 从而得到二者相等.
\end{frame}


\begin{frame}{例题: 长大不等式的应用*}
	\onslide<+->
	\begin{example}
		设 $f(z)$ 在 $z\neq a$ 处连续, 且 $\lim\limits_{z\to a}(z-a)f(z)=k$, 则
		\vspace{-0.9\baselineskip}
		\[\lim_{r\to0}\oint_{|z-a|=r}f(z)\diff z=2\pi ik.\]
	\end{example}

	\onslide<+->
	\begin{proof}
		$\forall \varepsilon>0,\exists\delta>0$ 使得当 $|z-a|<\delta$ 时, $|(z-a)f(z)-k|\le\varepsilon$.
		\onslide<+->{当 $0<r<\delta$ 时,
		\vspace{-0.5\baselineskip}
		\begin{align*}
		&\peq\abs{\oint_{|z-a|=r}f(z)\diff z-2\pi i k}
		=\abs{\oint_{|z-a|=r}\left[f(z)-\frac k{z-a}\right]\diff z}\\
		&\visible<+->{=\abs{\oint_{|z-a|=r}\frac{(z-a)f(z)-k}{z-a}\diff z}}
		\visible<+->{\le \frac{\varepsilon}r\cdot 2\pi r=2\pi\varepsilon.}
		\end{align*}}
		\onslide<+->{由于 $\varepsilon$ 是任意的, 因此命题得证.}
	\end{proof}
\end{frame}

