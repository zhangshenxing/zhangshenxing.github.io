\section{复变函数积分的概念}


\begin{frame}{有向曲线}
\onslide<+->
设 $C$ 是平面上一条光滑或逐段光滑的连续曲线,
\onslide<+->
也就是说 $z(t),a\le t\le b$ 除去有限个点之外都有非零导数.

\onslide<+->
固定它的一个方向, 称为\markdef{正方向}, 则我们得到一条\markdef{有向曲线}.
\onslide<+->
和这条曲线方向相反的记作 $C^-$, 它的方向被称为该曲线\markdef{负方向}.

\onslide<+->
对于闭路, 它的\markatt{正方向总是指逆时针方向}, 负方向总是指顺时针方向.
\onslide<+->
以后我们不加说明的话\markatt{默认是正方向}.

\onslide<1->
\begin{center}
\begin{tikzpicture}
\draw[cstaxis](-0.4,0)--(2.5,0);
\draw[cstaxis](0,-0.4)--(0,2.5);
\draw[cstcurve,dcolorb,smooth,domain=-35:125] plot ({1.1+1.3*cos(\x)}, {1.1+1.3*sin(\x)});
\draw[cstcurve,dcolorb,smooth,domain=40:45,cstarrow1to,visible on=<3->] plot ({1.1+1.3*cos(\x)}, {1.1+1.3*sin(\x)});
\draw[cstcurve,dcolora,smooth,domain=-65:30,visible on=<5->,cstarrow1to] plot ({1.1+0.7*cos(\x)}, {1.1+0.7*sin(\x)});
\draw[cstcurve,dcolora,smooth,domain=25:120,visible on=<5->,cstarrow1to] plot ({1.1+0.7*cos(\x)}, {1.1+0.7*sin(\x)});
\draw[cstcurve,dcolora,smooth,domain=115:210,visible on=<5->,cstarrow1to] plot ({1.1+0.7*cos(\x)}, {1.1+0.7*sin(\x)});
\draw[cstcurve,dcolora,smooth,domain=205:300,visible on=<5->,cstarrow1to] plot ({1.1+0.7*cos(\x)}, {1.1+0.7*sin(\x)});
\draw (2.4,0.3) node[dcolorb] {$A$};
\draw (0.4,1.9) node[dcolorb] {$B$};
\fill[cstdot,dcolorb] ({1.1+1.3*cos(-35)}, {1.1+1.3*sin(-35)}) circle;
\fill[cstdot,dcolorb] ({1.1+1.3*cos(125)}, {1.1+1.3*sin(125)}) circle;
\end{tikzpicture}
\end{center}
\end{frame}


\begin{frame}{复变函数积分的定义}
\onslide<+->
所谓的复变函数积分, 本质上仍然是第二类曲线积分.
\onslide<+->
设复变函数 $w=f(z)=u(x,y)+iv(x,y)$ 定义在区域 $D$ 内, 有向曲线 $C$ 包含在 $D$ 中.
\onslide<+->
形式地展开
\[f(z)\diff z=(u+iv)(\diff x+i\diff y)=(u\diff x-v\diff y)+i(u\diff y+v\diff x).\]

\begin{definition}
如果下述右侧两个线积分均存在, 则定义
\[\int_C f(z)\diff z=\int_C(u\diff x-v\diff y)+i\int_C(v\diff x+u\diff y)\]
为\markdef{函数 $f(z)$ 沿曲线 $C$ 的积分}.
\end{definition}
\end{frame}


\begin{frame}{复变函数积分的定义}
\onslide<+->
当然, 我们也可以像线积分那样直接定义.
\onslide<+->
在曲线 $C$ 上依次选择分点 $z_0=A,z_1,\dots,z_n=B$.
\onslide<+->
然后在每一段弧上任取 $\zeta_k\in\warc{z_{k-1}z_k}$ 并作和式
\[S_n=\sum_{k=1}^n f(\zeta_k)\Delta z_k,\quad \Delta z_k=z_k-z_{k-1}.\]
\onslide<+->
然后称 $n\to\infty$, 分割的弧长 $\ra 0$ 时 $S_n$ 的极限为复变函数积分.
\onslide<+->
这二者是等价的.
\onslide<2->
\begin{center}
\begin{tikzpicture}
\draw[cstcurve,dcolorc,smooth,domain=0:360] plot ({0.015*\x},{0.6*sin(\x)});
\draw[cstcurve,dcolorc,smooth,domain=195:200,cstarrowto] plot ({0.015*\x},{0.6*sin(\x)});
\fill[cstdot,dcolorc] (0,0) circle;
\fill[cstdot,dcolorb,visible on=<3->]({0.015*32},{0.6*sin(32)}) circle;
\fill[cstdot,dcolora,visible on=<2->]({0.015*64},{0.6*sin(64)}) circle;
\fill[cstdot,dcolorb,visible on=<3->]({0.015*96},{0.6*sin(96)}) circle;
\fill[cstdot,dcolora,visible on=<2->]({0.015*128},{0.6*sin(128)}) circle;
\fill[cstdot,dcolorb,visible on=<3->]({0.015*160},{0.6*sin(160)}) circle;
\fill[cstdot,dcolora,visible on=<2->]({0.015*232},{0.6*sin(232)}) circle;
\fill[cstdot,dcolorb,visible on=<3->]({0.015*264},{0.6*sin(264)}) circle;
\fill[cstdot,dcolora,visible on=<2->]({0.015*296},{0.6*sin(296)}) circle;
\fill[cstdot,dcolorb,visible on=<3->]({0.015*328},{0.6*sin(328)}) circle;
\fill[cstdot,dcolorc] ({0.015*360},0) circle;
\draw
	({0.015*32},{0.6*sin(32)-0.3}) node[dcolorb,visible on=<3->] {$\zeta_1$}
	({0.015*64},{0.6*sin(64)+0.3}) node[dcolora,visible on=<2->] {$z_1$}
	({0.015*96},{0.6*sin(96)-0.3}) node[dcolorb,visible on=<3->] {$\zeta_2$}
	({0.015*128},{0.6*sin(128)+0.3}) node[dcolora,visible on=<2->] {$z_2$}
	({0.015*160},{0.6*sin(160)-0.3}) node[dcolorb,visible on=<3->] {$\zeta_3$}
	({0.015*232},{0.6*sin(232)+0.3}) node[dcolora,visible on=<2->] {$z_{n-2}$}
	({0.015*264},{0.6*sin(264)-0.3}) node[dcolorb,visible on=<3->] {$\zeta_{n-1}$}
	({0.015*296},{0.6*sin(296)+0.3}) node[dcolora,visible on=<2->] {$z_{n-1}$}
	({0.015*328},{0.6*sin(328)-0.3}) node[dcolorb,visible on=<3->] {$\zeta_n$}
	({0.015*180},{0.6*sin(180)+0.5}) node[dcolora,visible on=<2->] {$\ddots$}
  (-0.3,0) node[dcolorc] {$A$}
  ({0.015*360+0.3},0) node[dcolorc] {$B$};
\end{tikzpicture}
\end{center}
\end{frame}


\begin{frame}{复变函数积分的定义}
\onslide<+->
如果 $C$ 是闭曲线, 则该积分记为 \markatt{$\displaystyle\oint_Cf(z)\diff z$}.
\onslide<+->
此时该积分不依赖端点的选取.

\onslide<+->
如果 $C$ 是实轴上的区间 $[a,b]$ 且此时 $f(z)=u(x)$, 
\onslide<+->
则
\[\int_Cf(z)\diff z=\int_a^bf(z)\diff z=\int_a^b u(x)\diff x\]
就是黎曼积分.
\end{frame}


\begin{frame}{积分存在的条件及其计算法}
\onslide<+->
根据线积分的存在性条件可知:
\begin{theorem}
如果 $f(z)$ 是 $D$ 内连续函数, $C$ 是光滑曲线, 则 $\displaystyle\int_Cf(z)\diff z$ 总存在.
\end{theorem}
\onslide<+->
在计算线积分时, 我们可以利用变量替换等技巧.
\onslide<+->
这些技巧可以照搬过来使用.
\onslide<+->
设 $C:z(t)=x(t)+iy(t),a\le t\le b$ 是一条光滑有向曲线, 正方向为 $t$ 增加的方向.
\onslide<+->
那么 $\diff z=z'(t)\diff t$,
\onslide<+->
\[\eqatt{\int_Cf(z)\diff z=\int_a^b f(z)z'(t)\diff t.}\]
\onslide<+->
如果 $C$ 的正方向是 $t$ 减少的方向, 则需要交换右侧积分的上下限.

\onslide<+->
如果 $C$ 是逐段光滑的, 则相应的积分就是各段的积分之和.
\onslide<+->
以后我们\marknot{只考虑逐段光滑曲线上的连续函数的积分}.
\end{frame}


%\begin{frame}{复变函数积分的定义}
%\begin{definition}
%设 $w=f(z)$ 定义在区域 $D$ 内, 有向曲线 $C$ 包含在 $D$ 中.
%\onslide<+->
%分点 $z_0=A,z_1,\dots,z_n=B$ 把曲线 $C$ 分成 $n$ 端弧.
%\onslide<+->
%在每一段弧上任取 $\zeta_k\in\warc{z_{k-1}z_k}$.
%\onslide<+->
%作和式
%\[S_n=\sum_{k=1}^n f(\zeta_k)\Delta z_k,\quad \Delta z_k=z_k-z_{k-1}.\]
%\onslide<+->
%设 $\Delta s_k$ 为 $\warc{z_{k-1}z_k}$ 的长度, $\delta=\max\limits_{1\le k\le n}\Delta s_k$.
%\end{definition}
%\onslide<1->
%\begin{center}
%\begin{tikzpicture}
%\draw[cstcurve,dcolorc,smooth,domain=0:360] plot ({0.015*\x},{0.6*sin(\x)});
%\draw[cstcurve,dcolorc,smooth,domain=195:200,cstarrowto] plot ({0.015*\x},{0.6*sin(\x)});
%\fill[cstdot,dcolorc] (0,0) circle;
%\fill[cstdot,dcolorb,visible on=<3->]({0.015*32},{0.6*sin(32)}) circle;
%\fill[cstdot,dcolora,visible on=<2->]({0.015*64},{0.6*sin(64)}) circle;
%\fill[cstdot,dcolorb,visible on=<3->]({0.015*96},{0.6*sin(96)}) circle;
%\fill[cstdot,dcolora,visible on=<2->]({0.015*128},{0.6*sin(128)}) circle;
%\fill[cstdot,dcolorb,visible on=<3->]({0.015*160},{0.6*sin(160)}) circle;
%\fill[cstdot,dcolora,visible on=<2->]({0.015*232},{0.6*sin(232)}) circle;
%\fill[cstdot,dcolorb,visible on=<3->]({0.015*264},{0.6*sin(264)}) circle;
%\fill[cstdot,dcolora,visible on=<2->]({0.015*296},{0.6*sin(296)}) circle;
%\fill[cstdot,dcolorb,visible on=<3->]({0.015*328},{0.6*sin(328)}) circle;
%\fill[cstdot,dcolorc] ({0.015*360},0) circle;
%\draw
%	({0.015*32},{0.6*sin(32)-0.3}) node[dcolorb,visible on=<3->] {$\zeta_1$}
%	({0.015*64},{0.6*sin(64)+0.3}) node[dcolora,visible on=<2->] {$z_1$}
%	({0.015*96},{0.6*sin(96)-0.3}) node[dcolorb,visible on=<3->] {$\zeta_2$}
%	({0.015*128},{0.6*sin(128)+0.3}) node[dcolora,visible on=<2->] {$z_2$}
%	({0.015*160},{0.6*sin(160)-0.3}) node[dcolorb,visible on=<3->] {$\zeta_3$}
%	({0.015*232},{0.6*sin(232)+0.3}) node[dcolora,visible on=<2->] {$z_{n-2}$}
%	({0.015*264},{0.6*sin(264)-0.3}) node[dcolorb,visible on=<3->] {$\zeta_{n-1}$}
%	({0.015*296},{0.6*sin(296)+0.3}) node[dcolora,visible on=<2->] {$z_{n-1}$}
%	({0.015*328},{0.6*sin(328)-0.3}) node[dcolorb,visible on=<3->] {$\zeta_n$}
%	({0.015*180},{0.6*sin(180)+0.5}) node[dcolora,visible on=<2->] {$\ddots$}
%  (-0.3,0) node[dcolorc] {$A$}
%  ({0.015*360+0.3},0) node[dcolorc] {$B$};
%\end{tikzpicture}
%\end{center}
%\end{frame}
%
%
%\begin{frame}{复变函数积分的定义}
%\begin{definition}
%如果当 $n\to\infty,\delta\to 0$ 时, 和式 $S_n$ 的极限总存在, 则称该极限值为\markdef{函数 $f(z)$ 沿曲线 $C$ 的积分}.
%\onslide<+->
%记作
%\[\int_C f(z)\diff z=\lim_{\delta\to0}\sum_{k=1}^n f(\zeta_k)\Delta z_k.\]
%\onslide<+->
%如果 $C$ 是闭曲线, 则该积分记为 \markdef{$\displaystyle\oint_Cf(z)\diff z$}.
%\onslide<+->
%此时该积分不依赖端点的选取.
%\end{definition}
%\onslide<+->
%如果 $C$ 是实轴上的区间 $[a,b]$ 且此时 $f(z)=u(x)$, 
%\onslide<+->
%则
%\[\int_Cf(z)\diff z=\int_a^bf(z)\diff z=\int_a^b u(x)\diff x.\]
%\end{frame}
%
%
%\begin{frame}{积分存在的条件及其计算法}
%\onslide<+->
%设 $C:z(t)=x(t)+iy(t),a\le t\le b$ 是一条光滑有向曲线, 正方向为 $t$ 增加的方向.
%\onslide<+->
%设 $\zeta_k=\xi_k+i\eta_k$, 则
%\begin{align*}
%S_n&=\sum_{k=1}^n f(\zeta_k)\Delta z_k
%\visible<+->{=\sum_{k=1}^n \bigl[u(\xi_k,\eta_k)+iv(\xi_k,\eta_k)\bigr]\cdot(\Delta x_k+i\Delta y_k)}\\
%&\visible<+->{=\sum_{k=1}^n\bigl[u(\xi_k,\eta_k)\Delta x_k-v(\xi_k,\eta_k)\Delta y_k\bigr]}\\
%&\qquad
%\visible<.->{+i\sum_{k=1}^n\bigl[u(\xi_k,\eta_k)\Delta y_k+v(\xi_k,\eta_k)\Delta x_k\bigr].}
%\end{align*}
%\onslide<+->
%由于 $f(z)=u+iv$ 连续, $u,v$ 也连续,
%\onslide<+->
%从而
%\[\int_C f(z)\diff z=\lim S_n=\int_C(u\diff x-v\diff y)+i\int_C(v\diff x+u\diff y).\]
%\end{frame}
%
%
%\begin{frame}{积分存在的条件及其计算法}
%\beqskip{5pt}
%\begin{theorem}
%如果 $f(z)$ 是 $D$ 内连续函数, $C$ 是光滑曲线, 则
%\[\int_Cf(z)\diff z=\int_C(u\diff x-v\diff y)+i\int_C(v\diff x+u\diff y).\]
%\end{theorem}
%\onslide<+->
%根据 $C$ 的参数方程, $\diff x=x'(t)\diff t,\diff y=y'(t)\diff t$.
%\onslide<+->
%因此
%\begin{align*}
%\markatt{\int_Cf(z)\diff z}&=\int_a^b\Bigl[
%\bigl(u x'(t)-v y'(t)\bigr)
%+i\bigl(v x'(t)+u y'(t)\bigr)\Bigr]\diff t\\
%&\visible<+->{\markatt{=\int_a^b f(z)z'(t)\diff t.}}
%\end{align*}
%\onslide<+->
%如果 $C$ 的正方向是 $t$ 减少的方向, 则需要交换右侧积分的上下限.
%
%\onslide<+->
%如果 $C$ 是逐段光滑的, 则相应的积分就是各段的积分之和.
%\onslide<+->
%以后我们\marknot{只考虑逐段光滑曲线上的连续函数的积分}.
%\endgroup
%\end{frame}


\begin{frame}{典型例题: 计算复变函数沿曲线的积分}
\begin{example}
求 $\displaystyle\int_Cz\diff z$, 其中 $C$ 是从原点到点 $3+4i$ 的直线段.
\end{example}
\begin{solution}
由于 $z=(3+4i)t,0\le t\le 1$,
\onslide<+->
因此
\begin{flalign*}
\int_Cz\diff z&=\int_0^1(3+4i)t\cdot(3+4i)\diff t&\\
&\visible<+->{=(3+4i)^2\int_0^1t\diff t}&\\
&\visible<+->{=\frac12(3+4i)^2=-\frac72+12i.}&
\end{flalign*}
\end{solution}
\onslide<1->
\begin{tikzpicture}[overlay,xshift=7cm,yshift=2cm]
\draw[cstaxis](0,0)--(3,0);
\draw[cstaxis](0,0)--(0,2.5);
\draw[cstcurve,dcolora,cstarrowto](0,0)--(1.5,2);
\draw
	(-0.2,-0.2) node {$O$}
	(2.8,-0.3) node {$x$}
	(-0.2,2.3) node {$y$}
  (2.8,2) node[dcolora,align=center,visible on=<2->] {$z=(3+4i)t$\\$0\le t\le 1$};
\end{tikzpicture}
\end{frame}


\begin{frame}{典型例题: 计算复变函数沿曲线的积分}
\vspace{-3pt}
\begin{example}
求 $\displaystyle\int_Cz\diff z$, 其中 $C$ 是抛物线 $y=\dfrac49x^2$ 上从原点到点 $3+4i$ 的曲线段.
\end{example}
\vspace{-3pt}
\begin{solution}
由于 $z=t+\dfrac49it^2,0\le t\le 3$,
\onslide<+->
因此
\vspace{-2pt}
\begin{flalign*}
\int_Cz\diff z&=\int_0^3\left(t+\frac{4}9it^2\right)\cdot\left(1+\frac89it\right)\diff t&\\
&\visible<+->{=\int_0^3\left(t+\frac43it^2-\frac{32}{81}t^3\right)\diff t}&\\
&\visible<+->{=\left(\frac12t^2+\frac49it^3-\frac8{81}t^4\right)\Big|_0^3}
\visible<+->{=-\frac72+12i.}&
\end{flalign*}
\vspace{-\baselineskip}
\end{solution}
\onslide<1->
\begin{tikzpicture}[overlay,xshift=7cm,yshift=2.5cm]
\draw[cstaxis](0,0)--(3,0);
\draw[cstaxis](0,0)--(0,2.5);
\draw[cstcurve,dcolora,domain=0:1.5,cstarrowto] plot({\x},{8*\x*\x/9});
\draw
	(-0.2,-0.2) node {$O$}
	(2.8,-0.3) node {$x$}
	(-0.2,2.3) node {$y$}
  (2.8,2) node[dcolora,align=center,visible on=<2->] {$z=t+\dfrac49it^2$\\$0\le t\le 3$};
\end{tikzpicture}
\vspace{-2pt}
\end{frame}


\begin{frame}{典型例题: 计算复变函数沿曲线的积分}
\begin{example}
求 $\displaystyle\int_C\Re z\diff z$, 其中 $C$ 是从原点到点 $1+i$ 的直线段.
\end{example}
\begin{solution}
由于 $z=(1+i)t,0\le t\le 1$,
\onslide<+->
因此 $\Re z=t$,
\onslide<+->
\begin{flalign*}
\int_C\Re z\diff z&=\int_0^1t\cdot(1+i)\diff t&\\
&\visible<+->{=(1+i)\int_0^1t\diff t}&\\
&\visible<+->{=\frac{1+i}2.}&
\end{flalign*}
\end{solution}
\onslide<1->
\begin{tikzpicture}[overlay,xshift=7cm,yshift=1.5cm]
\draw[cstaxis](0,0)--(3,0);
\draw[cstaxis](0,0)--(0,2.5);
\draw[cstcurve,dcolora,cstarrowto](0,0)--(2,2);
\draw
	(-0.2,-0.2) node {$O$}
	(2.8,-0.3) node {$x$}
	(-0.2,2.3) node {$y$}
  (2.3,0.9) node[dcolora,align=center,visible on=<2->] {$z=(1+i)t$\\$0\le t\le 1$};
\end{tikzpicture}
\end{frame}


\begin{frame}{典型例题: 计算复变函数沿曲线的积分}
\beqskip{0pt}
\begin{example}
求 $\displaystyle\int_C\Re z\diff z$, 其中 $C$ 是从原点到点 $i$ 再到 $1+i$ 的折线段.
\end{example}
\begin{solution}
\onslide<+->
由于第一段 $z=it,0\le t\le 1, \Re z=0$,
\onslide<+->
第二段 $z=t+i$, $0\le t\le 1$, $\Re z=t$.
\onslide<+->
因此
\[\int_C\Re z\diff z=\int_0^1 t\diff t=\frac12.\]
\onslide<2->
\vspace{-\baselineskip}
\begin{center}
\begin{tikzpicture}
\draw[cstaxis](0,0)--(3,0);
\draw[cstaxis](0,0)--(0,2.5);
\draw[cstcurve,cstarrowto,dcolora](0,0)--(0,2);
\draw[cstcurve,cstarrowto,dcolorb](0,2)--(2,2);
\draw
	(-0.2,-0.2) node {$O$}
	(2.8,-0.3) node {$x$}
	(-0.2,2.3) node {$y$}
  (-1.2,1.6) node[dcolora,align=center,visible on=<3->] {$z=it$\\$0\le t\le 1$}
  (3,1.9) node[dcolorb,align=center,visible on=<3->] {$z=t+i$\\$0\le t\le 1$};
\end{tikzpicture}
\end{center}
\vspace{-\baselineskip}
\end{solution}
\endgroup
\end{frame}


\begin{frame}[<*>]{典型例题: 计算复变函数沿曲线的积分}
\onslide<+->
可以看出, 即便起点和终点相同, 沿不同路径 $f(z)=\Re z$ 的积分也可能不同.
\onslide<+->
而 $f(z)=z$ 的积分则只和起点和终点位置有关, 与路径无关.
\onslide<+->
原因在于 $f(z)=z$ 是处处解析的, 我们以后会详加解释.
\onslide<+->
\begin{columns}
	\column{0.63\textwidth}
		\begin{exercise}
		求 $\displaystyle\int_C\Im z\diff z$, 其中 $C$ 是从原点沿 $y=x$ 到点 $1+i$ 再到 $i$ 的折线段.
		\end{exercise}
		\onslide<+->
		\begin{answer}
		$-\dfrac12+\dfrac i2$.
		\end{answer}
	\onslide<4->
	\column{0.33\textwidth}
		\begin{center}
		\begin{tikzpicture}
		\draw[cstaxis](0,0)--(2.5,0);
		\draw[cstaxis](0,0)--(0,2.5);
		\draw[cstcurve,cstarrowto,dcolora](0,0)--(2,2);
		\draw[cstcurve,cstarrowto,dcolorb](2,2)--(0,2);
		\draw
			(-0.2,-0.2) node {$O$}
			(2.3,-0.3) node {$x$}
			(-0.2,2.3) node {$y$};
		\end{tikzpicture}
		\end{center}
\end{columns}
\end{frame}


\begin{frame}{例题: 计算复变函数沿圆周的积分}
\beqskip{2pt}
\begin{example}
求 $\displaystyle\oint_{|z-z_0|=r}\frac{\diff z}{(z-z_0)^{n+1}}$, 其中 $n$ 为整数.
\end{example}
\begin{solution}
$C: |z-z_0|=r$ 的参数方程为 $z=z_0+re^{i\theta},0\le \theta\le 2\pi$.
\onslide<+->
于是 $\diff z=ire^{i\theta}\diff \theta$,
\onslide<+->
\[\oint_C\frac{\diff z}{(z-z_0)^{n+1}}
=\int_0^{2\pi}i(re^{i\theta})^{-n}\diff\theta
\visible<+->{=ir^{-n}\int_0^{2\pi}e^{-in\theta}\diff\theta.}\]
\onslide<2->
\vspace{-\baselineskip}
\begin{center}
\begin{tikzpicture}
\draw[cstcurve,dcolorb] ({1.3*cos(120)},{1.3*sin(120)})coordinate (C) -- (0,0)coordinate (B) --(1.3,0)coordinate (A) pic [draw,dcolorb,"$\theta$",angle eccentricity=1.7,angle radius=3mm] {angle};
\fill[cstdot,dcolorb](0,0) circle;
\draw[cstcurve,dcolora,cstarrow1from]({1.3*cos(45)},{1.3*sin(45)}) arc (45:-320:1.3);
\draw
	(-0.5,0.5) node[dcolorb] {$r$}
	(0,-0.3) node[dcolorb] {$z_0$};
\end{tikzpicture}
\end{center}
\vspace{-0.7\baselineskip}
\end{solution}
\endgroup
\end{frame}


\begin{frame}{例题: 计算复变函数沿圆周的积分}
\begin{solutionc}
当 $n=0$ 时, $\displaystyle\oint_C\frac{\diff z}{(z-z_0)^{n+1}}=2\pi i$.

\onslide<+->
当 $n\neq 0$ 时, 
\[\oint_C\frac{\diff z}{(z-z_0)^{n+1}}
=ir^{-n}\int_0^{2\pi}(\cos{n\theta}-i\sin{n\theta})\diff\theta=0.\]
\onslide<+->
所以
\[\markatt{\oint_{|z-z_0|=r}\frac{\diff z}{(z-z_0)^{n+1}}=\begin{cases}
2\pi i,&n=0;\\
0,&n\neq 0.
\end{cases}}\]
\end{solutionc}
\onslide<+->
这个积分以后经常用到, 它的特点是与积分圆周的圆心和半径都无关.
\end{frame}


\begin{frame}{积分的性质}
\begin{theorem}
\begin{enumerate}
\item $\displaystyle\int_Cf(z)\diff z=-\displaystyle\int_{C^-}f(z)\diff z$.
\item $\displaystyle\int_Ckf(z)\diff z=k\displaystyle\int_Cf(z)\diff z$.
\item $\displaystyle\int_C[f(z)\pm g(z)]\diff z
=\displaystyle\int_Cf(z)\diff z\pm\displaystyle\int_Cg(z)\diff z$.
\item (\markdef{长大不等式}) 设 $C$ 的长度为 $L$, $f(z)$ 在 $C$ 上满足 $|f(z)|\le M$, 则
\[\markatt{\abs{\int_Cf(z)\diff z}\le\int_C|f(z)|\diff s\le ML}.\]
\end{enumerate}
\end{theorem}
\end{frame}


\begin{frame}{积分的性质}
\begin{proof}
我们来证明下\enumnum4.
\onslide<+->
由
\[\abs{\sum_{k=1}^n f(\zeta_k)\Delta z_k}
\le\sum_{k=1}^n|f(\zeta_k)\Delta z_k|
\le\sum_{k=1}^n|f(\zeta_k)|\Delta s_k
\le M\sum_{k=1}^n\Delta s_k\]
\onslide<+->
可知
\[\abs{\int_Cf(z)\diff z}\le\int_C|f(z)|\diff s\le ML.\qedhere\]
\end{proof}
\onslide<+->
长大不等式常常用于估算一个积分和一个具体的数值之差不超过任意给定的 $\varepsilon$, 从而得到二者相等.
\end{frame}


\begin{frame}{例题: 长大不等式的应用*}
\beqskip{2pt}
\begin{example}
设 $f(z)$ 在 $z\neq a$ 处连续, 且 $\lim\limits_{z\to a}(z-a)f(z)=k$, 则
\[\lim
_{r\to0}\oint_{|z-a|=r}f(z)\diff z=2\pi ik.\]
\end{example}
\begin{proof}
$\forall \varepsilon>0,\exists\delta>0$ 使得当 $|z-a|<\delta$ 时, $|(z-a)f(z)-k|\le\varepsilon$.
\onslide<+->
当 $0<r<\delta$ 时,
\begin{align*}
&\peq\abs{\oint_{|z-a|=r}f(z)\diff z-2\pi i k}
=\abs{\oint_{|z-a|=r}\left[f(z)-\frac k{z-a}\right]\diff z}\\
&\visible<+->{=\abs{\oint_{|z-a|=r}\frac{(z-a)f(z)-k}{z-a}\diff z}}
\visible<+->{\le \frac{\varepsilon}r\cdot 2\pi r=2\pi\varepsilon.}
\end{align*}
\onslide<+->
由于 $\varepsilon$ 是任意的, 因此命题得证.
\end{proof}
\endgroup
\end{frame}

