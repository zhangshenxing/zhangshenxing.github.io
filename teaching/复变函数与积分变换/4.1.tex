\section{复数项级数}


\begin{frame}{复数项级数}
\onslide<+->
复数域上的级数与实数域上的级数并无本质差别.
\begin{definition}
\begin{itemize}
\item 设 $\{z_n\}_{n\ge1}$ 是一个复数列.
\onslide<+->
表达式 $\suml_{n=1}^\infty z_n$ 称为复数项\emph{无穷级数}.
\item 称
\[s_n=z_1+z_2+\cdots+z_n\]
为该级数的\emph{部分和}.
\item 如果部分和数列 $\set{s_n}_{n\ge 1}$ 极限存在, 则称 $\suml_{n=1}^\infty z_n$ \emph{收敛}, 并记 $\suml_{n=1}^\infty z_n=\lim\limits_{n\to\infty}s_n$ 为它的\emph{和}.
\onslide<+->
否则称之\emph{发散}.
\end{itemize}
\end{definition}
\end{frame}


\begin{frame}{复数项级数敛散性的判定}
\beqskip{8pt}
\begin{theorem}
$\suml_{n=1}^\infty z_n=a+bi$ 当且仅当 $\suml_{n=1}^\infty x_n=a,\suml_{n=1}^\infty y_n=b$.
\end{theorem}
\begin{proof}
设 $\suml_{n=1}^\infty x_n$ 的部分和为 $\sigma_n=x_1+x_2+\cdots+x_n$,
设 $\suml_{n=1}^\infty y_n$ 的部分和为
$\tau_n=y_1+y_2+\cdots+y_n,$
\onslide<+->
则 $\suml_{n=1}^\infty z_n$ 的部分和为
\[s_n=z_1+z_2+\cdots+z_n=\sigma_n+i\tau_n.\]
\onslide<+->
由复数列的敛散性判定条件可知
\[\lim_{n\to\infty}s_n=a+bi\iff	\lim_{n\to\infty}\sigma_n=a,\quad \lim_{n\to\infty}\tau_n=b.\]
\onslide<+->
由此命题得证.
\end{proof}
\endgroup
\end{frame}


\begin{frame}{复数项级数敛散性的判定}
\onslide<+->
如果 $\suml_{n=1}^\infty z_n$ 收敛, 则它的实部级数和虚部级数都收敛,
\onslide<+->
从而 $x_n,y_n\to 0$,
\onslide<+->
$z_n=x_n+iy_n\to 0$.
\onslide<+->
因此 \alert{$z_n\to0$ 是 $\suml_{n=1}^\infty z_n$ 收敛的必要条件}.
\begin{theorem}
如果实数项级数
\[\sum_{n=1}^\infty|z_n|=|z_1|+|z_2|+\cdots\]
收敛, 则 $\suml_{n=1}^\infty z_n$ 也收敛, 且 $\abs{\suml_{n=1}^\infty z_n}\le\suml_{n=1}^\infty |z_n|$.
\end{theorem}
\end{frame}


\begin{frame}{复数项级数敛散性的判定}
\beqskip{7pt}
\begin{proof}
\indent
因为 $|x_n|,|y_n|\le|z_n|$, 由比较判别法可知实数项级数 $\suml_{n=1}^\infty x_n$, $\suml_{n=1}^\infty y_n$ 绝对收敛, 从而收敛.
\onslide<+->
故 $\suml_{n=1}^\infty z_n$ 也收敛.

\onslide<+->
\indent
由三角不等式可知
\[\abs{\sum_{k=1}^n z_k}\le \sum_{k=1}^n|z_k|.\]
\onslide<+->
两边同时取极限即得级数的不等式关系
\[\abs{\sum_{n=1}^\infty z_n}=\abs{\lim_{n\to\infty}\sum_{k=1}^n z_k}=
\lim_{n\to\infty}\abs{\sum_{k=1}^n z_k}\le\lim_{n\to\infty}\sum_{k=1}^n|z_k|=\sum_{n=1}^\infty |z_n|,\]
\onslide<+->
其中第二个等式是因为绝对值函数 $|z|$ 连续.
\end{proof}
\endgroup
\end{frame}


\begin{frame}{绝对收敛和条件收敛}
\begin{definition}
\begin{itemize}
\item 如果级数 $\suml_{n=1}^\infty |z_n|$ 收敛, 则称 $\suml_{n=1}^\infty z_n$ \emph{绝对收敛}.
\item 称收敛但不绝对收敛的级数\emph{条件收敛}.
\end{itemize}
\end{definition}
\begin{theorem}
$\suml_{n=1}^\infty z_n$ 绝对收敛当且仅当它的实部和虚部级数都绝对收敛.
\end{theorem}
\begin{proof}
必要性由前一定理的证明已经知道,
\onslide<+->
充分性由 $|z_n|\le|x_n|+|y_n|$ 可得.
\end{proof}
\end{frame}


\begin{frame}{绝对收敛和条件收敛}
\begin{center}
\renewcommand\arraystretch{2}
\defaultrowcolors
\resizebox{\linewidth}{!}{
\begin{tabular}{|c|c|c|c|}
&\tht $\suml_{n=1}^\infty x_n$ 发散
&\tht $\suml_{n=1}^\infty x_n$ 条件收敛
&\tht $\suml_{n=1}^\infty x_n$ 绝对收敛\\
 \tht $\suml_{n=1}^\infty y_n$ 发散
&$\suml_{n=1}^\infty z_n$ 发散
&$\suml_{n=1}^\infty z_n$ 发散
&$\suml_{n=1}^\infty z_n$ 发散\\
 \tht $\suml_{n=1}^\infty y_n$ 条件收敛
&$\suml_{n=1}^\infty z_n$ 发散
&\cellcolor{lightg} $\suml_{n=1}^\infty z_n$ 条件收敛
&\cellcolor{lightg} $\suml_{n=1}^\infty z_n$ 条件收敛\\
 \tht $\suml_{n=1}^\infty y_n$ 绝对收敛
&$\suml_{n=1}^\infty z_n$ 发散
&\cellcolor{lightg} $\suml_{n=1}^\infty z_n$ 条件收敛
&\cellcolor{lightr} $\suml_{n=1}^\infty z_n$ 绝对收敛\\
\end{tabular}}
\end{center}
\end{frame}


\begin{frame}{绝对收敛和条件收敛}
\onslide<+->
绝对收敛的复级数各项可以任意重排次序而不改变其绝对收敛性, 且不改变其和.

\onslide<+->
一般的级数重排有限项不改变其敛散性与和, 但如果重排无限项则可能会改变其敛散性与和.
\begin{thinking}
什么时候 $\abs{\suml_{n=1}^\infty z_n}=\suml_{n=1}^\infty|z_n|$?
\end{thinking}
\begin{answer}
当且仅当非零的 $z_n$ 的辐角全都相同时成立.
\end{answer}
\end{frame}


\begin{frame}{例题: 判断级数的敛散性}
\beqskip{3pt}
\begin{example}
级数 $\displaystyle\sum_{n=1}^\infty\frac{1+i^n}n$ 发散、条件收敛、还是绝对收敛?
\end{example}
\begin{solution}
由于实部级数
\[\sum_{n=1}^\infty x_n=
1+\frac13+\frac24+\frac15+\frac17+\frac28+\cdots\]
发散, 所以该级数发散.
\end{solution}
\onslide<+->
事实上, 它的虚部级数
\[\sum_{n=1}^\infty y_n=
1-\frac13+\frac15-\frac17+\cdots\]
是条件收敛的.
\endgroup
\end{frame}


\begin{frame}{例题: 判断级数的敛散性}
\begin{example}
级数 $\displaystyle\sum_{n=1}^\infty\left[\frac{(-1)^n}n+\frac i{2^n}\right]$ 发散、条件收敛、还是绝对收敛?
\end{example}
\begin{solution}
因为实部级数 $\displaystyle\sum_{n=1}^\infty\frac{(-1)^n}n$ 条件收敛,
\onslide<+->
虚部级数 $\displaystyle\sum_{n=1}^\infty\frac1{2^n}$ 绝对收敛,
\onslide<+->
所以该级数条件收敛.
\end{solution}
\begin{example}
级数 $\displaystyle\sum_{n=1}^\infty\dfrac{i^n}n$ 发散、条件收敛、还是绝对收敛?
\end{example}
\end{frame}



\begin{frame}{例题: 复数项级数敛散性}
\begin{solution}
因为它的实部和虚部级数
\[\sum_{n=1}^\infty x_n=-\frac12+\frac14+\frac16-\frac18+\cdots\]
\onslide<+->
\[\sum_{n=1}^\infty y_n=1-\frac13+\frac15-\frac17+\cdots\]
均条件收敛,
\onslide<+->
所以原级数条件收敛.
\end{solution}
\end{frame}


\begin{frame}{例题: 复数项级数敛散性*}
\onslide<+->
可以证明
\[1-\frac13+\frac15-\frac17+\cdots=\arctan x|_{x=1}=\frac\pi4,\]
\onslide<+->
\[1-\frac12+\frac13-\frac14+\cdots=\ln(1+x)|_{x=1}=\ln2.\]
\onslide<+->
从而
\[\sum_{n=1}^\infty\dfrac{i^n}n=-\frac12\ln 2+\frac{\pi i}4.\]
\onslide<+->
事实上, 左侧是复变函数 $-\ln(1+z)$ 在 $z=-i$ 处的泰勒级数.
\end{frame}


\begin{frame}{例题: 判断级数的敛散性}
\begin{example}
级数 $\displaystyle\sum_{n=0}^\infty\frac{(8i)^n}{n!}$ 发散、条件收敛、还是绝对收敛?
\end{example}
\begin{solution}
因为 $\abs{\dfrac{(8i)^n}{n!}}=\dfrac{8^n}{n!}$, $\displaystyle\sum_{n=0}^\infty\frac{8^n}{n!}=e^8$ 收敛, 所以该级数绝对收敛.
\end{solution}
\onslide<+->
实际上, 它的实部和虚部级数分别为
\[1-\frac{8^2}{2!}+\frac{8^4}{4!}-\frac{8^6}{6!}+\cdots=\cos 8,\ 
8-\frac{8^3}{3!}+\frac{8^5}{5!}-\frac{8^7}{7!}+\cdots=\sin 8,\]
\onslide<+->
因此
\[\sum_{n=0}^\infty\frac{(8i)^n}{n!}=\cos 8+i\sin 8=e^{8i}.\]
\end{frame}


\begin{frame}{级数敛散性判别法}
对于正项级数 $\suml_{n=0}^\infty x_n$, 我们有若干判别法来判断它的敛散性.
\onslide<+->
由此可得: 设
\begin{enumerate}
\item \emph{达朗贝尔判别法(比值法)}: $\lambda=\displaystyle\lim_{n\to\infty}\abs{\frac{z_{n+1}}{z_n}}$ (假设存在);
\item \emph{柯西判别法(根式法)}: $\lambda=\displaystyle\lim_{n\to\infty}\sqrt[n]{\abs{z_n}}$ (假设存在);
\item \emph{柯西-Hadamard判别法}: $\lambda=\displaystyle\ov{\lim_{n\to\infty}}\sqrt[n]{\abs{z_n}}$ (所有子数列中极限的最大值).
\end{enumerate}
\onslide<+->
则当 $\lambda<1$ 时, $\suml_{n=0}^\infty z_n$ 绝对收敛;
\onslide<+->
当 $\lambda>1$ 时, $\suml_{n=0}^\infty z_n$ 发散.
\onslide<+->
其证明主要是通过将该级数与相应的等比级数做比较得到.
\onslide<+->
如果 $\lambda=1$, 则无法使用该方法判断.

\begin{block}{另解}
因为 $\displaystyle\lim_{n\to\infty}\abs{\frac{z_{n+1}}{z_n}}=\lim_{n\to\infty}\abs{\dfrac{8}{n+1}}=0$, 所以该级数绝对收敛.
\end{block}
\end{frame}





