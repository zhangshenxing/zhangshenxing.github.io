\section{解析函数与调和函数的关系}


\begin{frame}{调和函数}
\onslide<+->
调和函数是一类重要的二元实变函数, 它和解析函数有着紧密的联系.
\onslide<+->
为了简便, 我们用 $u_{xx},u_{yy}$ 来表示二阶偏导数.

\begin{definition}
如果二元实变函数 $u(x,y)$ 在区域 $D$ 内有二阶连续偏导数, 且满足拉普拉斯方程
\[\marknot{\Delta u:=u_{xx}+u_{yy}=0},\]
则称 $u(x,y)$ 是 $D$ 内的\markdef{调和函数}.
\end{definition}
\end{frame}


\begin{frame}{解析函数与调和函数的联系}
\begin{theorem}
区域 $D$ 内解析函数 $f(z)$ 的实部和虚部都是调和函数.
\end{theorem}
\begin{proof}
设 $f(z)=u(x,y)+iv(x,y)$, 则 $u,v$ 存在偏导数且
\[f'(z)=u_x+iv_x=v_y-iu_x.\]
\onslide<+->
由于 $f(z)$ 存在各阶导数, 因此 $u_x,u_y,v_x,v_y$ 存在连续偏导数.
\onslide<+->
由C-R方程 $u_x=v_y,u_y=-v_x$,
\onslide<+->
从而
\[\Delta u=u_{xx}+u_{yy}=v_{yx}-v_{xy}=0,\]
\vspace{-\baselineskip}
\onslide<+->
\[\Delta v=v_{xx}+v_{yy}=-u_{yx}+u_{xy}=0.\qedhere\]
\end{proof}
\end{frame}


\begin{frame}{解析函数与调和函数的联系}
\onslide<+->
反过来, 调和函数是否一定是某个解析函数的实部或虚部呢?
\onslide<+->
对于单连通的情形, 答案是肯定的.

\onslide<+->
如果 $u+iv$ 是区域 $D$ 内的解析函数, 则我们称 $v$ 是 $u$ 的\markdef{共轭调和函数}.
\onslide<+->
换言之 $u_x=v_y,u_y=-v_x$.
\onslide<+->
显然 $-u$ 是 $v$ 的共轭调和函数.

\begin{theorem}
设 $u(x,y)$ 是单连通域 $D$ 内的调和函数, 则线积分
\[v(x,y)=\int_{(x_0,y_0)}^{(x,y)}-u_y\diff x+u_x\diff y+C\]
是 $u$ 的共轭调和函数.
\end{theorem}
\onslide<+->
由此可知, 调和函数总具有任意阶连续偏导数.
\end{frame}


\begin{frame}{共轭调和函数的求法}
\onslide<+->
如果 $D$ 是多连通区域, 则未必存在共轭调和函数.
\onslide<+->
例如 $\ln(x^2+y^2)$ 是复平面去掉原点上的调和函数, 但它并不是某个解析函数的实部.
\onslide<+->
事实上, 它是 $2\Ln z$ 的实部.

\onslide<+->
在实际计算中, 我们一般不用线积分来得到共轭调和函数, 而是采用下述两种办法:
\begin{enumerate}
\item \markatt{偏积分法}: 通过 \boxatt{$v_y=u_x$ 解得 $v=\varphi(x,y)+\psi(x)$}, 其中 $\psi(x)$ 待定.
\onslide<+->
再代入 \boxatt{$u_y=-v_x$ 中解出 $\psi(x)$}.
\item \markatt{不定积分法}: 对 \boxatt{$f'(z)=u_x-iu_y=v_y+iv_x$ 求不定积分}得到 $f(z)$.
\end{enumerate}
\end{frame}


\begin{frame}{典型例题: 求共轭调和函数和相应的解析函数}
\begin{example}
证明 $u(x,y)=y^3-3x^2y$ 是调和函数, 并求其共轭调和函数以及由它们构成的解析函数.
\end{example}
\begin{solution}
\indent
因为 $u_x=-6xy,u_y=3y^2-3x^2$,
\onslide<+->
所以
\[u_{xx}+u_{yy}=-6y+6y=0,\]
\onslide<+->
故 $u$ 是调和函数.

\indent
\onslide<+->
由 $v_y=u_x=-6xy$ 得 $v=-3xy^2+\psi(x)$.

\indent
\onslide<+->
由 $v_x=-u_y=3x^2-3y^2$ 得 $\psi'(x)=3x^2$,
\onslide<+->
$\psi(x)=x^3+C$.
\end{solution}
\end{frame}


\begin{frame}{典型例题: 求共轭调和函数和相应的解析函数}
\begin{solutionc}
\indent
故 $v(x,y)=-3xy^2+x^3+C$,
\onslide<+->
\begin{align*}
f(z)&=u+iv=y^3-3x^2y+i(-3xy^2+x^3+C)\\
&\visible<+->{=i(x+iy)^3+iC=i(z^3+C).}
\end{align*}
\onslide<+->
或由
\[f'(z)=u_x-iu_y=-6xy-i(3y^2-3x^2)=3iz^2\]
得 $f(z)=iz^3+C$.
\end{solutionc}
\end{frame}


\begin{frame}{典型例题: 求共轭调和函数和相应的解析函数}
\begin{example}
求解析函数 $f(z)$ 使得它的虚部为
\[v(x,y)=e^x(y\cos y+x\sin y)+x+y.\]
\end{example}
\begin{solution}
由 $u_x=v_y=e^x(\cos y-y\sin y+x\cos y)+1$ 得
\[u=e^x(x\cos y-y\sin y)+x+\psi(y).\]
\onslide<+->
由 $u_y=-v_x=-e^x(y\cos y+x\sin y+\sin y)-1$ 得
\[\psi'(y)=-1,\quad\psi(y)=-y+C.\]
\end{solution}
\end{frame}


\begin{frame}{典型例题: 求共轭调和函数和相应的解析函数}
\begin{solutionc}
故
\vspace{-1.4\baselineskip}
\begin{align*}
f(z)&=u+iv\\
&=e^x(x\cos y-y\sin y)+x-y+C\\
&\qquad+i\bigl[e^x(y\cos y+x\sin y)+x+y\bigr]\\
&\visible<+->{=ze^z+(1+i)z+C.}
\end{align*}
\onslide<+->
或者由
\vspace{-1.4\baselineskip}
\begin{align*}
f'(z)&=v_y+iv_x\\
&=e^x(\cos y-y\sin y+x\cos y)+1\\
&\qquad+i\bigl[e^x(y\cos y+x\sin y+\sin y)+1\bigr]\\
&\visible<+->{=(z+1)e^z+1+i.}
\end{align*}
\onslide<+->
得 $f(z)=ze^z+(1+i)z+C$.
\end{solutionc}
\end{frame}


\begin{frame}{典型例题: 求共轭调和函数和相应的解析函数}
\begin{exercise}
证明 $u(x,y)=x^3-6x^2y-3xy^2+2y^3$ 是调和函数并求它的共轭调和函数.
\end{exercise}
\begin{answer}
$v(x,y)=2x^3+3x^2y-6xy^2-y^3+C$.
\end{answer}
\end{frame}

