% \ctexset{chapter/numbering=false}
\chapter{中外人名对照表}
% \phantomsection\addcontentsline{toc}{chapter}{中外人名对照表}

为便于读者查找相关英文资料, 在此列出本书中出现过的人名的外文名.
以下人名中不注明其爵位, 加粗为姓氏部分.

\begin{itemize}
  % A-G
  \item 阿贝尔 (Niels Henrik \emphf{Abel}, 1802--1829), 挪威数学家.
  \item 波尔文 (Jonathan Michael \emphf{Borwein}, 1951--2016), 苏格兰数学家.
  \item 波尔文 (Peter Benjamin \emphf{Borwein}, 1953--2020), 苏格兰数学家.
  \item 波尔查诺 (Bernard Placidus Johann Gonzal Nepomuk \emphf{Bolzano}, 1781--1848), 捷克数学家.
  \item 泊松 (Siméon-Denis \emphf{Poisson}, 1781--1840), 法国数学家、物理学家.
  \item 达朗贝尔 (Jean Le Rond d'Alembert, 1717--1783), 法国数学家.
  \item 狄拉克 (Paul Adrien Maurice \emphf{Dirac}, 1902--1984), 英国理论物理学家.
  \item 狄利克雷 (Johann Peter Gustav Lejeune \emphf{Dirichlet}, 1805--1859), 德国数学家.
  \item 棣莫弗 (Abraham \emphf{de Moivre}, 1667--1754), 法国数学家.
  \item 范德蒙 (Alexandre-Théophile \emphf{Vandermonde}, 1735--1796), 法国数学家.
  \item 费罗 (Scipione del \emphf{Ferro}, 1465--1526), 意大利数学家.
  \item 菲涅耳 (Augustin Jean \emphf{Fresnel}, 1788--1827), 法国物理学家.
  \item 傅里叶 (Jean Baptiste Joseph \emphf{Fourier}, 1768--1830), 法国数学家.
  \item 高斯 (Johann Carl Friedrich \emphf{Gauß}, 英文为 \emphf{Gauss}, 1777--1855), 德国数学家.
  \item 格林 (George \emphf{Green}, 1793--1841), 英国数学物理学家.
  \item 古萨 (Édouard Jean-Baptiste \emphf{Goursat}, 1858--1936), 法国数学家.
  % H-N
  \item 哈密顿 (William Rowan \emphf{Hamilton}, 1805--1865), 爱尔兰数学家、物理学家、天文学家.
  \item 赫维赛德 (Oliver \emphf{Heaviside}, 1850--1925), 英国物理学家、数学家.
  \item 莱布尼茨 (Gottfried Wilhelm von \emphf{Leibniz}, 1646--1716), 德国哲学家、数学家.
  \item 黎曼 (Georg Friedrich Bernhard \emphf{Riemann}, 1826--1866), 德国数学家.
  \item 卡尔达诺 (Girolamo 或 Hieronimo \emphf{Cardano}, 又名 Hieronymus \emphf{Cardanus} 或 Jerome \emphf{Cardan}, 1501--1576), 意大利数学家、物理学家、天文学家、哲学家.
  \item 科茨 (Roger \emphf{Cotes}, 1682--1716), 英国数学家.
  \item 柯西 (Augustin-Louis \emphf{Cauchy}, 1789--1857), 法国数学家、物理学家.
  \item 拉普拉斯 (Pierre-Simon \emphf{Laplace}, 1749--1827), 法国数学家、天文学家、物理学家.
  \item 洛朗 (Pierre Alphonse \emphf{Laurent}, 1813--1854), 法国数学家.
  \item 列维 (Paul Pierre \emphf{Lévy}, 1886--1971), 法国数学家.
  \item 刘维尔 (Joseph \emphf{Liouville}, 1809--1182), 法国数学家.
  \item 洛必达 (Guillaume-François-Antoine Marquis de \emphf{l'Hôpital}, 1661--1704), 法国数学家.
  \item 麦克劳林 (Colin \emphf{Maclaurin}, 1698--1746), 苏格兰数学家.
  \item 麦克斯韦 (James Clerk \emphf{Maxwell}, 1831--1879), 苏格兰物理学家、数学家.
  \item 莫比乌斯 (August Ferdinand \emphf{Möbius}, 1790--1868), 德国数学家、天文学家.
  \item 莫累拉 (Giacinto \emphf{Morera}, 1856--1907), 意大利数学家.
  \item 牛顿 (Isaac \emphf{Newton}, 1643--1727), 英国数学家.
  % O-T
  \item 欧拉 (Leonhard \emphf{Euler}, 1707--1783), 瑞士数学家、物理学家.
  \item 帕塞瓦尔 (Marc-Antoine \emphf{Parseval} des Chênes 1755--1836), 法国数学家.
  \item 皮卡 (Charles Émile \emphf{Picard}, 1856--1941), 法国数学家.
  % \item 切比雪夫 (\cmu{Пафну́тий Льво́вич {\color{main}{Чебышёв}}}, 英文为 \emphf{Chebyshev}, 1821--1894), 俄罗斯数学家、力学家.
  \item 儒科夫斯基 (\cmu{Николай Егорович {\color{main}{Жуковский}}}, 英文为 \emphf{Zhukovsky}, 1847--1921), 俄罗斯力学家.
  \item 儒歇 (Eugène \emphf{Rouché}, 1832--1910), 法国数学家.
  \item 若尔当 (Marie Ennemond Camille \emphf{Jordan}, 1838--1922), 法国数学家.
  \item 绍德尔 (Julius Pawel \emphf{Schauder}, 1899--1943), 波兰数学家.
  \item 斯坦尼兹 (Ernst \emphf{Steinitz}, 1871--1928), 德国数学家.
  \item 泰勒 (Brook \emphf{Taylor}, 1685--1731), 英国数学家.
  % U-Z
  \item 维布伦 (Oswald \emphf{Veblen}, 1880--1960), 美国数学家.
  \item 魏尔斯特拉斯 (Karl Theodor Wilhelm \emphf{Weierstrass}, 1815--1897), 德国数学家.
\end{itemize}

% 索引用
\index{A@{\vspace{1ex}\hspace{.1pt}\LARGE{\nouns{A}}}\phantom|phantom}
\index{B@{\vspace{1ex}\hspace{.1pt}\LARGE{\nouns{B}}}\phantom|phantom}
\index{C@{\vspace{1ex}\hspace{.1pt}\LARGE{\nouns{C}}}\phantom|phantom}
\index{D@{\vspace{1ex}\hspace{.1pt}\LARGE{\nouns{D}}}\phantom|phantom}
% \index{E@{\vspace{1ex}\hspace{.1pt}\LARGE{\nouns{E}}}\phantom|phantom}
\index{F@{\vspace{1ex}\hspace{.1pt}\LARGE{\nouns{F}}}\phantom|phantom}
\index{G@{\vspace{1ex}\hspace{.1pt}\LARGE{\nouns{G}}}\phantom|phantom}
\index{H@{\vspace{1ex}\hspace{.1pt}\LARGE{\nouns{H}}}\phantom|phantom}
% \index{I@{\vspace{1ex}\hspace{.1pt}\LARGE{\nouns{I}}}\phantom|phantom}
\index{J@{\vspace{1ex}\hspace{.1pt}\LARGE{\nouns{J}}}\phantom|phantom}
\index{K@{\vspace{1ex}\hspace{.1pt}\LARGE{\nouns{K}}}\phantom|phantom}
\index{L@{\vspace{1ex}\hspace{.1pt}\LARGE{\nouns{L}}}\phantom|phantom}
\index{M@{\vspace{1ex}\hspace{.1pt}\LARGE{\nouns{M}}}\phantom|phantom}
\index{N@{\vspace{1ex}\hspace{.1pt}\LARGE{\nouns{N}}}\phantom|phantom}
\index{O@{\vspace{1ex}\hspace{.1pt}\LARGE{\nouns{O}}}\phantom|phantom}
\index{P@{\vspace{1ex}\hspace{.1pt}\LARGE{\nouns{P}}}\phantom|phantom}
\index{Q@{\vspace{1ex}\hspace{.1pt}\LARGE{\nouns{Q}}}\phantom|phantom}
\index{R@{\vspace{1ex}\hspace{.1pt}\LARGE{\nouns{R}}}\phantom|phantom}
\index{S@{\vspace{1ex}\hspace{.1pt}\LARGE{\nouns{S}}}\phantom|phantom}
\index{T@{\vspace{1ex}\hspace{.1pt}\LARGE{\nouns{T}}}\phantom|phantom}
% \index{U@{\vspace{1ex}\hspace{.1pt}\LARGE{\nouns{U}}}\phantom|phantom}
% \index{V@{\vspace{1ex}\hspace{.1pt}\LARGE{\nouns{V}}}\phantom|phantom}
\index{W@{\vspace{1ex}\hspace{.1pt}\LARGE{\nouns{W}}}\phantom|phantom}
\index{X@{\vspace{1ex}\hspace{.1pt}\LARGE{\nouns{X}}}\phantom|phantom}
\index{Y@{\vspace{1ex}\hspace{.1pt}\LARGE{\nouns{Y}}}\phantom|phantom}
\index{Z@{\vspace{1ex}\hspace{.1pt}\LARGE{\nouns{Z}}}\phantom|phantom}












% 定义 \emph{哈密顿算子}
% \[\mbfnabla=i\pp{}{x}+j\pp{}{y}+k\pp{}{z}.\]
% 我们仿照着前述四元数的运算来定义 $\mbfnabla$ 与函数 $f$ 的``数乘''
% \[\mbfnabla f=i\pp{}{x}+j\pp{}{y}+k\pp{}{z},\]
% 与向量 $\bfD=iD_x+jD_y+kD_z$ 的``数量积''
% \[\mbfnabla\cdot \bfD=\pp{D_x}{x}+\pp{D_y}{y}+\pp{D_z}{z},\]
% 与向量 $\bfD=iD_x+jD_y+kD_z$ 的``向量积''
% \[\mbfnabla\times\bfD=(\pp{D_z}y-\pp{D_y}z)i+(\pp{D_x}z-\pp{D_z}x)j+(\pp{D_y}x-\pp{D_x}y)k.\]
% 由于这些运算规则和四元数的运算规则类似, 因此我们可仿照四元数的运算规则来得到 $\mbfnabla$ 的有关性质.
% \begin{homework}
%   \item 证明 $\mbfnabla\times\mbfnabla=0$, 由此得到
%   \[\mbfnabla\mbfnabla=-\mbfnabla\cdot\mbfnabla+\mbfnabla\times\mbfnabla=-\mbfnabla^2,\]
%   其中 $\mbfnabla^2$ 表示拉普拉斯算子 $\mbfnabla\cdot\mbfnabla$.
%   \item 证明 $\mbfnabla\times\mbfnabla f=0$.
%   提示: 考虑 $\mbfnabla\mbfnabla f$ 和乘法结合律.
%   \item 证明 $\mbfnabla\cdot(\mbfnabla\times \bfD)=0$ 和 $\mbfnabla\times(\mbfnabla\times\bfD)=\mbfnabla(\mbfnabla\cdot\bfD)-\mbfnabla^2\bfD$.
%   提示: 考虑 $\mbfnabla\mbfnabla\bfD$ 和乘法结合律.
% \end{homework}
% 事实上, $\mbfnabla f$ 是标量场 $f$ 的梯度, $\mbfnabla\cdot \bfD$ 是向量场 $\bfD$ 的散度, $\mbfnabla\times\bfD$ 是向量场 $\bfD$ 的旋度.



% 为了强调, 我们称通常的复数为\nouns{有限复数}\index{有限复数}, 通常的复平面为\nouns{有限复平面}\index{有限复平面}\footnote{也叫\emph{开复平面}.}.
% \begin{figure}[H]
%   \centering
%   \begin{tikzpicture}
%     \filldraw[cstcurve,cstfill] (0,1) circle (1);
%     \coordinate [label=above:\textcolor{third}{$N$}] (N) at (0,2);
%     \draw[cstdash] (0,0)--(N);
%     \draw[cstaxis] (-2,0)--(2.5,0);
%     \coordinate [label=below:\textcolor{main}{$x_1$}] (x1) at (2.2,0);
%     \coordinate [label=above right:\textcolor{main}{$X_1$}] (X1) at (1,1.1);
%     \coordinate [label=below:\textcolor{second}{$x_2$}] (x2) at (-1,0);
%     \coordinate [label=left:\textcolor{second}{$X_2$}] (X2) at (-.8,.4);
%     \draw[cstcurve,cstra,main] (0,2)--(x1);
%     \fill[cstdot,main] (X1) circle;
%     \draw[cstcurve,cstra,second] (0,2)--(x2);
%     \fill[cstdot,second] (X2) circle;
%     \fill[cstdot] (N) circle;
%   \end{tikzpicture}
%   \caption{圆周和实轴}
% \end{figure}

% \subsubsection{切比雪夫多项式}

% 对棣莫弗公式左侧进行二项式展开可以得到
% \begin{align*}
%   \cos 2\theta&=\hphantom{1}2\cos^2\theta-\hphantom{1}1,\\
%   \cos 3\theta&=\hphantom{1}4\cos^3\theta-\hphantom{1}3\cos\theta,\\
%   \cos 4\theta&=\hphantom{1}8\cos^4\theta-\hphantom{1}8\cos^2\theta+1,\\
%   \cos 5\theta&=16\cos^5\theta-20\cos^3\theta+5\cos\theta.
% \end{align*}
% 由递推关系
% \[
%   \cos (n+1)\theta+\cos(n-1)\theta=2\cos\theta\cos n\theta
% \]
% 可知 $\cos{n\theta}$ 是 $\cos\theta$ 的 $n$ 次多项式, 称这个多项式 $T_n(x)$ 为\nouns{切比雪夫多项式}\index{切比雪夫多项式}.
% 那么 $T_n$ 满足递推关系
% \[
%   T_0(x)=1,\quad T_1(x)=x,\quad
%   T_{n+1}(x)=2xT_n(x)-T_{n-1}(x),
% \]
% 且 $T_n$ 的首项系数为 $2^{n-1}$.

% 若 $T_n(\cos\theta)=\cos n\theta=0$, 则 $n\theta=(k+\frac12)\cpi$, $k\in\BZ$.
% 从而
% \[
%   \cos\theta=\cos\frac{(2k+1)\cpi}{2n},\quad
%   k=0,1,\cdots,n-1.
% \]
% 于是我们得到了 $T_n(x)$ 的所有根, 它们均位于 $(-1,1)$ 中.

% \begin{example}
%   设 $f(x)$ 是首项系数为 $2^{n-1}$ 的 $n$ 次多项式. 那么 $\max\limits_{x\in[-1,1]}\abs{f(x)}\ge 1$.
%   等号成立当且仅当 $f(x)=T_n(x)$. 
% \end{example}

% \begin{figure}[H]
%   \centering
%   \begin{tikzpicture}[scale=1.5]
%     \draw[cstaxis] (-1.5,0)--(1.5,0);
%     \draw[cstaxis] (0,-1.5)--(0,1.5);
%     \draw[cstcurve,fifth,domain=0:180,smooth] plot ({cos(\x)},{cos(5*\x)});
%   \end{tikzpicture}
%   \caption{切比雪夫多项式}
% \end{figure}

% \begin{proof}
%   设 $x_k=\cos\frac{k\cpi}{2n}, 0\le k\le 2n$, 则
%   \[
%     T_n(x_{4k})=1,\quad T_n(x_{4k+2})=-1,
%     \quad T_n(x_{2k+1})=0.
%   \]
%   若 $\max\limits_{x\in[-1,1]}\abs{f(x)}<1$, 则
%   \[
%     f(x_{4k})<1,\quad f(x_{4k+2})>-1.
%   \]
%   也就是说, 对于 $g=f-T_n$, 
%   \[
%     g(x_{4k})<0,\quad g(x_{4k+2})>0.
%   \]
%   于是由零点定理可知 $g(x)$ 在区间 $(x_{2n},x_{2n-2}), \cdots, (x_4,x_2),(x_2,x_0)$ 上都有零点.
%   但是 $g(x)$ 次数 $\le n-1$, 它不可能有 $n$ 个零点, 矛盾!
%   因此 $\max\limits_{x\in[-1,1]}\abs{f(x)}\ge 1$.

%   若 $\max\limits_{x\in[-1,1]}\abs{f(x)}=1$ 但 , 则 $g(x)$ 在区间 $[x_{2n},x_{2n-2}], \cdots, [x_4,x_2],[x_2,x_0]$ 上都有零点.
%   由于 $g(x)$ 最多只有 $n-1$
% \end{proof}

%   提示: 利用反证法, 考虑 $f-T_n$ 在 $T_n$ 零点的行为.

% 切比雪夫多项式的这些性质使得它在逼近理论中有重要的应用, 使用切比雪夫多项式的根进行插值的多项式能最大限度地降低\emph{龙格现象}, 并且提供多项式在连续函数的最佳一致逼近.


% \subsection{切比雪夫多项式}
% 对棣莫弗公式左侧进行二项式展开可以得到
% \begin{align*}
%   \cos(2\theta)&=\hphantom{1}2\cos^2\theta-\hphantom{1}1,\\
%   \cos(3\theta)&=\hphantom{1}4\cos^3\theta-\hphantom{1}3\cos\theta,\\
%   \cos(4\theta)&=\hphantom{1}8\cos^4\theta-\hphantom{1}8\cos^2\theta+1,\\
%   \cos(5\theta)&=16\cos^5\theta-20\cos^3\theta+5\cos\theta.
% \end{align*}
% 一般地, 可以证明 $\cos{n\theta}$ 是 $\cos\theta$ 的 $n$ 次多项式, 称这个多项式 $T_n(x)$ 为\emph{切比雪夫多项式}.
% \begin{homework}
%   \item 证明 $T_n$ 满足递推
%   \[
%     T_0(x)=1,\quad T_1(x)=x,\quad
%     T_{n+1}(x)=2xT_n(x)-T_{n-1}(x),
%   \]
%   并由此证明
%   \[
%     \sumf0 T_n(x)t^n=\frac{1-tx}{1-2tx+t^2}.
%   \]
%   \item 求 $T_n(x)$ 在 $[0,1]$ 中的所有根.
%   \item 证明 $T_n(x)$ 的前两项为
%   \[
%     T_n(x)=2^{n-1}x^n-n2^{n-3}x^{n-2}+\cdots
%   \]
%   \item 设 $f(x)$ 是首项系数为 $2^{n-1}$ 的 $n$ 次多项式. 那么 $\max\limits_{x\in[-1,1]}f(x)\ge 1$.
%   等号成立当且仅当 $f(x)=T_n(x)$. 
%   提示: 利用反证法, 考虑 $f-T_n$ 在 $T_n$ 零点的行为.
% \end{homework}

% 切比雪夫多项式的这些性质使得它在逼近理论中有重要的应用, 使用切比雪夫多项式的根进行插值的多项式能最大限度地降低\emph{龙格现象}, 并且提供多项式在连续函数的最佳一致逼近.




% \begin{minipage}[c][40mm][b]{.54\textwidth}
%   \centering
%   \begin{tikzpicture}
%     \fill[cstfill1] (-3.65,-.804)--(-1.85,.804)--(3.65,.804)--(1.85,-.804)--cycle;
%     \filldraw[cstcurve,cstfill] (0,1) circle (1);
%     \draw[cstdash] (0,1) circle (1 and 0.3);
%     \draw[cstdash] (0,0) circle (2 and 0.6);
%     \coordinate [label=above:\textcolor{third}{$N$}] (N) at (0,2);
%     \draw[cstdash] (0,0)--(N);
%     \draw[cstaxis] (0,0)--(2.5,0);
%     \draw[cstaxis] (0,0)--(-.8,-.9);
%     \coordinate [label=right:{$z_1$}] (z1) at (1.65,-.75);
%     \coordinate [label=left:{$Z_1$}] (Z1) at (.6,1);
%     \coordinate [label=left:{$z_2$}] (z2) at (-1,0);
%     \coordinate [label=below right:{$Z_2$}] (Z2) at (-.7,.6);
%     \draw[cstcurve,main,cstra] (N)--(z1);
%     \fill[cstdot,main] (Z1) circle;
%     \draw[cstcurve,cstra,second] (N)--(z2);
%     \fill[cstdot,second] (Z2) circle;
%     \fill[cstdot] (N) circle;
%   \end{tikzpicture}
%   \caption{复球面和复平面}
% \end{minipage}




% 若我们采用 \ref{enum:exp-series} 来定义, 则从 $\cos x$ 和 $\sin x$ 的泰勒展开
% \[
%   \cos x=1-\frac{x^2}{2!}+\frac{x^4}{4!}+\cdots\quad
%   \sin x=x-\frac{x^3}{3!}+\frac{x^5}{5!}\cdots
% \]
% 可以得到欧拉恒等式 $\ee^{\ii x}=\cos x+\ii\sin x$.
% 实际上这种推导方式并不严谨, 因为我们需要先铺垫好复变函数的级数理论才能这样代换, 这将会在\thmref{例}{exam:exp-taylor-expansion} 中得到解释.
% 而 \ref{enum:exp-euler} 和 \ref{enum:exp-expansion} 的等价性将可由\thmref{定理}{thm:zero-isolated} 推出.




% 
% 
% \begin{example}
% 求幂级数 $\sumf1 \frac{z^n}{n^p}$ 的收敛半径并讨论在收敛圆周上的情形, 其中 $p\in\BR$.
% \end{example}
% 
% \begin{solution}
% 由 $\liml_{n\ra\infty}\abs{\frac{c_{n+1}}{c_n}}=\lim_{n\ra\infty}\left(\frac n{n+1}\right)^p=1$ 可知收敛半径为 $1$.
% {设 $\abs{z}=1$.
% \begin{itemize}
% \item 若 $p>1$, $\sumf1 \abs{\frac{z^n}{n^p}}=\sumf1 \frac1{n^p}$ 收敛,
% {原级数在收敛圆周内处处绝对收敛.}
% \item 若 $p\le 0$, $\biggabs{\dfrac{z^n}{n^p}}=\dfrac1{n^p}\not\ra 0$,
% {原级数在收敛圆周内处处发散.}
% \end{itemize}}
% \end{solution}
% \end{frame}

% 
% 
% 回忆\emph{狄利克雷判别法}: 若 $\set{a_n}_{n\ge 1}$ 部分和有界, 实数项数列 $\set{b_n}_{n\ge 1}$ 单调趋于 $0$, 则 $\sumf1 a_nb_n$ 收敛.

% 
% \begin{solution}
% \begin{itemize}
% \item 若 $0<p\le1$, $\sumf1 \frac1{n^p}$ 发散, 
% {而在收敛圆周上其它点 $z\neq1$ 处,
% \[|z+z^2+\cdots+z^n|=\abs{\frac{z(1-z^n)}{1-z}}
% \le\frac{2}{\abs{1-z}}\]
% 有界, 数列 $\set{n^{-p}}_{n\ge 1}$ 单调趋于 $0$,}
% {因此 $\sumf1 \frac{z^n}{n^p}$ 收敛.}
% {故该级数在 $z=1$ 发散, 在收敛圆周上其它点收敛.}
% \end{itemize}
% \end{solution}
% \end{frame}




% \begin{theorem}
% 	设幂级数
% 	\[
% 		f(z)=\sumf0 a_nz^n,\abs{z}<R,
% 	\]
% 	设函数 $\varphi(z)$ 在集合 $D$ 上满足 $\abs{\varphi(z)}<R$.
% 	那么当 $z\in D$ 时,
% 	\[f[\varphi(z)]=\sumf0 a_n[\varphi(z)]^n.\]}
% \end{theorem}

% 设 $f(x)=\dfrac1x$, 那么
% \[
%   f(m)\le \int_{m-1/2}^{m+1/2}f(x)\d x
%   \le \frac12\bigl(f(m-\dfrac12)+f(m+\dfrac12)\bigr).
% \]
% \begin{center}
%   \begin{tikzpicture}
%     \draw[cstaxis] (-.5,0)--(3.5,0);
%     \draw[cstaxis] (0,-.5)--(0,3);
%     \fill[cstfill1] (1,0)--(1,1)--(2,.5)--(2,0)--cycle;
%     \fill[white] plot[domain=1:2,smooth cycle] (\x,{1/\x});
%     \draw[cstcurve,main,smooth,domain=.4:3] plot (\x,{1/\x});
%     \draw[cstcurve,fourth] (1,0)--(1,1);
%     \draw[cstcurve,fourth] (2,0)--(2,.5);
%     \draw[second] (1,1)--(2,.5);
%     \draw[second] (1,{8/9})--(2,{4/9});
%     \fill[cstdote,third] (1,1) circle;
%     \fill[cstdote,third] (2,.5) circle;
%     \fill[cstdote,third] (1.5,{2/3}) circle;
%   \end{tikzpicture}
% \end{center}
% 设 $S=f(n+1)+\cdots+f(2n)$, 则
% \begin{align*}
%   S&\le \int_{n+1/2}^{2n+1/2} f(x)\d x=\ln\frac{4n+1}{2n+1},\\
%   \ln 2&=\int_{n}^{2n}f(x)\d x
%   \le S+\frac12 f(n)-\frac12f(2n)=S+\frac1{4n},
% \end{align*}
% 于是由
% \[
%   -\frac14\le (S-\ln 2)n\le n\ln\frac{4n+1}{4n+2}
% \]
% 和夹逼准则可得 $\liml_{n\ra\infty}(S-\ln 2)n=-\dfrac14$.





















% \subsubsection{含对数函数和幂函数的积分}

% 考虑含对数函数或幂函数的广义积分时, 通常要重新选择复对数函数或幂函数的主值, 将辐角主值选择在范围 $[0,2\cpi)$ 内, 使其在正实轴上不连续.
% 然后采用\ref{fig:large-small-circle-negative-side} 所示的闭路, 其中 $\ell_1,\ell_2$ 分别为正实轴和正实轴下沿岸的线段且 $r\ra0^+,R\ra+\infty$.
% 由于这样做容易引起初学者理解上的困难, 我们采用变量替换 $x=\ee^t$ 将积分变形, 并考虑\ref{fig:rectangle-2pii} 所示的闭路.

% \begin{figure}[H]
%   \centering
%   \begin{minipage}{.48\textwidth}
%     \centering
%     \begin{tikzpicture}
%       \draw[
%         main,
%         cstcurve,
%         decoration={
%           markings,
%           mark = at position .5 with {
%             \arrowreversed[rotate=20]{Straight Barb}
%             \node[left] {$C_r^-$};
%           }
%         },
%         postaction={decorate}
%       ] ({.3*cos(16)},{.3*sin(16)}) arc(16:344:.3);
%       \draw[
%         second,
%         cstcurve,
%         decoration={
%           markings,
%           mark = at position .5 with {
%             \node[below] {$\ell_2$};
%           }
%         },
%         postaction={decorate}
%       ] ({.3*cos(16)},{.3*sin(16)})--({sqrt(4-.09*sin(16)*sin(16))},{.3*sin(16)});
%       \draw[
%         main,
%         cstcurve,
%         decoration={
%           markings,
%           mark = at position .5 with {
%             \arrow{Straight Barb}
%             \node[left] {$C_R$};
%           }
%         },
%         postaction={decorate}
%       ] ({sqrt(4-.09*sin(16)*sin(16))},{.3*sin(16)}) arc({asin(0.15*sin(16))}:{360-asin(0.15*sin(16))}:2);
%       \draw[
%         second,
%         cstcurve,
%         decoration={
%           markings,
%           mark = at position .5 with {
%             \node[above] {$\ell_1$};
%           }
%         },
%         postaction={decorate}
%       ] ({.3*cos(16)},{-.3*sin(16)})--({sqrt(4-.09*sin(16)*sin(16))},{-.3*sin(16)});
%     \end{tikzpicture}
%     \caption{大小圆周带负实轴沿岸闭路}
%     \label{fig:large-small-circle-negative-side}
%   \end{minipage}
%   \begin{minipage}{.48\textwidth}
%     \centering
%     \begin{tikzpicture}
%       \coordinate (A) at (-2,1.5);
%       \coordinate (B) at (2,1.5);
%       \coordinate (C) at (-2,0);
%       \coordinate (D) at (2,0);
%       \draw[cstaxis] (-3,0)--(3,0);
%       \draw[cstaxis] (0,-1)--(0,3);
%       \draw[
%         main,
%         cstcurve,
%         decoration={
%           markings,
%           mark = at position .5 with {
%             \arrow{Straight Barb}
%             \node[right] {$\ell_1$};
%           }
%         },
%         postaction={decorate}
%       ] (A)--(C);
%       \draw[
%         main,
%         cstcurve,
%         decoration={
%           markings,
%           mark = at position .5 with {
%             \arrow{Straight Barb}
%             \node[right] {$\ell_2$};
%           }
%         },
%         postaction={decorate}
%       ] (D)--(B);
%       \draw[
%         main,
%         cstcurve,
%         decoration={
%           markings,
%           mark = at position .4 with {
%             \arrow{Straight Barb}
%             \node[above] {$\ell$};
%           }
%         },
%         postaction={decorate}
%       ] (B)--(A);
%       \draw[
%         second,
%         cstcurve,
%         decoration={
%           markings,
%           mark = at position .4 with {
%             \arrow{Straight Barb}
%           }
%         },
%         postaction={decorate}
%       ] (C)--(D);
%       \draw
%         node[below] at (C) {$-R_1$}
%         node[below] at (D) {$R_2$}
%         node[below left] at (0,1.5) {$2\cpi\ii$};
%     \end{tikzpicture}
%     \caption{经过 $2\cpi\ii$ 的矩形闭路}
%     \label{fig:rectangle-2pii}
%   \end{minipage}
% \end{figure}

% 考虑 $\intf f(x)x^p\d x$, 其中实数 $p$ 不是整数, $f(x)$ 是一个有理函数, 分母没有正实根, 且满足
% \[
%   \lim_{x\ra 0} x^{p+1}f(x)=0,\quad
%   \lim_{x\ra \infty} x^{p+1}f(x)=0.
% \]
% 不难知道, 这也意味着
% \[
%   \lim_{z\ra 0} z^{p+1}f(z)=0,\quad
%   \lim_{z\ra \infty} z^{p+1}f(z)=0.
% \]
% 通过变量替换 $x=\ee^t$ 可将该积分化为
% \[
%    \intf f(x)x^p\d x
%   =\intff f(\ee^t)\ee^{(p+1)t}\d t.
% \]
% 设 $w=\ee^z, g(z)=f(\ee^z)\ee^{(p+1)z}$.
% 如\ref{fig:rectangle-2pii} 所示, 选择闭路 $C$ 为连接 $-R_1,R_2,R_2+2\cpi\ii,-R_1+2\cpi\ii$ 的矩形闭路.

% 对于 $z=-R_1+y\ii,0\le y\le 2\cpi$, 我们有 $\abs{w}=\ee^{-R_1}$. 于是当 $R_1\ra+\infty$ 时, $w\ra 0$,
% \[
%    \abs{g(z)}=\abs{f(w)w^{p+1}}\ra 0.
% \]
% 由于 $\ell_1$ 的长度为常值 $2\cpi$, 由\thmGrowUp 可知
% \[
%   \lim_{R_1\ra+\infty}\int_{\ell_1}g(z)\d z=0.
% \]
% 对于 $z=R_2+y\ii,0\le y\le 2\cpi$, 我们有 $\abs{w}=\ee^{R_2}$.
% 于是当 $R_2\ra+\infty$ 时, $w\ra \infty$,
% \[
%    |g(z)\abs{=}f(w)w^{p+1}|\ra 0.
% \]
% 同理可得
% \[
%   \lim_{R_2\ra+\infty}\int_{\ell_2}g(z)\d z=0.
% \]
% 由于 $\ell:z=x+2\cpi\ii,-R_1\le x\le R_2$, 因此
% \[
%    \int_\ell g(z)\d z
%   =\int_{R_2}^{-R_1}\ee^{2p\cpi\ii}\cdot f(\ee^x)\ee^{(p+1)x}\d x
%   =-\ee^{2p\cpi\ii}\int_{-R_1}^{R_2} g(x)\d x.
% \]

% 由我们的假设可知 $f(z)$ 在实轴上没有奇点, 从而 $g(z)$ 在实轴和 $\ell$ 上没有奇点.
% 令 $R_1,R_2$ 充分大, 使得 $g(z)$ 所有虚部位于 $(0,2\cpi)$ 的奇点都包含在 $C$ 中, 则由留数定理
% \begin{align*}
%    \oint_C g(z)\d z&
%   =\int_{-R_1}^{R_2} g(x)\d x
%     +\int_{\ell_1}g(z)\d z
%     -\ee^{2p\cpi\ii}\int_{-R_1}^{R_2} g(x)\d x
%     +\int_{\ell_2}g(z)\d z\\&
%   =2\cpi\ii\sum_{0<\Im a<2\cpi} \Res[g(z),a].
% \end{align*}
% 令 $R_1,R_2\ra +\infty$, 我们得到
% \[
%    (1-\ee^{2p\cpi\ii})\intf f(x)x^p\d x
%   =2\cpi\ii\sum_{0<\Im a<2\cpi} \Res[g(z),a].
% \]

% \begin{theorem}
%   \label{thm:integral-xp}
%   设实数 $p$ 不是整数, $f(x)$ 是一个有理函数, 分母没有正实根, 且满足
%   \[
%     \lim_{x\ra 0} x^{p+1}f(x)=0,\quad
%     \lim_{x\ra \infty} x^{p+1}f(x)=0.
%   \]
%   则
%   \[
%     \intf f(x)x^p\d x
%     =\frac{2\cpi\ii}{1-\ee^{2p\cpi\ii}}\sum_{0<\Im a<2\cpi} \Res[f(\ee^z)\ee^{(p+1)z},a].
%   \]
% \end{theorem}
% 这也说明了该广义积分是收敛的.
% 若用\ref{fig:large-small-circle-negative-side} 所示闭路可得
% \begin{equation}
%   \label{eq:integral-xp}
%    \intf f(x)x^p\d x
%   =-\frac{\cpi}{\sin{p\cpi}}\sum \Res[\ee^{p\ln(-z)}  f(z),a],
% \end{equation}
% 其中 $a$ 取除正实轴和零以外的奇点.
% 这两种形式是等价的.\footnote{
%   设 $0<\Im a<2\cpi$.
%   通过变量替换 $z=\ee^t$ 可将围绕点 $t=a$ 的闭路映射为围绕 $z=\ee^a$ 的闭路, 从而可以得到这两个公式中留数的关系.
%   此外, 由于 $0<\Im a<2\cpi$, 我们需要选择函数 $\ee^{p\ln(-z)+p\cpi\ii}$ 作为 $z^p$ 的单值分支, 因此系数变成了
%   \[
%      \dfrac{2\cpi\ii}{1-\ee^{2p\cpi\ii}}\cdot \ee^{p\cpi\ii}
%     =-\frac{\cpi}{\sin{p\cpi}}.
%   \]
% }

% \begin{example}
%   计算 $\intf\frac{x^p}{x(x+1)}\d x,0<p<1$.
% \end{example}

% \begin{solution}
%   设
%   \[
%     f(x)=\frac1{x(x+1)},\quad
%     g(z)=f(\ee^z)\ee^{(p+1)z}=\frac{\ee^{pz}}{\ee^z+1},
%   \]
%   则 $g(z)$ 在 $0<\Im a<2\cpi$ 范围内的奇点为 $a=\cpi\ii$, 且
%   \[
%     \Res[g(z),\cpi\ii]
%     =\lim_{z\ra \cpi\ii}\frac{\ee^{pz}(z-\cpi\ii)}{\ee^z+1}
%     =\ee^{p\cpi\ii}\lim_{z\ra \cpi\ii}\frac{1}{\ee^z}
%     =-\ee^{p\cpi\ii}.
%   \]
%   因此由\thmref{定理}{thm:integral-xp} 可得
%   \[
%      \intf\frac{x^p}{x(x+1)}\d x
%     =-\frac{2\cpi\ii}{1-\ee^{2p\cpi\ii}}\ee^{p\cpi\ii}
%     =\frac{2\cpi\ii}{\ee^{p\cpi\ii}-\ee^{-p\cpi\ii}}
%     =\frac{\cpi}{\sin p\cpi}.
%   \]
% \end{solution}

% \begin{solution}[另解]
%   设
%   \[
%     f(z)=\frac{\ee^{p\ln(-z)}}{z(z+1)},
%   \]
%   则 $f(z)$ 在正实轴和零以外的奇点为 $a=-1$, 且
%   \[
%      \Res[f(z),-1]
%     =\lim_{z\ra-1}\frac{\ee^{p\ln(-z)}}{z}
%     =-\ee^{p\ln 1}
%     =-1.
%   \]
%   因此由\eqref{eq:integral-xp} 可得
%   \[
%      \intf\frac{x^p}{x(x+1)}\d x
%     =-\frac{\cpi}{\sin{p\cpi}}\cdot(-1)
%     =\frac{\cpi}{\sin p\cpi}.
%   \]
% \end{solution}

% 考虑 $\intf f(x)\ln x\d x$, 其中 $f(x)$ 是一个有理函数, 分母没有正实根, 且分母至少比分子高 $2$ 次.
% 通过变量替换 $x=\ee^t$ 可将其化为
% \[
%    \intf f(x)\ln x\d x
%   =\intff f(\ee^t)t\ee^t\d t.
% \]
% 设 $w=\ee^z$, $g(z)=z^2\ee^zf(\ee^z)$.
% 考虑\ref{fig:rectangle-2pii} 所示的闭路, 类似可知
% \[
%   \lim_{R_1\ra+\infty}\int_{\ell_1}g(z)\d z=0,\quad 
%   \lim_{R_2\ra+\infty}\int_{\ell_2}g(z)\d z=0.
% \]
% 对于 $z=x+2\cpi\ii,-R_1\le x\le R_2$, 我们有
% \[
%   g(z)=(x+2\cpi\ii)^2\ee^xf(\ee^x)=g(x)+4\cpi\ii x\ee^xf(\ee^x)-4\cpi^2\ee^xf(\ee^x),
% \]
% 因此
% \begin{align*}
%    \int_\ell g(z)\d z
%   &=-\int_{-R_1}^{R_2}(x+2\cpi\ii)^2 \ee^xf(\ee^x)\d x\\
%   &=-\int_{-R_1}^{R_2} g(x)\d x
%   -4\cpi\ii \int_{-R_1}^{R_2}x\ee^xf(\ee^x)\d x
%   +4\cpi^2 \int_{-R_1}^{R_2}\ee^xf(\ee^x)\d x.
% \end{align*}
% 现在对整个闭路使用留数定理并令 $R_1,R_2\ra +\infty$, 我们得到
% \[
%   -4\cpi\ii \intff x\ee^xf(\ee^x)\d x
%   +4\cpi^2 \intff \ee^xf(\ee^x)\d x
%   =2\cpi\ii\sum_{0<\Im a<2\cpi} \Res[z^2\ee^zf(\ee^z),a],
% \]
% \[
%   -2\intf f(x)\ln x\d x
%   -2\cpi\ii\intf f(x)\d x
%   =\sum_{0<\Im a<2\cpi} \Res[z^2\ee^zf(\ee^z),a],
% \]

% \begin{theorem}
%   设 $f(x)$ 是一个有理函数, 分母没有正实根, 且分母至少比分子高 $2$ 次, 则
%   \[
%     \intf f(x)\ln x\d x
%     +\cpi\ii\intf f(x)\d x
%     =-\frac12 \sum_{0<\Im a<2\cpi} \Res[z^2\ee^zf(\ee^z),a].
%   \]
% \end{theorem}
% 若采用不同闭路, 该积分也会有其它表现形式.

% 对于含 $(\ln x)^n$ 的积分也可类似处理, 此时需要考虑 $z^k \ee^zf(\ee^z),k=2,3,\cdots,n+1$ 绕闭路的积分.

% \begin{exercise}
%   计算 $\intf\dfrac{\ln x}{x^2-2x+2}$.
% \end{exercise}

% \begin{solution}
%   设
%   \[
%     f(z)=\dfrac1{x^2-2x+2},\quad
%     g(z)=z^2\ee^zf(\ee^z)=\frac{z^2}{\ee^z-2+2\ee^{-z}},
%   \]
%   则 $g(z)$ 在 $0<\Im a<2\cpi$ 范围内的奇点为
%   \[
%     a=\ln\sqrt 2+\frac\cpi4i,\quad
%       \ln\sqrt2+\frac{7\cpi}4i,
%   \]
%   且
%   \begin{align*}
%     \Res[g(z),a]
%     &=\lim_{z\ra a}\frac{z^2(z-a)}{\ee^z-2+2\ee^{-z}}
%     =a^2\lim_{z\ra a}\frac1{\ee^z-2\ee^{-z}}\\
%     &=\begin{cases}
%       \dfrac1{2\ii}\Bigl(\ln\sqrt2+\dfrac\cpi4i\Bigr)^2,&a=\ln\sqrt2+\dfrac\cpi4i;\\[2\itemsep]
%       -\dfrac1{2\ii}\Bigl(\ln\sqrt2+\dfrac{7\cpi}4i\Bigr)^2,&a=\ln\sqrt2+\dfrac{7\cpi}4i.
%     \end{cases}
%   \end{align*}
%   因此
%   \[
%      \intf f(x)\ln x\d x
%     =\Re\Bigl(\frac{3\cpi}8\ln2+\frac34\cpi^2\ii\Bigr)
%     =\frac{3\cpi}8\ln2.
%   \]
% \end{solution}

% \begin{exercise}
%   计算 $\intf\frac{\ln x}{(x^2+1)^2}\d x$.
% \end{exercise}
