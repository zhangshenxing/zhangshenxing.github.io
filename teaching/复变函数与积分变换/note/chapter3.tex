
\chapter{复变函数的积分}
\section{复变函数积分的概念}

\subsection{复变函数积分的定义}

设 $C$ 是平面上一条光滑或逐段光滑的连续曲线, 也就是说它的参数方程 $z=z(t),a\le t\le b$ 除去有限个点之外都有非零导数.
固定它的一个方向, 称为\emph{正方向}, 则我们得到一条\emph{有向曲线}.
和这条曲线方向相反的记作 $C^-$, 它的方向被称为该曲线\emph{负方向}.

对于闭路, 规定它的\alert{正方向是指逆时针方向}, 负方向是指顺时针方向.
以后我们不加说明的话\alert{默认是正方向}.

\begin{figure}[hbpt]
	\centering
	\begin{tikzpicture}
		\draw[cstaxis](-3.5,0)--(3.5,0);
		\draw[cstaxis](0,-0.5)--(0,2.5);
		\begin{scope}[xshift=18mm,yshift=11mm,second,cstdot,cstcurve,smooth]
			\coordinate [label=left:{$A=z(a)$}] (A) at ({1.3*cos(-35)}, {1.3*sin(-35)});
			\coordinate [label=below:{$B=z(b)$}] (B) at ({1.3*cos(125)}, {1.3*sin(125)});
			\draw[main,domain=-35:125] plot ({1.3*cos(\x)}, {1.3*sin(\x)});
			\draw[main,domain=40:45,cstwra] plot ({1.3*cos(\x)}, {1.3*sin(\x)});
			\fill (A) circle;
			\fill (B) circle;
		\end{scope}
		\begin{scope}[xshift=-14mm,yshift=11mm,cstcurve,main,smooth,cstwla]
			\draw[domain=-65:30] plot ({-0.7*cos(\x)}, {0.7*sin(\x)});
			\draw[domain=25:120] plot ({-0.7*cos(\x)}, {0.7*sin(\x)});
			\draw[domain=115:210] plot ({-0.7*cos(\x)}, {0.7*sin(\x)});
			\draw[domain=205:300] plot ({-0.7*cos(\x)}, {0.7*sin(\x)});
		\end{scope}
	\end{tikzpicture}
	\caption{有向曲线}
\end{figure}

所谓的复变函数积分, 本质上仍然是第二类曲线积分.
设复变函数
\[
	w=f(z)=u(x,y)+iv(x,y)
\]
定义在区域 $D$ 内, 有向曲线 $C$ 包含在 $D$ 中.
形式地展开
\[
	f(z)\diff z=(u+iv)(\diff x+i\diff y)=(u\diff x-v\diff y)+i(u\diff y+v\diff x).
\]
\begin{definition}
	如果下述右侧两个线积分均存在, 则定义
	\[\int_C f(z)\diff z=\int_C(u\diff x-v\diff y)+i\int_C(v\diff x+u\diff y)\]
	为\emph{函数 $f(z)$ 沿曲线 $C$ 的积分}.
\end{definition}

我们也可以像线积分那样通过分割来定义.
在曲线 $C$ 上依次选择分点 $z_0=A,z_1,\dots,z_n=B$, 在每一段弧上任取 $\zeta_k\in\warc{z_{k-1}z_k}$ 并作和式
\[
	S_n=\sum_{k=1}^n f(\zeta_k)\Delta z_k,\quad \Delta z_k=z_k-z_{k-1}.
\]
称 $n\ra\infty$, 分割的最大弧长 $\ra 0$ 时 $S_n$ 的极限为复变函数积分.
这二者是等价的.

\begin{center}
	\begin{tikzpicture}
		\draw[cstcurve,third,smooth,domain=0:360] plot ({0.015*\x},{0.6*sin(\x)});
		\draw[cstcurve,third,smooth,domain=195:200,cstra] plot ({0.015*\x},{0.6*sin(\x)});
		\fill[cstdot,third] (0,0) circle;
		\fill[cstdot,second]({0.015*32},{0.6*sin(32)}) circle;
		\fill[cstdot,main]({0.015*64},{0.6*sin(64)}) circle;
		\fill[cstdot,second]({0.015*96},{0.6*sin(96)}) circle;
		\fill[cstdot,main]({0.015*128},{0.6*sin(128)}) circle;
		\fill[cstdot,second]({0.015*160},{0.6*sin(160)}) circle;
		\fill[cstdot,main]({0.015*232},{0.6*sin(232)}) circle;
		\fill[cstdot,second]({0.015*264},{0.6*sin(264)}) circle;
		\fill[cstdot,main]({0.015*296},{0.6*sin(296)}) circle;
		\fill[cstdot,second]({0.015*328},{0.6*sin(328)}) circle;
		\fill[cstdot,third] ({0.015*360},0) circle;
		\draw
		({0.015*32},{0.6*sin(32)-0.3}) node[second] {$\zeta_1$}
		({0.015*64},{0.6*sin(64)+0.3}) node[main] {$z_1$}
		({0.015*96},{0.6*sin(96)-0.3}) node[second] {$\zeta_2$}
		({0.015*128},{0.6*sin(128)+0.3}) node[main] {$z_2$}
		({0.015*160},{0.6*sin(160)-0.3}) node[second] {$\zeta_3$}
		({0.015*232},{0.6*sin(232)+0.3}) node[main] {$z_{n-2}$}
		({0.015*264},{0.6*sin(264)-0.3}) node[second] {$\zeta_{n-1}$}
		({0.015*296},{0.6*sin(296)+0.3}) node[main] {$z_{n-1}$}
		({0.015*328},{0.6*sin(328)-0.3}) node[second] {$\zeta_n$}
		({0.015*180},{0.6*sin(180)+0.5}) node[main] {$\ddots$}
		(-0.3,0) node[third] {$A$}
		({0.015*360+0.3},0) node[third] {$B$};
	\end{tikzpicture}
\end{center}

如果 $C$ 是闭曲线, 则该积分记为 \alert{$\displaystyle\oint_Cf(z)\diff z$}. 此时该积分不依赖端点的选取.

如果 $C$ 是实轴上的区间 $[a,b]$ 且 $f(z)=u(x)$, 则
\[
	\int_Cf(z)\diff z=\int_a^bf(z)\diff z=\int_a^b u(x)\diff x
\]
就是黎曼积分.

根据线积分的存在性条件可知:
\begin{theorem}
	如果 $f(z)$ 在 $D$ 内连续, $C$ 是光滑曲线, 则 $\displaystyle\int_Cf(z)\diff z$ 总存在.
\end{theorem}


\subsection{复变函数积分的计算法}

线积分中诸如变量替换等技巧可以照搬过来使用. 设
\[
	C:z(t)=x(t)+iy(t),\quad a\le t\le b
\]
是一条光滑有向曲线, 且正方向为 $t$ 增加的方向, 则 $\diff z=z'(t)\diff t$.

\begin{theorem}[复变函数积分计算方法I]
	\[
		\int_Cf(z)\diff z=\int_a^b f(z)z'(t)\diff t.
	\]
\end{theorem}
如果 $C$ 的正方向是从 $z(b)$ 到 $z(a)$, 则需要交换右侧积分的上下限.

如果 $C$ 是逐段光滑的, 则相应的积分就是各段的积分之和. 以后我们\alert{只考虑逐段光滑曲线上的连续函数的积分}.

\begin{example}
	求 $\displaystyle\int_Cz\diff z$, 其中 $C$ 是
	\begin{enumerate}
		\item 从原点到点 $3+4i$ 的直线段;
		\item 抛物线 $y=\dfrac49x^2$ 上从原点到点 $3+4i$ 的曲线段.
	\end{enumerate}
\end{example}

\begin{figure}[hbpt]
	\centering
	\begin{minipage}{.48\textwidth}
		\centering
		\begin{tikzpicture}
			\draw[cstaxis](0,0)--(3,0);
			\draw[cstaxis](0,0)--(0,2.5);
			\draw[cstcurve,main,cstra](0,0)--(1.5,2);
			\draw
				(2.5,1.2) node[main,align=center] {$z=(3+4i)t$\\$0\le t\le 1$};
		\end{tikzpicture}
	\end{minipage}
	\begin{minipage}{.48\textwidth}
		\centering
		\begin{tikzpicture}
			\draw[cstaxis](0,0)--(3,0);
			\draw[cstaxis](0,0)--(0,2.5);
			\draw[cstcurve,main,domain=0:1.5,cstra] plot({\x},{8*\x*\x/9});
			\draw
				(2.8,2) node[main,align=center] {$z=t+\dfrac49it^2$\\[1mm]$0\le t\le 3$};
		\end{tikzpicture}
	\end{minipage}
\end{figure}

\begin{solution}
	\begin{enumerate}
		\item 由于 $z=(3+4i)t,0\le t\le 1$, 因此
		\[
			 \int_Cz\diff z=\int_0^1(3+4i)t\cdot(3+4i)\diff t
			=(3+4i)^2\int_0^1t\diff t
			=\half (3+4i)^2=-\frac72+12i.
		\]
		\item 由于 $z=t+\dfrac49it^2,0\le t\le 3$, 因此
		\begin{align*}
			  \int_Cz\diff z
			&=\int_0^3\biggl(t+\frac{4}9it^2\biggr)\cdot\biggl(1+\frac89it\biggr)\diff t
			 =\int_0^3\biggl(t+\frac43it^2-\frac{32}{81}t^3\biggr)\diff t\\
			&=\biggl(\half t^2+\frac49it^3-\frac8{81}t^4\biggr)\Big|_0^3
			 =-\frac72+12i.
		\end{align*}
	\end{enumerate}
\end{solution}

\begin{example}
	求 $\displaystyle\int_C\Re z\diff z$, 其中 $C$ 是
	\begin{enumerate}
		\item 从原点到点 $1+i$ 的直线段;
		\item 从原点到点 $i$ 再到 $1+i$ 的折线段.
	\end{enumerate}
\end{example}

\begin{figure}[hbpt]
	\centering
	\begin{minipage}{.48\textwidth}
		\centering
		\begin{tikzpicture}
			\draw[cstaxis](0,0)--(3,0);
			\draw[cstaxis](0,0)--(0,2);
			\draw[cstcurve,main,cstra](0,0)--(2,2);
			\draw (2.4,0.9) node[align=center,main] {$z=(1+i)t$\\$0\le t\le 1$};
		\end{tikzpicture}
	\end{minipage}
	\begin{minipage}{.48\textwidth}
		\centering
		\begin{tikzpicture}
			\draw[cstaxis](0,0)--(2.5,0);
			\draw[cstaxis](0,0)--(0,2);
			\draw[cstcurve,cstra,main](0,0)--(0,1.5);
			\draw[cstcurve,cstra,second](0,1.5)--(1.5,1.5);
			\draw (0,.75) node[left,main,align=center] {$z=it$\\$0\le t\le 1$};
			\draw (1.5,1.5) node[right,second,align=center] {$z=t+i$\\$0\le t\le 1$};
		\end{tikzpicture}
	\end{minipage}
\end{figure}

\begin{solution}
	\begin{enumerate}
		\item 由于 $z=(1+i)t,0\le t\le 1$, 因此 $\Re z=t$,
			\[
				\int_C\Re z\diff z=\int_0^1t\cdot(1+i)\diff t
				=(1+i)\int_0^1t\diff t
				=\frac{1+i}2.
			\]
		\item 第一段 $z=it,0\le t\le 1, \Re z=0$,
			第二段 $z=t+i$, $0\le t\le 1$, $\Re z=t$. 因此
			\[
				\int_C\Re z\diff z=\int_0^1 t\diff t=\frac12.
			\]
	\end{enumerate}
\end{solution}

可以看出, 即便起点和终点相同, 沿不同路径 $f(z)=\Re z$ 的积分也可能不同.
而 $f(z)=z$ 的积分则只和起点和终点位置有关, 与路径无关.
原因在于 $f(z)=z$ 是处处解析的, 我们以后会详加解释.

\begin{exercise}
	求 $\displaystyle\int_C\Im z\diff z=$\fillblank[3cm][2mm]{}, 其中 $C$ 是从原点沿 $y=x$ 到点 $1+i$ 再到 $i$ 的折线段.
\end{exercise}
% \begin{center}
% 	\begin{tikzpicture}
% 		\draw[cstaxis](0,0)--(2.5,0);
% 		\draw[cstaxis](0,0)--(0,2.5);
% 		\draw[cstcurve,cstra,main](0,0)--(2,2);
% 		\draw[cstcurve,cstra,second](2,2)--(0,2);
% 	\end{tikzpicture}
% \end{center}

\begin{example}
	求 $\displaystyle\oint_{|z-z_0|=r}\frac{\diff z}{(z-z_0)^{n+1}}$, 其中 $n$ 为整数.
\end{example}
\begin{center}
	\begin{tikzpicture}
		\draw[cstcurve,second] ({1.3*cos(120)},{1.3*sin(120)})coordinate (C) -- (0,0)coordinate (B) --(1.3,0)coordinate (A) pic [draw,second,"$\theta$",angle eccentricity=1.7,angle radius=3mm] {angle};
		\fill[cstdot,second](0,0) circle;
		\draw[cstcurve,main,cstwla]({1.3*cos(45)},{1.3*sin(45)}) arc (45:-320:1.3);
		\draw
			(-0.5,0.5) node[second] {$r$}
			(0,-0.3) node[second] {$z_0$};
	\end{tikzpicture}
\end{center}
\begin{solution}
	$C: |z-z_0|=r$ 的参数方程为 $z=z_0+re^{i\theta},0\le \theta\le 2\pi$.
	于是 $\diff z=ire^{i\theta}\diff \theta$.
	\[
		\oint_C\frac{\diff z}{(z-z_0)^{n+1}}
		=\int_0^{2\pi}i(re^{i\theta})^{-n}\diff\theta
		=ir^{-n}\int_0^{2\pi}e^{-in\theta}\diff\theta.
	\]
	\begin{itemize}
		\item 当 $n=0$ 时, 该积分值为 $2\pi i$.
		\item 当 $n\neq 0$ 时, 该积分值 $=\dfrac{ir^{-n}}{-in}e^{-in\theta}\big|_0^{2\pi}=0$.
	\end{itemize}
\end{solution}
所以
\begin{theorem}[幂函数沿圆周的积分]
	\[\oint_{|z-z_0|=r}\frac{\diff z}{(z-z_0)^{n+1}}=\begin{cases}2\pi i,&n=0;\\0,&n\neq 0.\end{cases}\]
\end{theorem}

这个积分以后经常用到, 它的特点是积分值与圆周的圆心和半径都无关.

\begin{theorem}[积分的线性性质]
	\begin{enumerate}
		\item $\displaystyle\int_Cf(z)\diff z=-\displaystyle\int_{C^-}f(z)\diff z$.
		\item $\displaystyle\int_Ckf(z)\diff z=k\displaystyle\int_Cf(z)\diff z$.
		\item $\displaystyle\int_C[f(z)\pm g(z)]\diff z
		=\displaystyle\int_Cf(z)\diff z\pm\displaystyle\int_Cg(z)\diff z$.
	\end{enumerate}
\end{theorem}

\begin{theorem}[长大不等式]
	设 $C$ 的长度为 $L$, $f(z)$ 在 $C$ 上满足 $|f(z)|\le M$, 则
		\[\abs{\int_Cf(z)\diff z}\le\int_C|f(z)|\diff s\le ML.\]
\end{theorem}

\begin{proof}
	由
		\[\abs{\sum_{k=1}^n f(\zeta_k)\Delta z_k}
		\le\sum_{k=1}^n|f(\zeta_k)\Delta z_k|
		\le\sum_{k=1}^n|f(\zeta_k)|\Delta s_k
		\le M\sum_{k=1}^n\Delta s_k\]
	可知
		\[\abs{\int_Cf(z)\diff z}\le\int_C|f(z)|\diff s\le ML.\qedhere\]
\end{proof}

长大不等式常常用于证明等式: 估算一个积分和一个具体的数值之差不超过任意给定的 $\varepsilon$, 从而得到二者相等.

注意到: 如果被积函数 $f(z)$ 在 $C$ 上的点都连续, 那么 $|f(z)|$ 是 $C$ 的参变量 $t\in[a,b]$ 的连续函数, 从而有界, 即存在 $M$ 使得 $|f(z)|\le M,\forall z\in C$.

\begin{example}
	设 $f(z)$ 在 $z\neq a$ 处连续, 且 $\lim\limits_{z\ra a}(z-a)f(z)=k$, 则
	\[\lim_{r\ra0}\oint_{|z-a|=r}f(z)\diff z=2\pi ik.\]
\end{example}

\begin{proof}
	$\forall \varepsilon>0,\exists\delta>0$ 使得当 $|z-a|<\delta$ 时, $|(z-a)f(z)-k|\le\varepsilon$.
	当 $0<r<\delta$ 时,
	\begin{align*}
		 &\abs{\oint_{|z-a|=r}f(z)\diff z-2\pi i k}
		=\abs{\oint_{|z-a|=r}\left[f(z)-\frac k{z-a}\right]\diff z}\\
		=&\abs{\oint_{|z-a|=r}\frac{(z-a)f(z)-k}{z-a}\diff z}
		\le \frac{\varepsilon}r\cdot 2\pi r=2\pi\varepsilon.
	\end{align*}
	由于 $\varepsilon$ 是任意的, 因此命题得证.
\end{proof}



\section{柯西-古萨基本定理和复合闭路定理}

\subsection{柯西-古萨基本定理}

观察下方的两条曲线 $C_1,C_2$.
设 $C=C_1^-+C_2$.
可以看出
\[\int_{C_1}f(z)\diff z=\int_{C_2}f(z)\diff z\iff
\oint_Cf(z)\diff z=\int_{C_2}f(z)\diff z-\int_{C_1}f(z)\diff z=0.\]
所以 $f(z)$ 的积分只与起点终点有关 $\iff f(z)$ 绕任意闭路的积分为零.


\begin{center}
	\begin{tikzpicture}
		\draw
			(-0.2,-0.2) node[second] {$z_0$}
			(2.2,2.2) node[second] {$z$}
			(0.2,1.8) node[main] {$C_1$}
			(1.8,0.3) node[third] {$C_2$};
		\draw[cstcurve,third] (2,2) arc(0:-90:2);
		\draw[cstcurve,third,cstwra] ({2*cos(-50)},{2+2*sin(-50)}) arc(-50:-45:2);
		\draw[cstcurve,main] (0,0) arc(180:90:2);
		\draw[cstcurve,main,cstwra] ({2+2*cos(140)},{2*sin(140)}) arc(140:135:2);
		\draw[cstcurve,second,cstwla] ({1+0.3*cos(135)},{1+0.3*sin(135)}) arc(135:-110:0.3);
		\fill[cstdot,second] (0,0) circle;
		\fill[cstdot,second] (2,2) circle;
	\end{tikzpicture}
\end{center}

上一节中我们计算了 $f(z)=z,\Re z,\dfrac1{z-z_0}$ 的积分.
其中
\begin{itemize}
	\item $f(z)=z$ 处处解析, 积分只与起点终点有关 (闭路积分为零);
	\item $f(z)=\dfrac1{z-z_0}$ 有奇点 $z_0$, 沿绕 $z_0$ 闭路的积分非零;
	\item $f(z)=\Re z$ 处处不解析, 积分与路径有关 (闭路积分非零).
\end{itemize}
由此可见函数沿闭路积分为零,
与函数在闭路内部是否解析有关.

设 $C$ 是一条闭路, $D$ 是其内部区域.
设 \alert{$f(z)$ 在闭区域 $\ov D=D\cup C$ 上解析},
即存在区域 $B\supseteq\ov D$ 使得 $f(z)$ 在 $B$ 上解析.
为了简便假设 $f'(z)$ 连续,
则
\[\oint_Cf(z)\diff z=\oint_C(u\diff x-v\diff y)
+i\oint_C(v\diff x+u\diff y).\]
由格林公式和C-R方程可知
\[\oint_Cf(z)\diff z=-\iint_D(v_x+u_y)\diff x\diff y
+i\iint_D(u_x-v_y)\diff x\diff y=0.\]
也可以从
\[\oint_Cf(z)\diff z=-\iint_D\pp f{\ov z}\diff z\diff \ov z=2i\iint_D\pp f{\ov z}\diff x\diff y=0
\footnote{这是因为 \[\begin{vmatrix}
	\dpp zx&\dpp zy\\
	\dpp {\ov z}x&\dpp {\ov z}y
\end{vmatrix}=\begin{vmatrix}
	1&i\\
	1&-i
\end{vmatrix}=-2i.\]}\]
看出.

\begin{theorem}[柯西-古萨基本定理]
设 $f(z)$ 在闭路 $C$ 上连续, $C$ 内部解析, 则 $\displaystyle\oint_Cf(z)\diff z=0$.
\end{theorem}

\begin{corollary}
设 $f(z)$ 在单连通域 $D$ 内解析, $C$ 是 $D$ 内一条闭合曲线(可以不是闭路), 则 $\displaystyle\oint_Cf(z)\diff z=0$.
\end{corollary}

这是因为即使不是简单曲线也可以拆分为一些简单曲线.

\begin{center}
	\begin{tikzpicture}
		\draw[cstcurve,main,smooth,domain=-45:-15] plot({3*sqrt(cos(2*\x))*cos(\x)},{3*sqrt(cos(2*\x))*sin(\x)});
		\draw[cstcurve,main,smooth,domain=-20:45,cstwla] plot({3*sqrt(cos(2*\x))*cos(\x)},{3*sqrt(cos(2*\x))*sin(\x)});
		\draw[cstcurve,main,smooth,domain=-45:-15] plot({-2*sqrt(cos(2*\x))*cos(\x)},{2*sqrt(cos(2*\x))*sin(\x)});
		\draw[cstcurve,main,smooth,domain=-20:45,cstwla] plot({-2*sqrt(cos(2*\x))*cos(\x)},{2*sqrt(cos(2*\x))*sin(\x)});
		\draw[cstcurve,second,cstwla,ultra thick] (1.05,0.35) arc (135:-135:0.5);
		\draw[cstcurve,third,cstwla,ultra thick] (-0.75,-0.25) arc (315:45:0.3535);
	\end{tikzpicture}
\end{center}

\begin{example}
	求 $\displaystyle\oint_{|z|=1}\frac1{2z-3}\diff z$.
\end{example}

\begin{solution}
	由于 $\dfrac1{2z-3}$ 在 $|z|\le 1$ 上解析,
	{因此由柯西-古萨基本定理 $\displaystyle \oint_{|z|=1}\frac1{2z-3}\diff z=0$.}
\end{solution}

\begin{exercise}
	\begin{enumerate}
		\item $\displaystyle\oint_{|z-2|=1}\frac1{z^2+z}\diff z=$\fillblank{}.
		\item 求 $\displaystyle\oint_{|z|=2}\dfrac{\sin z}{|z|}\diff z=$\fillblank{}.
	\end{enumerate}
\end{exercise}

\begin{example}
	求 $\displaystyle\oint_C\frac1{z(z^2+1)}\diff z$, 其中 $C:|z-i|=\half $.
\end{example}

\begin{solution}
	注意到 
	\[\dfrac1{z(z^2+1)}=\dfrac1z-\dfrac12\left(\dfrac1{z+i}+\dfrac1{z-i}\right).\]
	由于 $\dfrac1z,\dfrac1{z+i}$ 在 $|z-i|\le\dfrac12$ 上解析,因此由柯西-古萨基本定理
		\[\oint_C\frac1z\diff z
		=\oint_C\frac1{z+i}\diff z=0,\]
		\[\oint_C\frac1{z(z^2+1)}\diff z
		=-\half\oint_C\frac1{z-i}\diff z=-\pi i.\]
\end{solution}


\subsection{复合闭路定理}

设 $C_0,C_1,\dots,C_n$ 是 $n+1$ 条简单闭曲线, $C_1,\dots,C_n$ 每一条都包含在其它闭路的外部, 而且它们都包含在 $C_0$ 的内部.
这样它们围成了一个多连通区域 $D$, 它的边界称为一个\emph{复合闭路}
\[C=C_0+C_1^-+\cdots+C_n^-.\]
沿着 $C$ 前进的点, $D$ 总在它的左侧, 因此我们这样规定它的正方向.

\begin{figure}[hbpt]
	\centering
	\begin{tikzpicture}
		\fill[cstcurve,cstfill] (0,0) circle(2.5 and 1.5);
		\fill[cstcurve,white] (1,0) circle(0.6);
		\fill[cstcurve,white] (-1,0) circle(0.6);
		\draw[cstcurve,main] (0,0) circle(2.5 and 1.5);
		\draw[cstcurve,main] (1,0) circle(0.6);
		\draw[cstcurve,main] (-1,0) circle(0.6);
		\draw[cstcurve,main,domain=85:90,cstwra] plot ({2.5*cos(\x)}, {1.5*sin(\x)});
		\draw[cstcurve,main,domain=135:140,cstwla] plot ({1+0.6*cos(\x)}, {0.6*sin(\x)});
		\draw[cstcurve,main,domain=135:140,cstwla] plot ({-1+0.6*cos(\x)}, {0.6*sin(\x)});
		\draw
			(2,1.3) node[main] {$C_0$}
			(-1,0.85) node[main] {$C_1^{-}$}
			(1,0.85) node[main] {$C_2^{-}$};
	\end{tikzpicture}
	\caption{复合闭路围成的区域}
\end{figure}

\begin{theorem}{复合闭路定理}
	设 $f(z)$ 在复合闭路 $C=C_0+C_1^-+\cdots+C_n^-$ 及其所围成的多连通区域内解析, 则
	\[\oint_{C_0}f(z)\diff z=
	\oint_{C_1}f(z)\diff z+\cdots+\oint_{C_n}f(z)\diff z.\]
\end{theorem}

\begin{proof}
	以曲线 $\gamma_1,\gamma_2,\dots,\gamma_{n+1}$ 把 $C_0,C_1,\dots,C_n$ 连接起来, 则它们把区域 $D$ 分成了两个单连通域 $D_1,D_2$.
	对 $D_1$ 和 $D_2$ 的边界应用柯西-古萨基本定理并相加, 则 $\gamma_i$ 对应的部分正好相互抵消,因此
		\[\oint_{C_0}f(z)\diff z-
	\oint_{C_1}f(z)\diff z-\cdots-\oint_{C_n}f(z)\diff z=0.\]于是定理得证.
\end{proof}
\begin{center}
	\begin{tikzpicture}
		\fill[cstcurve,cstfill] (0,0) circle(2.5 and 1.5);
		\fill[cstcurve,white] (1,0) circle(0.6);
		\fill[cstcurve,white] (-1,0) circle(0.6);

		\draw[cstcurve,main,domain=0:90,cstwra] plot ({2.5*cos(\x)}, {1.5*sin(\x)});
		\draw[cstcurve,main,domain=85:180] plot ({2.5*cos(\x)}, {1.5*sin(\x)});
		\draw[cstcurve,second,domain=180:270,cstwra] plot ({2.5*cos(\x)}, {1.5*sin(\x)});
		\draw[cstcurve,second,domain=265:360] plot ({2.5*cos(\x)}, {1.5*sin(\x)});

		\draw[cstcurve,main,domain=0:140] plot ({1+0.6*cos(\x)}, {0.6*sin(\x)});
		\draw[cstcurve,main,domain=135:180,cstwla] plot ({1+0.6*cos(\x)}, {0.6*sin(\x)});
		\draw[cstcurve,second,domain=180:320] plot ({1+0.6*cos(\x)}, {0.6*sin(\x)});
		\draw[cstcurve,second,domain=315:360,cstwla] plot ({1+0.6*cos(\x)}, {0.6*sin(\x)});

		\draw[cstcurve,main,domain=0:140] plot ({-1+0.6*cos(\x)}, {0.6*sin(\x)});
		\draw[cstcurve,main,domain=135:180,cstwla] plot ({-1+0.6*cos(\x)}, {0.6*sin(\x)});
		\draw[cstcurve,second,domain=180:320] plot ({-1+0.6*cos(\x)}, {0.6*sin(\x)});
		\draw[cstcurve,second,domain=315:360,cstwla] plot ({-1+0.6*cos(\x)}, {0.6*sin(\x)});

		\draw
			(2,1.3) node[main] {$C_0$}
			(-1,0.85) node[main] {$C_1^-$}
			(1,0.85) node[main] {$C_2^-$}
			(-2,-0.3) node {$\gamma_1$}
			(0,-0.3) node {$\gamma_2$}
			(2,-0.3) node {$\gamma_3$};
		\draw[cstcurve] (-2.5,0)--(-1.6,0);
		\draw[cstcurve] (-0.4,0)--(0.4,0);
		\draw[cstcurve] (1.6,0)--(2.5,0);
		\draw[cstcurve,thick,main] (-2.5,0.025)--(-1.6,0.025);
		\draw[cstcurve,thick,second] (-2.5,-0.025)--(-1.6,-0.025);
		\draw[cstcurve,thick,main] (-0.4,0.025)--(0.4,0.025);
		\draw[cstcurve,thick,second] (-0.4,-0.025)--(0.4,-0.025);
		\draw[cstcurve,thick,main] (1.6,0.025)--(2.5,0.025);
		\draw[cstcurve,thick,second] (1.6,-0.025)--(2.5,-0.025);
	\end{tikzpicture}
\end{center}

在实际应用中, 如果被积函数 $f(z)$ 在闭路 $C$ 的内部有有限多个奇点 $z_1,\dots,z_k$.
那么我们可以在 $C$ 内部构造闭路 $C_1,\dots,C_k$, 使得每个 $C_j$ 内部只包含一个奇点 $z_j$.
这样, 内部含多个奇点的情形就可以化成内部只含一个奇点的情形. 最后将这些闭路上的积分相加即可.

\begin{center}
	\begin{tikzpicture}
		\draw[cstcurve,main] (0,0) circle(2.2 and 1.2);
		\draw[cstcurve,second] (1,0) circle(0.4);
		\draw[cstcurve,second] (0,0) circle(0.4);
		\draw[cstcurve,second] (-1,0) circle(0.4);
		\fill[cstdot,second] (1,0) circle;
		\fill[cstdot,second] (0,0) circle;
		\fill[cstdot,second] (-1,0) circle;
		\draw
			(-1,-0.25) node[second] {$z_1$}
			(0,-0.25) node[second] {$z_2$}
			(1,-0.25) node[second] {$z_3$}
			(1.8,1.1) node[main] {$C_0$}
			(-1,0.65) node[second] {$C_1$}
			(0,0.65) node[second] {$C_2$}
			(1,0.65) node[second] {$C_3$};
	\end{tikzpicture}
\end{center}

此外, 从复合闭路定理还可以看出, 在计算积分 $\displaystyle \oint_C f(z)\diff z$ 时, $C$ 的具体形状无关紧要, 只要其内部奇点不变, $C$ 可以任意变形.
因为我们总可以选择一个包含这些奇点的闭路 $C'$, 使得 $C'$ 包含在 $C$ 及其变形后的闭路内部. 这样它们的积分自然都和 $C'$ 上的积分相同.

\begin{center}
	\begin{tikzpicture}
		\draw[cstcurve,main] (0,0) circle(2.2 and 1.2);
		\draw[cstcurve,third] (0,0) circle(1.2 and 2.2);
		\draw[cstcurve,second] (0,0) circle(1);
		\fill[cstdot,second] (.7,0.2) circle;
		\fill[cstdot,second] (0,-0.4) circle;
		\fill[cstdot,second] (-.5,0.1) circle;
		\draw
			(1.8,1.1) node[main] {$C_1$}
			(-1.1,1.8) node[third] {$C_2$}
			(-.4,0.65) node[second] {$C'$};
	\end{tikzpicture}
\end{center}

\begin{example}
	证明对于任意闭路 $C$, $\displaystyle\int_C(z-a)^n\diff z=0$, $n\neq -1$ 为整数.
\end{example}

\begin{proof}
	如果 $a$ 不在 $C$ 的内部,则 $(z-a)^n$ 在 $C$ 及其内部解析.由柯西-古萨基本定理, $\displaystyle\int_C(z-a)^n\diff z=0$.

	如果 $a$ 在 $C$ 的内部, 则在 $C$ 的内部取一个以 $a$ 为圆心的圆周 $C_1$.
	由复合闭路定理以及上一节的结论
		\[\int_C(z-a)^n\diff z=\int_{C_1}(z-a)^n\diff z=0.\qedhere\]
\end{proof}

同理, 由复合闭路定理和上一节的结论可知当 $a$ 在 $C$ 的内部且 $n=-1$ 时积分为 $2\pi i$.
\begin{theorem}[幂函数沿闭路的积分]
	当 $a$ 在 $C$ 的内部时,
	\[\oint_C\frac{\diff z}{(z-a)^{n+1}}=\begin{cases}2\pi i,&n=0;\\0,&n\neq 0.\end{cases}\]
\end{theorem}

\begin{example}
	求 $\displaystyle\int_\Gamma\frac{2z-1}{z^2-z}\diff z$, 其中 $\Gamma$ 是由 $2\pm i,-2\pm i$ 形成的矩形闭路.
\end{example}
\begin{center}
	\begin{tikzpicture}
		\draw[cstaxis] (-2.5,0)--(2.5,0);
		\draw[cstaxis] (0,-1.5)--(0,1.5);
		\draw[cstcurve,cstwra,main] (-1,-0.7)--(-0.5,-0.7);
		\draw[cstcurve,cstwra,second] (-0.4,0) arc(180:225:0.4);
		\draw[cstcurve,cstwra,second] (0.6,0) arc(180:225:0.4);
		\draw[cstcurve,second] (0,0) circle (0.4);
		\draw[cstcurve,second] (1,0) circle (0.4);
		\fill[cstdot,third] (1,0) circle;
		\fill[cstdot,third] (0,0) circle;
		\draw[cstcurve,main] (-2,-0.7) rectangle (2,0.7);
		\draw
			(-0.6,0.4) node[second] {$C_1$}
			(1.6,0.4) node[second] {$C_2$}
			(-1,-0.5) node[main] {$\Gamma$};
	\end{tikzpicture}
\end{center}

\begin{solution}
	函数 $\dfrac{2z-1}{z^2-z}$ 在 $\Gamma$ 内有两个奇点 $z=0,1$.
	设 $C_1,C_2$ 如图所示,由复合闭路定理
	\begin{align*}
			&\oint_\Gamma\frac{2z-1}{z^2-z}\diff z
		=\oint_{C_1}\frac{2z-1}{z^2-z}\diff z+\oint_{C_2}\frac{2z-1}{z^2-z}\diff z\\
		=&\oint_{C_1}\frac1z\diff z+\oint_{C_1}\frac1{z-1}\diff z
		+\oint_{C_2}\frac1z\diff z+\oint_{C_2}\frac1{z-1}\diff z\\
		=&2\pi i+0+0+2\pi i=4\pi i.
	\end{align*}
\end{solution}

\begin{example}
	求 $\displaystyle\int_\Gamma\frac{e^z}z\diff z$, 其中 $\Gamma=C_1+C_2^-$, $C_1:|z|=2, C_2:|z|=1$.
\end{example}

\begin{center}
	\begin{tikzpicture}
		\filldraw[cstcurve,main,cstfill] (0,0) circle (1.5);
		\draw[cstcurve,main,cstwla] (-1.06,1.06) arc(135:90:1.5);
		\filldraw[cstcurve,second,fill=white] (0,0) circle (0.75);
		\draw[cstcurve,second,cstwra] (-0.75,0) arc(180:130:0.75);
		\draw
			(-1.6,0.8) node[main] {$C_1$}
			(0.8,-0.7) node[second] {$C_2^-$};
		\draw[cstaxis] (-1.8,0)--(1.8,0);
		\draw[cstaxis] (0,-1.8)--(0,1.8);
	\end{tikzpicture}
\end{center}

\begin{solution}
	函数 $\dfrac{e^z}z$ 在 $C_1,C_2$ 围城的圆环域内解析.
	由复合闭路定理可知 $\displaystyle\int_\Gamma\frac{e^z}z\diff z=0$.
\end{solution}



\section{原函数和不定积分}

\subsection{原函数}

设 $f(z)$ 在单连通区域 $D$ 内解析, $C$ 是 $D$ 内一条起于 $z_0$ 终于 $z$ 的曲线.
由柯西-古萨基本定理可知, 积分 $\displaystyle\int_Cf(\zeta)\diff \zeta$ 与路径无关, 只与 $z_0,z$ 有关.
因此我们也将其记为 $\displaystyle\int_{z_0}^zf(\zeta)\diff\zeta$.
对于任意固定的 $z_0\in D$, 函数
\[F(z)=\int_{z_0}^zf(\zeta)\diff\zeta\]
定义了一个单值函数.

\begin{theorem}[原函数的存在性]
	$F(z)$ 是 $D$ 内的解析函数, 且 $F'(z)=f(z)$.
\end{theorem}

\begin{proof}
	以 $z$ 为中心作一包含在 $D$ 内的圆 $K$,
	取 $|\Delta z|$ 小于 $K$ 的半径.那么
	\[F(z+\Delta z)-F(z)=\int_{z_0}^{z+\Delta z}f(\zeta)\diff\zeta-\int_{z_0}^zf(\zeta)\diff\zeta
	=\int_z^{z+\Delta z}f(\zeta)\diff\zeta.\]

	容易知道
	\[\int_z^{z+\Delta z}f(z)\diff\zeta=f(z)\int_z^{z+\Delta z}\diff\zeta=f(z)\Delta z.\]我们需要比较上述两个积分, 其中 $z$ 到 $z+\Delta z$ 取直线.
	由于 $f(z)$ 解析, 因此连续.
	$\forall\varepsilon>0,\exists\delta>0$ 使得当 $|\zeta-z|<\delta$ 时, $z$ 落在 $K$ 中且 $|f(\zeta)-f(z)|<\varepsilon$.当 $|\Delta z|<\delta$ 时, 由长大不等式
		\begin{align*}
		\abs{\frac{F(z+\Delta z)-F(z)}{\Delta z}-f(z)}
		&{=\abs{\int_z^{z+\Delta z}\frac{f(\zeta)-f(z)}{\Delta z}\diff \zeta}}\\
		&{\le\frac{\varepsilon}{|\Delta z|}\cdot|\Delta z|=\varepsilon.}
		\end{align*}
	由于 $\varepsilon$ 是任意的, 因此
	\[f(z)=\lim_{\Delta z\ra 0}\frac{F(z+\Delta z)-F(z)}{\Delta z}=F'(z).\qedhere\]
\end{proof}

\begin{center}
	\begin{tikzpicture}
		\fill[cstcurve,main,rounded corners=0.5cm,cstfill] (-2.5,-0.8) rectangle (2,1);
		\draw[cstcurve,main] (0,0) circle(0.7);
		\draw[cstcurve,second] (-2,0)to [bend left](0,0);
		\draw[cstcurve,second] (0,0)--(0.4,0.4);
		\fill[cstdot,second] (-2,0) circle;
		\fill[cstdot,second] (0,0) circle;
		\fill[cstdot,second] (0.4,0.4) circle;
		\draw
			(-2,-0.3) node[second] {$z_0$}
			(0,-0.3) node[second] {$z$}
			(1.1,0.7) node[second] {$z+\Delta z$};
	\end{tikzpicture}
\end{center}


\subsection{牛顿-莱布尼兹定理}

\begin{theorem}[牛顿-莱布尼兹定理]
	设 $f(z)$ 在单连通区域 $D$ 上解析, $z_1$ 至 $z_2$ 的积分路径落在 $D$ 内, 则
	\begin{equation}
		\int_{z_1}^{z_2}f(z)\diff z=F(z_1)-F(z_2),\quad\text{其中}\quad F'(z)=f(z).
	\end{equation}
\end{theorem}

如果 $D$ 上的解析函数 $G(z)$ 满足 $G'(z)=f(z)$, 则称 $G(z)$ 是 $f(z)$ 的一个\emph{原函数}.
由于导函数为 $0$ 的解析函数只能是常值函数,
因此 $\displaystyle G(z)=\int_{z_0}^zf(z)\diff z+C$.
我们称之为 $f(z)$ 的\emph{不定积分}, 记为 \emph{$\displaystyle\int f(z)\diff z$}.

复变函数和实变函数的牛顿-莱布尼兹定理的差异在哪呢?
复变情形要求是\alert{单连通区域上解析函数}, 实变情形要求是\alert{闭区间上连续函数}.

\begin{example}
	求 $\displaystyle\int_{z_0}^{z_1}z\diff z$.
\end{example}

\begin{solution}
	由于 $f(z)=z$ 处处解析, 且 $\displaystyle\int z\diff z=\half  z^2+C$, 因此
	\[\int_{z_0}^{z_1}z\diff z=\half z^2\big|_{z_0}^{z_1}=\half (z_1^2-z_0^2).\]
\end{solution}
因此之前的例子中 $\displaystyle\int_0^{3+4i}z\diff z=-\frac72+12i$, 无论从 $0$ 到 $3+4i$ 的路径如何.

\begin{example}
	求 $\displaystyle\int_0^{\pi i}z\cos z^2\diff z$.
\end{example}

\begin{solution}
	由于 $f(z)=z\cos z^2$ 处处解析, 且
	\[\int z\cos z^2\diff z=\half\int \cos z^2\diff z^2=\half\sin z^2+C,\]
	因此
	\[\int_0^{\pi i}z\cos z^2\diff z=\half\sin z^2\big|_0^{\pi i}=-\half\sin \pi^2.\]
\end{solution}

这里我们使用了\alert{凑微分法}.

\begin{example}
	求 $\displaystyle\int_0^i z\cos z\diff z$.
\end{example}

\begin{solution}
	由于 $f(z)=z\cos z$ 处处解析,
	{且
		\[\int z\cos z\diff z
		=\int z\diff(\sin z)
		=z\sin z-\int \sin z\diff z
		{=z\sin z+\cos z+C,}\]因此
		\[\int_0^i z\cos z\diff z
		=(z\sin z+\cos z)\big|_0^i
		{=i\sin i+\cos i-1=e^{-1}-1.}
		\]}
\end{solution}

这里我们使用了\alert{分部积分法}.

\begin{example}
	求 $\displaystyle\int_1^{1+i} z e^z\diff z$.
\end{example}

\begin{solution}
	由于 $f(z)=ze^z$ 处处解析,
	{且
	\[\int z e^z\diff z=\int z\diff e^z=ze^z-\int e^z\diff z=(z-1)e^z+c,\]因此

		\begin{align*}
		\int_1^{1+i} z e^z\diff z&=(z-1)e^z\big|_1^{1+i}\\
		&{=ie^{1+i}=e(-\sin 1+i\cos 1).}
		\end{align*}}
\end{solution}

\begin{exercise}
	求 $\displaystyle\int_0^1 z\sin z\diff z=$\fillblank[4cm]{{$\sin 1-\cos 1$}}.
\end{exercise}
\begin{example}
	求 $\displaystyle\int_C(2z^2+8z+1)\diff z$, 其中 $C$ 是摆线
	$\displaystyle\begin{cases}
	x=a(\theta-\sin\theta),& \\ y=a(1-\cos\theta),
	\end{cases} 0\le \theta\le 2\pi.$

	\begin{center}
		\begin{tikzpicture}
			\draw[cstaxis](0,0)--(5,0);
			\draw[cstaxis](0,0)--(0,1.8);
			\draw[cstcurve,main,smooth,domain=0:360] plot ({0.7*(pi/180*\x-sin(\x))}, {0.7*(1-cos(\x))});
			\draw[cstcurve,second] (0,0.7) circle (0.7);
			\fill[cstdot,main] (0,0) circle;
			\draw[cstcurve,second] ({0.7*(pi/180*60)},0.7) circle (0.7);
			\fill[cstdot,main] ({0.7*(pi/180*60-sin(60))},{0.7*(1-cos(60))}) circle;
			\draw[cstcurve,second] ({0.7*(pi/180*120)},0.7) circle (0.7);
			\fill[cstdot,main] ({0.7*(pi/180*120-sin(120))},{0.7*(1-cos(120))}) circle;
			\draw[cstcurve,second] ({0.7*(pi/180*180)},0.7) circle (0.7);
			\fill[cstdot,main] ({0.7*(pi/180*180-sin(180))},{0.7*(1-cos(180))}) circle;
			\draw[cstcurve,second] ({0.7*(pi/180*240)},0.7) circle (0.7);
			\fill[cstdot,main] ({0.7*(pi/180*240-sin(240))},{0.7*(1-cos(240))}) circle;
			\draw[cstcurve,second] ({0.7*(pi/180*300)},0.7) circle (0.7);
			\fill[cstdot,main] ({0.7*(pi/180*300-sin(300))},{0.7*(1-cos(300))}) circle;
			\draw[cstcurve,second] ({0.7*(pi/180*360)},0.7) circle (0.7);
			\fill[cstdot,main] ({0.7*(pi/180*360-sin(360))},{0.7*(1-cos(360))}) circle;
		\end{tikzpicture}
	\end{center}
\end{example}

\begin{solution}
	由于 $f(z)=2z^2+8z+1$ 处处解析, 因此
		\begin{align*}
		&\int_C(2z^2+8z+1)\diff z=\int_0^{2\pi a}(2z^2+8z+1)\diff z\\
		=&\left(\frac23z^3+4z^2+z\right)\bigg|_0^{2\pi a}=\frac{16}3\pi^3a^3+16\pi^2a^2+2\pi a.
		\end{align*}
\end{solution}

\begin{example}
	设 $C$ 为沿着 $|z|=1$ 从 $1$ 到 $i$ 的逆时针圆弧, 求 $\displaystyle\int_C\frac{\ln(z+1)}{z+1}\diff z$.
\end{example}

\begin{solution}
	函数 $f(z)=\dfrac{\ln(z+1)}{z+1}$ 在 $\Re z\le -1$ 外的单连通区域解析.
	\[\int\frac{\ln(z+1)}{z+1}\diff z
	=\int\ln(z+1)\diff[\ln(z+1)]=\half\ln^2(z+1)+c.\]
	因此
	\begin{align*}
	&\int_C\frac{\ln(z+1)}{z+1}\diff z=\half\ln^2(z+1)\big|_1^i
		=\half\left[\ln^2(1+i)-\ln^22\right]\\
	=&\half\left[\left(\ln\sqrt2+\frac\pi4i\right)^2-\ln^22\right]
		=-\frac{\pi^2}{32}-\frac38\ln^22+\frac{\pi\ln2}{8}i.
	\end{align*}
\end{solution}

\section{柯西积分公式}

\subsection{柯西积分公式}

柯西-古萨基本定理是解析函数理论的基础, 但在很多情形下它由柯西积分公式表现.

\begin{theorem}{柯西积分公式}
	设
	\begin{itemize}
		\item 函数 $f(z)$ 在闭路或复合闭路 $C$ 及其内部 $D$ 解析,
		\item $z_0\in D$,
	\end{itemize}
	则
		\[f(z_0)=\frac1{2\pi i}\oint_C\frac{f(z)}{z-z_0}\diff z.\]
\end{theorem}

如果 $z_0\notin D$, 由柯西-古萨基本定理, 右侧的积分是 $0$.

解析函数可以用一个积分
\[f(z)=\frac1{2\pi i}\oint_C\frac{f(\zeta)}{\zeta-z}\diff\zeta,\quad z\in D\]
来表示, 这是研究解析函数理论的强有力工具.

解析函数在闭路 $C$ 内部的取值完全由它在 $C$ 上的值所确定. 这也是解析函数的特征之一.
特别地, 解析函数在圆心处的值等于它在圆周上的平均值.
设 $z=z_0+Re^{i\theta}$, 则 $\diff z=iRe^{i\theta}\diff\theta$,
\[f(z_0)=\frac1{2\pi i}\oint_C\frac{f(z)}{z-z_0}\diff z=\frac1{2\pi}\int_0^{2\pi}f(z_0+Re^{i\theta})\diff\theta.\]

\begin{proof}
	由连续性可知, $\forall\varepsilon>0,\exists\delta>0$ 使得当 $|z-z_0|\le\delta$ 时, $z\in D$ 且 $|f(z)-f(z_0)|<\varepsilon$.
	设 $\Gamma:|z-z_0|=\delta$,则
	\begin{align*}
		&\abs{\oint_C\frac{f(z)}{z-z_0}\diff z-2\pi i f(z_0)}
		\xeq{\text{复合闭路定理}}\abs{\oint_\Gamma\frac{f(z)}{z-z_0}\diff z-2\pi i f(z_0)}\\
		=&\abs{\oint_\Gamma\frac{f(z)}{z-z_0}\diff z-\oint_\Gamma\frac{f(z_0)}{z-z_0}\diff z}
		=\abs{\oint_\Gamma\frac{f(z)-f(z_0)}{z-z_0}\diff z}\\
		\le&\frac\varepsilon \delta\cdot 2\pi \delta=2\pi \varepsilon.
	\end{align*}
	由 $\varepsilon$ 的任意性可知 
	$\displaystyle\oint_C\frac{f(z)}{z-z_0}\diff z=2\pi i f(z_0)$.
\end{proof}

从柯西积分公式可以看出, 被积函数分子解析而分母形如 $z-z_0$ 时, 绕闭路的积分可以使用该公式计算.

\begin{example}
	求 $\displaystyle\oint_{|z|=4}\frac{\sin z}z\diff z$.
\end{example}

\begin{solution}
	函数 $\sin z$ 处处解析.
	{取 $f(z)=\sin z, z_0=0$ 并应用柯西积分公式得
		\[\oint_{|z|=4}\frac{\sin z}z\diff z
		=2\pi i \sin z|_{z=0}=0.\]}
\end{solution}

\begin{example}
	求 $\displaystyle\oint_{|z|=2}\frac{e^z}{z-1}\diff z$.
\end{example}

\begin{solution}
	由于函数 $e^z$ 处处解析,
	{取 $f(z)=e^z, z_0=1$ 并应用柯西积分公式得
		\[\oint_{|z|=2}\frac{e^z}{z-1}\diff z
		=2\pi i e^z|_{z=1}=2\pi ei.\]}
\end{solution}

\begin{exercise}
	求 $\displaystyle\oint_{|z|=2\pi}\frac{\cos z}{z-\pi}\diff z=$\fillblank{{$-2\pi i$}}.
\end{exercise}

\begin{example}
	设 $f(z)=\displaystyle\oint_{|\zeta|=\sqrt3}\frac{3\zeta^2+7\zeta+1}{\zeta-z}\diff \zeta$, 求 $f'(1+i)$.
\end{example}

\begin{solution}
	当 $|z|<\sqrt3$ 时,由柯西积分公式得
	\[
		f(z)=\oint_{|\zeta|=\sqrt3}\frac{3\zeta^2+7\zeta+1}{\zeta-z}\diff \zeta
		{=2\pi i(3\zeta^2+7\zeta+1)|_{\zeta=z}=2\pi i(3z^2+7z+1).}
	\]
	{因此 $f'(z)=2\pi i(6z+7)$,
		\[f'(1+i)=2\pi i(13+6i)=-12\pi+26\pi i.\]}
\end{solution}
注意当 $|z|>\sqrt3$ 时, $f(z)\equiv0$.

\begin{example}
	求 $\displaystyle\oint_{|z|=3}\frac{e^z}{z(z^2-1)}\diff z$.
\end{example}

\begin{solution}
	被积函数的奇点为 $0,\pm1$.
	设 $C_1,C_2,C_3$ 分别为绕 $0,1,-1$ 的分离圆周.由复合闭路定理和柯西积分公式
	\begin{align*}
		&\oint_{|z|=3}\frac{e^z}{z(z^2-1)}\diff z
		 =\oint_{C_1+C_2+C_3}\frac{e^z}{z(z^2-1)}\diff z\\
		=&2\pi i\left[\frac{e^z}{z^2-1}\bigg|_{z=0}+\frac{e^z}{z(z+1)}\bigg|_{z=1}+\frac{e^z}{z(z-1)}\bigg|_{z=-1}\right]\\
		=&2\pi i\left(-1+\frac e2+\frac{e^{-1}}2\right)=\pi i(e+e^{-1}-2).
	\end{align*}
	\begin{center}
		\begin{tikzpicture}
			\draw[cstaxis] (-2.1,0)--(2.1,0);
			\draw[cstaxis] (0,-2)--(0,2.1);
			\draw[cstcurve,second] (0,0) circle (1.8);
			\draw[cstcurve,main] (1.2,0) circle(0.5);
			\draw[cstcurve,main] (-1.2,0) circle(0.5);
			\draw[cstcurve,main] (0,0) circle(0.5);
			\fill[cstdot,second] (0,0) circle;
			\fill[cstdot,second] (1.2,0) circle;
			\fill[cstdot,second] (-1.2,0) circle;
			\draw
				(1.6,1.6) node[second] {$C$}
				(-0.4,0.8) node[main] {$C_1$}
				(1.2,-.8) node[main] {$C_2$}
				(-1.2,-.8) node[main] {$C_3$};
		\end{tikzpicture}
	\end{center}
\end{solution}

\subsection{高阶导数的柯西积分公式}

解析函数可以由它的积分所表示.
不仅如此, 通过积分表示, 还可以说明\alert{解析函数是任意阶可导的}.

\begin{theorem}{柯西积分公式}
	设函数 $f(z)$ 在闭路或复合闭路 $C$ 及其内部 $D$ 解析, 则对任意 $z_0\in D$,
	\[f^{(n)}(z_0)=\frac{n!}{2\pi i}\oint_C\frac{f(z)}{(z-z_0)^{n+1}}\diff z.\]
\end{theorem}

假如 $f(z)$ 有泰勒展开
\[f(z)=f(z_0)+f'(z_0)(z-z_0)+\cdots+\frac{f^{(n)}(z_0)}{n!}(z-z_0)^n+\cdots\]
那么由 $\displaystyle\oint_C \frac{\diff z}{(z-z_0)^n}$ 的性质可知上述公式右侧应当为 $f^{(n)}(z_0)$.

\begin{proof}
	先证明 $n=1$ 的情形.
	设 $\delta$ 为 $z_0$ 到 $C$ 的最短距离.当 $|h|<\delta$ 时, $z_0+h\in D$.由柯西积分公式,
		\[f(z_0)=\frac1{2\pi i}\oint_C\frac{f(z)}{z-z_0}\diff z,\ 
		f(z_0+h)=\frac1{2\pi i}\oint_C\frac{f(z)}{z-z_0-h}\diff z.\]
	两式相减得到
		\[\frac{f(z_0+h)-f(z_0)}h=\frac1{2\pi i}\oint_C\frac{f(z)}{(z-z_0)(z-z_0-h)}\diff z.\]
	当 $h\ra 0$ 时, 左边的极限是 $f'(z_0)$. 因此我们只需要证明右边的极限等于 $\displaystyle\frac1{2\pi i}\oint_C\frac{f(z)}{(z-z_0)^2}\diff z$.
	二者之差 $=\displaystyle\frac1{2\pi i}\oint_C\frac{h f(z)}{(z-z_0)^2(z-z_0-h)}\diff z$.
	由于 $f(z)$ 在 $C$ 上连续, 故存在 $M$ 使得 $|f(z)|\le M$. 注意到 $z\in C$, $|z-z_0|\ge \delta$, $|z-z_0-h|\ge\delta-|h|$. 由长大不等式,
		\[\abs{\oint_C\frac{h f(z)}{(z-z_0)^2(z-z_0-h)}\diff z}\le\frac{M|h|}{\delta^2(\delta-|h|)}\cdot L,\]
	其中 $L$ 是闭路 $C$ 的长度.当 $h\ra0$ 时, 它的极限为 $0$, 因此 $n=1$ 情形得证.

	对于一般的 $n$, 我们通过归纳法将 $f^{(n)}(z_0)$ 和 $f^{(n)}(z_0+h)$ 表达为积分形式. 比较 $\dfrac{f^{(n)}(z_0+h)-f^{(n)}(z_0)}h$ 与积分公式右侧之差, 并利用长大不等式证明 $h\ra 0$ 时, 差趋于零. 具体过程省略.
\end{proof}

\alert{柯西积分公式不是用来计算高阶导数的, 而是用高阶导数来计算积分的.}

\begin{example}
	求 $\displaystyle\oint_{|z|=2}\frac{\cos(\pi z)}{(z-1)^5}\diff z.$
\end{example}

\begin{solution}
	由于 $\cos(\pi z)$ 处处解析,
	{因此由柯西积分公式,
		\[
		\oint_{|z|=2}\frac{\cos(\pi z)}{(z-1)^5}\diff z
		=\frac{2\pi i}{4!}[\cos(\pi z)]^{(4)}\big|_{z=1}
		{=\frac{2\pi i}{24}\cdot \pi^4\cos \pi=-\frac{\pi^5 i}{12}.}
		\]}
\end{solution}

\begin{example}
	求 $\displaystyle\oint_{|z|=2}\frac{e^z}{(z^2+1)^2}\diff z.$
\end{example}

\begin{solution}
	$\dfrac{e^z}{(z^2+1)^2}$ 在 $|z|<2$ 的奇点为 $z=\pm i$.
	{取 $C_1,C_2$ 为以 $i,-i$ 为圆心的分离圆周.由复合闭路定理,
		\[\oint_{|z|=2}\frac{e^z}{(z^2+1)^2}\diff z
		=\oint_{C_1}\frac{e^z}{(z^2+1)^2}\diff z
		+\oint_{C_2}\frac{e^z}{(z^2+1)^2}\diff z.\]}

	由柯西积分公式,
	\begin{align*}
		&\oint_{C_1}\frac{e^z}{(z^2+1)^2}\diff z
		=\frac{2\pi i}{1}\left[\frac{e^z}{(z+i)^2}\right]'\Big|_{z=i}\\
		&=2\pi i\left[\frac{e^z}{(z+i)^2}-\frac{2e^z}{(z+i)^3}\right]\Big|_{z=i}
		=\frac{(1-i)e^i\pi}2.
	\end{align*}
	类似地, $\displaystyle\oint_{C_2}\frac{e^z}{(z^2+1)^2}\diff z=\frac{-(1+i)e^{-i}\pi}2$.
	故
		\[\oint_{|z|=2}\frac{e^z}{(z^2+1)^2}\diff z
		=\frac{(1-i)e^i\pi}2+\frac{-(1+i)e^{-i}\pi}2
		=\pi i(\sin1-\cos1).\]
\end{solution}

\begin{example}
	求 $\displaystyle\oint_{|z|=1}z^ne^z\diff z$, 其中 $n$ 是整数.
\end{example}

\begin{solution}
	当 $n\ge 0$ 时, $z^ne^z$ 处处解析.
	由柯西-古萨基本定理, 
	\[
		\oint_{|z|=1}z^ne^z\diff z=0.
	\]
	当 $n\le-1$ 时, $e^z$ 处处解析.由柯西积分公式,
	\[
		 \oint_{|z|=1}z^ne^z\diff z
		=\frac{2\pi i}{(-n-1)!}(e^z)^{(-n-1)}\big|_{z=0}
		=\frac{2\pi i}{(-n-1)!}.
	\]
\end{solution}

\begin{example}
	求 $\displaystyle\oint_{|z-3|=2}\frac1{(z-2)^2z^3}\diff z$ 和 $\displaystyle\oint_{|z-1|=2}\frac1{(z-2)^2z^3}\diff z$.
\end{example}

\begin{solution}
	\enumnum1 $\dfrac1{(z-2)^2z^3}$ 在 $|z-3|<2$ 的奇点为 $z=2$.
	{由柯西积分公式,
		\[\oint_{|z-3|=2}\frac1{(z-2)^2z^3}\diff z
		=\frac{2\pi i}{1!}\left(\frac1{z^3}\right)'\bigg|_{z=2}
		=-\frac{3\pi i}8.\]}

	\enumnum2 $\dfrac1{(z-2)^2z^3}$ 在 $|z-1|<3$ 的奇点为 $z=0,2$.
	取 $C_1,C_2$ 分别为以 $0$ 和 $2$ 为圆心的分离圆周.由复合闭路定理和柯西积分公式,
	\begin{align*}
		 &\oint_{|z-1|=3}\frac1{(z-2)^2z^3}\diff z=\oint_{C_1}\frac1{(z-2)^2z^3}\diff z+\oint_{C_2}\frac1{(z-2)^2z^3}\diff z\\
		=&\frac{2\pi i}{2!}\left[\frac1{(z-2)^2}\right]''\Big|_{z=0}+\frac{2\pi i}{1!}\left(\frac1{z^3}\right)'\Big|_{z=2}=0.
		\end{align*}
\end{solution}

\begin{exercise}
	$\displaystyle\oint_{|z-2i|=3}\frac1{z^2(z-i)}\diff z=$\fillblank{{$0$}}.
\end{exercise}

\begin{example}[莫累拉定理]
	设 $f(z)$ 在单连通域 $D$ 内连续, 且对于 $D$ 中任意闭路 $C$ 都有 $\displaystyle\oint_Cf(z)\diff z=0$, 则 $f(z)$ 在 $D$ 内解析.
\end{example}

该定理可视作柯西-古萨基本定理的逆定理.

\begin{proof}
	由题设可知 $f(z)$ 的积分与路径无关.
	固定 $z_0\in D$, 则
		\[F(z)=\int_{z_0}^zf(z)\diff z\]
	定义了 $D$ 内的一个函数.类似于原函数的证明可知 $F'(z)=f(z)$.故 $f(z)$ 作为解析函数 $F(z)$ 的导数也是解析的.
\end{proof}

高阶柯西积分公式说明解析函数的导数与实函数的导数有何不同?
高阶柯西积分公式说明, 函数 $f(z)$ 只要在区域 $D$ 中处处可导, 它就一定无限次可导, 并且各阶导数仍然在 $D$ 中解析.
\alert{这一点与实变量函数有本质的区别.}

同时我们也可以看出, 如果一个二元实函数 $u(x,y)$ 是一个解析函数的实部或虚部, 则 $u$ 也是具有任意阶偏导数.
这便引出了调和函数的概念.

\section{解析函数与调和函数的关系}

\subsection{调和函数}

调和函数是一类重要的二元实变函数, 它和解析函数有着紧密的联系.
为了简便, 我们用 $u_{xx},u_{yy}$ 来表示二阶偏导数.

\begin{definition}
	如果二元实变函数 $u(x,y)$ 在区域 $D$ 内有二阶连续偏导数, 且满足拉普拉斯方程
	\[\Delta u:=u_{xx}+u_{yy}=0,\]
	则称 $u(x,y)$ 是 $D$ 内的\emph{调和函数}.
\end{definition}

\begin{theorem}
	区域 $D$ 内解析函数 $f(z)$ 的实部和虚部都是调和函数.
\end{theorem}

\begin{proof}
	设 $f(z)=u(x,y)+iv(x,y)$, 则 $u,v$ 存在偏导数且
		\[f'(z)=u_x+iv_x=v_y-iu_y.\]
	{由于 $f(z)$ 任意阶可导, 因此 $u,v$ 存在任意阶偏导数.由C-R方程 $u_x=v_y,u_y=-v_x$可知
		\[\Delta u=u_{xx}+u_{yy}=v_{yx}-v_{xy}=0,\]}

	{
		\[\Delta v=v_{xx}+v_{yy}=-u_{yx}+u_{xy}=0.\qedhere\]}
\end{proof}

\subsection{共轭调和函数}

反过来, 调和函数是否一定是某个解析函数的实部或虚部呢?
对于单连通的情形, 答案是肯定的.

如果 $u+iv$ 是区域 $D$ 内的解析函数, 则我们称 $v$ 是 $u$ 的\emph{共轭调和函数}.
换言之 $u_x=v_y,u_y=-v_x$.
显然 $-u$ 是 $v$ 的共轭调和函数.
\begin{theorem}
	设 $u(x,y)$ 是单连通域 $D$ 内的调和函数, 则线积分
	\[v(x,y)=\int_{(x_0,y_0)}^{(x,y)}-u_y\diff x+u_x\diff y+C\]
	是 $u$ 的共轭调和函数.
\end{theorem}
由此可知, 区域 $D$ 上的调和函数在 $z\in D$ 的一个邻域内是一解析函数的实部, 从而在该邻域内具有任意阶连续偏导数.
而 $z$ 的任取的, 因此调和函数总具有任意阶连续偏导数.

如果 $D$ 是多连通区域, 则未必存在共轭调和函数.
例如 $\ln(x^2+y^2)$ 是复平面去掉原点上的调和函数, 但它并不是某个解析函数的实部.
事实上, 它是 $2\Ln z$ 的实部.

在实际计算中, 我们\alert{一般不用线积分}来得到共轭调和函数, 而是采用下述两种办法:
\begin{theorem}{偏积分法}
	通过 $v_y=u_x$ 解得 $v=\varphi(x,y)+\psi(x)$, 其中 $\psi(x)$ 待定.
	{再代入 $u_y=-v_x$ 中解出 $\psi(x)$.}
\end{theorem}
\begin{theorem}{不定积分法}
	对 $f'(z)=u_x-iu_y=v_y+iv_x$ 求不定积分得到 $f(z)$.
\end{theorem}

\begin{example}
	证明 $u(x,y)=y^3-3x^2y$ 是调和函数, 并求其共轭调和函数以及由它们构成的解析函数.
\end{example}

\begin{solution}
	由 $u_x=-6xy,u_y=3y^2-3x^2$ 可知 $u_{xx}+u_{yy}=-6y+6y=0$,
	{故 $u$ 是调和函数.}

	{由 $v_y=u_x=-6xy$ 得 $v=-3xy^2+\psi(x)$.}

	{由 $v_x=-u_y=3x^2-3y^2$ 得 $\psi'(x)=3x^2$,$\psi(x)=x^3+C$.}

	{故 $v(x,y)=-3xy^2+x^3+C$,
	\[
		f(z)=u+iv=y^3-3x^2y+i(-3xy^2+x^3+C)
		{=i(x+iy)^3+iC=i(z^3+C).}
	\]}
\end{solution}

当解析函数 $f(z)$ 为 $x,y$ 的多项式形式时, 将 $m$ 次齐次的项放一起, 则 $x^m$ 的系数就是 $f(z)$ 中 $z^m$ 的系数.

在上例中我们也可由另一种方法计算得到:
\[f'(z)=u_x-iu_y=-6xy-i(3y^2-3x^2)=3iz^2.\]
因此 $f(z)=iz^3+C$.

\begin{example}
	求解析函数 $f(z)$ 使得它的虚部为
	\[v(x,y)=e^x(y\cos y+x\sin y)+x+y.\]
\end{example}

\begin{solution}
	由 $u_x=v_y=e^x(\cos y-y\sin y+x\cos y)+1$ 得
	\[u=e^x(x\cos y-y\sin y)+x+\psi(y).\]
	{由 $u_y=-v_x=-e^x(y\cos y+x\sin y+\sin y)-1$ 得
	\[\psi'(y)=-1,\quad\psi(y)=-y+C.\]}
	故

	\begin{align*}
		f(z)&=u+iv\\
		&=e^x(x\cos y-y\sin y)+x-y+C
		+i\bigl[e^x(y\cos y+x\sin y)+x+y\bigr]\\
		&{=ze^z+(1+i)z+C,\quad C\in\BR.}
	\end{align*}
\end{solution}
这里, 我们只需看 $e^x\cos y$ 的系数 $x+iy=z$, 即 $f(z)$ 中 $e^z$ 的系数.

也可由
\begin{align*}
	f'(z)&=v_y+iv_x\\
	&=e^x(\cos y-y\sin y+x\cos y)+1
	+i\bigl[e^x(y\cos y+x\sin y+\sin y)+1\bigr]\\
	&{=(z+1)e^z+1+i.}
\end{align*}
得 $f(z)=ze^z+(1+i)z+C$.
\begin{exercise}
	证明 $u(x,y)=x^3-6x^2y-3xy^2+2y^3$ 是调和函数并求它的共轭调和函数.
\end{exercise}


\sectionHomework
\begin{homework}
	\item 选择题. \begin{exlist}
		\item 设 $C$ 为正向圆周 $|\zeta|=2$, $\displaystyle f(z)=\oint_C\frac{\sin \zeta}{\zeta-z}\diff \zeta$, 则 $f\left(\dfrac\pi6\right)=$ \fillbrace{}.
			\begin{taskschoice}(4)
				() $\pi i$
				() $-\pi i$
				() $0$
				() $2\pi i$
			\end{taskschoice}
		\item 下列命题中, 正确的是\fillbrace{}.
			\begin{taskschoice}(1)
				() 设 $v_1,v_2$ 在区域 $D$ 内均为 $u$ 的共轭调和函数, 则必有 $v_1=v_2$
				() 解析函数的实部是虚部的共轭调和函数
				() 以调和函数为实部与虚部的函数是解析函数
				() 若 $f(z)=u+iv$ 在区域 $D$ 内解析,则 $u_x$ 为 $D$ 内的调和函数
		\end{taskschoice}
	\end{exlist}
	\item 填空题. \begin{exlist}
		\item 设 $f(z)$ 在单连通区域 $D$ 内处处解析且不为零, 则 $\displaystyle\oint_C\frac{f''(z)+2f'(z)+f(z)}{f(z)}\diff z=$\fillblank{}, 其中 $C$ 为 $D$ 内一条闭路.
		\item 设 $C$ 为正向圆周 $|z|=1$, 则 $\displaystyle\oint_C\ov z\diff z=$\fillblank{}.
		\item 设 $C$ 为正向圆周 $|z|=2$, 则 $\displaystyle\oint_C\left(\frac{\ov z}{z}\right)\diff z=$\fillblank{}.
		\item 设 $C$ 为正向圆周 $|z|=2$, 则 $\displaystyle\oint_C\dfrac{\ov z}{|z|}\diff z=$\fillblank{}.
	\end{exlist}
	\item 计算题.
	\begin{exlist}
		\item 利用积分曲线的参数方程求 $\displaystyle\int_C z^2\diff z$, 其中 $C$ 为:
			\begin{tasks}
				\task 从 $0$ 到 $3+i$ 的直线段;
				\task 从 $0$ 到 $3$ 再到 $3+i$ 的折线段.
			\end{tasks}
		\item 试用观察法得出下列积分的值, 并说明为什么, 其中 $C:|z|=1$.
			\begin{tasks}(3)
				\task $\displaystyle\oint_C\frac{\diff z}{z-2}$;
				\task $\displaystyle\oint_C\frac{\diff z}{\cos z}$;
				\task $\displaystyle\oint_C\frac{e^z}{(z-2i)^2}\diff z$;
				\task $\displaystyle\oint_Ce^z\sin z\diff z$;
				\task $\displaystyle\oint_C\dfrac{\ 1\ }{\ov z}\diff z$;
				\task $\displaystyle\oint_C(|z|+e^z\cos z)\diff z$.
			\end{tasks}
		\item 设 $C$ 为正向圆周 $|z|=4$, 求 $\displaystyle\oint_C\dfrac{\sin z}{|z|^2}\diff z$.
		\item 设 $C$ 为从原点到 $1+i$ 的直线段, 求 $\displaystyle\int_C(z+1)^2\diff z$.
		\item 设 $C$ 为从 $i$ 到 $i-\pi$ 再到 $-\pi$ 的折线, 求 $\displaystyle\int_C\cos^2z\diff z$.
		\item 设 $C$ 为从原点到 $2$ 再到 $2+i$ 的折线段, 求 $\displaystyle\int_Cz^2\diff z$.
		\item 求 $\displaystyle\int_{-\pi i}^{3\pi i}e^{2z}\diff z$.
		\item 求 $\displaystyle\int_{-\pi i}^{\pi i}\sin^2z\diff z$.
		\item 求 $\displaystyle\int_0^i(z-i)e^{-z}\diff z$.
		\item 设 $C$ 为正向圆周 $|z-2|=1$, 求 
			$\displaystyle\oint_C\frac{e^z}{z-2}\diff z$.
		\item 设 $C$ 为正向圆周 $|z|=r<1$, 求 
			$\displaystyle\oint_C\frac{\diff z}{(z^2-1)(z^3-1)}$.
		\item 设 $C$ 为以 $\pm\dfrac12,\pm\dfrac65i$ 为顶点的菱形, 求 
			$\displaystyle\oint_C\frac{1}{z-i}\diff z$.
		\item 设 $C$ 为正向圆周 $|z|=2$, 求 
			$\displaystyle\oint_C\frac1{(z^2+1)(z^2+9)}\diff z$.
		\item 设 $C$ 为正向圆周 $|z-3|=4$, 求 
			$\displaystyle\oint_C\frac{e^{iz}}{z^2-3\pi z+2\pi^2}\diff z$.
		\item 设 $C_1$ 为正向圆周 $|z|=2$, $C_2^-$ 为负向圆周 $|z|=3$, $C=C_1+C_2^-$ 为复合闭路, 求 
			$\displaystyle\oint_C\frac{\cos z}{z^3}\diff z$.
		\item 设 $C$ 为正向圆周 $|z|=2$, 求 
			$\displaystyle\oint_C\frac{\sin z}{\left(z-\dfrac\pi 2\right)^2}\diff z$.
		\item 设 $C$ 为正向圆周 $|z|=1$, 求 
			$\displaystyle\oint_C\frac{\cos z}{z^{2023}}\diff z$.
		\item 设 $C$ 为正向圆周 $|z|=1.5$, 求 
			$\displaystyle\oint_C\frac{\ln(z+2)}{(z-1)^3}\diff z$.
		\item 设 $C$ 为正向圆周 $|\zeta|=2$, $\displaystyle f(z)=\oint_C\dfrac{\zeta^3+\zeta+1}{(z-\zeta)^2}\diff\zeta$.
			求 $f'(1+i)$ 和 $f'(4)$.
		\item 已知 $v(x,y)=x^3+y^3-axy(x+y)$ 为调和函数, 求参数 $a$ 以及解析函数 $f(z)$ 使得 $v(x,y)$ 是它的虚部.
		\item 已知 $f(z)=x^2+2xy-y^2+i(y^2+axy-x^2)$ 为解析函数, 求参数 $a$ 和 $f'(z)$.
		\item 已知 $f(z)=y^3+ax^2y+i(bx^3-3xy^2)$ 为解析函数, $a,b$ 为实数, 求参数 $a,b$ 和 $f'(z)$.	
		\item 设 $u$ 为区域 $D$ 内的调和函数, $f(z)=u_x-iu_y$.
			那么 $f(z)$ 是不是 $D$ 内的解析函数? 为什么?
		\item 请谈一谈复积分与实积分的区别.
	\end{exlist}
	\item 证明题.
	\begin{exlist}
		\item 设 $C_1$ 和 $C_2$ 为两条分离的闭路, 证明
			\[\frac1{2\pi i}\left[\oint_{C_1}\frac{z^2\diff z}{z-z_0}+\oint_{C_2}\frac{\sin z\diff z}{z-z_0}\right]=
			\begin{cases}
				z_0^2,&\text{当 $z_0$ 在 $C_1$ 内时,}\\
				\sin z_0,&\text{当 $z_0$ 在 $C_2$ 内时.}
			\end{cases}\]
		\item 设 $f(z)$ 和 $g(z)$ 在区域 $D$ 内处处解析, $C$ 为 $D$ 内任意一条闭路, 且 $C$ 的内部完全包含在 $D$ 中.
			如果 $f(z)=g(z)$ 在 $C$ 上所有的点处成立, 证明在 $C$ 内部所有点处 $f(z)=g(z)$ 也成立.
		\item 证明: 一对共轭调和函数的乘积仍为调和函数.
	\end{exlist}
\end{homework}


\sectionExtraReading
\begin{homework}
	\item 扩展阅读. 该部分作业不需要交, 有兴趣的同学可以做完后交到任课教师邮箱.
		\begin{exlist}
			\item 设 $f(z)=u+iv$.
			当 $u,v$ 是二元可微函数时, 我们也可以使用格林公式来计算 $f(z)$ 绕闭路的积分.
				\begin{tasks}
					\task 设 $C$ 是一条光滑或逐段光滑的闭路, $D$ 是其内部区域. 函数 $u(x,y),v(x,y)$ 在 $D$ 及其边界上连续可微. 证明
					\[\oint_C f(z)\diff z=-\iint_D(v_x+u_y)\diff x\diff y
					+i\iint_D(u_x-v_y)\diff x\diff y,\]
					并由此计算 $\displaystyle\oint_{|z|=1}\Re z\diff z$.
					\task 证明
					\[\oint_C f(z)\diff z=-\iint_D\frac{\partial f}{\partial \ov z}\diff z\diff \ov z=2i\iint_D\frac{\partial f}{\partial \ov z}\diff x\diff y,\]
					并由此计算 $\displaystyle\oint_{|z|=1}\Re z\diff z$.
				\end{tasks}
			\item 设 $f(z)$ 在闭路 $C$ 及其外部区域 $D$ 解析, $z_0\in D$. 是否有类似的柯西积分公式?
			我们假设 $\lim\limits_{z\to\infty}f(z)=A$ 存在.
			\begin{center}
			\begin{tikzpicture}
				\fill[cstfille5] (-4,4) rectangle (4,-4);
				\fill[cstdote,main] (0,0) circle;
				\draw[cstcurve,second] (0,0) circle (0.8);
				\draw[cstcurve,fourth] (0,0) circle (3.3);
				\filldraw[cstcurve,main,rounded corners=0.5cm,fill=white] (-3,0.8) rectangle (-1,-0.8);
				\draw
					(0,0.3) node[main] {$z_0$}
					(-1.5,-0.4) node[main] {$C$}
					(1.1,-0.3) node[second] {$C_1$}
					(2.5,1) node[fourth] {$C_2$};
			\end{tikzpicture}
			\end{center}
			\begin{tasks}
				\task 选取以 $z_0$ 为圆心的圆 $C_1,C_2$ 如图所示.
				利用长大不等式证明 $\displaystyle \frac1{2\pi i}\oint_{C_2}\frac{f(z)}{z-z_0}\diff z=A$.
				\task 利用复合闭路定理证明 $\displaystyle \frac1{2\pi i}\oint_C\frac{f(z)}{z-z_0}\diff z=A-f(z_0)$.
			\end{tasks}
		\end{exlist}
\end{homework}


\sectionExerciseAnswer
\exans $-\dfrac12+\dfrac i2$.
\exans \enumnum1 $0$; \enumnum2 $0$.
\exans $v(x,y)=2x^3+3x^2y-6xy^2-y^3+C$.
