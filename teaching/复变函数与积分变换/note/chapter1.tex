
\chapter{复数与复变函数}
\label{chapter:1}

\begin{introduction}
  % 复数的产生源于对数学中代数方程求解的需求.
  % 复数的概念最初由意大利数学家卡尔达诺在求解三次方程时提出.
  % 复数的引入, 使得数学体系得以扩展, 解决了许多实数范围内无法解决的问题.
  
  % 复数不仅在数学领域如几何、分析、表示论、动力系统等发挥着重要作用, 还在物理、工程、信息学、电子学等领域有着广泛应用.
  % 它不仅丰富了数学的内容,也为科学和技术的发展提供了有力的数学工具.

  本章中我们将学习复数、复变函数的基本概念, 以及复数列和复变函数的极限.
  我们将从一元三次方程求解问题出发, 逐步展示引入复数的必要性. 然后介绍复数的运算规则, 并展示复数的三角形式和指数形式在运算中所起的关键作用. 最后, 我们引入复变函数、复数列以及极限的概念, 并将之与实数情形进行对比.
\end{introduction}


\section{复数及其代数运算}

\subsection{复数的产生}

复数起源于多项式方程的求根问题. 
考虑一元二次方程 $x^2+bx+c=0$, 配方可得
  \[\left(x+\frac b2\right)^2=\frac{b^2-4c}4.\]
于是得到求根公式
  \[x=\frac{-b\pm\sqrt\Delta}2,\quad \Delta=b^2-4c.\]
\begin{enumpar}
  \item 当 $\Delta>0$ 时, 有两个不同的实根;
  \item 当 $\Delta=0$ 时, 有一个二重%
    \footnote{%
      如果 $x_0$ 是多项式方程 $f(x)=0$ 的根, 则 $x-x_0$ 是 $f(x)$ 的因式, 即存在多项式 $g(x)$ 使得 $f(x)=(x-x_0)g(x)$.
      如果 $(x-x_0)^k$ 是 $f(x)$ 的因式, 但 $(x-x_0)^{k+1}$ 不是, 则称 $x_0$ 是 \emph{$k$ 重根}.
      在第五章中我们将对一般的函数定义零点的重数.
    }%
    的实根;
  \item 当 $\Delta<0$ 时, 无实根.
\end{enumpar}

可以看出, 在一元二次方程中, 我们可以舍去包含\emph{负数开平方}的解. 然而在一元三次方程中, 即便只考虑实数根也会不可避免地引入负数开平方.

\begin{example}
  解方程 $x^3+6x-20=0$.
\end{example}

我们将使用由费罗%
  \footnote{%
    Scipione del Ferro (1465--1526), 意大利数学家.
    费罗发现了该方法后, 并没有发表他的结果, 因为当时人们常把他们的发现保密, 而向对手们提出挑战. 参考 \cite[第13章4节]{Kline1990}.
  }%
最先发现, 并由卡尔达诺%
  \footnote{%
    Girolamo Cardano (1501--1576), 意大利数学家、物理学家、天文学家、哲学家.%, 十六世纪人文主义代表人物, 古典概率论的创始人.
  }%
最先公开的解法.

\begin{solution}
  设 $x=u+v$, 则
    \[u^3+v^3+3uv(u+v)+6(u+v)-20=0.\]
  我们希望
    \[u^3+v^3=20,\qquad uv=-2,\]
  则 $u^3,v^3$ 满足一元二次方程 $X^2-20X-8=0$.
  解得
    \[u^3=10\pm\sqrt{108}{=(1\pm\sqrt3)^3.}\]
  所以 $u=1\pm\sqrt3, v=1\mp\sqrt 3$, $x=u+v=2$.
\end{solution}

那么这个方程是不是真的只有 $x=2$ 这一个实数解呢?
由其导数 $3x^2+6>0$ 可知其单调递增, 因此确实只有一个实数解.

% \begin{figure}[!h]
%   \centering
%   \begin{minipage}{.48\textwidth}
%     \centering
%     \begin{tikzpicture}
%       \filldraw[cstcurve,main,domain=-1.2:3.1,smooth,fill=white] plot ({\x*.5},{(\x*\x*\x+6*\x-20)*.1});
%       \draw[cstaxis] (-3,0)--(3,0);
%       \draw[cstaxis] (0,-3)--(0,3);
%       \coordinate [label=above left:$-20$] (A) at (0,-2);
%       \coordinate [label=above left:$2$] (B) at (1,0);
%       \fill[cstdot,second] (A) circle;
%       \fill[cstdot,second] (B) circle;
%     \end{tikzpicture}
%     \caption{$y=x^3+6x-20$\footnotemark}
%   \end{minipage}
%   \begin{minipage}{.48\textwidth}
%     \centering
%     \begin{tikzpicture}
%       \filldraw[cstcurve,main,domain=-3.6:3.2,smooth,fill=white] plot ({\x*.6},{(\x*\x*\x-7*\x+6)*.15});
%       \draw[cstaxis] (-3,0)--(3,0);
%       \draw[cstaxis] (0,-3)--(0,3);
%       \coordinate [label=below left:$1$] (A) at (.6,0);
%       \coordinate [label=above left:$2$] (B) at (1.2,0);
%       \coordinate [label=above left:$-3$] (C) at (-1.8,0);
%       \fill[cstdot,second] (A) circle;
%       \fill[cstdot,second] (B) circle;
%       \fill[cstdot,second] (C) circle;
%     \end{tikzpicture}
%     \caption{$y=x^3-7x+6$\footnotemark[1]}
%   \end{minipage}
% \end{figure}
% \footnotetext{图像的横纵坐标比例有放缩}

\begin{example}
  解方程 $x^3-7x+6=0$.
\end{example}

\begin{solution}
  同样地我们有 $x=u+v$, 其中
    \[u^3+v^3=-6,\qquad uv=\frac73.\]
  于是 $u^3,v^3$ 满足一元二次方程
    \[X^2+6X+\dfrac{343}{27}=0.\]
  然而这个方程没有实数解.

  我们可以强行解得
    \[u^3=-3+\frac{10}9\sqrt{-3},\]
    \[u=\sqrt[3]{-3+\frac{10}9\sqrt{-3}}
      =\frac{3+2\sqrt{-3}}3,\frac{-9+\sqrt{-3}}6,\frac{3-5\sqrt{-3}}6,\]
  相应地,
    \[v=\frac{3-2\sqrt{-3}}3,\frac{-9-\sqrt{-3}}6,\frac{3+5\sqrt{-3}}6,\]
  从而 $x=u+v=2,-3,1$.
\end{solution}

对于一般的三次方程 $x^3+px+q=0$ 而言, 类似可得:\footnote{%
    若 $p=0$, $q>0$, 则选择 $u^3=-\frac q2-\sqrt{\Delta}$ 以避免 $u=0$.
  }
  \[x=u-\frac p{3u},\quad u^3=-\frac q2+\sqrt{\Delta},\quad \Delta=\frac{q^2}4+\frac{p^3}{27}.\]
通过分析函数图像的极值点可以知道:\par
\begin{enumpar}
  \item 当 $\Delta>0$ 时, 有 $1$ 个实根.
  \item 当 $\Delta\le0$ 时, 有 $3$ 个实根(含重根情形).
\end{enumpar}

\begin{figure}[!h]
  \centering
  \begin{minipage}{.32\textwidth}
    \centering
    \begin{tikzpicture}
      \draw[cstaxis] (-2,0)--(2,0);
      \draw[cstaxis] (0,-2)--(0,2);
      \draw[cstcurve,main,domain=-2.6:3.3,smooth] plot ({(\x)*0.35},{(\x*\x*\x-3*\x-8.125)*0.1});
      \fill[cstdot,second] (.875,0) circle;
    \end{tikzpicture}
    \caption{$\Delta>0$}
  \end{minipage}
  \begin{minipage}{.32\textwidth}
    \centering
    \begin{tikzpicture}
      \draw[cstaxis] (-2,0)--(2,0);
      \draw[cstaxis] (0,-2)--(0,2);
      \draw[cstcurve,main,domain=-3.1:2.9,smooth] plot ({(\x)*0.35},{(\x*\x*\x-3*\x+2)*0.1});
      \fill[cstdot,second] (.35,0) circle;
      \fill[cstdot,second] (-.7,0) circle;
    \end{tikzpicture}
    \caption{$\Delta=0$}
  \end{minipage}
  \begin{minipage}{.32\textwidth}
    \centering
    \begin{tikzpicture}
      \draw[cstaxis] (-2,0)--(2,0);
      \draw[cstaxis] (0,-2)--(0,2);
      \draw[cstcurve,main,domain=-4.2:4,smooth] plot ({(\x)*0.3},{(\x*\x*\x-7.75*\x+3.75)*0.05});
      \fill[cstdot,second] (.75,0) circle;
      \fill[cstdot,second] (.15,0) circle;
      \fill[cstdot,second] (-.9,0) circle;
    \end{tikzpicture}
    \caption{$\Delta<0$}
  \end{minipage}
\end{figure}

所以我们想要使用一元三次方程求根公式的话, 就\emph{必须接受负数开方}.
那么为什么当 $\Delta<0$ 时, 从求根公式一定能得到 $3$ 个实根呢?
在学习了本章内容之后就可以回答这个问题了.

尽管在十六世纪, 人们已经得到了三次方程的求根公式, 然而对其中出现的虚数, 却是难以接受.
莱布尼茨%
  \footnote{%
    Gottfried Wilhelm Leibniz (1646--1716), 德国哲学家、数学家.
    参考 \cite[第13章2节]{Kline1990}.
  }%
对此有如下评论: \emph{圣灵在分析的奇观中找到了超凡的显示, 这就是那个理想世界的端兆, 那个介于存在与不存在之间的两栖物, 那个我们称之为虚的 $-1$ 的平方根.}



\subsection{复数的概念}

现在我们来正式介绍复数的概念.
由于方程 $x^2=-1$ 在复数范围内有两个不同的根, 为了避免记号 $\sqrt{-1}$ 带来的歧义, 我们引入符号 $i$ 来表示其中一个根.

\begin{definition}
  固定一个记号 $i$, \noun{复数}就是形如 $z=x+yi$ 的元素, 其中 $x,y$ 均是实数, 且不同的 $(x,y)$ 对应不同的复数.
  分别称 $x,y$ 为 $z$ 的\noun{实部} 和\noun{虚部}, 并记作 \noun{$\Re z$}${}=x$, \noun{$\Im z$}${}=y$.
\end{definition}

本书中, 我们将不言自明地使用 $x,y,x_1,y_1,\dots$ 等记号表示实数.
复数 $x+yi$ 也可表达为形式 $x+iy$.

记号 \noun{$i$} 叫作\noun{虚数单位}, 它最先是由莱昂哈德·欧拉%
  \footnote{%
    Leonhard Euler (1707--1783), 瑞士数学家、物理学家.
  }%
引入并使用.
将\emph{全体复数记作 $\BC$}, 全体实数记作 $\BR$\footnote{%
  全体复数、实数、有理数、整数、自然数构成的集合分别记作 $\BC,\BR,\BQ,\BZ,\BN$, 整数集合记号来自其德语 Zahlen, 其余来自它们的英文名称 complex number, real number, rational number, natural number.
  这些符号的叫做空心体, 手写时, 可在普通字母格式上添加一条竖线(对于 $\BZ$ 是斜线)来区分.
  有些文献使用黑粗体字母 $\mathbf{C,R,Q,Z,N}$ 来表示这些集合.
}.
由于实数 $x$ 可以自然地看成复数 $x+0i$, 在此观点下, 我们有 $\BR\subseteq \BC$.
我们简记 $x+0i=x,0+yi=yi$.

由定义可知, 每一个复数都可以唯一地表达成 $x+yi$ 这样的形式.
对于建立了直角坐标系的平面, 平面上的点 $(x,y)$ 和复数 $x+yi$ 一一对应.
将建立起这种对应的平面称为\noun{复平面}.
为了强调表示复数的字母 $z$, $w$ 等的不同, 也可将对应复平面称之为 $z$ 复平面、$w$ 复平面等等.
于是, 我们可将复数 $z$ 和其对应的点 $z$ 等同起来.
复数 $z$ 还与复平面上起于原点、终于点 $z$ 的向量 $\overrightarrow{OZ}$ 一一对应.

\begin{figure}[!h]
  \centering
  \begin{tikzpicture}
    \begin{scope}
      \draw[cstaxis] (-.5,0)--(3,0);
      \draw[cstaxis] (0,-.5)--(0,2.5);
      \coordinate [label=below left:$0$] (O) at (0,0);
      \coordinate [label=above:{$z=x+yi$}] (A) at (2,1.5);
      \coordinate (B) at (2,0);
      \coordinate (C) at (0,1.5);
      \draw[cstdash] (B)--(A)--(C);
      \fill[cstdot,main] (A) circle;
      \draw[Latex-Latex,line width=.5mm] (2.8,1)--(4,1) node[midway,below,emph] {一一对应};
    \end{scope}
    \begin{scope}[xshift=5cm]
      \coordinate [label=below left:$O$] (O) at (0,0);
      \coordinate [label=above:{$Z(x,y)$}] (A) at (2,1.5);
      \coordinate (B) at (2,0);
      \coordinate (C) at (0,1.5);
      \draw[cstdash] (B)--(A)--(C);
      \fill[cstdot,main] (A) circle;
      \draw[decorate,decoration={brace,amplitude=5},cstfill] (O)--(B) node[midway,above=2mm] {$x$};
      \draw[decorate,decoration={brace,amplitude=5},cstfill] (C)--(O) node[midway,right=2mm] {$y$};
      \draw[Latex-Latex,line width=.5mm] (2.8,1)--(4,1) node[midway,below,emph] {一一对应};
      \draw[cstaxis] (-.5,0)--(3,0);
      \draw[cstaxis] (0,-.5)--(0,2.5);
    \end{scope}
    \begin{scope}[xshift=10cm]
      \draw[cstaxis] (-.5,0)--(3,0);
      \draw[cstaxis] (0,-.5)--(0,2.5);
      \coordinate [label=below left:$O$] (O) at (0,0);
      \coordinate [label=above:{$\overrightarrow{OZ}=(x,y)$}] (A) at (2,1.5);
      \draw[cstcurve,cstra,second] (O)--(A);
    \end{scope}
  \end{tikzpicture}
  \caption{复数、平面上的点、平面向量一一对应}
\end{figure}

将复平面的 $x$ 轴和 $y$ 轴分别称为\noun{实轴}和\noun{虚轴}.
当 $\Im z=0$ 时, $z=x$ 是实数, 它落在实轴上.
称不是实数的复数是\noun{虚数}.
当 $\Re z=0$ 且 $z\neq 0$ 时, 称 $z=yi\neq 0$ 是\noun{纯虚数}, 它落在虚轴上.

\begin{figure}[!h]
  \centering
  \begin{minipage}{.48\textwidth}
    \centering
    \begin{tikzpicture}
      \coordinate [label=above:{实轴}] (R) at (3,0);
      \coordinate [label=right:{虚轴}] (I) at (0,2.5);
      \coordinate [label=below left:$0$] (O) at (0,0);
      \coordinate [label=above:{$z=x+yi$}] (A) at (2,1.5);
      \coordinate (B) at (2,0);
      \coordinate (C) at (0,1.5);
      \draw[cstdash] (B)--(A)--(C);
      \fill[cstdot] (A) circle;
      \draw[decorate,decoration={brace,amplitude=5},cstfill] (B)--(O) node[midway,below=1.5mm] {$\Re z$};
      \draw[decorate,decoration={brace,amplitude=5},cstfill] (C)--(O) node[midway,right=1.5mm] {$\Im z$};
      \draw[cstaxis] (-.5,0)--(R);
      \draw[cstaxis] (0,-.5)--(I);
      \draw[->,thick] (-2,.2)-|(.6,0);
      \draw[->,thick] (-2,1.3)--(0,1.3);
      \draw 
        (-2,.1) node[cstnode,emph] {实数}
        (-2,1.2) node[align=center,cstnode] {\textcolor{main}{\bfseries 纯虚数}\\不含原点};
    \end{tikzpicture}
  \end{minipage}
  \begin{minipage}{.48\textwidth}
    \centering
    \begin{tikzpicture}
      \filldraw[cstcurve,cstfill] (.8,0) circle (2.6 and 2);
      \coordinate (R) at (0,-.8);
      \filldraw[cstcurve,fill=white] (R) circle (1.2 and .7);
      \coordinate (I) at (0,.8);
      \draw (R) node[align=center] {\textcolor{main}{\bfseries 实数} \\$0,1,\sqrt2,\pi,e$};
      \draw (I) node[align=center] {\textcolor{main}{\bfseries 纯虚数} \\$i,-i,\pi i$};
      \draw[cstcurve] (I) circle (1.2 and .7);
      \draw 
        (3.7,0) node[align=center] {全\\体\\复\\数}
        (2,0) node[align=center] {虚数 \\$i,\pi i,\frac{-1+\sqrt 3 i}2$};
    \end{tikzpicture}
  \end{minipage}
  \caption{实数、纯虚数、复数和复平面的关系}
\end{figure}

\begin{example}
  实数 $x$ 取何值时, $z=(x^2-3x-4)+(x^2-5x-6)i$ 是:
  \begin{tasksexam}(4)
    \item 实数;
    \item 纯虚数.
  \end{tasksexam}
\end{example}
\begin{solution}\delspace
  \begin{enumnopar}[(i)]
    \item $\Im z=x^2-5x-6=0$, 即 $x=-1$ 或 $6$.
    \item $\Re z=x^2-3x-4=0$, 即 $x=-1$ 或 $4$.
      但同时要求 $\Im z=x^2-5x-6\neq 0$, 因此 $x\neq -1$.
      故 $x=4$.
  \end{enumnopar}
\end{solution}

\begin{exercise}
  若 $x^2(1+i)+x(5+4i)+4+3i$ 是纯虚数, 则实数 $x=$\fillblank{}.
\end{exercise}


\subsection{复数的代数运算}


\subsubsection{四则运算}
设 $z_1=x_1+y_1i,z_2=x_2+y_2i$.
由向量的加减法可自然定义出复数的加法和减法:
  \begin{align*}
    z_1+z_2&=(x_1+x_2)+(y_1+y_2)i,\\
    z_1-z_2&=(x_1-x_2)+(y_1-y_2)i.
  \end{align*}
显然, 复数的加法满足交换律和结合律:
\[z_1+z_2=z_2+z_1,\quad (z_1+z_2)+z_3=z_1+(z_2+z_3).\]

\begin{figure}[!h]
  \centering
  \begin{tikzpicture}
    \draw[cstaxis] (-3,0)--(4,0);
    \draw[cstaxis] (0,-1)--(0,3);
    \coordinate (O) at (0,0);
    \coordinate [label=above:{$z_1$}] (Z1) at (1,2);
    \coordinate [label=left:{$z_2$}] (Z2) at (-2.5,.5);
    \coordinate [label=above left:{$z_1+z_2$}] (P) at ($(Z1)+(Z2)$);
    \coordinate [label=right:{$z_1-z_2$}] (M) at ($(Z1)-(Z2)$);
    \coordinate [label=below:{$-z_2$}] (neg) at ($(O)-(Z2)$);
    \draw[cstcurve,cstra] (O)--(Z1);
    \draw[cstcurve,cstra] (O)--(Z2);
    \draw[cstcurve,cstra,second] (O)--(P);
    \draw[cstcurve,cstra,main] (O)--(M);
    \draw[cstdash,cstra] (O)--(neg);
    \draw[cstdash] (Z2)--(P)--(M)--(neg);
  \end{tikzpicture}
  \caption{复数的加法和减法}
\end{figure}

\emph{规定 $i\cdot i=-1$}.
我们希望 $\BC$ 上的运算满足乘法分配律, 那么复数的乘法应当定义为
  \begin{align*}
    z_1\cdot z_2&=(x_1+y_1i)(x_2+y_2i)\\
    &=x_1\cdot x_2+x_1\cdot y_2i+y_1i\cdot x_2+y_1i\cdot y_2i\\
    &=(x_1x_2-y_1y_2)+(x_1y_2+x_2y_1)i.
  \end{align*}
容易证明, 复数的乘法满足交换律、结合律和分配律:
\[z_1z_2=z_2z_1,\quad (z_1z_2) z_3=z_1(z_2 z_3),\quad z_1(z_2+z_3)=z_1z_2+z_1z_3.\]

当 $z=x+yi\neq0$ 时, 可以发现
  \[(x+yi)\cdot\frac{x-yi}{x^2+y^2}=1.\]
从而除法定义为
  \[\frac{z_1}{z_2}=(x_1+y_1i)\cdot\frac{x_2-y_2i}{x_2^2+y_2^2}=\frac{x_1x_2+y_1y_2}{x_2^2+y_2^2}+\frac{x_2y_1-x_1y_2}{x_2^2+y_2^2}i.\]
不难看出, 当 $z_1,z_2$ 是实数时, 相应的四则运算和实数的四则运算结果是相同的.

对于正整数 $n$, 定义 $z$ 的 \nouns{$n$ 次幂}{乘幂}为 $n$ 个 $z$ 相乘.
当 $z\neq 0$ 时, 还可以定义
\[z^0=1,\qquad z^{-n}=\frac1{z^n}.\]

\subsubsection{单位根}
\begin{example}\delspace
  \begin{enumnopar}[(i)]
    \item 容易知道 $i^2=-1,i^3=-i,i^4=1$.
    一般地, 对于整数 $n$, 
    \[i^{4n}=1,\quad i^{4n+1}=i,\quad i^{4n+2}=-1,\quad i^{4n+3}=-i.\]
    \item 令 $\omega=\dfrac12(-1+\sqrt 3i)$, 则
    \[\omega^2=\frac12(-1-\sqrt3i),\qquad\omega^3=1.\]
    \item 令 $z=1+i$, {则
    \[z^2=2i,\quad z^3=-2+2i,\quad z^4=-4,\quad z^8=16=2^4.\]}
  \end{enumnopar}

  对于正整数 $n$, 将满足 $z^n=1$ 的复数 $z$ 称为 \nouns{$n$ 次单位根}{单位根}.
  那么 $1,i,-1,-i$ 是 $4$ 次单位根, $1,\omega,\omega^2$ 是 $3$ 次单位根, $(1+i)/\sqrt2$ 是 $8$ 次单位根.
\end{example}

实数范围内的诸多代数恒等式在复数情形也是成立, 例如
\[1+q+q^2+\cdots+q^n=\begin{cases}
  \dfrac{1-q^{n+1}}{1-q},&\text{若}\ q\neq 1;\\
  n+1,&\text{若}\ q=1.
\end{cases}\]

\begin{example}
  化简 $1+i+i^2+i^3+i^4$.
\end{example}
\begin{solution}
  根据等比数列求和公式,
  \[1+i+i^2+i^3+i^4=\frac{i^5-1}{i-1}
  {=\frac{i-1}{i-1}=1.}\]
\end{solution}

\begin{exercise}
  化简 $\left(\dfrac{1-i}{1+i}\right)^{2020}$=\fillblank{}.
\end{exercise}

\subsubsection{复数域的性质\optional}
\label{sssec:complex-field}

若集合 $F$ 带有元素 $0,1$ 以及四则运算\footnote{%
  即有运算 $+$ 和 $\times$, 且对任意 $a\in S$, 存在 $b\in S$ 使得 $a+b=b+a=0$; 对任意非零 $a\in F$, 存在 $c\in F$ 使得 $a\times c=c\times a=1$.%
}, 且
\begin{itemize}
  \item 满足加法交换、结合律, 乘法交换、结合、分配律\footnote{即对任意 $a,b,c\in F$, 有 $a+b=b+a$, $a+(b+c)=(a+b)+c$, $ab=ba$, $a(bc)=(ab)c$, $a(b+c)=ab+ac$.};
  \item 对任意 $a\in F$, $a+0=a\times 1=a$,
\end{itemize}
则称 $F$ 构成一个\noun{域}.
有理数全体 $\BQ$, 实数全体 $\BR$, 复数全体 $\BC$ 均构成域.

与有理数域和实数域有着本质不同的是, 复数域是\nouns{代数闭域}{域!代数闭域}:
对于任意 $n\ge 1$ 个复数 $c_0,c_1,\dots,c_{n-1}$, 复系数多项式
  \[p(z)=z^n+c_{n-1}z^{n-1}+\cdots+c_1z+c_0,\]
总存在复数 $z_0$ 使得 $p(z_0)=0$.
由此归纳可知, 复系数多项式可以因式分解成一次多项式的乘积.
我们会在第五章证明该结论.

在 $\BQ,\BR$ 上可以定义出一个``好的''大小关系, 换言之它们是\nouns{有序域}{域!有序域}, 即存在一个满足下述性质的 $>$:
\begin{itemize}
  \item 若 $a\neq b$, 则要么 $a>b$, 要么 $b>a$;
  \item 若 $a>b$, 则对于任意 $c$, $a+c>b+c$;
  \item 若 $a>b,c>0$, 则 $ac>bc$.
\end{itemize}
而 \emph{$\BC$ 却不是有序域}.
这是因为: 如果 $i>0$, 则
  \[-1=i\cdot i>0,\quad -i=-1\cdot i>0.\]
于是 $0>i$, 矛盾! 同理 $i<0$ 也不可能.


\subsection{共轭复数}

\begin{definition}
  称 $z$ 在复平面关于实轴的对称点为它的\noun{共轭复数} \noun{$\ov z$}.
换言之, $\ov{x+yi}=x-yi$.
\end{definition}


\begin{figure}[!h]
  \centering
  \begin{tikzpicture}
    \draw[cstaxis] (-.5,0)--(4,0);
    \draw[cstaxis] (0,-1.5)--(0,1.5);
    \coordinate [label=below left:$0$] (O) at (0,0);
    \coordinate [label=right:{$z=x+yi$}] (Z) at (2,1);
    \coordinate [label=right:{$\ov z=x-yi$}] (Zbar) at (2,-1);
    \coordinate (R) at (2,0);
    \coordinate (I) at (0,1);
    \draw[cstdash] (I)--(Z)--(Zbar)--(0,-1);
    \fill[cstdot,main] (Z) circle;
    \fill[cstdot,main] (Zbar) circle;
    \draw[cstcurve,cstra,second] (O)--(Z);
    \draw[cstcurve,cstra,second] (O)--(Zbar);
  \end{tikzpicture}
  \caption{共轭复数}
\end{figure}

\begin{exercise}
  $z$ 关于虚轴的对称点是\fillblank{}.
\end{exercise}

从定义出发, 不难验证共轭复数满足如下性质:
\begin{enumpar}
  \item $z$ 是 $\ov z$ 的共轭复数, 即 $\ov{\ov z}=z$.
  \label{enum:conjugate-zbarbar}
  \item $\ov{z_1\pm z_2}=\ov{z_1}\pm\ov{z_2},\ 
  \ov{z_1\cdot z_2}=\ov{z_1}\cdot\ov{z_2},\ 
  \ov{\Bigl(\dfrac{z_1}{z_2}\Bigr)}=\dfrac{~\ov{z_1}~}{~\ov{z_2}~}$.
  \label{enum:conjugate-commutes}
  \item $z\ov{z}=(\Re z)^2+(\Im z)^2$.
  \item $z+\ov z=2\Re z,\ z-\ov z=2i\Im z$.
  \label{enum:conjugate-xy-zzbar}
  \item $z=\ov z$ 当且仅当 $z$ 是实数; $z=-\ov z$ 当且仅当 $z$ 是纯虚数或 $z=0$.
  \label{enum:conjugate-real-imag}
\end{enumpar}\par\noindent
\ref{enum:conjugate-commutes} 表明共轭运算和四则运算交换.
\ref{enum:conjugate-xy-zzbar} 表明了复数 $z$ 的实部 $x$ 和虚部 $y$ 可以用 $z,\ov z$ 表出.
这意味着使用共轭复数进行计算和证明,往往比直接使用 $x,y$ 表达的形式更简单.

\begin{example}
  证明 $z_1\cdot\ov{z_2}+\ov{z_1}\cdot z_2=2\Re{(z_1\cdot\ov{z_2})}$.
\end{example}

我们可以设 $z_1=x_1+y_1i,z_2=x_2+y_2i$, 然后代入等式两边化简并比较实部和虚部来证明.
但利用共轭复数可以更简单地证明它.

\begin{proof}
  根据共轭复数的性质 \ref{enum:conjugate-zbarbar} 和 \ref{enum:conjugate-commutes},
  \[\ov{z_1\cdot\ov{z_2}}=\ov{z_1}\cdot\ov{\ov{z_2}}=\ov{z_1}\cdot z_2.\]
  于是由共轭复数的性质 \ref{enum:conjugate-xy-zzbar} 可得
    \[z_1\cdot\ov{z_2}+\ov{z_1}\cdot z_2
      =z_1\cdot\ov{z_2}+\ov{z_1\cdot\ov{z_2}}
      =2\Re{(z_1\cdot\ov{z_2})}.\]
\end{proof}

\begin{example}
  设 $z=x+yi$ 且 $y\neq 0$. 证明: $x^2+y^2=1$ 当且仅当 $z+\dfrac1z$ 是实数.
\end{example}
\begin{proof}
  由共轭复数的性质 \ref{enum:conjugate-real-imag} 可知, 
  $z+\dfrac1z$ 是实数当且仅当
    \[z+\frac1z=\ov{\Bigl(z+\frac1z\Bigr)}=\ov z+\frac1{\ov z}.\]
  由于 $z\neq 0$, 化简可知上述等式等价于
    \[(z-\ov z)(z\ov z-1)=0.\]
  由 $y\neq0$ 可知 $z\neq \ov z$.
  故上述等式等价于 $z\ov z=1$, 即 $x^2+y^2=1$.
\end{proof}

\begin{exercise}
  将 $(x_1^2+y_1^2)(x_2^2+y_2^2)$ 表达为两个多项式的平方和.
\end{exercise}

由于 $z\ov z$ 是一个实数,
因此在做复数的除法运算时, 可以利用下式将其转化为乘法:
  \[\dfrac{z_1}{z_2}=\dfrac{z_1\ov{z_2}}{z_2\ov{z_2}}=\dfrac{z_1\ov{z_2}}{x_2^2+y_2^2}.\]
\begin{example}
  设 $z_1=10-5i,z_2=4+3i$, 求 $\ov{\left(\dfrac{z_1}{z_2}\right)}$.
\end{example}
\begin{solution}
  \begin{align*}
    \frac{z_1}{z_2}&=\frac{10-5i}{4+3i}
    =\frac{(10-5i)(4-3i)}{4^2+3^2}\\
    &=\frac{(40-15)+(-20-30)i}{25}=1-2i,
  \end{align*}
  因此 $\ov{\left(\dfrac{z_1}{z_2}\right)}=1+2i$.
\end{solution}

\begin{exercise}
  计算 $\dfrac{2-3i}{2-i}=$\fillblank[6em]{}.
\end{exercise}

\section{复数的三角与指数形式}

\subsection{复数的模和辐角}

由平面的极坐标表示, 我们可以得到复数的另一种表示方式.
以 $0$ 为极点, 正实轴为极轴, 逆时针为极角方向可以自然定义出复平面上的极坐标系.
设复数 $z=x+yi$ 对应点的极坐标为 $(r,\theta)$.
由极坐标和直角坐标的转化关系可知:
\[x=r\cos\theta,\qquad y=r\sin\theta,\]
\[r=\sqrt{x^2+y^2},\quad \theta=\arctan\dfrac yx+2k\pi\ \text{或}\ \arctan\dfrac yx+\pi+2k\pi,k\in\BZ.\]

\begin{figure}[!h]
  \centering
  \begin{tikzpicture}
    \coordinate [label=below left:$0$] (O) at (0,0);
    \coordinate [label=above:{$z=x+yi$}] (Z) at (3,2);
    \coordinate (X) at (3,0);
    \coordinate (Y) at (0,2);
    \draw[decorate,decoration={brace,amplitude=5},cstfill] (X)--(O) node[midway,below=2mm] {$x$};
    \draw[decorate,decoration={brace,amplitude=5},cstfill] (O)--(Y) node[midway,left=2mm] {$y$};
    \draw[main,thick,cstra] pic [cstfill1,draw=main, "$\theta$", angle eccentricity=1.3, angle radius=0.8cm] {angle=X--O--Z};
    \draw[cstaxis] (-.5,0)--(4,0);
    \draw[cstaxis] (0,-.5)--(0,3);
    \draw[cstcurve,main,cstra] (O)--(Z) node[midway,above,main] {$r$};
    \draw[cstdash] (X)--(Z)--(Y);
    \fill[cstdot,second] (Z) circle;
  \end{tikzpicture}
  \caption{复数的模和辐角}
\end{figure}

\begin{definition}
  \begin{enumpar}
    \item 称 $r$ 为 $z$ 的\noun{模}, 记为 \noun{$|z|$}${}=r=\sqrt{x^2+y^2}$.
    \item 称 $\theta$ 为 $z$ 的\noun{辐角}, 记为 \noun{$\Arg z$}${}=\theta$.
    约定 \alert{$0$ 的辐角没有意义}.
  \end{enumpar}
\end{definition}

任意非零复数 $z$ 的辐角有无穷多个.
这是因为若 $\theta$ 是 $z$ 的辐角, 则 $\theta+2k\pi$ 也是 $z$ 的辐角, 其中 $k$ 是任意整数.
我们固定选择 $\Arg z$ 中位于 $(-\pi,\pi]$ 的那个, 并称之为\nouns{辐角主值}{辐角!辐角主值}%
  \footnote{%
    部分文献选择位于 $[0,2\pi)$ 的辐角作为辐角主值. 由于这会导致后续对数函数主值在正实轴上不解析, 因此本书选择位于 $(-\pi,\pi]$ 的辐角作为辐角主值.
  }%
, 记作 \noun{$\arg z$}.
那么 $\Arg z=\arg z+2k\pi, k\in\BZ$.
根据 $z$ 所处的位置, 我们有
\begin{equation}\label{eq:arg}
  \arg z=\begin{cases}
    \arctan\dfrac yx,&x>0;\\
    \arctan\dfrac yx+\pi,&x<0,y\ge0;\\
    \arctan\dfrac yx-\pi,&x<0,y<0;\\
    \dfrac\pi2,&x=0,y>0;\\
    -\dfrac\pi2,&x=0,y<0.
  \end{cases}
\end{equation}

\begin{figure}[!h]
  \centering
  \begin{tikzpicture}
    \draw[cstaxis](-2.5,0)->(2.5,0); 
    \draw[cstaxis](0,-2)->(0,2);
    \draw[cstaxis,main,cstwla] (-1.5,0) arc(180:-180:1.5);
    \filldraw[cstdote,draw=main] (-1.5,-.07) circle;
    \coordinate [label=above:\textcolor{main}{$0$}] (A) at (1.7,0);
    \fill[cstdot,main] (A) circle;
    \coordinate [label=above right:{$\arctan\dfrac yx$}] (B) at (.9,.9);
    \fill[cstdot,main] (B) circle;
    \coordinate [label=right:{$\arctan\dfrac yx$}] (C) at (1.4,-.9);
    \fill[cstdot,main] (C) circle;
    \coordinate [label=above left:{$\arctan\dfrac yx+\pi$}] (D) at (-1.1,.8);
    \fill[cstdot,second] (D) circle;
    \coordinate [label=below:{$\pi$}] (E) at (-.9,0);
    \fill[cstdot,second] (E) circle;
    \coordinate [label=left:{$\arctan\dfrac yx-\pi$}] (F) at (-1.6,-.7);
    \fill[cstdot,main] (F) circle;
    \coordinate [label=left:{$\dfrac\pi2$}] (G) at (0,.5);
    \fill[cstdot,main] (G) circle;
    \coordinate [label=right:{$-\dfrac\pi2$}] (H) at (0,-.6);
    \fill[cstdot,main] (H) circle;
  \end{tikzpicture}
  \caption{辐角主值与复数位置的关系}
\end{figure}

\begin{exercise}
  $\arg \ov z=-\arg z$ 是否对于非零复数都成立?  
\end{exercise}

复数的模满足如下性质:
\begin{enumpar}
  \item $z\ov z=|z|^2=|\ov z|^2$;
  \label{enum:modulus-zzbar}
  \item $\abs{\Re z},\abs{\Im z}\le |z|\le\abs{\Re z}+\abs{\Im z}$;
  \label{enum:re-im-less}
  \item $\big||z_1|-|z_2|\big|\le|z_1\pm z_2|\le|z_1|+|z_2|$;
  \label{enum:modulus-triangle-inequality}
  \item $|z_1+z_2+\cdots+z_n|\le|z_1|+|z_2|+\cdots+|z_n|$.
  \label{enum:modulus-nangle-inequality}
\end{enumpar}\par\noindent
这些性质可通过\ref{fig:inequality_of_modulus}使用几何方式证明, 也可利用代数方法直接证明.

\begin{figure}[!h]
  \centering
  \begin{tikzpicture}
    \draw[cstaxis] (-4.4,0)--(4.3,0);
    \draw[cstaxis] (0,-3)--(0,3);
    \coordinate (O) at (0,0);
    \coordinate (Z) at (-2.5,1.5);
    \coordinate (R) at (-2.5,0);
    \draw[decorate,decoration={brace,amplitude=5},cstfill] (O)--(R) node[midway,below=2mm] {$\abs{\Re z}$};
    \draw[decorate,decoration={brace,amplitude=5},cstfill] (R)--(Z) node[midway,left=2mm] {$\abs{\Im z}$};
    \draw[cstcurve,second] (O)--(Z) node[midway,above] {$|z|$};
    \draw[cstcurve,main] (Z)--(R)--(O);
    \draw[thick] (R) ++(0,.3)--++(.3,0)--++(0,-.3);

    \coordinate [label=below right:{$z_1$}] (Z1) at (2.8,-.4);
    \coordinate (Z2) at (1.2,2);
    \coordinate (A) at (2.7,2.4);
    \coordinate [label=above right:{$z_1+z_2$}] (P) at ($(Z1)+(Z2)$);
    \coordinate [label=below:{$z_1-z_2$}] (M) at ($(Z1)-(Z2)$);
    \draw[decorate,decoration={brace,amplitude=5}] (Z1)--(O) node[midway,below,sloped] {$|z_1|$};
    \draw[decorate,decoration={brace,amplitude=5}] (P)--(Z1) node[midway,below,sloped] {$|z_2|$};
    \draw[decorate,decoration={brace,amplitude=5}] (O)--(P) node[midway,above,sloped] {$|z_1+z_2|$};
    \draw[decorate,decoration={brace,amplitude=5}] (Z1)--(M) node[midway,below,sloped] {$|z_2|$};
    \draw[decorate,decoration={brace,amplitude=5}] (M)--(O) node[midway,below,sloped] {$|z_1-z_2|$};
    \draw[decorate,decoration={brace,amplitude=5}] (A)--(P) node[midway,above,sloped] {$|z_3|$};
    \draw[decorate,decoration={brace,amplitude=5}] (O)--(A) node[midway,above=2mm,sloped] {$|z_1+z_2+z_3|$};
    \begin{scope}[cstcurve,cstra]
      \draw (O)--(Z1);
      \draw (Z1)--(P);
      \draw (P)--(A);
      \draw[main] (O)--(A);
      \draw[main] (O)--(P);
      \draw[main] (O)--(M);
      \draw (Z1)--(M);
    \end{scope}
  \end{tikzpicture}
  \caption{复数模的不等式关系}\label{fig:inequality_of_modulus}
\end{figure}

\begin{example}
  证明\delspace
  \begin{enumnopar}[(i)]
    \item $|z_1z_2|=|z_1\ov{z_2}|=|z_1|\cdot|z_2|$;
    \label{enum:modulus-z1z2-z1-z2}
    \item $|z_1+z_2|^2=|z_1|^2+|z_2|^2+2\Re(z_1\ov{z_2})$.
    \item $\bigl||z_1|-|z_2|\bigr|\le|z_1+z_2|$.
  \end{enumnopar}
\end{example}

\begin{proof}\delspace
  \begin{enumnopar}[(i)]
    \item 根据复数模的性质 \ref{enum:modulus-zzbar},
    \[|z_1z_2|^2=z_1z_2\cdot\ov{z_1}\ov{z_2}
    =z_1\cdot z_2\cdot \ov{z_1}\cdot \ov{z_2}=|z_1|^2\cdot|z_2|^2.\]
    由于复数模总是非负实数, 因此 $|z_1z_2|=|z_1|\cdot|z_2|$.
    将 $z_2$ 换成 $\ov{z_2}$ 得到
    \[|z_1\ov{z_2}|=|z_1|\cdot|\ov{z_2}|=|z_1|\cdot|z_2|.\]
    \item 因为
    \begin{align*}
        |z_1+z_2|^2
       =(z_1+z_2)(\ov{z_1}+\ov{z_2})
      &=z_1\ov{z_1}+z_2\ov{z_2}+z_1\ov{z_2}+\ov{z_1}z_2,\\
        |z_1|^2+|z_2|^2+2\Re(z_1\ov{z_2})
      &=z_1\ov{z_1}+z_2\ov{z_2}+z_1\ov{z_2}+\ov{z_1\ov{z_2}},
    \end{align*}
    而 $\ov{z_1\ov{z_2}}=\ov{z_1}z_2$, 所以两式相等.
    \item 根据 \ref{enum:modulus-z1z2-z1-z2}, 我们有
      \[\bigl||z_1|-|z_2|\bigr|^2=|z_1|^2-2|z_1|\cdot|z_2|+|z_2|^2=|z_1|^2-2|z_1 \ov{z_2}|+|z_2|^2.\]
      由于
      \[|z_1\ov{z_2}|=|{-z_1\ov{z_2}}|\ge\Re(-z_1\ov z_2)=-\Re(z_1\ov z_2),\]
      因此
      \[\bigl||z_1|-|z_2|\bigr|^2\le |z_1|^2+2\Re(z_1 \ov{z_2})+|z_2|^2=|z_1+z_2|^2.\]
      而复数模总是非负实数, 所以 $\bigl||z_1|-|z_2|\bigr|\le|z_1+z_2|$.
  \end{enumnopar}
\end{proof}

\begin{exercise}
  什么时候 $|z_1+z_2+\cdots+z_n|=|z_1|+|z_2|+\cdots+|z_n|$ 成立?
\end{exercise}



\subsection{复数的三角形式和指数形式}

由 $x=r\cos\theta,y=r\sin\theta$ 可得
  \[z=r(\cos\theta+i\sin\theta).\]
我们把这种形式称为复数的\noun{三角形式}.
定义 \emph{$e^{i\theta}:=\cos\theta+i\sin\theta$}, 则上述形式可改写为复数的\noun{指数形式}%
  \footnote{%
    $e^{i\theta}:=\cos\theta+i\sin\theta$ 叫作\noun{欧拉恒等式}, 它最先由英国数学家罗杰·科茨 (Roger Cotes, 1682--1716) 于 1714 年提出\cite{Cotes1714}, 后被欧拉于 1748 年重新发现.
    我们会在第二章说明为何如此定义.
  }%
  \[z=re^{i\theta}.\]
这两种形式是等价的, 指数形式可以认为是三角形式的一种缩写方式.
指数形式也可写作 $z=r\exp(i\theta)$.
求复数的三角和指数形式的\emph{关键在于计算模和辐角}.

\begin{example}
  将 $z=-2-\sqrt{12}i$ 化成三角形式和指数形式.
\end{example}

\begin{solution}
  显然 $|z|=\sqrt{4+12}=4$.
  由于 $z$ 在第三象限, 由\ref{eq:arg}可知
    \[\arg z=\arctan\frac{-\sqrt{12}}{-2}-\pi=\frac\pi3-\pi=-\frac{2\pi}3.\]
  因此 $z$ 的三角形式和指数形式为
    \[z=4\bigl(\cos(-\frac{2\pi}3)+i\sin(-
    \frac{2\pi}3)\bigr)=4e^{-\frac{2\pi i}3}.\]
\end{solution}

\begin{example}
  设 $0<\theta<\dfrac\pi2$. 将 $z=\sin\theta-i\cos\theta$ 化成三角形式和指数形式.
\end{example}
\begin{solution}
  显然 $|z|=\sqrt{\sin^2\theta+\cos^2\theta}=1$. 由于 $z$ 在第四象限, 由\ref{eq:arg}可知
  \[\arg z=\arctan(-\frac{\cos\theta}{\sin\theta})=-\arctan\tan(\frac\pi2-\theta)=-(\frac\pi2-\theta)=\theta-\frac\pi2.\]
  因此 $z$ 的三角形式和指数形式为
  \[z=\displaystyle\cos(\theta-\frac\pi2)+i\sin(\theta-\frac\pi2)=e^{i(\theta-\frac\pi2)}.\]
\end{solution}
\begin{solution}[另解]
  利用三角函数间的等式关系得到
  \begin{align*}
    \sin\theta&=\cos(\frac\pi2-\theta)=\cos(\theta-\frac\pi2),\\
    -\cos\theta&=-\sin(\frac\pi2-\theta)=\sin(\theta-\frac\pi2).
  \end{align*}
  因此 $z$ 的三角形式和指数形式为
  \[z=\displaystyle\cos(\theta-\frac\pi2)+i\sin(\theta-\frac\pi2)=e^{i(\theta-\frac\pi2)}.\]
\end{solution}

求复数的三角或指数形式时, 只需要任取一个辐角就可以了, 不要求必须是辐角主值.

\begin{exercise}
  将 $z=\sqrt 3-3i$ 化成三角形式和指数形式.
\end{exercise}

\begin{example}
  设 $z=e^{i\theta}$, 其中 $-\pi<\theta<\pi$.
  求 $1+z$ 的三角形式和指数形式.
\end{example}
\begin{solution}
  我们有
  \[1+z=(1+\cos\theta)+i\sin\theta=2\cos^2\frac\theta2+2i\sin\frac\theta2\cos\frac\theta2
  =2\cos\frac\theta2(\cos\frac\theta2+i\sin\frac\theta2).\]
  由 $-\pi<\theta<\pi$ 可知 $\cos\dfrac\theta2>0$.
  因此 $1+z$ 的三角形式和指数形式为
  \[1+z=2\cos\frac\theta2(\cos\frac\theta2+i\sin\frac\theta2)=2\cos\frac\theta2e^{\frac{i\theta}2}.\]
\end{solution}

\begin{figure}[!h]
  \centering
  \begin{tikzpicture}
    \coordinate [label=below left:0] (O) at (0,0);
    \coordinate [label=right:{$e^{i\varphi}$}] (Z1) at ({3*cos(18)},{3*sin(18)});
    \coordinate [label=left:{$e^{i\theta}$}] (Z2) at ({3*cos(130)},{3*sin(130)});
    \coordinate [label=above:{$e^{i\theta}+e^{i\varphi}$}] (P) at ($(Z1)+(Z2)$);
    \coordinate (M) at ($0.5*(P)$);
    \coordinate (X) at (2,0);
    \draw[thick] pic [cstfill, draw=third,"$\varphi$", angle eccentricity=1.4, angle radius=0.7cm] {angle=X--O--Z1};
    \draw[thick] pic [cstfill, draw=third, "$\frac{\theta-\varphi}2$", angle eccentricity=1.7] {angle=Z1--O--P};
    \draw[cstaxis] (-3,0)--(3,0);
    \draw[cstaxis] (0,-.4)--(0,3.5);
    \draw[cstcurve,cstra,main] (O)--(Z1);
    \draw[cstcurve,cstra,main] (O)--(Z2);
    \draw[cstcurve,cstra,second] (O)--(P);
    \draw[cstdash] (Z2)--(Z1)--(P)--(Z2);
    \draw[thick] (M)--++({.3*cos(16)},{-.3*sin(16)})--++({.3*sin(16)},{.3*cos(16)})--++({-.3*cos(16)},{.3*sin(16)});
  \end{tikzpicture}
  \caption{模相等的复数之和}\label{fig:sum_of_same_modulus}
\end{figure}

两个模相等的复数之和的三角和指数形式形式较为简单:
\begin{equation}\label{eq:sum-of-same-modulus}
  re^{i\theta}+re^{i\varphi}=2r\cos\frac{\theta-\varphi}2e^{\frac{\theta+\varphi}2i}.
\end{equation}
从\ref{fig:sum_of_same_modulus}可以很容易得到该等式.
注意 $\cos\dfrac{\theta-\varphi}2<0$ 时, 这离指数形式还差一步变形.

\section{复数的乘除、乘幂和方根}

三角形式和指数形式在进行复数的乘法、除法和幂次计算中非常方便.
\subsection{复数的乘除与三角形式和指数形式}

\begin{theorem}
\label{thm:multiply}
  设复数 $z_1,z_2\neq0$ 的三角形式为
  \[z_1=r_1(\cos\theta_1+i\sin\theta_1),\qquad
    z_2=r_2(\cos\theta_2+i\sin\theta_2),\]
  则
  \begin{align*}
    z_1z_2&=r_1r_2[\cos(\theta_1+\theta_2)+i\sin(\theta_1+\theta_2)],\\
    \frac{z_1}{z_2}&=\frac{r_1}{r_2}[\cos(\theta_1-\theta_2)+i\sin(\theta_1-\theta_2)].
  \end{align*}
\end{theorem}

若表达为指数形式, 上述等式变为
  \[
  r_1e^{i\theta_1}\cdot r_2e^{i\theta_2}=r_1r_2e^{i(\theta_1+\theta_2)},\qquad
  \frac{r_1e^{i\theta_1}}{r_2e^{i\theta_2}}=\frac{r_1}{r_2}e^{i(\theta_1-\theta_2)}.
  \]
该定理表明\footnote{多值函数相等是指两边所能取到的值构成的集合相等.
例如此处关于辐角的等式的含义是:
  \[\Arg{(z_1z_2)}=\{\theta_1+\theta_2\mid\theta_1\in\Arg z_1,\theta_2\in\Arg z_2\}.\]
  \[\Arg{\frac{z_1}{z_2}}=\{\theta_1-\theta_2\mid\theta_1\in\Arg z_1,\theta_2\in\Arg z_2\}.\]
}
  \[|z_1z_2|=|z_1|\cdot|z_2|,\quad\abs{\frac{z_1}{z_2}}=\frac{|z_1|}{|z_2|},\]
  \[\Arg{(z_1z_2)}=\Arg z_1+\Arg z_2,\quad
  \Arg{\frac{z_1}{z_2}}=\Arg z_1-\Arg z_2.\]
注意上述等式中 $\Arg$ 不能换成 $\arg$, 也就是说
  \[\arg{(z_1z_2)}=\arg z_1+\arg z_2,\quad
  \arg{\frac{z_1}{z_2}}=\arg z_1-\arg z_2\]
\emph{不一定成立}.
事实上, 当且仅当等式右侧落在区间 $(-\pi,\pi]$ 内时才成立, 否则等式两侧会相差 $\pm2\pi$.
例如 $z_1=z_2=e^{0.99\pi i}$, $z_1z_2=e^{1.98\pi i}$,
\[\arg z_1+\arg z_2=0.99\pi+0.99\pi=1.98\pi,\qquad
\arg(z_1z_2)=-0.02\pi.\]

\begin{proof}
  根据和差的正弦、余弦公式可知
  \begin{align*}
    z_1z_2&=r_1(\cos\theta_1+i\sin\theta_1)\cdot
    r_2(\cos\theta_2+i\sin\theta_2)\\
    &{=r_1r_2\bigl[(\cos\theta_1\cos\theta_2-\sin\theta_1\sin\theta_2)
    +i(\cos\theta_1\sin\theta_2+\sin\theta_1\cos\theta_2)\bigr]}\\
    &{=r_1r_2\bigl[\cos(\theta_1+\theta_2)+i\sin(\theta_1+\theta_2)\bigr]}
  \end{align*}
  因此乘法情形得证.

  设 $\dfrac{z_1}{z_2}=re^{i\theta}$, 则由乘法情形可知 $rr_2=r_1$ 且 $\theta+\theta_2$ 是 $z_1$ 的辐角.
  因此 $r=\dfrac{r_1}{r_2}$, $\theta$ 可取 $\theta_1-\theta_2$.
\end{proof}


\subsection{复数乘法的几何意义}
从定理~\ref{thm:multiply} 可以看出, 乘以复数 $z=re^{i\theta}$ 可以理解为\emph{模放大为 $r$ 倍, 并沿逆时针旋转角度 $\theta$}.
由此我们可以利用复数来解决平面几何中的问题.

\begin{figure}[!h]
  \centering
  \begin{tikzpicture}
    \coordinate [label=below:{$1$}] (X) at (1.6,0);
    \coordinate (O) at (0,0);
    \coordinate [label=right:{$z=re^{i\theta}$}] (Z) at ({2*cos(50)},{2*sin(50)});
    \coordinate [label=above:{$z_1$}] (Z1) at ({2.4*cos(80)},{2.4*sin(80)});
    \coordinate [label=above:{$zz_1$}] (ZZ1) at ({3*cos(130)},{3*sin(130)});
    \draw[cstcurve] pic [cstfill,draw, "$\theta$", angle eccentricity=1.4] {angle=X--O--Z};
    \draw[cstcurve] pic [cstfill,draw, "$\theta$", angle eccentricity=1.4] {angle=Z1--O--ZZ1};
    \draw[cstaxis] (-3,0)--(3,0);
    \draw[cstaxis] (0,-.5)--(0,3);
    \draw[cstcurve,cstra] (O)--(X);
    \draw[cstcurve,main,cstra] (O)--(Z);
    \draw[cstcurve,cstra] (O)--(Z1);
    \draw[cstcurve,second,cstra] (O)--(ZZ1);
  \end{tikzpicture}
  \caption{复数乘法的几何意义}
\end{figure}

\begin{example}
  已知正三角形的两个顶点为 $z_1=2$ 和 $z_2=1+i$, 求它的另一个顶点.
\end{example}

\begin{solution}
  由于 $\overrightarrow{z_1z_3}$ 为 $\overrightarrow{z_1z_2}$ 逆时针或顺时针旋转 $\dfrac\pi3$, 根据复数乘法的几何意义,
  \begin{align*}
    z_3-z_1&=(z_2-z_1)\exp(\pm\frac{\pi i}3)
    =(-1+i)(\half\pm\frac{\sqrt3}2i)\\
    &=-\frac{\sqrt3+1}2+\frac{1-\sqrt3}2i\ \text{或}\ \frac{\sqrt3-1}2+\frac{\sqrt3+1}2i.
  \end{align*}
  因此
  \[z_3=\frac{3-\sqrt3}2+\frac{1-\sqrt3}2i\ \text{或}\ \frac{3+\sqrt3}2+\frac{1+\sqrt3}2i.\]
\end{solution}

\begin{center}
  \begin{minipage}{.48\textwidth}
    \centering
    \begin{tikzpicture}[scale=1.5]
      \coordinate [label=below:{$z_1$}] (Z1) at (2,0);
      \coordinate [label=above:{$z_2$}] (Z2) at (1,1);
      \coordinate [label=left:{$z_3$}] (Z3) at ({(1.5-sqrt(3)/2)},{(.5-sqrt(3)/2)});
      \coordinate [label=right:{$z'_3$}] (Z3p) at ({(1.5+sqrt(3)/2)},{(.5+sqrt(3)/2)});
      \draw[cstcurve]
        pic [cstfill,draw, "$\pi/3$", angle eccentricity=1.8,angle radius=4mm] {angle=Z2--Z1--Z3}
        pic [cstfill,draw, "$\pi/3$", angle eccentricity=1.8] {angle=Z3p--Z1--Z2};
      \draw[cstaxis] (-.3,0)--(3,0);
      \draw[cstaxis] (0,-.3)--(0,2);
      \draw[cstcurve,main] (Z1)--(Z2)--(Z3)--cycle;
      \draw[cstdash] (Z1)--(Z3p)--(Z2);
    \end{tikzpicture}
  \end{minipage}
  \begin{minipage}{.48\textwidth}
    \centering
    \begin{tikzpicture}
      \coordinate [label=below left:{$A$}] (A) at (0,0);
      \coordinate [label=right:{$B=z$}] (B) at ({3*cos(60)},{3*sin(60)});
      \coordinate [label=below:{$C=1$}] (C) at (2,0);
      \coordinate [label=right:{$D=w$}] (D) at ($0.4*(B)+0.6*(C)$);
      \draw[cstcurve]
        pic [cstfill,draw] {angle=C--A--D}
        pic [cstfill,draw,angle radius=4mm] {angle=D--A--B};
      \draw[cstaxis] (-.45,0)--(3.5,0);
      \draw[cstaxis] (0,-.45)--(0,3);
      \draw[cstcurve,main] (B)--(A)--(C)--cycle;
      \draw[cstcurve] (A)--(D);
    \end{tikzpicture}
  \end{minipage}
\end{center}

\begin{example}
  设 $AD$ 是 $\triangle ABC$ 的角平分线, 证明 $\dfrac{AB}{AC}=\dfrac{DB}{DC}$.
\end{example}

\begin{proof}
  不妨设 $A=0,B=z,C=1,D=w$. 由于 $B,C,D$ 共线, 因此
  \[\lambda=\dfrac{DC}{BC}=\dfrac{w-1}{z-1}\in(0,1),\]
  那么
  \[w=1+\lambda(z-1)=\lambda z+(1-\lambda).\]
  由于 $\angle BAD=\angle DAC$, 根据复数乘法的几何意义,
  $\dfrac{z-0}{w-0}$ 是 $\dfrac{w-0}{1-0}$ 的正实数倍, 即
  \[\frac{w^2}z=\lambda^2 z+2\lambda(1-\lambda)+\frac{(1-\lambda)^2}z\in\BR,\]
  于是
    \[\lambda^2z+\dfrac{(1-\lambda)^2}z=\lambda^2\ov z+\dfrac{(1-\lambda)^2}{\ov z},\qquad
    \bigl(\lambda^2|z|^2-(1-\lambda)^2\bigr)(z-\ov z)=0.\]
  显然 $z\neq \ov z$. 又因为 $0<\lambda<1$, 故
    \[\frac{AB}{AC}=|z|=\frac{1-\lambda}{\lambda}
    =\frac{BC-DC}{DC}=\frac{DB}{DC}.\]
\end{proof}


\subsection{复数的乘幂}

根据定理~\ref{thm:multiply} 归纳可知
\begin{theorem}
  设 $z=r(\cos\theta+i\sin\theta)=re^{i\theta}$ 是一非零复数.
  对任意整数 $n$, 有
  \[z^n=r^n(\cos{n\theta}+i\sin{n\theta})=r^ne^{in\theta}.\]
\end{theorem}
特别地, 当 $r=1$ 时, 我们得到\nouns{棣莫弗\footnote{
  Abraham de Moivre (1667--1754), 法国数学家.
}公式}{棣莫弗公式}
\[(\cos\theta+i\sin\theta)^n=\cos{n\theta}+i\sin{n\theta}.\]

\begin{example}
  求 $(1+i)^n+(1-i)^n$.
\end{example}

\begin{solution}
  由于
  \[
    1+i=\sqrt2(\cos\frac\pi4+i\sin\frac\pi4),\quad
    1-i=\sqrt2(\cos\frac\pi4-i\sin\frac\pi4),
  \]
  因此
    \[
       (1+i)^n+(1-i)^n
      =2^{\frac n2}(\cos\frac{n\pi}4+i\sin\frac{n\pi}4 
       +\cos\frac{n\pi}4-i\sin\frac{n\pi}4)
      =2^{\frac n2+1}\cos\frac{n\pi}4.
    \]
\end{solution}

复数的乘幂可用于计算三角函数有关的求和式.
\begin{example}
  设 $0<\theta<2\pi$, 计算
  \begin{enumpar}[(i)]
    \item $A_n=1+\cos\theta+\cos{2\theta}+\cdots+\cos{n\theta}$;
    \item $B_n=\sin\theta+\sin{2\theta}+\cdots+\sin{n\theta}$.
  \end{enumpar}
\end{example}

\begin{solution}
  设 $z=\cos\theta+i\sin\theta=e^{i\theta}$, 则 $z\neq 1$,
  \[A_n+iB_n=1+z+z^2+\cdots+z^n=\frac{z^{n+1}-1}{z-1}.\]
  由\ref{eq:sum-of-same-modulus} 可知
  \[e^{i\varphi}-1=e^{i\varphi}+e^{i\pi}=2\cos\frac{\pi-\varphi}2e^{\frac{\varphi+\pi}2i}=2\sin\frac\varphi2 e^{\frac{\varphi+\pi}2i},\]
  于是
  \begin{align*}
    z^{n+1}-1=e^{i(n+1)\theta}-1&=2\sin{\frac{(n+1)\theta}2}e^{\frac{(n+1)\theta+\pi}2i},\\
    z-1=e^{i\theta}-1&=2\sin{\frac\theta2}e^{\frac{\theta+\pi}2i}.
  \end{align*}
  两式相除得到
  \[A_n+iB_n=\frac{z^{n+1}-1}{z-1}=\frac{\sin\frac{(n+1)\theta}2}{\sin\frac\theta2}e^{\frac{n\theta}2i}.\]
  它的实部和虚部分别为
  \[A_n=\frac{\sin\frac{(n+1)\theta}2}{\sin\frac\theta2}\cos\frac{n\theta}2,\qquad
  B_n=\frac{\sin\frac{(n+1)\theta}2}{\sin\frac\theta2}\sin\frac{n\theta}2.\]
\end{solution}

\begin{exercise}
  化简 $(\sqrt3+i)^12=$\fillblank[6em]{}.
\end{exercise}

注意当 $|n|\ge 2$ 时, \emph{$\Arg(z^n)=n\Arg z$ 不成立}.
这是因为
  \begin{align*}
    \Arg(z^n)&=n\arg z+2k\pi,\quad k\in\BZ,\\
    n\Arg z&=n\arg z+2nk\pi,\quad k\in\BZ,
  \end{align*}
二者能取到所有值形成的集合不相等.
而等式 $\arg(z^n)=n\arg z$ 也仅在 $\arg z\in(-\dfrac{\pi}n,\dfrac\pi n]$ 时成立.

\subsection{复数的方根}

我们利用复数乘幂公式来计算复数 $z$ 的 \emph{$n$ 次方根 $\sqrt[n]z$}, 其中 $n$ 是正整数.
设
  \[w^n=z=re^{i\theta}\neq0,\quad w=\rho e^{i\varphi},\]
则
  \[
    w^n=\rho^n(\cos{n\varphi}+i\sin{n\varphi})
       =r(\cos\theta+i\sin\theta).
  \]
比较两边的模可知 $\rho^n=r,\rho=\sqrt[n]r$.
为了避免记号冲突, 当 $r$ 是正实数时, $\sqrt[n]r$ 默认表示 $r$ 的唯一的 $n$ 次正实根, 称之为\noun{算术根}\footnote{%
  部分文献会使用 $(\sqrt[n]r)$ 来表示算术根, 以示区分.%
}.

由于 $n\varphi$ 和 $\theta$ 的正弦和余弦均相等, 因此存在整数 $k$ 使得
  \[n\varphi=\theta+2k\pi,\quad \varphi=\frac{\theta+2k\pi}n.\]
故 $w=w_k=\sqrt[n]r\exp(\dfrac{\theta+2k\pi}ni)$.
不难看出, $w_k=w_{k+n}$, 而 $w_0,w_1,\dots,w_{n-1}$ 两两不同.
因此只需取 $k=0,1,\dots,n-1$.
\begin{theorem}\label{thm:root}
  任意一个非零复数 $z=re^{i\theta}$ 的 $n$ 次方根有 $n$ 个值:
  \[
    \sqrt[n]z=\sqrt[n]r\exp(\dfrac{\theta+2k\pi}ni)\\
      =\sqrt[n]r(\cos\frac{\theta+2k\pi}n+i\sin\frac{\theta+2k\pi}n),\quad k=0,1,\dots,n-1.
  \]
\end{theorem}

若令 $\zeta=e^{2\pi i/n}$, 则
\[\sqrt[n]{z}=w,w\zeta,w\zeta^2,\dots,w\zeta^{n-1},\]
其中 $w$ 是 $z$ 的任一 $n$ 次方根.
于是 $z$ 的 $n$ 次方根的模都相等, 且 $w_k$ 和 $w_{k+1}$ 辐角相差 $\dfrac{2\pi}n$.
因此\emph{它们是以原点为中心, $\sqrt[n]r$ 为半径的圆的内接正 $n$ 边形的顶点}.

\begin{example}
  求 $\sqrt[4]{1+i}$.
\end{example}

\begin{solution}
  由于
    \[1+i=\sqrt2\exp(\dfrac{\pi i}4),\]
  因此
    \[\sqrt[4]{1+i}=\sqrt[8]2\exp\frac{(\frac\pi4+2k\pi)i}4,\quad k=0,1,2,3.\]
  于是该方根全部值为
    \[w_0=\sqrt[8]2e^{\frac{\pi i}{16}},\quad
    w_1=\sqrt[8]2e^{\frac{9\pi i}{16}},\quad
    w_2=\sqrt[8]2e^{\frac{17\pi i}{16}},\quad
    w_3=\sqrt[8]2e^{\frac{25\pi i}{16}}.\]
\end{solution}
显然 $w_{k+1}=iw_k$,
所以 $w_0,w_1,w_2,w_3$ 形成了一个正方形.

\begin{figure}[!h]
  \centering
  \begin{tikzpicture}[scale=1.6]
    \draw[cstaxis] (-1.5,0)--(1.5,0);
    \draw[cstaxis] (0,-1.5)--(0,1.5);
    \coordinate [label=below left:$0$] (O) at (0,0);
    \draw[cstcurve,thick,cstra] (0,0) circle (1);
    \coordinate (W0) at ({cos(11.25)},{sin(11.25)});
    \coordinate (W1) at ({cos(101.25)},{sin(101.25)});
    \coordinate (W2) at ({cos(191.25)},{sin(191.25)});
    \coordinate (W3) at ({cos(281.25)},{sin(281.25)});
    \begin{scope}[cstcurve,thick,cstra]
      \draw (O)--(W0) node[right] {$w_0$};
      \draw (O)--(W1) node[above] {$w_1$};
      \draw (O)--(W2) node[left] {$w_2$};
      \draw (O)--(W3) node[below] {$w_3$};
    \end{scope}
    \draw[cstcurve,main] (W0)--(W1)--(W2)--(W3)--cycle;
  \end{tikzpicture}
  \caption{$\sqrt[4]{1+i}$ 的所有值}
\end{figure}

\begin{exercise}
  计算 $\sqrt[6]{-1}=$\fillblank[12em]{}.
\end{exercise}

我们总有
  \[\Arg \sqrt[n]z=\dfrac1n\Arg z=\dfrac{\arg z+2k\pi}n,\quad k\in\BZ,\]
其中左边表示 $z$ 的所有 $n$ 次方根的所有辐角\footnote{此即多值函数复合的含义.}.
而 $n\ge 2$ 时, $\arg \sqrt[n]z=\dfrac 1n\arg z$ 总不成立, 因为左侧是多值的.


\subsection{应用: 实系数三次方程根的情况\optional}
现在我们来看三次方程 $z^3+pz+q=0$ 的根, $p\neq 0$.
回顾求根公式:
  \[z=u+v,\quad u^3=-\frac q2+\sqrt{\Delta},\quad uv=-\frac p3,\quad \Delta=\frac{q^2}4+\frac{p^3}{27}.\]
设该方程的三个根为 $z_1,z_2,z_3$, 那么
\begin{align*}
   &z^3+pz+q=(z-z_1)(z-z_2)(z-z_3)\\
  =&z^3-(z_1+z_2+z_3)z^2+(z_1z_2+z_2z_3+z_3z_1)z-z_1z_2z_3.
\end{align*}
设 $a=z_1z_2$, 则 $z_1+z_2=-z_3$, $p=a-z_3^2$, $q=-az_3$.
于是
\begin{align*}
   &\bigl((z_1-z_2)(z_2-z_3)(z_3-z_1)\bigr)^2\\
  =&(z_1^2-2z_1z_2+z_2^2)\bigl(z_3^2-(z_1+z_2)z_3+z_1z_2\bigr)^2\\
  =&(z_3^2-4a)(2z_3^2+a)^2=4z_3^6-12az_3^4-15a^2z_3^2-4a^3\\
  =&4(z_3^2-a)^3-27a^2z_3^2=-4p^3-27q^2=-108\Delta.\footnotemark
\end{align*}
\footnotetext{%
  一般 $n$ 次多项式的判别式定义为
  \[\prod_{1\le i<j\le n}(x_i-x_j)^2,\]
  其中 $x_1,\dots,x_n$ 表示其所有(复数)根.
  判别式总可以表示为多项式系数的多项式形式.
  这里, 三次多项式的判别式就是 $-108\Delta=-4p^3-27q^2$.
}
\begin{enumpar}
  \item 如果 $\Delta>0$, 设 $\omega=e^{2\pi i/3}$, 设实数 $\alpha$ 满足
    \[\alpha^3=-\frac q2+\sqrt{\Delta},\]
  则
    \[
      u=\alpha,\alpha\omega,\alpha\omega^2,\qquad
      z=\alpha-\frac p{3\alpha},\ 
        \alpha\omega-\frac p{3\alpha} \omega^2,\ 
        \alpha\omega^2-\frac p{3\alpha} \omega.
    \]
    第一个根是实根.
    由于
    \[\bigl((z_1-z_2)(z_2-z_3)(z_3-z_1)\bigr)^2=-108\Delta<0,\]
    因此该方程无重根, 且不能有三个实根.
    而后两个根互为共轭, 因此它们都是虚数.
  \item 如果 $\Delta\le 0$, 则 $p<0$, $|u|^2=-\dfrac p3>0$. 从而 $v=\ov u$.
    设
      \[\sqrt[3]{-\frac q2+\sqrt{\Delta}}=u_1,u_2,u_3,\]
    则 $u_1,u_2,u_3$ 都是虚数. 从而得到 $3$ 个实根
      \[x=u_1+\ov{u_1},\ u_2+\ov{u_2},\ u_3+\ov{u_3}.\]
    当且仅当 $\Delta=0$ 时有重根.
\end{enumpar}


\section{曲线和区域}

\subsection{复数表平面曲线}

很多的平面图形能用复数形式的方程来表示, 这种表示方式有些时候会显得更加直观和易于理解.

\begin{example}\delspace
  \begin{enumnopar}[(i)]
    \item $|z-1-i|=2$. 该方程表示与 $1+i$ 的距离为 $2$ 的点全体, 即圆心为 $1+i$ 半径为 $2$ 的圆.
    一般的圆方程为 $|z-z_0|=R$, 其中 $z_0$ 是圆心, $R>0$ 是半径.
    设 $z=x+yi,z_0=x_0+y_0i$, 两边同时平方化简可得
    \[R^2=|z-z_0|^2=(x-x_0)^2+(y-y_0)^2.\]
    \item $|z-2i|=|z+4|$. 该方程表示与 $2i$ 和 $-4$ 的距离相等的点, 即二者连线的垂直平分线.
    设 $z=x+yi$, 两边同时平方化简可得 $2x+y+3=0$.
    \item $|z-z_1|+|z-z_2|=2a$.
    \begin{itemnopar}
      \item 当 $2a>|z_1-z_2|$ 时, 该方程表示以 $z_1,z_2$ 为焦点, $a$ 为长半轴的椭圆;
      \item 当 $2a=|z_1-z_2|$ 时, 该方程表示连接 $z_1,z_2$ 的线段;
      \item 当 $2a<|z_1-z_2|$ 时, 该方程表示空集.
    \end{itemnopar}
    \item $|z-z_1|-|z-z_2|=2a$.
    \begin{itemnopar}
      \item 当 $2a<|z_1-z_2|$ 时, 该方程表示以 $z_1,z_2$ 为焦点, $a$ 为实半轴的双曲线的一支;
      \item 当 $2a=|z_1-z_2|$ 时, 该方程表示以 $z_2$ 为起点, 与 $z_2,z_1$ 连线反向的射线;
      \item 当 $2a>|z_1-z_2|$ 时, 该方程表示空集.
    \end{itemnopar}
  \end{enumnopar}
\end{example}
\begin{figure}[!h]
  \centering
  \begin{tikzpicture}
    \begin{scope}[xshift=-6mm,yshift=-6mm]
      \draw[cstaxis] (-.9,0)--(2.1,0);
      \draw[cstaxis] (0,-.9)--(0,2.1);
      \coordinate (A) at (.6,.6);
      \draw[cstcurve,main] (A) circle(1.2);
      \draw[cstcurve,cstra] (A)--++(.96,.72);
      \fill[cstdot] (A) circle node[below] {$1+i$};
    \end{scope}
    \begin{scope}[xshift=44mm,yshift=3mm]
      \draw[cstaxis] (-2.1,0)--(.9,0);
      \draw[cstaxis] (0,-1.8)--(0,1.2);
      \coordinate (A) at (-1.6,0);
      \coordinate (B) at (0,.8);
      \draw[cstdash] (A)--(B);
      \draw[cstcurve,main] (-1.04,.88)--(.28,-1.76);
      \fill[cstdot] (A) circle node[below] {$-4$};
      \fill[cstdot] (B) circle node[right] {$2i$};
    \end{scope}
    \begin{scope}[xshift=76mm]
      \begin{scope}[rotate=30]
        \coordinate (A) at (-.72,0);
        \coordinate (B) at (.72,0);
        \draw[cstcurve,main] (0,0) circle(1.2 and .96);
        \coordinate (Z) at (.72,.768);
        \draw[cstdash] (A)--(Z)--(B);
      \end{scope}
      \fill[cstdot] (A) circle node[right] {$z_1$};
      \fill[cstdot] (B) circle node[left] {$z_2$};
      \draw[cstaxis] (-1.5,0)--(1.5,0);
      \draw[cstaxis] (0,-1.5)--(0,1.5);
    \end{scope}
    \begin{scope}[xshift=114mm]
      \coordinate (A) at (-1,0);
      \coordinate (B) at (1,0);
      \draw[cstcurve,main,domain=-58:58,smooth] plot ({sec(\x)*.5},{.866*tan(\x)});
      \coordinate (Z) at (.9,{0.2*sqrt(42)});
      \draw[cstdash] (A)--(Z)--(B);
      \fill[cstdot] (A) circle node[below] {$z_1$};
      \fill[cstdot] (B) circle node[below] {$z_2$};
      \draw[cstaxis] (-1.5,0)--(1.5,0);
      \draw[cstaxis] (0,-1.5)--(0,1.5);
    \end{scope}
  \end{tikzpicture}
\end{figure}

\begin{exercise}
  $z^2+\ov z^2=1$ 和 $z^2-\ov z^2=i$ 分别表示什么图形?
\end{exercise}

设 $x(t),y(t),t\in[a,b]$ 是两个连续函数,
则参变量方程
  \[\begin{cases}x=x(t),& \\y=y(t),&\end{cases}t\in[a,b]\]
定义了一条\noun{连续曲线}\footnote{%
  这里考虑的都是连续曲线段, 不考虑无限长的曲线.
}.
这也等价于 $C:z=z(t)=x(t)+iy(t),t\in[a,b]$.
如果除了两个端点有可能重叠外, 其它情形不会出现重叠的点, 则称 $C$ 是\noun{简单曲线}或\nouns{若尔当\footnote{%
  Marie Ennemond Camille Jordan (1838--1922), 法国数学家.
}曲线}{若尔当曲线}.
如果还满足两个端点重叠, 即 $z(a)=z(b)$, 则称 $C$ 是\noun{简单闭曲线}或\noun{闭路}或\noun{若尔当闭曲线}.

\begin{figure}[!h]
  \centering
  \begin{tikzpicture}
    \begin{scope}[yshift=-3mm]
      \draw[cstaxis] (-1.3,.3)--(1.7,.3);
      \draw[cstaxis] (-.3,-1)--(-.3,1.6);
      \coordinate (A) at (-1,-.5);
      \coordinate (B) at (1,.5);
      \draw[cstcurve,main] (A)--(B);
      \fill[cstdot] (A) circle node[below] {$z(0)$};
      \fill[cstdot] (B) circle node[above] {$z(1)$};
    \end{scope}
    \begin{scope}[xshift=25mm]
      \draw[cstcurve,main,domain=0:360,smooth] plot ({\x*pi/360},{sin(\x)*.5});
      \coordinate (A) at (-1,-.5);
      \coordinate (B) at (1,.5);
      \fill[cstdot] (0,0) circle node[below left] {$z(0)$};
      \fill[cstdot] ({pi},0) circle node[above] {$z(2\pi)$};
      \draw[cstaxis] (-.5,0)--(3.5,0);
      \draw[cstaxis] (0,-1.3)--(0,1.3);
    \end{scope}
    \begin{scope}[xshift=78mm]
      \begin{scope}[rotate=30]
        \coordinate (A) at (-.72,0);
        \coordinate (B) at (.72,0);
        \draw[cstcurve,main] (0,0) circle(1.2 and .96);
      \end{scope}
      \fill[cstdot] (A) circle node[right] {$z_1$};
      \fill[cstdot] (B) circle node[left] {$z_2$};
      \draw[cstaxis] (-1.5,0)--(1.5,0);
      \draw[cstaxis] (0,-1.3)--(0,1.3);
    \end{scope}
    \begin{scope}[xshift=114mm]
      \draw[cstcurve,main,domain=-45:45,smooth] plot ({sqrt(2*cos(2*\x))*cos(\x)},{sqrt(2*cos(2*\x))*sin(\x)});
      \draw[cstcurve,main,domain=-45:45,smooth] plot ({-sqrt(2*cos(2*\x))*cos(\x)},{sqrt(2*cos(2*\x))*sin(\x)});
      \draw[cstaxis] (-1.8,0)--(1.8,0);
      \draw[cstaxis] (0,-1.3)--(0,1.3);
    \end{scope}
  \end{tikzpicture}
  \caption{连续曲线}
\end{figure}

\begin{example}\delspace
  \begin{enumnopar}[(i)]
    \item 直线段的参数方程可以写成
    \[z(t)=z_0+(z_1-z_0)t,\quad t\in[0,1],\]
    其中 $z_0,z_1$ 为两个端点.
    它是简单曲线.
    \item 正弦函数曲线段
    \[z(t)=\sin t,\quad t\in[0,2\pi]\]
    是简单曲线.
    \item 圆 $|z-1-i|=2$ 和椭圆 $|z-\sqrt5|+|z+\sqrt5|=6$ 是简单闭曲线,
    其参数方程可分别选为
    \[z=1+i+2e^{i\theta},\quad
    z=3\cos\theta+2i\sin\theta,\quad \theta\in[0,2\pi].
    \]
    \item 双纽线 $|z^2-1|=1$ 不是简单闭曲线.
  \end{enumnopar}
\end{example}



\subsection{区域和闭区域}

为了引入极限的概念, 我们需要考虑点的邻域.
类比于高等数学中的邻域和去心邻域, 在复变函数中, 称开圆盘\footnote{%
  为了避免混淆集合中的分割符号和模, 此处用 $:$ 而不是 $\mid$ 作为集合的分割符号.%
}
  \[U(z_0,\delta)=\{z:|z-z_0|<\delta\}\]
为 $z_0$ 的一个 \nouns{$\delta$ 邻域}{邻域}, 称去心开圆盘
  \[\Uc(z_0,\delta)=\{z:0<|z-z_0|<\delta\}\]
为 $z_0$ 的一个\nouns{去心 $\delta$ 邻域}{邻域!去心邻域}.

\begin{figure}[!h]
  \centering
  \begin{minipage}{.48\textwidth}
    \centering
    \begin{tikzpicture}
      \coordinate (A);
      \filldraw[cstcurve,main,cstfill1] (A) circle (1.5);
      \draw[cstcurve,cstra] (A)--(1.2,.9)
        node[midway,above left] {$\delta$};
      \fill[cstdot] (A) circle
        node[left] {$z_0$};
    \end{tikzpicture}
  \end{minipage}
  \begin{minipage}{.48\textwidth}
    \centering
    \begin{tikzpicture}
      \coordinate (A);
      \filldraw[cstcurve,main,cstfill1] (A) circle (1.5);
      \draw[cstcurve,cstra] (A)--(1.2,.9)
        node[midway,above left] {$\delta$};
      \filldraw[cstdote,draw=main] (A) circle
        node[left] {$z_0$};
    \end{tikzpicture}
  \end{minipage}
  \caption{邻域和去心邻域}
\end{figure}

设 $G$ 是复平面的一个子集, $z_0\in\BC$.
它们的位置关系有三种可能:
\begin{enumpar}
  \item 如果存在 $z_0$ 的一个邻域 $U$ 完全包含在 $G$ 中, 则称 $z_0$ 是 $G$ 的一个\noun{内点}.
  \item 如果存在 $z_0$ 的一个邻域 $U$ 完全不包含在 $G$ 中, 则称 $z_0$ 是 $G$ 的一个\noun{外点}.
  \item 如果 $z_0$ 的任何一个邻域 $U$, 都有属于和不属于 $G$ 的点, 则称 $z_0$ 是 $G$ 的一个\noun{边界点}.
\end{enumpar}\par\noindent
显然内点都属于 $G$, 外点都不属于 $G$, 边界点则都有可能.
这类比于区间的端点和区间的关系.

\begin{figure}
  \centering
  \begin{tikzpicture}
    \filldraw[cstcurve,main,cstfill1,smooth] plot coordinates {(2,0) (1.83,.9) (.64,1.46) (-.63,1.6) (-1.66,1.01) (-2.35,0) (-1.81,-1.06) (-.73,-1.68) (.74,-1.57) (1.82,-.91) (2,0)};
    \coordinate (A) at (-.7,0);
    \draw[cstcurve] (A) circle (.5) node[above] {$z_0$};
    \fill[cstdot] (A) circle;
    \coordinate (B) at (2,0);
    \draw[cstcurve] (B) circle (.5) node[right] {$z_0$};
    \fill[cstdot] (B) circle;
    \coordinate (C) at (4,0);
    \draw[cstcurve] (C) circle (.5) node[above] {$z_0$};
    \fill[cstdot] (C) circle;
    \draw (.5,1) node {$G$};
  \end{tikzpicture}
  \caption{点与集合的位置关系}
\end{figure}

\begin{definition}
  \begin{enumpar}
    \item 如果 $G$ 的所有点都是内点, 也就是说, $G$ 的边界点都不属于它, 称 $G$ 是一个\noun{开集}.
    \item 如果 $G$ 的所有边界点都属于 $G$, 称 $G$ 是一个\noun{闭集}.
  \end{enumpar}
\end{definition}
例如
  \[|z-z_0|<R,\quad 1<\Re z<3,\quad\frac\pi4<\arg z<\dfrac{3\pi}4\]
都给出开集\footnote{注意最后一个集合不包括原点.}.
$G$ 是一个闭集当且仅当它的补集是开集.
直观上看: 由 $>,<$ 的不等式给出的集合往往是开集, 由 $\ge,\le$ 的不等式给出的集合往往是闭集\footnote{%
  当不等式中的函数都是连续函数时, 这个结论才总成立.
  例如 $\arg z>\dfrac\pi2$ 包含了负实轴, 它既不是开集也不是闭集, 因为 $\arg z$ 在负实轴上不连续.%
}.

如果 $D$ 是某个开圆盘 $U(0,R)$ 的子集, 则称它是\noun{有界}的.
否则称它是\noun{无界}的.

\begin{definition}
  如果开集 $D$ 的任意两个点之间都可以用一条完全包含在 $D$ 中的折线段连接起来, 则称 $D$ 是一个\noun{区域}.
\end{definition}

\begin{figure}[!h]
  \centering
  \begin{tikzpicture}
    \filldraw[cstcurve,main,cstfill1,smooth] plot coordinates {(2.81,0) (2.37,1.03) (.91,1.7) (-.8,1.48) (-2.29,1.05) (-2.89,0) (-2.24,-1.03) (-.92,-1.64) (.81,-1.65) (2.38,-.93) (2.81,0)};
    \filldraw[cstcurve,main,fill=white,smooth] plot coordinates {(-.86,-.3) (-1.16,.31) (-1.62,.1) (-1.68,-.69) (-1.17,-.91) (-.86,-.3)};
    \filldraw[cstcurve,main,fill=white] (.5,.3) circle (.3);
    \fill[cstdot,main] (1.5,0) circle;
    \fill[cstdot,main] (1.6,-.5) circle;
    \draw[cstcurve,second,main] plot coordinates {(1,.5) (1.2,.3) (1.2,-.3) (1.4,.5)};
    \coordinate [label=left:{$z_1$}] (A) at (-1,.8);
    \coordinate [label=below:{$z_2$}] (B) at (1,-.8);
    \draw[cstcurve] (A)--(-.2,.5)--(.2,-.5)--(B);
  \end{tikzpicture}
  \caption{区域和它的边界}\label{fig:domain}
\end{figure}

区域\footnote{
  该定义中的折线段可以换成连续曲线段.
  它也等价于开集 $D$ 不能写成两个交集为空集的非空开集的并.
  也就是说, 区域是``连通''的开集.%
}和它的边界的并集叫作\noun{闭区域}或\noun{闭域}, 记作 $\ov D$.
它是一个闭集.

\begin{example}\delspace
  \begin{enumnopar}[(i)]
    \item 区域 $|z|>1$ 的边界是 $|z|=1$, 其闭区域是 $|z|\ge 1$.
    注意数学中边界的概念与日常所说的边界是两码事, 它并不表示集合被包含在边界内部. 
    \item $\BC-\{0\}$ 和 $\BC-\BR$ 的边界分别是 $\{0\}$ 和 $\BR$, 它们的闭区域都是 $\BC$.
    不同的区域可以拥有相同的闭区域.
    \item 在\ref{fig:domain} 中, 阴影部分(不包含线条部分)中任意两点可用折线连接, 因此它是一个区域.
    这些线条和点构成了它的边界.
    \item \ref{fig:normal-domain} 列出了几种常见区域, 它们由实部、虚部、模和辐角的不等式所表示. 
    \begin{figure}[!h]
      \centering
      \begin{tikzpicture}
        \begin{scope}[xshift=0mm]
          \draw[cstaxis](-1.5,0)--(1.5,0);
          \draw[cstaxis](0,-1.5)--(0,1.5);
          \fill[cstfille1] (-1.2,0) rectangle (1.2,.8);
          \draw (0,-1.5) node[below,align=center] {上半平面\\$\Im z>0$};
        \end{scope}
        \begin{scope}[xshift=37mm]
          \draw[cstaxis](-1.5,0)--(1.5,0);
          \draw[cstaxis](0,-1.5)--(0,1.5);
          \fill[cstfille1] (-1.2,0) rectangle (1.2,-.8);
          \draw (0,-1.5) node[below,align=center] {下半平面\\$\Im z<0$};
        \end{scope}
        \begin{scope}[xshift=74mm]
          \draw[cstaxis](-1.5,0)--(1.5,0);
          \draw[cstaxis](0,-1.5)--(0,1.5);
          \fill[cstfille1] (-1.2,-1) rectangle (0,1);
          \draw (0,-1.5) node[below,align=center] {左半平面\\$\Re z<0$};
        \end{scope}
        \begin{scope}[xshift=111mm]
          \draw[cstaxis](-1.5,0)--(1.5,0);
          \draw[cstaxis](0,-1.5)--(0,1.5);
          \fill[cstfille1] (0,1) rectangle (1.2,-1);
          \draw (0,-1.5) node[below,align=center] {右半平面\\$\Re z>0$};
        \end{scope}
        \begin{scope}[yshift=-45mm]
        \begin{scope}[xshift=0mm]
          \draw[cstaxis](-1.5,0)--(1.5,0);
          \draw[cstaxis](0,-1.5)--(0,1.5);
          \fill[cstfille1] (-.6,-1) rectangle (.2,1);
          \draw[cstcurve,main] (-.6,-1)--(-.6,1);
          \draw[cstcurve,main] (.2,-1)--(.2,1);
          \draw (0,-1.5) node[below,align=center] {竖直带状区域\\$x_1<\Re z<x_2$};
        \end{scope}
        \begin{scope}[xshift=37mm]
          \draw[cstaxis](-1.5,0)--(1.5,0);
          \draw[cstaxis](0,-1.5)--(0,1.5);
          \fill[cstfille1] (-1.1,-.4) rectangle (1.1,.4);
          \draw[cstcurve,main] (-1.1,-.4)--(1.1,-.4);
          \draw[cstcurve,main] (-1.1,.4)--(1.1,.4);
          \draw (0,-1.5) node[below,align=center] {水平带状区域\\$y_1<\Im z<y_2$};
        \end{scope}
        \begin{scope}[xshift=64mm,yshift=-10mm]
          \draw[cstaxis](-.5,0)--(2.5,0);
          \draw[cstaxis](0,-.5)--(0,2.5);
          \coordinate (A) at (0,0);
          \coordinate (B) at ({2.2*cos(60)},{2.2*sin(60)});
          \coordinate (C) at ({2.2*cos(10)},{2.2*sin(10)});
          \fill[cstfille1] (A)--(B) arc(60:10:2.2)--cycle;
          \draw[cstcurve,main] (C)--(A)--(B);
          \draw (1,-.5) node[below,align=center] {角状区域\\$\alpha_1<\arg z<\alpha_2$\footnotemark[1]};
        \end{scope}
        \begin{scope}[xshift=111mm]
          \filldraw[cstcurve,main,cstfill1] (0,0) circle (1.2);
          \filldraw[cstcurve,main,fill=white] (0,0) circle (.6);
          \draw[cstaxis](-1.5,0)--(1.5,0);
          \draw[cstaxis](0,-1.5)--(0,1.5);
          \draw (0,-1.5) node[below,align=center] {圆环域\\$r<|z|<R$};
        \end{scope}
        \end{scope}
      \end{tikzpicture}
      \caption{常见区域}\label{fig:normal-domain}
    \end{figure}
    \footnotetext{这里 $-\pi\le \alpha_1<\alpha_2\le \pi$.
    包含负实轴的角状区域无法用 $\arg z$ 的单个不等式表示.}
  \end{enumnopar}
\end{example}

\begin{exercise}
  上述区域对应的闭区域是什么?
\end{exercise}

\subsection{区域的特性}

闭路 $C$ 把复平面划分成了两个区域, 一个有界一个无界.
分别称这两个区域是 $C$ 的\emph{内部}和\emph{外部}.
$C$ 是它们的公共边界.
\footnote{捷克数学家波尔查诺
 (Bernard Placidus Johann Nepomuk Bolzano, 1781--1848) 最早明确陈述了这个定理, 并指出它是需要证明的. 1893 年, 若尔当首次给出了证明, 其中假设了该定理对于简单多边形成立 (这个情形并不难证明). 不过, 数学家普遍认为这种证明是有缺陷的. 不少数学家认为第一个完备证明是由美国数学家奥斯瓦尔德·维布伦 (Oswald Veblen, 1880--1960)于1905年给出的. 参考 \cite{BergJulianMinesRichman1975}.}

\begin{figure}[!h]
  \centering
  \begin{tikzpicture}
    \fill[cstfille] (-2,-1.5) rectangle (2,1.5);
    \filldraw[cstcurve,main,cstfill1,smooth] plot coordinates {(1.18,0) (.93,.64) (.33,1.06) (-.32,1.08) (-.83,.59) (-1.15,0) (-.83,-.68) (-.36,-1) (.36,-1.08) (.83,-.69) (1.18,0)};
  \end{tikzpicture}
  \caption{闭路的内部和外部}
\end{figure}

对于一些区域 $D$, 我们在区域中画一条闭路, 闭路的内部包含在 $D$ 内. 而对于另外一些区域 $D$, 却存在闭路使得其内部不完全包含在 $D$ 内.
\begin{definition}
  如果区域 $D$ 中的任一闭路的内部都包含在 $D$ 中, 则称 $D$ 是\noun{单连通区域}.
  否则称之为\noun{多连通区域}.
\end{definition}

\begin{figure}[!h]
  \centering
  \begin{tikzpicture}
    \filldraw[cstcurve,main,cstfill1,smooth] plot coordinates {(2.81,0) (2.37,1.03) (.91,1.7) (-.8,1.48) (-2.29,1.05) (-2.89,0) (-2.24,-1.03) (-.92,-1.64) (.81,-1.65) (2.38,-.93) (2.81,0)};
    \filldraw[cstcurve,main,fill=white,smooth] plot coordinates {(-.86,-.3) (-1.16,.31) (-1.62,.1) (-1.68,-.69) (-1.17,-.91) (-.86,-.3)};
    \filldraw[cstcurve,main,fill=white] (.5,.3) circle (.3);
    \fill[cstdot,main] (1.5,0) circle;
    \fill[cstdot,main] (1.6,-.5) circle;
    \draw[cstcurve,main] plot coordinates {(1,.5) (1.2,.3) (1.2,-.3) (1.4,.5)};
    \draw[cstcurve,smooth] plot coordinates {(1.94,-.2) (1.79,.41) (1.23,.77) (.58,.81) (.04,.35) (-.3,-.2) (.03,-.81) (.53,-1.19) (1.23,-1.08) (1.76,-.82) (1.94,-.2)};
  \end{tikzpicture}
  \caption{多连通区域}
\end{figure}

单连通区域内的任意一条闭路可以``连续地变形''成一个点\footnote{%
  即对任一闭路 $\ell: z=\varphi(t),t\in[0,1]$, 存在连续函数 $F:[0,1]\times[0,1]\to \BC$ (即实部虚部都是连续函数), 使得对 $0\le s<1$, $\ell_s: z=F(s,t),t\in[0,1]$
  都是闭路, 且 $F(0,t)=\varphi(t)$, 而 $F(1,t)$ 是常值函数.%
}. 这也等价于对区域内的任一两点 $A$, $B$, 若 $\ell_0,\ell_1$ 是从 $A$ 到 $B$ 的两条连续曲线, 则 $\ell_0$ 可以连续地变形为 $\ell_1$ 且保持端点不动\footnote{%
  即存在连续函数 $F:[0,1]\times[0,1]\to \BC$ 使得
  $\ell_0: z=F(0,t)$, $\ell_1: z=F(1,t),t\in[0,1]$, 且 $F(s,0)\equiv A$, $F(s,1)\equiv B$.
}.

\begin{example}\delspace
  \begin{enumnopar}[(i)]
    \item $\Re(z^2)\le1$. 设 $z=x+yi$, 则 $\Re(z^2)=x^2-y^2\le1$. 这是无界的单连通闭区域.
    \item $\arg z\neq \pi$. 即角状区域 $-\pi<\arg z<\pi$. 这是无界的单连通区域.
    \item $\abs{\dfrac1z}\le3$. 即 $|z|\ge\dfrac13$. 这是无界的多连通闭区域.
    \item $|z+1|+|z-1|<4$. 表示一个椭圆的内部. 这是有界的单连通区域.
  \end{enumnopar}
\end{example}

\begin{figure}[!h]
  \begin{minipage}{.24\textwidth}
    \centering
    \begin{tikzpicture}
      \fill[cstfille1] (-1.414,-1) rectangle (1.414,1);
      \filldraw[cstcurve,main,domain=-45:45,smooth,fill=white] plot ({sec(\x)},{tan(\x)});
      \filldraw[cstcurve,main,domain=-45:45,smooth,fill=white] plot ({-sec(\x)},{tan(\x)});
      \draw[cstaxis] (-1.8,0)--(1.8,0);
      \draw[cstaxis] (0,-1.5)--(0,1.5);
    \end{tikzpicture}
  \end{minipage}
  \begin{minipage}{.24\textwidth}
    \centering
    \begin{tikzpicture}
      \fill[cstfille1] (0,0) circle (1.2);
      \draw[cstaxis] (0,0)--(1.5,0);
      \draw[cstaxis] (0,-1.5)--(0,1.5);
      \draw[cstdash,main] (-1.5,0)--(0,0);
    \end{tikzpicture}
  \end{minipage}
  \begin{minipage}{.24\textwidth}
    \centering
    \begin{tikzpicture}
      \fill[cstfille1] (-1.2,-1.2) rectangle (1.2,1.2);
      \filldraw[cstcurve,main,fill=white] (0,0) circle (.5);
      \draw[cstaxis] (-1.5,0)--(1.5,0);
      \draw[cstaxis] (0,-1.5)--(0,1.5);
    \end{tikzpicture}
  \end{minipage}
  \begin{minipage}{.24\textwidth}
    \centering
    \begin{tikzpicture}
      \filldraw[cstdash,main,cstfill1] (0,0) circle (1 and {0.5*sqrt(3)});
      \draw[cstaxis] (-1.5,0)--(1.5,0);
      \draw[cstaxis] (0,-1.5)--(0,1.5);
    \end{tikzpicture}
  \end{minipage}
\end{figure}

\begin{exercise}
  集合 $|z+1|>|z-1|$ 有什么特点?
\end{exercise}


\section{复变函数}

\subsection{复变函数的定义}

回忆\noun{映射}是两个集合之间的一种对应 $f:A\to B$, 使得对于每一个 $a\in A$, 有一个唯一确定的 $b=f(a)$ 与之对应.
\begin{itemize}
  \item 当 $A$ 和 $B$ 都是实数集合的子集时, 它就是一个实变量函数.
  \item 当 $A$ 和 $B$ 都是复数集合的子集时, 它就是一个复变量函数, 或简称为\noun{复变函数}.
\end{itemize}

\begin{example}
  $f(z)=\Re z,\arg z,|z|$, $z^n$ ($n\neq0$ 为整数), $\dfrac{z+1}{z^2+1}$ 都是复变函数.
\end{example}

\begin{definition}
  \begin{enumpar}
    \item 称 $A$ 为 函数 $f$ 的\emph{定义域}.
    \item 称 $\set{w=f(z)\mid z\in A}$ 为它的\emph{值域}.\footnotemark
  \end{enumpar} 
\end{definition}
\footnotetext{值域和\emph{陪域} $B$ 往往不相同. 高等数学课程中的函数陪域总可选为 $\BR$, 复变函数课程中的函数陪域总可选为 $\BC$.
尽管在某些情形下, 具有不同陪域的函数视为不同的函数, 但在高等数学和复变函数课程中, 不考虑陪域是否相同, 只要定义域和对应关系相同, 就视为同一函数.}
\begin{exercise}
  上述函数的定义域和值域分别是什么?
\end{exercise}

在复变函数理论中, 常常会遇到\nouns{多值复变函数}{复变函数!多值复变函数}, 也就是说一个 $z\in A$ 可能有多个 $w$ 与之对应.
为了方便研究, 我们常常需要对每一个 $z$, 选取固定的一个 $f(z)$ 的值.
这样便得到了这个多值函数的一个\nouns{单值分支}{复变函数!单值分支}.
\begin{example}\delspace
  \begin{enumnopar}[(i)]
    \item $\arg z$ 是无穷多值函数 $\Arg z$ 的一个单值分支.
    复变函数中常常用开头字母大写来表示多值函数, 对应的单值分支则使用开头字母小写来表示.
    \item $\sqrt[n]{|z|}e^{i\arg z/n}$ 是 $n$ 值函数 $\sqrt[n]z$ 的一个单值分支, 其中 $z\neq 0$.
  \end{enumnopar}
\end{example}

在考虑多值的情况下, 复变函数总有反函数.
如果 $f$ 和 $f^{-1}$ 都是单值的, 则称 $f$ 是\nouns{一一对应}{复变函数!一一对应}.
\begin{example}
  $f(z)=z^n$ 的反函数就是 $f^{-1}(w)=\sqrt[n]{w}$.
  {当 $n=\pm1$ 时, $f$ 是一一对应.}
\end{example}
若无特别声明, 本书中\emph{复变函数总是指单值的复变函数}.


\subsection{映照}

大部分复变函数的图像无法在三维空间中表示出来.
为了直观理解和研究, 我们用 $z$ 复平面上的点表示自变量 $z$ 的值, $w$ 复平面上的点表示函数 $w=f(z)$ 的值.
然后用这两个复平面之间的变换或\noun{映照}来表示这种对应关系.
注意到 $w$ 的实部和虚部可以看作 $z$ 的实部和虚部的函数, 即
\[w=u+iv=u(x,y)+iv(x,y),\]
其中 $u,v$ 是两个二元实变函数.

\begin{example}
  函数 $w=\ov z$.
  如果把 $z$ 复平面和 $w$ 复平面重叠放置, 则这个映照对应的是关于 $z$ 轴的翻转变换.
  它把任一区域映成和它全等的区域, 且 $u=x,v=-y$.
\end{example}

\begin{center}
  \begin{tikzpicture}
    \begin{scope}[xshift=-25mm]
      \draw[cstaxis] (-2,0)--(2,0);
      \draw[cstaxis] (0,-1.5)--(0,1.5);
      \draw
        (2,0) node[above] {$x$}
        (0,1.5) node[left] {$y$}
        (0,-1.5) node[below] {$z$ 复平面};
      \draw[cstcurve,main,smooth] plot coordinates {(-1.5,0) (-1.7,-.4) (-.3,-.9) (.5,-.7) (.9,0) (1.1,1) (-.3,1.2) (-.7,1) (-1.5,0)};
      \coordinate (a) at (-1.2,-.3);
      \coordinate (b) at (.6,.9);
      \coordinate (c) at ($.8*(a)+.2*(b)$);
      \coordinate (d) at ($.5*(a)+.5*(b)$);
      \coordinate (e) at ($.2*(a)+.8*(b)$);
      \draw[cstcurve,main] (a)--(b);
    \end{scope}
    \begin{scope}[xshift=25mm]
      \draw[cstaxis] (-2,0)--(2,0);
      \draw[cstaxis] (0,-1.5)--(0,1.5);
      \draw
        (2,0) node[above] {$u$}
        (0,1.5) node[left] {$v$}
        (0,-1.5) node[below] {$w$ 复平面};
      \draw[cstcurve,second,smooth] plot coordinates {(-1.5,0) (-1.7,.4) (-.3,.9) (.5,.7) (.9,0) (1.1,-1) (-.3,-1.2) (-.7,-1) (-1.5,0)};
      \coordinate (A) at (-1.2,.3);
      \coordinate (B) at (.6,-.9);
      \coordinate (C) at ($.8*(A)+.2*(B)$);
      \coordinate (D) at ($.5*(A)+.5*(B)$);
      \coordinate (E) at ($.2*(A)+.8*(B)$);
      \draw[cstcurve,second] (A)--(B);
    \end{scope}
    \draw[cstdash,smooth,cstra] (c) to[bend right=15] (C);
    \draw[cstdash,smooth,cstra] (d) to[bend right=25] (D);
    \draw[cstdash,smooth,cstra] (e) to[bend left=45] (E);
    \fill[cstdot,main] (c) circle;
    \fill[cstdot,main] (d) circle;
    \fill[cstdot,main] (e) circle;
    \fill[cstdot,second] (C) circle;
    \fill[cstdot,second] (D) circle;
    \fill[cstdot,second] (E) circle;
  \end{tikzpicture}
\end{center}

\begin{example}
  函数 $w=az$.
  设 $a=re^{i\theta}$, 则这个映照对应的是一个旋转映照(逆时针旋转 $\theta$)和一个相似映照(放大为 $r$ 倍)的复合.
  它把任一区域映成和它相似的区域.
\end{example}

\begin{center}
  \begin{tikzpicture}
    \begin{scope}[xshift=-25mm]
      \draw[cstaxis] (-2,0)--(2,0);
      \draw[cstaxis] (0,-1.5)--(0,1.5);
      \draw
        (2,0) node[above] {$x$}
        (0,1.5) node[left] {$y$}
        (0,-1.5) node[below] {$z$ 复平面};
      \draw[cstcurve,main,smooth] plot coordinates {(-1.5,0) (-1.7,-.4) (-.3,-.9) (.5,-.7) (.9,0) (1.1,1) (-.3,1.2) (-.7,1) (-1.5,0)};
      \coordinate (a) at (-1.2,-.3);
      \coordinate (b) at (.6,.9);
      \coordinate (c) at ($.8*(a)+.2*(b)$);
      \coordinate (d) at ($.5*(a)+.5*(b)$);
      \coordinate (e) at ($.2*(a)+.8*(b)$);
      \draw[cstcurve,main] (a)--(b);
    \end{scope}
    \begin{scope}[xshift=25mm]
      \draw[cstaxis] (-2,0)--(2,0);
      \draw[cstaxis] (0,-1.5)--(0,1.5);
      \draw
        (2,0) node[above] {$u$}
        (0,1.5) node[left] {$v$}
        (0,-1.5) node[below] {$w$ 复平面};
      \draw[cstcurve,second,smooth,scale=.8,rotate=90] plot coordinates {(-1.5,0) (-1.7,-.4) (-.3,-.9) (.5,-.7) (.9,0) (1.1,1) (-.3,1.2) (-.7,1) (-1.5,0)};
      \coordinate (A) at (.24,-.96);
      \coordinate (B) at (-.72,.48);
      \coordinate (C) at ($.8*(A)+.2*(B)$);
      \coordinate (D) at ($.5*(A)+.5*(B)$);
      \coordinate (E) at ($.2*(A)+.8*(B)$);
      \draw[cstcurve,second] (A)--(B);
    \end{scope}
    \draw[cstdash,smooth,cstra] (c) to[bend right=25] (C);
    \draw[cstdash,smooth,cstra] (d) to[bend right=10] (D);
    \draw[cstdash,smooth,cstra] (e) to[bend left=20] (E);
    \fill[cstdot,main] (c) circle;
    \fill[cstdot,main] (d) circle;
    \fill[cstdot,main] (e) circle;
    \fill[cstdot,second] (C) circle;
    \fill[cstdot,second] (D) circle;
    \fill[cstdot,second] (E) circle;
  \end{tikzpicture}
\end{center}

\begin{example}\label{exam:wz2}
  函数 $w=z^2$.
  这个映照把 $z$ 的辐角增大一倍, 因此它会把角形区域变换为角形区域, 并将夹角放大一倍.
  
  由于 $u=x^2-y^2,v=2xy$.
  因此它把 $z$ 复平面上两族分别以直线 $y=\pm x$ 和坐标轴为渐近线的等轴双曲线 $x^2-y^2=c_1$, $2xy=c_2$分别映射为 $w$ 复平面上的两族平行直线 $u=c_1,v=c_2$.
\end{example}
  
\begin{center}
  \begin{tikzpicture}
    \begin{scope}[xshift=-25mm]
      \draw[cstaxis] (-2,0)--(2,0);
      \draw[cstaxis] (0,-1.5)--(0,1.5);
      \draw
        (2,0) node[above] {$x$}
        (0,1.5) node[left] {$y$}
        (0,-1.5) node[below] {$z$ 复平面};
      \fill[cstfille1] (0,0)--({1.5*cos(37.5)},{1.5*sin(37.5)}) arc(37.5:7.5:1.5)--cycle;
      \draw[cstcurve,main] (0,0)--({1.5*cos(37.5)},{1.5*sin(37.5)});
      \draw[cstcurve,main] (0,0)--({1.5*cos(7.5)},{1.5*sin(7.5)});
      \coordinate (a) at (0,1);
      \coordinate (b) at (.8,1.2);
      \coordinate (c) at (-.6,-.3);
    \end{scope}
    \begin{scope}[xshift=25mm]
      \draw[cstaxis] (-2,0)--(2,0);
      \draw[cstaxis] (0,-1.5)--(0,1.5);
      \draw
        (2,0) node[above] {$u$}
        (0,1.5) node[left] {$v$}
        (0,-1.5) node[below] {$w$ 复平面};
      \fill[cstfille2] (0,0)--({1.8*cos(75)},{1.8*sin(75)}) arc(75:15:1.8)--cycle;
      \draw[cstcurve,second] (0,0)--({1.8*cos(75)},{1.8*sin(75)});
      \draw[cstcurve,second] (0,0)--({1.8*cos(15)},{1.8*sin(15)});
      \coordinate (A) at (-1,0);
      \coordinate (B) at (-.8,1.92);
      \coordinate (C) at (.27,.36);
    \end{scope}
    \draw[cstdash,smooth,cstra] (a) to[bend left=10] (A);
    \draw[cstdash,smooth,cstra] (b) to[bend left=20] (B);
    \draw[cstdash,smooth,cstra] (c) to[bend right=25] (C);
    \fill[cstdot,fill=main] (a) circle;
    \fill[cstdot,fill=main] (b) circle;
    \fill[cstdot,fill=main] (c) circle;
    \fill[cstdot,fill=second] (A) circle;
    \fill[cstdot,fill=second] (B) circle;
    \fill[cstdot,fill=second] (C) circle;
  \end{tikzpicture}
\end{center}

\begin{center}
  \begin{tikzpicture}
    \begin{scope}[xshift=-25mm]
      \draw[cstaxis] (-2,0)--(2,0);
      \draw[cstaxis] (0,-1.5)--(0,1.5);
      \begin{scope}[cstcurve,main,smooth]
        \draw (-1.2,-1.2)--(1.2,1.2);
        \draw (-1.2,1.2)--(1.2,-1.2);
        \draw[domain=-35:35]
          plot ({sec(\x)},{tan(\x)})
          plot ({-sec(\x)},{tan(\x)})
          plot ({tan(\x)},{sec(\x)})
          plot ({tan(\x)},{-sec(\x)});
        \draw[domain=-46:46]
          plot ({(.8*sec(\x))},{0.8*tan(\x)})
          plot ({(-.8*sec(\x))},{0.8*tan(\x)})
          plot ({0.8*tan(\x)},{0.8*sec(\x)})
          plot ({0.8*tan(\x)},{0.8*-sec(\x)});
        \draw[domain=-57:57]
          plot ({(.6*sec(\x))},{0.6*tan(\x)})
          plot ({(-.6*sec(\x))},{0.6*tan(\x)})
          plot ({0.6*tan(\x)},{0.6*sec(\x)})
          plot ({0.6*tan(\x)},{0.6*-sec(\x)});
        \draw[domain=-68:68]
          plot ({(.4*sec(\x))},{0.4*tan(\x)})
          plot ({(-.4*sec(\x))},{0.4*tan(\x)})
          plot ({0.4*tan(\x)},{0.4*sec(\x)})
          plot ({0.4*tan(\x)},{0.4*-sec(\x)});
      \end{scope}
      \begin{scope}[cstcurve,smooth,rotate=45]
        \draw (-1.2,-1.2)--(1.2,1.2);
        \draw (-1.2,1.2)--(1.2,-1.2);
        \draw[domain=-35:35]
          plot ({sec(\x)},{tan(\x)})
          plot ({-sec(\x)},{tan(\x)})
          plot ({tan(\x)},{sec(\x)})
          plot ({tan(\x)},{-sec(\x)});
        \draw[domain=-46:46]
          plot ({(.8*sec(\x))},{0.8*tan(\x)})
          plot ({(-.8*sec(\x))},{0.8*tan(\x)})
          plot ({0.8*tan(\x)},{0.8*sec(\x)})
          plot ({0.8*tan(\x)},{0.8*-sec(\x)});
        \draw[domain=-57:57]
          plot ({(.6*sec(\x))},{0.6*tan(\x)})
          plot ({(-.6*sec(\x))},{0.6*tan(\x)})
          plot ({0.6*tan(\x)},{0.6*sec(\x)})
          plot ({0.6*tan(\x)},{0.6*-sec(\x)});
        \draw[domain=-68:68]
          plot ({(.4*sec(\x))},{0.4*tan(\x)})
          plot ({(-.4*sec(\x))},{0.4*tan(\x)})
          plot ({0.4*tan(\x)},{0.4*sec(\x)})
          plot ({0.4*tan(\x)},{0.4*-sec(\x)});
      \end{scope}
    \end{scope}
    \begin{scope}[xshift=25mm]
      \draw[cstaxis] (-2,0)--(2,0);
      \draw[cstaxis] (0,-1.5)--(0,1.5);
      \begin{scope}[cstcurve]
        \draw (-1.3,-1.2)--(1.3,-1.2);
        \draw (-1.3,-0.9)--(1.3,-.9);
        \draw (-1.3,-0.6)--(1.3,-.6);
        \draw (-1.3,-0.3)--(1.3,-.3);
        \draw (-1.3,0)--(1.3,0);
        \draw (-1.3,0.3)--(1.3,.3);
        \draw (-1.3,0.6)--(1.3,.6);
        \draw (-1.3,0.9)--(1.3,.9);
        \draw (-1.3,1.2)--(1.3,1.2);
      \end{scope}
      \begin{scope}[cstcurve,main,rotate=90]
        \draw (-1.3,-1.2)--(1.3,-1.2);
        \draw (-1.3,-0.9)--(1.3,-.9);
        \draw (-1.3,-0.6)--(1.3,-.6);
        \draw (-1.3,-0.3)--(1.3,-.3);
        \draw (-1.3,0)--(1.3,0);
        \draw (-1.3,0.3)--(1.3,.3);
        \draw (-1.3,0.6)--(1.3,.6);
        \draw (-1.3,0.9)--(1.3,.9);
        \draw (-1.3,1.2)--(1.3,1.2);
      \end{scope}
    \end{scope}
  \end{tikzpicture}
\end{center}

\begin{example}
  求下列集合在映照 $w=z^2$ 下的像.
  \begin{enumpar}[(i)]
    \item 线段 $0<|z|<2,\arg z=\dfrac\pi2$.
    \item 双曲线 $x^2-y^2=4$.
    \item 扇形区域 $0<\arg z<\dfrac\pi4,0<|z|<2$.
  \end{enumpar}
\end{example}
\begin{solution}\delspace
  \begin{enumnopar}[(i)]
    \item 设 $z=re^{\frac{\pi i}2}=ir$, 则 $w=z^2=-r^2$.
      因此它的像还是一条线段
      \[0<|w|<4,\arg w=-\pi.\]
    \item 由于
      \[w=u+iv=z^2=(x^2-y^2)+2xyi.\]
      因此 $u=x^2-y^2=4,v=2xy$.
      由于任何复数均存在平方根, 存在 $z=x+yi$ 使得 $f(z)=4+iv$, 其中 $x^2-y^2=4$.
      因此这条双曲线的像的确就是直线 $\Re w=4$.\footnote{部分文献往往会忽略检查所给的集合中的每个元素都有原像.}
    \item 设 $z=re^{i\theta}$, 则 $w=r^2e^{2i\theta}$.
      因此它的像是扇形区域
      \[0<\arg w<\dfrac\pi2,0<|w|<4.\]
  \end{enumnopar}
\end{solution}

\begin{example}
  求圆周 $|z|=2$ 在映照 $w=\dfrac{z+1}{z-1}$ 下的像.
\end{example}

\begin{solution}
  由于 $z=\dfrac{w+1}{w-1}$, $\abs{\dfrac{w+1}{w-1}}=2$,
  {因此
  \[|w+1|=2|w-1|,\quad w\ov w+w+\ov w+1=4w\ov w-4w-4\ov w+4,\]}
  {
    \[w\ov w-\frac53 w-\frac53\ov w+1=0,\quad \abs{w-\frac53}^2=\dfrac{16}9,\]即 $\abs{w-\dfrac53}=\dfrac43$, 是一个圆周.}
\end{solution}
这种形如 $\dfrac{az+b}{cz+d}$ 的映射叫作\noun{分式线性变换}或\noun{莫比乌斯变换}\footnote{
  August Ferdinand M\"obius (1790--1868), 德国数学家、天文学家.
}, 它总把直线和圆映为直线或圆.


\section{极限和连续性}

类似于实数的情形, 我们可以定义复数列和复变函数的极限.

\subsection{数列的极限}

\begin{definition}
  设 $\{z_n\}_{n\ge 1}$ 是一个复数列.
  如果存在复数 $z$ 满足: $\forall \varepsilon>0,\exists N$ 使得当 $n\ge N$ 时, $|z_n-z|<\varepsilon$, 则称 $z$ 是\nouns{数列 $\{z_n\}$ 的极限}{极限!复数列的极限}, 记作 \noun{$\lim\limits_{n\to\infty}z_n=z$}.
\end{definition}
此时称\nouns{极限存在}{极限!极限存在}或\nouns{数列收敛}{收敛!数列收敛}.
若不存在这样的 $z$, 则称\nouns{极限不存在}{极限!极限不存在}或\nouns{数列发散}{发散!数列发散}.
可以看出, $\lim\limits_{n\to\infty}z_n=z$ 等价于实极限 $\lim\limits_{n\to\infty}|z_n-z|=0$.

由于复数列极限的定义和实数列极限的定义在形式上完全相同, 因此类似于高等数学中的情形, 极限的四则运算法则对于复数列也是成立的.
\begin{theorem}
  设 $\lim\limits_{n\to\infty}z_n=z,\lim\limits_{n\to\infty}w_n=w$, 则
  \begin{enumpar}
    \item $\lim\limits_{n\to\infty}(z_n\pm w_n)=z\pm w$;
    \item $\lim\limits_{n\to\infty} z_nw_n=zw$;
    \item 当 $w\neq 0$ 时, $\lim\limits_{n\to\infty}\dfrac{z_n}{w_n}=\dfrac zw$.
  \end{enumpar}
\end{theorem}

下述定理保证了我们可以使用实数列的敛散性判定技巧.
\begin{theorem}
  设 $z_n=x_n+y_ni,z=x+yi$, 则
  \[\lim_{n\to\infty}z_n=z\iff \lim_{n\to\infty}x_n=x,\lim_{n\to\infty}y_n=y.\]
\end{theorem}

\begin{proof}
  由三角不等式
  \[|x_n-x|,|y_n-y|\le|z_n-z|\le|x_n-x|+|y_n-y|,\]
  夹逼准则和极限的四则运算法则可知
  \[\lim_{n\to\infty}|z_n-z|=0\iff
  \lim_{n\to\infty}|x_n-x|=0,\lim_{n\to\infty}|y_n-y|=0.\]
  由此定理得证.
\end{proof}

\begin{example}
  设 $z_n=\left(1+\dfrac1n\right)e^{\frac{\pi i}n}$. 数列 $\{z_n\}$ 是否收敛?
\end{example}

\begin{solution}
  由于
  \[x_n=\left(1+\frac1n\right)\cos\frac\pi n\to 1,\quad
  y_n=\left(1+\frac1n\right)\sin\frac\pi n\to 0.\]
  因此 $\{z_n\}$ 收敛且 $\lim\limits_{n\to\infty}z_n=1$.
\end{solution}


\subsection{无穷远点和复球面}


若 $\lim\limits_{n\to\infty}|z_n|=+\infty$, 则我们可将其记为 \nouns{$\lim\limits_{n\to\infty}z_n=\infty$}.
这也等价于: $\forall X>0,\exists N$ 使得当 $n\ge N$ 时 $|z_n|>X$.
如果称
  \[\Uc(\infty,X)=\{z\in\BC:|z|>X\}\]
为 $\infty$ 的\nouns{去心 $X$ 邻域}{去心邻域},
那么上述极限可统一表述为:

\begin{definition}
  $\lim\limits_{n\to\infty}z_n=z\in\BC\cup\set\infty$ 是指: 对 $z$ 的任意邻域 $U$, $\exists N$ 使得当 $n\ge N$ 时 $z_n\in U$.\footnotemark
\end{definition}
\footnotetext{一般地, 一个点的邻域是指包含它的任意一个开集.
可以说明, 把这里的任意 $\delta$ 邻域换成任意邻域, 并不会改变定义, 因为包含 $z$ 的开集一定也包含 $z$ 的 一个 $\delta$ 邻域.}


\begin{figure}[!h]
  \centering
  \begin{tikzpicture}
    \fill[cstfille1] (0,0) circle (1.2);
    \filldraw[cstcurve,main,fill=white] (0,0) circle (.5);
    \draw[cstaxis] (-1.5,0)--(1.5,0);
    \draw[cstaxis] (0,-1.5)--(0,1.5);
  \end{tikzpicture}
  \caption{$\infty$ 的(去心)邻域}
\end{figure}

那么有没有一种看法使得 $\infty$ 的邻域和普通复数的邻域没有差异呢?
我们将介绍复球面的概念, 它是复数的一种几何表示且自然包含无穷远点 $\infty$.
这种思想是在黎曼\footnote{
  Georg Friedrich Bernhard Riemann (1826--1866), 德国数学家.
}研究多值复变函数时引入的.

\begin{figure}[!h]
  \centering
  \begin{tikzpicture}
    \fill[cstfill1] (-3.65,-.804)--(-1.85,.804)--(3.65,.804)--(1.85,-.804)--cycle;
    \filldraw[cstcurve,cstfill] (0,1) circle (1);
    \draw[cstdash] (0,1) circle (1 and 0.3);
    \draw[cstdash] (0,0) circle (2 and 0.6);
    \coordinate [label=above:\textcolor{third}{$N$}] (N) at (0,2);
    \draw[cstdash] (0,0)--(N);
    \draw[cstaxis] (0,0)--(2.5,0);
    \draw[cstaxis] (0,0)--(-.8,-.9);
    \coordinate [label=right:{$z_1$}] (z1) at (1.65,-.75);
    \coordinate [label=left:{$Z_1$}] (Z1) at (.6,1);
    \coordinate [label=left:{$z_2$}] (z2) at (-1,0);
    \coordinate [label=below right:{$Z_2$}] (Z2) at (-.7,.6);
    \draw[cstcurve,main,cstra] (N)--(z1);
    \fill[cstdot,main] (Z1) circle;
    \draw[cstcurve,cstra,second] (N)--(z2);
    \fill[cstdot,second] (Z2) circle;
    \fill[cstdot] (N) circle;
  \end{tikzpicture}
  \caption{复球面和复平面}
\end{figure}

在三维空间中复平面上方取一个与复平面相切于原点 $O$ 的球面.
过 $O$ 做垂直于复平面的直线, 并与球面相交于另一点 $N$, 称之为北极.
\begin{itemize}
  \item 对于平面上的任意一点 $z$, 连接北极 $N$ 和 $z$ 的直线一定与球面相交于除 $N$ 以外的唯一一个点 $Z$.
  \item 反之, 球面上除了北极外的任意一点 $Z$, 直线 $NZ$ 一定与复平面相交于唯一一点.
\end{itemize}
这样, 球面上除北极外的所有点和全体复数建立了一一对应.

当 $|z|$ 越来越大时, 其对应球面上点也越来越接近 $N$.
如果我们在复平面上添加一个额外的``点''——\noun{无穷远点}, 记作 \noun{$\infty$}.
那么\noun{扩充复数集合 $\BC^*=\BC\cup\set\infty$} 就正好和球面上的点一一对应.
称这样的球面为\noun{复球面}, 称包含无穷远点的复平面为\noun{扩充复平面}或\noun{闭复平面}.
自然地, 称
\[U(\infty,X)=\{z\in\BC:|z|>X\}\cup\{\infty\}\subseteq\BC^*\]
为 $\infty$ 的 \noun{$X$ 邻域}{邻域!$X$ 邻域}.
% 它和实数列极限符号中的 $\infty$ 有什么联系呢?
% 选取上述图形的一个截面来看, 实轴可以和圆周去掉一点建立一一对应.
% 于是实数列极限符号中的 $\infty$ 在复球面上就是 $\infty$.

% \begin{figure}[!h]
%   \centering
%   \begin{tikzpicture}
%     \filldraw[cstcurve,cstfill] (0,1) circle (1);
%     \coordinate [label=above:\textcolor{third}{$N$}] (N) at (0,2);
%     \draw[cstdash] (0,0)--(N);
%     \draw[cstaxis] (-2,0)--(2.5,0);
%     \coordinate [label=below:\textcolor{main}{$x_1$}] (x1) at (2.2,0);
%     \coordinate [label=above right:\textcolor{main}{$X_1$}] (X1) at (1,1.1);
%     \coordinate [label=below:\textcolor{second}{$x_2$}] (x2) at (-1,0);
%     \coordinate [label=left:\textcolor{second}{$X_2$}] (X2) at (-.8,.4);
%     \draw[cstcurve,cstra,main] (0,2)--(x1);
%     \fill[cstdot,main] (X1) circle;
%     \draw[cstcurve,cstra,second] (0,2)--(x2);
%     \fill[cstdot,second] (X2) circle;
%     \fill[cstdot] (N) circle;
%   \end{tikzpicture}
%   \caption{圆周和实轴}
% \end{figure}

朴素地看, 复球面上任意一点可以定义 $\delta$ 邻域为与其距离小于 $\delta$ 的所有点.
特别地, $\infty$ 的去心邻域通过前面所说的对应关系, 可以对应到扩充复平面上 $\infty$ 的一个去心邻域\footnote{
  对于有限复数 $z$, 球面上 $z$ 的 $\delta$ 邻域对应到复平面上一包含 $z$ 的开集.
}.
所以在复球面上, 普通复数和 $\infty$ 的邻域具有同等地位.


\subsection{函数的极限}

\begin{definition}
  设复变函数 $f(z)$ 在 $z_0$ 的某个去心邻域内有定义.
  如果存在复数 $A$ 满足: $\forall\varepsilon>0$, $\exists \delta>0$ 使得当 $0<|z-z_0|<\delta$ 时, $|f(z)-A|<\varepsilon$, 则称 $A$ 为 \nouns{$f(z)$ 当 $z\to z_0$ 时的极限}{极限!函数的极限}, 记为 \noun{$\lim\limits_{z\to z_0}f(z)$}$=A$ 或 \noun{$f(z)\to A\ (z\to z_0)$}.
\end{definition}
此时称\nouns{极限存在}{极限!极限存在}.
若不存在这样的 $A$, 则称\nouns{极限不存在}{极限!极限不存在}.

该定义若用邻域的语言可重新表述为:
\begin{definition}
  设函数 $f(z)$ 在 $z_0$ 的某个去心邻域内有定义.
  如果存在复数 $A$ 满足: 对 $A$ 的任意邻域 $V$, 存在 $z_0$ 的去心邻域 $U^\circ$ 使得当 $z\in\Uc$ 时, $f(z)\in V$, 则称 $A$ 为 \nouns{$f(z)$ 当 $z\to z_0$ 时的极限}{极限!函数的极限}, 记为 \nouns{$\lim\limits_{z\to z_0}f(z)$}{limzz0f(z)A@$\lim\limits_{z\to z_0}f(z)$}$=A$ 或 \emph{$f(z)\to A (z\to z_0)$}.
\end{definition}
在此表述下, 将上述定义中的 $z_0$ 或 $A$ 换成 $\infty$, 即可得到 $z\to\infty$ 的极限定义, 以及 $\lim f(z)=\infty$ 的含义.

类似于数列情形, 极限的四则运算法则对于复变函数也是成立的.
\begin{theorem}\label{thm:limfunctionfour}
  设 $\lim\limits_{z\to z_0}f(z)=A,\lim\limits_{z\to z_0}g(z)=B$, 则
  \begin{enumpar}
    \item $\lim\limits_{z\to z_0}(f\pm g)(z)=A\pm B$;
    \item $\lim\limits_{z\to z_0}(fg)(z)=AB$;
    \item 当 $B\neq 0$ 时, $\lim\limits_{z\to z_0}\left(\dfrac fg\right)(z)=\dfrac AB$.
  \end{enumpar}
  这里, $z_0\in\BC^*$.
\end{theorem}

下述定理表明, 研究复变函数极限, 只需研究其实部、虚部两个二元实函数的极限.
\begin{theorem}\label{thm:limfunction}
  设 $f(z)=u(x,y)+iv(x,y),z_0=x_0+y_0i,A=u_0+v_0i$, 则
  \[\lim_{z\to z_0}f(z)=A\iff
  \lim_{\substack{x\to x_0\\y\to y_0}}u(x,y)=u_0,\quad
  \lim_{\substack{x\to x_0\\y\to y_0}}v(x,y)=v_0.\]
\end{theorem}

\begin{proof}
  由于
  \[\lim_{z\to z_0}f(z)=A\iff
  \lim_{\substack{x\to x_0\\y\to y_0}}|u(x,y)+iv(x,y)-u_0-iv_0|=0.\]
  因此与数列情形类似, 由三角不等式
  \[|u-u_0|,|v-v_0|\le|f(z)-A|\le|u-u_0|+|v-v_0|,\]
  夹逼准则和极限的四则运算法则可得该定理.
\end{proof}

在学习了复变函数的导数后, 我们也可以使用等价无穷小替换、洛必达法则等工具来计算极限. 相应结论证明与高等数学中相应结论证明类似, 此处不再赘述.

\begin{example}
  证明: 当 $z\to0$ 时, 函数 $f(z)=\dfrac{\Re z}{|z|}$ 的极限不存在.
\end{example}

\begin{proof}
  令 $z=x+yi$, 则 $f(z)=\dfrac x{\sqrt{x^2+y^2}}$.
  因此
    \[u(x,y)=\frac x{\sqrt{x^2+y^2}},\quad v(x,y)=0.\]
  当 $z$ 在实轴原点两侧分别趋向于 $0$ 时, $u(x,y)\to\pm1$.因此 $\lim\limits_{\substack{x\to 0\\y\to0}}u(x,y)$ 不存在,从而 $\lim\limits_{z\to z_0}f(z)$ 不存在.
\end{proof}


\subsection{函数的连续性}

有了复变函数的极限之后, 自然地可定义复变函数的连续性.
\begin{definition}
  \begin{enumpar}
    \item 如果 $\lim\limits_{z\to z_0}f(z)=f(z_0)$, 则称 $f(z)$ 在 \nouns{$z_0$ 处连续}{连续}.
    \item 如果 $f(z)$ 在区域 $D$ 内处处连续, 则称 $f(z)$ 在 \nouns{$D$ 内连续}{连续}.
  \end{enumpar}
\end{definition}

由定理~\ref{thm:limfunction} 可知:
\begin{theorem}
  函数 $f(z)=u(x,y)+iv(x,y)$ 在 $z_0=x_0+iy_0$ 处连续当且仅当 $u(x,y)$ 和 $v(x,y)$ 在 $(x_0,y_0)$ 处连续.
\end{theorem}

由于复变函数极限和连续的定义和实变量函数情形在形式上完全相同, 因此类似地有:
\begin{theorem}
  \begin{enumpar}
    \item 在 $z_0$ 处连续的两个函数 $f(z)$, $g(z)$ 之和、差、积、商($g(z_0)\neq 0$) 仍然在 $z_0$ 处连续.
    \item 如果函数 $g(z)$ 在 $z_0$ 处连续, 函数 $f(w)$ 在 $g(z_0)$ 处连续, 则 $f\bigl(g(z)\bigr)$ 在 $z_0$ 处连续.
  \end{enumpar}
\end{theorem}

\begin{example}\delspace
  \begin{enumnopar}
    \item 设
    \[f(z)=\ln(x^2+y^2)+i(x^2-y^2),\]
    则 $u(x,y)=\ln(x^2+y^2)$ 除原点外处处连续, $v(x,y)=x^2-y^2$ 处处连续. 因此 $f(z)$ 在 $z\neq0$ 处连续.
    \item 显然 $f(z)=z$ 是处处连续的, 故多项式函数
    \[P(z)=a_0+a_1z+a_2z^2+\cdots+a_nz^n\]
    也处处连续, 有理函数 $\dfrac{P(z)}{Q(z)}$ 在 $Q(z)$ 的零点以外处处连续.
  \end{enumnopar}
\end{example}

\begin{example}
  证明: 如果 $f(z)$ 在 $z_0$ 连续, 则 $\ov{f(z)}$ 在 $z_0$ 也连续.
\end{example}

\begin{proof}
  设 $f(z)=u(x,y)+iv(x,y),z_0=x_0+iy_0$.
  那么 $u(x,y),v(x,y)$ 在 $(x_0,y_0)$ 连续.从而 $-v(x,y)$ 也在 $(x_0,y_0)$ 连续.所以 $\ov{f(z)}=u(x,y)-iv(x,y)$ 在 $(x_0,y_0)$ 连续.
\end{proof}
\begin{proof}[另证]
  函数 $g(z)=\ov z=x-iy$ 处处连续,从而 $g\bigl(f(z)\bigr)=\ov{f(z)}$ 在 $z_0$ 处连续.
\end{proof}

可以看出, 由于 $\BC$ 和 $\BR$ 类似, 也是一个域, 所以复变函数与单变量实函数的极限和连续具有类似形式与结论.
另一方面, 由于 $\BC$ 和 $\BR^2$ 一一对应, 复变函数与两个二元实函数的极限和连续也没有太大区别.
这使得复变函数一方面具有和单变量实函数类似的导数和微分, 另一方面它的实部和虚部又具有偏导数和全微分.
这种特性使得可导的复变函数具有整洁优美的性质, 我们将在接下来的部分逐步揭开它的神秘面纱.




\sectionHomework
\begin{homework}
  \item 选择题.
  \begin{exlist}
    \item 以下哪一条不是 $z$ 是实数的充要条件?\fillbrace{}
    \begin{taskschoice}(4)
      \item $\Im z=0$
      \item $\arg z=0,\pi$
      \item $z=\ov z$
      \item $z^2=|z|^2$.
    \end{taskschoice}
    \item 以下哪一条是 $z$ 是纯虚数的充要条件?\fillbrace{}
    \begin{taskschoice}(4)
      \item $\Re z=0$
      \item $\arg z=\pm\dfrac\pi2$
      \item $z=-\ov z$
      \item $z^2=-|z|^2$.
    \end{taskschoice}
    \item 以下哪一个函数不是其定义域上的连续函数?\fillbrace{}
    \begin{taskschoice}(4)
      \item $\arg z$
      \item $|z|$
      \item 多项式
      \item $\ln(z\ov z)$.
    \end{taskschoice}
    \item 若 $f(z)$ 在 $0$ 处连续且 $f(0)=0$, $g(z)$ 在 $0$ 处连续, 则下述命题未必成立的是\fillbrace{}.
    \begin{taskschoice}(2)
      \item $f(z)+g(z)$ 在 $0$ 处连续
      \item $f(z)g(z)$ 在 $0$ 处连续
      \item $f\bigl(g(z)\bigr)$ 在 $0$ 处连续
      \item $g\bigl(f(z)\bigr)$ 在 $0$ 处连续
    \end{taskschoice}
    \item 下述图形不是圆周的是\fillbrace{}.
    \begin{taskschoice}(2)
      \item $z\ov z-(2+i)z-(2-i)\ov z=4$
      \item $\Re\dfrac{1}{z}=\dfrac{1}{9}$
      \item $\arg\dfrac{z-1}{z+1}=\dfrac{\pi}{3}$
      \item $z=i+2e^{i\theta},\theta\in[0,2\pi]$
    \end{taskschoice}
    \item 下述命题错误的是\fillbrace{}.
    \begin{taskschoice}(2)
      \item $|z|=\Re z+1$ 是抛物线
      \item $|z+i|=|z-i|$ 是直线
      \item $|z+i|-|z-i|=1$ 是双曲线
      \item $|z|+|z-2i|=2$ 是一条线段
    \end{taskschoice}
    \item \fillbrace{}是多连通区域的.
    \begin{taskschoice}(2)
      \item $-1<\arg z<\pi-1$
      \item $\arg z<\dfrac{3\pi}4$
      \item $1<|z|<2$
      \item $0<\Re z<1$
    \end{taskschoice}
    \item \fillbrace{}是有界的.
    \begin{taskschoice}(1)
      \item $z\ov z-(2+i)z-(2-i)\ov z\le 4$
      \item $\Im z\le0,\Re z\ge0$ 是无界单连通闭区域
      \item $|z-1|<|z+3|$ 是无界单连通区域
      \item $\Bigl|\dfrac{z+1}{z-1}\Bigr|<2$ 是无界多连通区域
    \end{taskschoice}
  \end{exlist}
  \item 填空题.
  \begin{exlist}
    \item 如果 $x,y$ 是实数且 $\dfrac{x+1+i(y-3)}{5+3i}=1+i$, 那么 $x+y=$\fillblank{}.
    \item 设 $z=-i$, 则 $1+z+z^2+z^3+z^4=$\fillblank{}.
    \item 化简 $\dfrac{(1+i)^{101}}{(1-i)^{99}}=$\fillblank{}.
    \item 化简 $(-1+i)^{10}-(-1-i)^{10}$=\fillblank{}.
    \item $\biggl(\dfrac{(1+i)^2}2\biggr)^{21}$ 的模是\fillblank{}.
    \item $z=\dfrac{3+i}{i}-\dfrac{10i}{3-i}$ 的模是\fillblank{}.
    \item $-1-i$ 的辐角是\fillblank{}.
    \item $-1+\sqrt 3i$ 的辐角主值是\fillblank{}.
    \item $2023-i$ 绕 $0$ 逆时针旋转 $\dfrac\pi2$ 后得到的复数是\fillblank{}.
    \item $8-i$ 绕 $0$ 逆时针旋转 $\dfrac\pi2$ 后得到的复数是\fillblank{}.
    \item 区域 $0<\arg z<\dfrac\pi3$ 在映射 $w=z^3$ 下的像是\fillblank[12em]{}.
    \item 已知映射 $w=z^3$, 则 $z=\sqrt3+i$ 在 $w$ 复平面上的像是\fillblank{}.
    \item 极限 $\displaystyle\lim_{z\to1+i}(1+z^2+2z^4)=$\fillblank{}.
    \item 极限 $\displaystyle\lim_{z\to\infty}\dfrac{2z^2-1}(z^2+z+1)=$\fillblank{}.
  \end{exlist}
  \item 计算题.
  \begin{exlist}
    \item $z_1=-z,z_2=\ov z,z_3=-\ov{z}$ 在复平面上对应的点分别与 $z$ 在复平面上对应的点是什么关系?
    \item 已知点 $z_1,z_2,z_3$ 不共线. 点 $\dfrac12(z_1+z_2)$ 和 $\dfrac13(z_1+z_2+z_3)$ 表示什么点?
    \item 设正方形的两个顶点分别为复数 $a$, $b$, 求它的其它两个顶点的所有可能.
    \item 求下列复数 $z$ 的实部与虚部, 共轭复数, 模和辐角主值:
    \begin{tasksexer}(2)
      \item $\dfrac{5+i}{2+3i}$;
      \item $\dfrac{3i}{1-i}-\dfrac1i$;
      \item $\dfrac{(3+4i)(2-5i)}{2i}$;
      \item $i^8-4i^{21}+i$.
    \end{tasksexer}
    \item 求下列复数 $z$ 的三角和指数形式:
    \begin{tasksexer}(3)
      \item $i$;
      \item $3-\sqrt 3i$;
      \item $\dfrac{2i}{1-i}$;
      \item $\ov{\biggl(\dfrac{4+3i}{1+2i}\biggr)}$;
      \item $\dfrac{3+i}{i}-\dfrac{10i}{3-i}$;
      \item $\dfrac{(\cos \varphi+i\sin \varphi)^5}{(\cos \varphi-i\sin \varphi)^4}$.
    \end{tasksexer}
    \item 计算
    \begin{tasksexer}(3)
      \item $(\sqrt3-i)^{-5}$;
      \item $\sqrt[3]{-8}$;
      \item $\sqrt[4]{-2+2i}$;
      \item $(-4-4i)^{1/5}$;
      \item $\sqrt[4]{-2}$;
      \item $(1-i)^{1/3}$.
    \end{tasksexer}
    \item 用复参数方程表示连接 $-1+i$ 与 $1-4i$ 的直线段.
    \item 将方程 $x^2+6x+y^2-18y=10$ 化为复数形式.
    \item 下列数列 $\set{z_n}$ 是否收敛? 如果收敛, 求出它们的极限:
    \begin{tasksexer}(2)
      \item $z_n=\dfrac{1+ni}{1-ni}$;
      \item $\displaystyle z_n=\bigl(1+\frac i2\bigr)^n$;
      \item $z_n=(-1)^n+\dfrac{i}{n+1}$;
      \item $z_n=\dfrac{(3+2i)^n}{(3+4i)^n}$;
      \item $z_n=\dfrac{n^2+in}{n^2-in}$;
      \item $z_n=\bigl(1+\dfrac{(-1)^n}n\bigr)e^{-\frac{n\pi i}2}$.
    \end{tasksexer}
    \item 构造一个在 $0$ 以外都有定义的函数 $f(z)$, 使得对任意非零复数 $z$, 极限 $\lim\limits_{t\to0}f(tz)$ 存在, 但 $\lim\limits_{z\to0}f(z)$ 不存在.
  \end{exlist}
  \item 证明题.
  \begin{exlist}
    \item 证明: 当 $|z|=1>|w|$ 时, $\bigl|\dfrac{z-w}{1-z\ov w}\bigr|=1$.
    \item 证明: 设 $|a|<1$. 证明 $|z|=1$ 当且仅当 $|z-a|=|1-\ov az|$.
    \item 证明: 如果复数 $a+ib$ 是实系数方程
      \[f(z)=a_0z^n+a_1z^{n-1}+\cdots+a_{n-1}z+a_n=0\]
      的根, 则 $a-ib$ 也是它的根.
    \item 设 $z=e^{it}$, 证明:\begin{tasksexam}(2)
      \item $z^n+\dfrac1{z^n}=2\cos{nt}$;
      \item $z^n-\dfrac1{z^n}=2i\sin{nt}$.
    \end{tasksexam}
    \item 设 $\dfrac{z_2-z_1}{z_3-z_1}=\dfrac{z_1-z_3}{z_2-z_3}$. 证明: $|z_1-z_2|=|z_2-z_3|=|z_3-z_1|$ 并说明这些等式的几何意义.
    \item 证明: 如果 $z_1+z_2+z_3=0$ 且 $|z_1|=|z_2|=|z_3|=1$, 则 $z_1,z_2,z_3$ 构成一个正三角形, 且单位圆(圆心为 $0$, 半径为 $1$ 的圆)是它的外接圆.
    \item 证明: $|z_1+z_2|^2+|z_1-z_2|^2=2(|z_1|^2+|z_2|^2)$ 并说明其几何意义.
    \item 证明: $f(z)=z{\ov z}^{-1}-\ov zz^{-1}$ 在 $z\ra 0$ 时极限不存在.
  \end{exlist}
\end{homework}



\newpage
\section{扩展阅读\optional}

\subsection{复数域的另一种构造方式}

我们已经在 \ref{sssec:complex-field}中了解到, 域 $F$ 是一个拥有如下要素
\begin{enumerate}
  \item 一个特定的元素 $0\in F$;
  \item 一个特定的元素 $1\in F$;
  \item 加法: 对任意 $a,b\in V$, 可作运算 $a+b\in F$;
  \item 数乘: 对任意 $a,b\in V$, 可作运算 $a\cdot b\in F$,
\end{enumerate}
且满足特定性质的集合.
如果 $F$, $K$ 是两个域, 映射 $f:F\to K$ 满足
\begin{itemize}
  \item $f(0)=0,\quad f(1)=1$ (注意两组 $0,1$ 分别位于 $F$ 和 $K$ 中);
  \item $\forall a,b\in F$, $f(a+b)=f(a)+f(b)$;
  \item $\forall a,b\in F$, $f(ab)=f(a)f(b)$,
\end{itemize}
则称 $f$ 是一个\emph{同态}.
如果 $f$ 还是一一对应, 则称 $f$ 是一个\emph{同构}.

\begin{exlist}
  \item 设 $f:\BQ\to\BQ$ 是有理数域之间的同构, 证明 $f$ 只能是恒等映射.
  \item 设 $f:\BR\to\BR$ 是实数域之间的连续的同构, 证明 $f$ 只能是恒等映射.
  \item 设 $f:\BC\to\BC$ 是复数域之间的连续的同构, 证明 $f$ 只能是恒等映射或复数共轭.
  这说明共轭是除了恒等映射外, 唯一一个与四则运算交换的 $\BC$ 到自身的连续双射, 它的重要性可见一斑.
  \item 设 $\bfE=\begin{pmatrix}1& \\&1\end{pmatrix}$,
  $\bfJ=\begin{pmatrix}&-1\\1&\end{pmatrix}$,
  \[F=\set{\begin{pmatrix}
    x&-y\\y&x\end{pmatrix}\;\Big|\; x,y\in\BR}
    =\set{x\bfE+y\bfJ\mid x,y\in\BR}\subseteq M_2(\BR).\]
  证明 $F$ (以及零矩阵 $\bfO$、单位阵 $\bfE$、矩阵加法和矩阵乘法) 是一个域 (需要验证\ref{sssec:complex-field}中的两个脚注).
  \item 设
  \begin{align*}
    f:\BC&\lra F\\
    x+yi&\lto x\bfE+y\bfJ.
  \end{align*}
  证明 $f$ 是域同构.
  可以看出 $f(\ov z)=f(z)^\rmT=f(z)^*$, 即复共轭对应矩阵转置(也是伴随), 而且矩阵 $f(z)$ 对应的线性变换将平面向量逆时针旋转 $\theta$ 并将模长放缩至 $r$ 倍.
\end{exlist}




\subsection{四元数}

我们知道, 用复数的乘法可以表示平面的旋转.
那么如何表示三维空间的旋转呢?
为此, 哈密尔顿\footnote{
  Sir William Rowan Hamilton (1805--1865), 爱尔兰数学家、物理学家、天文学家.}发明了\emph{四元数}:
\[\BH=\{a_0+a_1i+a_2j+a_3k\mid a,b,c,d\in\BR\}.\]
$\BH$ 上可以自然定义加法, 而它的乘法由如下规律所决定
\[i^2=j^2=k^2=-1,\quad ij=-ji=k,jk=-kj=i,ki=-ik=j.\]
这些运算满足除乘法交换律外的域的各种性质.

当 $a_0=0$ 时, $A=a_1i+a_2j+a_3k$ 对应一个三维向量 $(a_1,a_2,a_3)$.
因此我们可以用四元数来表示三维向量, 并把四元数分为标量部分与向量部分之和 $A=a_0+\bfa$.
\begin{exlist}
  \item 对于三维向量 $\bfa=a_1i+a_2j+a_3k, \bfb=b_1i+b_2j+b_3k$, 定义数量积
  \[\bfa\cdot\bfb=a_1b_1+a_2b_2+a_3b_3\]
  和向量积
  \[\bfa\times\bfb=\begin{vmatrix}
    i&j&k\\
    a_1&a_2&a_3\\
    b_1&b_2&b_3
  \end{vmatrix}.\]
  证明 $A=a_0+\bfa$ 与 $B=b_0+\bfb$ 的乘积为
  \[AB=(a_0b_0-\bfa\cdot\bfb)+(a_0\bfb+b_0\bfa+\bfa\times\bfb).\]
  \item 四元数上也可以定义共轭: $\tau(a_0+\bfa)=a_0-\bfa$. 
  \begin{exlist}
    \item 证明: $\tau(AB)=\tau(A)\tau(B)$.
    \item 证明: $\tau\bigl(\tau(A)\bigr)=A$ 且 $A\tau(A)=\tau(A)A$ 是非负实数. $A\tau(A)=0$ 当且仅当 $A=0$. 
    \item 证明: 对于任意非零四元数 $A$, 存在四元数 $B$ 使得 $AB=BA=1$. 
  \end{exlist}
  \item 定义 \emph{Nabla 算子}
  \[\nabla=i\pp{}{x}+j\pp{}{y}+k\pp{}{z}.\]
  我们仿照着前述四元数的运算来定义 $\nabla$ 与函数 $f$ 的``数乘''
  \[\nabla f=i\pp{}{x}+j\pp{}{y}+k\pp{}{z},\]
  与向量 $\bfD=iD_x+jD_y+kD_z$ 的``数量积''
  \[\nabla\cdot \bfD=\pp{D_x}{x}+\pp{D_y}{y}+\pp{D_z}{z},\]
  与向量 $\bfD=iD_x+jD_y+kD_z$ 的``向量积''
  \[\nabla\times\bfD=(\pp{D_z}y-\pp{D_y}z)i+(\pp{D_x}z-\pp{D_z}x)j+(\pp{D_y}x-\pp{D_x}y)k.\]
  由于这些运算规则和四元数的运算规则类似, 因此我们可仿照四元数的运算规则来得到 $\nabla$ 的有关性质.
  \begin{exlist}
    \item 证明 $\nabla\times\nabla=0$, 由此得到
    \[\nabla\nabla=-\nabla\cdot\nabla+\nabla\times\nabla=-\nabla^2,\]
    其中 $\nabla^2$ 表示拉普拉斯算子 $\nabla\cdot\nabla$.
    \item 证明 $\nabla\times\nabla f=0$.
    提示: 考虑 $\nabla\nabla f$ 和乘法结合律.
    \item 证明 $\nabla\cdot(\nabla\times \bfD)=0$ 和 $\nabla\times(\nabla\times\bfD)=\nabla(\nabla\cdot\bfD)-\nabla^2\bfD$.
    提示: 考虑 $\nabla\nabla\bfD$ 和乘法结合律.
  \end{exlist}
  事实上, $\nabla f$ 是标量场 $f$ 的梯度, $\nabla\cdot \bfD$ 是向量场 $\bfD$ 的散度, $\nabla\times\bfD$ 是向量场 $\bfD$ 的旋度.
  \item 设 $\bfP$ 是三维向量, $q$ 是四元数, 定义
  \[\Phi_q(\bfP)=q\bfP q^{-1}.\]
  \begin{exlist}
    \item 证明: $\Phi_q(\bfP_1\bfP_2)=\Phi_q(\bfP_1)\Phi_q(\bfP_2)$.
    由于 $\bfP_1\bfP_2=-\bfP_1\cdot\bfP_2+\bfP_1\times\bfP_2$, 因此 $\Phi_q$ 保持了向量的长度、夹角和手性. 换言之, 它是 $\BR^3$ 上的一个旋转变换.
    \item 证明: 若 $\bfA$ 是单位向量, $q=\cos\dfrac\theta2+\bfA\sin\dfrac\theta2$, 则 $\Phi_q$ 是绕单位向量 $\bfA$ 逆时针旋转 $\theta$ 的变换.
    由此可知, 所有的旋转变换均可用四元数来表示.
    \item 证明: 若 $q=w+xi+yj+zk$ 且 $q\tau(q)=1$, 则 $\Phi_q$ 对应的旋转矩阵为
    \[\begin{pmatrix}
      1-2y^2-2z^2&2xy-2wz&2xz+2wy\\
      2xy+2wz&1-2x^2-2x^2&2yz-2wx\\
      2xz-2wy&2yz+2wx&1-2x^2-2y^2
    \end{pmatrix}.\]
    它是行列式为 $1$ 的正交阵.
  \end{exlist}
  显然四元数乘法的复杂度比对应的矩阵乘法复杂度要低, 因此在计算机图形学中, 尤其是三维物体的旋转中, 四元数有着重要的作用.
\end{exlist}



\subsection{单位根的应用}
由复数方根的计算公式可知, $n$ 次单位根为 $z=e^{\frac{2k\pi i}n},k=0,1,\dots,n-1$.
单位根在代数、几何和组合中有着丰富的应用. 我们来看一个例子.
设集合 $A=\set{1,2,\dots,1001}$.
\begin{exlist}
  \item 集合 $A$ 有多少个子集? 试着将 $A$ 的每一个子集与
  \[N(x)=\prod_{a=1}^{1001}(1+x^a)\]
  的展开式中的每一项建立一个一一对应.
  \item 设 $S\subseteq A$. 定义
  \[f(S)=\prod_{a\in S}x^a=x^{\sum_{a\in S}a}.\]
  证明所有的 $S$ 对应的 $f(S)$ 之和就是 $N(x)$.
  \item 证明 $N(x)$ 的展开式合并同类项后 $x^k$ 的系数就是 $A$ 的那些满足元素之和是 $k$ 的子集的个数.
  \item 现在我们想知道 $A$ 有多少个子集满足元素之和是 $5$ 的倍数.
令 $x$ 是 $5$ 次单位根, 则 $N(x)$ 可以表为
\[N(x)=N_0+N_1x+N_2x^2+N_3x^3+N_4x^4,\]
  那么 $N_0$ 就是元素之和是 $5$ 的倍数的集合个数.
  \item 当 $x=1$ 时, 显然 $N(1)=2^{1001}$.
  当 $x\neq 1$ 是 $5$ 次单位根时, $1,x,x^2,x^3,x^4$ 是方程 $X^5-1=0$ 的所有根, 所以 $2,1+x,1+x^2,1+x^3,1+x^4$ 是方程 $(X-1)^5-1=0$ 的所有根. 由韦达定理可知
  \[(1+x^0)(1+x)(1+x^2)(1+x^3)(1+x^4)=2.\]
  由此证明
  \[N(x)=2^{405}(1+x).\]
  \item 计算 $N(1)+N(e^{2\pi i/5})+N(e^{4\pi i/5})+N(e^{6\pi i/5})+N(e^{8\pi i/5})$. 由此得到 $N_0=\dfrac{2^{1001}-2^{201}}5$.
  \item 想一想, $N_1,N_2,N_3,N_4$ 分别是多少?
\end{exlist}
% 更多细节可见: \url{https://www.bilibili.com/video/BV1R34y1W7Xn/}



\subsection{切比雪夫多项式}
对棣莫弗公式左侧进行二项式展开可以得到
\begin{align*}
  \cos(2\theta)&=\hphantom{1}2\cos^2\theta-\hphantom{1}1,\\
  \cos(3\theta)&=\hphantom{1}4\cos^3\theta-\hphantom{1}3\cos\theta,\\
  \cos(4\theta)&=\hphantom{1}8\cos^4\theta-\hphantom{1}8\cos^2\theta+1,\\
  \cos(5\theta)&=16\cos^5\theta-20\cos^3\theta+5\cos\theta.
\end{align*}
一般地, 可以证明 $\cos{n\theta}$ 是 $\cos\theta$ 的 $n$ 次多项式, 称这个多项式 $T_n(x)$ 为\emph{切比雪夫多项式}.
\begin{exlist}
  \item 证明 $T_n$ 满足递推
  \[T_0(x)=1,\quad T_1(x)=x,\quad
  T_{n+1}(x)=2xT_n(x)-T_{n-1}(x),\]
  并由此证明
  \[\sum_{n=0}^\infty T_n(x)t^n=\frac{1-tx}{1-2tx+t^2}.\]
  \item 求 $T_n(x)$ 在 $[0,1]$ 中的所有根.
  \item 证明 $T_n(x)$ 的前两项为
  \[T_n(x)=2^{n-1}x^n-n2^{n-3}T^{n-2}+\cdots\]
  \item 设 $f(x)$ 是首项系数为 $2^{n-1}$ 的 $n$ 次多项式. 那么 $\max\limits_{x\in[-1,1]}f(x)\ge 1$.
  等号成立当且仅当 $f(x)=T_n(x)$. (提示: 利用反证法, 考虑 $f-T_n$ 在 $T_n$ 零点的行为.)
\end{exlist}

