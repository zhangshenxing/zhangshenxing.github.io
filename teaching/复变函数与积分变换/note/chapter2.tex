
\chapter{解析函数}
\section{解析函数的概念}

\subsection{可导的函数}

由于 $\BC$ 和 $\BR$ 一样是域, 因此我们可以像一元实变函数一样去定义复变函数的导数和微分.

\begin{definition}[导数]
	设 $w=f(z)$ 的定义域是区域 $D$, $z_0\in D$.
	如果极限
		\[\lim_{z\to z_0}\frac{f(z)-f(z_0)}{z-z_0}
		=\lim_{\Delta z\to 0}\frac{f(z_0+\Delta z)-f(z_0)}{\Delta z}\]
	存在,则称 \emph{$f(z)$ 在 $z_0$ 可导}.
  这个极限值称为 \emph{$f(z)$ 在 $z_0$ 的导数},记作
		\[f'(z_0)=\frac{\diff w}{\diff z}\bigg|_{z=z_0}
		=\lim_{\Delta z\to 0}\frac{f(z_0+\Delta z)-f(z_0)}{\Delta z}.\]
	如果 $f(z)$ 在区域 $D$ 内处处可导, 称 \emph{$f(z)$ 在 $D$ 内可导}.
\end{definition}

\begin{example}
	函数 $f(z)=x+2yi$ 在哪些点处可导?
\end{example}

\begin{solution}
	\begin{align*}
		f'(z)&=\lim_{\Delta z\to 0}\frac{f(z+\Delta z)-f(z)}{\Delta z}\\
		&{=\lim_{\Delta z\to 0}\frac{(x+\Delta x)+2(y+\Delta y)i-(x+2yi)}{\Delta z}}\\
		&{=\lim_{\Delta z\to 0}\frac{\Delta x+2\Delta y i}{\Delta x+\Delta yi}.}
	\end{align*}
	当 $\Delta x=0, \Delta y\to 0$ 时, 上式$\to2$;当 $\Delta y=0, \Delta x\to 0$ 时, 上式$\to1$.因此该极限不存在, $f(z)$ 处处不可导.
\end{solution}

\begin{exercise}
  函数 $f(z)=\ov z=x-yi$ 在哪些点处可导? 
\end{exercise}

\begin{example}
	求 $f(z)=z^2$ 的导数.
\end{example}

\begin{solution}
	\[
	f'(z)=\lim_{\Delta z\to 0}\frac{f(z+\Delta z)-f(z)}{\Delta z}
	{=\lim_{\Delta z\to 0}\frac{(z+\Delta z)^2-z^2}{\Delta z}}
	{=\lim_{\Delta z\to 0}(2z+\Delta z)=2z.}
	\]
\end{solution}

和一元实变函数情形类似, 我们有如下求导法则:
\begin{theorem}[导函数的运算法则]
	\begin{itemize}
		\item $(c)'=0$, 其中 $c$ 为复常数;
		\item $(z^n)'=nz^{n-1}$, 其中 $n$ 为整数;
		\item $(f\pm g)'=f'\pm g',\quad (cf)'=cf'$;
		\item $(fg)'=f'g+fg',\quad \left(\dfrac fg\right)'=\dfrac{f'g-fg'}{g^2}$;
		\item $[f(g(z))]'=f'[g(z)]\cdot g'(z)$;
		\item $g'(z)=\dfrac1{f'(w)}, g=f^{-1}, w=g(z)$.
	\end{itemize}
\end{theorem}

\begin{theorem}[可导蕴含连续]
	若 $f(z)$ 在 $z_0$ 可导, 则 $f(z)$ 在 $z_0$ 连续.
\end{theorem}

该定理的证明和实变量情形完全相同.
\begin{proof}
	设
    \[\Delta w=f(z_0+\Delta z)-f(z_0),\]
  则
		\[
      \lim_{\Delta z\to 0}\Delta w
      =\lim_{\Delta z\to 0}\frac{\Delta w}{\Delta z}\cdot\Delta z
      =\lim_{\Delta z\to 0}\frac{\Delta w}{\Delta z}\cdot
        \lim_{\Delta z\to 0}\Delta z
      =f'(z_0)\cdot 0=0.\qedhere
		\]
\end{proof}


\subsection{可微的函数}

复变函数的微分也和一元实变函数情形类似.

\begin{definition}[微分]
	如果存在常数 $A$ 使得函数 $w=f(z)$ 满足
    \[\Delta w=f(z_0+\Delta z)-f(z_0)=A\Delta z+o(\Delta z),\]
	其中 $o(\Delta z)$ 表示 $\Delta z$ 的高阶无穷小量,
	则称 \emph{$f(z)$ 在 $z_0$ 处可微}, 称 $A\Delta z$ 为 \emph{$f(z)$ 在 $z_0$ 的微分}, 记作 $\diff w=A \Delta z$.
\end{definition}

和一元实变函数情形一样, 复变函数的可微和可导是等价的, 且 $\diff w=f'(z_0)\Delta z, \diff z=\Delta z$.
故
  \[\diff w=f'(z_0)\diff z, f'(z_0)=\dfrac{\diff w}{\diff z}.\]


\subsection{解析的函数}

\begin{definition}[解析和奇点]
	\begin{itemize}
		\item 若函数 $f(z)$ 在 $z_0$ 的一个邻域内处处可导, 则称 \emph{$f(z)$ 在 $z_0$ 解析}.
		\item 若 $f(z)$ 在区域 $D$ 内处处解析, 则称 $f(z)$ 在 $D$ 内解析, 或称 $f(z)$ 是 $D$ 内的一个\emph{解析函数}.
		\item 若 $f(z)$ 在 $z_0$ 不解析, 则称 $z_0$ 为 $f(z)$ 的一个\emph{奇点}.
	\end{itemize}
\end{definition}

无定义、不连续、不可导、可导但不解析, 都会导致奇点的产生.

由于区域 $D$ 是一个开集, 其中的任意 $z_0\in D$ 均存在一个包含在 $D$ 的邻域. 所以 \alert{$f(z)$ 在 $D$ 内解析和在 $D$ 内可导是等价的}.

如果 $f(z)$ 在 $z_0$ 解析, 则 $f(z)$ 在 $z_0$ 的一个邻域内处处可导, 从而在该邻域内解析. 因此 \alert{$f(z)$ 解析点全体是一个开集}.

\begin{exercise}
	函数 $f(z)$ 在点 $z_0$ 处解析是 $f(z)$ 在该点可导的\fillbrace{}.
	\begin{taskschoice}(2)
		() 充分条件
		() 必要条件
		() 充要条件
		() 既非充分也非必要条件
	\end{taskschoice}
\end{exercise}

\begin{example}
	研究函数 $f(z)=|z|^2$ 的解析性.
\end{example}
\begin{solution}
	由于
	\[
		\frac{f(z+\Delta z)-f(z)}{\Delta z}
		=\frac{(z+\Delta z)(\ov z+\ov{\Delta z})-z\ov z}{\Delta z}
		=\ov z+\ov{\Delta z}+z\frac{\Delta x-\Delta yi}{\Delta x+\Delta yi},
	\]
  \begin{enumerate}
    \item 若 $z=0$, 则当 $\Delta z\to 0$ 时该极限为 $0$.
    \item 若 $z\neq0$, 则当 $\Delta y=0,\Delta x\to 0$ 时该极限为 $\ov z+z$; 当 $\Delta x=0,\Delta y\to 0$ 时该极限为 $\ov z-z$.因此此时极限不存在.
  \end{enumerate}
	故 $f(z)$ 仅在 $z=0$ 处可导, 从而处处不解析.
\end{solution}


\section{函数解析的充要条件}

\subsection{柯西-黎曼方程}

通过对一些简单函数的分析, 我们发现可导的函数往往可以直接表达为 $z$ 的函数的形式, 而不解析的往往包含 $x,y,\ov z$ 等内容.
这种现象并不是孤立的.
我们来研究二元实变量函数的可微性与复变函数可导的关系.

为了简便我们用 $u_x,u_y,v_x,v_y$ 等记号表示偏导数.

设 \alert{$f$ 在 $z$ 处可导}, $f'(z)=a+bi$, 则
  \[
    \Delta u+i\Delta v=\Delta f
    =(a+bi)(\Delta x+i\Delta y)+o(\Delta z).
  \]
展开可知
  \begin{align*}
    \Delta u&=a\Delta x-b\Delta y+o(\Delta z),\\
    \Delta v&=b\Delta x+a\Delta y+o(\Delta z).
  \end{align*}
由于 $o(\Delta z)=o(|\Delta z|)=o(\sqrt{x^2+y^2})$,
因此 \alert{$u,v$ 可微且 $u_x=v_y=a,v_x=-u_y=b$}.

反过来, 假设 $u,v$ 可微且 $u_x=v_y, v_x=-u_y$. 由全微分公式
  \begin{align*}
    \diff u&=u_x\diff x+u_y\diff y
      =u_x\diff x-v_x\diff y,\\
    \diff v&=v_x\diff x+v_y\diff y=v_x\diff x+u_x\diff y,\\
    \diff f&=\diff (u+iv)=(u_x+i v_x)\diff x+(-v_x+i u_x)\diff y\\
      &=(u_x+i v_x)\diff(x+iy)\\
      &=(u_x+i v_x)\diff z=(v_y-i u_y)\diff z.
  \end{align*}
故 $f(z)$ 在 $z$ 处可导, 且 $f'(z)=u_x+i v_x=v_y-i u_y$.

由此得到
\begin{theorem}{柯西-黎曼方程}
	$f(z)$ 在 $z$ 可导当且仅当在 $z$ 点 $u,v$ 可微且满足柯西-黎曼方程(简称为 C-R 方程):
    \[u_x=v_y,\quad v_x=-u_y.\]
	此时
    \[f'(z)=u_x+iv_x=v_y-iu_y.\]
\end{theorem}

\begin{figure}[htbp]
  \begin{minipage}{.45\columnwidth}
    \centering
    \includegraphics[height=40mm]{../image/Cauchy.jpeg}
    \caption{柯西}
  \end{minipage}
  \begin{minipage}{.45\columnwidth}
    \centering
    \includegraphics[height=40mm]{../image/Riemann.jpeg}
    \caption{黎曼}
  \end{minipage}
\end{figure}

注意到 $x=\dfrac12z+\dfrac12\ov z,y=-\dfrac i2z+\dfrac i2\ov z$.
仿照着二元实函数偏导数在变量替换下的变换规则, 我们定义 $f$ 对 $z$ 和 $\ov z$ 的偏导数为
  \[\left\{\begin{aligned}
      \dpp fz&
    =\dpp xz\dpp fx+\dpp yz\dpp fy
    =\frac12\dpp fx-\frac i2\dpp fy,\\
      \dpp f{\ov z}&
    =\dpp x{\ov z}\dpp fx+\dpp y{\ov z}\dpp fy
    =\frac12\dpp fx+\frac i2\dpp fy.
  \end{aligned}\right.\]
如果把 $z,\ov z$ 看成独立变量, 那么当 $f$ 在 $z$ 处可导时, $\diff f=f'\diff z$.
当 $f$ 关于 $z,\ov z$ 可微时(即 $u,v$ 可微),
  \[\diff f=\dpp fz\diff z+\dpp f{\ov z}\diff\ov z.\]
所以 \alert{$f$ 在 $z$ 处可导当且仅当 $u,v$ 可微且 $\dpp f{\ov z}=0$, 此时 $f'(z)=\dpp fz$.}

由于二元函数的偏导数均连续蕴含可微, 因此我们有:

\begin{theorem}[连续偏导蕴含可导]
	\begin{itemize}
		\item 如果 $u_x,u_y,v_x,v_y$ 在 $z$ 处连续, 且满足C-R方程, 则 $f(z)$ 在 $z$ 可导.
		\item 如果 $u_x,u_y,v_x,v_y$ 在区域 $D$ 上处处连续, 且满足C-R方程, 则 $f(z)$ 在 $D$ 上可导(从而解析).
	\end{itemize}
\end{theorem}


\subsection{柯西-黎曼方程的应用}

\begin{example}
  \begin{enumerate}
    \item 函数 $f(z)=\ov z$ 在何处可导, 在何处解析?
    \item 函数 $f(z)=z\Re z$ 在何处可导, 在何处解析?
    \item 函数 $f(z)=e^x(\cos y+i\sin y)$ 在何处可导, 在何处解析?
  \end{enumerate}
\end{example}
\begin{solution}
  \begin{enumerate}
    \item 由 $u=x,v=-y$ 可知
      \begin{align*}
        u_x&=1,&u_y&=0,\\
        v_x&=0,&v_y&=-1.
      \end{align*}
    因为 $u_x=1\neq v_y=-1$, 所以该函数处处不可导, 处处不解析.
    \footnote{也可由 $\dpp f{\ov z}=1\neq0$ 看出.}
    \item 由 $f(z)=x^2+ixy,u=x^2,v=xy$ 可知
      \begin{align*}
        u_x&=2x,&u_y&=0,\\
        v_x&=y, &v_y&=x.
      \end{align*}
    由 $2x=x,0=-y$ 可知只有 $x=y=0,z=0$ 满足C-R方程.因此该函数只在 $0$ 可导, 处处不解析且 $f'(0)=u_x(0)+iv_x(0)=0$.
    \footnote{也可由 $f=\dfrac12 z(z+\ov z), \dpp f{\ov z}=\dfrac12 z$ 看出, $f'(0)=\dpp fz\Big|_{z=0}=z|_{z=0}=0$.}
    \item 由 $u=e^x\cos y,v=e^x\sin y$ 可知
      \begin{align*}
        u_x&=e^x\cos y,&u_y&=-e^x\sin y,\\
        v_x&=e^x\sin y,&v_y&=e^x\cos y.
      \end{align*}
    因此该函数处处可导, 处处解析, 且
      \[f'(z)=u_x+iv_x=e^x(\cos y+i\sin y)=f(z).\]
  \end{enumerate}
\end{solution}

实际上, \enumnum3 中的函数就是复变量的指数函数 $e^z$.

\begin{exercise}
	函数\fillbrace{}在 $z=0$ 处不可导.
	\begin{taskschoice}(4)
		() $2x+3yi$
		() $2x^2+3y^2i$
		() $e^x\cos y+i e^x\sin y$
		() $x^2-xyi$
	\end{taskschoice}
\end{exercise}

\begin{example}
	设函数 $f(z)=(x^2+axy+by^2)+i(cx^2+dxy+y^2)$ 在复平面内处处解析. 求实常数 $a,b,c,d$ 以及 $f'(z)$.
\end{example}
\begin{solution}
	由于
	\begin{align*}
		u_x&=2x+ay,&u_y&=ax+2by,\\
		v_x&=2cx+dy,&v_y&=dx+2y,
	\end{align*}
	因此
		\[2x+ay=dx+2y,\quad ax+2by=-(2cx+dy),\]
		\[a=d=2,\quad b=c=-1,\]
		\[f'(z)=u_x+iv_x=2x+2y+i(-2x+2y)=(2-2i)z.\]
\end{solution}

\begin{example}
	证明: 如果 $f'(z)$ 在区域 $D$ 内处处为零, 则 $f(z)$ 在 $D$ 内是一常数.
\end{example}

\begin{proof}
	由于
    \[f'(z)=u_x+iv_x=v_y-iu_y=0,\]
	因此 $u_x=v_x=u_y=v_y=0$, $u,v$ 均为常数,从而  $f(z)=u+iv$ 是常数.
\end{proof}
类似地可以证明, 若 $f(z)$ 在 $D$ 内解析, 则下述条件等价:
\begin{itemize}
	\item $f(z)$ 是一常数,
	\item $f'(z)=0$,
	\item $\arg{f(z)}$ 是一常数,
	\item $|f(z)|$ 是一常数,
	\item $\Re{f(z)}$ 是一常数,
	\item $\Im{f(z)}$ 是一常数,
	\item $v=u^2$,
	\item $u=v^2$.
\end{itemize}

\begin{example}
	证明: 如果 $f(z)$ 解析且 $f'(z)$ 处处非零, 则曲线族 $u(x,y)=c_1$ 和曲线族 $v(x,y)=c_2$ 互相正交.
\end{example}

\begin{proof}
	由于 $f'(z)=u_x-iu_y$, 因此 $u_x,u_y$ 不全为零.
	对 $u(x,y)=c_1$ 使用隐函数求导法则得 $u_x\diff x+u_y\diff y=0$,从而 $(u_y,-u_x)$ 是该曲线在 $z$ 处的非零切向量.

	同理 $(v_y,-v_x)$ 是 $v(x,y)=c_2$ 在 $z$ 处的非零切向量.由于
		\[u_yv_y+u_xv_x=u_yu_x-u_xu_y=0,\]
	因此二者正交.
\end{proof}

当 $f'(z_0)\neq 0$ 时, 经过 $z_0$ 的两条曲线 $C_1,C_2$ 的夹角和它们的像 $f(C_1),f(C_2)$ 在 $f(z_0)$ 处的夹角总是相同的.
这种性质被称为\emph{保角性}.
这是因为 $\diff f=f'(z_0)\diff z$.
局部来看 $f$ 把 $z_0$ 附近的点以 $z_0$ 为中心放缩 $f'(z_0)$ 倍并逆时针旋转 $\arg{f'(z_0)}$.
由 $w$ 复平面上曲线族 $u=c_1,v=c_2$ 正交可知上述例题成立.

最后我们来看复数在求导中的一个应用.

\begin{example}
	设 $f(z)=\dfrac1{1+z^2}$, 则它在除 $z=\pm i$ 外处处解析.
	当 $z=x$ 为实数时,
  \begin{align*}
    \biggl(\frac1{1+x^2}\biggr)^{(n)}&
      =f^{(n)}(x)=\frac i2\biggl(\frac1{x+i}-\frac1{x-i}\biggr)^{(n)}\\
    &=\frac i2\cdot(-1)^n n!\biggl(\frac1{(x+i)^{n+1}}-\frac1{(x-i)^{n+1}}\biggr)\\
    &=(-1)^{n+1}n!\Im\frac1{(x+i)^{n+1}}\\
    &=(-1)^nn!(x^2+1)^{-\frac{n+1}2}\sin\bigl((n+1)\arccot x\bigr).
  \end{align*}
	任意有理函数的高阶导数均可按此法计算.
\end{example}


\section{初等函数}

我们将实变函数中的初等函数推广到复变函数.
多项式函数和有理函数的解析性质已经介绍过, 这里不再重复.

\subsection{指数函数}

我们来定义指数函数. 
指数函数有多种等价的定义方式:
\begin{enumerate}
	\item $\exp z=e^x(\cos y+i\sin y)$ (欧拉恒等式);
	\item $\exp z=\lim\limits_{n\to\infty}\left(1+\dfrac zn\right)^n$ (极限定义);
	\item $\exp z=1+z+\dfrac{z^2}{2!}+\dfrac{z^3}{3!}+\cdots
	=\lim\limits_{n\to\infty}\sum\limits_{k=0}^n\dfrac{z^k}{k!}$ (级数定义);
	\item $\exp z$ 是唯一的一个处处解析的函数, 使得当 $z=x\in\BR$ 时, $\exp z=e^x$ ($e^x$ 的解析延拓).
\end{enumerate}

有些人会从 $e^x,\cos x,\sin x$ 的泰勒展开
\begin{align*}
	e^x&=1+x+\frac{x^2}{2!}+\frac{x^3}{3!}+\cdots\\
	\cos x&=1-\frac{x^2}{2!}+\frac{x^4}{4!}+\cdots\\
	\sin x&=x-\frac{x^3}{3!}+\frac{x^5}{5!}\cdots
\end{align*}
形式地带入得到欧拉恒等式 $e^{ix}=\cos x+i\sin x$.
事实上我们可以把它当做复指数函数的定义, 而不是欧拉恒等式的证明.
我们将在第四章说明\enumnum1、\enumnum3和\enumnum4是等价的.

我们来证明\enumnum1和\enumnum2等价.
\begin{align*}
	\lim_{n\to\infty}\abs{1+\frac zn}^n
	&=\lim_{n\to\infty}\left(1+\frac{2x}n+\frac{x^2+y^2}{n^2}\right)^{\frac n2}\quad
	{(1^\infty\ \text{型不定式})}\\
	&{=\exp\left[\lim_{n\to\infty}\frac n2
	\left(\frac{2x}n+\frac{x^2+y^2}{n^2}\right)\right]=e^x.}
\end{align*}
不妨设 $n>\abs{z}$, 这样 $1+\dfrac zn$ 落在右半平面,
  \[
    \lim_{n\to\infty} n\arg{\left(1+\frac zn\right)}
    =\lim_{n\to\infty} n\arctan \frac y{n+x}
    =\lim_{n\to\infty}\frac{ny}{n+x}=y.
  \]
故
  \[\lim_{n\to\infty}\left(1+\dfrac zn\right)^n=e^x(\cos y+i\sin y).\]

\begin{definition}{指数函数}
定义\emph{指数函数}
	\[\exp z:=e^x(\cos y+i\sin y).\]
\end{definition}
为了方便, 我们也记 \alert{$e^z=\exp z$}.
指数函数有如下性质:
\begin{itemize}
  \item $\exp z$ 处处解析, 且 $(\exp z)'=\exp z$.
	\item $\exp z\neq 0$.
	\item $\exp(z_1+z_2)=\exp z_1\cdot \exp z_2$.
	\item $\exp(z+2k\pi i)=\exp z$, 即 $\exp z$ 周期为 $2\pi i$.
	\item $\exp z_1=\exp z_2$ 当且仅当 $z_1=z_2+2k\pi i,k\in\BZ$.
	\item $\exp z$ 将直线族 $\Re z=c$ 映为圆周族 $\abs{w}=e^c$, 将直线族 $\Im z=c$ 映为射线族 $\Arg w=c$.
\end{itemize}

\begin{example}
	计算函数 $f(z)=\exp(z/6)$ 的周期.
\end{example}
\begin{solution}
	设 $f(z_1)=f(z_2)$, 则 $\exp(z_1/6)=\exp(z_2/6)$.
	{因此存在 $k\in\BZ$ 使得
		\[\frac{z_1}6=\frac{z_2}6+2k\pi i,\]从而 $z_1-z_2=12k\pi i$.所以 $f(z)$ 的周期是 $12\pi i$.}
\end{solution}

一般地, $\exp(az+b)$ 的周期是 $\dfrac{2\pi i}a$ (或写成 $-\dfrac{2\pi i}a$), $a\neq 0$.


\subsection{对数函数}

对数函数 $\Ln z$ 定义为指数函数 $\exp z$ 的反函数.
为什么我们用大写的 $\Ln$ 呢? 
在复变函数中, 很多函数是多值函数.
为了便于研究, 我们会固定它的一个单值分支.
我们将多值的这个开头字母大写, 而对应的单值的则是开头字母小写.
例如 $\Arg z$ 和 $\arg z$.

设 $z\neq 0$, $e^w=z=re^{i\theta}=e^{\ln r+i\theta}$,
则
\[w=\ln r+i\theta+2k\pi i,\quad k\in\BZ.\]
\begin{definition}[对数函数]
	\begin{enumerate}
		\item 定义\emph{对数函数}
			\[\Ln z=\ln\abs{z}+i\Arg z.\]
			它是一个多值函数.
		\item 定义\emph{对数函数主值}
			\[\ln z=\ln\abs{z}+i\arg z.\]
	\end{enumerate}
\end{definition}
对于每一个整数 $k$, $\ln z+2k\pi i$ 都给出了 $\Ln z$ 的一个单值分支.
特别地, 当 $z=x>0$ 是正实数时, $\ln z$ 就是实变的对数函数.

\begin{example}
	求 $\Ln 2,\Ln(-1)$ 以及它们的主值.
\end{example}

\begin{solution}
	\[\Ln2=\ln2+2k\pi i,\quad k\in\BZ,\]
	主值为 $\ln 2$.
	\[\Ln(-1)=\ln1+i\Arg(-1)=(2k+1)\pi i,\quad k\in\BZ,\]
	主值为 $\pi i$.
\end{solution}

\begin{example}
求 $\Ln(-2+3i),\Ln(3-\sqrt3 i)$.
\end{example}

\begin{solution}
	\[
    \Ln(-2+3i)=\ln\abs{-2+3i}+i\Arg(-2+3i)
      =\frac 12\ln 13+\left(-\arctan\frac 32+\pi+2k\pi\right)i,
      \quad k\in\BZ.
  \]
	\[
    \Ln(3-\sqrt3i)=\ln\abs{3+\sqrt 3i}+i\Arg(3-\sqrt 3i)
      =\ln 2\sqrt 3+\left(-\frac\pi6+2k\pi\right)i
      =\ln 2\sqrt 3+\left(2k-\frac16\right)\pi i,
      \quad k\in\BZ.
  \]
\end{solution}

\begin{example}
	解方程 $e^z-1-\sqrt 3i=0$.
\end{example}

\begin{solution}
	由于 $1+\sqrt 3 i=2e^{\frac{\pi i}3}$, 因此
	\[z=\Ln(1+\sqrt 3i)=\ln 2+\left(2k+\frac13\right)\pi i,\quad k\in\BZ.\]
\end{solution}

\begin{exercise}
	求 $\ln(-1-\sqrt3 i)=$\fillblank[2cm][3mm]{}.
\end{exercise}

对数函数与其主值的关系是
\[\Ln z=\ln z+\Ln 1=\ln z+2k\pi i,\quad k\in\BZ.\]
根据辐角以及主辐角的相应等式, 我们有
\[\Ln(z_1\cdot z_2)=\Ln z_1+\Ln z_2,\quad
  \Ln\frac{z_1}{z_2}=\Ln z_1-\Ln z_2,\]
\[\Ln \sqrt[n]z=\dfrac1n\Ln z.\]
而当 $\abs{n}\ge 2$ 时, \alert{$\Ln z^n=n\Ln z$ 不成立}.
以上等式换成 $\ln z$ 均不一定成立.


设 $x$ 是正实数, 则
\[\ln (-x)=\ln x+\pi i,\quad
  \lim_{y\to0^-}\ln (-x+yi)=\ln x-\pi i,\]
因此 $\ln z$ 在负实轴和零处不连续.
而在其它地方, $-\pi<\arg z<\pi$, $\ln z$ 是 $e^z$ 在区域 $-\pi<\Im z<\pi$ 上的单值反函数, 
从而 \alert{$(\ln z)'=\dfrac 1z$}, \alert{$\ln z$ 在除负实轴和零处的区域解析}.
\footnote{任取一条从 $0$ 到 $\infty$ 的简单曲线, 在去掉这条曲线后, 若固定一复数 $z_0$ 的辐角, 则多值函数 $\Arg z$ 可以在该区域内连续单值化, 简单来说就是沿着 $z_0$ 到 $z$ 的曲线让辐角连续变化. 同理, $\Ln z$ 也可以在该区域内单值化, 只需固定一复数 $z_0$ 的值.}

也可以通过C-R方程来得到 $\ln z$ 的解析性和导数: 当 $x>0$ 时,
\[\ln z=\half \ln(x^2+y^2)+i\arctan \frac yx,\]
\[u_x=v_y=\frac x{x^2+y^2},\qquad v_x=-u_y=-\frac y{x^2+y^2},\]
\[(\ln z)'=\frac{x-yi}{x^2+y^2}=\frac 1z.\]
其它情形可取虚部为 $\arccot\dfrac xy$ 或 $\arccot\dfrac xy-\pi$ 类似证明.

\subsection{幂函数}

\begin{definition}{幂函数}
	\begin{enumerate}
		\item 设 $a\neq 0$, $z\neq 0$, 定义\emph{幂函数}
		\[w=z^a=e^{a\Ln z}
		=\exp\bigl(a\ln\abs{z}+ia(\arg z+2k\pi)\bigr),\quad k\in\BZ.\]
		\item \emph{幂函数的主值}为
		\[w=e^{a\ln z}=\exp\bigl(a\ln\abs{z}+ia\arg z\bigr).\]
	\end{enumerate}
\end{definition}

根据 $a$ 的不同, 这个函数有着不同的性质.
\begin{enumerate}
  \item 当 $a$ 为整数时, 因为 $e^{2ak\pi i}=1$, 所以 $w=z^a$ 是单值的. 此时 $z^a$ 就是我们之前定义的乘幂. 
    当 $a$ 是非负整数时, $z^a$ 在复平面上解析;
    当 $a$ 是负整数时, $z^a$ 在 $\BC-\set0$ 上解析.
  \item 当 $a=\dfrac pq$ 为分数, $p,q$ 为互质的整数且 $q>1$ 时,
    \[z^{\frac pq}=\abs{z}^{\frac pq}\exp\biggl(\frac{ip(\arg z+2k\pi)}q\biggr),\quad k=0,1,\dots,q-1\]
    具有 $q$ 个值.
    去掉负实轴和 $0$ 之后, 它的主值 $w=\exp(a\ln z)$ 是处处解析的.
    事实上它就是 $\sqrt[q]{z^p}=(\sqrt[q]z)^p$.
		\begin{figure}[htbp]
			\centering
      \begin{tikzpicture}
				\coordinate [label=below left:{$0$}] (O) at (0,0);
				\coordinate [label=below:{$x$}] (X) at (2,0);
				\coordinate [label=left:{$y$}] (Y) at (0,2);
        \draw[cstaxis] (O)--(X);
				\draw[cstaxis] (0,-2)--(Y);
        \draw[draw=white,cstfille1] (O) circle (1.3);
        \draw[cstdash,main] (0,0)--(-2,0);
        \draw[cstdash,cstra,third] (1.7,.7)to [bend left] (4.5,.7);
        \draw (3,1.7) node[third] {$w=z^{2/9}$};
				\begin{scope}[xshift=5cm]
					\coordinate [label=below left:{$0$}] (O) at (0,0);
					\fill[cstfille2,pattern color=second] (O)--({1.4*cos(40)},{1.4*sin(40)}) arc (40:-40:1.4)--cycle;
					\draw[cstdash,second] ({1.4*cos(40)},{-1.4*sin(40)})--(0,0)--({1.4*cos(40)},{1.4*sin(40)});
					\coordinate [label=below left:{$0$}] (O) at (0,0);
					\coordinate [label=below:{$u$}] (X) at (2,0);
					\coordinate [label=left:{$v$}] (Y) at (0,2);
					\draw[cstaxis] (-2,0)--(X);
					\draw[cstaxis] (0,-2)--(Y);
				\end{scope}
      \end{tikzpicture}
			\caption{映照 $w=z^{2/9}$}
    \end{figure}
  \item 对于其它的 $a$, $z^a$ 具有无穷多个值.
    这是因为此时当 $k\neq0$ 时, $2k\pi a i$ 不可能是 $2\pi i$ 的整数倍. 
    从而不同的 $k$ 得到的是不同的值.
    去掉负实轴和 $0$ 之后, 它的主值 $w=\exp(a\ln z)$ 也是处处解析的.
		\footnote{对于 $\Ln\dfrac{z-a}{z-b},\sqrt{(z-a)(z-b)}$ 等类型的多值函数, 我们需要将它的``奇点''连接起来形成``割线''. 复平面上去掉这些割线得到的区域内, 这些函数也可以如同 $\Arg z,\Ln z$ 那样单值化.}
\end{enumerate}

\begin{center}
  \begin{tabular}{cccc} \toprule
  	$a$& $z^a$ 的值& $z^a$ 的解析区域\\ \midrule
  	&&$n\ge0$ 时处处解析\\
		\multirow{-2}*{整数 $n$}&\multirow{-2}*{单值}&$n<0$ 时除零点外解析\\ \midrule
		分数 $p/q$&$q$ 值&除负实轴和零点外解析\\ \midrule
		无理数或虚数&无穷多值&除负实轴和零点外解析\\ \bottomrule
  \end{tabular}
\end{center}

\begin{example}
	求 $1^{\sqrt 2}$ 和 $i^i$.
\end{example}
\begin{solution}
	\[
    1^{\sqrt2}=e^{\sqrt2\Ln1}
      =e^{\sqrt 2\cdot 2k\pi i}
      =\cos(2\sqrt 2k\pi)+i\sin(2\sqrt 2k\pi), \quad k\in\BZ.
  \]
	\[
    i^i=e^{i\Ln i}
      =\exp\biggl(i\cdot\Bigl(2k+\half\Bigr)\pi i\biggr)
			=\exp\Bigl(-2k\pi-\half\pi\Bigr), \quad k\in\BZ.
  \]
\end{solution}

\begin{exercise}
	$3^i$ 的主辐角是\fillblank{}.
\end{exercise}

幂函数与其主值有如下关系:
\[
  z^a=e^{a\ln z}\cdot 1^a
    =e^{a\ln z}\cdot e^{2ak\pi i},\quad k\in\BZ.
\]
对于幂函数的主值,
\[(z^a)'=\left(e^{a\ln z}\right)'=\frac{ae^{a\ln z}}z=az^{a-1}.\]
一般而言, $z^a\cdot z^b=z^{a+b}$ 和 $(z^a)^b=z^{ab}$ 都是不成立的.
\footnote{$z^a\cdot z^b=z^{a+b}$ 成立当且仅当 $\dfrac{a}{a+b}\in\BZ$. $(z^a)^b=z^{ab}$ 成立当且仅当 $\dfrac1a\in\BZ$.}

最后, 注意 $e^a$ 作为指数函数 $f(z)=e^z$ 在 $a$ 处的值和作为 $g(z)=z^a$ 在 $e$ 处的值是\alert{不同}的.
因为后者在 $a\not\in\BZ$ 时总是多值的.
前者实际上是后者的主值.
为避免混淆, 以后我们总\alert{默认 $e^a$ 表示指数函数 $\exp a$}.


\subsection{三角函数和反三角函数}

我们知道
  \[\cos x=\frac{e^{ix}+e^{-ix}}2,\quad
  \sin x=\frac{e^{ix}-e^{-ix}}{2i}\]
对于任意实数 $x$ 成立,
我们将其推广到复数情形.

\begin{definition}{余弦和正弦函数}
	定义\emph{余弦}和\emph{正弦}函数
	\[\cos z=\frac{e^{iz}+e^{-iz}}2,\quad
	\sin z=\frac{e^{iz}-e^{-iz}}{2i}.\]
\end{definition}
那么欧拉恒等式 \alert{$e^{iz}=\cos z+i\sin z$ 对任意复数 $z$ 均成立}.

不难得到
\[
	\cos(iy)=\dfrac{e^y+e^{-y}}2,\qquad
	{\sin(iy)=i\dfrac{e^y-e^{-y}}2.}
\]
当 $y\to\infty$ 时, $\cos(iy)$ 和 $\sin(iy)$ 都 $\to\infty$.
因此 \alert{$\sin z$ 和 $\cos z$ 并不有界}. 
这和实变情形不同.

容易看出 $\cos z$ 和 $\sin z$ 的零点都是实数.
于是可类似定义其它三角函数
\begin{align*}
	\tan z&=\frac{\sin z}{\cos z},z\neq\left(k+\half\right)\pi,&
	\cot z&=\frac{\cos z}{\sin z},z\neq k\pi,\\
	\sec z&=\frac{1}{\cos z},z\neq\left(k+\half\right)\pi,&
	\csc z&=\frac{1}{\sin z},z\neq k\pi.
\end{align*}
这些三角函数的奇偶性, 周期性和导数与实变情形类似,
  \[(\cos z)'=-\sin z,\quad
  (\sin z)'=\cos z,\]
且在定义域范围内是处处解析的.
三角函数的各种恒等式在复数情形也仍然成立, 例如
\begin{itemize}
	\item $\cos(z_1\pm z_2)=\cos z_1 \cos z_2\mp \sin z_1 \sin z_2$,
	\item $\sin(z_1\pm z_2)=\sin z_1 \cos z_2\pm\cos z_1 \sin z_2$,
	\item $\sin^2z+\cos^2z=1$.
\end{itemize}

类似的, 我们可以定义\emph{双曲函数}:
\begin{align*}
  \ch z&=\frac{e^z+e^{-z}}2=\cos iz,\\
  \sh z&=\frac{e^z-e^{-z}}2=-i\sin iz,\\
  \tanh z&=\frac{e^z-e^{-z}}{e^z+e^{-z}}
    =-i\tan iz,\quad z\neq \left(k+\half\right)\pi i.\
\end{align*}
它们的奇偶性和导数与实变情形类似, 在定义域范围内是处处解析的.
$\ch z,\sh z$ 的周期是 $2\pi i$, $\tanh z$ 的周期是 $\pi i$.

设 $z=\cos w=\dfrac{e^{iw}+e^{-iw}}2$, 则
  \[e^{2iw}-2ze^{iw}+1=0,\quad
    {e^{iw}=z+\sqrt{z^2-1}\footnote{注意右侧是双值函数}.}\]
因此\emph{反余弦函数}为
\[w=\Arccos z=-i\Ln(z+\sqrt{z^2-1}).\]
显然它是多值的. 同理, 我们有:
\begin{itemize}
	\item \emph{反正弦函数} $\Arcsin z=-i\Ln(iz+\sqrt{1-z^2})$;
	\item \emph{反正切函数} $\Arctan z=-\dfrac i2\Ln\dfrac{1+iz}{1-iz}, z\neq \pm i$;
	\item \emph{反双曲余弦函数} $\Arch z=\Ln(z+\sqrt{z^2-1})$;
	\item \emph{反双曲正弦函数} $\Arsh z=\Ln(z+\sqrt{z^2+1})$;
	\item \emph{反双曲正切函数} $\Arth z=\dfrac12\Ln\dfrac{1+z}{1-z}, z\neq \pm1$.
\end{itemize}

\begin{example}
	解方程 $\sin z=2$.
\end{example}

\begin{solution}
  由于
  \[\sin z=\dfrac{e^{iz}-e^{-iz}}{2i}=2,\]
  我们有
  \[e^{2iz}-4ie^{iz}-1=0.\]
  于是 $e^{iz}=(2\pm\sqrt 3)i$,
  \[z=-i\Ln[(2\pm\sqrt 3)i]=\left(2k+\half\right)\pi\pm i\ln(2+\sqrt3),\quad k\in\BZ.\]
\end{solution}

\begin{solution}[另解]
	由 $\sin z=2$ 可知
	\[\cos z=\sqrt{1-\sin^2 z}=\pm\sqrt 3i.\]
	于是 $e^{iz}=\cos z+i\sin z=(2\pm\sqrt 3)i$,
		\[z=-i\Ln[(2\pm\sqrt 3)i]=\left(2k+\half\right)\pi\pm i\ln(2+\sqrt3),\quad k\in\BZ.\]
\end{solution}
我们总有形式
\begin{align*}
	\Arcsin z&=(2k+\half)\pi\pm \theta,\\
	\Arccos z&=2k\pi\pm \theta,\\
	\Arctan z&=k\pi+\theta,\quad k\in\BZ.
\end{align*}


\sectionHomework
\begin{homework}
	\item 判断题.
		\begin{exlist}
			\item 如果 $f'(z_0)$ 存在, 那么 $f(z)$ 在 $z_0$ 解析.\fillbrace{}
			\item 如果 $z_0$ 是 $f(z)$ 的奇点, 那么 $f(z)$ 在 $z_0$ 不可导.\fillbrace{}
			\item 如果 $z_0$ 是 $f(z)$ 和 $g(z)$ 的奇点, 那么 $z_0$ 也是 $f(z)+g(z)$ 和 $f(z)/g(z)$ 的奇点.\fillbrace{}
			\item 如果 $u(x,y)$ 和 $v(x,y)$ 偏导数均存在, 那么 $f(z)=u+iv$ 亦可导.\fillbrace{}
      \item 如果 $f(z)$ 在区域 $D$ 内处处可导, 则 $f(z)$ 在区域 $D$ 解析. \fillbrace{}
			\item 对任意复数 $z$, 有 $\ov{e^z}=e^{\ov z}$.\fillbrace{}
			\item 对任意复数 $z$, 有 $\ov{\cos z}=\cos{\ov z}$.\fillbrace{}
			\item 对任意复数 $z$, 有 $\ov{\sin z}=\sin{\ov z}$.\fillbrace{}
			\item 对任意复数 $z$, 有 $\ch^2z-\sh^2z=1$.\fillbrace{}
		\end{exlist}
	\item 选择题.
		\begin{exlist}
			\item 函数 $f(z)$ 在点 $z_0$ 的邻域内可导是 $f(z)$ 在该邻域内解析的\fillbrace{}.
				\begin{taskschoice}(2)
					() 充分条件
					() 必要条件
					() 充要条件
					() 既非充分也非必要条件
				\end{taskschoice}
			\item 设 $f(z)=u(x,y)+iv(x,y)$. 将下述选项不重复地填入括号内:
			\[\fillbrace{}\implies
				\fillbrace{}\implies
				\fillbrace{}\implies
				\fillbrace{}\implies
				\fillbrace{}\implies
				\fillbrace{}\]
				\begin{taskschoice}(2)
					() $f(z)$ 在点 $z_0$ 有定义
					() $f(z)$ 在点 $z_0$ 连续
					() $f(z)$ 在点 $z_0$ 可导
					() $f(z)$ 在点 $z_0$ 解析
					() $f(z)$ 在点 $z_0$ 的一个邻域内解析
					() $u,v$ 均在点 $(x_0,y_0)$ 处有偏导数
				\end{taskschoice}
			\item 下列函数中, 为解析函数的是\fillbrace{}.
				\begin{taskschoice}(2)
					() $x^2-y^2-2xyi$
					() $x^2+xyi$
					() $2(x-1)y+i(y^2-x^2+2x)$
					() $x^3+iy^3$
				\end{taskschoice}
			\item 设 $n$ 是正整数, $z_1,z_2\neq0$. 下列式子一定正确的是\fillbrace{}.
				\begin{taskschoice}(2)
					() $\Arg(\sqrt z)=\dfrac12\Arg z$
					() $\Arg(z^n)=n\Arg z$
					() $\Arg(z_1z_2)=\Arg(z_1)+\Arg(z_2)$
					() $\arg(\sqrt z)=\dfrac12\arg z$
					() $\arg(z^{-n})=-n\arg z$
					() $\arg(z_1/z_2)=\arg(z_1)-\arg(z_2)$
					() $\Ln\sqrt[3] z=\dfrac13\Ln z$
					() $\Ln(z^{-n})=-n\Ln z$
					() $\Ln(z_1z_2)=\Ln(z_1)+\Ln(z_2)$
					() $\ln\sqrt z=\dfrac12\ln z$
					() $\ln(z^n)=n\ln z$
					() $\ln(z_1/z_2)=\ln(z_1)-\ln(z_2)$
				\end{taskschoice}
		\end{exlist}
	\item 填空题.
		\begin{exlist}
			\item 函数 $\dfrac{z+1}{z(z^2+1)}$ 的奇点为\fillblank{}.
			\item 函数 $\dfrac{z-2}{(z+1)^2(z^2+1)}$ 的奇点为\fillblank{}.
			\item 函数 $\dfrac{1}{\sin z}$ 的奇点为\fillblank{}.
			\item 如果函数 $f(z)=x^2-2xy-y^2+i(ax^2+bxy+cy^2)$ 在复平面上处处解析, 则 $a+b+c=$\fillblank{}.
			\item 计算 $\ln i=$\fillblank[2cm][3mm]{}.
			\item 设 $z=1^{\sqrt3}$, 则 $|z|=$\fillblank{}.
			\item $i^{-i}$ 的主值是\fillblank{}.
		\end{exlist}
  \item 计算题.
		\begin{exlist}
			\item 设 $f(z)=\dfrac15z^5-(1+i)z$, 解方程 $f'(z)=0$.
			\item 下列函数何处可导? 何处解析?
				\begin{tasks}(3)
					\task $f(z)=1/\ov z$;
					\task* $f(z)=x^3-3xy^2+i(3x^2y-y^3)$;
					\task $f(z)=2x^3+3y^3i$;
					\task $f(z)=xy^2+ix^2y$;
					\task $f(z)=e^{x^2+y^2}$;
					\task $f(z)=z\Im z$;
					\task* $f(z)=\sin x\ch y+i\cos x\sh y$.
				\end{tasks}
			\item 指出下列函数 $f(z)$ 的解析区域, 并求出其导数.
				\begin{tasks}(2)
					\task $(z-1)^5$;
					\task $z^3+2iz$;
					\task $\dfrac1{z^2-1}$;
					\task $\dfrac{az+b}{cz+d}$ ($c,d$ 不全为零).
				\end{tasks}
			\item 设 $my^3+nx^2y+i(x^3+lxy^2)$ 为解析函数, 试确定实数 $l,m,n$ 的值.
			\item 计算
				\begin{tasks}(4)
					\task $\Ln 4$;
					\task $2\Ln2$;
					\task $\ln(-i)$;
					\task $\ln(-3+4i)$;
					\task $\Im\sin(1+i)$;
					\task $\arg e^{1-4i}$;
					\task $\exp\left(1-\dfrac{\pi i}2\right)$;
					\task $\exp\left(\dfrac{1+\pi i}4\right)$;
					\task $3^i$;
					\task $(1+i)^i$.
				\end{tasks}
			\item 解方程 
				\begin{tasks}(3)
					\task $\sin z=0$;
					\task $\cos z=0$;
					\task $1+e^z=0$;
					\task $\sin z+\cos z=0$;
					\task $\sin z=2\cos z$.
				\end{tasks}
			\item 复变函数 $f(z)=\sin z$ 和实变量函数 $g(x)=\sin x$ 的性质有什么相似和不同之处? 试说出3点.
		\end{exlist}
\end{homework}
	

\sectionExtraReading
\begin{homework}
	\item 仿照复数的指数函数, 我们可以尝试在矩阵上定义指数函数. 设 $\bfA\in M_m(\BC)$ 是一个 $m\times m$ 的复矩阵, 我们想说明极限
	\[e^\bfA:=\lim_{n\to \infty}\left(1+\frac1n \bfA\right)^n\]
	存在.
	\begin{exlist}
		\item 当 $\bfA=\diag\set{a_1,a_2,\dots,a_m}$ 是一个对角矩阵时, 证明 $e^\bfA$ 存在且
		\[e^\bfA=\diag\set{e^{a_1},e^{a_2},\dots,e^{a_m}}.\]
		\item 当
		\[\bfA=\bfJ_m(a)=\begin{pmatrix}
			a&1&   &\\
			&a& 1 &\\
			& &\ddots&1\\
			& &      &a
		\end{pmatrix}\]
		是约当块时, 证明 $e^\bfA$ 存在.
		\item 每个方阵都可以相似于一些约当块构成的分块对角阵, 由此证明 $e^\bfA$ 总存在.
		\item 当 $\bfA=x\bfE+y\bfJ=\begin{pmatrix}
			x&y\\-y&x
		\end{pmatrix}$ 时, 证明 $e^\bfA=\begin{pmatrix}
			e^x\cos y&e^x\sin y\\-e^x\sin y&e^x\cos y
		\end{pmatrix}=e^x(cos y\bfE+\sin y \bfJ)$.
		\item 证明 $e^\bfA=\bfE+\bfA+\dfrac{\bfA^2}{2!}+\dfrac{\bfA^3}{3!}+\cdots$.
		\item 证明 $e^{\bfA+\bfB}=e^\bfA\cdot e^\bfB$.
	\end{exlist}
	
	\item 注意到 $x=\dfrac12z+\dfrac12\ov z,y=-\dfrac i2z+\dfrac i2\ov z$.
	仿照着二元实函数偏导数在变量替换下的变换规则, 我们定义 $f$ 对 $z$ 和 $\ov z$ 的偏导数为
	\[\left\{
	\begin{aligned}
		\frac{\partial f}{\partial z}&
	=\frac{\partial x}{\partial z}\frac{\partial f}{\partial x}
		+\frac{\partial y}{\partial z}\frac{\partial f}{\partial y}
	=\frac12\frac{\partial f}{\partial x}-\frac i2\frac{\partial f}{\partial y},\\
		\frac{\partial f}{\partial \ov z}&
	=\frac{\partial x}{\partial \ov z}\frac{\partial f}{\partial x}
		+\frac{\partial y}{\partial \ov z}\frac{\partial f}{\partial y}
	=\frac12\frac{\partial f}{\partial x}+\frac i2\frac{\partial f}{\partial y}.
	\end{aligned}\right.\]
	\begin{exlist}
	\item 证明C-R方程等价于 $\dfrac{\partial f}{\partial \ov z}
	=0$.
	所以我们也可以把 \emph{$\dfrac{\partial f}{\partial \ov z}=0$} 叫做C-R方程.

	\item 利用该结论求函数 $f(z)=\ov z, z\Im z, e^{z\ov z}$ 的可导点和解析点.
	\end{exlist}
\end{homework}



\sectionExerciseAnswer
\exans 处处不可导.
\exans A. 因为解析要求在 $z_0$ 的一个邻域内都可导才行.
\exans A. 根据C-R方程可知对于A, $u_x(0)=2\neq v_y(0)=3$. 对于BD, 各个偏导数在 $0$ 处取值都是 $0$. C则是处处都可导.

\exans $\ln 2-\dfrac{2\pi i}3$.
\exans $\ln 3$.



