
\chapter{解析函数}
\section{解析函数的概念}

\subsection{可导的函数}

由于 $\BC$ 和 $\BR$ 一样是域, 我们可以像单变量实函数一样去定义复变函数的导数和微分.

\begin{definition}
  设 $w=f(z)$ 在 $z_0$ 的邻域内有定义.
  如果极限
  \[
     \lim_{z\to z_0}\frac{f(z)-f(z_0)}{z-z_0}
    =\lim_{\Delta z\to 0}\frac{f(z_0+\Delta z)-f(z_0)}{\Delta z}
  \]
  存在,则称 \nouns{$f(z)$ 在 $z_0$ 可导}{可导}.
  这个极限值称为 \nouns{$f(z)$ 在 $z_0$ 的导数}{导数},记作
  \[
    f'(z_0)=\lim_{\Delta z\to 0}\frac{f(z_0+\Delta z)-f(z_0)}{\Delta z}.
  \]
  如果 $f(z)$ 在区域 $D$ 内处处可导, 称 \nouns{$f(z)$ 在 $D$ 内可导}{可导}.
\end{definition}

需要注意的是, 无论极限过程 $z\to z_0$ 中 $z$ 是沿何种方式趋于 $z_0$, 比值 $\dfrac{f(z_0+\Delta z)-f(z_0)}{\Delta z}$ 的极限都要存在且全都相等. 因此尽管复变函数导数定义的形式和单变量实函数情形类似, 但其限制实际上要严格得多.

\begin{example}
  函数 $f(z)=x+2yi$ 在哪些点处可导?
\end{example}

\begin{solution}
  由定义可知
  \begin{align*}
    f'(z)&=\lim_{\Delta z\to 0}\frac{f(z+\Delta z)-f(z)}{\Delta z}\\
    &{=\lim_{\Delta z\to 0}\frac{(x+\Delta x)+2(y+\Delta y)i-(x+2yi)}{\Delta z}}\\
    &{=\lim_{\Delta z\to 0}\frac{\Delta x+2\Delta y i}{\Delta x+\Delta yi}.}
  \end{align*}
  当 $\Delta x=0, \Delta y\to 0$ 时, 上式$\to2$;当 $\Delta y=0, \Delta x\to 0$ 时, 上式$\to1$.因此该极限不存在, $f(z)$ 处处不可导.
\end{solution}

\begin{exercise}
  函数 $f(z)=\ov z=x-yi$ 在哪些点处可导? 
\end{exercise}

可以看出, 即使 $f(z)=u+iv$ 的实部和虚部在 $z_0$ 都有偏导数, 甚至都可微, 也无法保证 $f(z)$ 在 $z_0$ 处可导.
我们还需要额外的条件来保证可导性, 具体会在下一节中介绍.

\begin{example}
  求 $f(z)=z^2$ 的导数.
\end{example}

\begin{solution}
  由定义可知
  \begin{align*}
  f'(z)&=\lim_{\Delta z\to 0}\frac{f(z+\Delta z)-f(z)}{\Delta z}
   =\lim_{\Delta z\to 0}\frac{(z+\Delta z)^2-z^2}{\Delta z}\\
  &=\lim_{\Delta z\to 0}(2z+\Delta z)=2z.
  \end{align*}
\end{solution}

事实上, 和单变量实函数情形类似, 复变函数也有如下求导法则.
由此可知, 多项式函数处处可导, 有理函数在其定义域内处处可导, 且其导数形式和单变量实函数情形类似.
\begin{theorem}
  \begin{enumpar}
    \item $(c)'=0$, 其中 $c$ 为复常数;
    \item $(z^n)'=nz^{n-1}$, 其中 $n$ 为整数;
    \item $(f\pm g)'=f'\pm g',\quad (cf)'=cf'$;
    \item $(fg)'=f'g+fg',\quad \left(\dfrac fg\right)'=\dfrac{f'g-fg'}{g^2}$;
    \item $\Bigl(f\bigl(g(z)\bigr)\Bigr)'=f'\bigl(g(z)\bigr)\cdot g'(z)$;
    \item $g'(z)=\dfrac1{f'(w)}, g=f^{-1}, w=g(z)$.
  \end{enumpar}
\end{theorem}

根据上述求导法则, 不难知道:
\begin{theorem}\label{thm:four-derivable}
  \begin{enumpar}
    \item 在 $z_0$ 处可导的两个函数 $f(z)$, $g(z)$ 之和、差、积、商($g(z_0)\neq 0$) 仍然在 $z_0$ 处可导.
    \item 如果函数 $g(z)$ 在 $z_0$ 处可导, 函数 $f(w)$ 在 $g(z_0)$ 处可导, 则 $f\bigl(g(z)\bigr)$ 在 $z_0$ 处可导.
  \end{enumpar}
\end{theorem}

\begin{theorem}
  若 $f(z)$ 在 $z_0$ 可导, 则 $f(z)$ 在 $z_0$ 连续.
\end{theorem}
即可导蕴含连续. 该定理的证明和单变量实函数情形完全相同.
\begin{proof}
  设
    \[\Delta w=f(z_0+\Delta z)-f(z_0),\]
  则
    \[
      \lim_{\Delta z\to 0}\Delta w
      =\lim_{\Delta z\to 0}\frac{\Delta w}{\Delta z}\cdot\Delta z
      =\lim_{\Delta z\to 0}\frac{\Delta w}{\Delta z}\cdot
        \lim_{\Delta z\to 0}\Delta z
      =f'(z_0)\cdot 0=0.
    \]
  从而 $f(z)$ 在 $z_0$ 连续.
\end{proof}


\subsection{可微的函数}


\begin{definition}
  如果存在常数 $A$ 使得函数 $w=f(z)$ 满足
    \[\Delta w=f(z_0+\Delta z)-f(z_0)=A\Delta z+o(\Delta z),\]
  其中 $o(\Delta z)$ 表示 $\Delta z$ 的高阶无穷小量,
  则称 \nouns{$f(z)$ 在 $z_0$ 处可微}{可微}, 称 $A\Delta z$ 为 \nouns{$f(z)$ 在 $z_0$ 的微分}{微分}, 记作 $\diff w=A \Delta z$.
\end{definition}

同导数一样, 复变函数微分的定义也和单变量实函数情形类似, 而且复变函数的可微和可导也是等价的, 且 $\diff w=f'(z_0)\Delta z, \diff z=\Delta z$.
故
  \[\diff w=f'(z_0)\diff z,\qquad f'(z_0)=\dfrac{\diff w}{\diff z}.\]
微分 $\diff w$ 是 $f(z)$ 在 $z_0$ 处的线性近似.

\subsection{解析的函数}

\begin{definition}
  \begin{enumpar}
    \item 若函数 $f(z)$ 在 $z_0$ 的一个邻域内处处可导, 则称 \nouns{$f(z)$ 在 $z_0$ 解析}{解析}.
    \item 若 $f(z)$ 在区域 $D$ 内处处解析, 则称 $f(z)$ 在 $D$ 内解析, 或称 $f(z)$ 是 $D$ 内的一个\noun{解析函数}\footnotemark.
    \item 若 $f(z)$ 在 $z_0$ 不解析, 则称 $z_0$ 为 $f(z)$ 的一个\noun{奇点}.
  \end{enumpar}
\end{definition}
\footnotetext{也可叫\emph{全纯函数}或\emph{正则函数}.}
由定义可知, 若 $f(z)$ 在 $z_0$ 解析, 则 $f(z)$ 在 $z_0$ 可导, 但反过来不成立.
无定义、不连续、不可导、可导但不解析, 都会导致奇点的产生.
不过, 若 $z_0$ 是 $f(z)$ 定义域的外点, 即存在 $z_0$ 的邻域与 $f(z)$ 定义域交集为空集, 这种情形不甚有趣, 因此我们不考虑这类奇点.

由于区域 $D$ 是一个开集, 其中的任意 $z_0\in D$ 均存在一个包含在 $D$ 的邻域. 所以 \emph{$f(z)$ 在 $D$ 内解析和在 $D$ 内可导是等价的}.
由于一个点的邻域也是一个开集, 因此若 $f(z)$ 在 $z_0$ 解析, 则 $f(z)$ 在 $z_0$ 的一个邻域内处处可导, 从而在该邻域内解析. 因此 \emph{$f(z)$ 解析点全体是一个开集}, 它是可导点集合的内点构成的集合.

\begin{exercise}
  函数 $f(z)$ 在点 $z_0$ 的邻域内解析是 $f(z)$ 在该邻域处处可导的\fillbrace{}.
  \begin{taskschoice}(2)
    \task 充分条件
    \task 必要条件
    \task 充要条件
    \task 既非充分也非必要条件
  \end{taskschoice}
\end{exercise}

\begin{example}
  研究函数 $f(z)=|z|^2$ 的解析性.
\end{example}
\begin{solution}
  注意到
  \[
    \frac{f(z+\Delta z)-f(z)}{\Delta z}
    =\frac{(z+\Delta z)(\ov z+\ov{\Delta z})-z\ov z}{\Delta z}
    =\ov z+\ov{\Delta z}+z\frac{\Delta x-\Delta yi}{\Delta x+\Delta yi}.
  \]
  \begin{itemize}
    \item 若 $z=0$, 则当 $\Delta z\to 0$ 时该极限为 $0$.
    \item 若 $z\neq0$, 则当 $\Delta y=0,\Delta x\to 0$ 时该极限为 $\ov z+z$; 当 $\Delta x=0,\Delta y\to 0$ 时该极限为 $\ov z-z$.因此此时极限不存在.
  \end{itemize}
  故 $f(z)$ 仅在 $z=0$ 处可导, 从而处处不解析.
\end{solution}

由定理~\ref{thm:four-derivable} 不难证明:
\begin{theorem}
  \begin{enumpar}
    \item 在 $z_0$ 处解析的两个函数 $f(z)$, $g(z)$ 之和、差、积、商($g(z_0)\neq 0$) 仍然在 $z_0$ 处解析.
    \item 在 $D$ 内解析的两个函数 $f(z)$, $g(z)$ 之和、差、积、商仍然在 $D$ (作商时需要去掉 $g(z)$ 的零点) 内解析.
    \item 如果函数 $g(z)$ 在 $z_0$ 处解析, 函数 $f(w)$ 在 $g(z_0)$ 处解析, 则 $f\bigl(g(z)\bigr)$ 在 $z_0$ 处解析.
    \item 如果函数 $g(z)$ 在 $D$ 内解析, 函数 $f(w)$ 在 $g(z_0)$ 处解析, 则 $f\bigl(g(z)\bigr)$ 在 $D$ 内解析.
  \end{enumpar}
\end{theorem}

由此可知, 多项式函数处处解析. 有理函数在其定义域内处处解析, 分母的零点是它的奇点.
我们来看复变函数在实变量函数求导中的一个应用.
\begin{example}
  设 $f(z)=\dfrac1{1+z^2}$, 则它在除 $z=\pm i$ 外处处解析.
  当 $z=x$ 为实数时,
  \begin{align*}
    \biggl(\frac1{1+x^2}\biggr)^{(n)}&
      =f^{(n)}(x)=\frac i2\biggl(\frac1{x+i}-\frac1{x-i}\biggr)^{(n)}\\
    &=\frac i2\cdot(-1)^n n!\biggl(\frac1{(x+i)^{n+1}}-\frac1{(x-i)^{n+1}}\biggr)\\
    &=(-1)^{n+1}n!\Im\frac1{(x+i)^{-n-1}}
  \end{align*}
  由于
  \[
    |x+i|=\sqrt{x^2+1},\qquad
    \arg(x+i)=\arccot x,
  \]
  因此
  \[
    \biggl(\frac1{1+x^2}\biggr)^{(n)}
    =(-1)^nn!(x^2+1)^{-\frac{n+1}2}\sin\bigl((n+1)\arccot x\bigr).
  \]
\end{example}
如果有理函数 $\dfrac{P(z)}{Q(z)}$ 分母的零点均能求出, 则可将其拆分为一个多项式和一些形如 $\dfrac{a}{(x-b)^k}$ 的分式之和, 从而其高阶导数可按此方法计算.

\section{函数解析的充要条件}

\subsection{柯西-黎曼定理}

在上一节中, 通过对一些简单函数的分析, 我们发现可导的函数往往可以直接表达为 $z$ 的函数的形式, 而不解析的往往包含 $x,y,\ov z$ 等内容.
这种现象并不是偶然的.
我们来研究二元实变量函数的可微性与复变函数可导的关系.
为了简便我们用
\[
  u_x=\dpp ux,\quad
  u_y=\dpp uy,\quad
  v_x=\dpp vx,\quad
  v_y=\dpp vy
\]
等记号表示偏导数.

设 $f$ 在 $z=x+yi$ 处可导, $f'(z)=a+bi$.
对于充分小 $|\Delta z|>0$, 令
\[
   \Delta f
  =f(z+\Delta z)-f(z)
  =\Delta u+i\Delta v.
\]
由于 $f$ 在 $z$ 处可微, 因此
\[
   \Delta f
  =\Delta u+i\Delta v
  =(a+bi)(\Delta x+i\Delta y)+o(\Delta z).
\]
设 $\rho=|\Delta z|=\sqrt{\Delta x^2+\Delta y^2}$.
由于 $\Delta z$ 的高阶无穷小量 $o(\Delta z)=o(\rho)$ 的实部和虚部也是 $\rho$ 的高阶无穷小量, 展开可知
\begin{align*}
  \Delta u&=a\Delta x-b\Delta y+o(\rho),\\
  \Delta v&=b\Delta x+a\Delta y+o(\rho),
\end{align*}
因此 $u,v$ 在 $(x_0,y_0)$ 处可微且 $u_x=v_y=a,v_x=-u_y=b$.

反过来, 假设 $u,v$ 在 $(x_0,y_0)$ 处可微且 $u_x=v_y, v_x=-u_y$. 由全微分公式
\begin{align*}
  \Delta u&=u_x\Delta x+u_y\Delta y+o(\rho)
    =u_x\Delta x-v_x\Delta y+o(\rho),\\
  \Delta v&=v_x\Delta x+v_y\Delta y+o(\rho)
    =v_x\Delta x+u_x\Delta y+o(\rho),\\
  \Delta f&=\Delta u+i\Delta v
    =(u_x+i v_x)\Delta x+(-v_x+i u_x)\Delta y+o(\rho)\\
   &=(u_x+i v_x)\Delta(x+iy)+o(\rho)
    =(u_x+i v_x)\Delta z+o(\rho).
\end{align*}
故 $f(z)$ 在 $z$ 处可微, 从而可导, 且 $f'(z)=u_x+i v_x=v_y-i u_y$.

由此得到:
\begin{theorem}[柯西-黎曼定理]
  $f(z)$ 在 $z=x+yi$ 处可导当且仅当 $u,v$ 在 $(x,y)$ 处可微, 且满足\noun{柯西-黎曼方程}:
    \[u_x=v_y,\quad v_x=-u_y.\]
  此时
    \[f'(z)=u_x+iv_x=v_y-iu_y.\]
\end{theorem}

\begin{figure}[!h]
  \centering
  \includegraphics[height=40mm]{../image/Cauchy.jpeg}
  \hspace{5em}
  \includegraphics[height=40mm]{../image/Riemann.jpeg}
  \caption{柯西和黎曼}
\end{figure}

柯西\footnote{%
  Augustin-Louis Cauchy (1789--1857), 法国数学家、物理学家.
}-黎曼方程可简称为 C-R 方程.

当 $f$ 可导时其导数形式也可直接看出.
当极限
\[
  \lim\limits_{\Delta z\to 0}\dfrac{\Delta u+i\Delta v}{\Delta z}=f'(z)
\]
存在时, 它沿着水平方向和竖直方向的极限
\[
  \lim\limits_{\Delta x\to 0}\dfrac{\Delta u+i\Delta v}{\Delta x}=u_x+iv_x,\qquad
  \lim\limits_{\Delta y\to 0}\dfrac{\Delta u+i\Delta v}{\Delta y i}=-iu_y+v_y
\]
也都存在, 且三者极限相同.
因此 $f'(z)=u_x+iv_x=-iu_y+v_y$.

下面我们来介绍柯西-黎曼方程的等价形式.
注意到
\[
  x=\dfrac12z+\dfrac12\ov z,\qquad
  y=-\dfrac i2z+\dfrac i2\ov z.
\]
仿照着二元实函数偏导数在变量替换下的变换规则, 定义 $f$ 对 $z$ 和 $\ov z$ 的偏导数为
  \[\begin{aligned}
      \dpp fz&
    =\dpp xz\dpp fx+\dpp yz\dpp fy
    =\frac12\dpp fx-\frac i2\dpp fy,\\
      \dpp f{\ov z}&
    =\dpp x{\ov z}\dpp fx+\dpp y{\ov z}\dpp fy
    =\frac12\dpp fx+\frac i2\dpp fy.
  \end{aligned}\]
和前面的计算类似可知: 当 $f$ 在 $z$ 处可导时,
\[
  \Delta f=f'(z)\Delta z+o(\rho).
\]
当 $u,v$ 可微时,
\[
  \Delta f=\dpp fz\Delta z+\dpp f{\ov z}\Delta\ov z+o(\rho).
\]
由于极限 $\lim\limits_{\Delta\to 0}\dfrac{\Delta \ov z}{\Delta z}$ 不存在, 因此:
\begin{theorem}[柯西-黎曼定理的等价形式]
  $f(z)$ 在 $z=x+yi$ 处可导当且仅当 $u,v$ 在 $(x,y)$ 处可微, 且满足\noun{柯西-黎曼方程}:
    \[\dpp f{\ov z}=0.\]
  此时
    \[f'(z)=\dpp fz.\]
\end{theorem}
从该定理便可解释, 为何含有 $x,y,\ov z$ 形式的函数往往不可导, 而可导的函数往往可以直接表达为 $z$ 的形式.

由于二元函数的偏导数均连续蕴含可微, 因此我们有:
\begin{theorem}
  \begin{enumpar}
    \item 如果 $u_x,u_y,v_x,v_y$ 在 $(x,y)$ 处连续, 且满足C-R方程, 则 $f(z)$ 在 $z=x+yi$ 处可导.
    \item 如果 $u_x,u_y,v_x,v_y$ 在区域 $D$ 上处处连续, 且满足C-R方程, 则 $f(z)$ 在 $D$ 上处处可导, 从而解析.
  \end{enumpar}
\end{theorem}
尽管这些条件不是充要条件, 但在实际应用中, 确实很多情形下 $u_x,u_y,v_x,v_y$ 是处处连续的.


\subsection{柯西-黎曼定理的应用}

在下面几个例子中, 我们利用柯西黎曼定理来研究函数的可导性和解析性.
\begin{example}
  求下列函数的可导点和解析区域.
  \begin{tasksexam}(2)
    \task {$f(z)=\ov z$;}
    \task {$f(z)=\ov z(2z+\ov z)$;}
    \task {$f(z)=e^{|z|^2}$;}
    \task {$f(z)=e^x(\cos y+i\sin y)$.}\label{enum:exp}
  \end{tasksexam}
\end{example}

\begin{solution}\delspace
  \begin{enumnopar}[(i)]
    \item 由 $u=x,v=-y$ 可知
      \begin{alignat*}{2}
        u_x&=1,\qquad&u_y&=0,\\
        v_x&=0,\qquad&v_y&=-1.
      \end{alignat*}
      这些偏导数都是连续的.
      因为 $u_x=1\neq v_y=-1$, 所以该函数处处不可导, 处处不解析.

      也可从 $\dpp f{\ov z}=1\neq0$ 看出.
    \item 由 $f(z)=3x^2+y^2-2xyi,u=3x^2+y^2,v=-2xy$ 可知
      \begin{alignat*}{2}
        u_x&=6x,\qquad&u_y&=2y,\\
        v_x&=-2y, \qquad&v_y&=-2x.
      \end{alignat*}
      这些偏导数都是连续的.
      由 $u_x=v_y,v_x=-u_y$ 可知只有 $x=\Re z=0$ 满足C-R方程.
      因此该函数在虚轴上可导, 处处不解析且 $f'(yi)=(u_x+iv_x)|_{(0,0)}=-2yi$.

      也可从 $\dpp f{\ov z}=2z+2\ov z=4x$ 看出, 且有 $f'(yi)=\dpp fz\Big|_{z=yi}=2\ov z|_{z=yi}=-2yi$.
    \item 由 $f(z)=e^{x^2+y^2},u=e^{x^2+y^2},v=0$ 可知
      \begin{alignat*}{2}
        u_x&=2xe^{x^2+y^2},\qquad&u_y&=2ye^{x^2+y^2},\\
        v_x&=0, \qquad&v_y&=0.
      \end{alignat*}
      这些偏导数都是连续的.
      由 $u_x=v_y,v_x=-u_y$ 可知只有 $x=y=0,z=0$ 满足C-R方程.
      因此该函数只在 $0$ 可导, 处处不解析且 $f'(0)=(u_x+iv_x)|_{(0,0)}=0$.

      也可从 $f(z)=e^{z\ov z}, \dpp f{\ov z}=ze^{z\ov z}$ 看出, 且有 $f'(0)=\dpp fz\Big|_{z=0}=\ov ze^{z\ov z}|_{z=0}=0$.
    \item 由 $u=e^x\cos y,v=e^x\sin y$ 可知
      \begin{alignat*}{2}
        u_x&=e^x\cos y,\qquad&u_y&=-e^x\sin y,\\
        v_x&=e^x\sin y,\qquad&v_y&=e^x\cos y.
      \end{alignat*}
      这些偏导数都是连续的, 且处处满足C-R方程.
      因此该函数处处可导, 处处解析, 且
      \[f'(z)=u_x+iv_x=e^x(\cos y+i\sin y)=f(z).\]
  \end{enumnopar}
\end{solution}

我们发现, \ref{enum:exp} 中的函数满足导数等于自身, 后面我们会看到它就是复变量的指数函数 $e^z$.

\begin{exercise}
  函数\fillbrace{}在 $z=0$ 处不可导.
  \begin{taskschoice}(4)
    \task $2x+3yi$
    \task $2x^2+3y^2i$
    \task $x^2-xyi$
    \task $e^x\cos y+i e^x\sin y$
  \end{taskschoice}
\end{exercise}

\begin{example}
  设函数 $f(z)=(x^2+axy+by^2)+i(cx^2+dxy+y^2)$ 在复平面内处处解析. 求实常数 $a,b,c,d$ 以及 $f'(z)$.
\end{example}
\begin{solution}
  注意到
  \begin{alignat*}{2}
    u_x&=2x+ay,\qquad&u_y&=ax+2by,\\
    v_x&=2cx+dy,\qquad&v_y&=dx+2y.
  \end{alignat*}
  由C-R方程可知
    \[2x+ay=dx+2y,\quad ax+2by=-(2cx+dy),\]
  因此 $a=d=2$, $b=c=-1$, 且
    \[f'(z)=u_x+iv_x=2x+2y+i(-2x+2y)=(2-2i)z.\]
\end{solution}

\begin{example}\label{exam:zero-deriv-constant}
  证明: 如果 $f'(z)$ 在区域 $D$ 内处处为零, 则 $f(z)$ 在 $D$ 内是一常数.
\end{example}

\begin{proof}
  由于
    \[f'(z)=u_x+iv_x=v_y-iu_y=0,\]
  因此 $u_x=v_x=u_y=v_y=0$, $u,v$ 均为常数,从而  $f(z)=u+iv$ 是常数.
\end{proof}
类似地可以证明, 若 $f(z)$ 在 $D$ 内解析, 则下述任一条件均可推出 $f(z)$ 是一常数:
\begin{tasksexam}(2)
  \task {$\arg{f(z)}$ 是一常数;}
  \task {$|f(z)|$ 是一常数;}
  \task {$\Re{f(z)}$ 是一常数;}
  \task {$\Im{f(z)}$ 是一常数;}
  \task {$v=u^2$;}
  \task {$u=v^2$.}
\end{tasksexam}

\begin{example}
  证明: 如果 $f(z)$ 解析且 $f'(z)$ 处处非零, 则曲线族 $u(x,y)=c_1$ 和曲线族 $v(x,y)=c_2$ 互相正交.
\end{example}

\begin{proof}
  由于 $f'(z)=u_x-iu_y$, 因此 $u_x,u_y$ 不全为零.
  对 $u(x,y)=c_1$ 使用隐函数求导法则得 $u_x\diff x+u_y\diff y=0$,从而 $(u_y,-u_x)$ 是该曲线在 $z$ 处的非零切向量.

  同理 $(v_y,-v_x)$ 是 $v(x,y)=c_2$ 在 $z$ 处的非零切向量.由于
    \[u_yv_y+u_xv_x=u_yu_x-u_xu_y=0,\]
  因此这两个切向量正交, 从而曲线正交.
\end{proof}

当 $f'(z_0)\neq 0$ 时, 经过 $z_0$ 的两条曲线 $C_1,C_2$ 的夹角和它们的像 $f(C_1),f(C_2)$ 在 $f(z_0)$ 处的夹角总是相同的.
这种性质被称为\noun{保角性}.
这是因为 $\diff f=f'(z_0)\diff z$.
从复数乘法的几何意义可知, 局部上 $f$ 把 $z_0$ 附近的点以 $z_0$ 为中心放缩 $|f'(z_0)|$ 倍并逆时针旋转 $\arg{f'(z_0)}$.
由 $w$ 复平面上曲线族 $u=c_1,v=c_2$ 正交可知上述例题成立, 特别地, 例~\ref{exam:wz2} 中的曲线族 $x^2-y^2=c_1$, $2xy=c_2$ 正交.


\section{初等函数}

我们将实变函数中的初等函数推广到复变函数.
多项式函数和有理函数的解析性质已经介绍过, 这里不再重复.

\subsection{指数函数}

复指数函数有多种等价的定义方式:
\begin{enumerate}
  \item $\exp z=e^x(\cos y+i\sin y)$ (欧拉恒等式);\label{enum:exp-euler}
  \item $\exp z=\lim\limits_{n\to\infty}\left(1+\dfrac zn\right)^n$ (极限定义);\label{enum:exp-limit}
  \item $\exp z=1+z+\dfrac{z^2}{2!}+\dfrac{z^3}{3!}+\cdots
  =\lim\limits_{n\to\infty}\sum\limits_{k=0}^n\dfrac{z^k}{k!}$ (级数定义);\label{enum:exp-series}
  \item $\exp z$ 是唯一的一个处处解析的函数, 使得当 $z=x\in\BR$ 时, $\exp z=e^x$ (解析延拓).\label{enum:exp-expansion}
\end{enumerate}
这几种定义方式都是等价的.

若我们采用\ref{enum:exp-series} 来定义, 则从 $\cos x$ 和 $\sin x$ 的泰勒展开
\[
  \cos x=1-\frac{x^2}{2!}+\frac{x^4}{4!}+\cdots\quad
  \sin x=x-\frac{x^3}{3!}+\frac{x^5}{5!}\cdots
\]
可以得到欧拉恒等式 $e^{ix}=\cos x+i\sin x$.
然而, 这实际上需要我们先铺垫好复级数的基础理论, 这会在第四章例\ref{exam:exp-taylor-expansion}中得到解释.
而\ref{enum:exp-euler} 和\ref{enum:exp-expansion} 的等价性我们将在第五章由定理\ref{thm:zero-isolated}推出.

现在我们来证明\ref{enum:exp-euler} 和\ref{enum:exp-limit} 是等价的\footnote{%
  欧拉也是从实指数函数的极限定义
  \[e^x=\lim\limits_{n\to\infty}(1+\dfrac xn)^n\]
  得到复指数函数的极限定义, 并证明了欧拉恒等式.
  参考 \cite[第19章2,3节]{Kline1990}.
}.
\begin{align*}
  \lim_{n\to\infty}\abs{1+\frac zn}^n
  &=\lim_{n\to\infty}\left(1+\frac{2x}n+\frac{x^2+y^2}{n^2}\right)^{\frac n2}\quad
  {(1^\infty\ \text{型不定式})}\\
  &{=\exp\left[\lim_{n\to\infty}\frac n2
  \left(\frac{2x}n+\frac{x^2+y^2}{n^2}\right)\right]=e^x.}
\end{align*}
不妨设 $n>\abs{z}$, 这样 $1+\dfrac zn$ 落在右半平面,
  \[
    \lim_{n\to\infty} n\arg{\left(1+\frac zn\right)}
    =\lim_{n\to\infty} n\arctan \frac y{n+x}
    =\lim_{n\to\infty}\frac{ny}{n+x}=y.
  \]
故
  \[\lim_{n\to\infty}\left(1+\dfrac zn\right)^n=e^x(\cos y+i\sin y).\]

\begin{definition}{指数函数}
定义\noun{指数函数}
  \[\exp z:=e^x(\cos y+i\sin y).\]
\end{definition}
为了方便, 我们也记 \alert{$e^z=\exp z$}\index{$e^z$}\index{$\exp z$}.
指数函数有如下性质:
\begin{itemize}
  \item $\exp z$ 处处解析, 且 $(\exp z)'=\exp z$.
  \item $\exp z\neq 0$.
  \item $\exp(z_1+z_2)=\exp z_1\cdot \exp z_2$.
  \item $\exp(z+2k\pi i)=\exp z$, 即 $\exp z$ 周期为 $2\pi i$.
  \item $\exp z_1=\exp z_2$ 当且仅当 $z_1=z_2+2k\pi i,k\in\BZ$.
  \item $\exp z$ 将直线族 $\Re z=c$ 映为圆周族 $\abs{w}=e^c$, 将直线族 $\Im z=c$ 映为射线族 $\Arg w=c$.
\end{itemize}

\begin{example}
  计算函数 $f(z)=\exp(z/6)$ 的周期.
\end{example}
\begin{solution}
  设 $f(z_1)=f(z_2)$, 则 $\exp(z_1/6)=\exp(z_2/6)$.
  {因此存在 $k\in\BZ$ 使得
    \[\frac{z_1}6=\frac{z_2}6+2k\pi i,\]从而 $z_1-z_2=12k\pi i$.所以 $f(z)$ 的周期是 $12\pi i$.}
\end{solution}

一般地, $\exp(az+b)$ 的周期是 $\dfrac{2\pi i}a$ (或写成 $-\dfrac{2\pi i}a$), $a\neq 0$.


\subsection{对数函数}

对数函数 $\Ln z$ 定义为指数函数 $\exp z$ 的反函数.
为什么我们用大写的 $\Ln$ 呢? 
在复变函数中, 很多函数是多值函数.
为了便于研究, 我们会固定它的一个单值分支.
我们将多值的这个开头字母大写, 而对应的单值的则是开头字母小写.
例如 $\Arg z$ 和 $\arg z$.

设 $z\neq 0$, $e^w=z=re^{i\theta}=e^{\ln r+i\theta}$,
则
\[w=\ln r+i\theta+2k\pi i,\quad k\in\BZ.\]
\begin{definition}[对数函数]
  \begin{enumerate}
    \item 定义\noun{对数函数}\index{$\Ln z$}
      \[\Ln z=\ln\abs{z}+i\Arg z.\]
      它是一个多值函数.
    \item 定义\nouns{对数函数主值}{对数函数!对数函数主值}\index{$\ln z$}
      \[\ln z=\ln\abs{z}+i\arg z.\]
  \end{enumerate}
\end{definition}
对于每一个整数 $k$, $\ln z+2k\pi i$ 都给出了 $\Ln z$ 的一个单值分支.
特别地, 当 $z=x>0$ 是正实数时, $\ln z$ 就是实变的对数函数.

\begin{example}
  求 $\Ln 2,\Ln(-1)$ 以及它们的主值.
\end{example}

\begin{solution}
  \[\Ln2=\ln2+2k\pi i,\quad k\in\BZ,\]
  主值为 $\ln 2$.
  \[\Ln(-1)=\ln1+i\Arg(-1)=(2k+1)\pi i,\quad k\in\BZ,\]
  主值为 $\pi i$.
\end{solution}

\begin{example}
求 $\Ln(-2+3i),\Ln(3-\sqrt3 i)$.
\end{example}

\begin{solution}
  \[
    \Ln(-2+3i)=\ln\abs{-2+3i}+i\Arg(-2+3i)
      =\frac 12\ln 13+\left(-\arctan\frac 32+\pi+2k\pi\right)i,
      \quad k\in\BZ.
  \]
  \[
    \Ln(3-\sqrt3i)=\ln\abs{3+\sqrt 3i}+i\Arg(3-\sqrt 3i)
      =\ln 2\sqrt 3+\left(-\frac\pi6+2k\pi\right)i
      =\ln 2\sqrt 3+\left(2k-\frac16\right)\pi i,
      \quad k\in\BZ.
  \]
\end{solution}

\begin{example}
  解方程 $e^z-1-\sqrt 3i=0$.
\end{example}

\begin{solution}
  由于 $1+\sqrt 3 i=2e^{\frac{\pi i}3}$, 因此
  \[z=\Ln(1+\sqrt 3i)=\ln 2+\left(2k+\frac13\right)\pi i,\quad k\in\BZ.\]
\end{solution}

\begin{exercise}
  求 $\ln(-1-\sqrt3 i)=$\fillblank[2cm][3mm]{}.
\end{exercise}

对数函数与其主值的关系是
\[\Ln z=\ln z+\Ln 1=\ln z+2k\pi i,\quad k\in\BZ.\]
根据辐角以及辐角主值的相应等式, 我们有
\[\Ln(z_1\cdot z_2)=\Ln z_1+\Ln z_2,\quad
  \Ln\frac{z_1}{z_2}=\Ln z_1-\Ln z_2,\]
\[\Ln \sqrt[n]z=\dfrac1n\Ln z.\]
而当 $\abs{n}\ge 2$ 时, \alert{$\Ln z^n=n\Ln z$ 不成立}.
以上等式换成 $\ln z$ 均不一定成立.


设 $x$ 是正实数, 则
\[\ln (-x)=\ln x+\pi i,\quad
  \lim_{y\to0^-}\ln (-x+yi)=\ln x-\pi i,\]
因此 $\ln z$ 在负实轴和零处不连续.
而在其它地方, $-\pi<\arg z<\pi$, $\ln z$ 是 $e^z$ 在区域 $-\pi<\Im z<\pi$ 上的单值反函数, 
从而 \alert{$(\ln z)'=\dfrac 1z$}, \alert{$\ln z$ 在除负实轴和零处的区域解析}.
\footnote{任取一条从 $0$ 到 $\infty$ 的简单曲线, 在去掉这条曲线后, 若固定一复数 $z_0$ 的辐角, 则多值函数 $\Arg z$ 可以在该区域内连续单值化, 简单来说就是沿着 $z_0$ 到 $z$ 的曲线让辐角连续变化. 同理, $\Ln z$ 也可以在该区域内单值化, 只需固定一复数 $z_0$ 的值.}

也可以通过C-R方程来得到 $\ln z$ 的解析性和导数: 当 $x>0$ 时,
\[\ln z=\half \ln(x^2+y^2)+i\arctan \frac yx,\]
\[u_x=v_y=\frac x{x^2+y^2},\qquad v_x=-u_y=-\frac y{x^2+y^2},\]
\[(\ln z)'=\frac{x-yi}{x^2+y^2}=\frac 1z.\]
其它情形可取虚部为 $\arccot\dfrac xy$ 或 $\arccot\dfrac xy-\pi$ 类似证明.

\subsection{幂函数}

\begin{definition}{幂函数}
  \begin{enumerate}
    \item 设 $a\neq 0$, $z\neq 0$, 定义\noun{幂函数}
    \[w=z^a=e^{a\Ln z}
    =\exp\bigl(a\ln\abs{z}+ia(\arg z+2k\pi)\bigr),\quad k\in\BZ.\]
    \item \nouns{幂函数的主值}{幂函数!幂函数的主值}为
    \[w=e^{a\ln z}=\exp\bigl(a\ln\abs{z}+ia\arg z\bigr).\]
  \end{enumerate}
\end{definition}

根据 $a$ 的不同, 这个函数有着不同的性质.
\begin{enumerate}
  \item 当 $a$ 为整数时, 因为 $e^{2ak\pi i}=1$, 所以 $w=z^a$ 是单值的. 此时 $z^a$ 就是我们之前定义的乘幂. 
    当 $a$ 是非负整数时, $z^a$ 在复平面上解析;
    当 $a$ 是负整数时, $z^a$ 在 $\BC-\set0$ 上解析.
  \item 当 $a=\dfrac pq$ 为分数, $p,q$ 为互质的整数且 $q>1$ 时,
    \[z^{\frac pq}=\abs{z}^{\frac pq}\exp\biggl(\frac{ip(\arg z+2k\pi)}q\biggr),\quad k=0,1,\dots,q-1\]
    具有 $q$ 个值.
    去掉负实轴和 $0$ 之后, 它的主值 $w=\exp(a\ln z)$ 是处处解析的.
    事实上它就是 $\sqrt[q]{z^p}=(\sqrt[q]z)^p$.
    \begin{figure}[!h]
      \centering
      \begin{tikzpicture}
        \coordinate [label=below left:{$0$}] (O) at (0,0);
        \coordinate [label=below:{$x$}] (X) at (2,0);
        \coordinate [label=left:{$y$}] (Y) at (0,2);
        \draw[cstaxis] (O)--(X);
        \draw[cstaxis] (0,-2)--(Y);
        \draw[draw=white,cstfille1] (O) circle (1.3);
        \draw[cstdash,main] (0,0)--(-2,0);
        \draw[cstdash,cstra,third] (1.7,.7)to [bend left] (4.5,.7);
        \draw (3,1.7) node[third] {$w=z^{2/9}$};
        \begin{scope}[xshift=5cm]
          \coordinate [label=below left:{$0$}] (O) at (0,0);
          \fill[cstfille2,pattern color=second] (O)--({1.4*cos(40)},{1.4*sin(40)}) arc (40:-40:1.4)--cycle;
          \draw[cstdash,second] ({1.4*cos(40)},{-1.4*sin(40)})--(0,0)--({1.4*cos(40)},{1.4*sin(40)});
          \coordinate [label=below left:{$0$}] (O) at (0,0);
          \coordinate [label=below:{$u$}] (X) at (2,0);
          \coordinate [label=left:{$v$}] (Y) at (0,2);
          \draw[cstaxis] (-2,0)--(X);
          \draw[cstaxis] (0,-2)--(Y);
        \end{scope}
      \end{tikzpicture}
      \caption{映照 $w=z^{2/9}$}
    \end{figure}
  \item 对于其它的 $a$, $z^a$ 具有无穷多个值.
    这是因为此时当 $k\neq0$ 时, $2k\pi a i$ 不可能是 $2\pi i$ 的整数倍. 
    从而不同的 $k$ 得到的是不同的值.
    去掉负实轴和 $0$ 之后, 它的主值 $w=\exp(a\ln z)$ 也是处处解析的.
    \footnote{对于 $\Ln\dfrac{z-a}{z-b},\sqrt{(z-a)(z-b)}$ 等类型的多值函数, 我们需要将它的``奇点''连接起来形成``割线''. 复平面上去掉这些割线得到的区域内, 这些函数也可以如同 $\Arg z,\Ln z$ 那样单值化.}
\end{enumerate}

\begin{center}
  \begin{tabular}{cccc} \toprule
    $a$& $z^a$ 的值& $z^a$ 的解析区域\\ \midrule
    &&$n\ge0$ 时处处解析\\
    \multirow{-2}*{整数 $n$}&\multirow{-2}*{单值}&$n<0$ 时除零点外解析\\ \midrule
    分数 $p/q$&$q$ 值&除负实轴和零点外解析\\ \midrule
    无理数或虚数&无穷多值&除负实轴和零点外解析\\ \bottomrule
  \end{tabular}
\end{center}

\begin{example}
  求 $1^{\sqrt 2}$ 和 $i^i$.
\end{example}
\begin{solution}
  \[
    1^{\sqrt2}=e^{\sqrt2\Ln1}
      =e^{\sqrt 2\cdot 2k\pi i}
      =\cos(2\sqrt 2k\pi)+i\sin(2\sqrt 2k\pi), \quad k\in\BZ.
  \]
  \[
    i^i=e^{i\Ln i}
      =\exp\biggl(i\cdot\Bigl(2k+\half\Bigr)\pi i\biggr)
      =\exp\Bigl(-2k\pi-\half\pi\Bigr), \quad k\in\BZ.
  \]
\end{solution}

\begin{exercise}
  $3^i$ 的辐角主值是\fillblank{}.
\end{exercise}

幂函数与其主值有如下关系:
\[
  z^a=e^{a\ln z}\cdot 1^a
    =e^{a\ln z}\cdot e^{2ak\pi i},\quad k\in\BZ.
\]
对于幂函数的主值,
\[(z^a)'=\left(e^{a\ln z}\right)'=\frac{ae^{a\ln z}}z=az^{a-1}.\]
一般而言, $z^a\cdot z^b=z^{a+b}$ 和 $(z^a)^b=z^{ab}$ 都是不成立的.
\footnote{$z^a\cdot z^b=z^{a+b}$ 成立当且仅当 $\dfrac{a}{a+b}\in\BZ$. $(z^a)^b=z^{ab}$ 成立当且仅当 $\dfrac1a\in\BZ$.}

最后, 注意 $e^a$ 作为指数函数 $f(z)=e^z$ 在 $a$ 处的值和作为 $g(z)=z^a$ 在 $e$ 处的值是\alert{不同}的.
因为后者在 $a\not\in\BZ$ 时总是多值的.
前者实际上是后者的主值.
为避免混淆, 以后我们总\alert{默认 $e^a$ 表示指数函数 $\exp a$}.


\subsection{三角函数和反三角函数}

我们知道
  \[\cos x=\frac{e^{ix}+e^{-ix}}2,\quad
  \sin x=\frac{e^{ix}-e^{-ix}}{2i}\]
对于任意实数 $x$ 成立,
我们将其推广到复数情形.

\begin{definition}{余弦和正弦函数}
  定义\noun{余弦函数}和\noun{正弦函数}
  \[\cos z=\frac{e^{iz}+e^{-iz}}2,\quad
  \sin z=\frac{e^{iz}-e^{-iz}}{2i}.\]
\end{definition}
那么欧拉恒等式 \alert{$e^{iz}=\cos z+i\sin z$ 对任意复数 $z$ 均成立}.

不难得到
\[
  \cos(iy)=\dfrac{e^y+e^{-y}}2,\qquad
  {\sin(iy)=i\dfrac{e^y-e^{-y}}2.}
\]
当 $y\to\infty$ 时, $\cos(iy)$ 和 $\sin(iy)$ 都 $\to\infty$.
因此 \alert{$\sin z$ 和 $\cos z$ 并不有界}. 
这和实变情形不同.

容易看出 $\cos z$ 和 $\sin z$ 的零点都是实数.
于是可类似定义其它三角函数\index{正切函数}
\begin{align*}
  \tan z&=\frac{\sin z}{\cos z},z\neq\left(k+\half\right)\pi,&
  \cot z&=\frac{\cos z}{\sin z},z\neq k\pi,\\
  \sec z&=\frac{1}{\cos z},z\neq\left(k+\half\right)\pi,&
  \csc z&=\frac{1}{\sin z},z\neq k\pi.
\end{align*}
这些三角函数的奇偶性, 周期性和导数与实变情形类似,
  \[(\cos z)'=-\sin z,\quad
  (\sin z)'=\cos z,\]
且在定义域范围内是处处解析的.
三角函数的各种恒等式在复数情形也仍然成立, 例如
\begin{itemize}
  \item $\cos(z_1\pm z_2)=\cos z_1 \cos z_2\mp \sin z_1 \sin z_2$,
  \item $\sin(z_1\pm z_2)=\sin z_1 \cos z_2\pm\cos z_1 \sin z_2$,
  \item $\sin^2z+\cos^2z=1$.
\end{itemize}

类似的, 我们可以定义\noun{双曲函数}:
\begin{align*}
  \ch z&=\frac{e^z+e^{-z}}2=\cos iz,\\
  \sh z&=\frac{e^z-e^{-z}}2=-i\sin iz,\\
  \tanh z&=\frac{e^z-e^{-z}}{e^z+e^{-z}}
    =-i\tan iz,\quad z\neq \left(k+\half\right)\pi i.\
\end{align*}
它们的奇偶性和导数与实变情形类似, 在定义域范围内是处处解析的.
$\ch z,\sh z$ 的周期是 $2\pi i$, $\tanh z$ 的周期是 $\pi i$.

设 $z=\cos w=\dfrac{e^{iw}+e^{-iw}}2$, 则
  \[e^{2iw}-2ze^{iw}+1=0,\quad
    {e^{iw}=z+\sqrt{z^2-1}\footnote{注意右侧是双值函数}.}\]
因此\noun{反余弦函数}为
\[w=\Arccos z=-i\Ln(z+\sqrt{z^2-1}).\]
显然它是多值的. 同理, 我们有:
\begin{itemize}
  \item \noun{反正弦函数} $\Arcsin z=-i\Ln(iz+\sqrt{1-z^2})$;
  \item \noun{反正切函数} $\Arctan z=-\dfrac i2\Ln\dfrac{1+iz}{1-iz}, z\neq \pm i$;
  \item \noun{反双曲余弦函数} $\Arch z=\Ln(z+\sqrt{z^2-1})$;
  \item \noun{反双曲正弦函数} $\Arsh z=\Ln(z+\sqrt{z^2+1})$;
  \item \noun{反双曲正切函数} $\Arth z=\dfrac12\Ln\dfrac{1+z}{1-z}, z\neq \pm1$.
\end{itemize}

\begin{example}
  解方程 $\sin z=2$.
\end{example}

\begin{solution}
  由于
  \[\sin z=\dfrac{e^{iz}-e^{-iz}}{2i}=2,\]
  我们有
  \[e^{2iz}-4ie^{iz}-1=0.\]
  于是 $e^{iz}=(2\pm\sqrt 3)i$,
  \[z=-i\Ln[(2\pm\sqrt 3)i]=\left(2k+\half\right)\pi\pm i\ln(2+\sqrt3),\quad k\in\BZ.\]
\end{solution}

\begin{solution}[另解]
  由 $\sin z=2$ 可知
  \[\cos z=\sqrt{1-\sin^2 z}=\pm\sqrt 3i.\]
  于是 $e^{iz}=\cos z+i\sin z=(2\pm\sqrt 3)i$,
    \[z=-i\Ln[(2\pm\sqrt 3)i]=\left(2k+\half\right)\pi\pm i\ln(2+\sqrt3),\quad k\in\BZ.\]
\end{solution}
我们总有形式
\begin{align*}
  \Arcsin z&=(2k+\half)\pi\pm \theta,\\
  \Arccos z&=2k\pi\pm \theta,\\
  \Arctan z&=k\pi+\theta,\quad k\in\BZ.
\end{align*}


\sectionHomework




\item $i^{-i}$ 的主值是\fillblank{}.
\item $2^{-i}$ 的辐角主值是\fillblank{}.
\item 下面哪个函数在 $z=0$ 处不可导?~(~~~~)
\xx{$2x+3yi$}{$2x^2+3y^2i$}{$x^2-xyi$}{$e^x\cos y+i e^x\sin y$}
\item 解方程 $\sin z=2\cos z$.
\item 如果函数 $f(z)=e^{ax}(\cos y-i\sin y)$ 在复平面上处处解析, 则实数 $a=$\fillblank{}.
\item 函数 $f(z)=u(x,y)+iv(x,y)$ 在 $z_0=x_0+iy_0$ 处可导的充要条件是(~~~~).
\xx{$u,v$ 均在 $(x_0,y_0)$ 处连续}{$u,v$ 均在 $(x_0,y_0)$ 处有偏导数}{$u,v$ 均在 $(x_0,y_0)$ 处可微}{$u,v$ 均在 $(x_0,y_0)$ 处可微且满足C-R方程}
\item 解方程 $\cos z=\dfrac{3\sqrt2}4$.
\item 求 $\Ln(1+\sqrt3i)$.
\item 已知 $f(z)=u+iv$ 是解析函数, 其中 $u(x,y)=x^2+axy-y^2, v=2x^2-2y^2+2xy$ 且 $a$ 是实数.
求参数 $a$ 以及解析函数 $f'(z)$, 其中 $f'(z)$ 需要写成 $z$ 的表达式.
\item 设 $C$ 为有向曲线 $z(t)=\sin t+it,0\le t\le \pi$, 求 $\displaystyle\int_C ze^z \diff z$.
\item 设 $C$ 为正向圆周 $|z-1|=4$, 求 $\displaystyle\oint_C\frac{\sin z}{z^2+1}\diff z$.
\item 假设 $u(x,y)=x^3+ax^2y+bxy^2-3y^3$ 是调和函数,求参数 $a,b$ 以及 $v(x,y)$ 使得 $v(0,0)=0$ 且 $f(z)=u+iv$ 是解析函数.






\begin{homework}
  \item 判断题.
    \begin{exlist}
      \item 如果 $f'(z_0)$ 存在, 那么 $f(z)$ 在 $z_0$ 解析.\fillbrace{}
      \item 如果 $z_0$ 是 $f(z)$ 的奇点, 那么 $f(z)$ 在 $z_0$ 不可导.\fillbrace{}
      \item 如果 $z_0$ 是 $f(z)$ 和 $g(z)$ 的奇点, 那么 $z_0$ 也是 $f(z)+g(z)$ 和 $f(z)/g(z)$ 的奇点.\fillbrace{}
      \item 如果 $u(x,y)$ 和 $v(x,y)$ 偏导数均存在, 那么 $f(z)=u+iv$ 亦可导.\fillbrace{}
      \item 如果 $f(z)$ 在区域 $D$ 内处处可导, 则 $f(z)$ 在区域 $D$ 解析. \fillbrace{}
      \item 对任意复数 $z$, 有 $\ov{e^z}=e^{\ov z}$.\fillbrace{}
      \item 对任意复数 $z$, 有 $\ov{\cos z}=\cos{\ov z}$.\fillbrace{}
      \item 对任意复数 $z$, 有 $\ov{\sin z}=\sin{\ov z}$.\fillbrace{}
      \item 对任意复数 $z$, 有 $\ch^2z-\sh^2z=1$.\fillbrace{}
    \end{exlist}
  \item 选择题.
    \begin{exlist}
      \item 函数 $f(z)$ 在点 $z_0$ 的邻域内可导是 $f(z)$ 在该邻域内解析的\fillbrace{}.
        \begin{taskschoice}(2)
          \task 充分条件
          \task 必要条件
          \task 充要条件
          \task 既非充分也非必要条件
        \end{taskschoice}
      \item 设 $f(z)=u(x,y)+iv(x,y)$. 将下述选项不重复地填入括号内:
      \[\fillbrace{}\implies
        \fillbrace{}\implies
        \fillbrace{}\implies
        \fillbrace{}\implies
        \fillbrace{}\implies
        \fillbrace{}\]
        \begin{taskschoice}(2)
          \task $f(z)$ 在点 $z_0$ 有定义
          \task $f(z)$ 在点 $z_0$ 连续
          \task $f(z)$ 在点 $z_0$ 可导
          \task $f(z)$ 在点 $z_0$ 解析
          \task $f(z)$ 在点 $z_0$ 的一个邻域内解析
          \task $u,v$ 均在点 $(x_0,y_0)$ 处有偏导数
        \end{taskschoice}
      \item 下列函数中, 为解析函数的是\fillbrace{}.
        \begin{taskschoice}(2)
          \task $x^2-y^2-2xyi$
          \task $x^2+xyi$
          \task $2(x-1)y+i(y^2-x^2+2x)$
          \task $x^3+iy^3$
        \end{taskschoice}
      \item 设 $n$ 是正整数, $z_1,z_2\neq0$. 下列式子一定正确的是\fillbrace{}.
        \begin{taskschoice}(2)
          \task $\Arg(\sqrt z)=\dfrac12\Arg z$
          \task $\Arg(z^n)=n\Arg z$
          \task $\Arg(z_1z_2)=\Arg(z_1)+\Arg(z_2)$
          \task $\arg(\sqrt z)=\dfrac12\arg z$
          \task $\arg(z^{-n})=-n\arg z$
          \task $\arg(z_1/z_2)=\arg(z_1)-\arg(z_2)$
          \task $\Ln\sqrt[3] z=\dfrac13\Ln z$
          \task $\Ln(z^{-n})=-n\Ln z$
          \task $\Ln(z_1z_2)=\Ln(z_1)+\Ln(z_2)$
          \task $\ln\sqrt z=\dfrac12\ln z$
          \task $\ln(z^n)=n\ln z$
          \task $\ln(z_1/z_2)=\ln(z_1)-\ln(z_2)$
        \end{taskschoice}
    \end{exlist}
  \item 填空题.
    \begin{exlist}
      \item 函数 $\dfrac{z+1}{z(z^2+1)}$ 的奇点为\fillblank{}.
      \item 函数 $\dfrac{z-2}{(z+1)^2(z^2+1)}$ 的奇点为\fillblank{}.
      \item 函数 $\dfrac{1}{\sin z}$ 的奇点为\fillblank{}.
      \item 如果函数 $f(z)=x^2-2xy-y^2+i(ax^2+bxy+cy^2)$ 在复平面上处处解析, 则 $a+b+c=$\fillblank{}.
      \item 计算 $\ln i=$\fillblank[2cm]{}.
      \item 设 $z=1^{\sqrt3}$, 则 $|z|=$\fillblank{}.
      \item $i^{-i}$ 的主值是\fillblank{}.
    \end{exlist}
  \item 计算题.
    \begin{exlist}
      \item 设 $f(z)=\dfrac15z^5-(1+i)z$, 解方程 $f'(z)=0$.
      \item 下列函数何处可导? 何处解析?
        \begin{tasks}(3)
          \task $f(z)=1/\ov z$;
          \task* $f(z)=x^3-3xy^2+i(3x^2y-y^3)$;
          \task $f(z)=2x^3+3y^3i$;
          \task $f(z)=xy^2+ix^2y$;
          \task $f(z)=e^{x^2+y^2}$;
          \task $f(z)=z\Im z$;
          \task* $f(z)=\sin x\ch y+i\cos x\sh y$.
        \end{tasks}
      \item 指出下列函数 $f(z)$ 的解析区域, 并求出其导数.
        \begin{tasks}(2)
          \task $(z-1)^5$;
          \task $z^3+2iz$;
          \task $\dfrac1{z^2-1}$;
          \task $\dfrac{az+b}{cz+d}$ ($c,d$ 不全为零).
        \end{tasks}
      \item 设 $my^3+nx^2y+i(x^3+lxy^2)$ 为解析函数, 试确定实数 $l,m,n$ 的值.
      \item 计算
        \begin{tasks}(4)
          \item 计算 $\Arccos 2$.
          \task $\Ln 4$;
          \task $2\Ln2$;
          \task $\ln(-i)$;
          \task $\ln(-3+4i)$;
          \task $\Im\sin(1+i)$;
          \task $\arg e^{1-4i}$;
          \task $\exp\left(1-\dfrac{\pi i}2\right)$;
          \task $\exp\left(\dfrac{1+\pi i}4\right)$;
          \task $3^i$;
          \task $(1+i)^i$.
        \end{tasks}
      \item 解方程 
        \begin{tasks}(3)
          \task $\sin z=0$;
          \task $\cos z=0$;
          \task $1+e^z=0$;
          \task $\sin z+\cos z=0$;
          \task $\sin z=2\cos z$.
        \end{tasks}
      \item 验证 $e^x(x\cos y-y\sin y)+i e^x(y\cos y+x\sin y)$ 在全平面解析, 并求出其导数. 它在无穷远解析吗? 为何?

      \item 计算 $\displaystyle\int_0^{+\infty}\frac{x^2+1}{x^4+1}\diff x$.
      \item 复变函数 $f(z)=\sin z$ 和实变量函数 $g(x)=\sin x$ 的性质有什么相似和不同之处? 试说出3点.
      \item 求出 $\dfrac{1}{\sin z-2}$ 的解析区域.
      \item 证明: 若整函数(在整个复平面解析) $f$ 将实轴和虚轴均映为实数, 则 $f'(0)=0$.
    \end{exlist}
\end{homework}
  

\sectionExtraReading
\begin{homework}
  \item 仿照复数的指数函数, 我们可以尝试在矩阵上定义指数函数. 设 $\bfA\in M_m(\BC)$ 是一个 $m\times m$ 的复矩阵, 我们想说明极限
  \[e^\bfA:=\lim_{n\to \infty}\left(1+\frac1n \bfA\right)^n\]
  存在.
  \begin{exlist}
    \item 当 $\bfA=\diag\set{a_1,a_2,\dots,a_m}$ 是一个对角矩阵时, 证明 $e^\bfA$ 存在且
    \[e^\bfA=\diag\set{e^{a_1},e^{a_2},\dots,e^{a_m}}.\]
    \item 当
    \[\bfA=\bfJ_m(a)=\begin{pmatrix}
      a&1&   &\\
      &a& 1 &\\
      & &\ddots&1\\
      & &      &a
    \end{pmatrix}\]
    是约当块时, 证明 $e^\bfA$ 存在.
    \item 每个方阵都可以相似于一些约当块构成的分块对角阵, 由此证明 $e^\bfA$ 总存在.
    \item 当 $\bfA=x\bfE+y\bfJ=\begin{pmatrix}
      x&y\\-y&x
    \end{pmatrix}$ 时, 证明 $e^\bfA=\begin{pmatrix}
      e^x\cos y&e^x\sin y\\-e^x\sin y&e^x\cos y
    \end{pmatrix}=e^x(cos y\bfE+\sin y \bfJ)$.
    \item 证明 $e^\bfA=\bfE+\bfA+\dfrac{\bfA^2}{2!}+\dfrac{\bfA^3}{3!}+\cdots$.
    \item 证明 $e^{\bfA+\bfB}=e^\bfA\cdot e^\bfB$.
  \end{exlist}
  
  \item 注意到 $x=\dfrac12z+\dfrac12\ov z,y=-\dfrac i2z+\dfrac i2\ov z$.
  仿照着二元实函数偏导数在变量替换下的变换规则, 我们定义 $f$ 对 $z$ 和 $\ov z$ 的偏导数为
  \[\left\{
  \begin{aligned}
    \frac{\partial f}{\partial z}&
  =\frac{\partial x}{\partial z}\frac{\partial f}{\partial x}
    +\frac{\partial y}{\partial z}\frac{\partial f}{\partial y}
  =\frac12\frac{\partial f}{\partial x}-\frac i2\frac{\partial f}{\partial y},\\
    \frac{\partial f}{\partial \ov z}&
  =\frac{\partial x}{\partial \ov z}\frac{\partial f}{\partial x}
    +\frac{\partial y}{\partial \ov z}\frac{\partial f}{\partial y}
  =\frac12\frac{\partial f}{\partial x}+\frac i2\frac{\partial f}{\partial y}.
  \end{aligned}\right.\]
  \begin{exlist}
  \item 证明C-R方程等价于 $\dfrac{\partial f}{\partial \ov z}
  =0$.
  所以我们也可以把 \emph{$\dfrac{\partial f}{\partial \ov z}=0$} 叫做C-R方程.

  \item 利用该结论求函数 $f(z)=\ov z, z\Im z, e^{z\ov z}$ 的可导点和解析点.
  \end{exlist}
\end{homework}

