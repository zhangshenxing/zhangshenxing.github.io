\documentclass[11pt,a4paper,twoside,openright,scheme=chinese]{ctexbook}
\usepackage{book}
\usepackage{makeidx}
\usepackage{pgfplots}
\usepgfplotslibrary{patchplots}
\usetheme{bamboo}
\usepackage{bm}
\usepackage{extarrows}
\usepackage{mathrsfs}
\usepackage{stmaryrd}
\usepackage{makecell}
\usepackage{tikz}
\usepackage{color,calc}

% 符号
\renewcommand{\labelenumi}{{\upshape(\arabic{enumi})}}
\newcommand\enumnum[1]{{\textcolor{fourth}{\mdseries\upshape{(#1)}}}}
\newcommand\peq{\mathrel{\phantom{=}}} % 用于对齐的等号幻影

\usepackage{tikz}
\usepackage{color,calc}
% TIKZ 设置
\usetikzlibrary{
	quotes,
	shapes.arrows,
	arrows.meta,
	positioning,
	shapes.geometric,
	patterns,
	calc,
	angles,
	decorations.pathreplacing,
	backgrounds % 背景边框
}
\tikzset{
	background rectangle/.style={semithick,draw=fourth,fill=white,rounded corners},
  % arrow
	cstra/.style      ={-Stealth},        % right arrow
	cstla/.style      ={Stealth-},        % left arrow
	cstlra/.style     ={Stealth-Stealth}, % left-right arrow
	cstwra/.style     ={-Straight Barb},  % wide ra
	cstwla/.style     ={Straight Barb-},
	cstwlra/.style    ={Straight Barb-Straight Barb},
	cstnarrow/.style      ={-Latex, line width=0.1cm}, %文本框间箭头
	cstaxis/.style        ={-Stealth, thick}, %坐标轴
  % curve
	cstcurve/.style       ={very thick}, %一般曲线
	cstdash/.style        ={thick, dash pattern= on 0.2cm off 0.05cm}, %虚线
  % dot
	cstdot/.style         ={radius=.08}, %实心点
	cstdote/.style        ={radius=.07, fill=white}, %空心点
  % fill
	cstfill/.style       ={fill=black!10},
	cstfille/.style      ={pattern=north east lines, pattern color=black},
	cstfill1/.style       ={fill=main!20},
	cstfille1/.style      ={pattern=north east lines, pattern color=main},
	cstfill2/.style        ={fill=second!20},
	cstfille2/.style       ={pattern=north east lines, pattern color=second},
	cstfill3/.style        ={fill=third!20},
	cstfille3/.style       ={pattern=north east lines, pattern color=third},
	cstfill4/.style        ={fill=fourth!20},
	cstfille4/.style       ={pattern=north east lines, pattern color=fourth},
	cstfill5/.style        ={fill=fifth!20},
	cstfille5/.style       ={pattern=north east lines, pattern color=fifth},
  % node
	cstnode/.style        ={fill=white,draw=black,text=black,rounded corners=0.2cm,line width=1pt},
	cstnode1/.style       ={fill=main!15,draw=main!80,text=black,rounded corners=0.2cm,line width=1pt},
	cstnode2/.style       ={fill=second!15,draw=second!80,text=black,rounded corners=0.2cm,line width=1pt},
	cstnode3/.style       ={fill=third!15,draw=third!80,text=black,rounded corners=0.2cm,line width=1pt},
	cstnode4/.style       ={fill=fourth!15,draw=fourth!80,text=black,rounded corners=0.2cm,line width=1pt},
	cstnode5/.style       ={fill=fifth!15,draw=fifth!80,text=black,rounded corners=0.2cm,line width=1pt}
}
\ExplSyntaxOn
\cs_new_protected:Npn \fpstepfromto#1#2#3 
  {% from, to, nums
    \fp_step_inline:nnnn {#1} { (#2-(#1))/(#3-1)*0.99 } {#2}
  }
\pgfmathdeclarefunction{nrand}{0}
  {% \tex_normaldeviate:D 生成均值为 0,标准差为 10000 的随机整数
    \tl_set:Nx \pgfmathresult { \fp_eval:n { \tex_normaldeviate:D/10000 } }
  }
\pgfmathdeclarefunction{rdv}{0}{\pgfmathparse{1+nrand/100}}
\ExplSyntaxOff
\newcommand{\randpts}[3][10]{
  \foreach\i in {0,1,...,#1}{
    \pgfmathparse{rdv}\let\rdv\pgfmathresult
    \coordinate (\i) at ({#2*\rdv*cos(360/#1*\i)},{#3*\rdv*sin(360/#1*\i)});
  }
}
\newcommand{\randep}[2]{
  \randpts{#1}{#2}
  \filldraw[cstcurve,main,cstfill3,smooth] plot coordinates {(0) (1) (2) (3) (4) (5) (6) (7) (8) (9) (0)};
}

\let\question\relax
\elegantnewtheorem{question}{问题}{defstyle}{que}

\newfontface\cmunrm{cmunrm.otf}
\newcommand\cmu[1]{{\cmunrm{#1}}}
\newcommand{\alert}[1]{\textcolor{main}{\bf #1}}
\renewcommand{\emph}[1]{\textcolor{second}{\bf #1}}
\newcommand{\cnen}[2]{{\kaishu$\overset{\text{{#2}}}{\text{#1}}$}}
\newcommand{\nounen}[2]{{\color{second}\kaishu\cnen{#1}{#2}}\index{{#1}}}
\newcommand{\noun}[1]{{\color{second}\kaishu #1}\index{{#1}}}
\newcommand{\nouns}[2]{{\color{second}\kaishu #1}\index{{#2}}}
\newcommand{\nounsen}[3]{{\color{second}\kaishu\cnen{#1}{#2}}\index{{#3}}}

\setcounter{tocdepth}{2}
\newfontfamily\couriernew{Courier New}
\lstset{language=[LaTeX]TeX,
	basicstyle=\couriernew,
  morekeywords={AUTHOR, KEY, TITLE, YEAR, PAGES, HOWPUBLISHED, URL, LANGUAGE},
  keywordstyle=\color{winered}
}




\RequirePackage{extarrows}
\usepackage[normalem]{ulem}
\NewDocumentCommand\fillblank{O{1cm} O{0cm} m}{\uline{\makebox[#1]{\raisebox{#2}{#3}}}}
\newcommand\fillbrace[1]{{(\nolinebreak\hspace{0.5em minus 0.5em}{#1}\hspace{0.5em minus 0.5em}\nolinebreak)}}
\newcommand\resizet[1]{\resizebox{!}{#1\baselineskip}}
\newcommand{\trueex}{$\checkmark$}
\newcommand{\falseex}{$\times$}

% arrows
\newcommand\ra{\rightarrow}
\newcommand\lra{\longrightarrow}
\newcommand\la{\leftarrow}
\newcommand\lla{\longleftarrow}
\newcommand\sqra{\rightsquigarrow}
\newcommand\sqlra{\leftrightsquigarrow}
\newcommand\inj{\hookrightarrow}
\newcommand\linj{\hookleftarrow}
\newcommand\surj{\twoheadrightarrow}
\newcommand\simto{\stackrel{\sim}{\longrightarrow}}
\newcommand\sto[1]{\stackrel{#1}{\longrightarrow}}
\newcommand\lsto[1]{\stackrel{#1}{\longleftarrow}}
\newcommand\xto{\xlongrightarrow}
\newcommand\xeq{\xlongequal}
\newcommand\luobida{\xeq{\text{洛必达}}}
\newcommand\djwqx{\xeq{\text{等价无穷小}}}
\newcommand\eqprob{\stackrel{\mathrm{P}}{=}}
\newcommand\lto{\longmapsto}
\renewcommand\vec[1]{\overrightarrow{#1}}

% decorations
\newcommand\wh{\widehat}
\newcommand\wt{\widetilde}
\newcommand\ov{\overline}
\newcommand\ul{\underline}
\newlength{\larc}
\NewDocumentCommand\warc{o m}{%
	\IfNoValueTF {#1}%
	{%
		\settowidth{\larc}{$#2$}%
		\stackrel{\rotatebox{-90}{\ensuremath{\left(\rule{0ex}{0.7\larc}\right.}}}{#2}%
	}%
	{%
		\stackrel{\rotatebox{-90}{\ensuremath{\left(\rule{0ex}{#1}\right.}}}{#2}%
	}%
}

% braces
\newcommand\set[1]{{\left\{#1\right\}}}
\newcommand\setm[2]{{\left\{#1\,\middle\vert\, #2\right\}}}
\newcommand\abs[1]{\left|#1\right|}
\newcommand\pair[1]{\langle{#1}\rangle}
\newcommand\norm[1]{\!\parallel\!{#1}\!\parallel\!}
\newcommand\dbb[1]{\llbracket#1 \rrbracket}
\newcommand\floor[1]{\lfloor#1\rfloor}

% symbols
\renewcommand\le{\leqslant}
\renewcommand\ge{\geqslant}
\newcommand\vare{\varepsilon}
\newcommand\varp{\varphi}
\newcommand\ilim{\varinjlim\limits}
\newcommand\plim{\varprojlim\limits}
\newcommand\half{\frac{1}{2}}
\newcommand\mmid{\parallel}
\font\cyr=wncyr10\newcommand\Sha{\hbox{\cyr X}}
\newcommand\Uc{\stackrel{\circ}{U}\!\!}
\newcommand\hil[3]{\left(\frac{{#1},{#2}}{#3}\right)}
\newcommand\leg[2]{\Bigl(\frac{{#1}}{#2}\Bigr)}
\newcommand\aleg[2]{\Bigl[\frac{{#1}}{#2}\Bigr]}
\newcommand\stsc[2]{\genfrac{}{}{0pt}{}{#1}{#2}}

% categories
\newcommand\cA{{\mathsf{A}}}
\newcommand\cb{{\mathsf{b}}}
\newcommand\cB{{\mathsf{B}}}
\newcommand\cC{{\mathsf{C}}}
\newcommand\cD{{\mathsf{D}}}
\newcommand\cM{{\mathsf{M}}}
\newcommand\cR{{\mathsf{R}}}
\newcommand\cP{{\mathsf{P}}}
\newcommand\cT{{\mathsf{T}}}
\newcommand\cX{{\mathsf{X}}}
\newcommand\cx{{\mathsf{x}}}
\newcommand\cAb{{\mathsf{Ab}}}
\newcommand\cBT{{\mathsf{BT}}}
\newcommand\cBun{{\mathsf{Bun}}}
\newcommand\cCharLoc{{\mathsf{CharLoc}}}
\newcommand\cCoh{{\mathsf{Coh}}}
\newcommand\cComm{{\mathsf{Comm}}}
\newcommand\cEt{{\mathsf{Et}}}
\newcommand\cFppf{{\mathsf{Fppf}}}
\newcommand\cFpqc{{\mathsf{Fpqc}}}
\newcommand\cFunc{{\mathsf{Func}}}
\newcommand\cGroups{{\mathsf{Groups}}}
\newcommand\cGrpd{{\{\mathsf{Grpd}\}}}
\newcommand\cHo{{\mathsf{Ho}}}
\newcommand\cIso{{\mathsf{Iso}}}
\newcommand\cLoc{{\mathsf{Loc}}}
\newcommand\cMod{{\mathsf{Mod}}}
\newcommand\cModFil{{\mathsf{ModFil}}}
\newcommand\cNilp{{\mathsf{Nilp}}}
\newcommand\cPerf{{\mathsf{Perf}}}
\newcommand\cPN{{\mathsf{PN}}}
\newcommand\cRep{{\mathsf{Rep}}}
\newcommand\cRings{{\mathsf{Rings}}}
\newcommand\cSets{{\mathsf{Sets}}}
\newcommand\cStack{{\mathsf{Stack}}}
\newcommand\cSch{{\mathsf{Sch}}}
\newcommand\cTop{{\mathsf{Top}}}
\newcommand\cVect{{\mathsf{Vect}}}
\newcommand\cZar{{\mathsf{Zar}}}
\newcommand\cphimod{{\varphi\txt{-}\mathsf{Mod}}}
\newcommand\cphimodfil{{\varphi\txt{-}\mathsf{ModFil}}}

% font
\newcommand\rma{{\mathrm{a}}}
\newcommand\rmb{{\mathrm{b}}}
\newcommand\rmc{{\mathrm{c}}}
\newcommand\rmd{{\mathrm{d}}}
\newcommand\rme{{\mathrm{e}}}
\newcommand\rmf{{\mathrm{f}}}
\newcommand\rmg{{\mathrm{g}}}
\newcommand\rmh{{\mathrm{h}}}
\newcommand\rmi{{\mathrm{i}}}
\newcommand\rmj{{\mathrm{j}}}
\newcommand\rmk{{\mathrm{k}}}
\newcommand\rml{{\mathrm{l}}}
\newcommand\rmm{{\mathrm{m}}}
\newcommand\rmn{{\mathrm{n}}}
\newcommand\rmo{{\mathrm{o}}}
\newcommand\rmp{{\mathrm{p}}}
\newcommand\rmq{{\mathrm{q}}}
\newcommand\rmr{{\mathrm{r}}}
\newcommand\rms{{\mathrm{s}}}
\newcommand\rmt{{\mathrm{t}}}
\newcommand\rmu{{\mathrm{u}}}
\newcommand\rmv{{\mathrm{v}}}
\newcommand\rmw{{\mathrm{w}}}
\newcommand\rmx{{\mathrm{x}}}
\newcommand\rmy{{\mathrm{y}}}
\newcommand\rmz{{\mathrm{z}}}
\newcommand\rmA{{\mathrm{A}}}
\newcommand\rmB{{\mathrm{B}}}
\newcommand\rmC{{\mathrm{C}}}
\newcommand\rmD{{\mathrm{D}}}
\newcommand\rmE{{\mathrm{E}}}
\newcommand\rmF{{\mathrm{F}}}
\newcommand\rmG{{\mathrm{G}}}
\newcommand\rmH{{\mathrm{H}}}
\newcommand\rmI{{\mathrm{I}}}
\newcommand\rmJ{{\mathrm{J}}}
\newcommand\rmK{{\mathrm{K}}}
\newcommand\rmL{{\mathrm{L}}}
\newcommand\rmM{{\mathrm{M}}}
\newcommand\rmN{{\mathrm{N}}}
\newcommand\rmO{{\mathrm{O}}}
\newcommand\rmP{{\mathrm{P}}}
\newcommand\rmQ{{\mathrm{Q}}}
\newcommand\rmR{{\mathrm{R}}}
\newcommand\rmS{{\mathrm{S}}}
\newcommand\rmT{{\mathrm{T}}}
\newcommand\rmU{{\mathrm{U}}}
\newcommand\rmV{{\mathrm{V}}}
\newcommand\rmW{{\mathrm{W}}}
\newcommand\rmX{{\mathrm{X}}}
\newcommand\rmY{{\mathrm{Y}}}
\newcommand\rmZ{{\mathrm{Z}}}
\newcommand\bfa{{\mathbf{a}}}
\newcommand\bfb{{\mathbf{b}}}
\newcommand\bfc{{\mathbf{c}}}
\newcommand\bfd{{\mathbf{d}}}
\newcommand\bfe{{\mathbf{e}}}
\newcommand\bff{{\mathbf{f}}}
\newcommand\bfg{{\mathbf{g}}}
\newcommand\bfh{{\mathbf{h}}}
\newcommand\bfi{{\mathbf{i}}}
\newcommand\bfj{{\mathbf{j}}}
\newcommand\bfk{{\mathbf{k}}}
\newcommand\bfl{{\mathbf{l}}}
\newcommand\bfm{{\mathbf{m}}}
\newcommand\bfn{{\mathbf{n}}}
\newcommand\bfo{{\mathbf{o}}}
\newcommand\bfp{{\mathbf{p}}}
\newcommand\bfq{{\mathbf{q}}}
\newcommand\bfr{{\mathbf{r}}}
\newcommand\bfs{{\mathbf{s}}}
\newcommand\bft{{\mathbf{t}}}
\newcommand\bfu{{\mathbf{u}}}
\newcommand\bfv{{\mathbf{v}}}
\newcommand\bfw{{\mathbf{w}}}
\newcommand\bfx{{\mathbf{x}}}
\newcommand\bfy{{\mathbf{y}}}
\newcommand\bfz{{\mathbf{z}}}
\newcommand\bfA{{\mathbf{A}}}
\newcommand\bfB{{\mathbf{B}}}
\newcommand\bfC{{\mathbf{C}}}
\newcommand\bfD{{\mathbf{D}}}
\newcommand\bfE{{\mathbf{E}}}
\newcommand\bfF{{\mathbf{F}}}
\newcommand\bfG{{\mathbf{G}}}
\newcommand\bfH{{\mathbf{H}}}
\newcommand\bfI{{\mathbf{I}}}
\newcommand\bfJ{{\mathbf{J}}}
\newcommand\bfK{{\mathbf{K}}}
\newcommand\bfL{{\mathbf{L}}}
\newcommand\bfM{{\mathbf{M}}}
\newcommand\bfN{{\mathbf{N}}}
\newcommand\bfO{{\mathbf{O}}}
\newcommand\bfP{{\mathbf{P}}}
\newcommand\bfQ{{\mathbf{Q}}}
\newcommand\bfR{{\mathbf{R}}}
\newcommand\bfS{{\mathbf{S}}}
\newcommand\bfT{{\mathbf{T}}}
\newcommand\bfU{{\mathbf{U}}}
\newcommand\bfV{{\mathbf{V}}}
\newcommand\bfW{{\mathbf{W}}}
\newcommand\bfX{{\mathbf{X}}}
\newcommand\bfY{{\mathbf{Y}}}
\newcommand\bfZ{{\mathbf{Z}}}
\newcommand\BA{{\mathbb{A}}}
\newcommand\BB{{\mathbb{B}}}
\newcommand\BC{{\mathbb{C}}}
\newcommand\BD{{\mathbb{D}}}
\newcommand\BE{{\mathbb{E}}}
\newcommand\BF{{\mathbb{F}}}
\newcommand\BG{{\mathbb{G}}}
\newcommand\BH{{\mathbb{H}}}
\newcommand\BI{{\mathbb{I}}}
\newcommand\BJ{{\mathbb{J}}}
\newcommand\BK{{\mathbb{K}}}
\newcommand\BL{{\mathbb{L}}}
\newcommand\BM{{\mathbb{M}}}
\newcommand\BN{{\mathbb{N}}}
\newcommand\BO{{\mathbb{O}}}
\newcommand\BP{{\mathbb{P}}}
\newcommand\BQ{{\mathbb{Q}}}
\newcommand\BR{{\mathbb{R}}}
\newcommand\BS{{\mathbb{S}}}
\newcommand\BT{{\mathbb{T}}}
\newcommand\BU{{\mathbb{U}}}
\newcommand\BV{{\mathbb{V}}}
\newcommand\BW{{\mathbb{W}}}
\newcommand\BX{{\mathbb{X}}}
\newcommand\BY{{\mathbb{Y}}}
\newcommand\BZ{{\mathbb{Z}}}
\newcommand\CA{{\mathcal{A}}}
\newcommand\CB{{\mathcal{B}}}
\newcommand\CC{{\mathcal{C}}}
\providecommand\CD{{\mathcal{D}}}
\newcommand\CE{{\mathcal{E}}}
\newcommand\CF{{\mathcal{F}}}
\newcommand\CG{{\mathcal{G}}}
\newcommand\CH{{\mathcal{H}}}
\newcommand\CI{{\mathcal{I}}}
\newcommand\CJ{{\mathcal{J}}}
\newcommand\CK{{\mathcal{K}}}
\newcommand\CL{{\mathcal{L}}}
\newcommand\CM{{\mathcal{M}}}
\newcommand\CN{{\mathcal{N}}}
\newcommand\CO{{\mathcal{O}}}
\newcommand\CP{{\mathcal{P}}}
\newcommand\CQ{{\mathcal{Q}}}
\newcommand\CR{{\mathcal{R}}}
\newcommand\CS{{\mathcal{S}}}
\newcommand\CT{{\mathcal{T}}}
\newcommand\CU{{\mathcal{U}}}
\newcommand\CV{{\mathcal{V}}}
\newcommand\CW{{\mathcal{W}}}
\newcommand\CX{{\mathcal{X}}}
\newcommand\CY{{\mathcal{Y}}}
\newcommand\CZ{{\mathcal{Z}}}
\newcommand\RA{{\mathrm{A}}}
\newcommand\RB{{\mathrm{B}}}
\newcommand\RC{{\mathrm{C}}}
\newcommand\RD{{\mathrm{D}}}
\newcommand\RE{{\mathrm{E}}}
\newcommand\RF{{\mathrm{F}}}
\newcommand\RG{{\mathrm{G}}}
\newcommand\RH{{\mathrm{H}}}
\newcommand\RI{{\mathrm{I}}}
\newcommand\RJ{{\mathrm{J}}}
\newcommand\RK{{\mathrm{K}}}
\newcommand\RL{{\mathrm{L}}}
\newcommand\RM{{\mathrm{M}}}
% \newcommand\RN{{\mathrm{N}}}
\newcommand\RO{{\mathrm{O}}}
\newcommand\RP{{\mathrm{P}}}
\newcommand\RQ{{\mathrm{Q}}}
\newcommand\RR{{\mathrm{R}}}
\newcommand\RS{{\mathrm{S}}}
\newcommand\RT{{\mathrm{T}}}
\newcommand\RU{{\mathrm{U}}}
\newcommand\RV{{\mathrm{V}}}
\newcommand\RW{{\mathrm{W}}}
\newcommand\RX{{\mathrm{X}}}
\newcommand\RY{{\mathrm{Y}}}
\newcommand\RZ{{\mathrm{Z}}}
\newcommand\msa{\mathscr{A}}
\newcommand\msb{\mathscr{B}}
\newcommand\msc{\mathscr{C}}
\newcommand\msd{\mathscr{D}}
\newcommand\mse{\mathscr{E}}
\newcommand\msf{\mathscr{F}}
\newcommand\msg{\mathscr{G}}
\newcommand\msh{\mathscr{H}}
\newcommand\msi{\mathscr{I}}
\newcommand\msj{\mathscr{J}}
\newcommand\msk{\mathscr{K}}
\newcommand\msl{\mathscr{L}}
\newcommand\msm{\mathscr{M}}
\newcommand\msn{\mathscr{N}}
\newcommand\mso{\mathscr{O}}
\newcommand\msp{\mathscr{P}}
\newcommand\msq{\mathscr{Q}}
\newcommand\msr{\mathscr{R}}
\newcommand\mss{\mathscr{S}}
\newcommand\mst{\mathscr{T}}
\newcommand\msu{\mathscr{U}}
\newcommand\msv{\mathscr{V}}
\newcommand\msw{\mathscr{W}}
\newcommand\msx{\mathscr{X}}
\newcommand\msy{\mathscr{Y}}
\newcommand\msz{\mathscr{Z}}
\newcommand\fa{{\mathfrak{a}}}
\newcommand\fb{{\mathfrak{b}}}
\newcommand\fc{{\mathfrak{c}}}
\newcommand\fd{{\mathfrak{d}}}
\newcommand\fe{{\mathfrak{e}}}
\newcommand\ff{{\mathfrak{f}}}
\newcommand\fg{{\mathfrak{g}}}
\newcommand\fh{{\mathfrak{h}}}
\newcommand\fii{{\mathfrak{i}}} % be careful about \fii
\newcommand\fj{{\mathfrak{j}}}
\newcommand\fk{{\mathfrak{k}}}
\newcommand\fl{{\mathfrak{l}}}
\newcommand\fm{{\mathfrak{m}}}
\newcommand\fn{{\mathfrak{n}}}
\newcommand\fo{{\mathfrak{o}}}
\newcommand\fp{{\mathfrak{p}}}
\newcommand\fq{{\mathfrak{q}}}
\newcommand\fr{{\mathfrak{r}}}
\newcommand\fs{{\mathfrak{s}}}
\newcommand\ft{{\mathfrak{t}}}
\newcommand\fu{{\mathfrak{u}}}
\newcommand\fv{{\mathfrak{v}}}
\newcommand\fw{{\mathfrak{w}}}
\newcommand\fx{{\mathfrak{x}}}
\newcommand\fy{{\mathfrak{y}}}
\newcommand\fz{{\mathfrak{z}}}
\newcommand\fA{{\mathfrak{A}}}
\newcommand\fB{{\mathfrak{B}}}
\newcommand\fC{{\mathfrak{C}}}
\newcommand\fD{{\mathfrak{D}}}
\newcommand\fE{{\mathfrak{E}}}
\newcommand\fF{{\mathfrak{F}}}
\newcommand\fG{{\mathfrak{G}}}
\newcommand\fH{{\mathfrak{H}}}
\newcommand\fI{{\mathfrak{I}}}
\newcommand\fJ{{\mathfrak{J}}}
\newcommand\fK{{\mathfrak{K}}}
\newcommand\fL{{\mathfrak{L}}}
\newcommand\fM{{\mathfrak{M}}}
\newcommand\fN{{\mathfrak{N}}}
\newcommand\fO{{\mathfrak{O}}}
\newcommand\fP{{\mathfrak{P}}}
\newcommand\fQ{{\mathfrak{Q}}}
\newcommand\fR{{\mathfrak{R}}}
\newcommand\fS{{\mathfrak{S}}}
\newcommand\fT{{\mathfrak{T}}}
\newcommand\fU{{\mathfrak{U}}}
\newcommand\fV{{\mathfrak{V}}}
\newcommand\fW{{\mathfrak{W}}}
\newcommand\fX{{\mathfrak{X}}}
\newcommand\fY{{\mathfrak{Y}}}
\newcommand\fZ{{\mathfrak{Z}}}

% a
\newcommand\ab{{\mathrm{ab}}}
\newcommand\ad{{\mathrm{ad}}}
\newcommand\Ad{{\mathrm{Ad}}}
\newcommand\adele{ad\'{e}le}
\newcommand\Adele{Ad\'{e}le}
\newcommand\adeles{ad\'{e}les}
\newcommand\adelic{ad\'{e}lic}
\newcommand\AJ{{\mathrm{AJ}}}
\newcommand\alb{{\mathrm{alb}}}
\newcommand\Alb{{\mathrm{Alb}}}
\newcommand\alg{{\mathrm{alg}}}
\newcommand\an{{\mathrm{an}}}
\newcommand\ann{{\mathrm{ann}}}
\newcommand\Ann{{\mathrm{Ann}}}
\DeclareMathOperator\Arcsin{Arcsin}
\DeclareMathOperator\Arccos{Arccos}
\DeclareMathOperator\Arctan{Arctan}
\DeclareMathOperator\arsh{arsh}
\DeclareMathOperator\Arsh{Arsh}
\DeclareMathOperator\arch{arch}
\DeclareMathOperator\Arch{Arch}
\DeclareMathOperator\arth{arth}
\DeclareMathOperator\Arth{Arth}
\DeclareMathOperator\arccot{arccot}
\DeclareMathOperator\Arg{Arg}
\newcommand\arith{{\mathrm{arith}}}
\newcommand\Art{{\mathrm{Art}}}
\newcommand\AS{{\mathrm{AS}}}
\newcommand\Ass{{\mathrm{Ass}}}
\newcommand\Aut{{\mathrm{Aut}}}
% b
\newcommand\Bun{{\mathrm{Bun}}}
\newcommand\Br{{\mathrm{Br}}}
\newcommand\bs{\backslash}
\newcommand\BWt{{\mathrm{BW}}}
% c
\newcommand\can{{\mathrm{can}}}
\newcommand\cc{{\mathrm{cc}}}
\newcommand\cd{{\mathrm{cd}}}
\DeclareMathOperator\ch{ch}
\newcommand\Ch{{\mathrm{Ch}}}
\let\char\relax
\DeclareMathOperator\char{char}
\DeclareMathOperator\Char{char}
\newcommand\Chow{{\mathrm{CH}}}
\newcommand\circB{{\stackrel{\circ}{B}}}
\newcommand\cl{{\mathrm{cl}}}
\newcommand\Cl{{\mathrm{Cl}}}
\newcommand\cm{{\mathrm{cm}}}
\newcommand\cod{{\mathrm{cod}}}
\DeclareMathOperator\coker{coker}
\DeclareMathOperator\Coker{Coker}
\DeclareMathOperator\cond{cond}
\DeclareMathOperator\codim{codim}
\newcommand\cont{{\mathrm{cont}}}
\newcommand\Conv{{\mathrm{Conv}}}
\newcommand\corr{{\mathrm{corr}}}
\newcommand\Corr{{\mathrm{Corr}}}
\DeclareMathOperator\coim{coim}
\DeclareMathOperator\coIm{coIm}
\DeclareMathOperator\corank{corank}
\DeclareMathOperator\covol{covol}
\newcommand\cris{{\mathrm{cris}}}
\newcommand\Cris{{\mathrm{Cris}}}
\newcommand\CRIS{{\mathrm{CRIS}}}
\newcommand\crit{{\mathrm{crit}}}
\newcommand\crys{{\mathrm{crys}}}
\newcommand\cusp{{\mathrm{cusp}}}
\newcommand\CWt{{\mathrm{CW}}}
\newcommand\cyc{{\mathrm{cyc}}}
% d
\newcommand\Def{{\mathrm{Def}}}
\newcommand\diag{{\mathrm{diag}}}
\newcommand\diff{\,\mathrm{d}}
\newcommand\disc{{\mathrm{disc}}}
\newcommand\dist{{\mathrm{dist}}}
\renewcommand\div{{\mathrm{div}}}
\newcommand\Div{{\mathrm{Div}}}
\newcommand\dR{{\mathrm{dR}}}
\newcommand\Drin{{\mathrm{Drin}}}
% e
\newcommand\End{{\mathrm{End}}}
\newcommand\ess{{\mathrm{ess}}}
\newcommand\et{{\text{\'{e}t}}}
\newcommand\etale{{\'{e}tale}}
\newcommand\Etale{{\'{E}tale}}
\newcommand\Ext{\mathrm{Ext}}
\newcommand\CExt{\mathcal{E}\mathrm{xt}}
% f
\newcommand\Fil{{\mathrm{Fil}}}
\newcommand\Fix{{\mathrm{Fix}}}
\newcommand\fppf{{\mathrm{fppf}}}
\newcommand\Fr{{\mathrm{Fr}}}
\newcommand\Frac{{\mathrm{Frac}}}
\newcommand\Frob{{\mathrm{Frob}}}
% g
\newcommand\Ga{\mathbb{G}_a}
\newcommand\Gm{\mathbb{G}_m}
\newcommand\hGa{\widehat{\mathbb{G}}_{a}}
\newcommand\hGm{\widehat{\mathbb{G}}_{m}}
\newcommand\Gal{{\mathrm{Gal}}}
\newcommand\gl{{\mathrm{gl}}}
\newcommand\GL{{\mathrm{GL}}}
\newcommand\GO{{\mathrm{GO}}}
\newcommand\geom{{\mathrm{geom}}}
\newcommand\Gr{{\mathrm{Gr}}}
\newcommand\gr{{\mathrm{gr}}}
\newcommand\GSO{{\mathrm{GSO}}}
\newcommand\GSp{{\mathrm{GSp}}}
\newcommand\GSpin{{\mathrm{GSpin}}}
\newcommand\GU{{\mathrm{GU}}}
% h
\newcommand\hg{{\mathrm{hg}}}
\newcommand\Hk{{\mathrm{Hk}}}
\newcommand\HN{{\mathrm{HN}}}
\newcommand\Hom{{\mathrm{Hom}}}
\newcommand\CHom{\mathcal{H}\mathrm{om}}
% i
\newcommand\id{{\mathrm{id}}}
\newcommand\Id{{\mathrm{Id}}}
\newcommand\idele{id\'{e}le}
\newcommand\Idele{Id\'{e}le}
\newcommand\ideles{id\'{e}les}
\let\Im\relax
\DeclareMathOperator\Im{Im}
\DeclareMathOperator\im{im}
\newcommand\Ind{{\mathrm{Ind}}}
\newcommand\cInd{{\mathrm{c}\textrm{-}\mathrm{Ind}}}
\newcommand\ind{{\mathrm{ind}}}
\newcommand\Int{{\mathrm{Int}}}
\newcommand\inv{{\mathrm{inv}}}
\newcommand\Isom{{\mathrm{Isom}}}
% j
\newcommand\Jac{{\mathrm{Jac}}}
\newcommand\JL{{\mathrm{JL}}}
% k
\newcommand\Katz{\mathrm{Katz}}
\DeclareMathOperator\Ker{Ker}
\newcommand\KS{{\mathrm{KS}}}
\newcommand\Kl{{\mathrm{Kl}}}
\newcommand\CKl{{\mathcal{K}\mathrm{l}}}
% l
\newcommand\lcm{\mathrm{lcm}}
\newcommand\length{\mathrm{length}}
\DeclareMathOperator\Li{Li}
\DeclareMathOperator\Ln{Ln}
\newcommand\Lie{{\mathrm{Lie}}}
\newcommand\lt{\mathrm{lt}}
\newcommand\LT{\mathcal{LT}}
% m
\newcommand\mex{{\mathrm{mex}}}
\newcommand\MW{{\mathrm{MW}}}
\renewcommand\mod{\, \mathrm{mod}\, }
\newcommand\mom{{\mathrm{mom}}}
\newcommand\Mor{{\mathrm{Mor}}}
\newcommand\Morp{{\mathrm{Morp}\,}}
% n
\newcommand\new{{\mathrm{new}}}
\newcommand\Newt{{\mathrm{Newt}}}
\newcommand\nd{{\mathrm{nd}}}
\newcommand\NP{{\mathrm{NP}}}
\newcommand\NS{{\mathrm{NS}}}
\newcommand\ns{{\mathrm{ns}}}
\newcommand\Nm{{\mathrm{Nm}}}
\newcommand\Nrd{{\mathrm{Nrd}}}
\newcommand\Neron{N\'{e}ron}
% o
\newcommand\Obj{{\mathrm{Obj}\,}}
\newcommand\odd{{\mathrm{odd}}}
\newcommand\old{{\mathrm{old}}}
\newcommand\op{{\mathrm{op}}}
\newcommand\Orb{{\mathrm{Orb}}}
\newcommand\ord{{\mathrm{ord}}}
% p
\newcommand\pd{{\mathrm{pd}}}
\newcommand\Pet{{\mathrm{Pet}}}
\newcommand\PGL{{\mathrm{PGL}}}
\newcommand\Pic{{\mathrm{Pic}}}
\newcommand\pr{{\mathrm{pr}}}
\newcommand\Proj{{\mathrm{Proj}}}
\newcommand\proet{\text{pro\'{e}t}}
\newcommand\Poincare{\text{Poincar\'{e}}}
\newcommand\Prd{{\mathrm{Prd}}}
\newcommand\prim{{\mathrm{prim}}}
% r
\newcommand\Rad{{\mathrm{Rad}}}
\newcommand\rad{{\mathrm{rad}}}
\DeclareMathOperator\rank{rank}
\let\Re\relax
\DeclareMathOperator\Re{Re}
\newcommand\rec{{\mathrm{rec}}}
\newcommand\red{{\mathrm{red}}}
\newcommand\reg{{\mathrm{reg}}}
\newcommand\res{{\mathrm{res}}}
\newcommand\Res{{\mathrm{Res}}}
\newcommand\rig{{\mathrm{rig}}}
\newcommand\Rig{{\mathrm{Rig}}}
\newcommand\rk{{\mathrm{rk}}}
\newcommand\Ros{{\mathrm{Ros}}}
\newcommand\rs{{\mathrm{rs}}}
% s
\newcommand\sd{{\mathrm{sd}}}
\newcommand\Sel{{\mathrm{Sel}}}
\newcommand\sep{{\mathrm{sep}}}
\DeclareMathOperator{\sgn}{sgn}
\newcommand\Sh{{\mathrm{Sh}}}
\DeclareMathOperator\sh{sh}
\newcommand\Sht{\mathrm{Sht}}
\newcommand\Sim{{\mathrm{Sim}}}
\newcommand\sign{{\mathrm{sign}}}
\newcommand\SK{{\mathrm{SK}}}
\newcommand\SL{{\mathrm{SL}}}
\newcommand\SO{{\mathrm{SO}}}
\newcommand\Sp{{\mathrm{Sp}}}
\newcommand\Spa{{\mathrm{Spa}}}
\newcommand\Span{{\mathrm{Span}}}
\DeclareMathOperator\Spec{Spec}
\newcommand\Spf{{\mathrm{Spf}}}
\newcommand\Spin{{\mathrm{Spin}}}
\newcommand\Spm{{\mathrm{Spm}}}
\newcommand\srs{{\mathrm{srs}}}
\newcommand\rss{{\mathrm{ss}}}
\newcommand\ST{{\mathrm{ST}}}
\newcommand\St{{\mathrm{St}}}
\newcommand\st{{\mathrm{st}}}
\newcommand\Stab{{\mathrm{Stab}}}
\newcommand\SU{{\mathrm{SU}}}
\newcommand\Sym{{\mathrm{Sym}}}
\newcommand\sub{{\mathrm{sub}}}
\newcommand\rsum{{\mathrm{sum}}}
\newcommand\supp{{\mathrm{supp}}}
\newcommand\Supp{{\mathrm{Supp}}}
\newcommand\Swan{{\mathrm{Sw}}}
\newcommand\suml{\sum\limits}
% t
\newcommand\td{{\mathrm{td}}}
\let\tanh\relax
\DeclareMathOperator\tanh{th}
\newcommand\tor{{\mathrm{tor}}}
\newcommand\Tor{{\mathrm{Tor}}}
\newcommand\tors{{\mathrm{tors}}}
\newcommand\tr{{\mathrm{tr}\,}}
\newcommand\Tr{{\mathrm{Tr}}}
\newcommand\Trd{{\mathrm{Trd}}}
\newcommand\TSym{{\mathrm{TSym}}}
\newcommand\tw{{\mathrm{tw}}}
% u
\newcommand\uni{{\mathrm{uni}}}
\newcommand\univ{\mathrm{univ}}
\newcommand\ur{{\mathrm{ur}}}
\newcommand\USp{{\mathrm{USp}}}
% v
\newcommand\vQ{{\breve \BQ}}
\newcommand\vE{{\breve E}}
\newcommand\Ver{{\mathrm{Ver}}}
\newcommand\vF{{\breve F}}
\newcommand\vK{{\breve K}}
\newcommand\vol{{\mathrm{vol}}}
\newcommand\Vol{{\mathrm{Vol}}}
% w
\newcommand\wa{{\mathrm{wa}}}
% z
\newcommand\Zar{{\mathrm{Zar}}}

\usebeamercolor{structure}
\colorlet{heavyb}{structure.fg!80}
\colorlet{lightb}{structure.fg!10}
\usebeamercolor{example text}
\colorlet{heavyg}{example text.fg!80}
\colorlet{lightg}{example text.fg!10}
\usebeamercolor{alerted text}
\colorlet{heavyr}{alerted text.fg!80}
\colorlet{lightr}{alerted text.fg!10}
% 表格色
\newcommand\defaultrowcolors{\rowcolors[\hline]{1}{lightb}{lightb}}
\newcommand\tht{\cellcolor{heavyb}\color{white}}
% 文本色
\usebeamercolor{structure}
\renewcommand\emph[1]{{\color{structure.fg!50!blue}{#1}}}
\newcommand\abox[1]{\colorbox{yellow}{\alert{#1}}}
\newcommand\aboxeq[1]{\abox{$\displaystyle #1$}}
% 图形色
\colorlet{dcolora}{red}
\colorlet{dcolorb}{blue}
\colorlet{dcolorc}{red!50!blue}
% 枚举数字引用标志
\newcommand\enumnum[1]{{\mdseries\upshape\usebeamerfont{enumerate item}\usebeamercolor[fg]{enumerate item}(#1)}}
\newcommand\enumnumhk[1]{{
  \usebeamercolor[fg]{enumerate subitem}%
  \begin{pgfpicture}{-1ex}{0ex}{0.5ex}{2ex}
    \pgfpathcircle{\pgfpoint{0pt}{.75ex}}{1ex}
    \pgfusepath{stroke}
    \pgftext[base]{\color{fg}#1}
  \end{pgfpicture}%
}}
% 手动证毕标志
\newcommand\mqed{\visible<.->{\qedsymbol}}
% 用于对齐的等号幻影
\newcommand\peq{\mathrel{\phantom{=}}}
% 减少公式垂直间距, 配合\endgroup
\newcommand\beqskip[1]{\begingroup\abovedisplayskip=#1\belowdisplayskip=#1\belowdisplayshortskip=#1}
% 缩进
\renewcommand{\indent}{\hspace*{1.5em}}
\setlength{\parindent}{1.5em}
% 作业
% \usepackage{bookmark}
% \newcommand{\homework}{%
% 	\bookmark[page=\thepage,level=3]{作业}%
% 	\setbeamertemplate{subsection in footline}[homework]%
% 	\setbeamercolor{subsection in head/foot}{parent=alerted text}%
% 	\setbeamercolor{frametitle}{parent=alerted text}%
% 	\setbeamercolor{seperate line}{use=alerted text,bg=alerted text.fg}
% 	\setnaviboxexercise
% }
% \NewDocumentEnvironment{homeworks}{o}{%
% 	\setbeamercolor{enumerate item}{use=alerted text,fg=alerted text.fg}%
% 	\setbeamercolor{enumerate subitem}{use=structure,fg=structure.fg}%
% 	\begin{enumerate}%
% 		\IfValueT{#1}{\setcounter{enumi}{#1}}}{%
% 	\end{enumerate}%
% }
% 选择题, 根据选项内容长度自动排版
\newlength{\ltemp}
\newlength{\lxxmax}
\newlength{\lquar}
\newlength{\lhalf}
\newlength{\lfull}
\newcounter{lxxtype}
\NewDocumentCommand\xx{O{0} m m m m}{%
	\setlength{\lfull}{\columnwidth}%
	\addtolength{\lfull}{-\leftmargin}%
	\setlength{\lhalf}{0.5\lfull}%
	\setlength{\lquar}{0.25\lfull}%
	\setcounter{lxxtype}{0}%
	\ifnum#1=1\setcounter{lxxtype}{1}\fi%
	\ifnum#1=2\setcounter{lxxtype}{2}\fi%
	\ifnum#1=4\setcounter{lxxtype}{4}\fi%
	\settowidth{\lxxmax}{(A)~#2~}% 获取最长选项长度
	\settowidth{\ltemp}{(B)~#3~}%
	\ifdimcomp\ltemp>\lxxmax{\setlength{\lxxmax}{\ltemp}}{}%
	\settowidth{\ltemp}{(C)~#4~}%
	\ifdimcomp\ltemp>\lxxmax{\setlength{\lxxmax}{\ltemp}}{}%
	\settowidth{\ltemp}{(D)~#5~}%
	\ifdimcomp\ltemp>\lxxmax{\setlength{\lxxmax}{\ltemp}}{}%
	\ifnum\value{lxxtype}=0%
		\setcounter{lxxtype}{4}%
		\ifdimcomp\lxxmax>\lquar{\setcounter{lxxtype}{2}}{}%
		\ifnum\value{lxxtype}=2%
			\ifdimcomp\lxxmax>\lhalf{\setcounter{lxxtype}{1}}{}%
		\fi%
	\fi%
	\vspace{5pt}%
	\ifnum\value{lxxtype}=1%
		\\\makebox[\lfull][l]{(A)~#2}%
		\\\makebox[\lfull][l]{(B)~#3}%
		\\\makebox[\lfull][l]{(C)~#4}%
		\\\makebox[\lfull][l]{(D)~#5}%
	\fi%
	\ifnum\value{lxxtype}=2%
		\\\makebox[\lhalf][l]{(A)~#2}%
			\makebox[\lhalf][l]{(B)~#3}%
		\\\makebox[\lhalf][l]{(C)~#4}%
			\makebox[\lhalf][l]{(D)~#5}%
	\fi%
	\ifnum\value{lxxtype}=4%
		\\\makebox[\lquar][l]{(A)~#2}% 
			\makebox[\lquar][l]{(B)~#3}%
			\makebox[\lquar][l]{(C)~#4}%
			\makebox[\lquar][l]{(D)~#5}%
	\fi%
}
\NewDocumentCommand\subex{m o o o o}{%
	\setlength{\lfull}{\columnwidth}%
	\addtolength{\lfull}{-\leftmargin}%
	\vspace{5pt}%
	\ifnum#1=1%
		\\\makebox[\lfull][l]{#2}%
	\fi%
	\ifnum#1=2%
		\\\makebox[0.48\lfull][l]{#2}%
			\makebox[0.48\lfull][l]{#3}%
	\fi%
	\ifnum#1=3%
		\\\makebox[0.32\lfull][l]{#2}% 
			\makebox[0.32\lfull][l]{#3}%
			\makebox[0.32\lfull][l]{#4}%
	\fi%
	\ifnum#1=4%
		\\\makebox[0.24\lfull][l]{#2}% 
			\makebox[0.24\lfull][l]{#3}%
			\makebox[0.24\lfull][l]{#4}%
			\makebox[0.24\lfull][l]{#5}%
	\fi%
}
\usepackage[normalem]{ulem}
\NewDocumentCommand\fillblank{O{1cm} O{0cm} m}{\uline{\makebox[#1]{\raisebox{#2}{\alert{#3}}}}}
\newcommand{\fillbrace}{{(\nolinebreak\hspace{2em minus 1em}\nolinebreak)}}
% TIKZ 设置
\usetikzlibrary{
	quotes,
	shapes.arrows,
	arrows.meta,
	positioning,
	shapes.geometric,
	overlay-beamer-styles,
	patterns,
	calc,
	angles
}
\tikzset{
	cstarrowto/.style     ={-Stealth}, %尖箭头
	cstarrowfrom/.style   ={Stealth-},
	cstarrowdouble/.style ={Stealth-Stealth},
	cstarrow1to/.style    ={-Straight Barb}, %宽箭头
	cstarrow1from/.style  ={Straight Barb-},
	cstarrow1double/.style={Straight Barb-Straight Barb},
	cstnarrow/.style      ={-Latex, line width=0.1cm}, %文本框间箭头
	cstaxis/.style        ={-Stealth, thick}, %坐标轴
	cstcurve/.style       ={very thick}, %一般曲线
	cstdash/.style        ={thick, dash pattern= on 0.2cm off 0.05cm}, %虚线
	cstdot/.style         ={radius=0.07}, %实心点
	cstdote/.style        ={radius=0.06, fill=white}, %空心点
	cstfill/.style        ={fill=cyan!40}, %填充
	cstfille/.style       ={pattern=north east lines, pattern color=cyan}, %虚线填充
	cstnode/.style        ={fill=white,draw=black,text=black,rounded corners=0.2cm,line width=1pt},
	cstnodeb/.style       ={fill=lightb,draw=heavyb,text=black,rounded corners=0.2cm,line width=1pt},
	cstnodeg/.style       ={fill=lightg,draw=heavyg,text=black,rounded corners=0.2cm,line width=1pt},
	cstnoder/.style       ={fill=lightr,draw=heavyr,text=black,rounded corners=0.2cm,line width=1pt}
}
% 标题
\title{复变函数与积分变换}
\author{张神星}
\institute{合肥工业大学}
\email{zhangshenxing@hfut.edu.cn}
\website{https://zhangshenxing.gitee.io}
\office{翡翠科教楼B1810东}
\titlegraphic{misc/hfut.png}
\institutegraphic{misc/hfutname.png}


\makeindex
\title{复变函数与积分变换}
\subtitle{工程数学教材}
\author{张神星}

\newcommand{\wip}{
\begin{exercise}
  \alert{WIP 这里需要一道填空或选择.}
\end{exercise}}




\begin{document}

\maketitle

\hypersetup{pageanchor=true}
\frontmatter

\chapter*{引\qquad 言}
\addcontentsline{toc}{chapter}{引言}

在当今科学与工程的广阔领域中,复变函数与积分变换作为数学工具的核心组成部分,扮演着举足轻重的角色。它们不仅是理论研究的基石,更是解决实际问题的利器。本书《复变函数与积分变换》旨在为读者提供一本内容全面、结构清晰、易于理解的教材,旨在帮助读者系统地掌握这一领域的基本理论和实际应用,为未来的学术研究和职业生涯打下坚实的基础。

复变函数,作为数学的一个独特分支,将实数域扩展到复数域,从而揭示了许多在实数域中难以发现的数学规律和性质。通过复变函数,我们可以更加深入地理解自然界的许多现象,如电磁场的分布、流体的流动、振动与波动等。本书首先介绍了复数的基本概念、运算规则和几何意义,为读者后续学习复变函数提供了必要的数学基础。随后,我们深入探讨了复变函数的极限、连续性、可导性和可积性等基本性质,以及复变函数的解析性、零点与极值、全纯函数与调和函数等核心理论。

积分变换,特别是傅里叶变换和拉普拉斯变换,是现代科学与工程中广泛应用的数学工具。它们能够将复杂的微分方程转化为易于求解的代数方程,从而大大简化了问题的求解过程。本书在介绍复变函数的基础上,进一步探讨了积分变换的基本原理、性质和应用。通过详细的例子和丰富的习题,读者可以逐步掌握积分变换的基本方法和技巧,为解决实际工程问题提供有力的数学支持。

在编写过程中,我们注重理论与实践的结合,力求使内容既具有理论深度,又易于理解和应用。同时,我们也注重教材的趣味性和可读性,通过生动的语言、丰富的插图和有趣的数学故事,激发读者的学习兴趣和好奇心。此外,本书还配备了丰富的参考资料和进一步阅读的指引,以便读者在深入学习复变函数与积分变换时能够有更多的选择和参考。

我们相信,通过本书的学习,读者不仅能够掌握复变函数与积分变换的基本理论和方法,还能够培养自己的数学思维和解决问题的能力。无论是在学术研究、工程实践还是日常生活中,复变函数与积分变换都将成为读者不可或缺的数学工具。

最后,我们要感谢所有为本书编写提供支持和帮助的同事、朋友和家人。他们的无私奉献和辛勤付出,使得本书能够顺利出版并与广大读者见面。我们衷心希望本书能够成为读者学习复变函数与积分变换的良师益友,陪伴读者在数学探索的道路上不断前行。

让我们一起,以复变函数与积分变换为钥匙,打开科学与工程的神秘大门,探索未知的数学奥秘吧!

\vskip 0.5cm
\begin{flushright}
编者\\
\zhtoday
\end{flushright}


\tableofcontents




\mainmatter


% \begin{tikzpicture}
%   \begin{axis}[
%     color=white,
%     xlabel=$x$,
%     ylabel=$y$
%   ]
%     \addplot3[
%       surf,
%       % shader=interp,
%       patch,
%       patch type=bilinear,
%       mesh/color input=explicit mathparse,
%       mesh/colorspace explicit color output=rgb,
%       domain=0:2,
%       domain y=0:2,
%       samples=30,
%       point meta={
%         symbolic={
%           (sin(deg(y*pi*2))+1)/2, 
%           (sin(deg((y+0.5)*pi*2))+1)/2,
%           (sin(deg((y+1)*pi*2))+1)/2
%         }
%       }
%     ]
%     {exp(-x)};
%   \end{axis}
% \end{tikzpicture}


\chapter{复数与复变函数}

复数起源于多项式方程的求根问题. 
考虑一元二次方程 $x^2+bx+c=0$, 配方可得
  \[\left(x+\frac b2\right)^2=\frac{b^2-4c}4.\]
于是得到求根公式
  \[x=\frac{-b\pm\sqrt\Delta}2,\quad \Delta=b^2-4c.\]
\begin{enumerate}
  \item 当 $\Delta>0$ 时, 有两个不同的实根;
  \item 当 $\Delta=0$ 时, 有一个二重\footnote{如果 $x_0$ 是多项式方程 $f(x)=0$ 的根, 则 $x-x_0$ 是 $f(x)$ 的因式, 即存在多项式 $g(x)$ 使得 $f(x)=(x-x_0)g(x)$.
  如果 $(x-x_0)^k$ 是 $f(x)$ 的因式, 但 $(x-x_0)^{k+1}$ 不是, 则称 $x_0$ 是 \emph{$k$ 重根}.}的实根;
  \item 当 $\Delta<0$ 时, 无实根.
\end{enumerate}

可以看出, 在一元二次方程中, 我们可以舍去包含\alert{负数开平方}的解. 然而在一元三次方程中, 即便只考虑实数根也会不可避免地引入负数开平方.

\begin{example}
  解方程 $x^3+6x-20=0$.
\end{example}
我们将使用由 Scipione dal Ferro (1465--1526) 最先发现, 最终由 Cardan 公开的解法\footnote{Ferro 发现了该方法后, 并没有发表他的结果, 因为当时人们常把他们的发现保密, 而向对手们提出挑战.}.
\begin{solution}
  设 $x=u+v$, 则
    \[u^3+v^3+3uv(u+v)+6(u+v)-20=0.\]
  我们希望 $u^3+v^3=20, uv=-2$, 则 $u^3,v^3$ 满足一元二次方程 $X^2-20X-8=0$.
  解得
    \[u^3=10\pm\sqrt{108}{=(1\pm\sqrt3)^3.}\]
  所以 $u=1\pm\sqrt3, v=1\mp\sqrt 3$, $x=u+v=2$.
\end{solution}

那么这个方程是不是真的只有 $x=2$ 这一个解呢?
由 $f'(x)=3x^2+6>0$ 可知其单调递增, 因此确实只有一个解.

\begin{figure}[hbpt]
  \centering
  \begin{minipage}{.48\textwidth}
    \centering
    \begin{tikzpicture}
      \filldraw[cstcurve,main,domain=-1.2:3.1,smooth,fill=white] plot ({\x*.5},{(\x*\x*\x+6*\x-20)*.1});
      \draw[cstaxis] (-3,0)--(3,0);
      \draw[cstaxis] (0,-3)--(0,3);
      \coordinate [label=above left:$-20$] (A) at (0,-2);
      \coordinate [label=above left:$2$] (B) at (1,0);
      \fill[cstdot,second] (A) circle;
      \fill[cstdot,second] (B) circle;
    \end{tikzpicture}
    \caption{$y=x^3+6x-20$\footnotemark}
  \end{minipage}
  \begin{minipage}{.48\textwidth}
    \centering
    \begin{tikzpicture}
      \filldraw[cstcurve,main,domain=-3.6:3.2,smooth,fill=white] plot ({\x*.6},{(\x*\x*\x-7*\x+6)*.15});
      \draw[cstaxis] (-3,0)--(3,0);
      \draw[cstaxis] (0,-3)--(0,3);
      \coordinate [label=below left:$1$] (A) at (.6,0);
      \coordinate [label=above left:$2$] (B) at (1.2,0);
      \coordinate [label=above left:$-3$] (C) at (-1.8,0);
      \fill[cstdot,second] (A) circle;
      \fill[cstdot,second] (B) circle;
      \fill[cstdot,second] (C) circle;
    \end{tikzpicture}
    \caption{$y=x^3-7x+6$\footnotemark[1]}
  \end{minipage}
\end{figure}
\footnotetext{图像的横纵坐标比例有放缩}

\begin{example}
  解方程 $x^3-7x+6=0$.
\end{example}

\begin{solution}
  同样地我们有 $x=u+v$, 其中
    \[u^3+v^3=-6,\quad uv=\frac73.\]
  于是 $u^3,v^3$ 满足一元二次方程 $X^2+6X+\dfrac{343}{27}=0$.
  然而这个方程没有实数解.

  我们可以强行解得
    \[u^3=-3+\frac{10}9\sqrt{-3},\]
    \[u=\sqrt[3]{-3+\frac{10}9\sqrt{-3}}
      =\frac{3+2\sqrt{-3}}3,\frac{-9+\sqrt{-3}}6,\frac{3-5\sqrt{-3}}6,\]
  相应地
    \[v=\frac{3-2\sqrt{-3}}3,\frac{-9-\sqrt{-3}}6,\frac{3+5\sqrt{-3}}6,\]
  从而 $x=u+v=2,-3,1$.
\end{solution}

所以我们从一条``\alert{错误的路径}''走到了正确的目的地?

对于一般的三次方程 $x^3+px+q=0$ 而言, 类似可得:
\footnote{一般 $n$ 次多项式的判别式定义为 $\prod_{1\le i<j\le n}(x_i-x_j)^2$, 其中 $x_1,\dots,x_n$ 表示其所有(复数)根.
三次多项式的判别式应当是 $-108\Delta$, 这里为了运算方便取作如此形式.}
  \[x=u-\frac p{3u},\quad u^3=-\frac q2+\sqrt{\Delta},\quad \Delta=\frac{q^2}4+\frac{p^3}{27}.\]
由于 $p=0$ 情形较为简单, 所以我们不考虑这种情形.
通过分析函数图像的极值点可以知道:
\begin{enumerate}
  \item 当 $\Delta>0$ 时, 有 $1$ 个实根.
  \item 当 $\Delta=0$ 时, 有 $2$ 个实根 $x=-\sqrt[3]{4q},\dfrac12\sqrt[3]{4q}$ ($2$重).
  \item 当 $\Delta<0$ 时, 有 $3$ 个实根.
\end{enumerate}

\begin{figure}[hbpt]
  \centering
  \begin{minipage}{.32\textwidth}
    \centering
    \begin{tikzpicture}
      \draw[cstaxis] (-2,0)--(2,0);
      \draw[cstaxis] (0,-2)--(0,2);
      \draw[cstcurve,main,domain=-2.6:3.3,smooth] plot ({(\x)*0.35},{(\x*\x*\x-3*\x-10)*0.1});
      \fill[cstdot,second] (.92,0) circle;
    \end{tikzpicture}
    \caption{$\Delta>0$}
  \end{minipage}
  \begin{minipage}{.32\textwidth}
    \centering
    \begin{tikzpicture}
      \draw[cstaxis] (-2,0)--(2,0);
      \draw[cstaxis] (0,-2)--(0,2);
      \draw[cstcurve,main,domain=-3.1:2.8,smooth] plot ({(\x)*0.35},{(\x*\x*\x-3*\x+2)*0.1});
      \fill[cstdot,second] (.37,0) circle;
      \fill[cstdot,second] (-.69,0) circle;
    \end{tikzpicture}
    \caption{$\Delta=0$}
  \end{minipage}
  \begin{minipage}{.32\textwidth}
    \centering
    \begin{tikzpicture}
      \draw[cstaxis] (-2,0)--(2,0);
      \draw[cstaxis] (0,-2)--(0,2);
      \draw[cstcurve,main,domain=-4:3.9,smooth] plot ({(\x)*0.3},{(\x*\x*\x-7*\x+1)*0.05});
      \fill[cstdot,second] (.76,0) circle;
      \fill[cstdot,second] (.05,0) circle;
      \fill[cstdot,second] (-.78,0) circle;
    \end{tikzpicture}
    \caption{$\Delta<0$}
  \end{minipage}
\end{figure}

所以我们想要使用求根公式的话, 就\alert{必须接受负数开方}.
那么为什么当 $\Delta<0$ 时, 从求根公式一定能得到 $3$ 个实根呢?
在学习了本章内容之后就可以回答这个问题了.

尽管在十六世纪, 人们已经得到了三次方程的求根公式, 然而对其中出现的虚数, 却是难以接受.

\begin{tcolorbox}[
  common,
  borderline={0pt}{0pt}{fourth,cstdash},
  colbacktitle=fourth,
  fontlower=\itshape,
  halign lower=flush right,
  lower separated=true]
	圣灵在分析的奇观中找到了超凡的显示, 这就是那个理想世界的端兆, 那个介于存在与不存在之间的两栖物, 那个我们称之为虚的 $-1$ 的平方根。
  \tcblower
  莱布尼兹 (Leibniz)
\end{tcolorbox}

我们将在下一节使用更为现代的语言来解释和运用复数.


\section{复数及其代数运算}

\subsection{复数的概念}

现在我们来正式介绍复数的概念.
为了避免记号 $\sqrt{-1}$ 带来的歧义, 我们先引入抽象符号 $i$, 再通过定义它的运算来构造复数.\footnote{记号 $i$ 被称为\emph{虚数单位}, 它最先是由欧拉引入并使用.}

\begin{definition}[复数]
  固定一个记号 $i$, \noun{复数}就是形如 $z=x+yi$ 的元素, 其中 $x,y$ 均是实数, 且不同的 $(x,y)$ 对应不同的复数.
\end{definition}

换言之, 每一个复数可以唯一地表达成 $x+yi$ 这样的形式.

于是复数全体构成一个二维实线性空间, $\{1,i\}$ 是一组基. 而且实数 $x$ 可以自然地看成复数 $x+0i$.
将\emph{全体复数记作 $\BC$}, 全体实数记作 $\BR$, 则 $\BC=\BR+\BR i$, $\BR\subseteq \BC$.\footnote{
  全体复数、实数、有理数、整数、自然数集合分别记作 $\BC,\BR,\BQ,\BZ,\BN$, 整数来自德语 Zahlen, 其余来自它们的英文名称 complex number, real number, rational number, natural number.
}\footnote{
  这些符号的叫做空心体, 书写时, 可在普通字母格式上添加一条竖线(对于 $\BZ$ 是斜线)来区分.
}
由此, $\BC$ 和平面上的点可以建立一一对应, 并将建立起这种对应的平面称为\emph{复平面}.

\begin{figure}[hbpt]
  \centering
  \begin{tikzpicture}
    \begin{scope}
      \draw[cstaxis] (-.5,0)--(3,0);
      \draw[cstaxis] (0,-.5)--(0,2.5);
      \coordinate [label=below left:$0$] (O) at (0,0);
      \coordinate [label=above:\textcolor{second}{$z=x+yi$}] (A) at (2,1.5);
      \coordinate (B) at (2,0);
      \coordinate (C) at (0,1.5);
      \draw[cstdash] (B)--(A)--(C);
      \fill[cstdot,second] (A) circle;
      \draw[third,Latex-Latex,line width=.5mm] (2.8,1)--(4,1) node[midway,below,third] {一一对应};
    \end{scope}
    \begin{scope}[xshift=5cm]
      \coordinate [label=below left:$O$] (O) at (0,0);
      \coordinate [label=above:\textcolor{second}{$Z(x,y)$}] (A) at (2,1.5);
      \coordinate (B) at (2,0);
      \coordinate (C) at (0,1.5);
      \draw[cstdash] (B)--(A)--(C);
      \fill[cstdot,second] (A) circle;
      \draw[decorate,decoration={brace,amplitude=5},main,cstfill1] (O)--(B) node[midway,above=2mm] {$x$};
      \draw[decorate,decoration={brace,amplitude=5},main,cstfill1] (C)--(O) node[midway,right=2mm] {$y$};
      \draw[third,Latex-Latex,line width=.5mm] (2.8,1)--(4,1) node[midway,below,third] {一一对应};
      \draw[cstaxis] (-.5,0)--(3,0);
      \draw[cstaxis] (0,-.5)--(0,2.5);
    \end{scope}
    \begin{scope}[xshift=10cm]
      \draw[cstaxis] (-.5,0)--(3,0);
      \draw[cstaxis] (0,-.5)--(0,2.5);
      \coordinate [label=below left:$O$] (O) at (0,0);
      \coordinate [label=above:\textcolor{second}{$\overrightarrow{OZ}=(x,y)$}] (A) at (2,1.5);
      \draw[cstcurve,cstra,second] (O)--(A);
    \end{scope}
  \end{tikzpicture}
  \caption{复数、平面上的点、平面向量一一对应}
\end{figure}

当 $y=0$ 时, $z=x$ 就是一个实数.
它对应复平面上的点就是直角坐标系的 $x$ 轴上的点.
因此我们称 $x$ 轴为\emph{实轴}.
相应地, 称 $y$ 轴为\emph{虚轴}.
称 $z=x+yi$ 在实轴和虚轴的投影为它的\emph{实部 $\Re z=x$} 和\emph{虚部 $\Im z=y$}.

当 $\Im z=0$ 时, $z$ 是实数.
不是实数的复数是\emph{虚数}.
当 $\Re z=0$ 且 $z\neq 0$ 时, 称 $z$ 是\emph{纯虚数}.

\begin{figure}[hbpt]
  \centering
  \begin{minipage}{.48\textwidth}
    \centering
    \begin{tikzpicture}
      \coordinate [label=above:\textcolor{main}{实轴}] (R) at (3,0);
      \coordinate [label=right:\textcolor{second}{虚轴}] (I) at (0,2.5);
      \coordinate [label=below left:$0$] (O) at (0,0);
      \coordinate [label=above:\textcolor{third}{$z=x+yi$}] (A) at (2,1.5);
      \coordinate (B) at (2,0);
      \coordinate (C) at (0,1.5);
      \draw[cstdash] (B)--(A)--(C);
      \fill[cstdot,third] (A) circle;
      \draw[decorate,decoration={brace,amplitude=5},main,cstfill1] (B)--(O) node[midway,below=1.5mm] {$\Re z$};
      \draw[decorate,decoration={brace,amplitude=5},second,cstfill2] (C)--(O) node[midway,right=1.5mm] {$\Im z$};
      \draw[cstaxis,main] (-.5,0)--(R);
      \draw[cstaxis,second] (0,-.5)--(I);
      \draw[main,->,thick] (-2,.2)-|(.6,0);
      \draw[second,->,thick] (-2,1.3)--(0,1.3);
      \draw 
        (-2,.1) node[cstnode,draw=main,text=main] {实数}
        (-2,1.2) node[align=center,cstnode,draw=second,text=second] {纯虚数\\不含原点};
    \end{tikzpicture}
  \end{minipage}
  \begin{minipage}{.48\textwidth}
    \centering
    \begin{tikzpicture}
      \filldraw[cstcurve,cstfill] (.8,0) circle (2.6 and 2);
      \coordinate (R) at (0,-.8);
      \filldraw[cstcurve,main,fill=white] (R) circle (1.2 and .7);
      \coordinate (I) at (0,.8);
      \draw (R) node[align=center,main] {实数 \\$0,1,\sqrt2,\pi,e$};
      \draw (I) node[align=center,second] {纯虚数 \\$i,-i,\pi i$};
      \draw[cstcurve,second] (I) circle (1.2 and .7);
      \draw 
        (3.7,0) node[align=center] {全\\体\\复\\数}
        (2,0) node[align=center,third] {虚数 \\$i,\pi i,\frac{-1+\sqrt 3 i}2$};
    \end{tikzpicture}
  \end{minipage}
  \caption{实数、纯虚数、复数和复平面的关系}
\end{figure}

\begin{example}
  实数 $x$ 取何值时, $z=(x^2-3x-4)+(x^2-5x-6)i$ 是:
  \begin{enumerate}
    \item 实数;
    \item 纯虚数.
  \end{enumerate}
\end{example}
\begin{solution}
  \begin{enumerate}
    \item $\Im z=x^2-5x-6=0$, 即 $x=-1$ 或 $6$.
    \item $\Re z=x^2-3x-4=0$, 即 $x=-1$ 或 $4$.
      但同时要求 $\Im z=x^2-5x-6\neq 0$, 因此 $x\neq -1$.
      故 $x=4$.
  \end{enumerate}
\end{solution}

\begin{exercise}
  若 $x^2(1+i)+x(5+4i)+4+3i$ 是纯虚数, 则实数 $x=$\fillblank{}.
\end{exercise}


\subsection{复数的代数运算}

我们将不言自明地使用 $x,y,x_1,y_1,\dots$ 等记号表示实数.

\subsubsection*{四则运算}
设 $z_1=x_1+y_1i,z_2=x_2+y_2i$.
由 $\BC$ 是二维实线性空间可得复数的加法和减法:
  \begin{align*}
    z_1+z_2&=(x_1+x_2)+(y_1+y_2)i,\\
    z_1-z_2&=(x_1-x_2)+(y_1-y_2)i.
  \end{align*}
复数的加减法与其对应的向量 $\overrightarrow{OZ}$ 的加减法是一致的.

\begin{figure}[hbpt]
  \centering
  \begin{tikzpicture}
    \draw[cstaxis] (-2,0)--(4,0);
    \draw[cstaxis] (0,-3)--(0,2.5);
    \coordinate (O) at (0,0);
    \coordinate [label=right:\textcolor{main}{$z_1$}] (Z1) at (2.5,-1);
    \coordinate [label=above:\textcolor{main}{$z_2$}] (Z2) at (1.5,2);
    \coordinate [label=above right:\textcolor{second}{$z_1+z_2$}] (P) at ($(Z1)+(Z2)$);
    \coordinate [label=below:\textcolor{third}{$z_1-z_2$}] (M) at ($(Z1)-(Z2)$);
    \coordinate [label=left:{$-z_2$}] (neg) at ($(O)-(Z2)$);
    \draw[cstcurve,cstra,main] (O)--(Z1);
    \draw[cstcurve,cstra,main] (O)--(Z2);
    \draw[cstcurve,cstra,second] (O)--(P);
    \draw[cstcurve,cstra,third] (O)--(M);
    \draw[cstdash,cstra] (O)--(neg);
    \draw[cstdash] (Z2)--(P)--(M)--(neg);
  \end{tikzpicture}
  \caption{复数的加法和减法}
\end{figure}

\alert{规定 $i\cdot i=-1$} 并要求实数与复数的乘法和标量乘法(数乘)一致.
我们希望 $\BC$ 上的运算满足乘法分配律, 则
  \begin{align*}
    z_1\cdot z_2&=(x_1+y_1i)(x_2+y_2i)\\
    &=x_1\cdot x_2+x_1\cdot y_2i+y_1i\cdot x_2+y_1i\cdot y_2i\\
    &=(x_1x_2-y_1y_2)+(x_1y_2+x_2y_1)i.
  \end{align*}
由此可得 $z\neq0$ 时,
  \[\frac1{z}=\frac{x-yi}{x^2+y^2},\]
从而
  \[\frac{z_1}{z_2}=z_1\cdot\frac1z=\frac{x_1x_2+y_1y_2}{x_2^2+y_2^2}+\frac{x_2y_1-x_1y_2}{x_2^2+y_2^2}i.\]

对于正整数 $n$, 定义 $z$ 的 \emph{$n$ 次幂}为 $n$ 个 $z$ 相乘.
当 $z\neq 0$ 时, 还可以定义 $z^0:=1,z^{-n}:=\dfrac1{z^n}$.

\subsubsection*{单位根}
\begin{example}
  \begin{enumerate}
    \item $i^2=-1,i^3=-i,i^4=1$.
    一般地, 对于整数 $n$, 
    \[i^{4n}=1,\quad i^{4n+1}=i,\quad i^{4n+2}=-1,\quad i^{4n+3}=-i.\]
    \item 令 $\omega=\dfrac{-1+\sqrt 3i}2$, 则 $\omega^2=\dfrac{-1-\sqrt3i}2,\omega^3=1$.
    \item 令 $z=1+i$, {则
    \[z^2=2i,\quad z^3=-2+2i,\quad z^4=-4,\quad z^8=16=2^4.\]}
  \end{enumerate}
  将满足 $z^n=1$ 的复数 $z$ 称为 \emph{$n$ 次单位根}.
  那么 $1,i,-1,-i$ 是 $4$ 次单位根, $1,\omega,\omega^2$ 是 $3$ 次单位根.
\end{example}

\begin{example}
  化简 $1+i+i^2+i^3+i^4$.
\end{example}
\begin{solution}
  根据等比数列求和公式,
  \[1+i+i^2+i^3+i^4=\frac{i^5-1}{i-1}
  {=\frac{i-1}{i-1}=1.}\]
\end{solution}

\begin{exercise}
  化简 $\left(\dfrac{1-i}{1+i}\right)^{2020}$=\fillblank{}.
\end{exercise}

\subsubsection*{复数域的性质}

复数全体构成一个\emph{域}.
所谓的域, 是指带有如下内容和性质的集合:
\begin{itemize}
  \item 包含 $0,1$, 且有四则运算;\footnote{即有运算 $+$ 和 $\times$, 且对任意 $a$, 存在 $b$ 使得 $a+b=b+a=0$; 对任意 $a\neq 0$, 存在 $c$ 使得 $a\times c=c\times a=1$.}
  \item 满足加法结合、交换律, 乘法结合、交换、分配律;
  \item 对任意 $a$, $a+0=a\times 1=a$.
\end{itemize}
有理数全体 $\BQ$, 实数全体 $\BR$ 也构成域, 它们是 $\BC$ 的子域.
与有理数域和实数域有着本质不同的是, 复数域是\emph{代数闭域}:
对于任何次数 $n\ge 1$ 的复系数多项式
  \[p(z)=z^n+c_{n-1}z^{n-1}+\cdots+c_1z+c_0,\]
都存在复数 $z_0$ 使得 $p(z_0)=0$.
由此不难知道, 复系数多项式可以因式分解成一次多项式的乘积.
我们会在第五章证明该结论.

在 $\BQ,\BR$ 上可以定义出一个``好的''大小关系, 换言之它们是\emph{有序域}, 即存在一个满足下述性质的 $>$:
\begin{itemize}
  \item 若 $a\neq b$, 则要么 $a>b$, 要么 $b>a$;
  \item 若 $a>b$, 则对于任意 $c$, $a+c>b+c$;
  \item 若 $a>b,c>0$, 则 $ac>bc$.
\end{itemize}
而 \alert{$\BC$ 却不是有序域}.
如果 $i>0$, 则
  \[-1=i\cdot i>0,\quad -i=-1\cdot i>0.\]
于是 $0>i$, 矛盾! 同理 $i<0$ 也不可能.


\subsection{共轭复数}

\begin{definition}[共轭复数]
  称 $z$ 在复平面关于实轴的对称点为它的\emph{共轭复数 $\ov z$}.
换言之, $\ov{x+yi}=x-yi$.
\end{definition}
从定义出发, 不难验证共轭复数满足如下性质:
\begin{enumerate}
  \item $z$ 是 $\ov z$ 的共轭复数.
  \item $\ov{z_1\pm z_2}=\ov{z_1}\pm\ov{z_2},\ 
  \ov{z_1\cdot z_2}=\ov{z_1}\cdot\ov{z_2},\ 
  \ov{z_1/z_2}=\ov{z_1}/\ov{z_2}$.
  \item $z\ov{z}=(\Re z)^2+(\Im z)^2$.
  \item $z+\ov z=2\Re z,\ z-\ov z=2i\Im z$.
  \item $z=\ov z\iff z$ 是实数; $z=-\ov z\iff z$ 是纯虚数或 $z=0$.
\end{enumerate}
\enumnum4表明了 $x,y$ 可以用 $z,\ov z$ 表出.
\enumnum2表明共轭复数和四则运算交换.
这意味着使用共轭复数进行计算和证明,往往比直接使用 $x,y$ 表达的形式更简单.

\begin{exercise}
  $z$ 关于虚轴的对称点是\fillblank{}.
\end{exercise}
\begin{example}
  证明 $z_1\cdot\ov{z_2}+\ov{z_1}\cdot z_2=2\Re(z_1\cdot\ov{z_2})$.
\end{example}

我们可以设 $z_1=x_1+y_1i,z_2=x_2+y_2i$, 然后代入等式两边化简并比较实部和虚部得到.
但利用共轭复数可以更简单地证明它.

\begin{proof}
  由于 $\ov{z_1\cdot\ov{z_2}}=\ov{z_1}\cdot\ov{\ov{z_2}}=\ov{z_1}\cdot z_2$, 因此
    \[z_1\cdot\ov{z_2}+\ov{z_1}\cdot z_2
      =z_1\cdot\ov{z_2}+\ov{z_1\cdot\ov{z_2}}
      =2\Re(z_1\cdot\ov{z_2}).\qedhere\]
\end{proof}

\begin{example}
  设 $z=x+yi$ 且 $y\neq 0,\pm1$. 证明: $x^2+y^2=1$ 当且仅当 $\dfrac z{1+z^2}$ 是实数.
\end{example}
\begin{proof}
  $\dfrac z{1+z^2}$ 是实数当且仅当
    \[\frac z{1+z^2}=\ov{\left(\frac z{1+z^2}\right)}=\frac{\ov z}{1+{\ov z}^2},\]
  即
    \[z(1+{\ov z}^2)=\ov z(1+z^2),\quad (z-\ov z)(z\ov z-1)=0.\]%
  由 $y\neq0$ 可知 $z\neq \ov z$.
  故上述等式等价于 $z\ov z=1$, 即 $x^2+y^2=1$.
\end{proof}

由于 $z\ov z$ 是一个实数,
因此在做复数的除法运算时, 可以利用下式将其转化为乘法:
  \[\dfrac{z_1}{z_2}=\dfrac{z_1\ov{z_2}}{z_2\ov{z_2}}=\dfrac{z_1\ov{z_2}}{x_2^2+y_2^2}.\]
\begin{example}
  设 $z=-\dfrac1i-\dfrac{3i}{1-i}$, 求 $\Re z,\Im z$ 以及 $z\ov z$.
\end{example}
\begin{solution}
  \[z=-\frac1i-\frac{3i}{1-i}
  {=i-\frac{3i-3}2=\frac32-\half i,}\]
  因此
    \[\Re z=\frac32,\quad\Im z=-\half ,\quad
    z\ov z=\left(\frac32\right)^2+\left(-\half\right)^2=\frac52.\]
\end{solution}

\begin{example}
  设 $z_1=5-5i,z_2=-3+4i$, 求 $\ov{\left(\dfrac{z_1}{z_2}\right)}$.
\end{example}
\begin{solution}
  \begin{align*}
    \frac{z_1}{z_2}&=\frac{5-5i}{-3+4i}
    =\frac{(5-5i)(-3-4i)}{(-3)^2+4^2}\\
    &=\frac{(-15-20)+(-20+15)i}{25}=-\frac75-\frac15i,
  \end{align*}
  因此 $\ov{\left(\dfrac{z_1}{z_2}\right)}=-\dfrac75+\dfrac15i$.
\end{solution}


\section{复数的三角与指数形式}

\subsection{复数的模和辐角}

由平面的极坐标表示, 我们可以得到复数的另一种表示方式.
以 $0$ 为极点, 正实轴为极轴, 逆时针为极角方向可以自然定义出复平面上的极坐标系.

通过极坐标和直角坐标的转化关系可知:
\[ x=r\cos\theta,\qquad y=r\sin\theta,\]
\[r=\sqrt{x^2+y^2},\qquad \theta=\arctan\dfrac yx\text{ 或 }\arctan\dfrac yx\pm\pi.\]

\begin{definition}[模和辐角]
  \begin{itemize}
    \item 称 $r$ 为 $z$ 的\emph{模}, 记为 \emph{$|z|=r$}.
    \item 称 $\theta$ 为 $z$ 的\emph{辐角}, 记为 \emph{$\Arg z=\theta$}.
    \alert{$0$ 的辐角没有意义}.
  \end{itemize}
\end{definition}


\begin{figure}[hbpt]
  \begin{minipage}{.45\textwidth}
    \centering
    \begin{tikzpicture}
      \coordinate [label=below left:$0$] (O) at (0,0);
      \coordinate [label=above:\textcolor{second}{$z=x+yi$}] (Z) at (3,2);
      \coordinate (X) at (3,0);
      \coordinate (Y) at (0,2);
      \draw[decorate,decoration={brace,amplitude=5},main,cstfill1] (X)--(O) node[midway,below=2mm] {$x$};
      \draw[decorate,decoration={brace,amplitude=5},main,cstfill1] (O)--(Y) node[midway,left=2mm] {$y$};
      \draw[third,thick,cstra] pic [cstfill3,draw=third, "$\theta$", angle eccentricity=1.3, angle radius=0.8cm] {angle=X--O--Z};
      \draw[cstaxis] (-.5,0)--(4,0);
      \draw[cstaxis] (0,-.5)--(0,3);
      \draw[cstcurve,third,cstra] (O)--(Z) node[midway,above,third] {$r$};
      \draw[cstdash] (X)--(Z)--(Y);
      \fill[cstdot,second] (Z) circle;
    \end{tikzpicture}
    \caption{复数的模和辐角}
  \end{minipage}
  \begin{minipage}{.45\textwidth}
    \centering
    \begin{tikzpicture}
      \draw[cstaxis](-2.5,0)->(2.5,0); 
      \draw[cstaxis](0,-2)->(0,2);
      \draw[cstaxis,main,cstwla] (-1.5,0) arc(180:-180:1.5);
      \filldraw[cstdote,draw=main] (-1.5,-.07) circle;
      \coordinate [label=above:\textcolor{main}{$0$}] (A) at (1.7,0);
      \fill[cstdot,main] (A) circle;
      \coordinate [label=above right:\textcolor{main}{$\arctan\dfrac yx$}] (B) at (.9,.9);
      \fill[cstdot,main] (B) circle;
      \coordinate [label=right:\textcolor{main}{$\arctan\dfrac yx$}] (C) at (1.4,-.9);
      \fill[cstdot,main] (C) circle;
      \coordinate [label=above left:\textcolor{second}{$\arctan\dfrac yx+\pi$}] (D) at (-1.1,.8);
      \fill[cstdot,second] (D) circle;
      \coordinate [label=below:\textcolor{second}{$\pi$}] (E) at (-.9,0);
      \fill[cstdot,second] (E) circle;
      \coordinate [label=left:\textcolor{third}{$\arctan\dfrac yx-\pi$}] (F) at (-1.6,-.7);
      \fill[cstdot,third] (F) circle;
      \coordinate [label=left:\textcolor{fourth}{$\dfrac\pi2$}] (G) at (0,.5);
      \fill[cstdot,fourth] (G) circle;
      \coordinate [label=right:\textcolor{fourth}{$-\dfrac\pi2$}] (H) at (0,-.6);
      \fill[cstdot,fourth] (H) circle;
    \end{tikzpicture}
    \caption{主辐角与复数位置的关系}
  \end{minipage}
\end{figure}

任意 $z\neq 0$ 的辐角有无穷多个.
我们固定选择其中位于 $(-\pi,\pi]$ 的那个, 并称之为\emph{主辐角}或\emph{辐角主值}\footnote{选择位于 $[0,2\pi)$ 的那个作为辐角主值也是一种常见的选择, 但这会导致后续中对数函数主值在正实轴上不解析. 因此我们作此选择.}, 记作 \emph{$\arg z$}.
那么 \alert{$\Arg z=\arg z+2k\pi, k\in\BZ$},
\[\arg z=\begin{cases}
  \arctan\dfrac yx,&x>0;\\
  \arctan\dfrac yx+\pi,&x<0,y\ge0;\\
  \arctan\dfrac yx-\pi,&x<0,y<0;\\
  \dfrac\pi2,&x=0,y>0;\\
  -\dfrac\pi2,&x=0,y<0.
\end{cases}\]

注意 \alert{$\arg \ov z=-\arg z$ 未必成立}, 仅当 $z$ 不是负实数和 $0$ 时成立.

复数的模满足如下性质:
\begin{enumerate}
  \item $z\ov z=|z|^2=|\ov z|^2$;
  \item $\abs{\Re z},\abs{\Im z}\le |z|\le\abs{\Re z}+\abs{\Im z}$;
  \item $\big||z_1|-|z_2|\big|\le|z_1\pm z_2|\le|z_1|+|z_2|$;
  \item $|z_1+z_2+\cdots+z_n|\le|z_1|+|z_2|+\cdots+|z_n|$.
\end{enumerate}

\begin{figure}[hbpt]
  \centering
  \begin{tikzpicture}
    \draw[cstaxis] (-4.4,0)--(4.3,0);
    \draw[cstaxis] (0,-3)--(0,3);
    \coordinate (O) at (0,0);
    \coordinate (Z) at (-2.5,1.5);
    \coordinate (R) at (-2.5,0);
    \draw[decorate,decoration={brace,amplitude=5},main,cstfill1] (O)--(R) node[midway,below=2mm,main] {$\abs{\Re z}$};
    \draw[decorate,decoration={brace,amplitude=5},main,cstfill1] (R)--(Z) node[midway,left=2mm,main] {$\abs{\Im z}$};
    \draw[cstcurve,second] (O)--(Z) node[midway,above,second] {$|z|$};
    \draw[cstcurve,main] (Z)--(R)--(O);
    \draw[thick] (R) ++(0,.3)--++(.3,0)--++(0,-.3);

    \coordinate [label=below right:\textcolor{main}{$z_1$}] (Z1) at (2.8,-.4);
    \coordinate (Z2) at (1.2,2);
    \coordinate (A) at (2.7,2.4);
    \coordinate [label=above right:\textcolor{main}{$z_1+z_2$}] (P) at ($(Z1)+(Z2)$);
    \coordinate [label=below:\textcolor{third}{$z_1-z_2$}] (M) at ($(Z1)-(Z2)$);
    \draw[decorate,decoration={brace,amplitude=5},main] (Z1)--(O) node[midway,below,sloped] {$|z_1|$};
    \draw[decorate,decoration={brace,amplitude=5},main] (P)--(Z1) node[midway,below,sloped] {$|z_2|$};
    \draw[decorate,decoration={brace,amplitude=5},second] (O)--(P) node[midway,above,sloped] {$|z_1+z_2|$};
    \draw[decorate,decoration={brace,amplitude=5},main] (Z1)--(M) node[midway,below,sloped] {$|z_2|$};
    \draw[decorate,decoration={brace,amplitude=5},third] (M)--(O) node[midway,below,sloped] {$|z_1-z_2|$};
    \draw[decorate,decoration={brace,amplitude=5},main] (A)--(P) node[midway,above,sloped] {$|z_3|$};
    \draw[decorate,decoration={brace,amplitude=5},fourth] (O)--(A) node[midway,above=2mm,sloped] {$|z_1+z_2+z_3|$};
    \begin{scope}[cstcurve,cstra]
      \draw[main] (O)--(Z1);
      \draw[main] (Z1)--(P);
      \draw[main] (P)--(A);
      \draw[fourth] (O)--(A);
      \draw[second] (O)--(P);
      \draw[third] (O)--(M);
      \draw[main] (Z1)--(M);
    \end{scope}
  \end{tikzpicture}
  \caption{复数模的不等式关系}
\end{figure}

\begin{exercise}
  什么时候 $|z_1+z_2+\cdots+z_n|=|z_1|+|z_2|+\cdots+|z_n|$?
\end{exercise}

\begin{example}
  证明
  \begin{enumerate}
    \item $|z_1z_2|=|z_1\ov{z_2}|=|z_1|\cdot|z_2|$;
    \item $|z_1+z_2|^2=|z_1|^2+|z_2|^2+2\Re(z_1\ov{z_2})$.
  \end{enumerate}
\end{example}

\begin{proof}
  \begin{enumerate}
    \item 因为
    \[|z_1z_2|^2=z_1z_2\cdot\ov{z_1}\ov{z_2}
    =z_1z_2\ov{z_1}\ov{z_2}=|z_1|^2\cdot|z_2|^2,\]
    所以 $|z_1z_2|=|z_1|\cdot|z_2|$.
    因此 $|z_1\ov{z_2}|=|z_1|\cdot|\ov{z_2}|=|z_1|\cdot|z_2|$.
    \item 因为
    \begin{align*}
      \text{左边}&=(z_1+z_2)(\ov{z_1}+\ov{z_2})
      {=z_1\ov{z_1}+z_2\ov{z_2}+z_1\ov{z_2}+\ov{z_1}z_2,}\\
      \text{右边}&=z_1\ov{z_1}+z_2\ov{z_2}+z_1\ov{z_2}+\ov{z_1\ov{z_2}},
    \end{align*}
    而 $\ov{z_1\ov{z_2}}=\ov{z_1}z_2$, 所以两侧相等.\qedhere
  \end{enumerate}
\end{proof}


\subsection{复数的三角形式和指数形式}

由 $x=r\cos\theta,y=r\sin\theta$ 可得

\begin{definition}[复数的三角形式]
  \[z=r(\cos\theta+i\sin\theta).\]	
\end{definition}

定义 \alert{$e^{i\theta}=\exp(i\theta):=\cos\theta+i\sin\theta$}\footnote{此即欧拉恒等式, 我们会在第二章说明为何如此定义.}, 则我们得到

\begin{definition}[复数的指数形式]
  \[z=re^{i\theta}=r\exp(i\theta).\]
\end{definition}
这两种形式的等价的, 指数形式可以认为是三角形式的一种缩写方式.

求复数的三角和指数形式的\alert{关键在于计算模和辐角}.

\begin{example}
  将 $z=-\sqrt{12}-2i$ 化成三角形式和指数形式.
\end{example}

\begin{solution}
  $r=|z|=\sqrt{12+4}=4$.
  由于 $z$ 在第三象限, 因此
    \[\arg z=\arctan\frac{-2}{-\sqrt{12}}-\pi=\frac\pi6-\pi=-\frac{5\pi}6.\]
  故
    \[z=4\left[\cos\left(-\frac{5\pi}6\right)+i\sin\left(-
    \frac{5\pi}6\right)\right]=4e^{-\frac{5\pi i}6}.\]
\end{solution}

\begin{example}
  将 $z=\sin\dfrac\pi5+i\cos\dfrac\pi5$ 化成三角形式和指数形式.
\end{example}
\begin{solution}
  $r=|z|=1$. 由于 $z$ 在第一象限, 因此
  \[\arg z=\arctan\frac{\cos(\pi/5)}{\sin(\pi/5)}=\arctan\cot\frac\pi 5=\frac\pi2-\frac\pi5=\frac{3\pi}{10}.\]
  故
  \[z=\displaystyle\cos\frac{3\pi}{10}+i\sin\frac{3\pi}{10}=e^{\frac{3\pi i}{10}}.\]
\end{solution}
\begin{solution}[另解]
  \[z=\sin\frac\pi5+i\cos\frac\pi5
  =\cos\left(\frac\pi2-\frac\pi5\right)+i\sin\left(\frac\pi2-\frac\pi5\right)
  =\cos\frac{3\pi}{10}+i\sin\frac{3\pi}{10}=e^{\frac{3\pi i}{10}}.\]
\end{solution}

求复数的三角或指数形式时, 只需要任取一个辐角就可以了, 不要求必须是主辐角.

\begin{exercise}
  将 $z=\sqrt 3-3i$ 化成三角形式和指数形式.
\end{exercise}

两个模相等的复数之和的三角和指数形式形式较为简单:
\[e^{i\theta}+e^{i\varphi}=2\cos\frac{\theta-\varphi}2e^{\frac{\theta+\varphi}2i}.\]
注意 $\cos\dfrac{\theta-\varphi}2<0$ 时, 这离指数形式还差一步变形.

\begin{example}
  如果 $|z|=1,\arg z=\theta$, 则 $z+1=2\cos\dfrac\theta2 e^{\frac{\theta i}2}$.
\end{example}

\begin{figure}[hbpt]
  \centering
  \begin{tikzpicture}
    \coordinate [label=below left:0] (O) at (0,0);
    \coordinate [label=right:\textcolor{main}{$e^{i\varphi}$}] (Z1) at ({3*cos(18)},{3*sin(18)});
    \coordinate [label=left:\textcolor{main}{$e^{i\theta}$}] (Z2) at ({3*cos(130)},{3*sin(130)});
    \coordinate [label=above:\textcolor{second}{$e^{i\theta}+e^{i\varphi}$}] (P) at ($(Z1)+(Z2)$);
    \coordinate (M) at ($0.5*(P)$);
    \coordinate (X) at (2,0);
    \draw[thick,main] pic [cstfill1, draw=main,"$\varphi$", angle eccentricity=1.4, angle radius=0.7cm] {angle=X--O--Z1};
    \draw[thick,second] pic [cstfill2, draw=second, "$\frac{\theta-\varphi}2$", angle eccentricity=1.7] {angle=Z1--O--P};
    \draw[cstaxis] (-3,0)--(3,0);
    \draw[cstaxis] (0,-.4)--(0,3.5);
    \draw[cstcurve,cstra,main] (O)--(Z1);
    \draw[cstcurve,cstra,main] (O)--(Z2);
    \draw[cstcurve,cstra,second] (O)--(P);
    \draw[cstdash] (Z2)--(Z1)--(P)--(Z2);
    \draw[thick] (M)--++({.3*cos(16)},{-.3*sin(16)})--++({.3*sin(16)},{.3*cos(16)})--++({-.3*cos(16)},{.3*sin(16)});
  \end{tikzpicture}
  \caption{模相等的复数之和}
\end{figure}

\section{复数的乘除、乘幂和方根}

\subsection{复数的乘除与三角、指数形式}

三角和指数形式在进行复数的乘法、除法和幂次计算中非常方便.

\begin{theorem}[复数的乘除与三角、指数形式]
  设
  \[z_1=r_1(\cos\theta_1+i\sin\theta_1)=r_1e^{i\theta_1},\]
  \[z_2=r_2(\cos\theta_2+i\sin\theta_2)=r_2e^{i\theta_2}\neq 0,\]
  则
  \begin{align*}
    z_1z_2&=r_1r_2[\cos(\theta_1+\theta_2)+i\sin(\theta_1+\theta_2)]=r_1r_2e^{i(\theta_1+\theta_2)},\\
    \frac{z_1}{z_2}&=\frac{r_1}{r_2}[\cos(\theta_1-\theta_2)+i\sin(\theta_1-\theta_2)]=\frac{r_1}{r_2}e^{i(\theta_1-\theta_2)}.
  \end{align*}
\end{theorem}

换言之\footnote{多值函数相等是指两边所能取到的值构成的集合相等.
例如此处关于辐角的等式的含义是:
  \[\Arg(z_1z_2)=\{\theta_1+\theta_2\mid\theta_1\in\Arg z_1,\theta_2\in\Arg z_2\}.\]
  \[\Arg\Bigl(\frac{z_1}{z_2}\Bigr)=\{\theta_1-\theta_2\mid\theta_1\in\Arg z_1,\theta_2\in\Arg z_2\}.\]
},
  \[|z_1z_2|=|z_1|\cdot|z_2|,\quad\abs{\frac{z_1}{z_2}}=\frac{|z_1|}{|z_2|},\]
  \[\Arg(z_1z_2)=\Arg z_1+\Arg z_2,\quad
  \Arg\Bigl(\frac{z_1}{z_2}\Bigr)=\Arg z_1-\Arg z_2.\]
注意上述等式中 $\Arg$ 不能换成 $\arg$, 也就是说
  \[\arg(z_1z_2)=\arg z_1+\arg z_2,\quad
  \arg\Bigl(\frac{z_1}{z_2}\Bigr)=\arg z_1-\arg z_2\]
\alert{不一定成立}.
事实上, 当且仅当等式右侧落在区间 $(-\pi,\pi]$ 内时才成立, 否则等式两侧会相差 $\pm2\pi$.
例如 $z_1=z_2=e^{0.99\pi i}$, $z_1z_2=e^{1.98\pi i}$,
\[\arg z_1+\arg z_2=0.99\pi+0.99\pi=1.98\pi,\qquad
\arg(z_1z_2)=-.02\pi.\]

\begin{proof}
  根据和差的正弦、余弦公式可知
  \begin{align*}
    z_1z_2&=r_1(\cos\theta_1+i\sin\theta_1)\cdot
    r_2(\cos\theta_2+i\sin\theta_2)\\
    &{=r_1r_2\bigl[(\cos\theta_1\cos\theta_2-\sin\theta_1\sin\theta_2)
    +i(\cos\theta_1\sin\theta_2+\sin\theta_1\cos\theta_2)\bigr]}\\
    &{=r_1r_2\bigl[\cos(\theta_1+\theta_2)+i\sin(\theta_1+\theta_2)\bigr]}
  \end{align*}
  因此乘法情形得证.

  设 $\dfrac{z_1}{z_2}=re^{i\theta}$, 则由乘法情形可知
    \[rr_2=r_1,\quad \theta+2k\pi+\Arg z_2=\Arg z_1.\]
  因此 $r=\dfrac{r_1}{r_2}$, $\theta$ 可取 $\theta_1-\theta_2$.
\end{proof}


\subsubsection*{复数乘法的几何意义}
从该定理可以看出, 乘以复数 $z=re^{i\theta}$ 可以理解为\alert{模放大为 $r$ 倍, 并沿逆时针旋转角度 $\theta$}.

\begin{figure}[hbpt]
  \centering
  \begin{tikzpicture}
    \coordinate [label=below:\textcolor{main}{$1$}] (X) at (1.6,0);
    \coordinate (O) at (0,0);
    \coordinate [label=right:\textcolor{main}{$z=re^{i\theta}$}] (Z) at ({2*cos(50)},{2*sin(50)});
    \coordinate [label=right:\textcolor{second}{$z_1$}] (Z1) at ({2.4*cos(80)},{2.4*sin(80)});
    \coordinate [label=right:\textcolor{second}{$zz_1$}] (ZZ1) at ({3*cos(130)},{3*sin(130)});
    \draw[cstcurve,main] pic [cstfill1,draw=main, "$\theta$", angle eccentricity=1.4] {angle=X--O--Z};
    \draw[cstcurve,second] pic [cstfill2,draw=second, "$\theta$", angle eccentricity=1.4] {angle=Z1--O--ZZ1};
    \draw[cstaxis] (-3,0)--(3,0);
    \draw[cstaxis] (0,-.5)--(0,3);
    \draw[cstcurve,main,cstra] (O)--(X);
    \draw[cstcurve,main,cstra] (O)--(Z);
    \draw[cstcurve,second,cstra] (O)--(Z1);
    \draw[cstcurve,second,cstra] (O)--(ZZ1);
  \end{tikzpicture}
  \caption{复数乘法的几何意义}
\end{figure}

\begin{example}
  已知正三角形的两个顶点为 $z_1=1$ 和 $z_2=2+i$, 求它的另一个顶点.
\end{example}

\begin{center}
  \begin{tikzpicture}
    \coordinate [label=below:\textcolor{third}{$z_1$}] (Z1) at (1.5,0);
    \coordinate [label=right:\textcolor{third}{$z_2$}] (Z2) at (3,1.5);
    \coordinate [label=left:\textcolor{main}{$z_3$}] (Z3) at ({1.5*(1.5-sqrt(3)/2)},{1.5*(.5+sqrt(3)/2)});
    \coordinate [label=right:\textcolor{second}{$z'_3$}] (Z3p) at ({1.5*(1.5+sqrt(3)/2)},{1.5*(.5-sqrt(3)/2)});
    \draw[cstcurve,main] pic [cstfill1,draw=main, "$\pi/3$", angle eccentricity=1.7,angle radius=4mm] {angle=Z2--Z1--Z3};
    \draw[cstcurve,second] pic [cstfill2,draw=second, "$\pi/3$", angle eccentricity=1.8] {angle=Z3p--Z1--Z2};
    \draw[cstaxis] (-.5,0)--(4,0);
    \draw[cstaxis] (0,-.5)--(0,2.5);
    \draw[cstcurve,third] (Z1)--(Z2);
    \draw[cstcurve,main] (Z2)--(Z3)--(Z1);
    \draw[cstdash,second] (Z1)--(Z3p)--(Z2);
  \end{tikzpicture}
\end{center}

\begin{solution}
  由于 $\overrightarrow{Z_1Z_3}$ 为 $\overrightarrow{Z_1Z_2}$ 顺时针或逆时针旋转 $\dfrac\pi3$, 因此
  \begin{align*}
    z_3-z_1&=(z_2-z_1)\exp\left(\pm\frac{\pi i}3\right)
    =(1+i)\left(\half\pm\frac{\sqrt3}2i\right)\\
    &=\frac{1-\sqrt3}2+\frac{1+\sqrt3}2i\ \text{或}\ \frac{1+\sqrt3}2+\frac{1-\sqrt3}2i,\\
    z_3&=\frac{3-\sqrt3}2+\frac{1+\sqrt3}2i\ \text{或}\ \frac{3+\sqrt3}2+\frac{1-\sqrt3}2i.
  \end{align*}
\end{solution}

\begin{example}
  设 $AD$ 是 $\triangle ABC$ 的角平分线, 证明 $\dfrac{AB}{AC}=\dfrac{DB}{DC}$.
\end{example}

\begin{center}
  \begin{tikzpicture}
    \coordinate [label=below left:\textcolor{main}{$A$}] (A) at (0,0);
    \coordinate [label=right:\textcolor{main}{$B=z$}] (B) at ({3*cos(60)},{3*sin(60)});
    \coordinate [label=below:\textcolor{main}{$C=1$}] (C) at (2,0);
    \coordinate [label=right:\textcolor{second}{$D=w$}] (D) at ($0.4*(B)+0.6*(C)$);
    \draw[cstcurve] pic [cstfill2,draw=second] {angle=C--A--D};
    \draw[cstcurve] pic [cstfill3,draw=third,angle radius=4mm] {angle=D--A--B};
    \draw[cstaxis] (-.3,0)--(3,0);
    \draw[cstaxis] (0,-.4)--(0,2);
    \draw[cstcurve,main] (B)--(A)--(C)--cycle;
    \draw[cstcurve,second] (A)--(D);
  \end{tikzpicture}
\end{center}

\begin{proof}
  不妨设 $A=0,B=z,C=1,D=w$, 设
  \[\lambda=\dfrac{DC}{BC}=\dfrac{w-1}{z-1}\in(0,1).\]
  那么
  \[w=1+\lambda(z-1)=\lambda z+(1-\lambda).\]
  由于 $\angle BAD=\angle DAC$, 根据复数乘法的几何意义,
  $\dfrac{z-0}{w-0}$ 是 $\dfrac{w-0}{1-0}$ 的正实数倍, 即
  \[\frac{w^2}z=\lambda^2 z+2\lambda(1-\lambda)+\frac{(1-\lambda)^2}z\in\BR,\]
  于是
    \[\lambda^2z+\dfrac{(1-\lambda)^2}z=\lambda^2\ov z+\dfrac{(1-\lambda)^2}{\ov z},\qquad
    \bigl(\lambda^2|z|^2-(1-\lambda)^2\bigr)(z-\ov z)=0.\]
  显然 $z\neq \ov z$. 又因为 $0<\lambda<1$, 故
    \[\frac{AB}{AC}=|z|=\frac{1-\lambda}{\lambda}
    =\frac{BC-DC}{DC}=\frac{DB}{DC}.\qedhere\]
\end{proof}


\subsection{复数的乘幂}

设 $z=r(\cos\theta+i\sin\theta)=re^{i\theta}\neq0$.
根据复数三角和指数形式的乘法和除法运算法则, 我们有
\begin{theorem}[复数的乘幂]
  \[z^n=r^n(\cos{n\theta}+i\sin{n\theta})=r^ne^{in\theta},\quad\forall n\in\BZ.\]
\end{theorem}
特别地, 当 $r=1$ 时, 我们得到\emph{棣莫弗公式}
\[(\cos\theta+i\sin\theta)^n=\cos{n\theta}+i\sin{n\theta}.\]
对棣莫弗公式左侧进行二项式展开可以得到
\begin{align*}
  \cos(2\theta)&=\hphantom{1}2\cos^2\theta-\hphantom{1}1,\\
  \cos(3\theta)&=\hphantom{1}4\cos^3\theta-\hphantom{1}3\cos\theta,\\
  \cos(4\theta)&=\hphantom{1}8\cos^4\theta-\hphantom{1}8\cos^2\theta+1,\\
  \cos(5\theta)&=16\cos^5\theta-20\cos^3\theta+5\cos\theta.
\end{align*}
一般地, 可以证明 $\cos{n\theta}$ 是 $\cos\theta$ 的 $n$ 次多项式,
这个多项式
\[g_n(T)=2^{n-1}T^n-n2^{n-3}T^{n-2}+\cdots\]
叫做\emph{切比雪夫多项式}.
它在计算数学的逼近理论中有着重要作用.

\begin{example}
  求 $(1+i)^n+(1-i)^n$.
\end{example}

\begin{solution}
  由于
  \[
    1+i=\sqrt2\left(\cos\frac\pi4+i\sin\frac\pi4\right),\quad
    1-i=\sqrt2\left(\cos\frac\pi4-i\sin\frac\pi4\right),
  \]
  因此
    \[
       (1+i)^n+(1-i)^n
      =2^{\frac n2}\left(\cos\frac{n\pi}4+i\sin\frac{n\pi}4 
       +\cos\frac{n\pi}4-i\sin\frac{n\pi}4\right)
      =2^{\frac n2+1}\cos\frac{n\pi}4.
    \]
\end{solution}

\begin{exercise}
  化简 $(\sqrt3+i)^{2022}=$\fillblank[2cm]{}.
\end{exercise}


\subsection{复数的方根}

我们利用复数乘幂公式来计算复数 $z$ 的 \emph{$n$ 次方根 $\sqrt[n]z$}.
设
  \[w^n=z=re^{i\theta}\neq0,\quad w=\rho e^{i\varphi},\]
则
  \[
    w^n=\rho^n(\cos{n\varphi}+i\sin{n\varphi})
       =r(\cos\theta+i\sin\theta).
  \]
比较两边的模可知 $\rho^n=r,\rho=\sqrt[n]r$.
为了避免记号冲突, 当 $r$ 是正实数时, $\sqrt[n]r$ 默认表示 $r$ 的唯一的 $n$ 次正实根, 称之为\emph{算术根}.

由于 $n\varphi$ 和 $\theta$ 的正弦和余弦均相等, 因此存在整数 $k$ 使得
  \[n\varphi=\theta+2k\pi,\quad \varphi=\frac{\theta+2k\pi}n.\]
故 $w=w_k=\sqrt[n]r\exp\Bigl(\dfrac{\theta+2k\pi}ni\Bigr)$.
不难看出, $w_k=w_{k+n}$, 而 $w_0,w_1,\dots,w_{n-1}$ 两两不同.
因此只需取 $k=0,1,\dots,n-1$.
\begin{theorem}[复数的方根]
  任意一个非零复数 $z$ 的 $n$ 次方根有 $n$ 个值:
  \[
    \sqrt[n]z=\sqrt[n]r\exp\Bigl(\dfrac{\theta+2k\pi}ni\Bigr)\\
      =\sqrt[n]r\Bigl(\cos\frac{\theta+2k\pi}n+i\sin\frac{\theta+2k\pi}n\Bigr),\quad k=0,1,\dots,n-1.
  \]
\end{theorem}
这些根的模都相等, 且 $w_k$ 和 $w_{k+1}$ 辐角相差 $\dfrac{2\pi}n$.
因此\alert{它们是以原点为中心, $\sqrt[n]r$ 为半径的圆的内接正 $n$ 边形的顶点}.

\begin{example}
  求 $\sqrt[4]{1+i}$.
\end{example}

\begin{solution}
  由于
    \[1+i=\sqrt2\exp\left(\dfrac{\pi i}4\right),\]
  因此
    \[\sqrt[4]{1+i}=\sqrt[8]2\exp\frac{(\frac\pi4+2k\pi)i}4,\quad k=0,1,2,3.\]
  于是该方根全部值为
    \[w_0=\sqrt[8]2e^{\frac{\pi i}{16}},\quad
    w_1=\sqrt[8]2e^{\frac{9\pi i}{16}},\quad
    w_2=\sqrt[8]2e^{\frac{17\pi i}{16}},\quad
    w_3=\sqrt[8]2e^{\frac{25\pi i}{16}}.\]
\end{solution}
显然 $w_{k+1}=iw_k$,
所以 $w_0,w_1,w_2,w_3$ 形成了一个正方形.

\begin{figure}[hbpt]
  \centering
  \begin{tikzpicture}
    \draw[cstaxis] (-2.3,0)--(2.3,0);
    \draw[cstaxis] (0,-2.3)--(0,2.3);
    \coordinate [label=below left:$0$] (O) at (0,0);
    \draw[cstcurve,thick,third,cstra] (0,0) circle (1.6);
    \coordinate (W0) at ({1.6*cos(11.25)},{1.6*sin(11.25)});
    \coordinate (W1) at ({1.6*cos(101.25)},{1.6*sin(101.25)});
    \coordinate (W2) at ({1.6*cos(191.25)},{1.6*sin(191.25)});
    \coordinate (W3) at ({1.6*cos(281.25)},{1.6*sin(281.25)});
    \draw[cstcurve,thick,second,cstra] (O)--(W0)
      node[right] {$w_0$};
    \draw[cstcurve,thick,second,cstra] (O)--(W1)
      node[above] {$w_1$};
    \draw[cstcurve,thick,second,cstra] (O)--(W2)
      node[left] {$w_2$};
    \draw[cstcurve,thick,second,cstra] (O)--(W3)
      node[below] {$w_3$};
    \draw[cstcurve,main] (W0)--(W1)--(W2)--(W3)--cycle;
  \end{tikzpicture}
  \caption{$\sqrt[4]{1+i}$ 的所有值}
\end{figure}

\begin{exercise}
  计算 $\sqrt[6]{-1}=$\fillblank[5cm][2mm]{}.
\end{exercise}

注意当 $|n|\ge 2$ 时, \alert{$\Arg(z^n)=n\Arg z$ 不成立}.
这是因为
  \begin{align*}
    \Arg(z^n)&=n\arg z+2k\pi,\quad k\in\BZ,\\
    n\Arg z&=n\arg z+2nk\pi,\quad k\in\BZ.
  \end{align*}
不过我们总有
  \[\Arg \sqrt[n]z=\dfrac1n\Arg z=\dfrac{\arg z+2k\pi}n,\quad k\in\BZ,\]
其中左边表示 $z$ 的所有 $n$ 次方根的所有辐角\footnote{此即多值函数复合的含义.}.


\subsubsection*{应用: 实系数三次方程根的情况}
现在我们来看三次方程 $x^3+px+q=0$ 的根, $p\neq 0$.
回顾求根公式:
  \[x=u+v,\quad u^3=-\frac q2+\sqrt{\Delta},\quad uv=-\frac p3,\quad \Delta=\frac{q^2}4+\frac{p^3}{27}.\]
\begin{enumerate}
  \item 如果 $\Delta>0$, 设 $\omega=e^{2\pi i/3}$, 设实数 $\alpha$ 满足
    \[\alpha^3=-\frac q2+\sqrt{\Delta},\]
  则
    \[
      u=\alpha,\alpha\omega,\alpha\omega^2,\qquad
      x=\alpha-\frac p{3\alpha},\ 
        \alpha\omega-\frac p{3\alpha} \omega^2,\ 
        \alpha\omega^2-\frac p{3\alpha} \omega.
    \]
    容易证明后两个根都是虚数.
  \item 如果 $\Delta\le 0$, 则 $p<0$, $|u|^2=-\dfrac p3>0$. 从而 $v=\ov u$.
    设
      \[\sqrt[3]{-\frac q2+\sqrt{\Delta}}=u_1,u_2,u_3,\]
    则我们得到 $3$ 个实根
      \[x=u_1+\ov{u_1},\ u_2+\ov{u_2},\ u_3+\ov{u_3}.\]
    不难验证, 若有重根则 $\Delta=0$.
\end{enumerate}


\section{曲线和区域}

\subsection{复数表平面曲线}

很多的平面图形能用复数形式的方程来表示, 这种表示方程有些时候会显得更加直观和易于理解.

\begin{example}
  \begin{enumerate}
    \item $|z+i|=2$. 该方程表示与 $-i$ 的距离为 $2$ 的点全体, 即圆心为 $-i$ 半径为 $2$ 的圆.
    一般的圆方程为 $|z-z_0|=R$, 其中 $z_0$ 是圆心, $R$ 是半径.
    
\begin{figure}[hbpt]
  \begin{minipage}{.48\textwidth}
    \centering
    \begin{tikzpicture}
      \draw[cstaxis] (-1.5,0)--(1.5,0);
      \draw[cstaxis] (0,-2)--(0,1);
      \coordinate (A) at (0,-.6);
      \fill[cstdot,second] (A) circle
        node[left] {$-i$};
      \draw[cstcurve,main] (A) circle(1.2);
    \end{tikzpicture}
  \end{minipage}
  \begin{minipage}{.48\textwidth}
    \centering
    \begin{tikzpicture}
      \draw[cstaxis] (-1.5,0)--(1.5,0);
      \draw[cstaxis] (0,-1.5)--(0,1.5);
      \coordinate (A) at (-1,0);
      \fill[cstdot,second] (A) circle node[above] {$-2$};
      \coordinate (B) at (0,1);
      \fill[cstdot,second] (B) circle node[left] {$2i$};
      \draw[cstcurve,main] (-1.2,1.2)--(1.2,-1.2);
    \end{tikzpicture}
  \end{minipage}
\end{figure}
    \item $|z-2i|=|z+2|$. 该方程表示与 $2i$ 和 $-2$ 的距离相等的点, 即二者连线的垂直平分线.两边同时平方化简可得 $x+y=0$.
    \item $\Im(i+\ov z)=4$. 设 $z=x+yi$, 则 $\Im(i+\ov z)=1-y=4$, 因此 $y=-3$.
    \item $|z-z_1|+|z-z_2|=2a$.
    \begin{itemize}
      \item 当 $2a>|z_1-z_2|$ 时, 该方程表示以 $z_1,z_2$ 为焦点, $a$ 为长半轴的椭圆;
      \item 当 $2a=|z_1-z_2|$ 时, 该方程表示连接 $z_1,z_2$ 的线段;
      \item 当 $2a<|z_1-z_2|$ 时, 该方程表示空集.
    \end{itemize}
    \item $|z-z_1|-|z-z_2|=2a$.
    \begin{itemize}
      \item 当 $2a<|z_1-z_2|$ 时, 该方程表示以 $z_1,z_2$ 为焦点, $a$ 为实半轴的双曲线的一支;
      \item 当 $2a=|z_1-z_2|$ 时, 该方程表示以 $z_2$ 为起点, 与 $z_2,z_1$ 连线反向的射线;
      \item 当 $2a>|z_1-z_2|$ 时, 该方程表示空集.
    \end{itemize}
  \end{enumerate}
\end{example}

\begin{exercise}
  $z^2+\ov z^2=1$ 和 $z^2-\ov z^2=i$ 分别表示什么图形?
\end{exercise}


\subsection{区域和闭区域}

为了引入极限的概念, 我们需要考虑点的邻域.
类比于高等数学中的邻域和去心邻域, 我们在复变函数中, 称开圆盘
  \[U(z_0,\delta)=\{z:|z-z_0|<\delta\}\]
为 $z_0$ 的一个 \emph{$\delta$ 邻域}, 称去心开圆盘
  \[\Uc(z_0,\delta)=\{z:0<|z-z_0|<\delta\}\]
为 $z_0$ 的一个\emph{去心 $\delta$ 邻域}.

\begin{figure}[hbpt]
  \centering
  \begin{minipage}{.48\textwidth}
    \centering
    \begin{tikzpicture}
      \coordinate (A);
      \filldraw[cstcurve,main,cstfill1] (A) circle (1.5);
      \draw[cstcurve,cstra,second] (A)--(1.2,.9)
        node[midway,above left] {$\delta$};
      \fill[cstdot,second] (A) circle
        node[left] {$z_0$};
    \end{tikzpicture}
  \end{minipage}
  \begin{minipage}{.48\textwidth}
    \centering
    \begin{tikzpicture}
      \coordinate (A);
      \filldraw[cstcurve,main,cstfill1] (A) circle (1.5);
      \draw[cstcurve,cstra,second] (A)--(1.2,.9)
        node[midway,above left] {$\delta$};
      \filldraw[cstdote,draw=main] (A) circle
        node[left,second] {$z_0$};
    \end{tikzpicture}
  \end{minipage}
  \caption{邻域和去心邻域}
\end{figure}

设 $G$ 是复平面的一个子集, $z_0\in\BC$.
它们的位置关系有三种可能:
\begin{enumerate}
  \item 如果存在 $z_0$ 的一个邻域 $U$ 完全包含在 $G$ 中, 则称 $z_0$ 是 $G$ 的一个\emph{内点}.
  \item 如果存在 $z_0$ 的一个邻域 $U$ 完全不包含在 $G$ 中, 则称 $z_0$ 是 $G$ 的一个\emph{外点}.
  \item 如果 $z_0$ 的任何一个邻域 $U$, 都有属于和不属于 $G$ 的点, 则称 $z_0$ 是 $G$ 的一个\emph{边界点}.
\end{enumerate}
显然内点都属于 $G$, 外点都不属于 $G$, 而边界点则都有可能.
这类比于区间的端点和区间的关系.

\begin{figure}
  \centering
  \begin{tikzpicture}
    \filldraw[cstcurve,main,cstfill1,smooth] plot coordinates {(2,0) (1.83,.9) (.64,1.46) (-.63,1.6) (-1.66,1.01) (-2.35,0) (-1.81,-1.06) (-.73,-1.68) (.74,-1.57) (1.82,-.91) (2,0)};
    \coordinate (A) at (-.7,0);
    \draw[cstcurve,second] (A) circle (.5) node[above] {$z_0$};
    \fill[cstdot,second] (A) circle;
    \coordinate (B) at (2,0);
    \draw[cstcurve,third] (B) circle (.5) node[right] {$z_0$};
    \fill[cstdot,third] (B) circle;
    \coordinate (C) at (4,0);
    \draw[cstcurve,fourth] (C) circle (.5) node[above] {$z_0$};
    \fill[cstdot,fourth] (C) circle;
    \draw (.5,0) node[main] {$G$};
  \end{tikzpicture}
  \caption{点与集合的位置关系}
\end{figure}

\begin{definition}[开集和闭集]
  \begin{enumerate}
    \item 如果 $G$ 的所有点都是内点, 也就是说, $G$ 的边界点都不属于它, 称 $G$ 是一个\emph{开集}.
    \item 如果 $G$ 的所有边界点都属于 $G$, 称 $G$ 是一个\emph{闭集}.
  \end{enumerate}
\end{definition}
例如
  \[|z-z_0|<R,\quad 1<\Re z<3,\quad\frac\pi4<\arg z<\dfrac{3\pi}4\]
都是开集\footnote{最后一个集合不包括原点}.
$G$ 是一个闭集当且仅当它的补集是开集.
直观上看: 开集往往由 $>,<$ 的不等式给出, 闭集往往由 $\ge,\le$ 的不等式给出.
不过注意这并不是绝对的.

如果 $D$ 可以被包含在某个开圆盘 $U(0,R)$ 中, 则称它是\emph{有界}的.
否则称它是\emph{无界}的.

\begin{definition}[区域]
  如果开集 $D$ 的任意两个点之间都可以用一条完全包含在 $D$ 中的折线连接起来, 则称 $D$ 是一个\emph{区域}.
  也就是说, 区域是连通的开集.
\end{definition}
区域和它的边界一起构成了\emph{闭区域}, 记作 $\ov D$.
它是一个闭集.

观察下方的图案, 阴影部分(不包含线条部分)中任意两点可用折线连接, 因此它是一个区域.
这些线条和点构成了它的边界.

\begin{figure}[hbpt]
  \centering
  \begin{tikzpicture}
    \filldraw[cstcurve,main,cstfill1,smooth] plot coordinates {(2.81,0) (2.37,1.03) (.91,1.7) (-.8,1.48) (-2.29,1.05) (-2.89,0) (-2.24,-1.03) (-.92,-1.64) (.81,-1.65) (2.38,-.93) (2.81,0)};
    \filldraw[cstcurve,main,fill=white,smooth] plot coordinates {(-.86,-.3) (-1.16,.31) (-1.62,.1) (-1.68,-.69) (-1.17,-.91) (-.86,-.3)};
    \filldraw[cstcurve,main,fill=white] (.5,.3) circle (.3);
    \fill[cstdot,main] (1.5,0) circle;
    \fill[cstdot,main] (1.6,-.5) circle;
    \draw[cstcurve,second,main] plot coordinates {(1,.5) (1.2,.3) (1.2,-.3) (1.4,.5)};
    \coordinate [label=left:\textcolor{second}{$z_1$}] (A) at (-1,.8);
    \coordinate [label=below:\textcolor{second}{$z_2$}] (B) at (1,-.8);
    \draw[cstcurve,second] (A)--(-.2,.5)--(.2,-.5)--(B);
  \end{tikzpicture}
  \caption{区域和它的边界}
\end{figure}

数学中边界的概念与日常所说的边界是两码事. 例如区域 $|z|>1$ 的边界是 $|z|=1$, 其闭区域是 $|z|\ge 1$.

很多区域可以由复数的实部、虚部、模和辐角的不等式所确定.
\begin{exercise}
  下方区域对应的闭区域是什么?
\end{exercise}

\begin{figure}[hbpt]
  \begin{minipage}{.24\textwidth}
    \centering
    \begin{tikzpicture}
      \draw[cstaxis](-1.5,0)--(1.5,0);
      \draw[cstaxis](0,-1.5)--(0,1.5);
      \fill[cstfille1] (-1.2,0) rectangle (1.2,.8);
      \draw (0,-1.5) node[below,align=center] {上半平面\\$\Im z>0$};
    \end{tikzpicture}
  \end{minipage}
  \begin{minipage}{.24\textwidth}
    \centering
    \begin{tikzpicture}
      \draw[cstaxis](-1.5,0)--(1.5,0);
      \draw[cstaxis](0,-1.5)--(0,1.5);
      \fill[cstfille1] (-1.2,0) rectangle (1.2,-.8);
      \draw (0,-1.5) node[below,align=center] {下半平面\\$\Im z<0$};
    \end{tikzpicture}
  \end{minipage}
  \begin{minipage}{.24\textwidth}
    \centering
    \begin{tikzpicture}
      \draw[cstaxis](-1.5,0)--(1.5,0);
      \draw[cstaxis](0,-1.5)--(0,1.5);
      \fill[cstfille1] (-1.2,-1) rectangle (0,1);
      \draw (0,-1.5) node[below,align=center] {左半平面\\$\Re z<0$};
    \end{tikzpicture}
  \end{minipage}
  \begin{minipage}{.24\textwidth}
    \centering
    \begin{tikzpicture}
      \draw[cstaxis](-1.5,0)--(1.5,0);
      \draw[cstaxis](0,-1.5)--(0,1.5);
      \fill[cstfille1] (0,1) rectangle (1.2,-1);
      \draw (0,-1.5) node[below,align=center] {右半平面\\$\Re z>0$};
    \end{tikzpicture}
  \end{minipage}
\end{figure}

\begin{figure}[hbpt]
  \begin{minipage}{.24\textwidth}
    \centering
    \begin{tikzpicture}
      \draw[cstaxis](-1.5,0)--(1.5,0);
      \draw[cstaxis](0,-1.5)--(0,1.5);
      \fill[cstfille1] (-.6,-1) rectangle (.2,1);
      \draw[cstcurve,main] (-.6,-1)--(-.6,1);
      \draw[cstcurve,main] (.2,-1)--(.2,1);
      \draw (0,-1.5) node[below,align=center] {竖直带状区域\\$x_1<\Re z<x_2$};
    \end{tikzpicture}
  \end{minipage}
  \begin{minipage}{.24\textwidth}
    \centering
    \begin{tikzpicture}
      \draw[cstaxis](-1.5,0)--(1.5,0);
      \draw[cstaxis](0,-1.5)--(0,1.5);
      \fill[cstfille1] (-.5,-.4) rectangle (1.1,.4);
      \draw[cstcurve,main] (-.5,-.4)--(1.1,-.4);
      \draw[cstcurve,main] (-.5,.4)--(1.1,.4);
      \draw (0,-1.5) node[below,align=center] {水平带状区域\\$y_1<\Im z<y_2$};
    \end{tikzpicture}
  \end{minipage}
  \begin{minipage}{.24\textwidth}
    \centering
    \begin{tikzpicture}
      \draw[cstaxis](-.5,0)--(2.5,0);
      \draw[cstaxis](0,-.5)--(0,2.5);
      \coordinate (A) at (0,0);
      \coordinate (B) at ({2.2*cos(60)},{2.2*sin(60)});
      \coordinate (C) at ({2.2*cos(10)},{2.2*sin(10)});
      \fill[cstfille1] (A)--(B) arc(60:10:2.2)--cycle;
      \draw[cstcurve,main] (C)--(A)--(B);
      \draw (1,-.5) node[below,align=center] {角状区域\\$\alpha_1<\arg z<\alpha_2$};
    \end{tikzpicture}
  \end{minipage}
  \begin{minipage}{.24\textwidth}
    \centering
    \begin{tikzpicture}
      \filldraw[cstcurve,main,cstfill1] (0,0) circle (1.2);
      \filldraw[cstcurve,main,fill=white] (0,0) circle (.6);
      \draw[cstaxis](-1.5,0)--(1.5,0);
      \draw[cstaxis](0,-1.5)--(0,1.5);
      \draw (0,-1.5) node[below,align=center] {圆环域\\$r<|z|<R$};
    \end{tikzpicture}
  \end{minipage}
\end{figure}


\subsection{区域的特性}

设 $x(t),y(t),t\in[a,b]$ 是两个连续函数,
则参变量方程
  \[\begin{cases}x=x(t),& \\y=y(t),&\end{cases}t\in[a,b]\]
定义了一条\emph{连续曲线}.
这也等价于 $C:z=z(t)=x(t)+iy(t),t\in[a,b]$.
如果除了两个端点有可能重叠外, 其它情形不会出现重叠的点, 则称 $C$ 是\emph{简单曲线}.
如果还满足两个端点重叠, 即 $z(a)=z(b)$, 则称 $C$ 是\emph{简单闭曲线}或\emph{闭路}.

\begin{figure}[hbpt]
  \centering
  \begin{tikzpicture}
    \draw[cstaxis](-.3,0)--(9.5,0);
    \draw[cstaxis](0,-.3)--(0,2.5);
    \coordinate (A) at (.7,.9);
    \coordinate (B) at (3.5,.9);
    \draw[cstcurve,main,smooth] plot coordinates {(A) (1.5,1.6) (2.5,.6) (B)};
    \fill[cstdot,second] (A) circle node[below] {$z(a)$};
    \fill[cstdot,second] (B) circle node[above] {$z(b)$};
    \draw[cstcurve,second,smooth] plot coordinates {(4.8,.7) (5.76,1.25) (6.31,2) (5.77,2.71) (4.77,2.45) (4.78,1.55) (6.2,.7) };
    \draw[cstcurve,main,smooth] plot coordinates {(9.02,1.5) (8.9,1.98) (8.33,2.27) (7.69,2.18) (7.07,1.95) (6.94,1.5) (7.11,1.04) (7.68,.75) (8.34,.82) (8.93,1.08) (9.02,1.5)};
  \end{tikzpicture}
  \caption{简单曲线、非简单曲线、闭路}
\end{figure}

闭路 $C$ 把复平面划分成了两个区域, 一个有界一个无界.
分别称这两个区域是 $C$ 的\emph{内部}和\emph{外部}.
$C$ 是它们的公共边界.
\footnote{B. Bolzano 最早明确陈述了这个定理, 并指出它是需要证明的. 1893 年, C. Jordan 首次给出了证明, 其中假设了该定理对于简单多边形成立 (这个情形并不难证明). 不少数学家认为第一个给出完备证明的是美国数学家 O. Veblen(1905).}

\begin{figure}[hbpt]
  \centering
  \begin{tikzpicture}
    \fill[cstfille1] (-2,-1.5) rectangle (2,1.5);
    \filldraw[cstcurve,main,cstfill1,smooth] plot coordinates {(1.18,0) (.93,.64) (.33,1.06) (-.32,1.08) (-.83,.59) (-1.15,0) (-.83,-.68) (-.36,-1) (.36,-1.08) (.83,-.69) (1.18,0)};
  \end{tikzpicture}
  \caption{闭路的内部和外部}
\end{figure}

在前面所说的几个常见区域的例子中, 我们在区域中画一条闭路.
除了圆环域之外, 闭路的内部仍然包含在这个区域内.

\begin{definition}[单连通区域和多连通区域]
  如果区域 $D$ 中的任一闭路的内部都包含在 $D$ 中, 则称 $D$ 是\emph{单连通区域}.
  否则称之为\emph{多连通区域}.
\end{definition}

\begin{figure}[hbpt]
  \centering
  \begin{tikzpicture}
    \filldraw[cstcurve,main,cstfill1,smooth] plot coordinates {(2.81,0) (2.37,1.03) (.91,1.7) (-.8,1.48) (-2.29,1.05) (-2.89,0) (-2.24,-1.03) (-.92,-1.64) (.81,-1.65) (2.38,-.93) (2.81,0)};
    \filldraw[cstcurve,main,fill=white,smooth] plot coordinates {(-.86,-.3) (-1.16,.31) (-1.62,.1) (-1.68,-.69) (-1.17,-.91) (-.86,-.3)};
    \filldraw[cstcurve,main,fill=white] (.5,.3) circle (.3);
    \fill[cstdot,main] (1.5,0) circle;
    \fill[cstdot,main] (1.6,-.5) circle;
    \draw[cstcurve,main] plot coordinates {(1,.5) (1.2,.3) (1.2,-.3) (1.4,.5)};
    \draw[cstcurve,second,smooth] plot coordinates {(1.94,-.2) (1.79,.41) (1.23,.77) (.58,.81) (.04,.35) (-.3,-.2) (.03,-.81) (.53,-1.19) (1.23,-1.08) (1.76,-.82) (1.94,-.2)};
  \end{tikzpicture}
  \caption{多连通区域}
\end{figure}

单连通区域内的任一闭路可以``连续地变形''成一个点.\footnote{不妨设 $\ell: z=x(t)+iy(t),t\in[0,1]$ 是闭路.
如果存在连续函数 $X,Y:[0,1]\times[0,1]\to \BR$ 使得对 $0\le s<1$
$\ell_s: z=X(s,t)+iY(s,t),t\in[0,1]$
都是闭路, 且 $\ell_0=\ell,\ell_1=a+bi$,
则称闭路 $\ell$ 可以连续地变形为点 $a+bi$.}
这也等价于: 设 $\ell_0,\ell_1$ 是从 $A$ 到 $B$ 的两条连续曲线, 则 $\ell_0$ 可以连续地变形为 $\ell_1$ 且保持端点不动.\footnote{不妨设 $\ell_0: z=x_0(t)+iy_0(t),\ell_1: z=x_1(t)+iy_1(t),t\in[0,1]$.
如果存在连续函数 $X,Y:[0,1]\times[0,1]\to \BR$ 使得
\[\ell_0: z=X(0,t)+iY(0,t),\quad \ell_1: z=X(1,t)+iY(1,t),\]
则称 $\ell_0$ 可以连续地变形为 $\ell_1$ 且保持端点不动.
}

\begin{example}
  \begin{enumerate}
    \item $\Re(z^2)\le1$. 设 $z=x+yi$, 则 $\Re(z^2)=x^2-y^2\le1$. 这是无界的单连通闭区域.
    \item $\arg z\neq \pi$. 即角状区域 $-\pi<\arg z<\pi$. 这是无界的单连通区域.
    \item $\abs{\dfrac1z}\le3$. 即 $|z|\ge\dfrac13$. 这是无界的多连通闭区域.
    \item $|z+1|+|z-1|<4$. 表示一个椭圆的内部. 这是有界的单连通区域.
  \end{enumerate}
\end{example}

\begin{figure}[hbpt]
  \begin{minipage}{.24\textwidth}
    \centering
    \begin{tikzpicture}
      \fill[cstfille1] (-1.414,-1) rectangle (1.414,1);
      \filldraw[cstcurve,main,domain=-45:45,smooth,fill=white] plot ({sec(\x)},{tan(\x)});
      \filldraw[cstcurve,main,domain=-45:45,smooth,fill=white] plot ({-sec(\x)},{tan(\x)});
      \draw[cstaxis] (-1.8,0)--(1.8,0);
      \draw[cstaxis] (0,-1.5)--(0,1.5);
    \end{tikzpicture}
  \end{minipage}
  \begin{minipage}{.24\textwidth}
    \centering
    \begin{tikzpicture}
      \fill[cstfille1] (0,0) circle (1.2);
      \draw[cstaxis] (0,0)--(1.5,0);
      \draw[cstaxis] (0,-1.5)--(0,1.5);
      \draw[cstdash,main] (-1.5,0)--(0,0);
    \end{tikzpicture}
  \end{minipage}
  \begin{minipage}{.24\textwidth}
    \centering
    \begin{tikzpicture}
      \fill[cstfille1] (-1.2,-1.2) rectangle (1.2,1.2);
      \filldraw[cstcurve,main,fill=white] (0,0) circle (.5);
      \draw[cstaxis] (-1.5,0)--(1.5,0);
      \draw[cstaxis] (0,-1.5)--(0,1.5);
    \end{tikzpicture}
  \end{minipage}
  \begin{minipage}{.24\textwidth}
    \centering
    \begin{tikzpicture}
      \filldraw[cstdash,main,cstfill1] (0,0) circle (1 and {0.5*sqrt(3)});
      \draw[cstaxis] (-1.5,0)--(1.5,0);
      \draw[cstaxis] (0,-1.5)--(0,1.5);
    \end{tikzpicture}
  \end{minipage}
\end{figure}

\begin{exercise}
  $|z+1|+|z-1|\ge 1$ 表示什么集合?
\end{exercise}


\section{复变函数}

\subsection{复变函数的定义}

所谓的\emph{映射}, 就是两个集合之间的一种对应 $f:A\to B$, 使得对于每一个 $a\in A$, 有一个唯一确定的 $b=f(a)$ 与之对应.
\begin{itemize}
  \item 当 $A$ 和 $B$ 都是实数集合的子集时, 它就是一个实变函数.
  \item 当 $A$ 和 $B$ 都是复数集合的子集时, 它就是一个\emph{复变函数}.
\end{itemize}

\begin{example}
  $f(z)=\Re z,\arg z,|z|$, $z^n$ ($n$ 为整数), $\dfrac{z+1}{z^2+1}$ 都是复变函数.
\end{example}

\begin{definition}[复变函数的定义域和值域]
  \begin{itemize}
    \item 称 $A$ 为 函数 $f$ 的\emph{定义域}.
    \item 称 $\set{w=f(z)\mid z\in A}$ 为它的\emph{值域}.\footnotemark
  \end{itemize} 
\end{definition}
\footnotetext{值域和\emph{陪域} $B$ 往往不相同. 在高等数学中的函数陪域总可选为 $\BR$, 本课程中复变函数陪域总可选为 $\BC$.
尽管在某些情形下不同陪域的函数视为不同, 但在高等数学和本课程中, 不考虑陪域是否相同, 只要定义域和对应关系相同, 就视为同一函数.}
\begin{exercise}
  上述函数的定义域和值域分别是什么?
\end{exercise}

在复变函数理论中, 常常会遇到\emph{多值的复变函数}, 也就是说一个 $z\in A$ 可能有多个 $w$ 与之对应.
例如 $\Arg z,\sqrt[n]z$ 等.
为了方便研究, 我们常常需要对每一个 $z$, 选取固定的一个 $f(z)$ 的值.
这样便得到了这个多值函数的一个\emph{单值分支}.
\begin{example}
  $\arg z$ 是无穷多值函数 $\Arg z$ 的一个单值分支.
\end{example}

在考虑多值的情况下, 复变函数总有反函数.
如果 $f$ 和 $f^{-1}$ 都是单值的, 则称 $f$ 是\emph{一一对应}.
\begin{example}
  $f(z)=z^n$ 的反函数就是 $f^{-1}(w)=\sqrt[n]{w}$.
  {当 $n=\pm1$ 时, $f$ 是一一对应.}
\end{example}
若无特别声明, 本书中\alert{复变函数总是指单值的复变函数}.


\subsection{映照}

大部分复变函数的图像无法在三维空间中表示出来.
为了直观理解和研究, 我们用两个复平面($z$ 复平面和 $w$ 复平面)之间的\emph{映照}来表示这种对应关系,
其中 
\[w=u+iv=u(x,y)+iv(x,y)\]
的实部和虚部是两个二元实变函数.

\begin{figure}[hbpt]
  \centering
  \begin{tikzpicture}
    \begin{scope}[xshift=-25mm]
      \draw[cstaxis] (-2,0)--(2,0);
      \draw[cstaxis] (0,-1.5)--(0,1.5);
      \draw
        (2,0) node[above] {$x$}
        (0,1.5) node[left] {$y$}
        (0,-1.5) node[below,main] {$z$ 复平面};
      \draw[cstcurve,main,smooth] plot coordinates {(-1.5,0) (-1.7,-.4) (-.3,-.9) (.5,-.7) (.9,0) (1.1,1) (-.3,1.2) (-.7,1) (-1.5,0)};
      \coordinate (a) at (.5,.8);
      \coordinate (b) at (.5,.5);
      \coordinate (c) at (-.3,.3);
    \end{scope}
    \begin{scope}[xshift=25mm]
      \draw[cstaxis] (-2,0)--(2,0);
      \draw[cstaxis] (0,-1.5)--(0,1.5);
      \draw
        (2,0) node[above] {$u$}
        (0,1.5) node[left] {$v$}
        (0,-1.5) node[below,second] {$w$ 复平面};
      \draw[cstcurve,smooth,second] plot coordinates {(-1.3,0) (-.5,-.5) (0,-.8) (.5,-.5) (1,0) (1.3,.9) (.8,1.2) (-.5,.8) (-1.3,0)};
      \coordinate (A) at (.3,.7) circle;
      \coordinate (B) at (-.3,-.3) circle;
    \end{scope}
    \draw[cstdash,smooth,third,cstra] (a) to [bend left=25] (A);
    \draw[cstdash,smooth,third,cstra] (b)to [bend right=15] (B);
    \draw[cstdash,smooth,third,cstra] (c) to [bend right=25] (B);
    \fill[cstdot,main] (a) circle;
    \fill[cstdot,main] (b) circle;
    \fill[cstdot,main] (c) circle;
    \fill[cstdot,second] (A) circle;
    \fill[cstdot,second] (B) circle;
  \end{tikzpicture}
  \caption{映照}
\end{figure}

\begin{example}
  函数 $w=\ov z$.
  如果把 $z$ 复平面和 $w$ 复平面重叠放置, 则这个映照对应的是关于 $z$ 轴的翻转变换.
  它把任一区域映成和它全等的区域, 且 $u=x,v=-y$.
\end{example}

\begin{center}
  \begin{tikzpicture}
    \begin{scope}[xshift=-25mm]
      \draw[cstaxis] (-2,0)--(2,0);
      \draw[cstaxis] (0,-1.5)--(0,1.5);
      \draw
        (2,0) node[above] {$x$}
        (0,1.5) node[left] {$y$}
        (0,-1.5) node[below,main] {$z$ 复平面};
      \draw[cstcurve,main,smooth] plot coordinates {(-1.5,0) (-1.7,-.4) (-.3,-.9) (.5,-.7) (.9,0) (1.1,1) (-.3,1.2) (-.7,1) (-1.5,0)};
      \coordinate (a) at (-1.2,-.3);
      \coordinate (b) at (.6,.9);
      \coordinate (c) at ($.8*(a)+.2*(b)$);
      \coordinate (d) at ($.5*(a)+.5*(b)$);
      \coordinate (e) at ($.2*(a)+.8*(b)$);
      \draw[cstcurve,main] (a)--(b);
    \end{scope}
    \begin{scope}[xshift=25mm]
      \draw[cstaxis] (-2,0)--(2,0);
      \draw[cstaxis] (0,-1.5)--(0,1.5);
      \draw
        (2,0) node[above] {$u$}
        (0,1.5) node[left] {$v$}
        (0,-1.5) node[below,second] {$w$ 复平面};
      \draw[cstcurve,second,smooth] plot coordinates {(-1.5,0) (-1.7,.4) (-.3,.9) (.5,.7) (.9,0) (1.1,-1) (-.3,-1.2) (-.7,-1) (-1.5,0)};
      \coordinate (A) at (-1.2,.3);
      \coordinate (B) at (.6,-.9);
      \coordinate (C) at ($.8*(A)+.2*(B)$);
      \coordinate (D) at ($.5*(A)+.5*(B)$);
      \coordinate (E) at ($.2*(A)+.8*(B)$);
      \draw[cstcurve,second] (A)--(B);
    \end{scope}
    \draw[cstdash,smooth,third,cstra] (c) to[bend right=15] (C);
    \draw[cstdash,smooth,third,cstra] (d) to[bend right=25] (D);
    \draw[cstdash,smooth,third,cstra] (e) to[bend left=45] (E);
    \fill[cstdot,main] (c) circle;
    \fill[cstdot,main] (d) circle;
    \fill[cstdot,main] (e) circle;
    \fill[cstdot,second] (C) circle;
    \fill[cstdot,second] (D) circle;
    \fill[cstdot,second] (E) circle;
  \end{tikzpicture}
\end{center}

\begin{example}
  函数 $w=az$.
  设 $a=re^{i\theta}$, 则这个映照对应的是一个旋转映照(逆时针旋转 $\theta$)和一个相似映照(放大为 $r$ 倍)的复合.
  它把任一区域映成和它相似的区域.
\end{example}

\begin{center}
  \begin{tikzpicture}
    \begin{scope}[xshift=-25mm]
      \draw[cstaxis] (-2,0)--(2,0);
      \draw[cstaxis] (0,-1.5)--(0,1.5);
      \draw
        (2,0) node[above] {$x$}
        (0,1.5) node[left] {$y$}
        (0,-1.5) node[below,main] {$z$ 复平面};
      \draw[cstcurve,main,smooth] plot coordinates {(-1.5,0) (-1.7,-.4) (-.3,-.9) (.5,-.7) (.9,0) (1.1,1) (-.3,1.2) (-.7,1) (-1.5,0)};
      \coordinate (a) at (-1.2,-.3);
      \coordinate (b) at (.6,.9);
      \coordinate (c) at ($.8*(a)+.2*(b)$);
      \coordinate (d) at ($.5*(a)+.5*(b)$);
      \coordinate (e) at ($.2*(a)+.8*(b)$);
      \draw[cstcurve,main] (a)--(b);
    \end{scope}
    \begin{scope}[xshift=25mm]
      \draw[cstaxis] (-2,0)--(2,0);
      \draw[cstaxis] (0,-1.5)--(0,1.5);
      \draw
        (2,0) node[above] {$u$}
        (0,1.5) node[left] {$v$}
        (0,-1.5) node[below,second] {$w$ 复平面};
      \draw[cstcurve,second,smooth,scale=.8,rotate=90] plot coordinates {(-1.5,0) (-1.7,-.4) (-.3,-.9) (.5,-.7) (.9,0) (1.1,1) (-.3,1.2) (-.7,1) (-1.5,0)};
      \coordinate (A) at (.24,-.96);
      \coordinate (B) at (-.72,.48);
      \coordinate (C) at ($.8*(A)+.2*(B)$);
      \coordinate (D) at ($.5*(A)+.5*(B)$);
      \coordinate (E) at ($.2*(A)+.8*(B)$);
      \draw[cstcurve,second] (A)--(B);
    \end{scope}
    \draw[cstdash,smooth,third,cstra] (c) to[bend right=25] (C);
    \draw[cstdash,smooth,third,cstra] (d) to[bend right=10] (D);
    \draw[cstdash,smooth,third,cstra] (e) to[bend left=20] (E);
    \fill[cstdot,main] (c) circle;
    \fill[cstdot,main] (d) circle;
    \fill[cstdot,main] (e) circle;
    \fill[cstdot,second] (C) circle;
    \fill[cstdot,second] (D) circle;
    \fill[cstdot,second] (E) circle;
  \end{tikzpicture}
\end{center}

\begin{example}
  函数 $w=z^2$.
  这个映照把 $z$ 的辐角增大一倍, 因此它会把角形区域变换为角形区域, 并将夹角放大一倍.
  
  由于 $u=x^2-y^2,v=2xy$.
  因此它把 $z$ 复平面上两族分别以直线 $y=\pm x$ 和坐标轴为渐近线的等轴双曲线 $x^2-y^2=c_1,2xy=c_2$分别映射为 $w$ 复平面上的两族平行直线 $u=c_1,v=c_2$.
\end{example}
  
\begin{center}
  \begin{tikzpicture}
    \begin{scope}[xshift=-25mm]
      \draw[cstaxis] (-2,0)--(2,0);
      \draw[cstaxis] (0,-1.5)--(0,1.5);
      \draw
        (2,0) node[above] {$x$}
        (0,1.5) node[left] {$y$}
        (0,-1.5) node[below,main] {$z$ 复平面};
      \fill[cstfille1] (0,0)--({1.5*cos(37.5)},{1.5*sin(37.5)}) arc(37.5:7.5:1.5)--cycle;
      \draw[cstcurve,main] (0,0)--({1.5*cos(37.5)},{1.5*sin(37.5)});
      \draw[cstcurve,main] (0,0)--({1.5*cos(7.5)},{1.5*sin(7.5)});
      \coordinate (a) at (0,1);
      \coordinate (b) at (.8,1.2);
      \coordinate (c) at (-.6,-.3);
    \end{scope}
    \begin{scope}[xshift=25mm]
      \draw[cstaxis] (-2,0)--(2,0);
      \draw[cstaxis] (0,-1.5)--(0,1.5);
      \draw
        (2,0) node[above] {$u$}
        (0,1.5) node[left] {$v$}
        (0,-1.5) node[below,second] {$w$ 复平面};
      \fill[cstfille2] (0,0)--({1.8*cos(75)},{1.8*sin(75)}) arc(75:15:1.8)--cycle;
      \draw[cstcurve,second] (0,0)--({1.8*cos(75)},{1.8*sin(75)});
      \draw[cstcurve,second] (0,0)--({1.8*cos(15)},{1.8*sin(15)});
      \coordinate (A) at (-1,0);
      \coordinate (B) at (-.8,1.92);
      \coordinate (C) at (.27,.36);
    \end{scope}
    \draw[cstdash,smooth,third,cstra] (a) to[bend left=10] (A);
    \draw[cstdash,smooth,third,cstra] (b) to[bend left=20] (B);
    \draw[cstdash,smooth,third,cstra] (c) to[bend right=25] (C);
    \fill[cstdot,fill=main] (a) circle;
    \fill[cstdot,fill=main] (b) circle;
    \fill[cstdot,fill=main] (c) circle;
    \fill[cstdot,fill=second] (A) circle;
    \fill[cstdot,fill=second] (B) circle;
    \fill[cstdot,fill=second] (C) circle;
  \end{tikzpicture}
\end{center}

\begin{center}
  \begin{tikzpicture}
    \begin{scope}[xshift=-25mm]
      \draw[cstaxis] (-2,0)--(2,0);
      \draw[cstaxis] (0,-1.5)--(0,1.5);
      \begin{scope}[cstcurve,main,smooth]
        \draw (-1.2,-1.2)--(1.2,1.2);
        \draw (-1.2,1.2)--(1.2,-1.2);
        \draw[domain=-35:35]
          plot ({sec(\x)},{tan(\x)})
          plot ({-sec(\x)},{tan(\x)})
          plot ({tan(\x)},{sec(\x)})
          plot ({tan(\x)},{-sec(\x)});
        \draw[domain=-46:46]
          plot ({(.8*sec(\x))},{0.8*tan(\x)})
          plot ({(-.8*sec(\x))},{0.8*tan(\x)})
          plot ({0.8*tan(\x)},{0.8*sec(\x)})
          plot ({0.8*tan(\x)},{0.8*-sec(\x)});
        \draw[domain=-57:57]
          plot ({(.6*sec(\x))},{0.6*tan(\x)})
          plot ({(-.6*sec(\x))},{0.6*tan(\x)})
          plot ({0.6*tan(\x)},{0.6*sec(\x)})
          plot ({0.6*tan(\x)},{0.6*-sec(\x)});
        \draw[domain=-68:68]
          plot ({(.4*sec(\x))},{0.4*tan(\x)})
          plot ({(-.4*sec(\x))},{0.4*tan(\x)})
          plot ({0.4*tan(\x)},{0.4*sec(\x)})
          plot ({0.4*tan(\x)},{0.4*-sec(\x)});
      \end{scope}
      \begin{scope}[cstcurve,second,smooth,rotate=45]
        \draw (-1.2,-1.2)--(1.2,1.2);
        \draw (-1.2,1.2)--(1.2,-1.2);
        \draw[domain=-35:35]
          plot ({sec(\x)},{tan(\x)})
          plot ({-sec(\x)},{tan(\x)})
          plot ({tan(\x)},{sec(\x)})
          plot ({tan(\x)},{-sec(\x)});
        \draw[domain=-46:46]
          plot ({(.8*sec(\x))},{0.8*tan(\x)})
          plot ({(-.8*sec(\x))},{0.8*tan(\x)})
          plot ({0.8*tan(\x)},{0.8*sec(\x)})
          plot ({0.8*tan(\x)},{0.8*-sec(\x)});
        \draw[domain=-57:57]
          plot ({(.6*sec(\x))},{0.6*tan(\x)})
          plot ({(-.6*sec(\x))},{0.6*tan(\x)})
          plot ({0.6*tan(\x)},{0.6*sec(\x)})
          plot ({0.6*tan(\x)},{0.6*-sec(\x)});
        \draw[domain=-68:68]
          plot ({(.4*sec(\x))},{0.4*tan(\x)})
          plot ({(-.4*sec(\x))},{0.4*tan(\x)})
          plot ({0.4*tan(\x)},{0.4*sec(\x)})
          plot ({0.4*tan(\x)},{0.4*-sec(\x)});
      \end{scope}
    \end{scope}
    \begin{scope}[xshift=25mm]
      \draw[cstaxis] (-2,0)--(2,0);
      \draw[cstaxis] (0,-1.5)--(0,1.5);
      \begin{scope}[cstcurve,second]
        \draw (-1.3,-1.2)--(1.3,-1.2);
        \draw (-1.3,-0.9)--(1.3,-.9);
        \draw (-1.3,-0.6)--(1.3,-.6);
        \draw (-1.3,-0.3)--(1.3,-.3);
        \draw (-1.3,0)--(1.3,0);
        \draw (-1.3,0.3)--(1.3,.3);
        \draw (-1.3,0.6)--(1.3,.6);
        \draw (-1.3,0.9)--(1.3,.9);
        \draw (-1.3,1.2)--(1.3,1.2);
      \end{scope}
      \begin{scope}[cstcurve,main,rotate=90]
        \draw (-1.3,-1.2)--(1.3,-1.2);
        \draw (-1.3,-0.9)--(1.3,-.9);
        \draw (-1.3,-0.6)--(1.3,-.6);
        \draw (-1.3,-0.3)--(1.3,-.3);
        \draw (-1.3,0)--(1.3,0);
        \draw (-1.3,0.3)--(1.3,.3);
        \draw (-1.3,0.6)--(1.3,.6);
        \draw (-1.3,0.9)--(1.3,.9);
        \draw (-1.3,1.2)--(1.3,1.2);
      \end{scope}
    \end{scope}
  \end{tikzpicture}
\end{center}

\begin{example}
  求下列集合在映照 $w=z^2$ 下的像.
  \begin{enumerate}
    \item 线段 $0<|z|<2,\arg z=\dfrac\pi2$.
    \item 双曲线 $x^2-y^2=4$.
    \item 扇形区域 $0<\arg z<\dfrac\pi4,0<|z|<2$.
  \end{enumerate}
\end{example}
\begin{solution}
  \begin{enumerate}
    \item 设 $z=re^{\frac{\pi i}2}=ir$, 则 $w=z^2=-r^2$.
      因此它的像还是一条线段 $0<|w|<4,\arg w=-\pi$.
    \item 由于
      \[w=u+iv=z^2=(x^2-y^2)+2xyi.\]
      因此 $u=x^2-y^2=4,v=2xy$.
      由于 
      \[f\biggl(\sqrt{\sqrt{4+v^2/4}+2}+i\dfrac{v}{2\sqrt{\sqrt{4+v^2/4}+2}}\biggr)=4+iv,\]
      因此这条双曲线的像的确就是直线 $\Re w=4$.\footnote{在很多教材或习题册中, 往往会忽略检查所给的集合中的每个元素都有原像.}
    \item 设 $z=re^{i\theta}$, 则 $w=r^2e^{2i\theta}$.
      因此它的像是扇形区域 $0<\arg w<\dfrac\pi2,0<|w|<4$.
  \end{enumerate}
\end{solution}

\begin{example}
  求圆周 $|z|=2$ 在映照 $w=\dfrac{z+1}{z-1}$ 下的像.
\end{example}

\begin{solution}
  由于 $z=\dfrac{w+1}{w-1}$, $\abs{\dfrac{w+1}{w-1}}=2$,
  {因此
  \[|w+1|=2|w-1|,\quad w\ov w+w+\ov w+1=4w\ov w-4w-4\ov w+4,\]}
  {
    \[w\ov w-\frac53 w-\frac53\ov w+1=0,\quad \abs{w-\frac53}^2=\dfrac{16}9,\]即 $\abs{w-\dfrac53}=\dfrac43$, 是一个圆周.}
\end{solution}


\section{极限和连续性}

\subsection{无穷远点}

类似于实变函数情形, 我们可以定义复变函数的极限.

\subsubsection*{数列极限}
先来看数列极限的定义.

\begin{definition}[数列极限的定义]
  设 $\{z_n\}_{n\ge 1}$ 是一个复数列.
  如果 $\forall \varepsilon>0,\exists N$ 使得当 $n\ge N$ 时 $|z_n-z|<\varepsilon$, 则称 $z$ 是\emph{数列 $\{z_n\}$ 的极限}, 记作 \emph{$\lim\limits_{n\to\infty}z_n=z$}.
\end{definition}

\begin{figure}[hbpt]
  \centering
  \begin{tikzpicture}
    \fill[cstfille1] (0,0) circle (1.2);
    \filldraw[cstcurve,main,fill=white] (0,0) circle (.5);
    \draw[cstaxis] (-1.5,0)--(1.5,0);
    \draw[cstaxis] (0,-1.5)--(0,1.5);
  \end{tikzpicture}
  \caption{$\infty$ 的(去心)邻域}
\end{figure}

如果 $\forall X>0,\exists N$ 使得当 $n\ge N$ 时 $|z_n|>X$, 则记 \emph{$\lim\limits_{n\to\infty}z_n=\infty$}.
如果称
  \[\Uc(\infty,X)=\{z\in\BC\mid|z|>X\}\]
为 \emph{$\infty$ 的(去心) $X$ 邻域},
那么上述定义可统一表述为:

\begin{definition}[数列极限的等价定义]
  $\lim\limits_{n\to\infty}z_n=z\in\BC\cup\set\infty$ 是指: 对 $z$ 的任意 $\delta$ 邻域 $U$, $\exists N$ 使得当 $n\ge N$ 时 $z_n\in U$.\footnotemark
\end{definition}
\footnotetext{一般地, 一个点的邻域是指包含它的任意一个开集.
可以说明, 把这里的任意 $\delta$ 邻域换成任意邻域, 并不会改变定义, 因为包含 $z$ 的开集一定包含一个 $z$ 的 $\delta$ 领域.}

\subsubsection*{复球面和扩充复平面}
那么有没有一种看法使得 $\infty$ 的邻域和普通复数的邻域没有差异呢?
我们将介绍复球面的概念, 它是复数的一种几何表示且自然包含无穷远点 $\infty$.
这种思想是在黎曼研究多值复变函数时引入的.

\begin{figure}[hbpt]
  \centering
  \begin{tikzpicture}
    \fill[cstfill1] (-3.65,-.804)--(-1.85,.804)--(3.65,.804)--(1.85,-.804)--cycle;
    \filldraw[cstcurve,cstfill] (0,1) circle (1);
    \draw[cstdash] (0,1) circle (1 and 0.3);
    \draw[cstdash,third] (0,0) circle (2 and 0.6);
    \coordinate [label=above:\textcolor{third}{$N$}] (N) at (0,2);
    \draw[cstdash] (0,0)--(N);
    \draw[cstaxis] (0,0)--(2.5,0);
    \draw[cstaxis] (0,0)--(-.8,-.9);
    \coordinate [label=right:\textcolor{main}{$z_1$}] (z1) at (1.65,-.75);
    \coordinate [label=left:\textcolor{main}{$Z_1$}] (Z1) at (.6,1);
    \coordinate [label=left:\textcolor{second}{$z_2$}] (z2) at (-1,0);
    \coordinate [label=below right:\textcolor{second}{$Z_2$}] (Z2) at (-.7,.6);
    \draw[cstcurve,main,cstra] (N)--(z1);
    \fill[cstdot,main] (Z1) circle;
    \draw[cstcurve,cstra,second] (N)--(z2);
    \fill[cstdot,second] (Z2) circle;
    \fill[cstdot,third] (N) circle;
  \end{tikzpicture}
  \caption{复球面和复平面}
\end{figure}

取一个与复平面相切于原点 $z=0$ 的球面.
过 $O$ 做垂直于复平面的直线, 并与球面相交于另一点 $N$, 称之为北极.
\begin{itemize}
  \item 对于平面上的任意一点 $z$, 连接北极 $N$ 和 $z$ 的直线一定与球面相交于除 $N$ 以外的唯一一个点 $Z$.
  \item 反之, 球面上除了北极外的任意一点 $Z$, 直线 $NZ$ 一定与复平面相交于唯一一点.
\end{itemize}
这样, 球面上除北极外的所有点和全体复数建立了一一对应.

当 $|z|$ 越来越大时, 其对应球面上点也越来越接近 $N$.
如果我们在复平面上添加一个额外的"点"——\emph{无穷远点}, 记作 \emph{$\infty$}.
那么\emph{扩充复数集合 $\BC^*=\BC\cup\set\infty$} 就正好和球面上的点一一对应.
称这样的球面为\emph{复球面}, 称包含无穷远点的复平面为\emph{扩充复平面}或\emph{闭复平面}.

它和实数列极限符号中的 $\infty$ 有什么联系呢?
选取上述图形的一个截面来看, 实轴可以和圆周去掉一点建立一一对应.
于是实数列极限符号中的 $\infty$ 在复球面上就是 $\infty$.

\begin{figure}[hbpt]
  \centering
  \begin{tikzpicture}
    \filldraw[cstcurve,cstfill] (0,1) circle (1);
    \coordinate [label=above:\textcolor{third}{$N$}] (N) at (0,2);
    \draw[cstdash] (0,0)--(N);
    \draw[cstaxis] (-2,0)--(2.5,0);
    \coordinate [label=below:\textcolor{main}{$x_1$}] (x1) at (2.2,0);
    \coordinate [label=above right:\textcolor{main}{$X_1$}] (X1) at (1,1.1);
    \coordinate [label=below:\textcolor{second}{$x_2$}] (x2) at (-1,0);
    \coordinate [label=left:\textcolor{second}{$X_2$}] (X2) at (-.8,.4);
    \draw[cstcurve,cstra,main] (0,2)--(x1);
    \fill[cstdot,main] (X1) circle;
    \draw[cstcurve,cstra,second] (0,2)--(x2);
    \fill[cstdot,second] (X2) circle;
    \fill[cstdot,third] (N) circle;
  \end{tikzpicture}
  \caption{圆周和实轴}
\end{figure}

朴素地看, 复球面上任意一点可以定义 $\delta$ 邻域为与其距离小于 $\delta$ 的所有点.
特别地, $\infty$ 的邻域通过前面所说的对应关系, 可以对应到扩充复平面上 $\infty$ 的一个邻域.
所以在复球面上, 我们将普通复数和 $\infty$ 的邻域可以视为相同的概念.


\subsection{数列的极限}

下述定理保证了我们可以使用实数列的敛散性判定技巧.

\begin{theorem}[复数列极限的等价刻画]
  设 $z_n=x_n+y_ni,z=x+yi$, 则
  \[\lim_{n\to\infty}z_n=z\iff \lim_{n\to\infty}x_n=x,\lim_{n\to\infty}y_n=y.\]
\end{theorem}

\begin{proof}
  由三角不等式
  \[|x_n-x|,|y_n-y|\le|z_n-z|\le|x_n-x|+|y_n-y|\]
  易证.
\end{proof}

由此可知极限的四则运算法则对于数列也是成立的.
\begin{theorem}[数列极限的四则运算法则]
  设 $\lim\limits_{n\to\infty}z_n=z,\lim\limits_{n\to\infty}w_n=w$, 则
  \begin{enumerate}
    \item $\lim\limits_{n\to\infty}(z_n\pm w_n)=z\pm w$;
    \item $\lim\limits_{n\to\infty} z_nw_n=zw$;
    \item 当 $w\neq 0$ 时, $\lim\limits_{n\to\infty}\dfrac{z_n}{w_n}=\dfrac zw$.
  \end{enumerate}
\end{theorem}

\begin{example}
  设 $z_n=\left(1+\dfrac1n\right)e^{\frac{\pi i}n}$. 数列 $\{z_n\}$ 是否收敛?
\end{example}

\begin{solution}
  由于
  \[x_n=\left(1+\frac1n\right)\cos\frac\pi n\to 1,\quad
  y_n=\left(1+\frac1n\right)\sin\frac\pi n\to 0.\]
  因此 $\{z_n\}$ 收敛且 $\lim\limits_{n\to\infty}z_n=1$.
\end{solution}

\subsection{函数的极限}

\begin{definition}
  设函数 $f(z)$ 在点 $z_0$ 的某个去心邻域内有定义.
  如果存在复数 $A$, 使得对 $A$ 的任意邻域 $U(A,\varepsilon),\exists\delta>0$ 使得
    \[z\in\Uc(z_0,\delta)\implies f(z)\in U(A,\varepsilon),\]
  则称 $A$ 为 \emph{$f(z)$ 当 $z\to z_0$ 时的极限}, 记为 \emph{$\lim\limits_{z\to z_0}f(z)=A$} 或 \emph{$f(z)\to A (z\to z_0)$}.
\end{definition}
此时我们称\emph{极限存在}.

上述定义中的 $z_0$ 和 $A$ 可换成 $\infty$, 从而得到 $z\to\infty$ 的极限定义, 以及 $\lim f(z)=\infty$ 的含义.

不难看出, 复变函数的极限和二元实函数的极限定义是类似的:
即 $z\to z_0$ 沿任一曲线趋向于 $z_0$ 的极限都是相同的.

\begin{theorem}[函数极限的等价刻画]
  设 $f(z)=u(x,y)+iv(x,y),z_0=x_0+y_0i,A=u_0+v_0i$, 则
  \[\lim_{z\to z_0}f(z)=A\iff
  \lim_{\substack{x\to x_0\\y\to y_0}}u(x,y)=u_0,\quad
  \lim_{\substack{x\to x_0\\y\to y_0}}v(x,y)=v_0.\]
\end{theorem}

\begin{proof}
  由三角不等式
  \[|u-u_0|,|v-v_0|\le|f(z)-A|\le|u-u_0|+|v-v_0|\]
  易证.
\end{proof}

由此可知极限的四则运算法则对于复变函数也是成立的.

\begin{theorem}[函数极限的四则运算法则]
  设 $\lim\limits_{z\to z_0}f(z)=A,\lim\limits_{z\to z_0}g(z)=B$, 则
  \begin{enumerate}
    \item $\lim\limits_{z\to z_0}(f\pm g)(z)=A\pm B$;
    \item $\lim\limits_{z\to z_0}(fg)(z)=AB$;
    \item 当 $B\neq 0$ 时, $\lim\limits_{z\to z_0}\left(\dfrac fg\right)(z)=\dfrac AB$.
  \end{enumerate}
\end{theorem}

在学习了复变函数的导数后, 我们也可以使用等价无穷小替换、洛必达法则等工具来计算极限.

\begin{example}
  证明: 当 $z\to0$ 时, 函数 $f(z)=\dfrac{\Re z}{|z|}$ 的极限不存在.
\end{example}

\begin{proof}
  令 $z=x+yi$, 则 $f(z)=\dfrac x{\sqrt{x^2+y^2}}$.
  因此
    \[u(x,y)=\frac x{\sqrt{x^2+y^2}},\quad v(x,y)=0.\]
  当 $z$ 在实轴原点两侧分别趋向于 $0$ 时, $u(x,y)\to\pm1$.因此 $\lim\limits_{\substack{x\to 0\\y\to0}}u(x,y)$ 不存在,从而 $\lim\limits_{z\to z_0}f(z)$ 不存在.
\end{proof}


\subsection{函数的连续性}

\begin{definition}[连续]
  \begin{itemize}
    \item 如果 $\lim\limits_{z\to z_0}f(z)=f(z_0)$, 则称 $f(z)$ 在 \emph{$z_0$ 处连续}.
    \item 如果 $f(z)$ 在区域 $D$ 内处处连续, 则称 $f(z)$ 在 \emph{$D$ 内连续}.
  \end{itemize}
\end{definition}

根据前面的极限判定定理可知:
\begin{theorem}[连续的等价刻画]
  函数 $f(z)=u(x,y)+iv(x,y)$ 在 $z_0=x_0+iy_0$ 处连续当且仅当 $u(x,y)$ 和 $v(x,y)$ 在 $(x_0,y_0)$ 处连续.
\end{theorem}

\begin{example}
  设 $f(z)=\ln(x^2+y^2)+i(x^2-y^2)$.
  $u(x,y)=\ln(x^2+y^2)$ 除原点外处处连续, $v(x,y)=x^2-y^2$ 处处连续.因此 $f(z)$ 在 $z\neq0$ 处连续.
\end{example}

\begin{theorem}[连续函数的四则运算和复合]
  \begin{itemize}
    \item 在 $z_0$ 处连续的两个函数 $f(z),g(z)$ 之和、差、积、商($g(z_0)\neq 0$) 在 $z_0$ 处仍然连续.
    \item 如果函数 $g(z)$ 在 $z_0$ 处连续, 函数 $f(w)$ 在 $g(z_0)$ 处连续, 则 $f\bigl(g(z)\bigr)$ 在 $z_0$ 处连续.
  \end{itemize}
\end{theorem}

显然 $f(z)=z$ 是处处连续的, 故多项式函数
\[P(z)=a_0+a_1z+a_2z^2+\cdots+a_nz^n\]
也处处连续, 有理函数 $\dfrac{P(z)}{Q(z)}$ 在 $Q(z)$ 的零点以外处处连续.

\begin{example}
  证明: 如果 $f(z)$ 在 $z_0$ 连续, 则 $\ov{f(z)}$ 在 $z_0$ 也连续.
\end{example}

\begin{proof}
  设 $f(z)=u(x,y)+iv(x,y),z_0=x_0+iy_0$.
  那么 $u(x,y),v(x,y)$ 在 $(x_0,y_0)$ 连续.从而 $-v(x,y)$ 也在 $(x_0,y_0)$ 连续.所以 $\ov{f(z)}=u(x,y)-iv(x,y)$ 在 $(x_0,y_0)$ 连续.
\end{proof}
\begin{proof}[另证]
  函数 $g(z)=\ov z=x-iy$ 处处连续,从而 $g\bigl(f(z)\bigr)=\ov{f(z)}$ 在 $z_0$ 处连续.
\end{proof}

可以看出, 在极限和连续性上, 复变函数和两个二元实函数没有什么差别.
那么复变函数和多变量微积分的差异究竟是什么导致的呢?
归根到底就在于 $\BC$ 是一个域, 上面可以做除法.
这就导致了复变函数有\alert{导数}, 而不是像多变量实函数只有偏导数.
这种特性使得可导的复变函数具有整洁优美的性质, 我们将逐步揭开它的神秘面纱.

\begin{homework}
  \item 判断题.
    \begin{exlist}
      \item $z$ 是实数当且仅当 $z=\ov z$. \fillbrace{}
      \item $z$ 是纯虚数当且仅当 $z=-\ov z$. \fillbrace{}
      \item $z$ 是实数当且仅当 $\arg z=0,\pi$. \fillbrace{}
      \item $z$ 是纯虚数当且仅当 $\arg z=\pm\dfrac\pi2$. \fillbrace{}
      \item 若 $f(z)$ 在 $z_0$ 处连续, $g(z)$ 在 $z_0$ 处不连续, 则 $(f+g)(z)$ 在 $z_0$ 一定不连续. \fillbrace{}
    \end{exlist}
  \item 选择题.
    \begin{exlist}
      \item \begin{tasks}
        \task $|z|=\Re z+1$ 是\fillbrace{}.
        \task $|z+i|=|z-i|$ 是\fillbrace{}.
        \task $\bigl||z+i|-|z-i|\bigr|=1$ 是\fillbrace{}.
        \task $|z|+|z-2i|=2$ 是\fillbrace{}.
        \task $\Re(i\ov z)=3$ 是\fillbrace{}.
        \task $z\ov z-(2+i)z-(2-i)\ov z=4$ 是\fillbrace{}.
        \task $z=1+it,-1\le t\le 1$ 是\fillbrace{}.
        \task $z=i+2e^{i\theta},0\le \theta\le 2\pi$ 是\fillbrace{}.
      \end{tasks}
        \begin{taskschoice}(4)
          () 直线
          () 圆周
          () 不是圆的椭圆
          () 双曲线
          () 双曲线的一支
          () 抛物线
          () 一个点
          () 一条线段
        \end{taskschoice}
      \item \begin{tasks}
        \task $z\ov z-(2+i)z-(2-i)\ov z\le 4$ 是\fillbrace{}.
        \task $-1<\arg z<\pi-1$ 的\fillbrace{}.
        \task $1<|z|<2$ 是\fillbrace{}.
        \task $0<\Re z<1$ 是\fillbrace{}.
        \task $\Im z\le0,\Re z\ge0$ 是\fillbrace{}.
        \task $|z-1|<|z+3|$ 是\fillbrace{}.
        \task $\left|\dfrac{z+1}{z-1}\right|<2$ 是\fillbrace{}.
        \task $\arg z<\dfrac{3\pi}4$ 是\fillbrace{}.
      \end{tasks}
        \begin{taskschoice}(2)
          () 有界单连通区域
          () 有界多连通区域
          () 无界单连通区域
          () 无界多连通区域
          () 有界单连通闭区域
          () 有界多连通闭区域
          () 无界单连通闭区域
          () 无界多连通闭区域
        \end{taskschoice}
    \end{exlist}
  \item 填空题.
    \begin{exlist}
      \item 如果 $x,y$ 是实数且 $\dfrac{x+1+i(y-3)}{5+3i}=1+i$, 那么 $x+y=$\fillblank{}.
      \item 设 $z=-i$, 则 $1+z+z^2+z^3+z^4=$\fillblank{}.
      \item 化简 $(-1+i)^{10}-(-1-i)^{10}$=\fillblank{}.
      \item 化简 $i^{2022}-(-i)^{2022}=$\fillblank{}.
      \item 化简 $\dfrac{(1+i)^{101}}{(1-i)^{99}}=$\fillblank{}.
      \item $\biggl(\dfrac{(1+i)^2}2\biggr)^{2021}$ 的模是\fillblank{}.
      \item $2^{-i}$ 的辐角主值是\fillblank{}.
      \item $-1-i$ 的辐角主值是\fillblank{}.
      \item $2023-i$ 绕 $0$ 逆时针旋转 $\dfrac\pi2$ 后得到的复数是\fillblank{}.
      \item 区域 $0<\arg z<\dfrac\pi3$ 在映射 $w=z^3$ 下的像是\fillblank[4cm]{}.
      \item 已知映射 $w=z^3$, 则 $z=\sqrt3+i$ 在 $w$ 复平面上的像是\fillblank{}.
      \item 极限 $\displaystyle\lim_{z\to1+i}(1+z^2+2z^4)=$\fillblank{}.
    \end{exlist}
  \item 计算题.
    \begin{exlist}
      \item $z_1=-z,z_2=\ov z,z_3=-\ov{z}$ 在复平面上对应的点分别与 $z$ 在复平面上对应的点是什么关系?
      \item 已知点 $z_1,z_2,z_3$ 不共线. 点 $\dfrac12(z_1+z_2)$ 和 $\dfrac13(z_1+z_2+z_3)$ 表示什么点?
      \item 求下列复数 $z$ 的实部与虚部, 共轭复数, 模和主辐角:
        \begin{tasks}(4)
          \task $\dfrac{5+i}{2+3i}$;
          \task $\dfrac{3i}{1-i}-\dfrac1i$;
          \task $\dfrac{(3+4i)(2-5i)}{2i}$;
          \task $i^8-4i^{21}+i$.
        \end{tasks}
      \item 求下列复数 $z$ 的三角和指数形式:
        \begin{tasks}(4)
          \task $i$;
          \task $1+i\sqrt3$;
          \task $3-\sqrt 3i$;
          \task $\dfrac{2i}{1-i}$;
          \task $\ov{\biggl(\dfrac{4+3i}{1+2i}\biggr)}$;
          \task $\dfrac{3+i}{i}-\dfrac{10i}{3-i}$;
          \task $\dfrac{(\cos \varphi+i\sin \varphi)^5}{(\cos \varphi-i\sin \varphi)^3}$.
        \end{tasks}
      \item 计算
        \begin{tasks}(3)
          \task $(\sqrt3-i)^5$;
          \task $(1+i)^6$;
          \task $\sqrt[3]{-8}$;
          \task $\sqrt[4]{-2+2i}$;
          \task $\sqrt[4]{-2}$;
          \task $(1-i)^{1/3}$.
        \end{tasks}
      \item 用复参数方程表示连接 $-1+i$ 与 $1-4i$ 的直线段.
      \item 用复参数方程表示以 $z_0$ 为圆心, $R$ 为半径的圆周.
      \item 讨论极限 $\displaystyle\lim_{z\to0}\left(\frac{z}{~\ov z~}-\frac{~\ov z~}z\right)$ 是否存在. 若存在请求出具体的值, 若不存在请证明.
      \item 下列数列 $\set{z_n}$ 是否收敛? 如果收敛, 求出它们的极限:
        \begin{tasks}(3)
          \task $z_n=\dfrac{1+ni}{1-ni}$;
          \task $\displaystyle z_n=\left(1+\frac i2\right)^n$;
          \task $z_n=(-1)^n+\dfrac{i}{n+1}$;
          \task $z_n=\dfrac{(3+2i)^n}{(3+4i)^n}$;
          \task $z_n=\biggl(1+\dfrac{(-1)^n}n\biggr)e^{-\frac{n\pi i}2}$.
        \end{tasks}
    \end{exlist}
  \item 证明题.
    \begin{exlist}
      \item 证明: 当 $|z|=1>|w|$ 时, $\left|\dfrac{z-w}{1-z\ov w}\right|=1$.
      \item 证明: 如果复数 $a+ib$ 是实系数方程
      	\[a_0z^n+a_1z^{n-1}+\cdots+a_{n-1}z+a_n=0\]
      	的根, 则 $a-ib$ 也是它的根.
      \item 设 $\dfrac{z_2-z_1}{z_3-z_1}=\dfrac{z_1-z_3}{z_2-z_3}$. 证明: $|z_1-z_2|=|z_2-z_3|=|z_3-z_1|$ 并说明这些等式的几何意义.
      \item 设 $z=e^{it}$, 证明:\begin{tasks}(2)
          \task $z^n+\dfrac1{z^n}=2\cos{nt}$;
          \task $z^n-\dfrac1{z^n}=2i\sin{nt}$.
        \end{tasks}
    \end{exlist}
  \item 扩展阅读. 该部分作业不需要交, 有兴趣的同学可以做完后交到任课教师邮箱.
    \begin{exlist}
    \item 我们知道, 对于任意两个集合 $A,B$, 我们可以定义 $A\to B$ 的映射.
      在数学中, 很多对象是带有``结构''的集合, 例如实线性空间 $V$ 是一个拥有如下结构:
      \begin{center}
        零元 $0\in V$;\qquad 加法 $v_1+v_2\in V$;\qquad 数乘 $\lambda v$,
      \end{center}
      且满足一些特定性质的集合.
      如果 $A,B$ 具有同一种结构, 映射 $f:A\to B$ ``保持''了这些结构, 则我们称 $f$ 是\emph{同态}.	
      例如实线性空间之间的同态就是指一个映射 $f:V\to W$, 使得
      \[f(0)=0;\qquad f(v_1+v_2)=f(v_1)+f(v_2);\qquad f(\lambda v)=\lambda f(v).\]
    
      再比如域是带有如下结构:
      \begin{center}
        零元 $0$;\qquad 幺元 $1$;\qquad 加法;\qquad 减法;\qquad 乘法;\qquad 除法, 
      \end{center}
      且满足特定性质的集合(交换律分配律之类的). 所以域之间的同态就是指一个 $f:F\to K$, 使得
      \begin{itemize}
        \item $f(0)=0,\qquad f(1)=1$;
        \item $f(x+y)=f(x)+f(y),\qquad f(x-y)=f(x)-f(y)$;
        \item $f(xy)=f(x)f(y),\qquad f(x/y)=f(x)/f(y)$.
      \end{itemize}
      如果一个同态是双射(一一对应), 则称之为\emph{同构}.
      \begin{tasks}
        \task 设 $f:\BQ\to\BQ$ 是有理数域之间的同构, 证明 $f$ 只能是恒等映射.
        \task 设 $f:\BR\to\BR$ 是实数域之间的连续的同构, 证明 $f$ 只能是恒等映射.
        \task 设 $f:\BC\to\BC$ 是复数域之间的连续的同构, 证明 $f$ 只能是恒等映射或复共轭.
        \task 如果 $F=\BR+\BR t$ 是一个真包含 $\BR$ 的域, 证明 $F$ 同构于 $\BC$.
        \task 设
        \[F=\set{\begin{pmatrix}
        x&y\\-y&x\end{pmatrix}: x,y\in\BR}
        =\set{xE+yJ: x,y\in\BR}\subseteq M_2(\BR),\]
        其中 $E=\begin{pmatrix}1& \\&1\end{pmatrix}$,
        $J=\begin{pmatrix}&1\\-1&\end{pmatrix}$.
        证明 $F$ 是一个域且同构于 $\BC$.
      \end{tasks}
    \item 满足 $z^n=1$ 的复数 $z$ 被称为 \emph{$n$ 次单位根}.
      不难看出 $z=e^{\frac{2k\pi i}n},k=0,1,\dots,n-1$.
      单位根在代数, 几何和组合中有着丰富的应用. 我们来看一个例子.
      设集合 $A=\set{1,2,\dots,2023}$.
      \begin{tasks}
        \task 集合 $A$ 有多少个子集? 试着将 $A$ 的每一个子集与
        \[N(x)=\prod_{a=1}^{2023}(1+x^a)\]
        的展开式中的每一项建立一个一一对应.
        \task 设 $S\subseteq A$. 定义
        \[f(S)=\prod_{a\in S}x^a=x^{\sum_{a\in S}a}.\]
        证明所有的 $S$ 对应的 $f(S)$ 之和就是 $N(x)$.
        \task 证明 $N(x)$ 的展开式合并同类项后 $x^k$ 的系数就是 $A$ 的那些满足元素之和是 $k$ 的子集的个数.
        \task 现在我们想知道 $A$ 有多少个子集满足元素之和是 $5$ 的倍数.
      令 $x$ 是 $5$ 次单位根, 则 $N(x)$ 可以表为
      \[N(x)=N_0+N_1x+N_2x^2+N_3x^3+N_4x^4,\]
        那么 $N_0$ 就是元素之和是 $5$ 的倍数的集合个数.
        \task 当 $x=1$ 时, 显然 $N(1)=2^{2023}$.
        当 $x\neq 1$ 是 $5$ 次单位根时, $1,x,x^2,x^3,x^4$ 是方程 $X^5-1=0$ 的所有根, 所以 $2,1+x,1+x^2,1+x^3,1+x^4$ 是方程 $(X-1)^5-1=0$ 的所有根. 由韦达定理可知
        \[(1+x^0)(1+x)(1+x^2)(1+x^3)(1+x^4)=2.\]
        由此证明
        \[N(x)=2^{404}(1+x^0)(1+x)(1+x^2)=2^{405}(1+x+x^2+x^3).\]
        \task 计算 $N(1)+N(e^{2\pi i/5})+N(e^{4\pi i/5})+N(e^{6\pi i/5})+N(e^{8\pi i/5})$. 由此得到 $N_0=\dfrac{2^{2023}+4\cdot 2^{405}}5$.
        \task 想一想, $N_1,N_2,N_3,N_4$ 分别是多少?
      \end{tasks}
      更多细节可见: \url{https://www.bilibili.com/video/BV1R34y1W7Xn/}
  \end{exlist}  
\end{homework}

\exerciseanswer

\ansno{1.1} $-4$.
\ansno{1.2} $1$.
\ansno{1.3} $-\ov z$.

\ansno{2.1} $z_1,\dots,z_n$ 中的非零元辐角相等.
\ansno{2.2} $\displaystyle z=2\sqrt3\left(\cos\frac{-\pi}3+i\sin\frac{-\pi}3\right)=2\sqrt3e^{-\frac{\pi i}3}$, 写成 $\dfrac{5\pi}3$ 也可以.

\ansno{3.1} $-2^{2022}$.
\ansno{3.2} $\pm\dfrac{\sqrt3+i}2,\pm i,\pm\dfrac{\sqrt3-i}2$.

\ansno{4.1} 双曲线 $x^2-y^2=\dfrac12$ 和双曲线 $xy=\dfrac14$.
\ansno{4.2}
\begin{enumerate}
	\item 上半平面对应的闭区域为 $\Im z\ge0$.
	\item 下半平面对应的闭区域为 $\Im z\le0$.
	\item 左半平面对应的闭区域为 $\Re z\le0$.
	\item 右半平面对应的闭区域为 $\Re z\ge0$.
	\item 竖直带状区域对应的闭区域为 $x_1\le\Re z\le x_2$.
	\item 水平带状区域对应的闭区域为 $y_1\le\Im z\le y_2$.
	\item 角状区域对应的闭区域为 $\alpha_1\le \arg z\le \alpha_2$ 以及原点. 如果 $\alpha_1=-\pi,\alpha_2=\pi$, 则为 $\BC$.
	\item 圆环域对应的闭区域为 $r\le|z|\le R$.
\end{enumerate}
\ansno{4.3} 整个复平面.

\ansno{5.1}
\begin{enumerate}
	\item $\Re z$ 的定义域为 $\BC$, 值域为 $\BR$.
	\item $\arg z$ 的定义域为 $\{z\in\BC\mid z\neq 0\}$, 值域为 $(-\pi,\pi]$.
	\item $|z|$ 的定义域为 $\BC$, 值域为 $\{x\in\BR\mid x\ge0\}$.
	\item 当 $n>0$ 时, $z^n$ 的定义域为 $\BC$, 值域为 $\BC$.
	当 $n\le 0$ 时, $z^n$ 的定义域为 $\{z\in\BC\mid z\neq 0\}$, 值域为 $\{z\in\BC\mid z\neq 0\}$.
	\item $\dfrac{z+1}{z^2+1}$ 的定义域为 $\{z\in\BC\mid z\neq \pm i\}$, 值域为 $\BC$.
\end{enumerate}

% 
\chapter{解析函数}
\section{解析函数的概念}

\subsection{可导的函数}

由于 $\BC$ 和 $\BR$ 一样是域, 我们可以像单变量实函数一样去定义复变函数的导数和微分.

\begin{definition}
  设 $w=f(z)$ 在 $z_0$ 的邻域内有定义.
  如果极限
  \[
     \lim_{z\to z_0}\frac{f(z)-f(z_0)}{z-z_0}
    =\lim_{\Delta z\to 0}\frac{f(z_0+\Delta z)-f(z_0)}{\Delta z}
  \]
  存在,则称 \nouns{$f(z)$ 在 $z_0$ 可导}{可导}.
  这个极限值称为 \nouns{$f(z)$ 在 $z_0$ 的导数}{导数},记作
  \[
    f'(z_0)=\lim_{\Delta z\to 0}\frac{f(z_0+\Delta z)-f(z_0)}{\Delta z}.
  \]
  如果 $f(z)$ 在区域 $D$ 内处处可导, 称 \nouns{$f(z)$ 在 $D$ 内可导}{可导}.
\end{definition}

需要注意的是, 无论极限过程 $z\to z_0$ 中 $z$ 是沿何种方式趋于 $z_0$, 比值 $\dfrac{f(z_0+\Delta z)-f(z_0)}{\Delta z}$ 的极限都要存在且全都相等. 因此尽管复变函数导数定义的形式和单变量实函数情形类似, 但其限制实际上要严格得多.

\begin{example}
  函数 $f(z)=x+2yi$ 在哪些点处可导?
\end{example}

\begin{solution}
  由定义可知
  \begin{align*}
    f'(z)&=\lim_{\Delta z\to 0}\frac{f(z+\Delta z)-f(z)}{\Delta z}\\
    &{=\lim_{\Delta z\to 0}\frac{(x+\Delta x)+2(y+\Delta y)i-(x+2yi)}{\Delta z}}\\
    &{=\lim_{\Delta z\to 0}\frac{\Delta x+2\Delta y i}{\Delta x+\Delta yi}.}
  \end{align*}
  当 $\Delta x=0, \Delta y\to 0$ 时, 上式$\to2$;当 $\Delta y=0, \Delta x\to 0$ 时, 上式$\to1$.因此该极限不存在, $f(z)$ 处处不可导.
\end{solution}

\begin{exercise}
  函数 $f(z)=\ov z=x-yi$ 在哪些点处可导? 
\end{exercise}

可以看出, 即使 $f(z)=u+iv$ 的实部和虚部在 $z_0$ 都有偏导数, 甚至都可微, 也无法保证 $f(z)$ 在 $z_0$ 处可导.
我们还需要额外的条件来保证可导性, 具体会在下一节中介绍.

\begin{example}
  求 $f(z)=z^2$ 的导数.
\end{example}

\begin{solution}
  由定义可知
  \begin{align*}
  f'(z)&=\lim_{\Delta z\to 0}\frac{f(z+\Delta z)-f(z)}{\Delta z}
   =\lim_{\Delta z\to 0}\frac{(z+\Delta z)^2-z^2}{\Delta z}\\
  &=\lim_{\Delta z\to 0}(2z+\Delta z)=2z.
  \end{align*}
\end{solution}

事实上, 和单变量实函数情形类似, 复变函数也有如下求导法则.
由此可知, 多项式函数处处可导, 有理函数在其定义域内处处可导, 且其导数形式和单变量实函数情形类似.
\begin{theorem}
  \begin{enumpar}
    \item $(c)'=0$, 其中 $c$ 为复常数;
    \item $(z^n)'=nz^{n-1}$, 其中 $n$ 为整数;
    \item $(f\pm g)'=f'\pm g',\quad (cf)'=cf'$;
    \item $(fg)'=f'g+fg',\quad \left(\dfrac fg\right)'=\dfrac{f'g-fg'}{g^2}$;
    \item $\Bigl(f\bigl(g(z)\bigr)\Bigr)'=f'\bigl(g(z)\bigr)\cdot g'(z)$;
    \item $g'(z)=\dfrac1{f'(w)}, g=f^{-1}, w=g(z)$.
  \end{enumpar}
\end{theorem}

根据上述求导法则, 不难知道:
\begin{theorem}\label{thm:four-derivable}
  \begin{enumpar}
    \item 在 $z_0$ 处可导的两个函数 $f(z)$, $g(z)$ 之和、差、积、商($g(z_0)\neq 0$) 仍然在 $z_0$ 处可导.
    \item 如果函数 $g(z)$ 在 $z_0$ 处可导, 函数 $f(w)$ 在 $g(z_0)$ 处可导, 则 $f\bigl(g(z)\bigr)$ 在 $z_0$ 处可导.
  \end{enumpar}
\end{theorem}

\begin{theorem}
  若 $f(z)$ 在 $z_0$ 可导, 则 $f(z)$ 在 $z_0$ 连续.
\end{theorem}
即可导蕴含连续. 该定理的证明和单变量实函数情形完全相同.
\begin{proof}
  设
    \[\Delta w=f(z_0+\Delta z)-f(z_0),\]
  则
    \[
      \lim_{\Delta z\to 0}\Delta w
      =\lim_{\Delta z\to 0}\frac{\Delta w}{\Delta z}\cdot\Delta z
      =\lim_{\Delta z\to 0}\frac{\Delta w}{\Delta z}\cdot
        \lim_{\Delta z\to 0}\Delta z
      =f'(z_0)\cdot 0=0.
    \]
  从而 $f(z)$ 在 $z_0$ 连续.
\end{proof}


\subsection{可微的函数}


\begin{definition}
  如果存在常数 $A$ 使得函数 $w=f(z)$ 满足
    \[\Delta w=f(z_0+\Delta z)-f(z_0)=A\Delta z+o(\Delta z),\]
  其中 $o(\Delta z)$ 表示 $\Delta z$ 的高阶无穷小量,
  则称 \nouns{$f(z)$ 在 $z_0$ 处可微}{可微}, 称 $A\Delta z$ 为 \nouns{$f(z)$ 在 $z_0$ 的微分}{微分}, 记作 $\diff w=A \Delta z$.
\end{definition}

同导数一样, 复变函数微分的定义也和单变量实函数情形类似, 而且复变函数的可微和可导也是等价的, 且 $\diff w=f'(z_0)\Delta z, \diff z=\Delta z$.
故
  \[\diff w=f'(z_0)\diff z,\qquad f'(z_0)=\dfrac{\diff w}{\diff z}.\]
微分 $\diff w$ 是 $f(z)$ 在 $z_0$ 处的线性近似.

\subsection{解析的函数}

\begin{definition}
  \begin{enumpar}
    \item 若函数 $f(z)$ 在 $z_0$ 的一个邻域内处处可导, 则称 \nouns{$f(z)$ 在 $z_0$ 解析}{解析}.
    \item 若 $f(z)$ 在区域 $D$ 内处处解析, 则称 $f(z)$ 在 $D$ 内解析, 或称 $f(z)$ 是 $D$ 内的一个\noun{解析函数}\footnotemark.
    \item 若 $f(z)$ 在 $z_0$ 不解析, 则称 $z_0$ 为 $f(z)$ 的一个\noun{奇点}.
  \end{enumpar}
\end{definition}
\footnotetext{也可叫\emph{全纯函数}或\emph{正则函数}.}
由定义可知, 若 $f(z)$ 在 $z_0$ 解析, 则 $f(z)$ 在 $z_0$ 可导, 但反过来不成立.
无定义、不连续、不可导、可导但不解析, 都会导致奇点的产生.
不过, 若 $z_0$ 是 $f(z)$ 定义域的外点, 即存在 $z_0$ 的邻域与 $f(z)$ 定义域交集为空集, 这种情形不甚有趣, 因此我们不考虑这类奇点.

由于区域 $D$ 是一个开集, 其中的任意 $z_0\in D$ 均存在一个包含在 $D$ 的邻域. 所以 \emph{$f(z)$ 在 $D$ 内解析和在 $D$ 内可导是等价的}.
由于一个点的邻域也是一个开集, 因此若 $f(z)$ 在 $z_0$ 解析, 则 $f(z)$ 在 $z_0$ 的一个邻域内处处可导, 从而在该邻域内解析. 因此 \emph{$f(z)$ 解析点全体是一个开集}, 它是可导点集合的内点构成的集合.

\begin{exercise}
  函数 $f(z)$ 在点 $z_0$ 的邻域内解析是 $f(z)$ 在该邻域处处可导的\fillbrace{}.
  \begin{taskschoice}(2)
    \task 充分条件
    \task 必要条件
    \task 充要条件
    \task 既非充分也非必要条件
  \end{taskschoice}
\end{exercise}

\begin{example}
  研究函数 $f(z)=|z|^2$ 的解析性.
\end{example}
\begin{solution}
  注意到
  \[
    \frac{f(z+\Delta z)-f(z)}{\Delta z}
    =\frac{(z+\Delta z)(\ov z+\ov{\Delta z})-z\ov z}{\Delta z}
    =\ov z+\ov{\Delta z}+z\frac{\Delta x-\Delta yi}{\Delta x+\Delta yi}.
  \]
  \begin{itemize}
    \item 若 $z=0$, 则当 $\Delta z\to 0$ 时该极限为 $0$.
    \item 若 $z\neq0$, 则当 $\Delta y=0,\Delta x\to 0$ 时该极限为 $\ov z+z$; 当 $\Delta x=0,\Delta y\to 0$ 时该极限为 $\ov z-z$.因此此时极限不存在.
  \end{itemize}
  故 $f(z)$ 仅在 $z=0$ 处可导, 从而处处不解析.
\end{solution}

由定理~\ref{thm:four-derivable} 不难证明:
\begin{theorem}
  \begin{enumpar}
    \item 在 $z_0$ 处解析的两个函数 $f(z)$, $g(z)$ 之和、差、积、商($g(z_0)\neq 0$) 仍然在 $z_0$ 处解析.
    \item 在 $D$ 内解析的两个函数 $f(z)$, $g(z)$ 之和、差、积、商仍然在 $D$ (作商时需要去掉 $g(z)$ 的零点) 内解析.
    \item 如果函数 $g(z)$ 在 $z_0$ 处解析, 函数 $f(w)$ 在 $g(z_0)$ 处解析, 则 $f\bigl(g(z)\bigr)$ 在 $z_0$ 处解析.
    \item 如果函数 $g(z)$ 在 $D$ 内解析, 函数 $f(w)$ 在 $g(z_0)$ 处解析, 则 $f\bigl(g(z)\bigr)$ 在 $D$ 内解析.
  \end{enumpar}
\end{theorem}

由此可知, 多项式函数处处解析. 有理函数在其定义域内处处解析, 分母的零点是它的奇点.
我们来看复变函数在实变量函数求导中的一个应用.
\begin{example}
  设 $f(z)=\dfrac1{1+z^2}$, 则它在除 $z=\pm i$ 外处处解析.
  当 $z=x$ 为实数时,
  \begin{align*}
    \biggl(\frac1{1+x^2}\biggr)^{(n)}&
      =f^{(n)}(x)=\frac i2\biggl(\frac1{x+i}-\frac1{x-i}\biggr)^{(n)}\\
    &=\frac i2\cdot(-1)^n n!\biggl(\frac1{(x+i)^{n+1}}-\frac1{(x-i)^{n+1}}\biggr)\\
    &=(-1)^{n+1}n!\Im\frac1{(x+i)^{-n-1}}
  \end{align*}
  由于
  \[
    |x+i|=\sqrt{x^2+1},\qquad
    \arg(x+i)=\arccot x,
  \]
  因此
  \[
    \biggl(\frac1{1+x^2}\biggr)^{(n)}
    =(-1)^nn!(x^2+1)^{-\frac{n+1}2}\sin\bigl((n+1)\arccot x\bigr).
  \]
\end{example}
如果有理函数 $\dfrac{P(z)}{Q(z)}$ 分母的零点均能求出, 则可将其拆分为一个多项式和一些形如 $\dfrac{a}{(x-b)^k}$ 的分式之和, 从而其高阶导数可按此方法计算.

\section{函数解析的充要条件}

\subsection{柯西-黎曼定理}

在上一节中, 通过对一些简单函数的分析, 我们发现可导的函数往往可以直接表达为 $z$ 的函数的形式, 而不解析的往往包含 $x,y,\ov z$ 等内容.
这种现象并不是偶然的.
我们来研究二元实变量函数的可微性与复变函数可导的关系.
为了简便我们用
\[
  u_x=\dpp ux,\quad
  u_y=\dpp uy,\quad
  v_x=\dpp vx,\quad
  v_y=\dpp vy
\]
等记号表示偏导数.

设 $f$ 在 $z=x+yi$ 处可导, $f'(z)=a+bi$.
对于充分小 $|\Delta z|>0$, 令
\[
   \Delta f
  =f(z+\Delta z)-f(z)
  =\Delta u+i\Delta v.
\]
由于 $f$ 在 $z$ 处可微, 因此
\[
   \Delta f
  =\Delta u+i\Delta v
  =(a+bi)(\Delta x+i\Delta y)+o(\Delta z).
\]
设 $\rho=|\Delta z|=\sqrt{\Delta x^2+\Delta y^2}$.
由于 $\Delta z$ 的高阶无穷小量 $o(\Delta z)=o(\rho)$ 的实部和虚部也是 $\rho$ 的高阶无穷小量, 展开可知
\begin{align*}
  \Delta u&=a\Delta x-b\Delta y+o(\rho),\\
  \Delta v&=b\Delta x+a\Delta y+o(\rho),
\end{align*}
因此 $u,v$ 在 $(x_0,y_0)$ 处可微且 $u_x=v_y=a,v_x=-u_y=b$.

反过来, 假设 $u,v$ 在 $(x_0,y_0)$ 处可微且 $u_x=v_y, v_x=-u_y$. 由全微分公式
\begin{align*}
  \Delta u&=u_x\Delta x+u_y\Delta y+o(\rho)
    =u_x\Delta x-v_x\Delta y+o(\rho),\\
  \Delta v&=v_x\Delta x+v_y\Delta y+o(\rho)
    =v_x\Delta x+u_x\Delta y+o(\rho),\\
  \Delta f&=\Delta u+i\Delta v
    =(u_x+i v_x)\Delta x+(-v_x+i u_x)\Delta y+o(\rho)\\
   &=(u_x+i v_x)\Delta(x+iy)+o(\rho)
    =(u_x+i v_x)\Delta z+o(\rho).
\end{align*}
故 $f(z)$ 在 $z$ 处可微, 从而可导, 且 $f'(z)=u_x+i v_x=v_y-i u_y$.

由此得到:
\begin{theorem}[柯西-黎曼定理]
  $f(z)$ 在 $z=x+yi$ 处可导当且仅当 $u,v$ 在 $(x,y)$ 处可微, 且满足\noun{柯西-黎曼方程}:
    \[u_x=v_y,\quad v_x=-u_y.\]
  此时
    \[f'(z)=u_x+iv_x=v_y-iu_y.\]
\end{theorem}

\begin{figure}[!h]
  \centering
  \includegraphics[height=40mm]{../image/Cauchy.jpeg}
  \hspace{5em}
  \includegraphics[height=40mm]{../image/Riemann.jpeg}
  \caption{柯西和黎曼}
\end{figure}

柯西\footnote{%
  Augustin-Louis Cauchy (1789--1857), 法国数学家、物理学家.
}-黎曼方程可简称为 C-R 方程.

当 $f$ 可导时其导数形式也可直接看出.
当极限
\[
  \lim\limits_{\Delta z\to 0}\dfrac{\Delta u+i\Delta v}{\Delta z}=f'(z)
\]
存在时, 它沿着水平方向和竖直方向的极限
\[
  \lim\limits_{\Delta x\to 0}\dfrac{\Delta u+i\Delta v}{\Delta x}=u_x+iv_x,\qquad
  \lim\limits_{\Delta y\to 0}\dfrac{\Delta u+i\Delta v}{\Delta y i}=-iu_y+v_y
\]
也都存在, 且三者极限相同.
因此 $f'(z)=u_x+iv_x=-iu_y+v_y$.

下面我们来介绍柯西-黎曼方程的等价形式.
注意到
\[
  x=\dfrac12z+\dfrac12\ov z,\qquad
  y=-\dfrac i2z+\dfrac i2\ov z.
\]
仿照着二元实函数偏导数在变量替换下的变换规则, 定义 $f$ 对 $z$ 和 $\ov z$ 的偏导数为
  \[\begin{aligned}
      \dpp fz&
    =\dpp xz\dpp fx+\dpp yz\dpp fy
    =\frac12\dpp fx-\frac i2\dpp fy,\\
      \dpp f{\ov z}&
    =\dpp x{\ov z}\dpp fx+\dpp y{\ov z}\dpp fy
    =\frac12\dpp fx+\frac i2\dpp fy.
  \end{aligned}\]
和前面的计算类似可知: 当 $f$ 在 $z$ 处可导时,
\[
  \Delta f=f'(z)\Delta z+o(\rho).
\]
当 $u,v$ 可微时,
\[
  \Delta f=\dpp fz\Delta z+\dpp f{\ov z}\Delta\ov z+o(\rho).
\]
由于极限 $\lim\limits_{\Delta\to 0}\dfrac{\Delta \ov z}{\Delta z}$ 不存在, 因此:
\begin{theorem}[柯西-黎曼定理的等价形式]
  $f(z)$ 在 $z=x+yi$ 处可导当且仅当 $u,v$ 在 $(x,y)$ 处可微, 且满足\noun{柯西-黎曼方程}:
    \[\dpp f{\ov z}=0.\]
  此时
    \[f'(z)=\dpp fz.\]
\end{theorem}
从该定理便可解释, 为何含有 $x,y,\ov z$ 形式的函数往往不可导, 而可导的函数往往可以直接表达为 $z$ 的形式.

由于二元函数的偏导数均连续蕴含可微, 因此我们有:
\begin{theorem}
  \begin{enumpar}
    \item 如果 $u_x,u_y,v_x,v_y$ 在 $(x,y)$ 处连续, 且满足C-R方程, 则 $f(z)$ 在 $z=x+yi$ 处可导.
    \item 如果 $u_x,u_y,v_x,v_y$ 在区域 $D$ 上处处连续, 且满足C-R方程, 则 $f(z)$ 在 $D$ 上处处可导, 从而解析.
  \end{enumpar}
\end{theorem}
尽管这些条件不是充要条件, 但在实际应用中, 确实很多情形下 $u_x,u_y,v_x,v_y$ 是处处连续的.


\subsection{柯西-黎曼定理的应用}

在下面几个例子中, 我们利用柯西黎曼定理来研究函数的可导性和解析性.
\begin{example}
  求下列函数的可导点和解析区域.
  \begin{tasksexam}(2)
    \task {$f(z)=\ov z$;}
    \task {$f(z)=\ov z(2z+\ov z)$;}
    \task {$f(z)=e^{|z|^2}$;}
    \task {$f(z)=e^x(\cos y+i\sin y)$.}\label{enum:exp}
  \end{tasksexam}
\end{example}

\begin{solution}\delspace
  \begin{enumnopar}[(i)]
    \item 由 $u=x,v=-y$ 可知
      \begin{alignat*}{2}
        u_x&=1,\qquad&u_y&=0,\\
        v_x&=0,\qquad&v_y&=-1.
      \end{alignat*}
      这些偏导数都是连续的.
      因为 $u_x=1\neq v_y=-1$, 所以该函数处处不可导, 处处不解析.

      也可从 $\dpp f{\ov z}=1\neq0$ 看出.
    \item 由 $f(z)=3x^2+y^2-2xyi,u=3x^2+y^2,v=-2xy$ 可知
      \begin{alignat*}{2}
        u_x&=6x,\qquad&u_y&=2y,\\
        v_x&=-2y, \qquad&v_y&=-2x.
      \end{alignat*}
      这些偏导数都是连续的.
      由 $u_x=v_y,v_x=-u_y$ 可知只有 $x=\Re z=0$ 满足C-R方程.
      因此该函数在虚轴上可导, 处处不解析且 $f'(yi)=(u_x+iv_x)|_{(0,0)}=-2yi$.

      也可从 $\dpp f{\ov z}=2z+2\ov z=4x$ 看出, 且有 $f'(yi)=\dpp fz\Big|_{z=yi}=2\ov z|_{z=yi}=-2yi$.
    \item 由 $f(z)=e^{x^2+y^2},u=e^{x^2+y^2},v=0$ 可知
      \begin{alignat*}{2}
        u_x&=2xe^{x^2+y^2},\qquad&u_y&=2ye^{x^2+y^2},\\
        v_x&=0, \qquad&v_y&=0.
      \end{alignat*}
      这些偏导数都是连续的.
      由 $u_x=v_y,v_x=-u_y$ 可知只有 $x=y=0,z=0$ 满足C-R方程.
      因此该函数只在 $0$ 可导, 处处不解析且 $f'(0)=(u_x+iv_x)|_{(0,0)}=0$.

      也可从 $f(z)=e^{z\ov z}, \dpp f{\ov z}=ze^{z\ov z}$ 看出, 且有 $f'(0)=\dpp fz\Big|_{z=0}=\ov ze^{z\ov z}|_{z=0}=0$.
    \item 由 $u=e^x\cos y,v=e^x\sin y$ 可知
      \begin{alignat*}{2}
        u_x&=e^x\cos y,\qquad&u_y&=-e^x\sin y,\\
        v_x&=e^x\sin y,\qquad&v_y&=e^x\cos y.
      \end{alignat*}
      这些偏导数都是连续的, 且处处满足C-R方程.
      因此该函数处处可导, 处处解析, 且
      \[f'(z)=u_x+iv_x=e^x(\cos y+i\sin y)=f(z).\]
  \end{enumnopar}
\end{solution}

我们发现, \ref{enum:exp} 中的函数满足导数等于自身, 后面我们会看到它就是复变量的指数函数 $e^z$.

\begin{exercise}
  函数\fillbrace{}在 $z=0$ 处不可导.
  \begin{taskschoice}(4)
    \task $2x+3yi$
    \task $2x^2+3y^2i$
    \task $x^2-xyi$
    \task $e^x\cos y+i e^x\sin y$
  \end{taskschoice}
\end{exercise}

\begin{example}
  设函数 $f(z)=(x^2+axy+by^2)+i(cx^2+dxy+y^2)$ 在复平面内处处解析. 求实常数 $a,b,c,d$ 以及 $f'(z)$.
\end{example}
\begin{solution}
  注意到
  \begin{alignat*}{2}
    u_x&=2x+ay,\qquad&u_y&=ax+2by,\\
    v_x&=2cx+dy,\qquad&v_y&=dx+2y.
  \end{alignat*}
  由C-R方程可知
    \[2x+ay=dx+2y,\quad ax+2by=-(2cx+dy),\]
  因此 $a=d=2$, $b=c=-1$, 且
    \[f'(z)=u_x+iv_x=2x+2y+i(-2x+2y)=(2-2i)z.\]
\end{solution}

\begin{example}\label{exam:zero-deriv-constant}
  证明: 如果 $f'(z)$ 在区域 $D$ 内处处为零, 则 $f(z)$ 在 $D$ 内是一常数.
\end{example}

\begin{proof}
  由于
    \[f'(z)=u_x+iv_x=v_y-iu_y=0,\]
  因此 $u_x=v_x=u_y=v_y=0$, $u,v$ 均为常数,从而  $f(z)=u+iv$ 是常数.
\end{proof}
类似地可以证明, 若 $f(z)$ 在 $D$ 内解析, 则下述任一条件均可推出 $f(z)$ 是一常数:
\begin{tasksexam}(2)
  \task {$\arg{f(z)}$ 是一常数;}
  \task {$|f(z)|$ 是一常数;}
  \task {$\Re{f(z)}$ 是一常数;}
  \task {$\Im{f(z)}$ 是一常数;}
  \task {$v=u^2$;}
  \task {$u=v^2$.}
\end{tasksexam}

\begin{example}
  证明: 如果 $f(z)$ 解析且 $f'(z)$ 处处非零, 则曲线族 $u(x,y)=c_1$ 和曲线族 $v(x,y)=c_2$ 互相正交.
\end{example}

\begin{proof}
  由于 $f'(z)=u_x-iu_y$, 因此 $u_x,u_y$ 不全为零.
  对 $u(x,y)=c_1$ 使用隐函数求导法则得 $u_x\diff x+u_y\diff y=0$,从而 $(u_y,-u_x)$ 是该曲线在 $z$ 处的非零切向量.

  同理 $(v_y,-v_x)$ 是 $v(x,y)=c_2$ 在 $z$ 处的非零切向量.由于
    \[u_yv_y+u_xv_x=u_yu_x-u_xu_y=0,\]
  因此这两个切向量正交, 从而曲线正交.
\end{proof}

当 $f'(z_0)\neq 0$ 时, 经过 $z_0$ 的两条曲线 $C_1,C_2$ 的夹角和它们的像 $f(C_1),f(C_2)$ 在 $f(z_0)$ 处的夹角总是相同的.
这种性质被称为\noun{保角性}.
这是因为 $\diff f=f'(z_0)\diff z$.
从复数乘法的几何意义可知, 局部上 $f$ 把 $z_0$ 附近的点以 $z_0$ 为中心放缩 $|f'(z_0)|$ 倍并逆时针旋转 $\arg{f'(z_0)}$.
由 $w$ 复平面上曲线族 $u=c_1,v=c_2$ 正交可知上述例题成立, 特别地, 例~\ref{exam:wz2} 中的曲线族 $x^2-y^2=c_1$, $2xy=c_2$ 正交.


\section{初等函数}

我们将实变函数中的初等函数推广到复变函数.
多项式函数和有理函数的解析性质已经介绍过, 这里不再重复.

\subsection{指数函数}

复指数函数有多种等价的定义方式:
\begin{enumerate}
  \item $\exp z=e^x(\cos y+i\sin y)$ (欧拉恒等式);\label{enum:exp-euler}
  \item $\exp z=\lim\limits_{n\to\infty}\left(1+\dfrac zn\right)^n$ (极限定义);\label{enum:exp-limit}
  \item $\exp z=1+z+\dfrac{z^2}{2!}+\dfrac{z^3}{3!}+\cdots
  =\lim\limits_{n\to\infty}\sum\limits_{k=0}^n\dfrac{z^k}{k!}$ (级数定义);\label{enum:exp-series}
  \item $\exp z$ 是唯一的一个处处解析的函数, 使得当 $z=x\in\BR$ 时, $\exp z=e^x$ (解析延拓).\label{enum:exp-expansion}
\end{enumerate}
这几种定义方式都是等价的.

若我们采用\ref{enum:exp-series} 来定义, 则从 $\cos x$ 和 $\sin x$ 的泰勒展开
\[
  \cos x=1-\frac{x^2}{2!}+\frac{x^4}{4!}+\cdots\quad
  \sin x=x-\frac{x^3}{3!}+\frac{x^5}{5!}\cdots
\]
可以得到欧拉恒等式 $e^{ix}=\cos x+i\sin x$.
然而, 这实际上需要我们先铺垫好复级数的基础理论, 这会在第四章例\ref{exam:exp-taylor-expansion}中得到解释.
而\ref{enum:exp-euler} 和\ref{enum:exp-expansion} 的等价性我们将在第五章由定理\ref{thm:zero-isolated}推出.

现在我们来证明\ref{enum:exp-euler} 和\ref{enum:exp-limit} 是等价的\footnote{%
  欧拉也是从实指数函数的极限定义
  \[e^x=\lim\limits_{n\to\infty}(1+\dfrac xn)^n\]
  得到复指数函数的极限定义, 并证明了欧拉恒等式.
  参考 \cite[第19章2,3节]{Kline1990}.
}.
\begin{align*}
  \lim_{n\to\infty}\abs{1+\frac zn}^n
  &=\lim_{n\to\infty}\left(1+\frac{2x}n+\frac{x^2+y^2}{n^2}\right)^{\frac n2}\quad
  {(1^\infty\ \text{型不定式})}\\
  &{=\exp\left[\lim_{n\to\infty}\frac n2
  \left(\frac{2x}n+\frac{x^2+y^2}{n^2}\right)\right]=e^x.}
\end{align*}
不妨设 $n>\abs{z}$, 这样 $1+\dfrac zn$ 落在右半平面,
  \[
    \lim_{n\to\infty} n\arg{\left(1+\frac zn\right)}
    =\lim_{n\to\infty} n\arctan \frac y{n+x}
    =\lim_{n\to\infty}\frac{ny}{n+x}=y.
  \]
故
  \[\lim_{n\to\infty}\left(1+\dfrac zn\right)^n=e^x(\cos y+i\sin y).\]

\begin{definition}{指数函数}
定义\noun{指数函数}
  \[\exp z:=e^x(\cos y+i\sin y).\]
\end{definition}
为了方便, 我们也记 \alert{$e^z=\exp z$}\index{$e^z$}\index{$\exp z$}.
指数函数有如下性质:
\begin{itemize}
  \item $\exp z$ 处处解析, 且 $(\exp z)'=\exp z$.
  \item $\exp z\neq 0$.
  \item $\exp(z_1+z_2)=\exp z_1\cdot \exp z_2$.
  \item $\exp(z+2k\pi i)=\exp z$, 即 $\exp z$ 周期为 $2\pi i$.
  \item $\exp z_1=\exp z_2$ 当且仅当 $z_1=z_2+2k\pi i,k\in\BZ$.
  \item $\exp z$ 将直线族 $\Re z=c$ 映为圆周族 $\abs{w}=e^c$, 将直线族 $\Im z=c$ 映为射线族 $\Arg w=c$.
\end{itemize}

\begin{example}
  计算函数 $f(z)=\exp(z/6)$ 的周期.
\end{example}
\begin{solution}
  设 $f(z_1)=f(z_2)$, 则 $\exp(z_1/6)=\exp(z_2/6)$.
  {因此存在 $k\in\BZ$ 使得
    \[\frac{z_1}6=\frac{z_2}6+2k\pi i,\]从而 $z_1-z_2=12k\pi i$.所以 $f(z)$ 的周期是 $12\pi i$.}
\end{solution}

一般地, $\exp(az+b)$ 的周期是 $\dfrac{2\pi i}a$ (或写成 $-\dfrac{2\pi i}a$), $a\neq 0$.


\subsection{对数函数}

对数函数 $\Ln z$ 定义为指数函数 $\exp z$ 的反函数.
为什么我们用大写的 $\Ln$ 呢? 
在复变函数中, 很多函数是多值函数.
为了便于研究, 我们会固定它的一个单值分支.
我们将多值的这个开头字母大写, 而对应的单值的则是开头字母小写.
例如 $\Arg z$ 和 $\arg z$.

设 $z\neq 0$, $e^w=z=re^{i\theta}=e^{\ln r+i\theta}$,
则
\[w=\ln r+i\theta+2k\pi i,\quad k\in\BZ.\]
\begin{definition}[对数函数]
  \begin{enumerate}
    \item 定义\noun{对数函数}\index{$\Ln z$}
      \[\Ln z=\ln\abs{z}+i\Arg z.\]
      它是一个多值函数.
    \item 定义\nouns{对数函数主值}{对数函数!对数函数主值}\index{$\ln z$}
      \[\ln z=\ln\abs{z}+i\arg z.\]
  \end{enumerate}
\end{definition}
对于每一个整数 $k$, $\ln z+2k\pi i$ 都给出了 $\Ln z$ 的一个单值分支.
特别地, 当 $z=x>0$ 是正实数时, $\ln z$ 就是实变的对数函数.

\begin{example}
  求 $\Ln 2,\Ln(-1)$ 以及它们的主值.
\end{example}

\begin{solution}
  \[\Ln2=\ln2+2k\pi i,\quad k\in\BZ,\]
  主值为 $\ln 2$.
  \[\Ln(-1)=\ln1+i\Arg(-1)=(2k+1)\pi i,\quad k\in\BZ,\]
  主值为 $\pi i$.
\end{solution}

\begin{example}
求 $\Ln(-2+3i),\Ln(3-\sqrt3 i)$.
\end{example}

\begin{solution}
  \[
    \Ln(-2+3i)=\ln\abs{-2+3i}+i\Arg(-2+3i)
      =\frac 12\ln 13+\left(-\arctan\frac 32+\pi+2k\pi\right)i,
      \quad k\in\BZ.
  \]
  \[
    \Ln(3-\sqrt3i)=\ln\abs{3+\sqrt 3i}+i\Arg(3-\sqrt 3i)
      =\ln 2\sqrt 3+\left(-\frac\pi6+2k\pi\right)i
      =\ln 2\sqrt 3+\left(2k-\frac16\right)\pi i,
      \quad k\in\BZ.
  \]
\end{solution}

\begin{example}
  解方程 $e^z-1-\sqrt 3i=0$.
\end{example}

\begin{solution}
  由于 $1+\sqrt 3 i=2e^{\frac{\pi i}3}$, 因此
  \[z=\Ln(1+\sqrt 3i)=\ln 2+\left(2k+\frac13\right)\pi i,\quad k\in\BZ.\]
\end{solution}

\begin{exercise}
  求 $\ln(-1-\sqrt3 i)=$\fillblank[2cm][3mm]{}.
\end{exercise}

对数函数与其主值的关系是
\[\Ln z=\ln z+\Ln 1=\ln z+2k\pi i,\quad k\in\BZ.\]
根据辐角以及辐角主值的相应等式, 我们有
\[\Ln(z_1\cdot z_2)=\Ln z_1+\Ln z_2,\quad
  \Ln\frac{z_1}{z_2}=\Ln z_1-\Ln z_2,\]
\[\Ln \sqrt[n]z=\dfrac1n\Ln z.\]
而当 $\abs{n}\ge 2$ 时, \alert{$\Ln z^n=n\Ln z$ 不成立}.
以上等式换成 $\ln z$ 均不一定成立.


设 $x$ 是正实数, 则
\[\ln (-x)=\ln x+\pi i,\quad
  \lim_{y\to0^-}\ln (-x+yi)=\ln x-\pi i,\]
因此 $\ln z$ 在负实轴和零处不连续.
而在其它地方, $-\pi<\arg z<\pi$, $\ln z$ 是 $e^z$ 在区域 $-\pi<\Im z<\pi$ 上的单值反函数, 
从而 \alert{$(\ln z)'=\dfrac 1z$}, \alert{$\ln z$ 在除负实轴和零处的区域解析}.
\footnote{任取一条从 $0$ 到 $\infty$ 的简单曲线, 在去掉这条曲线后, 若固定一复数 $z_0$ 的辐角, 则多值函数 $\Arg z$ 可以在该区域内连续单值化, 简单来说就是沿着 $z_0$ 到 $z$ 的曲线让辐角连续变化. 同理, $\Ln z$ 也可以在该区域内单值化, 只需固定一复数 $z_0$ 的值.}

也可以通过C-R方程来得到 $\ln z$ 的解析性和导数: 当 $x>0$ 时,
\[\ln z=\half \ln(x^2+y^2)+i\arctan \frac yx,\]
\[u_x=v_y=\frac x{x^2+y^2},\qquad v_x=-u_y=-\frac y{x^2+y^2},\]
\[(\ln z)'=\frac{x-yi}{x^2+y^2}=\frac 1z.\]
其它情形可取虚部为 $\arccot\dfrac xy$ 或 $\arccot\dfrac xy-\pi$ 类似证明.

\subsection{幂函数}

\begin{definition}{幂函数}
  \begin{enumerate}
    \item 设 $a\neq 0$, $z\neq 0$, 定义\noun{幂函数}
    \[w=z^a=e^{a\Ln z}
    =\exp\bigl(a\ln\abs{z}+ia(\arg z+2k\pi)\bigr),\quad k\in\BZ.\]
    \item \nouns{幂函数的主值}{幂函数!幂函数的主值}为
    \[w=e^{a\ln z}=\exp\bigl(a\ln\abs{z}+ia\arg z\bigr).\]
  \end{enumerate}
\end{definition}

根据 $a$ 的不同, 这个函数有着不同的性质.
\begin{enumerate}
  \item 当 $a$ 为整数时, 因为 $e^{2ak\pi i}=1$, 所以 $w=z^a$ 是单值的. 此时 $z^a$ 就是我们之前定义的乘幂. 
    当 $a$ 是非负整数时, $z^a$ 在复平面上解析;
    当 $a$ 是负整数时, $z^a$ 在 $\BC-\set0$ 上解析.
  \item 当 $a=\dfrac pq$ 为分数, $p,q$ 为互质的整数且 $q>1$ 时,
    \[z^{\frac pq}=\abs{z}^{\frac pq}\exp\biggl(\frac{ip(\arg z+2k\pi)}q\biggr),\quad k=0,1,\dots,q-1\]
    具有 $q$ 个值.
    去掉负实轴和 $0$ 之后, 它的主值 $w=\exp(a\ln z)$ 是处处解析的.
    事实上它就是 $\sqrt[q]{z^p}=(\sqrt[q]z)^p$.
    \begin{figure}[!h]
      \centering
      \begin{tikzpicture}
        \coordinate [label=below left:{$0$}] (O) at (0,0);
        \coordinate [label=below:{$x$}] (X) at (2,0);
        \coordinate [label=left:{$y$}] (Y) at (0,2);
        \draw[cstaxis] (O)--(X);
        \draw[cstaxis] (0,-2)--(Y);
        \draw[draw=white,cstfille1] (O) circle (1.3);
        \draw[cstdash,main] (0,0)--(-2,0);
        \draw[cstdash,cstra,third] (1.7,.7)to [bend left] (4.5,.7);
        \draw (3,1.7) node[third] {$w=z^{2/9}$};
        \begin{scope}[xshift=5cm]
          \coordinate [label=below left:{$0$}] (O) at (0,0);
          \fill[cstfille2,pattern color=second] (O)--({1.4*cos(40)},{1.4*sin(40)}) arc (40:-40:1.4)--cycle;
          \draw[cstdash,second] ({1.4*cos(40)},{-1.4*sin(40)})--(0,0)--({1.4*cos(40)},{1.4*sin(40)});
          \coordinate [label=below left:{$0$}] (O) at (0,0);
          \coordinate [label=below:{$u$}] (X) at (2,0);
          \coordinate [label=left:{$v$}] (Y) at (0,2);
          \draw[cstaxis] (-2,0)--(X);
          \draw[cstaxis] (0,-2)--(Y);
        \end{scope}
      \end{tikzpicture}
      \caption{映照 $w=z^{2/9}$}
    \end{figure}
  \item 对于其它的 $a$, $z^a$ 具有无穷多个值.
    这是因为此时当 $k\neq0$ 时, $2k\pi a i$ 不可能是 $2\pi i$ 的整数倍. 
    从而不同的 $k$ 得到的是不同的值.
    去掉负实轴和 $0$ 之后, 它的主值 $w=\exp(a\ln z)$ 也是处处解析的.
    \footnote{对于 $\Ln\dfrac{z-a}{z-b},\sqrt{(z-a)(z-b)}$ 等类型的多值函数, 我们需要将它的``奇点''连接起来形成``割线''. 复平面上去掉这些割线得到的区域内, 这些函数也可以如同 $\Arg z,\Ln z$ 那样单值化.}
\end{enumerate}

\begin{center}
  \begin{tabular}{cccc} \toprule
    $a$& $z^a$ 的值& $z^a$ 的解析区域\\ \midrule
    &&$n\ge0$ 时处处解析\\
    \multirow{-2}*{整数 $n$}&\multirow{-2}*{单值}&$n<0$ 时除零点外解析\\ \midrule
    分数 $p/q$&$q$ 值&除负实轴和零点外解析\\ \midrule
    无理数或虚数&无穷多值&除负实轴和零点外解析\\ \bottomrule
  \end{tabular}
\end{center}

\begin{example}
  求 $1^{\sqrt 2}$ 和 $i^i$.
\end{example}
\begin{solution}
  \[
    1^{\sqrt2}=e^{\sqrt2\Ln1}
      =e^{\sqrt 2\cdot 2k\pi i}
      =\cos(2\sqrt 2k\pi)+i\sin(2\sqrt 2k\pi), \quad k\in\BZ.
  \]
  \[
    i^i=e^{i\Ln i}
      =\exp\biggl(i\cdot\Bigl(2k+\half\Bigr)\pi i\biggr)
      =\exp\Bigl(-2k\pi-\half\pi\Bigr), \quad k\in\BZ.
  \]
\end{solution}

\begin{exercise}
  $3^i$ 的辐角主值是\fillblank{}.
\end{exercise}

幂函数与其主值有如下关系:
\[
  z^a=e^{a\ln z}\cdot 1^a
    =e^{a\ln z}\cdot e^{2ak\pi i},\quad k\in\BZ.
\]
对于幂函数的主值,
\[(z^a)'=\left(e^{a\ln z}\right)'=\frac{ae^{a\ln z}}z=az^{a-1}.\]
一般而言, $z^a\cdot z^b=z^{a+b}$ 和 $(z^a)^b=z^{ab}$ 都是不成立的.
\footnote{$z^a\cdot z^b=z^{a+b}$ 成立当且仅当 $\dfrac{a}{a+b}\in\BZ$. $(z^a)^b=z^{ab}$ 成立当且仅当 $\dfrac1a\in\BZ$.}

最后, 注意 $e^a$ 作为指数函数 $f(z)=e^z$ 在 $a$ 处的值和作为 $g(z)=z^a$ 在 $e$ 处的值是\alert{不同}的.
因为后者在 $a\not\in\BZ$ 时总是多值的.
前者实际上是后者的主值.
为避免混淆, 以后我们总\alert{默认 $e^a$ 表示指数函数 $\exp a$}.


\subsection{三角函数和反三角函数}

我们知道
  \[\cos x=\frac{e^{ix}+e^{-ix}}2,\quad
  \sin x=\frac{e^{ix}-e^{-ix}}{2i}\]
对于任意实数 $x$ 成立,
我们将其推广到复数情形.

\begin{definition}{余弦和正弦函数}
  定义\noun{余弦函数}和\noun{正弦函数}
  \[\cos z=\frac{e^{iz}+e^{-iz}}2,\quad
  \sin z=\frac{e^{iz}-e^{-iz}}{2i}.\]
\end{definition}
那么欧拉恒等式 \alert{$e^{iz}=\cos z+i\sin z$ 对任意复数 $z$ 均成立}.

不难得到
\[
  \cos(iy)=\dfrac{e^y+e^{-y}}2,\qquad
  {\sin(iy)=i\dfrac{e^y-e^{-y}}2.}
\]
当 $y\to\infty$ 时, $\cos(iy)$ 和 $\sin(iy)$ 都 $\to\infty$.
因此 \alert{$\sin z$ 和 $\cos z$ 并不有界}. 
这和实变情形不同.

容易看出 $\cos z$ 和 $\sin z$ 的零点都是实数.
于是可类似定义其它三角函数\index{正切函数}
\begin{align*}
  \tan z&=\frac{\sin z}{\cos z},z\neq\left(k+\half\right)\pi,&
  \cot z&=\frac{\cos z}{\sin z},z\neq k\pi,\\
  \sec z&=\frac{1}{\cos z},z\neq\left(k+\half\right)\pi,&
  \csc z&=\frac{1}{\sin z},z\neq k\pi.
\end{align*}
这些三角函数的奇偶性, 周期性和导数与实变情形类似,
  \[(\cos z)'=-\sin z,\quad
  (\sin z)'=\cos z,\]
且在定义域范围内是处处解析的.
三角函数的各种恒等式在复数情形也仍然成立, 例如
\begin{itemize}
  \item $\cos(z_1\pm z_2)=\cos z_1 \cos z_2\mp \sin z_1 \sin z_2$,
  \item $\sin(z_1\pm z_2)=\sin z_1 \cos z_2\pm\cos z_1 \sin z_2$,
  \item $\sin^2z+\cos^2z=1$.
\end{itemize}

类似的, 我们可以定义\noun{双曲函数}:
\begin{align*}
  \ch z&=\frac{e^z+e^{-z}}2=\cos iz,\\
  \sh z&=\frac{e^z-e^{-z}}2=-i\sin iz,\\
  \tanh z&=\frac{e^z-e^{-z}}{e^z+e^{-z}}
    =-i\tan iz,\quad z\neq \left(k+\half\right)\pi i.\
\end{align*}
它们的奇偶性和导数与实变情形类似, 在定义域范围内是处处解析的.
$\ch z,\sh z$ 的周期是 $2\pi i$, $\tanh z$ 的周期是 $\pi i$.

设 $z=\cos w=\dfrac{e^{iw}+e^{-iw}}2$, 则
  \[e^{2iw}-2ze^{iw}+1=0,\quad
    {e^{iw}=z+\sqrt{z^2-1}\footnote{注意右侧是双值函数}.}\]
因此\noun{反余弦函数}为
\[w=\Arccos z=-i\Ln(z+\sqrt{z^2-1}).\]
显然它是多值的. 同理, 我们有:
\begin{itemize}
  \item \noun{反正弦函数} $\Arcsin z=-i\Ln(iz+\sqrt{1-z^2})$;
  \item \noun{反正切函数} $\Arctan z=-\dfrac i2\Ln\dfrac{1+iz}{1-iz}, z\neq \pm i$;
  \item \noun{反双曲余弦函数} $\Arch z=\Ln(z+\sqrt{z^2-1})$;
  \item \noun{反双曲正弦函数} $\Arsh z=\Ln(z+\sqrt{z^2+1})$;
  \item \noun{反双曲正切函数} $\Arth z=\dfrac12\Ln\dfrac{1+z}{1-z}, z\neq \pm1$.
\end{itemize}

\begin{example}
  解方程 $\sin z=2$.
\end{example}

\begin{solution}
  由于
  \[\sin z=\dfrac{e^{iz}-e^{-iz}}{2i}=2,\]
  我们有
  \[e^{2iz}-4ie^{iz}-1=0.\]
  于是 $e^{iz}=(2\pm\sqrt 3)i$,
  \[z=-i\Ln[(2\pm\sqrt 3)i]=\left(2k+\half\right)\pi\pm i\ln(2+\sqrt3),\quad k\in\BZ.\]
\end{solution}

\begin{solution}[另解]
  由 $\sin z=2$ 可知
  \[\cos z=\sqrt{1-\sin^2 z}=\pm\sqrt 3i.\]
  于是 $e^{iz}=\cos z+i\sin z=(2\pm\sqrt 3)i$,
    \[z=-i\Ln[(2\pm\sqrt 3)i]=\left(2k+\half\right)\pi\pm i\ln(2+\sqrt3),\quad k\in\BZ.\]
\end{solution}
我们总有形式
\begin{align*}
  \Arcsin z&=(2k+\half)\pi\pm \theta,\\
  \Arccos z&=2k\pi\pm \theta,\\
  \Arctan z&=k\pi+\theta,\quad k\in\BZ.
\end{align*}


\sectionHomework




\item $i^{-i}$ 的主值是\fillblank{}.
\item $2^{-i}$ 的辐角主值是\fillblank{}.
\item 下面哪个函数在 $z=0$ 处不可导?~(~~~~)
\xx{$2x+3yi$}{$2x^2+3y^2i$}{$x^2-xyi$}{$e^x\cos y+i e^x\sin y$}
\item 解方程 $\sin z=2\cos z$.
\item 如果函数 $f(z)=e^{ax}(\cos y-i\sin y)$ 在复平面上处处解析, 则实数 $a=$\fillblank{}.
\item 函数 $f(z)=u(x,y)+iv(x,y)$ 在 $z_0=x_0+iy_0$ 处可导的充要条件是(~~~~).
\xx{$u,v$ 均在 $(x_0,y_0)$ 处连续}{$u,v$ 均在 $(x_0,y_0)$ 处有偏导数}{$u,v$ 均在 $(x_0,y_0)$ 处可微}{$u,v$ 均在 $(x_0,y_0)$ 处可微且满足C-R方程}
\item 解方程 $\cos z=\dfrac{3\sqrt2}4$.
\item 求 $\Ln(1+\sqrt3i)$.
\item 已知 $f(z)=u+iv$ 是解析函数, 其中 $u(x,y)=x^2+axy-y^2, v=2x^2-2y^2+2xy$ 且 $a$ 是实数.
求参数 $a$ 以及解析函数 $f'(z)$, 其中 $f'(z)$ 需要写成 $z$ 的表达式.
\item 设 $C$ 为有向曲线 $z(t)=\sin t+it,0\le t\le \pi$, 求 $\displaystyle\int_C ze^z \diff z$.
\item 设 $C$ 为正向圆周 $|z-1|=4$, 求 $\displaystyle\oint_C\frac{\sin z}{z^2+1}\diff z$.
\item 假设 $u(x,y)=x^3+ax^2y+bxy^2-3y^3$ 是调和函数,求参数 $a,b$ 以及 $v(x,y)$ 使得 $v(0,0)=0$ 且 $f(z)=u+iv$ 是解析函数.






\begin{homework}
  \item 判断题.
    \begin{exlist}
      \item 如果 $f'(z_0)$ 存在, 那么 $f(z)$ 在 $z_0$ 解析.\fillbrace{}
      \item 如果 $z_0$ 是 $f(z)$ 的奇点, 那么 $f(z)$ 在 $z_0$ 不可导.\fillbrace{}
      \item 如果 $z_0$ 是 $f(z)$ 和 $g(z)$ 的奇点, 那么 $z_0$ 也是 $f(z)+g(z)$ 和 $f(z)/g(z)$ 的奇点.\fillbrace{}
      \item 如果 $u(x,y)$ 和 $v(x,y)$ 偏导数均存在, 那么 $f(z)=u+iv$ 亦可导.\fillbrace{}
      \item 如果 $f(z)$ 在区域 $D$ 内处处可导, 则 $f(z)$ 在区域 $D$ 解析. \fillbrace{}
      \item 对任意复数 $z$, 有 $\ov{e^z}=e^{\ov z}$.\fillbrace{}
      \item 对任意复数 $z$, 有 $\ov{\cos z}=\cos{\ov z}$.\fillbrace{}
      \item 对任意复数 $z$, 有 $\ov{\sin z}=\sin{\ov z}$.\fillbrace{}
      \item 对任意复数 $z$, 有 $\ch^2z-\sh^2z=1$.\fillbrace{}
    \end{exlist}
  \item 选择题.
    \begin{exlist}
      \item 函数 $f(z)$ 在点 $z_0$ 的邻域内可导是 $f(z)$ 在该邻域内解析的\fillbrace{}.
        \begin{taskschoice}(2)
          \task 充分条件
          \task 必要条件
          \task 充要条件
          \task 既非充分也非必要条件
        \end{taskschoice}
      \item 设 $f(z)=u(x,y)+iv(x,y)$. 将下述选项不重复地填入括号内:
      \[\fillbrace{}\implies
        \fillbrace{}\implies
        \fillbrace{}\implies
        \fillbrace{}\implies
        \fillbrace{}\implies
        \fillbrace{}\]
        \begin{taskschoice}(2)
          \task $f(z)$ 在点 $z_0$ 有定义
          \task $f(z)$ 在点 $z_0$ 连续
          \task $f(z)$ 在点 $z_0$ 可导
          \task $f(z)$ 在点 $z_0$ 解析
          \task $f(z)$ 在点 $z_0$ 的一个邻域内解析
          \task $u,v$ 均在点 $(x_0,y_0)$ 处有偏导数
        \end{taskschoice}
      \item 下列函数中, 为解析函数的是\fillbrace{}.
        \begin{taskschoice}(2)
          \task $x^2-y^2-2xyi$
          \task $x^2+xyi$
          \task $2(x-1)y+i(y^2-x^2+2x)$
          \task $x^3+iy^3$
        \end{taskschoice}
      \item 设 $n$ 是正整数, $z_1,z_2\neq0$. 下列式子一定正确的是\fillbrace{}.
        \begin{taskschoice}(2)
          \task $\Arg(\sqrt z)=\dfrac12\Arg z$
          \task $\Arg(z^n)=n\Arg z$
          \task $\Arg(z_1z_2)=\Arg(z_1)+\Arg(z_2)$
          \task $\arg(\sqrt z)=\dfrac12\arg z$
          \task $\arg(z^{-n})=-n\arg z$
          \task $\arg(z_1/z_2)=\arg(z_1)-\arg(z_2)$
          \task $\Ln\sqrt[3] z=\dfrac13\Ln z$
          \task $\Ln(z^{-n})=-n\Ln z$
          \task $\Ln(z_1z_2)=\Ln(z_1)+\Ln(z_2)$
          \task $\ln\sqrt z=\dfrac12\ln z$
          \task $\ln(z^n)=n\ln z$
          \task $\ln(z_1/z_2)=\ln(z_1)-\ln(z_2)$
        \end{taskschoice}
    \end{exlist}
  \item 填空题.
    \begin{exlist}
      \item 函数 $\dfrac{z+1}{z(z^2+1)}$ 的奇点为\fillblank{}.
      \item 函数 $\dfrac{z-2}{(z+1)^2(z^2+1)}$ 的奇点为\fillblank{}.
      \item 函数 $\dfrac{1}{\sin z}$ 的奇点为\fillblank{}.
      \item 如果函数 $f(z)=x^2-2xy-y^2+i(ax^2+bxy+cy^2)$ 在复平面上处处解析, 则 $a+b+c=$\fillblank{}.
      \item 计算 $\ln i=$\fillblank[2cm]{}.
      \item 设 $z=1^{\sqrt3}$, 则 $|z|=$\fillblank{}.
      \item $i^{-i}$ 的主值是\fillblank{}.
    \end{exlist}
  \item 计算题.
    \begin{exlist}
      \item 设 $f(z)=\dfrac15z^5-(1+i)z$, 解方程 $f'(z)=0$.
      \item 下列函数何处可导? 何处解析?
        \begin{tasks}(3)
          \task $f(z)=1/\ov z$;
          \task* $f(z)=x^3-3xy^2+i(3x^2y-y^3)$;
          \task $f(z)=2x^3+3y^3i$;
          \task $f(z)=xy^2+ix^2y$;
          \task $f(z)=e^{x^2+y^2}$;
          \task $f(z)=z\Im z$;
          \task* $f(z)=\sin x\ch y+i\cos x\sh y$.
        \end{tasks}
      \item 指出下列函数 $f(z)$ 的解析区域, 并求出其导数.
        \begin{tasks}(2)
          \task $(z-1)^5$;
          \task $z^3+2iz$;
          \task $\dfrac1{z^2-1}$;
          \task $\dfrac{az+b}{cz+d}$ ($c,d$ 不全为零).
        \end{tasks}
      \item 设 $my^3+nx^2y+i(x^3+lxy^2)$ 为解析函数, 试确定实数 $l,m,n$ 的值.
      \item 计算
        \begin{tasks}(4)
          \item 计算 $\Arccos 2$.
          \task $\Ln 4$;
          \task $2\Ln2$;
          \task $\ln(-i)$;
          \task $\ln(-3+4i)$;
          \task $\Im\sin(1+i)$;
          \task $\arg e^{1-4i}$;
          \task $\exp\left(1-\dfrac{\pi i}2\right)$;
          \task $\exp\left(\dfrac{1+\pi i}4\right)$;
          \task $3^i$;
          \task $(1+i)^i$.
        \end{tasks}
      \item 解方程 
        \begin{tasks}(3)
          \task $\sin z=0$;
          \task $\cos z=0$;
          \task $1+e^z=0$;
          \task $\sin z+\cos z=0$;
          \task $\sin z=2\cos z$.
        \end{tasks}
      \item 验证 $e^x(x\cos y-y\sin y)+i e^x(y\cos y+x\sin y)$ 在全平面解析, 并求出其导数. 它在无穷远解析吗? 为何?

      \item 计算 $\displaystyle\int_0^{+\infty}\frac{x^2+1}{x^4+1}\diff x$.
      \item 复变函数 $f(z)=\sin z$ 和实变量函数 $g(x)=\sin x$ 的性质有什么相似和不同之处? 试说出3点.
      \item 求出 $\dfrac{1}{\sin z-2}$ 的解析区域.
      \item 证明: 若整函数(在整个复平面解析) $f$ 将实轴和虚轴均映为实数, 则 $f'(0)=0$.
    \end{exlist}
\end{homework}
  

\sectionExtraReading
\begin{homework}
  \item 仿照复数的指数函数, 我们可以尝试在矩阵上定义指数函数. 设 $\bfA\in M_m(\BC)$ 是一个 $m\times m$ 的复矩阵, 我们想说明极限
  \[e^\bfA:=\lim_{n\to \infty}\left(1+\frac1n \bfA\right)^n\]
  存在.
  \begin{exlist}
    \item 当 $\bfA=\diag\set{a_1,a_2,\dots,a_m}$ 是一个对角矩阵时, 证明 $e^\bfA$ 存在且
    \[e^\bfA=\diag\set{e^{a_1},e^{a_2},\dots,e^{a_m}}.\]
    \item 当
    \[\bfA=\bfJ_m(a)=\begin{pmatrix}
      a&1&   &\\
      &a& 1 &\\
      & &\ddots&1\\
      & &      &a
    \end{pmatrix}\]
    是约当块时, 证明 $e^\bfA$ 存在.
    \item 每个方阵都可以相似于一些约当块构成的分块对角阵, 由此证明 $e^\bfA$ 总存在.
    \item 当 $\bfA=x\bfE+y\bfJ=\begin{pmatrix}
      x&y\\-y&x
    \end{pmatrix}$ 时, 证明 $e^\bfA=\begin{pmatrix}
      e^x\cos y&e^x\sin y\\-e^x\sin y&e^x\cos y
    \end{pmatrix}=e^x(cos y\bfE+\sin y \bfJ)$.
    \item 证明 $e^\bfA=\bfE+\bfA+\dfrac{\bfA^2}{2!}+\dfrac{\bfA^3}{3!}+\cdots$.
    \item 证明 $e^{\bfA+\bfB}=e^\bfA\cdot e^\bfB$.
  \end{exlist}
  
  \item 注意到 $x=\dfrac12z+\dfrac12\ov z,y=-\dfrac i2z+\dfrac i2\ov z$.
  仿照着二元实函数偏导数在变量替换下的变换规则, 我们定义 $f$ 对 $z$ 和 $\ov z$ 的偏导数为
  \[\left\{
  \begin{aligned}
    \frac{\partial f}{\partial z}&
  =\frac{\partial x}{\partial z}\frac{\partial f}{\partial x}
    +\frac{\partial y}{\partial z}\frac{\partial f}{\partial y}
  =\frac12\frac{\partial f}{\partial x}-\frac i2\frac{\partial f}{\partial y},\\
    \frac{\partial f}{\partial \ov z}&
  =\frac{\partial x}{\partial \ov z}\frac{\partial f}{\partial x}
    +\frac{\partial y}{\partial \ov z}\frac{\partial f}{\partial y}
  =\frac12\frac{\partial f}{\partial x}+\frac i2\frac{\partial f}{\partial y}.
  \end{aligned}\right.\]
  \begin{exlist}
  \item 证明C-R方程等价于 $\dfrac{\partial f}{\partial \ov z}
  =0$.
  所以我们也可以把 \emph{$\dfrac{\partial f}{\partial \ov z}=0$} 叫做C-R方程.

  \item 利用该结论求函数 $f(z)=\ov z, z\Im z, e^{z\ov z}$ 的可导点和解析点.
  \end{exlist}
\end{homework}


% 

\chapter{类域论}
\begin{introduction}
\item 抽象类域论 \ref{sec:abstract class field theory}
\item 局部类域论 \ref{sec:local class field theory}
\item 整体类域论 \ref{sec:global class field theory}
\end{introduction}

我们称伽罗瓦群为交换群的扩张为\noun{阿贝尔扩张}.
\begin{question}{}{}
给定数域 $K$, 如何确定它的所有阿贝尔扩张?
\end{question}

类域论的主要目的是为了分类给定域 $K$ 的所有阿贝尔扩张, 这些扩张应当由 $K$ 自身的结构所导出. 它建立了这些扩张和与 $K$ 有关的交换群 $A_K$ 的子群的一一对应.
在局部域的情形 $A_K=K^\times$, 在整体域的情形 $A_K$ 是理想类群的变种——伊代尔类群. 类域论的核心是互反映射
  \[r: G(L/K)^\ab\simto A_K/\bfN_{L/K} A_L.\]
我们先研究抽象的类域论, 然后再应用到具体情形.

\section{抽象类域论}
\label{sec:abstract class field theory}
抽象类域论又称 class formation. 它的目的是为了统一地处理各种类域论. 在阅读过程中, 我们可以先代入局部类域论的情形来理解.

\subsection{射影有限群}
为了研究无限伽罗瓦扩张, 我们需要学习射影有限群.
\begin{proposition}{}{}
\noun{射影有限群}是指有限群的逆向极限, 每个分量赋予离散拓扑. 这等价于一个紧豪斯多夫群, 且存在由正规子群形成幺元的一组邻域基.
\end{proposition}
\begin{proof}
设 $G$ 是一个紧豪斯多夫群, 正规子群 $\set{N_i}$ 形成 $1\in G$ 的一组邻域基. 则 $N_i$ 的陪集形成 $G$ 的一组开覆盖, 因此 $G/N_i$ 有限. 设
  \[f:G\to \plim G_i,\quad G_i=G/N_i.\]
由于 $G$ 是豪斯多夫, $G_i$ 都是离散的.
由于 $\cap N_i=\set{1}$, 因此 $f$ 单.
对于 $\plim G_i$ 的开集 
  \[U_S=\prod_{i\in S}\set{1_{G_i}} \times  \prod_{i\notin S} \plim G_i,\quad \#S<\infty,\]
$f^{-1}(U_S)=\cap_{i\in S} N_i$ 开, 因此 $f$ 连续.
对于任意 $x=(x_i)\in\plim G_i$ 和邻域 $xU_S$, 令 $N=\cap_{i\in S} N_i$, 则存在 $y\in G$ 使得 $x_i=y\mod N$, 因此 $f(y)\in xU_S$, $f$ 是稠密的. 由于 $G$ 紧, $f$ 将闭集映为闭集, 开集映为开集, 因此 $f$ 是满的.

反之则可以直接验证得到.
\end{proof}

\begin{exercise}\label{exe:closed_open_subgroups}
拓扑群的开子群是闭子群, 有限指标的闭子群是开子群.
\end{exercise}

\begin{example}
设 $\Omega/k$ 是任意伽罗瓦扩张, $G=G(\Omega/k)$ 为其伽罗瓦群. 我们有\footnote{称该拓扑为\noun{克鲁尔拓扑}.}
  \[G(\Omega/k)\cong\plim G(K/k),\]
其中 $K$ 取遍 $k$ 的有限伽罗瓦子扩张.
\end{example}

\begin{example}
设 $\CO$ 是一个剩余域有限的完备离散赋值环, $\fp$ 是极大理想, 则
  \[\CO=\plim_n \CO/\fp^n,\quad \CO^\times=\plim_n \CO^\times/U^{(n)}.\]
\end{example}

\begin{example}
令
  \[\wh \BZ=\plim_n \BZ/n\BZ.\]
则 $\wh \BZ/n\wh \BZ\cong \BZ/n\BZ$ 且
  \[\wh \BZ=\prod_p \BZ_p.\]
我们有自然的嵌入 $\BZ\inj \wh \BZ$, 它是稠密的.
\end{example}

\begin{example}
我们有 
  \[G(\BF_{q^n}/\BF_q)\cong \BZ/n\BZ,\quad \varphi_q\mapsto 1,\]
其中 $\varphi_q(x)=x^q$ 是弗罗贝尼乌斯. 因此
  \[G(\ov\BF_q/\BF_q)\cong \wh \BZ.\]
\end{example}

\begin{example}
如果 $K$ 是一个完备离散赋值域, 且剩余域 $\kappa$ 有限. 则我们知道 $K^\ur=K W(\ov\kappa)$ 是它的极大非分歧扩张, 且
  \[G(K^\ur/K)\cong G(\ov \kappa/\kappa)\cong \wh \BZ.\]
\end{example}

\begin{example}
设 $\BQ^\cyc=\cup_n \BQ(\zeta_n)$, 则
  \[G(\BQ^\cyc/\BQ)\cong \plim (\BZ/n\BZ)^\times=\wh\BZ^\times\cong\prod_p \BZ_p^\times.\]
\end{example}

\begin{example}
如果一个射影有限群 $G$ 由一个元素 $\sigma\in G$ 拓扑生成, 即 $G=\ov{\pair{\sigma}}$, 则称之为\noun{射影循环群}. 它的开子群一定形如 $G^n$. 例如 $G=\BZ_p,\wh \BZ$ 均是射影循环群.
\end{example}

\begin{example}
设 $A$ 是一个交换挠群, 定义它的\noun{庞特里亚金对偶}为
  \[A^\vee=\Hom(A,\BQ/\BZ).\]
如果 $A=\cup_i A_i$ 是一些有限子群的并, 则
  \[A^\vee=\plim A_i^\vee\]
是射影有限群. 例如 $A=\BQ/\BZ=\cup_n \frac{1}{n}\BZ/\BZ$,
则 $(\frac{1}{n}\BZ/\BZ)^\vee=\BZ/n\BZ$,
  \[A^\vee=\plim \BZ/n\BZ=\wh \BZ.\]
\end{example}

\begin{example}
对于任意群 $G$, $N$ 取遍它的有限指标的正规子群. 定义
  \[\wh G=\plim_N G/N\]
为它的\noun{射影完备化}. 例如 $\BZ$ 的射影完备化就是 $\wh \BZ$.
\end{example}

\begin{exercise}
射影有限(乘法)群 $G$ 均可视为 $\wh \BZ$ 模, 即对于任意 $\sigma\in G,a,b\in\wh\BZ$, $(\sigma^a)^b=\sigma^{ab},\sigma^{a+b}=\sigma^a\sigma^b$. 其作用限制在 $\BZ$ 就是通常的 $\BZ$ 作用.
\end{exercise}

\begin{exercise}
如果 $G=\plim_i G_i$ 是射影有限的, 则 $G^\ab=\plim_i G_i^\ab$.
\end{exercise}


\subsection{抽象伽罗瓦理论}
设 $G$ 是一个射影有限群.
$\set{G_E}_{E\in X}$ 是它的闭子群构成的一个集族, 且其中包含 $G$ 和 $\set{1}$.
我们把指标 $E$ 叫做(抽象)\noun{域}.
那么存在域 $k$ 和 $\bar k$ 满足 $G_k=G,G_{\bar k}=1$.

如果 $G_L\subseteq G_K$, 我们记 $K\subseteq L$ 或 $L/K$, 并称之为\noun{域扩张}.
如果 $n=[G_K:G_L]$ 有限, 称 $L/K$ 为 $n$ 次\noun{有限扩张}, 此时 $G_L$ 是 $G_K$ 的开子群.
如果 $G_L\triangleleft G_K$ 是正规子群, 称 $L/K$ 为\noun{正规扩张}或\noun{伽罗瓦扩张}, 并定义 $G(L/K)=G_K/G_L$.
不难证明
\[G=\plim_{[K:k]<\infty} G(K/k),\]
其中 $K$ 取遍 $k$ 的所有有限伽罗瓦扩张.
类似地, 我们可以定义域的交、复合、共轭以及循环扩张、阿贝尔扩张等概念, 即对应的 $G_E$ 复合、交、共轭以及 $G(L/K)$ 是循环群、阿贝尔群等.

\begin{example}
设 $E$ 是一个域, $\Omega$ 是它的一个伽罗瓦扩张, $G=G(\Omega/E)$, $X$ 为所有包含在 $\Omega$ 的 $E$ 的有限扩张 $F$ 以及 $\Omega$, $G_F=G(\Omega/F)$. 特别地, 我们可以取 $\Omega=E^\sep$.
\end{example}

设 $A$ 是一个拓扑 $G$ 模($A$ 赋予离散拓扑), 即 $G$ 在 $A$ 上的作用 $G\times A\to A$ 连续, 则对于任意 $a\in A$, 存在 $(1,a)\in G\times A$ 的一个邻域 $G_K\times \set{a}$ 落在它的原像, 即 $a\in A_K=A^{G_K}$. 因此
  \[A=\bigcup_{[K:k]<\infty} A_K.\]
定义范映射
  \[\fct{\bfN_{L/K}: A_L}{A_K}{a}{\prod_\sigma a^\sigma,}\]
其中 $\sigma$ 取遍 $G_K\bs G_L$ 的一组代表元. 对于伽罗瓦扩张, $A_L$ 是 $G(L/K)$ 模且 $A_L^{G(L/K)}=A_K$.
因此相应的泰特上同调为
  \[\begin{split}
	\rmH^0\bigl(G(L/K),A_L\bigr)&=A_K/\bfN_{L/K}A_L,\\
	\rmH^{-1}\bigl(G(L/K),A_L\bigr)&=A_L^{\bfN=1}/I_{G(L/K)} A_L.
	\end{split}\]

\subsection{抽象分歧理论}
固定一个满射
  \[d:G\to \wh \BZ.\]
记 $\wt k$ 为 $I=\ker d$ 的固定域, 则 $G(\wt k/k)\cong \wh \BZ$.
对于任意域 $K$, 称 
  \[I_K=G_K\cap I=G_K\cap G_{\wt k}=G_{K\wt k},\]
为 $K$ 的\nouns{惯性群}{惯性!惯性群}, 它是 $\wt K:=K\wt k$ 的固定域, 称之为 $K$ 的\noun{极大非分歧扩张}. 令
  \[f_K=[\wh \BZ:d(G_K)],\quad e_K=[I:I_K].\]
如果 $f_K$ 有限, 则我们有满同态
  \[d_K=\frac{1}{f_K}d:G_K\to \wh \BZ,\]
核为 $I_K$, 因此 $d_K:G(\wt K/K)\simto \wh \BZ.$

  \[\xymatrix{
&& \bar k\\
\wt k\ar@{-}[r]\ar@{-}[rru]^I & \wt K\ar@{-}[ru]_{I_K} &\\
k\ar@{-}[r]\ar@{-}[u]^{\wh \BZ}& K \ar@{-}[u]^{d(G_K)}\ar@/_/@{-}[ruu]_{G_K}&
}\]

\begin{definition}{弗罗贝尼乌斯}{frobenius}
称 $G(\wt K/K)$ 的拓扑生成元 $\varphi_K=d_K^{-1}(1)$ 为 $K$ 的\noun{弗罗贝尼乌斯}.
\end{definition}

对于域扩张 $L/K$, 定义\nouns{惯性指数}{惯性!惯性指数} $f_{L/K}=[d(G_K):d(G_L)]$, $e_{L/K}=[I_K:I_L]$.
我们有 $\wt L=L\wt K$, 于是 $f_{L/K}=[L\cap \wt K:K].$
对于域扩张 $K\subseteq L\subseteq M$, 显然
  \[f_{M/K}=f_{L/K}f_{M/L},\quad
e_{M/K}=e_{L/K}e_{M/L}.\]

\begin{proposition}{}{}
我们有 $[L:K]=f_{L/K}e_{L/K}$.
\end{proposition}
\begin{proof}
考虑交换图表
  \[\xymatrix{
    1\ar[r] & I_L\ar[r]\ar[d] & G_L\ar[r]\ar[d] & d(G_L)\ar[r]\ar[d] &1\\
    1\ar[r] & I_K\ar[r] & G_K\ar[r] & d(G_K)\ar[r] &1.
  }\]
如果 $L/K$ 伽罗瓦, 则
  \[1\ra I_K/I_L\ra G(L/K)\ra d(G_K)/d(G_L)\ra 1\]
正合. 如果 $L/K$ 不是伽罗瓦, 考虑它的伽罗瓦闭包 $M/K$, 然后利用 $e,f$ 关于扩张的可乘性即可. 
\end{proof}

如果 $f_{L/K}=1$, 称之为\noun{完全分歧}, 即 $L\cap \wt K=K$.
如果 $e_{L/K}=1$, 称之为\noun{非分歧}, 即 $L\subseteq \wt K$.
此时,
  \[G(\wt K/K)\ra G(L/K)\]
是满射. 若 $f_K$ 有限, 称 $\varphi_K$ 的像 $\varphi_{L/K}$ 为 $L/K$ 的\noun{弗罗贝尼乌斯}.

\begin{definition}{亨泽尔赋值}{henselian valuation}
$A_k$ 的一个\noun{亨泽尔赋值}是指同态
  \[v:A_k\to \wh \BZ,\]
满足
\begin{enumerate}
\item $v(A_k)=Z\supseteq \BZ$ 和 $\BZ/n\BZ\cong Z/nZ, \forall n\in \BN$,
\item $v(\bfN_{K/k}A_K)=f_kZ$.
\end{enumerate}
\end{definition}


\begin{proposition}{}{}
对有限扩张 $K/k$, 
  \[v_K=\frac{1}{f_K}v\circ\bfN_{K/k}:A_K\to Z\]
定义了一个满同态, 且
\begin{enumerate}
\item $v_K=v_{K^\sigma}\circ \sigma,\forall \sigma\in G$,
\item 对于有限扩张 $L/K$,
  \[\xymatrix{
A_L\ar[r]^{v_L}\ar[d]_{\bfN_{L/K}} &\wh \BZ\ar[d]^{f_{L/K}}\\
A_K\ar[r]^{v_K} &\wh\BZ.
}\]
\end{enumerate}
\end{proposition}
\begin{proof}
(1) 设 $\tau$ 取遍 $G_k/G_K$ 的一组代表元, 则 $\sigma^{-1}\tau\sigma$ 取遍 $G_k/\sigma^{-1}G_K\sigma=G_k/G_{K^\sigma}$ 的一组代表元. 因此
  \[v_{K^\sigma}(a^\sigma)=\frac{1}{f_{K^\sigma}}v(\prod_\tau a^{\tau\sigma})=\frac{1}{f_K}v(\prod_\tau a^\tau)=v_K(a).\]

(2) 由范数关于域扩张 $L/K/k$ 的分解性得到.
\end{proof}

\begin{definition}{素元}{prime element}
$A_K$ 的\noun{素元}是指满足 $v_K(\pi_K)=1$ 的 $\pi_K\in A_K$. 记
  \[U_K=\set{u\in A_K\mid v_K(u)=0}.\]
\end{definition}

容易知道, 当 $L/K$ 非分歧时, $K$ 的素元仍然是 $L$ 的素元; 当 $L/K$ 完全分歧时, $L$ 素元的范数是 $K$ 的素元.


\subsection{互反映射}
对于抽象的伽罗瓦群 $G$ 和一个 $G$ 模 $A$, 我们固定了
  \[d:G\to \wh \BZ,\quad v:A_k\to \wh \BZ.\]
我们现在只考虑有限扩张 $K/k$.
我们要求 $G$ 和 $A$ 满足如下的\noun{类域论公理}:

\begin{axiom}{类域论公理}{class_field_axiom}
对于任意循环扩张 $L/K$,
  \[\#\rmH^0\bigl(G(L/K),A_L\bigr)=[L:K],\quad \rmH^{-1}\bigl(G(L/K),A_L\bigr)=1.\]
\end{axiom}

\begin{proposition}{}{}
对于任意非分歧扩张 $L/K$,
  \[\rmH^0\bigl(G(L/K),U_L\bigr)=\rmH^{-1}\bigl(G(L/K),U_L\bigr)=1.\]
\end{proposition}

\begin{proof}
由于 $L/K$ 非分歧, 因此 $\pi_K$ 是 $L$ 的素元. 由于 $\rmH^{-1}\bigl(G(L/K),A_L\bigr)=1$, 因此对于任意 $u\in U_L^{\bfN=1}$, 存在 $a\in A_L$ 使得 $u=a^{\sigma-1}$. 设 $a=\varepsilon \pi_K^m, \varepsilon\in U_L$, 则 $u=\varepsilon^{\sigma-1}$. 因此 $\rmH^{-1}\bigl(G(L/K),U_L\bigr)=1.$

满同态 $v_K:A_K\to Z$ 诱导了满同态
  \[v_K: A_K/\bfN A_L\to Z/nZ\cong \BZ/n\BZ,\quad n=f_{L/K}=[L:K].\]
由于 $\# A_K/\bfN A_L=n$, 因此这是一个同构. 对于 $u\in U_K, v_K(u)=0$, 因此存在 $a\in A_L$ 使得 $u=\bfN a$. 而 $v_K(u)=nv_L(a)$, 因此 $a\in U_L$, $\rmH^0\bigl(G(L/K),U_L\bigr)=1$.
\end{proof}

对于无限扩张 $L/K$, 我们令
  \[\bfN_{L/K}A_L:=\bigcap_M \bfN_{M/K}A_M,\]
其中 $M/K$ 取遍 $L/K$ 的所有有限子扩张. 
所谓的\noun{互反映射}指的是如下的一个典范同态
  \[r_{L/K}:G(L/K)\to A_K/\bfN_{L/K}A_L.\]
考虑 $\wt L/K$, 则 $G_{\wt L}=I_L\subseteq I_K$. 因此 $d_K:G_K\to\wh\BZ$ 诱导了 $d_K:G(\wt L/K)\to \wh \BZ$.
定义
  \[\Frob(\wt L/K)=\set{\sigma\in G(\wt L/K)\mid d_K(\sigma)\in \BN^+}.\]

\begin{theorem}{互反映射}{reciprocity map}
定义
  \[\fct{r_{\wt L/K}:\Frob(\wt L/K)}{A_K/\bfN_{\wt L/K}A_{\wt L}}{\sigma}{\bfN_{\Sigma/K}(\pi_\Sigma)\mod \bfN_{\wt L/K}A_{\wt L},}\]
其中 $\Sigma$ 是 $\sigma$ 的固定域, $\pi_\Sigma$ 是它的一个素元. 它可以下降为
  \[r_{L/K}:G(L/K)\to A_K/\bfN_{L/K}A_L.\]
\end{theorem}

\begin{lemma}{}{}
设 $\Sigma$ 是 $\sigma\in\Frob(\wt L/K)$ 的固定域, 则
  \[f_{\Sigma/K}=d_K(\sigma),\quad
[\Sigma:K]<\infty,\quad
\wt\Sigma=\wt L,\quad
\sigma=\varphi_\Sigma.\]
\end{lemma}
\begin{proof}
(1) 由于 $\Sigma\cap \wt K$ 是 $\sigma|_{\wt K}=\varphi_K^{d_K(\sigma)}$ 的固定域, 因此 
  \[f_{\Sigma/K}=[\Sigma\cap \wt K:K]=d_K(\sigma).\]

(2) 由 $\wt K\subseteq \Sigma \wt K=\wt \Sigma\subseteq \wt L$ 知
  \[e_{\Sigma/K}=(I_K:I_\Sigma)=\# G(\wt \Sigma/\wt K)\le \# G(\wt L/\wt K)\]
有限, 因此 $[\Sigma:K]=f_{\Sigma/K}e_{\Sigma/K}$ 有限.

(3) 由于 $\Gamma=G(\wt L/\Sigma)=\ov{\pair{\sigma}}$ 是射影循环群, $(\Gamma:\Gamma^n)\le n$. 因此满射 $\Gamma\to G(\wt \Sigma/\Sigma)\cong\wt \BZ$ 诱导了双射 $\Gamma/\Gamma^n\cong\wt \BZ/n\wt\BZ$, 从而 $\Gamma\cong G(\wt \Sigma/\Sigma),\wt \Sigma=\wt L$.

(4) 由于 $f_{\Sigma/K}d_\Sigma(\sigma)=d_K(\sigma)=f_{\Sigma/K}$, 因此 $d_\Sigma(\wt \sigma)=1,\sigma=\varphi_\Sigma$.
\end{proof}

\begin{proof}[定理~\ref{thm:reciprocity map}的证明]
先考虑有限伽罗瓦扩张.

\noindent{\bf 第一步}.
对于有限伽罗瓦扩张 $L/K$,
  \[\fct{\Frob(\wt L/K)}{G(L/K)}{\sigma}{\sigma|_L}\]
是满射. 设 $\varphi\in G(\wt L/K)$ 是 $\varphi_K$ 的一个提升, 即 $\varphi|_{\wt K}=\varphi_K$ 且 $d_K(\varphi)=1$. 由于 $L\cap\wt K/K$ 非分歧, 因此 $\varphi|_{L\cap \wt K}=\varphi_{L\cap \wt K/K}$. 对于任意 $\sigma\in G(L/K)$, $\sigma$ 在 $L\cap \wt K$ 上的限制是生成元 $\varphi_{L\cap \wt K/K}$ 的一个幂次, 不妨设 $\sigma|_{L\cap \wt K}=\varphi_{L\cap \wt K/K}^n,n\in\BN^+$. 由于 $\wt L=L\wt K$, 因此
  \[G(\wt L/\wt K)\cong G(L/L\cap \wt K).\]
设 $\tau\in G(\wt L/\wt K)$ 是 $\sigma\varphi^{-n}|_L$ 的原像, 则 $\wt\sigma=\tau\varphi^n$ 满足 $\wt\sigma|_L=\sigma$ 和 $\wt \sigma|_{\wt K}=\varphi_K^n$, 即 $d_K(\wt\sigma)=n$.

\noindent{\bf 第二步}.
$r_{\wt L/K}$ 不依赖于 $\pi_\Sigma$ 的选取.
设 $u\in U_\Sigma$, 对于任意 $\Sigma\subseteq M\subseteq \wt\Sigma=\wt L$, 由类域论公理, 存在 $\varepsilon\in U_M$ 使得 $u=\bfN_{M/\Sigma}(\varepsilon)$, 因此 $\bfN_{\Sigma/K}(u)=\bfN_{M/K}(\varepsilon)\in\bfN_{M/K}A_M$. 从而 $\bfN_{\Sigma/K}(u)\in\bfN_{\wt L/K}A_{\wt L}$, 因此 $r_{\wt L/K}$ 不依赖于素元的选取.

\noindent{\bf 第三步}.
$r_{\wt L/K}$ 具有可乘性. 对于任意 $\varphi\in G(\wt L/K)$, 考虑自同态
  \[\varphi-1:A_{\wt L}\to A_{\wt L},\quad a\mapsto a^\varphi/a,\]
  \[\varphi_n:A_{\wt L}\to A_{\wt L},\quad a\mapsto \prod_{i=0}^{n-1}a^{\varphi^i}.\]
显然 $(\varphi-1)\circ\varphi_n=\varphi_n\circ(\varphi-1)=\varphi^n-1$.
设 $\sigma_1=\sigma_2\sigma_3\in\Frob(\wt L/K)$, $n_i=d_K(\wt \sigma_i)$, $\Sigma_i$ 为 $\sigma_i$ 的固定域, $\pi_i\in A_{\Sigma_i}$ 是素元. 固定 $\varphi\in G(\wt L/K)$ 使得 $d_K(\varphi)=1$, 令 $\tau_i=\sigma_i^{-1}\varphi^{n_i}\in G(\wt L/\wt K)$. 我们知道 $n_3=n_1+n_2$,
  \[\tau_3=\sigma_2^{-1}\sigma_1^{-1}\varphi^{n_1+n_2}=(\sigma_2^{-1}\varphi^{n_2})(\varphi^{-n_2}\sigma_1\varphi^{n_2})^{-1}\varphi^{n_1}.\]
令 $\sigma_4=\varphi^{-n_2}\sigma_1\varphi^{n_2}$, $n_4=d_K(\sigma_4)=n_1$, $\Sigma_4=\Sigma_1^{\varphi^{n_2}},\pi_4=\pi_1^{\varphi^{n_2}}\in A_{\Sigma_4}$, $\tau_4=\sigma_4^{-1}\varphi^{n_4}$, 则 $\tau_3=\tau_2\tau_4$ 且 $\bfN_{\Sigma_4/K}(\pi_4)=\bfN_{\Sigma_1/K}(\pi_1)$.
我们只需证明
  \[\bfN_{\Sigma_3/K}(\pi_3)\equiv \bfN_{\Sigma_2/K}(\pi_2)\bfN_{\Sigma_4/K}(\pi_4)\mod \bfN_{\wt L/K}A_{\wt L}.\]

\begin{lemma}{}{}
设 $\varphi,\sigma\in\Frob(\wt L/K)$, $d_K(\varphi)=1,d_K(\sigma)=n$. 若 $\Sigma$ 是 $\sigma$ 的固定域, $a\in A_{\Sigma}$, 则
  \[\bfN_{\Sigma/K}(a)=(\bfN\circ \varphi_n)(a)=(\varphi_n\circ \bfN)(a),\]
其中 $\bfN=\bfN_{\wt L/\wt K}.$
\end{lemma}
\begin{proof}
$K$ 在 $\Sigma$ 的极大非分歧扩张 $\Sigma^0=\wt K\cap \sigma/K$ 的扩张次数为 $n$, 伽罗瓦群 $G(\Sigma^0/K)$ 由 $\varphi_{\Sigma^0/K}=\varphi_K|_{\Sigma^0}=\varphi|_{\Sigma^0}.$ 因此 $\bfN_{\Sigma^0/K}=\varphi_n|_{A_{\Sigma^0}}$. 又因为 $\Sigma \wt K=\wt \Sigma,\Sigma\cap \wt K=\Sigma^0$, $\bfN_{\Sigma/\Sigma^0}=\bfN|_{A_\Sigma}.$ 对于 $a\in A_{\Sigma}$, 
  \[\bfN_{\Sigma/K}(a)=\bfN_{\Sigma^0/K}\bigl(\bfN_{\Sigma/\Sigma^0}(a)\bigr)=\bfN(a)^{\varphi_n}=\bfN(a^{\varphi_n}).\]
最后由 $\varphi$ 正规化 $G(\wt L/\wt K)$ 可知第二个等式成立.
\end{proof}

现在我们知道 $\bfN_{\Sigma_i/K}(\pi_i)=\bfN(\pi_i^{\varphi_{n_i}})$. 因此我们只需证明 $\bfN(u)\in\bfN_{\wt L/K}A_{\wt L}$, $u=\pi_3^{\varphi_{n_3}}\pi_4^{-\varphi_{n_4}}\pi_2^{-\varphi_{n_2}}$.
注意到 $I_{G(\wt L/\wt K)}U_{\wt L}$ 中的元素在 $\bfN$ 下的像为 $1$, 因此 $\bfN$ 诱导了同态 $\rmH_0\bigl(G(\wt L/\wt K),U_{\wt L}\bigr)\to U_{\wt K}$.

\begin{lemma}{}{}
如果 $x\in \rmH_0\bigl(G(\wt L/\wt K),U_{\wt L}\bigr)$ 被 $\varphi\in G(\wt L/K)$ 固定, 且 $d_K(\varphi)=1$, 则 $\bfN(x)\in\bfN_{\wt L/K}U_{\wt L}$.
\end{lemma}
\begin{proof}
设 $x=u\mod I_{G(\wt L/\wt K)}U_{\wt L}$, 则
  \[u^{\varphi-1}=\prod_{i=1}^r u_i^{\tau_i-1}, \quad \tau_i\in G(\wt L/\wt K),u_i\in U_{\wt L}.\]
设 $M/K$ 是 $\wt L/K$ 的有限伽罗瓦子扩张, 不妨设 $u,u_i\in M$ 且 $L\subseteq M$. 设 $n=[M:K],\sigma=\varphi^n\in G(\wt L/M)$, $\Sigma$ 是 $\sigma$ 的固定域, $\Sigma_n$ 是 $\sigma^n=\varphi_\Sigma^n$ 的固定域, 则 $\Sigma_n/\Sigma$ 是 $n$ 次非分歧扩张. 由类域论公理, 存在 $\wt u,\wt u_i\in U_{\Sigma_n}$ 使得
  \[u=\bfN_{\Sigma_n/\Sigma}(\wt u)=\wt u^{\sigma_n},\quad
u_i=\bfN_{\Sigma_n/\Sigma}(\wt u_i)=\wt u_i^{\sigma_n}.\]
因此 $\wt u^{\varphi-1}$ 和 $\prod_i \wt u_i^{\varphi-1}$ 仅相差一个 $\wt x\in U_{\Sigma_n}$ 满足 $\bfN_{\Sigma_n/\Sigma}(\wt x)=1$. 再次由类域论公理, 存在 $\wt y\in U_{\Sigma_n},\wt x=\wt y^{\sigma-1}=\wt y^{\varphi_n(\varphi -1)}$. 从而
  \[\bfN(\wt u)^{\varphi -1}=\bfN(\wt y^{\varphi_n})^{\varphi-1},\quad
\bfN(\wt u)=\bfN(\wt y^{\varphi_n})z,\]
其中 $z\in U_{\wt K},z^{\varphi-1}=1$, 即 $z\in U_K$. 设 $y=\wt y^{\sigma_n}=\bfN_{\Sigma_n/\Sigma}(\wt y)\in U_\Sigma$,
  \[\bfN(u)=\bfN(\wt u)^{\sigma_n}=\bfN(\wt y^{\varphi_n})^{\sigma_n}z^{\sigma_n}=\bfN(y^{\varphi_n})z^n=\bfN_{\Sigma/K}(y)\bfN_{M/K}(z)\]
属于 $\bfN_{M/K}U_M$.
\end{proof}

现在返回到原命题. 由于 $\varphi_{n_i}\circ(\varphi -1)=\varphi^{n_i}-1$, $\pi_i^{\varphi_{n_i}-1}=\pi_i^{\tau_i-1}$, 因此
  \[u^{\varphi-1}=\pi_3^{\tau_3-1}\pi_4^{1-\tau_4}\pi_2^{1-\tau_2}.\]
由于 $\tau_3=\tau_2\tau_4$ 知道 $(\tau_3-1)+(1-\tau_2)+(1-\tau_4)=(1-\tau_2)(1-\tau_4)$. 设
  \[\pi_3=u_3\pi_4,\quad \pi_2=u_2^{-1}\pi_4,\quad \pi_4^{\tau_2}=u_4\pi_4,\]
则 $u^{\varphi-1}=\prod_{i=2}^4 u_i^{\tau_i-1}$, 从而 $\bfN(u)\in \bfN_{\wt L/K}A_{\wt L}.$

\noindent{\bf 第四步}.
互反映射 $r_{L/K}$ 良定.
由于 $\Frob(\wt L/K)\to G(L/K)$ 是满射, 且 $\bfN_{\wt L/K}A_{\wt L}\subseteq \bfN_{L/K}A_L$, 因此
$r_{\wt L/K}:\Frob(\wt L/K)\to A_K/\bfN_{\wt L/K}A_{\wt L}$ 可以下降为
  \[r_{L/K}:G(L/K)\to A_K/\bfN_{L/K}A_L.\]
我们只需说明该定义不依赖于 $\sigma\in G(L/K)$ 在 $\Frob(\wt L/K)$ 中提升 $\wt \sigma$ 的选取. 设 $\wt\sigma'$ 也提升 $\sigma$, $\Sigma,\pi_{\Sigma'}\in A_{\Sigma'}$ 是素元. 如果 $d_K(\wt\sigma)=d_K(\wt \sigma')$, 则二者在 $\wt K$ 和 $L$ 上均相同, 从而二者相同. 如果 $d_K(\wt \sigma)<d_K(\wt\sigma)$, 则存在 $\wt\tau\in\Frob(\wt L/K)$ 使得 $\wt\sigma'=\wt\sigma\wt\tau$ 且 $\wt\tau|_L=1$. 因此 $\wt \tau$ 的固定域 $\Sigma''$ 包含 $L$, 从而 $r_{\wt L/K}(\wt\tau)\equiv \bfN_{\Sigma''/K}(\pi_{\Sigma''})\equiv 1\mod\bfN_{L/K}A_L$, 因此 $r_{\wt L/K}(\wt\sigma')=r_{\wt L/K}(\wt\sigma)$.

\noindent{\bf 第五步}.
根据定义不难证明互反映射满足下列函子性质.
从而可知命题对于无限伽罗瓦扩张 $L/K$ 也成立.
\end{proof}

\begin{proposition}{}{}
对于有限伽罗瓦扩张 $L/K,L'/K'$, 如果 $K\subseteq K',L\subseteq L'$, $\sigma\in G(L/K)$, 则下列图表交换
  \[\xymatrix{
G(L'/K')\ar[rr]^{r_{L'/K'}} \ar[d]&& A_{K'}/\bfN_{L'/K'}A_{L'}\ar[d]^{\bfN_{K'/K}}\\
G(L/K)\ar[rr]^{r_{L/K}}&&A_K/\bfN_{L/K}A_L,
}\]
其中左竖直箭头为限制在 $L$ 上, 
  \[\xymatrix{
    G(L/K)\ar[rr]^{r_{L/K}} \ar[d]&& A_K/\bfN_{L/K}A_L\ar[d]^{\sigma}\\
    G(L^\sigma/K^\sigma)\ar[rr]^{r_{L^\sigma/K^\sigma}}&&A_{K^\sigma}/\bfN_{L^\sigma/K^\sigma}A_{L^\sigma}
  }\]
其中左竖直箭头为 $\sigma$ 共轭作用. 
\end{proposition}

如果 $L=L'$, 则有自然的嵌入 $A_K/\bfN_{L/K}A_L\inj A_{K'}/\bfN_{L/K'}A_L$, 它在伽罗瓦群的反映为\noun{变换映射}(Verlagerung). 设 $H$ 是 $G$ 的指标有限的子群, 则我们可以定义典范的同态
  \[\Ver:G^\ab\to H^\ab.\]
对于 $\sigma\in G$, 我们有双陪集分解
  \[G=\bigsqcup_{\tau}\pair{\sigma}\tau H.\]
对于任意 $\tau$, 设 $f(\tau)$ 是最小的正整数使得 $\sigma_\tau=(\tau^{-1}\sigma\tau)^{f(\tau)}\in H$, 定义
  \[\Ver(\sigma\mod G')=\prod_{\tau}\sigma_\tau\mod H'.\]

\begin{exercise}
证明 $\Ver$ 是一个良定义的同态.
\end{exercise}

\begin{proposition}{}{}
对于有限伽罗瓦扩张 $L/K$ 的中间域 $K'$, 我们有交换图表
  \[\xymatrix{
G(L/K')^\ab\ar[rr]^{r_{L/K'}}&& A_{K'}/\bfN_{L/K'}A_L\\
G(L/K)^\ab\ar[u]^{\Ver}\ar[rr]^{r_{L/K}}&&A_K/\bfN_{L/K}A_L\ar[u],
}\]
其中右侧由嵌入诱导.
\end{proposition}
\begin{proof}
我们暂时记 $G=G(\wt L/K),H=G(\wt L/K')$.
设 $\sigma\in G(L/K)$, $\wt \sigma\in\Frob(\wt L/K)$ 是 $\sigma$ 的原像, $\Sigma$ 为固定域, $S=G(\wt L/\Sigma)=\ov{\pair{\wt \sigma}}$. 考虑双陪集分解
  \[G=\bigsqcup_\tau S\tau H.\]
令 $S_\tau=\tau^{-1}S\tau\cap H,\wt\sigma_\tau=\tau^{-1}\wt\sigma^{f(\tau)}\tau$. 令
  \[\ov G=G(L/K),\ \ov H=G(L/K'),\ \ov S=\pair{\sigma},\ 
\ov\tau=\tau|_L,\ \sigma_\tau=\wt\sigma_\tau|_L,\]
则显然有双陪集分解
  \[\ov G=\bigsqcup_\tau \ov S\ov\tau \ov H.\]
因此
  \[\Ver(\sigma\mod G(L/K)')=\prod_\tau\sigma_\tau\mod G(L/K')'.\]
对于任意 $\tau$, 设有陪集分解
  \[H=\bigsqcup_{\omega_\tau}S_\tau\omega_\tau,\quad
G=\bigsqcup_{\tau,\omega_\tau}S\tau\omega_\tau.\]
设  $\Sigma_\tau$ 是 $\wt\sigma_\tau$ 的固定域, $\Sigma^\tau$ 是 $\tau^{-1}\wt \sigma\tau$ 的固定域, 则 $\Sigma_\tau/\Sigma^\tau$ 是 $f(\tau)$ 次非分歧扩张. 如果 $\pi\in A_\Sigma$ 是 $\Sigma$ 的素元, 则 $\pi^\tau\in A_{\Sigma^\tau}$ 是 $\Sigma^\tau$ 的素元, 因此也是 $\Sigma_\tau$ 的素元. 从而,
  \[\bfN_{\Sigma/K}(\pi)=\prod_{\tau,\omega_\tau}\pi^{\tau\omega_\tau}=\prod_\tau\bfN_{\Sigma_\tau/K'}(\pi^\tau),\]
  \[r_{L/K}(\sigma)\equiv \prod_\tau r_{L/K'}(\sigma_\tau)
\equiv r_{L/K'}(\prod_\tau\sigma_\tau)\equiv r_{L/K'}\Bigl(\Ver\bigl(\sigma\mod G(L/K')'\bigr)\Bigr).\]
\end{proof}


\begin{proposition}{}{}
如果 $L/K$ 非分歧, 则 $r_{L/K}$ 由 $r_{L/K}(\varphi_{L/K})=\pi_K\mod\bfN_{L/K}A_L$ 确定, 此时 $r_{L/K}$ 是同构.
\end{proposition}
\begin{proof}
此时 $\wt L=\wt K$, $\varphi_K\in G(\wt K/K)$ 是 $\varphi_{L/K}$ 的原像, 固定域为 $K$, 因此 $r_{L/K}$ 由此给出. 赋值 $v_K$ 诱导了同构 $A_K/\bfN_{L/K}A_L\simto Z/nZ\cong \BZ/n\BZ$, 这是因为对于 $v_K(a)\equiv \mod nZ$, $a=u\pi_K^{dn}$, 而 $u=\bfN_{L/K}(\varepsilon),\varepsilon\in U_L$, 从而 $a=\bfN_{L/K}(\varepsilon \pi_K^d)$. 最后由于生成元 $\varphi_{L/K}$ 的像为素元 $\pi_K$, 即生成元相互对应, 从而 $r_{L/K}$ 是同构.
\end{proof}

\subsection{互反律}
类域论的主定理如下所述:

\begin{theorem}{互反律}{reciprocity_map}
\index{互反律}
对于任意有限伽罗瓦扩张 $L/K$, 互反映射
  \[r_{L/K}:G(L/K)^\ab\to A_L/\bfN_{L/K}A_K\]
是同构.
\end{theorem}
\begin{proof}
设 $M/K$ 是 $L/K$ 的伽罗瓦子扩张, 则我们有交换图表
  \[\xymatrix{
    1\ar[r]&G(L/M)\ar[d]^{r_{L/M}}\ar[r]&G(L/K)\ar[d]^{r_{L/K}}\ar[r]&G(M/K)\ar[d]^{r_{M/K}}\ar[r]&1\\
           &A_M/\bfN_{L/M}A_L\ar[r]^{\bfN_{M/K}}&A_K/\bfN_{L/K}A_L\ar[r]&A_K/\bfN_{M/K}A_M\ar[r]&1.
  }\]

\noindent{\bf 第一步}.
约化到 $G(L/K)$ 是交换群情形. 若此时已成立, 则取 $M=L^\ab/K$ 为其极大阿贝尔子扩张, 即 $G(M/K)=G(L/K)^\ab$. 此时 $G(L/M)$ 是 $r_{L/K}$ 的核, 因此 $r_{L/K}$ 是单射. 要证明满射, 我们对次数进行归纳. 当 $G(L/K)$ 是可解群时, $M=L$ 或 $[L:M]<[L:K]$, 如果 $r_{L/M}$ 和 $r_{M/K}$ 都是满射, 则 $r_{L/K}$ 也是满射. 一般情形下, 设 $M$ 是 $G(L/K)$ 的一个希洛夫 $p$ 子群的固定域, 则我们只需证明 $\bfN_{M/K}$ 的像是 $A_K/\bfN_{L/K}A_L$ 的希洛夫 $p$ 子群 $S_p$ 即可, 这诱导了 $r_{L/K}$ 是满射. 嵌入 $A_K\inj A_M$ 诱导了
  \[i:A_K/\bfN_{L/K}A_L\to A_M/\bfN_{L/M}A_L,\] 
他满足 $\bfN_{M/K}\circ i=[M:K]$. 由于 $p\nmid[M:K]$, 因此 $[M:K]:S_p\ra S_p$ 是同构, 从而 $S_p$ 落在 $\bfN_{M/K}$ 的像, 因此 $r_{L/K}$ 是满射.

\noindent{\bf 第二步}.
约化到循环扩张情形. 若 $M/K$ 取遍所有循环子扩张, $r_{L/K}$ 的核落在 $G(L/K)\to\prod G(M/K)$ 的核中. 由于 $G(L/K)$ 是交换群, 该映射是单的, 因此 $r_{L/K}$ 是单射. 由于此时 $G(L/K)$ 是可解的, 因此对次数进行和第一步相同的归纳可知 $r_{L/K}$ 是满射.

\noindent{\bf 第三步}.
设 $L/K$ 是循环扩张. 不妨设 $f_{L/K}=1$. 实际上, 设 $M=L\cap \wt K$, 则 $f_{L/M}=1, r_{M/K}$ 是同构. 由类域论公理可知图表中下面一行的群的阶, 从而它是正合的, 因此 $r_{L/M}$ 是同构蕴含 $r_{L/K}$ 是同构.

设 $L/K$ 是循环完全分歧扩张. 设 $G(L/K)=\pair{\sigma}$. 我们把 $\sigma$ 在 $G(L/K)\cong G(\wt L/\wt K)$ 下的像仍记为 $\sigma$, 则 $\wt \sigma=\sigma\varphi_L\in\Frob(\wt L/K)$ 是 $\sigma$ 的一个原像, 且 $d_K(\wt \sigma)=1=f_{L/K}$. 设 $\Sigma$ 是 $\wt \sigma$ 的固定域, 则 $f_{\Sigma/K}=1$, 因此 $\Sigma\cap \wt K=K$. 设 $M/K$ 是 $\wt L/K$ 的伽罗瓦子扩张, 包含 $\Sigma L$, $M^0=M\cap\wt K$. 令 $\bfN=\bfN_{M/M^0}$, 则 $\bfN_{A_\Sigma}=\bfN_{\Sigma/K},\bfN_{A_L}=\bfN_{L/K}.$

设 $r_{L/K}(\sigma^k)=1$, $0\le k<n=[L:K]$. 由于 $\pi_\Sigma$ 和 $\pi_L$ 都是 $M$ 的素元, 因此 $\pi_\Sigma^k=u\pi_L^k$, $u\in U_M$, 于是
  \[r_{L/K}(\sigma^k)\equiv \bfN(\pi_\Sigma^k)\equiv \bfN(u)\bfN(\pi_L^k)\equiv \bfN(u)\mod \bfN_{L/K}A_L.\]
由 $r_{L/K}(\sigma^k)=1$ 知存在 $v\in U_L$ 使得 $\bfN(uv^{-1})=1$, $uv^{-1}=a^{\sigma-1},a\in A_M$, 从而
  \[(\pi_L^k v)^{\sigma-1}=(\pi_L^k v)^{\wt \sigma-1}=(\pi_\Sigma^k u^{-1}v)^{\wt \sigma-1}=(a^{\sigma-1})^{\wt \sigma-1}=(a^{\wt \sigma-1})^{\sigma-1},\] 
因此 $x=\pi_L^k v a^{1-\wt \sigma}\in A_{M^0}$. 由于 $nv_{M^0}(x)=v_M(x)=k$, 因此 $k=0$, $r_{L/K}$ 是单射. 由类域论公理知 $\#A_K/\bfN_{L/K}A_L=n$, 因此 $r_{L/K}$ 是同构.
\end{proof}

我们考虑互反映射的逆诱导的同态
  \[(~,L/K):A_K\to G(L/K)^\ab,\]
它的核是 $\bfN_{L/K}A_L$, 我们称该映射为\noun{范剩余符号}. 由互反映射的函子性我们自然有范剩余符号的函子性. 对于无限伽罗瓦扩张 $L/K$, 
  \[G(L/K)^\ab=\plim_i G(L_i/K)^\ab,\]
其中 $L_i/K$ 取遍所有有限子扩张. 由于 $(a,L'/K)|_{L^\ab}=(a,L/K)$, 因此这定义出 $G(L/K)^\ab$ 的一个元素, 即此时也有范剩余符号.

\begin{proposition}{}{}
我们有
  \[(a,\wt K/K)=\varphi_K^{v_K(a)},\quad d_K\circ (~~,\wt K/K)=v_K.\]
\end{proposition}
\begin{proof}
考虑它的有限 $f$ 次子扩张 $L/K$, 则 $v_K(a)=n+fz,n\in\BZ,z\in Z$, 即 $a=u\pi_K^nb^f,u\in U_K,b\in A_K$. 于是
  \[(a,\wt K/K)|_L=(a,L/K)=(\pi_K,L/K)^n(b,L/K)^f=\varphi_{L/K}^n=\varphi_K^{v_K(a)}|_L.\]
从而 $(a,\wt K/K)=\varphi_K^{v_K(a)}$.
\end{proof}

对于任意 $K$, 我们赋予 $A_K$ 拓扑为 $\bfN_{L/K}A_L$ 形成单位元的邻域基, 其中 $L/K$ 取遍有限伽罗瓦扩张, 我们称之为\noun{范数拓扑}.

\begin{proposition}{}{}
(1) $A_K$ 的开子群是它的有限指标闭子群.

(2) $v_K:A_K\to\wh \BZ$ 是连续的.

(3) 对于有限扩张 $L/K$, $\bfN_{L/K}:A_L\to A_K$ 是连续的.

(4) $A_K$ 是豪斯多夫的当且仅当 $A_K^0=\cap_L \bfN_{L/K}A_L=0$.
\end{proposition}
\begin{proof}
(1) 根据习题~\ref{exe:closed_open_subgroups}, 我们只需说明开子群是有限指标的, 而这由它包含某个 $\bfN_{L/K}A_L$ 可知.

(2) $f\wh \BZ,f\ge 1$ 形成 $0\in\wh \BZ$ 的领域基. 设 $L/K$ 是 $f$ 次非分歧扩张, 则 $v_K(\bfN_{L/K}A_L)=f v_L(A_L)\subset f\wh \BZ$, 因此 $v_K$ 连续.

(3) 由于 $\bfN_{M/K}A_M$ 形成 $a\in A_K$ 的领域基, 而它的原像包含 $\bfN_{M/L}A_M$, 因此 $\bfN_{L/K}$ 连续.

(4) 显然.
\end{proof}

通过互反律, 我们可以给出 $K$ 的有限阿贝尔扩张的一种刻画.

\begin{theorem}{}{abelian_extension_correspondence}
映射
  \[L\mapsto \CN_L=\bfN_{L/K}A_L\]
给出了有限阿贝尔扩张 $L/K$ 和 $A_K$ 的开子群间的一一对应.
\end{theorem}
\begin{proof}
证明是比较直接的, 见\cite[Theorem~4.6.7]{Neukirch1999}.
\end{proof}
我们称 $A_K$ 的开子群 $\CN$ 对应 $L$ 为其对应的\noun{类域}, 则
  \[G(L/K)\cong A_K/\CN.\]


\section{局部类域论}
\label{sec:local class field theory}
\subsection{局部互反律}

对于任意域 $k$, $G=G(k^\sep/k)$, $A=(k^\sep)^\times$, 我们都有如下结论:
\begin{theorem}{希尔伯特 90}{}
对于有限循环扩张 $L/K$, $\rmH^{-1}\bigl(G(L/K),L^\times\bigr)=1$.
\end{theorem}

\begin{proposition}{诺特}{first_cohom_trivial}
对于有限伽罗瓦扩张 $L/K$, $\rmH^1\bigl(G(L/K),L^\times\bigr)=1$.
\end{proposition}
由于循环扩张情形, $\rmH^1=\rmH^{-1}$, 因此这蕴含希尔伯特 90.
\begin{proof}
设 $f:G\to L^\times$ 是一个 $1$ 余循环. 对于 $c\in L^\times$, 令
  \[\alpha=\sum_{\sigma\in G(L/K)} f(\sigma) c^\sigma.\]
由 $1,\sigma,\dots,\sigma^{n-1}$ 的线性无关性~\ref{pro:independent_of_embeddings}, 存在 $c\in L^\times$ 使得 $\alpha\neq 0$. 我们有
  \[\alpha^\tau=\sum_\sigma f(\sigma)^\tau c^{\sigma\tau}=\sum_\sigma f(\tau)^{-1}f(\sigma\tau)c^{\sigma\tau}=f(\tau)^{-1}\alpha,\]
因此 $f(\sigma)=(\alpha^{-1})^{\sigma-1}$ 是 $1$ 余边界.
\end{proof}

设 $k$ 是 $\BQ_p$ 的有限扩张.
\begin{theorem}{}{}
设 $G=G(\ov k/k)$, $A=\ov k^\times$, 则它们满足类域论公理~\ref{axi:class_field_axiom}.
\end{theorem}
\begin{proof}
$\rmH^{-1}=1$ 由希尔伯特 90 得到.
设 $G=G(L/K)$, 则我们有 $G$ 模正合列
  \[0\ra U_L\to L^\times\to \BZ\to 0,\]
于是
  \[h(G,L^\times)=h(G,\BZ)h(G,U_L)=[L:K]h(G,U_L).\]
选取 $L/K$ 的一组正规基 $\set{a^\sigma\mid \sigma\in G}$, $\alpha\in\CO_L$. 设 $M=\suml_{\sigma\in G}\CO_K \alpha^\sigma\subseteq \CO_L$. 则 
  \[V^n=1+\pi_K^n M,\quad n=1,2,\dots\]
形成了 $1\in U_L$ 的一组邻域基. 由于 $M$ 是开子群, 因此存在 $N$ 使得 $\pi_K^N\CO_L\subseteq M$, 于是对于 $n\ge N$,
  \[(\pi_K^nM)(\pi_K^n M)=\pi_K^{2n}MM\subseteq \pi_K^{2n}\CO_L\subseteq\pi_K^{2n-N}M\subseteq \pi_K^n M,\]
即 $V^nV^n\subseteq V^n$. 显然 $V^n$ 包含其中元素的逆. 我们有 $G$ 模同构
  \[\begin{split}
V^n/V^{n+1}&\simto M/\pi_K M\\
1+\pi_K^n \alpha&\mapsto \alpha\mod \pi_K M.
\end{split}\]
而 
  \[M/\pi_K M=\bigoplus_{\sigma \in G} (\CO_K/\pi_K) \alpha^\sigma=\Ind_G^1(\CO_K/\pi_K),\]
由命题~\ref{pro:cohomology_of_induced_modules}, $\rmH^i(G,V^n/V^{n+1})=1,i=0,-1$. 设 $a\in (V^n)^G$, 则存在 $b_0\in V^n,a_1\in(V^{n+1})^G$ 使得 $a=(\bfN b_0)b_1$. 归纳地, 我们得到一串 $b_i\in V^{n+i}$ 使得 $a=\bfN(\prod_{i=0}^\infty b_i)$, 这里这个乘积是收敛的. 因此 $\rmH^0(G,V^n)=1$. 同理可得 $\rmH^{-1}(G,V^n)=1$, 因此 $h(G,V^n)=1$. 而 $U_L/V^n$ 有限, 因此
  \[h(G,U_L)=h(G,U_L/V^n)h(G,V^n)=1.\]
\end{proof}




设 $\kappa$ 为 $k$ 的剩余域, $\wt k$ 为 $k$ 的极大非分歧扩张, 则
  \[G(\wt k/k)\cong G(\bar \kappa/\kappa)\cong \wh\BZ.\]
设 $q=\# \kappa$, 则 $1\in\wh \BZ$ 对应弗罗贝尼乌斯 $x\mapsto x^q$. 于是 $\varphi_k\in G(\wt k/k)$ 由
  \[a^{\varphi_k}\equiv a^q\mod\fp_{\wt k},\quad a\in\CO_{\wt k}\]
决定. 我们有自然的满同态
  \[d:G\to\wh\BZ.\]
设 $v_K:K^\times\to\BZ$ 为 $K$ 的归一化赋值. 对于任意有限扩张 $K/k$, $\frac{1}{e_K}v_K$ 是 $v_k$ 在 $K^\times$ 上的延拓. 我们有
  \[\frac{1}{e_K}v_K=\frac{1}{[K:k]}v_k\circ\bfN_{K/k},\]
因此 $v_K(\bfN_{K/k}K^\times)=f_K v_K(K^\times)=f_K\BZ$, 即 $v_k$ 是一个亨泽尔赋值,
  \[d:G\to\wh\BZ,\quad v_k:k^\times\to\BZ\]
满足类域论的假设.
因此我们有局部域的互反律定理~\ref{thm:reciprocity_map}和阿贝尔扩张刻画定理~\ref{thm:abelian_extension_correspondence}.

\begin{theorem}{}{}
对于任意局部域的有限伽罗瓦扩张,
  \[r_{L/K}:G(L/K)^\ab\to L^\times/\bfN_{L/K}K^\times\]
是同构.
\end{theorem}
\begin{theorem}{}{}
  \[L\mapsto \CN_L=\bfN_{L/K}L^\times\]
给出了局部域的有限阿贝尔扩张 $L/K$ 和 $K^\times$ 的有限指标开子群间的一一对应.
\end{theorem}
\begin{proof}
这里需要证明赋值给出的拓扑中有限指标开子群和范数拓扑的开子群一致, 见\cite[Theorem~5.1.4]{Neukirch1999}. 由范数拓扑下开子群包含 $\bfN_{L/K}U_L=U_K$ 知道它是开的, 显然它是有限指标的. 反之, 我们需要考虑库默尔理论.

设 $\wp:A\to A$ 是 $G$ 模满同态, 核 $\mu_\wp$ 为 $n$ 阶循环群. 在我们的情形, $a^\wp:=a^n$ 即可. 设 $K\supseteq \mu_\wp$. 对于集合 $B\subseteq A$, 记 $K(B)$ 为
  \[H=\set{\sigma\in G_K\mid \sigma b=b,\forall b\in B}\]
的固定域. 显然如果 $B$ 是 $G_K$ 不变的, $K(B)/K$ 是伽罗瓦的. \noun{库默尔扩张}是指形如 $K\bigl(\wp^{-1}(\Delta)\bigr)/K$ 的扩张, 其中 $\Delta\subseteq A_K$.

\begin{proposition}{}{}
库默尔扩张和 $K$ 的指数为 $n$ 的阿贝尔扩张一一对应. 设 $L/K$ 是指数为 $n$ 的阿贝尔扩张, 则
  \[L=K\bigl(\wp^{-1}(\Delta)\bigr),\quad \Delta=
A_L^\wp\cap A_K.\]
特别地, 如果 $L/K$ 是循环扩张, 则存在 $\alpha^\wp\in A_K$ 使得 $L=K(\alpha)$.
\end{proposition}
\begin{proof}
我们有单射
  \[G\Bigl(K\bigl(\wp^{-1}(a)\bigr)/K\Bigr)\inj \mu_\wp
\quad \sigma\mapsto \alpha^{\sigma-1},\]
其中 $\wp(\alpha)=a$. 由于 $\mu_\wp\subseteq A_K$, 因此该映射不依赖于 $\alpha$ 的选取. 从而
  \[G(L/K)\inj \prod_{a\in\Delta}G\Bigl(K\bigl(\wp^{-1}(a)\bigr)/K\Bigr)\inj \mu_\wp^\Delta\]
是单射.

反之我们有 $\wp^{-1}(\Delta)\subseteq A_L$. 由于 $L/K$ 是它循环子扩张的合成, 考虑 $M/K$ 是其中之一. 我们只需证 $M\subseteq K\bigl(\wp^{-1}(\Delta)\bigr)$. 设 $\sigma$ 生成 $G(M/K)$, $\zeta_\wp$ 生成 $\mu_\wp$, $d=[M:K],d'=n/d,\xi=\zeta^{n/d}$. 由于 $\bfN_{M/K}(\xi)=\xi^d=1$, 由类域论公理存在 $\alpha\in A_M$ 使得 $\xi=\alpha^{\sigma-1}$, 因此 $K\subseteq K(\alpha)\subseteq M$. 由于 $\alpha^{\sigma^i}=\xi^i\alpha$, 因此 $\alpha^{\sigma^i}=\alpha$ 当且仅当 $d\mid i$, 所以 $M=K(\alpha)$. 最后 $(\alpha^\wp)^{\sigma-1}=\xi^\wp=1$, $a=\alpha^\wp\in A_K$, 即 $\alpha\in\wp^{-1}(\Delta)$, 从而 $M\subseteq K\bigl(\wp^{-1}(\Delta)\bigr)$.
\end{proof}
\begin{theorem}{}{}
映射
  \[\Delta\mapsto L=K\bigl(\wp^{-1}(\Delta)\bigr)\]
给出了群 $A_K^\wp\subseteq \Delta\subseteq A_K$ 和指数为 $n$ 的阿贝尔扩张 $L/K$ 的一一对应, 其中 $\Delta=A_L^\wp\cap A_K$. 此时,
  \[\Delta/A_K^\wp\cong\Hom(G(L/K),\mu_\wp),\quad a\mod A_K^\wp\mapsto \chi_a,\]
其中 $\chi_a(\sigma)=\alpha^{\sigma-1},\alpha\in\wp^{-1}(a)$.
\end{theorem}
\begin{proof}
考虑
  \[\Delta\to \Hom(L/K,\mu_\wp),\quad a\mapsto \chi_a.\]
$\chi_a=1$ 当且仅当 $\alpha^{\sigma-1}=1,\forall \sigma$, 从而 $\alpha\in A_K,a\in A_K^\wp$. 因此定理中映射为单射. 要证明满射, 任一 $\chi$ 的核的固定域 $M/K$ 是 $d$ 次循环扩张, 且诱导了 $\ov\chi:G(M/K)\to \mu_\wp$. 设 $\sigma$ 生成 $G(M/K)$, 则 $\bfN_{M/K}\bigl(\ov\chi(\sigma)\bigr)=\ov\chi(\sigma)^d=1$, 从而存在 $\alpha\in A_M,\ov\chi(\sigma)=\alpha^{\sigma-1}$. 很明显, $a=\alpha^\wp\in \Delta=A_L^\wp\cap A_K$. 对于 $\tau\in G(L/K)$, $\chi(\tau)=\ov\chi(\tau|_M)=\alpha^{\tau-1}=\chi_a(\tau)$, 即 $\chi=\chi_a$. 从而
  \[\Delta/A_K^\wp\cong\Hom(G(L/K),\mu_\wp).\]
根据这个对应可知, 如果 $\Delta$ 对应 $L$, 则它的大小是确定的, 

最后我们来说明一一对应. 设 $A_K^\wp\subseteq\Delta\subseteq A_K$, 且 $L=K\bigl(\wp^{-1}(\Delta)\bigr)$, 则 $\Delta\subseteq \Delta'=A_L^\wp\cap A_K$. 因此
  \[\Delta'/A_K^\wp\cong \Hom(G(L/K),\mu_\wp)\]
的子群对应
  \[\Delta/A_K^\wp\cong \Hom(G(L/K)/H,\mu_\wp),\]
其中
  \[H=\set{\sigma\in G(L/K)\mid \chi_a(\sigma)=1,\forall a\in\Delta}.\]
由于 $\chi_a(\sigma)=\sigma^{\sigma-1}$, 因此 $H$ 固定 $\wp^{-1}(\Delta)$, 从而固定 $L$, 即 $H=1$, $\Delta=\Delta'$.
\end{proof}

现在返回到局部类域论. 设 $\CN$ 是指标为 $n$ 的子群, 则 $K^{\times n}\subset \CN$. 我们可不妨设 $\mu_n\subseteq K^\times$, 不然令 $K_1=K(\mu_n)$. 若 $K_1$ 包含 $\bfN_{L_1/K_1}L_1^\times$, $L/K$ 是包含 $L_1$ 的伽罗瓦扩张, 则 $\bfN_{L/K}L^\times\subseteq K^\times$.

现在设 $\mu_n\subseteq K$, $L=K(\sqrt[n]{K^\times})$ 是 $K$ 的极大指数 $n$ 阿贝尔扩张, 则
  \[K^\times/\bfN_{L/K}L^\times\cong \Hom(G(L/K),\mu_n)\cong K^\times/K^{\times n}.\]
而根据 $K^\times$ 的结构, 它是有限的, 从而 $L/K$ 有限. 由于 $K^\times/\bfN_{L/K}L^\times\cong G(L/K)$ 指数 $n$, 因此 $K^{\times n}\subseteq \bfN_{L/K}L^\times$. 比较指数大小可知二者一致. 
\end{proof}


\subsection{阿贝尔扩域}
\begin{definition}{导子}{conductor of abelian extension}
设 $L/K$ 是有限阿贝尔扩张, $n$ 为最小的满足 $U_K^{(n)}\subseteq \bfN_{L/K}L^\times$ 的正整数. 称 $\ff_{L/K}=\fp_K^n$ 为 $L/K$ 的\noun{导子}.
\end{definition}

\begin{proposition}{}{}
有限阿贝尔扩张 $L/K$ 非分歧当且仅当 $\ff_{L/K}=1$.
\end{proposition}
\begin{proof}
如果 $L/K$ 非分歧, 则 $U_K=\bfN_{L/K}U_L$, $\ff=1$. 如果 $\ff=1$, 则 $U_K\subseteq \bfN_{L/K}U_L$, $\pi_K^n\in \bfN_{L/K}L^\times$, 其中 $n=(K^\times:\bfN_{L/K}L^\times)$.
设 $M/K$ 是 $n$ 次非分歧扩张, 则 $\bfN_{M/K}M^\times=\pi_K^{n\BZ}\times U_K\subseteq\bfN_{L/K}L^\times$, 因此 $M\supseteq L$, $L/K$ 非分歧.
\end{proof}

任何 $K^\times$ 的有限指标开子群都包含某个有限指标开子群 $\pi^{f\BZ}\times U_K^{(n)}$. 因此每个有限阿贝尔扩张 $L/K$ 都包含在某个 $\pi^{f\BZ}\times U_K^{(n)}$ 对应的类域. 换言之, 这样的类域对于研究 $K$ 的阿贝尔扩张是至关重要的. 现在我们考虑 $\BQ_p$ 的情形.

\begin{proposition}{}{}
$\BQ_p(\mu_{p^n})/\BQ_p$ 的范数群为 $p^\BZ\times U_{\BQ_p}^\times$.
\end{proposition}
\begin{proof}
设 $K=\BQ_p, L=\BQ_p(\mu_{p^n})$. 由于 $\zeta=\zeta_{p^n}$ 的极小多项式为 $\Phi(x)=x^{p^{n-1}(p-1)}+\cdots+x^{p^{n-1}}+1$, 因此
  \[\bfN_{L/K}(1-\zeta)=\prod_\sigma (1-\sigma\zeta)=\Phi(1)=p.\]
如果 $v_L$ 延拓 $v_K$, 则 $v_L(\zeta)=\frac{1}{p^{n-1}(p-1)} v_K(p)=\frac{1}{p^{n-1}(p-1)}$, $p\CO_L=(1-\zeta)^{p^{n-1}(p-1)}$, $p$ 完全分歧. 考虑
  \[\exp:\fp_K^\nu\to U_K^{(\nu)},\]
其中 $\nu=v_p(2p)=1$ 或 $2$. 由于 
  \[\fct{\fp_K^\nu}{\fp_K^{\nu+s-1}}{a}{p^{s-1}(p-1)a}\]
是一个同构, 它诱导了 $(U_K^{(1)})^{p^{n-1}(p-1)}=U_K^{(n)},p\ge 3$ 和 $(U_K^{(2)})^{2^{n-2}}=U_K^{(n)}$, $p=2$. 因此, $p\ge3$ 时 $U_K^{(n)}\subseteq \bfN_{L/K}L^\times$; $p=2$ 时
  \[U_K^{(2)}=U_K^{(3)}\cup 5U_K^{(3)}=(U_K^{(2)})^2\cup 5(U_K^{(2)})^2,\]
  \[U_K^{(n)}=(U_K^{(2)})^{2^{n-1}}\cup 5^{2^{n-2}}(U_K^{(2)})^{2^{n-1}}.\]
而 $5^{2^{n-2}}=\bfN_{L/K}(2+i)$, 因此 $U_K^{(n)}\subseteq \bfN_{L/K}L^\times$. 从而 $p^\BZ\times U_K^{(n)}\subseteq \bfN_{L/K}L^\times$. 又因为二者在 $K^\times$ 中的指标相同, 因此二者相等.
\end{proof}

\begin{corollary}{}{}
$\BQ_p$ 的任意有限阿贝尔扩张都包含在某个 $\BQ_p(\mu_n)$ 中. 特别地 $\BQ_p^\ab=\BQ_p(\mu_\infty)$.
\end{corollary}
\begin{proof}
设 $\zeta=\zeta_n$, $p\nmid n$, $\Phi(X)$ 为其极小多项式, $L=\BQ_p(\zeta_n)$. 由于 $\ov\Phi\mid X^n-1$ 可分, 由亨泽尔引理它不可约, 从而 $\ov\Phi(x)$ 是 $\ov\zeta\equiv \zeta\mod\fp_L$ 在 $\BF_p$ 上的极小多项式, $[L:\BQ_p]=\deg\Phi=\deg\ov\Phi=f$, $L/\BQ_p$ 非分歧, $X^n-1$ 在剩余域 $\kappa_L$ 上完全分解为一次多项式乘积, 从而 $\kappa_L=\BF_{p^f}$ 由 $\mu_n$ 生成, 即 $n\mid p^f-1$. 因此 $\BQ_p(\mu_{p^f-1})/\BQ_p$ 是 $f$ 次非分歧扩张.

对于一般的阿贝尔扩张 $M$, 设 $p^{f\BZ}\times U_{\BQ_p}^{(n)}\subseteq \bfN_{M/K}M^\times$, 则 $M$ 包含在
  \[p^{f\BZ}\times U_{\BQ_p}^{(n)}=\bigl(p^{f\BZ}\times U_{\BQ_p}\bigr)\cap \bigl(p^{\BZ}\times U_{\BQ_p}^{(n)}\bigr)\]
对应的类域 $L\BQ_p(\mu_{p^n})=\BQ_p(\mu_{(p^f-1)p^n})$ 中.
\end{proof}

\begin{theorem}{克罗内克-韦伯}{}
$\BQ$ 的任意有限阿贝尔扩张都包含在某个 $\BQ(\mu_n)$ 中. 特别地 $\BQ_p^\ab=\BQ_p(\mu_\infty)$.
\end{theorem}
\begin{proof}
设 $L/\BQ$ 是阿贝尔扩张, $S$ 包含所有在 $L$ 中分歧的素数, $\fp\mid p$, 则 $L_\fp/\BQ_p$ 是阿贝尔扩张, 从而 $L_\fp\subseteq \BQ_p(\mu_{n_p})$. 设 $e_p=v_p(n_p)$, 
  \[n=\prod_{p\in S}p^{e_p},\]
$M=L(\mu_n)$, 则 $M/\BQ$ 是阿贝尔扩张且分歧的有限素位均包含在 $S$ 中. 设 $\fP\mid \fp$,
  \[M_\fP=L_\fp(\mu_n)=\BQ_p(\mu_{p^{e_p}n'})=\BQ_p(\mu_{p^{e_p}})\BQ_p(\mu_{n'}),p\nmid n',\]
由于 $\BQ_p(\mu_{n'})$ 是 $M_\fP/\BQ_p$ 的极大非分歧子扩张, 其惯性群
  \[I_p=G(\BQ_p(\mu_{p^{e_p}})/\BQ_p)\]
大小为 $\varphi(p^{e_p})$. 设 $I\subseteq G(M/\BQ)$ 由所有惯性群 $I_p,p\in S$ 生成, 则 $I$ 的固定域非分歧, 它只能是 $\BQ$. 由此
  \[\#I\le\prod_p\# I_p=\prod_p \varphi(p^{e_p})=\varphi(n)=[\BQ(\mu_n):\BQ],\]
$[M:\BQ]=[\BQ(\mu_n):\BQ]$, 即 $M=\BQ(\mu_n)$.
\end{proof}

\subsection{卢宾-泰特形式群}

对于一般的局部域, 我们需要利用卢宾-泰特形式群来描述其阿贝尔扩域. 

\begin{definition}{形式群}{formal group}
环 $R$ 上的\noun{形式群}指的是满足如下条件的形式幂级数 $\CF(X,Y)$ $\in$ $R\ldb X,Y\rdb$\footnote{高维的形式群也可以类似地定义.}
\begin{enumerate}
\item $\CF(X,Y)\equiv X+Y\mod\deg 2$;
\item $\CF(X,Y)=\CF(Y,X)$;
\item $\CF\bigl(X,\CF(Y,Z)\bigr)=\CF\bigl(\CF(X,Y),Z\bigr)$.
\end{enumerate}
我们记 $X+_\CF Y:=\CF(X,Y)$.
\end{definition}

\begin{exercise}
(1) 验证 $\Ga(X,Y)=X+Y$ 是形式群, 称为\nouns{形式加法群}{形式群!形式加法群}.

(2) 验证 $\Gm(X,Y)=X+Y+XY$ 是形式群, 称为\nouns{形式乘法群}{形式群!形式乘法群}.

(3) 设 $f(X)=a_1 X+a_2X^2+\dots\in R\ldb X\rdb$, $a_1\in R^\times$, 则存在 $f^{-1}(X)\in R\ldb X\rdb$ 使得 $f\bigl(f^{-1}(X)\bigr)=f^{-1}\bigl(f(X)\bigr)=X$. 此时 $\CF(X,Y)=f^{-1}\bigl(f(X)+f(Y)\bigr)$ 是形式群, $f$ 称为 $\CF$ 的\noun{对数}.
\end{exercise}


设 $\CF,\CG$ 是形式群. 如果 $f\in XR\ldb X\rdb$ 满足
  \[f\bigl(\CF(X,Y)\bigr)=\CG\bigl(f(X),f(Y)\bigr),\]
称之为形式群的\noun{同态} $f:\CF\to \CG$. 如果 $f\in R\ldb X\rdb^\times$, 即存在 $f^{-1}:\CG\to \CF$, 称之为\noun{同构}. 容易验证 $\CF$ 的自同态全体在加法和复合意义下构成环 $\End_R(\CF)$.

\begin{exercise}
设 $R$ 是 $\BQ$ 代数. 对于任意 $R$ 上形式群 $\CF$, 存在唯一的形式群同构 $\log_\CF:F\simto \Ga$, 使得 $\log_\CF(X)\equiv X\mod\deg 2$, 称之为 $\CF$ 的\noun{对数}.
\end{exercise}

\begin{exercise}
$\log_{\Gm}=\log(1+X)=\suml_{n=1}^\infty(-1)^{n+1}\frac{X^n}{n}.$
\end{exercise}

\begin{definition}{形式模}{formal module}
设 $\CF$ 是 $R$ 上的形式群. 如果环同态
  \[\fct{R}{\End_R(\CF)}{a}{[a]_\CF(X)}\]
满足 $[a]_\CF(X)\equiv aX\mod \deg 2$, 称之为\nouns{形式 $R$ 模}{形式群!形式 $R$ 模}, 或简称 $\CF$ 为形式 $R$ 模. 自然地, 形式模之间的同态为满足 $f\bigl([a]_\CF(X)\bigr)=[a]_\CG\bigl(f(X)\bigr)$ 的形式群同态 $f:\CF\to \CG$.
\end{definition}

设 $K$ 是一个局部环, $\pi$ 为其素元, $q=\#\kappa$. 

\begin{definition}{卢宾-泰特级数}{lubin-tate series}
如果 $e(X)\in \CO_K\ldb X\rdb$ 满足
  \[e(X)\equiv \pi X\mod \deg 2\quad e(X)\equiv X^q\mod\pi,\]
称之为关于 $\pi$ 的\noun{卢宾-泰特级数}.
\end{definition}

\begin{theorem}{}{}
(1) 对于任意卢宾-泰特级数 $e(X)$, 存在唯一的形式 $\CO_K$ 模 $\CF$ 使得 
  \[e\in\End_{\CO_K}(\CF),\quad [\pi]_\CF(X)=e(X),\]
称之为\noun{卢宾-泰特形式群}.

(2) 如果 $e'(X)$ 也是关于 $\pi$ 的卢宾-泰特级数, 则存在 $[a]_{\CF,\CF'}(X)\in\CO_K\ldb X\rdb$ 使得
  \[[a]_{\CF,\CF'}:\CF\to\CF'\]
是形式 $\CO_K$ 模同态. 如果 $a$ 是一个单位, 它是形式模同构.
\end{theorem}
\begin{example}
设 $K=\BQ_p$, $e(X)=(1+X)^p-1$, 则
  \[\fct{\BZ_p}{\End_{\BZ_p}(\Gm)}{a}{[a]_{\Gm}(X)=(1+X)^a-1}\]
使得 $\Gm$ 成为对应的形式 $\BZ_p$ 模.
\end{example}

\begin{proposition}{}{}
设 $e,e'$ 分别是关于素元 $\pi,\pi'$ 的卢宾-泰特级数, $\varphi$ 是弗罗贝尼乌斯的一个提升, $\vK=\wh{\wt K}$.
如果 $a_i\in \CO_{\vK}$ 满足 $\pi a_i=\pi' \varphi(a_i)$, 
$L(X_1,\dots,X_n)=\sum_{i=1}^na_iX_i$, 则存在唯一的幂级数 
  \[\CF(X_1,\dots, X_n)\in\CO_{\vK}\ldb X_1,\dots,X_n\rdb\]
满足
  \[\CF\equiv L\mod\deg 2,\quad e\circ \CF=\CF^{\varphi}\circ e'.\]
\end{proposition}
该命题可归纳地解出 $\CF$ 的每一个齐次项, 这里不做详解. 当 $\pi=\pi'$, $a_i\in\CO_K$ 时, $\CF$ 还是 $\CO_K$ 系数的. 由此可得前述定理, 且同一个 $\pi$ 对应的卢宾-泰特形式群是同构的, 

如果 $R$ 是一个完备赋值环, $\fp$ 是其极大理想, 则 
  \[x+_\CF y:=\CF(x,y)\]
定义了 $\fp$ 上一个交换群的结构. 如果 $\CF$ 还是一个形式 $R$ 模, 则 $\fp$ 成为 $R$ 模.

设 $\ov K$ 为局部域 $K$ 的代数闭包, $\bar \fp$ 为其极大理想, $\pi$ 为 $K$ 的一个素元, $\LT$ 为其关联的一个卢宾-泰特形式群, 它在同构下是唯一的. 定义
  \[\LT[\pi^n]=\set{\lambda\in \bar \fp\mid [\pi^n]_\LT(\lambda)=0}=\ker [\pi^n]_\LT\]
为其 $\pi^n$ 等分点群. 易知它是一个 $\CO_K/\pi^n\CO_K$ 模.

\begin{proposition}{}{}
$\LT[\pi^n]$ 是秩为 $1$ 的自由 $\CO_K/\pi^n\CO_K$ 模. 因此 $[~~]_\CF$ 定义了同构
  \[\CO_K/\pi^n\CO_K\simto \End_{\CO_K}(\LT[\pi^n]),\quad
U_K/U_K^{(n)}\simto \Aut_{\CO_K}(\LT[\pi^n]).\]
\end{proposition}
\begin{proof}
不妨设 $e(X)=X^q+\pi X=[\pi]_\CF(X)$, 则 $\LT[\pi^n]$ 是 $e^n(X)=0$ 的根, 归纳可知它是可分多项式. 设 $\lambda_n\in\LT[\pi^n]-\LT[\pi^{n-1}]$, 则
  \[\CO_K\to \LT[\pi^n],\quad a\mapsto [a]_\CF(\lambda_n)\]
诱导了 $\CO_K$ 模同构 $\CO_K/\pi^n\CO_K\simto \LT[\pi^n]$.
\end{proof}

设 $L_n=K(\LT[\pi^n])$. 由于不同的卢宾-泰特形式群之间是同构的, $f:\CF\to \CG$, 因此 $\CG[\pi^n]=f(\CF[\pi^n]), K(\CG[\pi^n])\subseteq K(\CF[\pi^n])$, $L_n$  不依赖于 $\LT$ 的选取.
\begin{example}
如果 $K=\BQ_p,e(X)=(1+X)^p-1$, 则
  \[\LT[\pi^n]=\set{\zeta-1\mid \zeta\in\mu_{p^n}},\]
于是 $L_n=\BQ_p(\mu_{p^n})$.
\end{example}

\begin{theorem}{}{}
$L_n/K$ 是完全分歧的 $q^{n-1}(q-1)$ 次阿贝尔扩张, 伽罗瓦群为
  \[G(L_n/K)\cong\Aut_{\CO_K}(\LT[\pi^n])\cong U_K/U_K^{(n)}, \]
其中 
  \[\sigma\mapsto u\mod U_K^{(n)},\quad \lambda^\sigma=[u]_F(\lambda),\lambda\in\LT[\pi^n].\]
如果 $\lambda_n\in\LT[\pi^n]-\LT[\pi^{n-1}]$, 则 $L_n=K(\lambda_n)$ 是 $L_n$ 的素元, 且
  \[\phi_n(X)=\frac{e^n(X)}{e^{n-1}(X)}=X^{q^{n-1}(q-1)}+\cdots+\pi\in\CO_X[X]\]
是它的极小多项式, $\bfN_{L_n/K}(-\lambda_n)=\pi$. 
\end{theorem}
\begin{proof}
如果
  \[e(X)=X^q+\pi(a_{q-1}X^{q-1}+\cdots+a_2X^2)+\pi X,\]
则
  \[\phi_n(X)=e^{n-1}(X)^{q-1}+\pi(a_{q-1}X^{q-2}+\cdots+a_2X)+\pi \]
是艾森斯坦多项式, 从而是 $\lambda_n$ 的极小多项式. 于是 $\lambda_n$ 是完全分歧扩张 $K(\lambda_n)/K$ 的素元. 任一 $\sigma\in G(L_n/K)$ 诱导了 $\LT[\pi^n]$ 的自同构, 从而
  \[G(L_n/K)\to \Aut_{\CO_K}(\LT[\pi^n])\cong U_K/U_K^{(n)}.\]
由于 $L_n$ 由 $\LT[\pi^n]$ 生成, 它是单的. 而
  \[\#G(L_n/K)\ge [K(\lambda_n):K]=q^{n-1}(q-1)=\# U_K/U_K^{(n)},\]
因此它是同构.
\end{proof}

\begin{theorem}{}{}
设 $a=u\pi^{v_K(a)}\in K^\times$, $u\in U_K$, 则
  \[(a,L_n/K)\lambda=[u^{-1}]_\LT(\lambda),\quad\lambda\in\LT[\pi^n].\]
\end{theorem}
\begin{proof}
设 $\sigma\in G(L_n/K)$ 对应 $u\in U_K$, $\wt \sigma\in\Frob(\wt L_n/K)$ 是 $\sigma$ 的提升且满足 $d_K(\wt\sigma)=1$. 我们将 $\wt \sigma$ 视为 $\breve L_n=L_n\vK$ 的自同构. 设 $\Sigma$ 是 $\wt\sigma$ 的固定域, 则 $f_{\Sigma/K}=d_K(\wt \sigma)=1$, $\Sigma/K$ 完全分歧. 由于 $\Sigma\cap \wt K=K,\wt\Sigma=\Sigma\wt K=\wt L_n$, 因此 $[\Sigma:K]=[\wt L_n:\wt K]=[L_n:K]=q^{n-1}(q-1)$.

设 $e,e'$ 分别为对应 $\pi,\pi'$ 的卢宾-泰特级数, $\pi=u\pi'$, $\CF,\CF'$ 为 $e,e'$ 对应的卢宾-泰特形式群. 则存在 $\theta=\varepsilon X+O(X^2)\in\CO_{\breve K}\ldb X\rdb$, 使得 $\varepsilon\in U_{\breve K}$,
  \[\theta^\varphi=\theta\circ [u]_{\CF'},\quad
\theta^\varphi\circ e'=e\circ \theta, \quad\varphi=\varphi_K.\]
设 $\lambda_n\in \CF[\pi^n]-\CF[\pi^{n-1}]$, $\pi_\Sigma=\theta(\lambda_n)$. 则 
  \[\pi_{\Sigma}^{\wt \sigma}=\theta^{\varphi}(\lambda_n^\sigma)=\theta^\varphi\bigl([u^{-1}]_\CF (\lambda_n)\bigr)=\theta(\lambda_n)=\pi_\Sigma.\]
由于 $i=n$ 时, ${e'}^i\bigl(\theta(\lambda_n)\bigr)=\theta^{\varphi^i}\bigl({e}^i(\lambda_n)\bigr)=0$, $i=n-1$ 时, 它非零. 因此 $\pi_\Sigma\in \CF'[{\pi'}^n]-\CF'[{\pi'}^{n-1}]$, $\Sigma=K(\pi_\Sigma)$, $\bfN_{\Sigma/K}(-\pi_\Sigma)=\pi'=u\pi$,
  \[r_{L_n/K}(\sigma)=\bfN_{\Sigma/K}(-\pi_\Sigma)=\pi'\equiv u\mod \bfN_{L_n/K}L_n^\times,\]
因此
  \[(a,L_n/K)=(\pi^{v_K(a)},L_n/K)(u,L_n/K)=(u,L_n/K)=\sigma.\]
\end{proof}

\begin{example}
当 $K=\BQ,\LT=\Gm$ 时, $a=u p^{v_p(a)}\in \BQ_p^\times,u\in\BZ_p^\times,\lambda=\zeta-1$, 其中 $\zeta$ 是本原 $p^n$ 次单位根, 则
  \[\bigl(a,\BQ_p(\mu_{p^n})/\BQ_p\bigr)\zeta=\zeta^{u^{-1}}.\]
\end{example}

\begin{theorem}{}{}
$L_n/K$ 的范数群为 $(\pi)\times U_K^{(n)}$. 因此, $K^\ab=\wt K(\LT[\pi^\infty])$ 为 $K$ 的极大阿贝尔扩张.
\end{theorem}
\begin{proof}
由于 $a=u\pi_K^{v_K(a)}$ 时, $a\in \bfN_{L_n/K}L_n^\times$ 当且仅当 $[u^{-1}]_\LT=\id_{\LT[\pi^n]}$, 即 $u\in U_K^{(n)}$. 设 $L/K$ 是有限阿贝尔扩张, 则 $\pi^{f\BZ}\times U_K^{(n)}\in \bfN_{L/K}L^\times$. 和分圆域情形类似, 此时 $\pi^{f\BZ}$ 对应的是非分歧扩张, 因此 $L\subseteq \wt K(\LT[\pi^\infty])$.
\end{proof}

对于 $K^\times$ 的高阶单位群, 它在范剩余符号下的像是 $G(L/K)$ 的高阶分歧群.
\begin{theorem}{}{}
设 $L/K$ 是有限阿贝尔扩张, 范剩余符号
  \[(\ , L/K):K^\times\to G(L/K)\]
将 $U^{(n)}_K$ 映为 $G^n(L/K)$. 特别地, $\set{G^t(L/K)}_{t\ge -1}$ 仅在整数处有跳跃.
\end{theorem}
\begin{proof}
见\cite[Chapter V, \S 6]{Neukirch1999}, 这里省略.
\end{proof}


\subsection{希尔伯特符号}
我们称互反映射的逆
  \[(~,L/K):K^\times\to G(L/K)^\ab\]
为\noun{局部范数剩余符号}. 对于 $\BC/\BR$ 情形, $(a,\BC/\BR)=\sgn(a)\in G(\BC/\BR)$.

假设 $K\supseteq \mu_n$. 设 $L=K(\sqrt[n]{K^\times})$, 则 $\bfN_{L/K}L^\times=K^{\times n}$, 因此
  \[G(L/K)\cong K^\times/K^{\times n}.\]
另一方面, 我们有自然同构
  \[\begin{split}
    \Hom(G(L/K),\mu_n)&\cong K^\times/K^{\times n}\\
    \left(\sigma\mapsto \frac{(\sqrt[n]{a})^\sigma}{\sqrt[n]{a}}\right)&\mapsfrom a.
  \end{split}\]
因此双线性映射
  \[\fct{G(L/K)\times \Hom(G(L/K),\mu_n)}{\mu_n}{(\sigma,\chi)}{\chi(\sigma)}\]
诱导了
  \[\hil{\ }{\ }{\fp}: K^\times/K^{\times n}\times K^\times/K^{\times n} \to \mu_n.\]
称之为 $n$ 次\noun{希尔伯特符号}.

由定义我们有:
\begin{proposition}{}{}
对于 $a,b\in K^\times$,
  \[(a,K(\sqrt[n]{b})/K)\sqrt[n]{b}=\hil{a}{b}{\fp}\sqrt[n]{b}.\]
\end{proposition}

一般的希尔伯特符号和我们在\ref{2:hilbert_symbol}节中接触的二次希尔伯特符号一样, 也有着一系列易于计算的性质.
\begin{proposition}{}{}
(1) $\hil{aa'}{b}{\fp}=\hil{a}{b}{\fp}\hil{a'}{b}{\fp}$, $\hil{a}{bb'}{\fp}=\hil{a}{b}{\fp}\hil{a}{b'}{\fp}$;

(2) $\hil{a}{b}{\fp}=1\iff a\in\bfN_{K(\sqrt[n]{b})/K}K(\sqrt[n]{b})^\times$;

(3) $\hil{a}{b}{\fp}=1,\forall b\implies a\in K^{\times n}$;

(4) $\hil{a}{b}{\fp}=\hil{b}{a}{\fp}^{-1}$;

(5) $\hil{a}{-a}{\fp}=\hil{a}{1-a}{\fp}=1$.
\end{proposition}
\begin{proof}
(1-3)由定义和互反映射的性质可得. 设 $b\in K^\times,x\in K$ 使得 $x^n-b\neq 0$, 设 $d$ 是 $n$ 的最大的因子使得 $\sqrt[d]{b}\in K$. 设 $\beta^n=b$, 则 $K(\beta)/K$ 是 $m=n/d$ 次循环扩张且 $i\equiv j\mod d$ 时, $x-\zeta^i\beta$ 是 $x-\zeta^j\beta$ 的共轭元, 其中 $\zeta$ 是一个 $n$ 次本原单位根. 于是
  \[x^n-b=\prod_{i=0}^{d-1}\bfN_{K(\beta)/K}(x-\zeta^i \beta)\]
是 $K(\sqrt[n]{b})/K$ 的一个范数, 因此
  \[\hil{x^n-b}{b}{\fp}=1.\]
由此可得(5). 最后
  \[1=\hil{ab}{-ab}{\fp}=\hil{a}{b}{\fp}\hil{a}{-a}{\fp}\hil{b}{a}{\fp}\hil{b}{-b}{\fp}=\hil{a}{b}{\fp}\hil{b}{a}{\fp}\]
得到(4).
\end{proof}

\begin{exercise}
对于 $\BC/\BR$, $n=2$, 证明 $\hil{a}{b}{\infty}=(-1)^{\frac{\sgn a-1}{2}\cdot\frac{\sgn b-1}{2}}.$
\end{exercise}

设 $K$($\neq\BR,\BC$) 的剩余特征为 $p$, 假设 $p\nmid n$. 我们来计算该情形即所谓的\noun{温希尔伯特符号}. 由于 $K$ 的单位根群为 $\mu_{q-1}$, 因此 $n\mid q-1$.

\begin{lemma}{}{}
设 $p\nmid n$, $x\in K^\times$, 则 $K(\sqrt[n]{x})/K$ 非分歧当且仅当 $x\in U_K K^{\times n}$.
\end{lemma}
\begin{proof}
设 $x=uy^n,u\in U_K,y\in K^\times$, 则 $K(\sqrt[n]{x})=K(\sqrt[n]{u})$. 设 $\kappa'$ 为 $X^n-u$ 在 $\kappa$ 上的分裂域, $K'/K$ 是非分歧扩张且 $\kappa'$ 是 $K'$ 的剩余域. 由亨泽尔引理, $X^n-u$ 在 $K'$ 上完全分解, 因此 $K(\sqrt[n]{u})\subseteq K'$ 非分歧. 反之, 若 $L=K(\sqrt[n]{x})$ 非分歧, 令 $x=u\pi^r,u\in U_K$, $\pi$ 是素元, 则 $v_L(\sqrt[n]{u\pi^r})=r/n\in\BZ$, 从而 $n\mid r$.
\end{proof}

由于 $U_K=\mu_{q-1}\times U_K^{(1)}$, 因此任一 $u\in U_K$ 可分解为
  \[u=\omega(u)\pair{u},\]
其中 $\omega(u)\in \mu_{q-1},\pair{u}\in U_K^{(1)}$.

\begin{proposition}{}{}
设 $p\nmid n$, $a,b\in K^\times$, 则
  \[\hil{a}{b}{\fp}=\omega \left((-1)^{\alpha\beta}\frac{b^\alpha}{a^\beta}\right)^{(q-1)/n},\]
其中 $\alpha=v_K(a),\beta=v_K(b)$.
\end{proposition}
\begin{proof}
由于右侧是双线性的, 因此我们只需证明 $a=\pi,b=-\pi u,u\in U_K$ 的情形. 而 $\hil{\pi}{-\pi}{\fp}=1$, 因此只需证 $a=\pi,b=u$ 的情形.
设 $y=\sqrt[n]{u}$, $K'=K(y)$, 则 $K(y)/K$ 非分歧, 因此 $(\pi,K(y)/K)$ 是弗罗贝尼乌斯映射 $\varphi=\varphi_{K(y)/K}$, 于是
  \[\hil{\pi}{u}{\fp}=\frac{\varphi y}{y}\equiv y^{q-1}\equiv u^{(q-1)/n}\equiv \omega(u)^{(q-1)/n}\mod\fp.\]
由于 $\mu_n\subseteq \mu_{q-1}=\kappa^\times$, 因此两侧相等.
\end{proof}

我们可以看出 $\hil{\pi}{u}{\fp}$ 不依赖于 $\pi$ 的选取, 定义\noun{勒让德符号}(或 \nouns{$n$ 次剩余符号}{n 次剩余符号@$n$ 次剩余符号})
  \[\leg{u}{\fp}:=\hil{\pi}{u}{\fp}=\omega(u)^{(q-1)/n}.\]

\begin{exercise}
$\leg{u}{\fp}=1$ 当且仅当 $u\mod \fp$ 是一个 $n$ 次方.
\end{exercise}

我们来证明本节定义的希尔伯特符号和\ref{2:hilbert_symbol}节定义的一致. 我们只需证明 $K=\BQ_2$, $n=2$ 时,
  \[\hil{2}{a}{2}=(-1)^{(a^2-1)/8},\quad \hil{a}{b}{2}=(-1)^{\frac{a-1}{2}\cdot\frac{b-1}{2}},\quad a,b\in U_{\BQ_2}.\]
由于 $U_{\BQ_2}/U_{\BQ_2}^2=\pair{-1,5}$, 我们只需要考虑 $a,b=-1$ 或 $5$. $\hil{-1}{x}{2}=1$ 当且仅当 $x\in\bfN_{\BQ(\sqrt{-1})/\BQ}$, 因此 $\hil{-1}{2}{2}=\hil{-1}{5}{2}=1$. 但是 $-1$ 不是个平方, 因此 $\hil{-1}{-1}{2}=-1$. 由于 $\hil{2}{2}{2}=\hil{2}{-1}{2}=1$, 而 $2$ 不是个平方, 这迫使 $\hil{2}{5}{2}=-1$.

\begin{exercise}
证明 $U_{\BQ_2}/U_{\BQ_2}^2=\pair{-1,5}$.
\end{exercise}

可以看出, $p\mid n$ 的情形(\noun{野希尔伯特符号})相对而言更复杂. 具体的结果由 Br\"uckner 于 1964年给出, 见\cite[Theorem~5.3.7]{Neukirch1999}.


\begin{exercise}
了解和学习整体函数域情形的局部类域论.
\end{exercise}


\section{整体类域论}
\label{sec:global class field theory}

整体类域论中的 $G$ 模 $A$ 是所谓的伊代尔类群.
\subsection{阿代尔和伊代尔}

\begin{definition}{阿代尔环}{adele ring}
称
  \[\BA_K:={\prod_v}' K_v\]
为数域 $K$ 的\noun{阿代尔环}\index{A K@$\BA_K$}.
这里的 $'$ 表示相对于 $\CO_v$ 的\noun{限制直积}, 即除有限多个素位外, $\alpha_v\in\CO_v$\footnote{一般地, 给出一族局部紧群 $(G_\lambda)$, 对除有限多个 $\lambda$ 外给了一个紧开子群 $U_\lambda$, 则其直积中满足除有限多个 $\lambda$ 外, $x_\lambda\in U_\lambda$ 的元素称之为限制直积.}.
它的拓扑定义为相应的乘积拓扑的限制拓扑.
其中的元素被称为 $K$ 的\noun{阿代尔} (\noun{\adele}).
\end{definition}

\begin{definition}{伊代尔群}{idele group}
称阿代尔环的乘法群 
  \[\BI_K=\BA_K^\times={\prod_v}' K_v^\times\]
为 $K$ 的\noun{伊代尔群}\index{A K times@$\BA_K^\times$}\index{I K@$\BI_K$}, 其中的元素被称为\noun{伊代尔} (\noun{\idele}). 容易看出, 伊代尔群是 $K_v^\times$ 相对于 $\CO_v^\times$ 的限制直积.
\end{definition}

嵌入 $K\inj K_v$ 诱导了对角嵌入
  \[K^\times\inj \BI_K.\]
我们称 $K^\times$ 的像为\noun{主伊代尔}.

\begin{definition}{伊代尔类群}{idele class group}
商群 $C_K=\BI_K/K^\times$ 被称为 $K$ 的\noun{伊代尔类群}\index{C_K@$C_K$}.
\end{definition}

对于 $v\mid \infty$, 定义
  \[\CO_v=\begin{cases}
\BR_{\ge 0},&v\text{ 是实素位};\\
\BC,&v\text{ 是复素位}.
\end{cases}\]
设 $S$ 是 $K$ 的素位的一个有限集合, 我们记 $\BI^S_K$\index{ISK@$\BI^S_K$} 为 $S$ 以外的素位处都属于 $\CO_v^\times$ 的伊代尔构成的子群. 显然 
  \[\BI_K=\bigcup_S \BI^S_K.\]
记
  \[K^S=K^\times\cap \BI_K^S.\]
如果 $S_\infty=\set{v\mid \infty}$ 包含所有无穷素位, 则 $K^{S_\infty}=\CO_K^\times$.

考虑群同态
  \[\fct{(~):\BI_K}{\CI_K}{\alpha}{(\alpha)=\prod_{\fp\nmid \infty}\fp^{v_\fp(\alpha_\fp)}.}\]
它的核是
  \[\BI_K^{S_\infty}=\prod_{\fp\nmid \infty}\CO_\fp^\times \times\prod_{\fp\mid \infty}K_\fp^\times.\]
这诱导了群同态
  \[C_K\to \Cl_K,\]
核为 $\BI_K^{S_\infty} K^\times/K^\times$.

\begin{proposition}{}{}
$K^\times$ 是 $\BI_K$ 的离散闭子群.
\end{proposition}
\begin{proof}
只需证明 $1$ 有一个不包含其它主伊代尔的开集即可. 取
  \[U=\set{\alpha\in \BI_K\mid v\nmid \infty\ \text{时}, \alpha_v\in\CO_v^\times; v\mid \infty\ \text{时}, |\alpha_v-1|_v<1.}\]
如果 $1\neq x\in U$ 是一个主伊代尔, 则
  \[1=\prod_v |x-1|_v<\prod_{v\nmid \infty} |x-1|_v\le \prod_{v\nmid\infty}\max\set{|x|_v,1}=1,\]
矛盾.

由于 $(x,y)\mapsto xy^{-1}$ 是连续的, 存在 $1$ 的一个开邻域 $V$ 使得 $VV^{-1}\subseteq U$. 对于任意 $y\in \BI_K$, 如果 $yV$ 包含两个不同的主伊代尔 $x_1=yv_1,x_2=yv_2\in K^\times$, 则 $x_1x_2^{-1}=v_1v_2^{-1}\in U$, 矛盾! 因此 $yV$ 里只有至多一个主伊代尔, 因此 $K^\times$ 是闭子群.
\end{proof}

\begin{exercise}
(1) $\BA_\BQ=(\wh \BZ\otimes_\BZ\BQ)\times\BR$.

(2) $\BA_\BQ/\BZ$ 是紧的, 连通的.

(3) $\BA_\BQ/\BZ$ 任意唯一可除, 即对于任意 $y\in\BA_\BQ/\BZ,n\in\BZ$, 存在唯一的 $x\in\BZ_\BQ/\BZ$ 使得 $nx=y$.
\end{exercise}

\subsection{域扩张中的伊代尔}
设 $L/K$ 是数域的有限扩张. 我们记 
  \[L_\fp=\prod_{\fP\mid \fp} L_\fP=L\otimes_K K_\fp.\]
它是 $[L:K]$ 次 $K_\fp$ 代数. 自然的对角嵌入 $K_\fp\inj L_\fp$ 诱导了 $\BA_K\inj \BA_L$ 以及
  \[\fct{\BI_K}{\BI_L}{\alpha}{\alpha',}\]
其中 $\alpha'_\fP=\alpha_\fp\in K_\fp^\times\subseteq L_\fP^\times,\fP\mid\fp$.

设 $\sigma:L\simto \sigma L$ 是一个同构, 则它同样诱导了同构 $\sigma:\BI_L\to \BI_{\sigma L}$. 对于 $L$ 的任一素位 $\fP$, 设 $\alpha\in L_\fP$ 是 $\set{\alpha_i\in L}$ 在 $|\cdot|_\fP$ 下的极限, 令 $\sigma\alpha\in (\sigma L)_{\sigma\fP}$ 是的 $\set{\sigma\alpha_i\in \sigma L}$ 在 $|\cdot|_{\sigma\fP}$ 下的极限, 则 $\sigma:L_\fP\simto (\sigma L)_{\sigma\fP}$. 对于 $\alpha\in \BI_L$, 我们有 $(\sigma\alpha)_{\sigma\fP}=\sigma\alpha_\fP\in(\sigma L)_{\sigma\fP}.$
如果 $L/K$ 是伽罗瓦扩张, 则任一 $\sigma\in G=G(L/K)$ 诱导了同构 $\sigma:\BI_L\to \BI_L$, 因此 $\BI_L$ 是 $G$ 模.

\begin{proposition}{}{}
如果 $L/K$ 是有限伽罗瓦扩张, 则 $\BI_L^G=\BI_K$.
\end{proposition}
\begin{proof}
对于 $\alpha\in \BI_L$, $\sigma\in G$ 诱导了 $K_\fp$ 同构 $\sigma:L_\fP\to L_{\sigma\fP}$, $\fP\mid\fp$. 因此
  \[(\sigma\alpha)_{\sigma\fP}=\sigma\alpha_\fP=\alpha_\fP=\alpha_{\sigma\fP},\]
即 $\sigma\alpha=\alpha,\alpha\in \BI_L^G$. 反之, 若 $\alpha\in\BI_L^G$, 则
  \[(\sigma\alpha)_{\sigma\fP}=\sigma\alpha_\fP=\alpha_{\sigma\fP}.\]
如果 $\sigma\in G_\fP=G(L_\fP/K_\fp)$, 则 $\sigma\fP=\fP$, $\sigma\alpha_\fP=\alpha_\fP$, $\alpha_\fP\in K_\fp^\times$. 一般地, $\alpha_\fP=\sigma\alpha_\fP=\alpha_{\sigma\fP}$. 而 $G$ 在 $\set{\fP\mid \fp}$ 上传递, 因此 $\alpha\in\BI_K$.
\end{proof}

任一 $\alpha_\fp \in L_\fp^\times$ 诱导了 $K_\fp$ 向量空间 $L_\fp$ 上的自同构 
  \[\fct{\alpha_\fp:L_\fp}{L_\fp}{x}{\alpha_\fp x,}\]
它的行列式记为 $\bfN_{L_\fp/K_\fp}(\alpha_\fp)$. 由此定义了群同态
  \[\bfN_{L_\fp/K_\fp}:L_\fp^\times\to K_\fp^\times\]
以及
  \[\bfN_{L/K}:\BI_L\to \BI_K.\]
显然, $\alpha_\fp=(\alpha_\fP)$ 诱导的同构是 $\alpha_\fP:L_\fP\to L_\fP$ 的直和, 因此
  \[\bfN_{L/K}(\alpha)_\fp=\prod_{\fP\mid \fp}\bfN_{L_\fP/K_\fp}(\alpha_\fP).\]

\begin{proposition}{}{}
(1) 对于 $K\subseteq L\subseteq M$, $\bfN_{M/K}=\bfN_{L/K}\circ \bfN_{M/L}$.

(2) 如果 $M/K$ 是伽罗瓦扩张, $L$ 是中间域, $G=G(M/K)$, $H=G(M/L)$, 则对于 $x\in\BI_L$, $\bfN_{L/K}(\alpha)=\prod_{\sigma\in G/H}\sigma\alpha$.

(3) $\bfN_{L/K}(\alpha)=\alpha^{[L:K]},\alpha\in \BI_K$.

(4) 主伊代尔\ $x\in L^\times$ 的范数是 $\bfN_{L/K}(x)$ 对应的主伊代尔.
\end{proposition}
\begin{proof}
类似域扩张情形.
\end{proof}

由于 $\bfN_{L/K}:\BI_L\to\BI_K$ 将主伊代尔映为主伊代尔, 因此它诱导了 $\bfN_{L/K}:C_L\to C_K$.

现在我们考虑域扩张下伊代尔类群的关系.
\begin{proposition}{}{}
设 $L/K$ 是有限扩张, 则 $\BI_K\inj \BI_L$ 诱导了嵌入 $C_K\to C_L.$
\end{proposition}
\begin{proof}
我们只需证明 $\BI_K\cap L^\times=K^\times$. 设 $M$ 是 $L/K$ 的伽罗瓦闭包, $G=G(M/K),$
  \[\BI_K\cap L^\times\subseteq \BI_K\cap M^\times= (\BI_K\cap M^\times)^G=\BI_K\cap M^{\times G}=\BI_K\cap K^\times=K^\times.\]
\end{proof}

\begin{proposition}{}{}
设 $L/K$ 是有限伽罗瓦扩张, $G=G(L/K)$, 则 $C_L$ 是自然的 $G$ 模且 $C_L^G=C_K$. 
\end{proposition}
\begin{proof}
$L^\times$ 是 $\BI_L$ 的 $G$ 子模, 因此 $C_L$ 是自然的商模. 我们有 $G$ 模短正合列
  \[1\ra L^\times\ra \BI_L\ra C_L\ra 1,\]
这诱导了长正合列
  \[1\ra L^{\times G}\ra \BI_L^G\ra C_L^G\ra \rmH^1(G,L^\times)\]
由命题~\ref{pro:first_cohom_trivial}~知 $\rmH^1(G,L^\times)=1$, 因此
$C_L^G=\BI_L^G/L^{\times G}=\BI_K/K^\times=C_K.$
\end{proof}


\subsection{整体域的埃尔布朗商}
设 $L/K$ 是数域的有限伽罗瓦扩张, $G=G(L/K)$.
设 $\fP\mid\fp$, $G_\fP=G(L_\fP/K_\fp)\subseteq G$ 是其分解群. 我们有
  \[L_\fp=\prod_{\sigma\in G/G_\fP} L_{\sigma\fP}=\prod_{\sigma\in G/G_\fP} \sigma(L_\fP),\]
因此
  \[L_\fp^\times=\Ind_G^{G_\fP}L_\fP^\times,\quad U_{L,\fp}=\Ind_G^{G_\fP} U_\fP\]
是诱导模. 由诱导模性质知
  \[\rmH^i(G,L_\fp^\times)\cong \rmH^i(G_\fP.L_\fP^\times),\quad\rmH^i(G,U_{L,\fp})=\rmH^i(G_\fP,U_\fP).\]
\begin{proposition}{}{}
设 $L/K$ 是数域循环扩张, $S$ 包含 $K$ 的所有在 $L$ 中分歧的素位, 则对于 $i=0,-1$,
  \[\rmH^i(G,\BI_L^S)\cong\bigoplus_{\fp\in S}\rmH^i(G_\fP,L_\fP^\times),\quad\rmH^i(G,\BI_L)\cong\bigoplus_{\fp}\rmH^i(G_\fP,L_\fP^\times).\]
这里, 每个 $\fp$ 之上选取一个 $\fP$.
\end{proposition}
\begin{proof}
由于 $\BI_L^S=\bigl(\bigoplus_{\fp\in S}L_\fp^\times\bigr)\oplus V, V=\prod_{\fp\notin S}U_{L,\fp}$, 因此我们有
  \[\rmH^i(G,\BI_L^S)\cong\bigoplus_{\fp\in S}\rmH^i(G,L_\fp^\times)\oplus \rmH^i(G,V),\]
以及单射 
  \[\rmH^i(G,V)\inj \prod_{\fp\notin S}\rmH^i(G,U_{L,\fp}).\]
对于 $\fp\notin S$, $L_\fP/K_\fp$ 非分歧, 因此 $\rmH^i(G,U_{L,\fp})=\rmH^i(G_\fP,U_\fP)=1$. 由此我们得到第一个同构.
第二个由
  \[\rmH^i(G,\BI_L)=\ilim_S \rmH^i(G,\BI_L^S)\cong \ilim_S \bigoplus_{\fp\in S}\rmH^i(G_\fP,L_\fP^\times)=\bigoplus_\fp\rmH^i(G_\fP,L_\fP^\times)\]
可得.
\end{proof}

\begin{remark}
对于 $G=G(\BC/\BR)$, 我们有 
  \[\rmH^{-1}(G,\BC^\times)=1,\#\rmH^0(G,\BC^\times)=2.\]
因此在无穷素位的相应上同调也满足类域论公理.
\end{remark}

\begin{exercise}
设 $i=0,-1$.

(1) 对于任意多个 $G$ 模 $A_k$, $\rmH^i(G,\oplus_k A_k)=\prod_k \rmH^i(G,A_k)$. 

(2) 对于任意多个 $G$ 模 $A_k$, $\rmH^i(G,\prod_k A_k)\inj \prod_k \rmH^i(G,A_k)$ 是单射.

(3) 对于 $G$ 模正向系 $A_k$, $\rmH^i(G,\ilim_k A_k)=\ilim_k \rmH^i(G,A_k)$.
\end{exercise}


根据希尔伯特 90, $\rmH^{-1}(G_\fP,L_\fP^\times)=1$, 因此 $\rmH^{-1}(G,\BI_L)=1$. 换言之, 一个伊代尔是范当且仅当在每个局部如此.

设 $n_\fp=[L_\fP:K_\fp]$, 则由上述结论可知:
\begin{proposition}{}{}
设 $L/K$ 是循环扩张, $S$ 包含 $K$ 的所有在 $L$ 中分歧的素位, 则
  \[\rmH^{-1}(G,\BI_L^S)=1,\quad h(G,\BI_L^S)=\prod_{\fp\in S} n_\fp.\]
\end{proposition}
\begin{proposition}{}{}
设 $L/K$ 是 $n$ 次循环扩张, $S$ 包含 $K$ 的所有在 $L$ 中分歧的素位, 则
  \[h(G,L^S)=\frac{1}{n}\prod_{\fp\in S} n_\fp.\]
\end{proposition}
\begin{proof}
设 $\ov S$ 是 $L$ 中 $S$ 之上的素位全体, $\set{e_\fP}$ 是 $V=\prod_{\fP\in \ov S}\BR$ 的标准基, 则同态
  \[\lambda:L^S\to V,\quad \lambda(a)=\sum_{\fP\in\ov S}\log|a|_\fP e_\fP\]
的核为 $\mu(L)$, 像为 $(s-1)$ 维格, $s=\#\ov S$. 考虑 $G$ 在 $V$ 上的作用 $\sigma e_\fP=e_{\sigma\fP}$, 则 $\lambda$ 是 $G$ 模同态. 因此 $e_0=\sum_\fP e_\fP$ 和 $\lambda(L^S)$ 生成 $G$ 不变的完全格 $\Gamma$. 由于作为 $G$ 模, $\BZ e_0\cong\BZ$, 我们有
  \[0\ra \BZ e_0\ra \Gamma\ra \Gamma/\BZ e_0\ra 0,\] 
$\Gamma/\BZ e_0=\lambda(L^S)$, 因此
  \[h(G,L^S)=h\bigl(G,\lambda(L^S)\bigr)=h(G,\BZ)^{-1}h(G,\Gamma)=\frac{1}{n}h(G,\Gamma).\]

我们断言, 存在子完全格 $\Gamma'\subseteq \Gamma$ 使得
  \[\Gamma'=\sum \BZ w_\fP,\]
$\sigma w_\fP=w_{\sigma\fP}$.
设 $|\sum a_\fP e_\fP|=\max_\fP |a_\fP|$. 存在 $b>0$ 使得对任意 $x\in V,\gamma\in\Gamma,|x-\gamma|<b$. 选取充分大的 $t\in\BR$ 和 $\gamma\in\Gamma$, 使得 $y=te_{\fP_0}-\gamma$ 满足 $|y|<b$.
定义 $w_\fP=\sum_{\sigma \fP_0=\fP}\sigma \gamma$, 则 $\sigma w_\fP=w_{\sigma\fP}$. 我们来说明它们线性无关. 如果 $\sum c_\fP w_\fP=0$ 系数不全为零, 不妨设 $|c_\fP|\le 1$ 且存在 $c_{\fP'}=1$, 则 
  \[w_\fP=\sum_{\sigma\fP_0=\fP} \sigma\gamma=tn_\fP e_\fP-y_\fP,\] 
其中 $|y_\fP|\le gb$, $g=\#G$.
因此
  \[0=\sum c_\fP w_\fP=t\sum c_\fP n_\fP e_\fP-z,\quad |z|\le sgb.\]
而 $t$ 充分大时这不可能成立, 因此 $w_\fP$ 线性无关.

现在
  \[\Gamma'=\bigoplus_\fP \BZ w_\fP=\bigoplus_{\fp\in S}\bigoplus_{\fP\mid\fp}\BZ w_\fP=\bigoplus_{\fp\in S} \Gamma_\fp'.\]
而 $\Gamma_\fp'=\Ind_G^{G_\fp}\BZ w_{\fP_0}$ 是诱导模, $\Gamma'$ 在 $\Gamma$ 中有限指标, 从而
  \[h(G,L^S)=\frac{1}{n}h(G,\Gamma')=\frac{1}{n}\prod_{\fp\in S} (G_\fp,\BZ w_{\fP_0})=\frac{1}{n}\prod_\fp n_\fp.\]
\end{proof}

\begin{theorem}{}{}
设 $L/K$ 是 $n$ 次循环扩张, 则
  \[h(G,C_L)=n.\]
\end{theorem}
\begin{proof}
设 $\fa_1,\dots,\fa_h$ 是 $\Cl_L$ 的一组代表元, $S$ 为包含分歧、无穷素位以及 $\fa_i$ 在 $K$ 中分解的素因子的一个素位的有限集合, 则 $\BI_L=\BI_L^S L^\times$. 实际上, 对于任意 $\alpha\in \BI_L$, 设 $(\alpha)=\prod_{\fP\nmid\infty}\fP^{v_\fP(\alpha_\fP)}$ 为对应的分式理想. 设 $(\alpha)=\fa_i(a),a\in L^\times$, $\alpha'=\alpha a^{-1}$. 对于 $\fP\notin \ov S$, $v_\fP(\alpha'_\fP)=0,\alpha'_\fP\in U_\fP$, 因此 $\alpha'\in \BI_L^S$, 从而 $\BI_L=\BI_L^S L^\times$.
由正合列
  \[1\ra L^S\ra \BI_L^S\ra \BI_L^SL^\times/L^\times\ra 1\]
可知 $h(G,C_L)=h(G,\BI_L^S)/h(G,L^S)=n$.
\end{proof}

\begin{corollary}{}{}
如果 $L/K$ 是 $n=p^\nu$ 次循环扩张, 则有无穷多 $K$ 的素位在 $L$ 中不分裂.
\end{corollary}
\begin{proof}
假设不分裂的素位集合 $S$ 有限. 设 $M/K$ 是 $p$ 次子扩张. 对于任意 $\fp\notin S$, 相应的分解群 $G_\fp\neq G$. 因此 $G_\fp\subseteq G(L/M)$, 即所有 $\fp\notin S$ 在 $M/K$ 中完全分裂.

对于任意 $\alpha\in\BI_K$, 由逼近定理, 存在 $a\in K^\times$ 使得对任意 $\fp\in S$, $\alpha_\fp a^{-1}$ 包含在 $\bfN_{M_\fP/K_\fp}M_\fP^\times$ 的一个开子群中. 而 $\fp\notin S$ 时 $M_\fP=K_\fp$, 因此 $\alpha_\fp a^{-1}$ 自然落在 $\bfN_{M_\fP/K_\fp}M_\fP^\times$ 中. 因此 $\alpha a^{-1}$ 局部处处是范, 从而它是 $\BI_M$ 的范, $\alpha$ 在 $C_K$ 中的类落在 $\bfN_{M/K}C_M$ 中. 故 $C_K=\bfN_{M/K}C_M$, 这与 $h(G(M/K),C_M)=p$ 矛盾!
\end{proof}

\begin{corollary}{}{}
设 $L/K$ 是数域的有限扩张. 如果 $K$ 几乎所有的素位都在 $L$ 中完全分裂, 则 $L=K$.
\end{corollary}
\begin{proof}
设 $M/K$ 是其伽罗瓦闭包, $G=G(M/K),H=G(M/L)$. 设 $\fP\mid\fp$ 是 $M$ 的素位, 则 $L$ 中 $\fp$ 之上的素位个数为双陪集 $H\bs G/ G_\fP$ 的大小. 因此 $\fp$ 在 $L$ 中完全分裂当且仅当 $\# H\bs G/G_\fP=[L:K]=\# H\bs G$, 这意味着 $G_\fP=1$, 即 $\fp$ 在 $M$ 中完全分裂.

设 $\sigma\in G(M/K)$ 为素数阶元, 固定域为 $K'$, 则 $K'$ 几乎所有的素位在 $M$ 中完全分裂, 这与前述推论矛盾! 因此 $M=K,L=K$.
\end{proof}


\subsection{整体互反律}
现在我们知道循环扩张 $L/K$ 中 $C_L$ 的埃尔布朗商为 $n=[L:K]$, 因此我们只需要说明 $\rmH^0$ 的大小等于 $n$ 即可得到类域论公理. 我们略去证明过程.
\begin{theorem}{}{}
设 $L/K$ 是 $n$ 次循环扩张, 则
  \[\#\rmH^0\bigl(G(L/K),C_L\bigr)=n,\quad \rmH^{-1}\bigl(G(L/K),C_L\bigr)=1.\]
\end{theorem}
\begin{corollary}{}{}
设 $L/K$ 是循环扩张, 则 $K$ 中元素是 $L$ 的范当且仅当在每个局部如此.
\end{corollary}
\begin{proof}
设 $G=G(L/K)$.
由于
  \[1\ra L^\times\ra \BI_L\ra C_L\ra 1\]
正合, 我们有正合列
  \[1=\rmH^{-1}(G,C_L)\ra \rmH^0(G,L^\times)\ra \rmH^0(G,\BI_L).\]
由于 $\rmH^0(G,\BI_L)=\bigoplus_\fp \rmH^0(G_\fP,L_\fP^\times)$, 因此该命题成立.
\end{proof}

\begin{proposition}{}{}
设 $T$ 是 $G\bigl(\BQ(\mu_\infty)/\BQ\bigr)$ 的挠部分, 则 $T$ 的固定子域 $\wt\BQ$ 是 $\BQ$ 的 $\wh \BZ$ 扩张. 
\end{proposition}
\begin{proof}
由于
  \[G\bigl(\BQ(\mu_\infty)/\BQ\bigr)=\plim_n G\bigl(\BQ(\mu_n)/\BQ\bigr)\cong \plim_n (\BZ/n\BZ)^\times=\wh \BZ^\times,\]
而 $\wh \BZ=\prod_p \BZ_p,\BZ_p^\times\cong \BZ_p\times \BZ/(p-1)\BZ,p\neq 2$ 或 $\BZ_2\times \BZ/2\BZ,p=2$. 因此
  \[G\bigl(\BQ(\mu_\infty)/\BQ\bigr)\cong\wh \BZ^\times\cong \wh \BZ\times\wh T,\quad \wh T=\prod_{p\neq 2}\BZ/(p-1)\BZ\times \BZ/2\BZ,\]
而 $T=\bigoplus_{p\neq 2} \BZ/(p-1)\BZ\oplus \BZ/2\BZ$ 的闭包是 $\wh T$, 因此 $T$ 和 $\wh T$ 的固定域相同, $G(\wt \BQ/\BQ)=G\bigl(\BQ(\mu_\infty)/\BQ\bigr)/\wh T\cong \wh \BZ$.
\end{proof}

固定 $G(\wt \BQ/\BQ)\simto \wh\BZ$, 则我们有一个连续的满同态
  \[d:G_\BQ\to\wh \BZ.\]
对于数域 $K$, 令 $f_K=[K\cap \wt \BQ:\BQ]$, 则我们有满同态
  \[d_K=\frac{1}{f_K}d:G_K\to \wh\BZ.\]
这给出了一个 $\wh\BZ$ 扩张 $\wt K=K\wt\BQ/K$, 称之为 $K$ 的\noun{分圆扩张}.

对于有限次阿贝尔扩张 $L/K$, 定义
  \[\fct{[~,L/K]:\BI_K}{G(L/K)}{\alpha}{\prod_\fp(\alpha_\fp,L_\fP/K_\fp).}\]
由于对几乎所有的素位 $\fp$, $L_\fP/K_\fp$ 非分歧, 因此 $\alpha_\fp\in U_\fp$ 的像是 $1$, 从而右侧乘积有意义.

和局部类域论类似, 我们可以对于无穷次阿贝尔扩张定义范剩余符号, 且范剩余符号满足相应的函子性. 我们不做详解.

\begin{proposition}{}{}
对于任一单位根 $\zeta$ 和主伊代尔 $a\in K^\times$, $[a,K(\zeta)/K]=1$.
\end{proposition}
\begin{proof}
由函子性我们有 $[\bfN_{K/\BQ}(a),\BQ(\zeta)/\BQ]=[a,K(\zeta)/K]|_{\BQ(\zeta)}$. 因此我们只需对 $K=\BQ$ 情形证明即可. 出于同样的理由, 我们可以不妨设 $\zeta$ 是 $\ell^m\neq 2$ 阶. 对于 $a\in\BQ^\times$, 设 $a=u_pp^{v_p(a)}$. 对于 $p\neq \ell,\infty$, $\BQ_p(\zeta)/\BQ_p$ 非分歧, $(p,\BQ_p(\zeta)/\BQ_p)$ 是弗罗贝尼乌斯 $\varphi_p:\zeta\mapsto \zeta^p$. 由局部类域论可知
  \[(a,\BQ_p(\zeta)/\BQ_p)\zeta=\zeta^{n_p},\quad n_p=\begin{cases}
    p^{v_p(a)}, & p\neq \ell,\infty,\\
    u_p^{-1},   & p=\ell,\\
    \sgn(a),    & p=\infty.
  \end{cases}\]
因此
  \[[a,\BQ(\zeta)/\BQ]\zeta=\zeta^\alpha,\]
其中
  \[\alpha=\prod_p n_p=\sgn(a)\prod_{p\nmid \ell\infty} p^{v_p(a)}u_\ell^{-1}=1.\]
\end{proof}

由此可知 $[a,\wt K/K]=1,\forall a\in K^\times$.
因此我们有
  \[[~~,\wt K/K]:C_K\to G(\wt K/K).\]
通过复合 $d_K:G(\wt K/K)\to\wh\BZ$, 我们得到
  \[v_K:C_K\to\wh\BZ.\]
\begin{proposition}{}{}
$v_K$ 是满同态且是亨泽尔的.
\end{proposition}
\begin{proof}
我们对有限伽罗瓦子扩张 $L/K$ 来证明 $[~~,L/K]$ 是满射. 由于在每个局部范剩余符号是满的, 因此 $[\BI_K,L/K]$ 包含所有的分解群, 从而它的固定域 $M$ 上所有 $\fp$ 完全分裂, 这意味着 $M=K$, $[\BI_K,L/K]=G(L/K)$. 所以 $[\BI_K,\wt K/K]=[C_K,\wt K/K]$ 在 $G(\wt K/K)$ 中稠密. 

对于任意 $\alpha\in\BI_K$, 定义
  \[|\alpha|=\prod_{\fp} |\alpha_\fp|_\fp^{-1}.\]
由乘积公式可知 $|~|$ 在主伊代尔上平凡, 因此它诱导了连续同态 $|~|:C_K\to \BR_+^\times$. 我们不加证明地断言它的核 $C_K^0$ 是紧的. 通过将 $\BR_+^\times$ 看成一个无穷素位处完备化的正实数部分, 我们固定一个 $|~|$ 的截面, 从而 $C_K=C_K^0\times \BR_+^\times$. 而 $\BR_+^\times$ 的像是平凡的, 因此 $C_K$ 和 $C_K^0$ 的像相同, 它是一个闭集, 因此它的像等于它的闭包 $G(\wt K/K)$. 故 $v_K$ 是满的.

对于有限扩张 $L/K$, 由函子性我们有
  \[\begin{split}
    v_K(\bfN_{L/K}C_L)&=v_K(\bfN_{L/K}\BI_L)=d_K[\bfN_{L/K}\BI_L,\wt K/K]\\
    &=f_{L/K}d_L [\BI_L,\wt L/L]=f_{L/K}v_L(C_L)=f_{L/K}\wh \BZ.
  \end{split}\]
\end{proof}


于是
  \[d_\BQ:G_\BQ\to\wh\BZ,\quad v_\BQ:C_\BQ\to\wh\BZ\]
满足类域论公理. 我们有
\begin{theorem}{}{}
设 $L/K$ 是数域的伽罗瓦扩张, 我们有典范同构
  \[r_{L/K}:G(L/K)^\ab\simto C_K/\bfN_{L/K}C_L.\]
\end{theorem}
它的逆
  \[(~,L/K):C_K\to G(L/K)^\ab\]
被称为\noun{整体范剩余符号}.
\begin{proposition}{乘积公式}{}
设 $L/K$ 是数域的伽罗瓦扩张, 我们有典范同构
  \[(a,L/K)=\prod_\fp (a_\fp,L_\fP/K_\fp),\quad a\in \BA_K^\times.\]
特别地, 对于主伊代尔 $a\in K^\times$, 我们有乘积公式
  \[\prod_\fp(a,L_\fP/K_\fp)=1.\]
\end{proposition}
\begin{proof}
见\cite[Chapter VI, Proposition~5.6, Corollary~5.7]{Neukirch1999}, 这里省略.
\end{proof}


\subsection{整体类域}
\begin{theorem}{}{}
$L\mapsto \CN_L=\bfN_{L/K}C_L$ 给出了 $K$ 的有限阿贝尔扩张 $L$ 和 $C_K$ 的有限指标闭子群的一一对应.
\end{theorem}
\begin{proof}
我们需要证明范拓扑下开子群和和通常拓扑下有限指标闭子群一致. 我们省略.
\end{proof}

设 $\fm=\prod_{\fp\nmid\infty}\fp^{n_\fp}$ 为 $K$ 的理想, 记
  \[\BI_K^\fm={\prod_\fp}' U_\fp^{(n_\fp)}.\]

\begin{definition}{同余子群和射线类群}{congruence group and ray class field}
称
  \[C_K^\fm=\BI_K^\fm K^\times/K^\times\]
为模 $\fm$ 的\noun{同余子群}, 称 $C_K/C_K^\fm$ 为模 $\fm$ 的\noun{射线类群}.
\end{definition}

\begin{proposition}{}{}
$C_K$ 的子群是有限指标闭子群当且仅当其包含一个同余子群.
\end{proposition}
\begin{proof}
由于 $\BI_K^\fm\subseteq \BI_K$ 开, 因此 $C_K^\fm$ 是开子群. 由于 $\BI_K^\fm\subseteq \BI_K^{S_\infty}$, 而 $(C_K:\BI_K^{S_\infty}K^\times/K^\times)=h_K<\infty$, 因此
  \[\begin{split}
(C_K:C_K^\fm)&=h_K(\BI_K^{S_\infty}K^\times:\BI_K^\fm K^\times)\le h(\BI_K^{S_\infty}:\BI_K^\fm)\\
&=h_K\prod_{\fp\mid\fm} (U_\fp:U_\fp^{(n_\fp)})2^{r_1}
\end{split}\]
有限. 从而 $C_K^\fm$ 是有限指标闭子群, 包含 $C_K^\fm$ 的子群是有限多个陪集的不交并, 也是有限指标闭子群.

反之, 设 $\CN$ 是有限指标闭子群, 则 $\CN$ 是开子群. 于是它在 $\BI_K$ 中的原像 $U$ 是开集, 它包含某个
  \[W=\prod_{\fp\in S-S_\infty}U_\fp^{(n_\fp)}\times\prod_{\fp\in S\cap S_\infty}W_\fp \times\prod_{\fp\notin S}U_\fp,\]
其中 $\fp\in S\cap S_\infty$ 时, $W_\fp\subset K_\fp^\times$ 是开集, 它必然生成整个 $U_\fp$, 从而 $U$ 包含某个 $\BI_K^\fm$, $\CN\supseteq C_K^\fm$.
\end{proof}

设 $\CI_K^\fm$ 为所有和 $\fm$ 互素的分式理想, $\CP_K^\fm$ 为所有主分式理想 $(a)$, 其中 $a\equiv 1\mod\fm$ 且 $a$ 全正(即在所有实嵌入下的像大于 $0$). 令
  \[\Cl_K^\fm=\CI_K^\fm/\CP_K^\fm.\]

\begin{proposition}{}{}
自然同态 $(~):\BI_K\to \CI_K$ 诱导了同构 $C_K/C_K^\fm\cong \Cl_K^\fm.$ 特别地, $\Cl_\BQ^m\cong(\BZ/m\BZ)^\times$.
\end{proposition}
\begin{proof}
记
  \[\BI_K^{(\fm)}=\set{\alpha\in\BI_K\mid \alpha_\fp\in U_\fp^{(n_\fp)},\fp\mid \fm\infty}.\]
我们有 $\BI_K=\BI_K^{(m)}K^\times$, 这是因为对于 $\alpha\in\BI_K$, 由逼近定理, 存在 $a\in K^\times$ 使得 $\alpha_\fp a\equiv 1\mod \fp^{n_\fp},\fp\mid\fm$ 以及 $\alpha_\fp a>0,\fp$ 是实素位. 因此 $\beta=\alpha a\in\BI_K^{(\fm)}$, $\alpha=\beta a^{-1}\in\BI_K^{(\fm)}K^\times$. 显然 $\BI_K^{(\fm)}\cap K^\times$ 对应的 $\CP_K^\fm$ 中的理想, 因此我们有满同态
  \[C_K=\BI_K^{(\fm)}K^\times/K^\times=\BI_K^{(\fm)}/\BI_K^{(\fm)}\cap K^\times\to \CI_K^\fm/\CP_K^\fm.\]
显然 $C_K^\fm=\BI_K^\fm K^\times/K^\times$ 在核中. 如果 $[\alpha]$ 在核中, $\alpha\in \BI_K^{(\fm)}$, 则存在 $(a)\in \CP_K^\fm$ 使得 $(\alpha)=(a)$. 由于 $a\in \BI_K^{(\fm)}\cap K^\times$, 因此 $\beta=\alpha a^{-1}\in \BI_K^\fm$, $[\alpha]=[\beta]\in C_K^\fm$, 核为 $C_K^\fm$.

对每个 $\Cl_\BQ^m$ 中的代表元, 我们选择其正生成元, 则我们得到满同态 $\CI_K^m\to (\BZ/m\BZ)^\times$, 显然它的核是 $a\equiv m,a>0$ 生成的理想.
\end{proof}

称 $C_K^\fm$ 对应的类域 $K^\fm$ 为\noun{射线类域}, $K$ 的任意有限阿贝尔扩张均包含在其中. 当 $K=\BQ$ 时, 模 $m$ 的射线类域就是 $\BQ(\mu_m)$, 换言之, 射线类域是分圆域的推广.

对于一个有限阿贝尔扩张, 想要判断它落在哪个射线类域中, 我们需要引入导子.

\begin{definition}{导子}{conductor}
有限阿贝尔扩张 $L/K$ 的\noun{导子} $\ff$ 为满足 $L\subseteq K^\ff$ 的最小的 $\ff$.
\end{definition}

\begin{proposition}{}{}
$\ff_{L/K}=\prod_\fp \ff_{L_\fP/K_\fp}.$
\end{proposition}
\begin{proof}
设 $\CN=\bfN_{L/K}C_L$, 则对于 $\fm=\prod_\fp \fp^{n_\fp}$,
  \[C_K^\fm\subseteq \CN\iff \ff\mid \fm,\quad
\prod_\fp \ff_\fp\mid\fm \iff \fp\mid \fp^{n_\fp},\forall\fp\]
因此我们只需证 $C_K^\fm\subseteq\CN\iff \ff_\fp\mid \fp^{n_\fp},\forall\fp$. 我们知道一个伊代尔是范当且仅当每个局部是范, 因此 $C_K^\fm\subseteq \CN$ 当且仅当每个局部 $U_\fp^{(n_\fp)}\subseteq \bfN L_\fP^\times$, 即 $\ff_\fp\mid\fp^{n_\fp}$.
\end{proof}

\begin{corollary}{}{}
对于有限阿贝尔扩张 $L/K$, $\fp$ 分歧当且仅当 $\fp\mid\ff_{L/K}$.
\end{corollary}

令 $\fm=1$, 我们称 $K^1$ 为 $K$ 的\noun{大希尔伯特类域}, 它是 $K$ 的极大非分歧阿贝尔扩张.
\begin{proposition}{}{}
我们有正合列
  \[1\ra \CO^\times/\CO_+^\times\ra \prod_{\text{实素位}\fp}\BR^\times/\BR_+^\times\ra\Cl_K^1\ra \Cl_K\ra 1,\]
其中 $\CO_+^\times$ 表示所有的全正单位.
\end{proposition}
\begin{proof}
显然我们有
  \[1\ra \BI_K^{S_\infty}K^\times/\BI_K^1 K^\times\ra \Cl_K^1\ra \Cl_K\ra 1,\]
另一方面
  \[1\ra \BI_K^{S_\infty}\cap K^\times/\BI_K^1\cap K^\times\ra \BI_K^{S_\infty}/\BI_K^1\ra \BI_K^{S_\infty}K^\times/\BI_K^1 K^\times\ra 1.\]
由 $\BI_K^{S_\infty}\cap K^\times=\CO^\times, \BI_K^1\cap K^\times=\CO_+^\times$, $\BI_K^{S_\infty}/\BI_K^1=\prod\limits_{\fp\mid\infty}K_\fp^\times/U_\fp=\prod\limits_{\text{实素位}\fp}\BR^\times/\BR_+^\times$ 知该命题成立.
\end{proof}

$K^1$ 中无穷素位完全分裂(即实素位变成实素位)的极大子扩张被称为 $K$ 的\noun{希尔伯特类域}.
\begin{proposition}{}{hilbertclassfield}
$G(H_K/K)\cong\Cl_K$.
\end{proposition}
\begin{proof}
$H_K/K$ 是所有 $G(K^1_\fp/K_\fp),\fp\mid\infty$ 生成的子群 $G_\infty$ 的固定域, 它在范剩余符号下的像为所有 $K_\fp^\times$ 在 $G(K^1/K)\cong\BI_K/\BI_K^1 K^\times$ 中生成的子群, 即
  \[(\prod_{\fp\mid\infty} K_\fp^\times)\BI_K^1 K^\times/\BI_K^1 K^\times=\BI_K^{S_\infty}K^\times/\BI_K^1 K^\times,\]
因此
  \[G(H_K/K)=G(K^1/K)/G_\infty\cong \BI_K/\BI_K^{S_\infty}K^\times\cong\Cl_K.\]
\end{proof}

我们知道 $\BQ$ 的射线类域由单位根生成, 那么对于一般的数域而言, 是否可以通过添加解析函数的特殊值来得到呢? 目前为止, 人们仅知道虚二次域的情形.  该情形下, 射线类域由添加理想类的 $j$ 函数值得到, 在此不做详解.


设 $L/K$ 是 $n$ 次阿贝尔扩张, $\fp$ 非分歧, 则
  \[\varphi_\fp=(\pi_\fp,L_\fP/K_\fp)\]
不依赖于 $\pi_\fp$ 的选取, 且生成 $G_\fp=G(L_\fP/K_\fp)$. 记
  \[\leg{L/K}{\fp}:=\varphi_\fp.\]
设 $L\subseteq K^\fm$, 即 $\ff\mid\fm$, 则 $\fp\nmid\fm$ 时 $\fp$ 非分歧, 定义\noun{阿廷符号}
  \[\fct{\leg{L/K}{~}:\CI_K^\fm}{G(L/K)}{\fa=\prod_\fp \fp^{v_\fp}}{\prod_\fp \leg{L/K}{\fp}^{\nu_\fp}.}\]
显然
  \[\leg{L/K}{\fa}=(\prod_\fp \pair{\pi_\fp}^{\nu_\fp},L/K),\]
因此它在 $\CP_K^\fm$ 上平凡.

\begin{theorem}{}{}
设 $L/K$ 是 $n$ 次阿贝尔扩张, $\ff\mid\fm$, 则我们有满射
  \[\leg{L/K}{~}:\Cl_K^\fm/H^\fm\ra G(L/K),\]
它的核为 $H^\fm/\CP_K^\fm$, 其中 $H^\fm=(\bfN_{L/K}\CI_L^\fm)\CP_K^\fm$, 且我们有正合列的交换图
  \[\xymatrix{
    1\ar[r]&\bfN_{L/K}C_L\ar[r]\ar[d]&C_K\ar[r]^-{(~,L/K)}\ar[d]&G(L/K)\ar[r]\ar@{=}[d]&1\\
    1\ar[r]&H^\fm/\CP_K^\fm\ar[r]&\Cl_K^\fm\ar[r]^-{\leg{L/K}{~}}&G(L/K)\ar[r]&1.
  }\]
换言之, 我们有同构
  \[\leg{L/K}{~}:\CI_K^\fm/H^\fm\ra G(L/K).\]
\end{theorem}

\begin{theorem}{}{}
设 $L/K$ 是 $n$ 次阿贝尔扩张, $\ff\mid\fm$, $\fp$ 非分歧.
设 $\fp$ 在 $\CI^\fm_K/H^\fm$ 中的阶为 $f$, 则 $\fp$ 在 $L$ 中分解为 $n/f$ 个不同素理想的乘积. 特别地, 完全分解的素理想为 $H^\ff$ 中的素理想全体.
\end{theorem}
\begin{proof}
这是因为 $\fP\mid\fp$ 的分解群的大小为弗罗贝尼乌斯 $\varphi_\fp$ 的阶, 即 $\fp$ 在 $\CI^\fm_K/H^\fm$ 中的阶.
\end{proof}

\begin{example}
当 $K=\BQ$ 时, $p$ 在 $\BQ(\mu_m)/\BQ$ 中分解为 $\varphi(m)/f$ 个不同素理想乘积, 其中 $f$ 为 $p\mod m$ 的阶. 特别地, 当且仅当 $p\equiv 1\mod m$ 时, $p$ 完全分解.
\end{example}

我们知道 $G(H_K/K)=\Cl_K=\CI_K/\CP_K$, 因此该扩张对应的 $H^1=\CP_K$.
\begin{corollary}{}{}
$K$ 中素理想在希尔伯特类域中完全分解当且仅当它是主理想.
\end{corollary}

\begin{theorem}{}{}
$K$ 中任一理想在希尔伯特类域中变成主理想.
\end{theorem}
\begin{proof}
见\cite[Chapter VI, Proposition~7.5]{Neukirch1999}, 这里省略.
\end{proof}

一般而言, $K$ 的希尔伯特类域类数不为 $1$. 
一个自然的问题是, 
  \[K=K_0\subseteq K_1\subseteq \cdots\]
是否会在某一层停止?其中 $K_{i+1}=H_{K_i}$. 答案是否定的, E. S. Golod 和沙法列维奇证明了存在 $K$ 使得这个扩张塔可以一直增长下去.


\subsection{希尔伯特符号的整体性质}
设 $K$ 包含 $n$ 次单位根.
\begin{proposition}{}{}
对于 $a,b\in K^\times$,
  \[\prod_\fp \hil{a}{b}{\fp}=1.\]
\end{proposition}
对于与 $n$ 互素的理想 $\fb=\prod_{\fp\nmid n}\fp^{\nu_\fp}$, 以及和 $\fb$ 互素的元素 $a$, 定义 \nouns{$n$ 次剩余符号}{n 次剩余符号@$n$ 次剩余符号}
  \[\leg{a}{\fb}=\prod_{\fp\nmid n}\leg{a}{\fp}^{\nu_\fp}.\]
\begin{theorem}{}{}
如果 $a,b\in K^\times$ 互素且均和 $n$ 互素, 则
  \[\leg{a}{b}\leg{b}{a}^{-1}=\prod_{\fp\mid n\infty}\hil{a}{b}{\fp}.\]
\end{theorem}
当我们取 $n=2, K=\BQ$ 时, 这就是高斯的二次互反律.


\begin{exercise}
了解和学习整体函数域情形的整体类域论.
\end{exercise}

\begin{exercise}
了解和学习几何类域论.
\end{exercise}



% \chapter{级数}
\label{chapter:4}

复变函数的级数理论研究如何把复变函数展开成幂级数或双边幂级数的形式, 这样复变函数的一些性质就显得较为简单或容易研究.
与实变量情形有所不同, 只要复变函数在圆域或圆环域内处处解析, 就一定能展开成幂级数或双边幂级数的形式.
这些展开均来自于\thmCIH, 其中圆环域内解析函数的双边幂级数展开, 即洛朗展开, 可与该函数绕闭路积分联系, 这便引出了复变函数的留数理论.
最后, 我们针对不同的洛朗级数特点给出孤立奇点的分类, 为留数的计算做好准备.



\section{复数项级数}

\subsection{定义和敛散性}

和数列类似, 我们可仿照实数域上级数得到复数域上级数.

\begin{definition}
  \begin{enuma}
    \item 设 $\{z_n\}_{n\ge1}$ 是复数列. 称表达式 $\sumf1 z_n$ 为复数项\noun{无穷级数}, 简称\noun{级数}.\footnotemark
    \item 称 $s_n=z_1+z_2+\cdots+z_n$ 为该级数的\noun{部分和}.
    \item 若部分和数列 $\set{s_n}_{n\ge 1}$ 极限存在, 则称 $\sumf1 z_n$ \noun{收敛}, 并记 $\sumf1 z_n=\liml_{n\ra\infty}s_n$ 为它的\noun{和}. 否则称该级数\noun{发散}.
  \end{enuma}
\end{definition}
\footnotetext{复数列或无穷级数的下标也可以从 $0$ 或任意整数开始.}

若 $\sumf1 z_n=A$ 收敛, 则
\[
   \lim_{n\ra\infty} z_n
  =\lim_{n\ra\infty}s_n-\lim_{n\ra\infty}s_{n-1}
  =A-A=0.
\]
因此 \alert{$\liml_{n\ra\infty}z_n=0$ 是 $\sumf1 z_n$ 收敛的必要条件}.

\begin{theorem}
  设 $z_n=x_n+y_n\ii$, 则对于实数 $a$ 和 $b$,
  \[
    \sumf1 z_n=a+b\ii\iff
    \sumf1 x_n=a\ \text{且}\ 
    \sumf1 y_n=b.
  \]
\end{theorem}

\begin{proof}
  设部分和
  \[
    \sigma_n=x_1+x_2+\cdots+x_n,\quad
    \tau_n=y_1+y_2+\cdots+y_n,
  \]
  \[
    s_n=z_1+z_2+\cdots+z_n=\sigma_n+\ii \tau_n.
  \]
  由复数列的敛散性判定\thmref{定理}{thm:sequence-re-im} 可知
  \[
    \lim_{n\ra\infty}s_n=a+b\ii\iff	
    \lim_{n\ra\infty}\sigma_n=a\ \text{且}\ 
    \lim_{n\ra\infty}\tau_n=b.
  \]
  由此命题得证.
\end{proof}

\begin{theorem}
  \label{thm:absolute-convergent}
  若实数项级数
  \[
    \sumf1 \abs{z_n}=|z_1|+|z_2|+\cdots
  \]
  收敛, 则 $\sumf1 z_n$ 也收敛, 且 $\abs{\sumf1 z_n}\le\sumf1 \abs{z_n}$.
\end{theorem}

此即绝对收敛蕴含收敛.

\begin{proof}
  因为 $\abs{x_n},\abs{y_n}\le\abs{z_n}$, 由正项级数的比较审敛法可知实数项级数 $\sumf1 x_n$, $\sumf1 y_n$ 绝对收敛, 从而收敛.
  故 $\sumf1 z_n$ 也收敛.

  由三角不等式可知
  \[
    \abs{\sum_{k=1}^n z_k}\le \sum_{k=1}^n|z_k|.
  \]
  两边同时取极限即得级数的不等式关系
  \[
     \abs{\sumf1 z_n}
    =\abs{\lim_{n\ra\infty}\sum_{k=1}^n z_k}
    =\lim_{n\ra\infty}\abs{\sum_{k=1}^n z_k}
    \le\lim_{n\ra\infty}\sum_{k=1}^n|z_k|
    =\sumf1 \abs{z_n},
  \]
  其中第二个等式是因为绝对值函数 $|z|$ 连续.
\end{proof}

\begin{exercise}
  什么时候 $\abs{\sumf1 z_n}=\sumf1 \abs{z_n}$?
\end{exercise}

\begin{definition}
  \begin{enuma}
    \item 若级数 $\sumf1 \abs{z_n}$ 收敛, 则称 $\sumf1 z_n$ \noun{绝对收敛}.
    \item 称收敛但不绝对收敛的级数\noun{条件收敛}.
  \end{enuma}
\end{definition}

\begin{theorem}
  $\sumf1 z_n$ 绝对收敛当且仅当它的实部和虚部级数都绝对收敛.
\end{theorem}

\begin{proof}
  必要性由\thmref{定理}{thm:absolute-convergent} 的证明已经知道,
  充分性由 $\abs{z_n}\le\abs{x_n}+\abs{y_n}$ 以及正项级数的比较审敛法可得.
\end{proof}

由此我们得到复数项级数的敛散性与实部、虚部级数敛散性的关系\ref{tab:convergent-re-im}.

\begin{table}[!htb]
  \centering
  \begin{tabular}{ccc} \topcolorrule
    \bf 实部级数&\bf 虚部级数&\bf 复数项级数\\ \topcolorrule
    绝对收敛&绝对收敛&绝对收敛\\ \midcolorrule
    绝对收敛&条件收敛&条件收敛\\ \midcolorrule
    条件收敛&绝对收敛&条件收敛\\ \midcolorrule
    条件收敛&条件收敛&条件收敛\\ \midcolorrule
    发散&任意情形&发散\\ \midcolorrule
    任意情形&发散&发散\\ \bottomcolorrule
  \end{tabular}
  \caption{复数项级数与实部、虚部级数敛散性的关系}
  \label{tab:convergent-re-im}
\end{table}

绝对收敛的复级数各项可以任意重排次序而不改变其绝对收敛性以及和.
一般的级数重排有限项不改变其敛散性与和, 但若重排无限项则可能会改变其敛散性与和.\footnote{
  实数项条件收敛级数重排后可以收敛到任意实数, 也可发散到 $+\infty$, 也可发散到 $-\infty$, 此即\emph{黎曼重排定理}.
  而对于复数项条件收敛级数, 重排后能取到的和为全体复数(也可发散到 $\infty$), 或者为复平面内一条直线(也可发散到 $\infty$), 此即\emph{列维-斯坦尼兹定理}.
}

\begin{example}
  级数 $\sumf1 \frac{1+\ii^n}n$ 发散、条件收敛、还是绝对收敛?
\end{example}

\begin{solution}
  由于实部级数\footnote{正项级数可以任意重排而不改变敛散性以及和.}
  \[
    \sumf1 x_n
    =1+\frac13+\frac24+\frac15+\frac17+\frac28+\cdots
    =\sumf1 \frac1n
  \]
  发散, 所以该级数发散.
\end{solution}

\begin{example}
  \label{exam:lni}
  级数 $\sumf1 \dfrac{\ii^n}n$ 发散、条件收敛、还是绝对收敛?
\end{example}

\begin{solution}
  因为它的实部和虚部级数
  \begin{align*}
    \sumf1 x_n&=-\half +\frac14+\frac16-\frac18+\cdots\\
    \sumf1 y_n&=1-\frac13+\frac15-\frac17+\cdots
  \end{align*}
  均条件收敛, 所以原级数条件收敛.
\end{solution}

\begin{exercise}
  级数 $\sumf1 \Bigl(\frac{(-1)^n}n+\frac \ii{2^n}\Bigr)$ 发散、条件收敛、还是绝对收敛?
\end{exercise}


\subsection{判别法}

由正项级数的比值审敛法\footnote{又名\emph{达朗贝尔判别法}. 在使用该判别法之前, 可以先移除级数中所有的零项, 该判别法仍然适用.}可得:
\begin{theorem}[比值审敛法]
  若极限 $\lambda=\liml_{n\ra\infty}\abs{\dfrac{z_{n+1}}{z_n}}$ 存在或为 $+\infty$, 则
  \begin{enuma}
    \item 当 $\lambda<1$ 时, $\sumf0 z_n$ 绝对收敛;
    \item 当 $\lambda>1$ 时, $\sumf0 z_n$ 发散.
  \end{enuma}
\end{theorem}
当 $\lambda=1$ 时, 无法使用该方法判断敛散性.

将上述结论中的 $\lambda$ 换成 \alert{$\lambda=\liml_{n\ra\infty}\sqrt[n]{\abs{z_n}}$} 也成立, 叫做\noun{根式审敛法}\footnote{
  又名\emph{柯西判别法}.
  一般情形下, 我们有\emph{柯西-阿达马判别法}: 
  $\lambda=\liml_{k\ra\infty}\max\limits_{n\ge k}\sqrt[n]{\abs{z_n}}\in\BR\cup\{+\infty\}$, $\lambda$ 即该数列收敛子数列的极限的最大值, 也叫作\emph{上极限}.
}.

\begin{example}
  级数 $\sumf0 \frac{(8\ii)^n}{n!}$ 发散、条件收敛、还是绝对收敛?
\end{example}

\begin{solution}
  由
  \[
     \lim_{n\ra\infty}\abs{\frac{z_{n+1}}{z_n}}
    =\lim_{n\ra\infty}\abs{\dfrac{8\ii}{n+1}}
    =0<1
  \]
  可知该级数绝对收敛.
\end{solution}

实际上, 它的实部和虚部级数分别为
\[
  1-\frac{8^2}{2!}+\frac{8^4}{4!}-\cdots=\cos 8,\quad
  8-\frac{8^3}{3!}+\frac{8^5}{5!}-\cdots=\sin 8,
\]
因此
\[
   \sumf0 \frac{(8\ii)^n}{n!}
  =\cos 8+\ii \sin 8
  =\ee^{8\ii}.
\]
我们能否像高等数学情形一样, 直接将函数 $\ee^z$ 进行幂级数展开来得到该级数的和呢?
事实上任意解析函数都可以在解析点处展开成幂级数.



\section{幂级数}

\subsection{幂级数及其收敛域}

复变函数项级数与实变量函数项级数也是类似的.

\begin{definition}
  \begin{enuma}
    \item 设 $\bigl\{f_n(z)\bigr\}_{n\ge 1}$ 是一个复变函数列, 其中每一项都在区域 $D$ 上有定义.
    称表达式 $\sumf1 f_n(z)$ 为\noun{复变函数项级数}.
    \item 对于 $z_0\in D$, 若级数 $\sumf1 f_n(z_0)$ 收敛, 则称 \noun{$\sumf1 f_n(z)$ 在 $z_0$ 处收敛}.
    \item 若 $\sumf1 f_n(z)$ 在 $D$ 内处处收敛, 则随着 $z$ 的变化, 该级数的和是一个函数, 称为\noun{和函数}.
  \end{enuma}
\end{definition}

\begin{definition}
  称形如 $\sumf0 c_n(z-a)^n$ 的函数项级数为\noun{幂级数}.\footnotemark
\end{definition}
\footnotetext{尽管 $z=a$ 时 $(z-a)^0$ 无意义, 但为了简便我们约定幂级数在 $z=a$ 时取值为 $c_0$.}

我们只需要考虑 $a=0$ 情形的幂级数, 因为这和对应的一般情形幂级数的收敛范围以及和函数只是差一个平移.

和实变量情形相同, 复幂级数也有阿贝尔定理.

\begin{theorem}[阿贝尔定理]
  \label{thm:abel-first}
  \begin{enuma}
    \item 若 $\sumf0 c_nz^n$ 在 $z_0\neq 0$ 处收敛, 则对任意满足 $|z|<|z_0|$ 的 $z$, 该级数必绝对收敛.\label{enum:abel-theorem-convergent}
    \item 若 $\sumf0 c_nz^n$ 在 $z_0\neq 0$ 处发散, 则对任意满足 $|z|>|z_0|$ 的 $z$, 该级数必发散.
  \end{enuma}
\end{theorem}

\begin{proofenuma}
  \item 因为级数 $\sumf0 c_nz^n$ 收敛, 所以 $\liml_{n\ra\infty}c_n z_0^n=0$.
  故存在 $M$ 使得 $\abs{c_nz_0^n}\le M$.
  对于 $|z|<|z_0|$,
  \[
      \sumf0 \abs{c_nz^n}
    =\sumf0 \abs{c_nz_0^n}\cdot\abs{\frac z{z_0}}^n
    \le M\sumf0 \abs{\frac z{z_0}}^n
    =\frac{M}{1-\abs{\dfrac z{z_0}}}
  \]
  所以级数在 $z$ 处绝对收敛.
  \item 即 \ref{enum:abel-theorem-convergent} 的逆否命题.\qedhere
\end{proofenuma}

设 $R$ 是实幂级数 $\sumf0 |c_n|x^n$ 的收敛半径.
\begin{enuma}
  \item 若 $R=+\infty$, 由\thmAF 可知 $\sumf0 c_nz^n$ 处处绝对收敛.
  \item 若 $0<R<+\infty$, 则 $\sumf0 c_nz^n$ 在 $|z|<R$ 上绝对收敛, 在 $|z|>R$ 上发散.
  \item 若 $R=0$, 则 $\sumf0 c_nz^n$ 仅在 $z=0$ 处收敛, 对任意 $z\neq 0$ 都发散.
\end{enuma}
我们称 $R$ 为该幂级数的\noun{收敛半径}, 那么\alert{幂级数的收敛域就是圆域 $|z|<R$}.\footnote{收敛域是指收敛的最大区域. 当 $R=0$ 时, 该幂级数仅在 $z=0$ 处收敛, 没有收敛域.}

\begin{figure}[!hbt]
  \centering
  \begin{tikzpicture}
    \fill[cstfille2] (-2,-1.5) rectangle (2,1.5)
      node[second,below right] {发散};
    \filldraw[cstcurve,cstfill1] (0,0) circle (1.2);
    \fill[cstdot] (0,0) circle
      node[below,main] {绝对收敛};
    \draw[cstra] (0,0)--(0.96,0.72)
      node[pos=.5,below right] {$R$};
    \draw[cstra] (-1.2,0)--(-2.4,0)
      node[left] {都有可能};
  \end{tikzpicture}
  \caption{幂级数的收敛范围}
\end{figure}

\begin{example}
  求幂级数 $\sumf0 z^n=1+z+z^2+\cdots$ 的收敛半径与和函数.
\end{example}

\begin{solution}
  若该幂级数收敛, 则由 $\liml_{n\ra\infty}z^n=0$ 可知 $|z|<1$.
  当 $|z|<1$ 时, 该幂级数的和函数为
  \[
     \lim_{n\ra\infty}s_n
    =\lim_{n\ra\infty}\frac{1-z^{n+1}}{1-z}
    =\frac1{1-z}.
  \]
  因此收敛半径为 $1$.
\end{solution}

若每次都从定义来求收敛半径, 显然是不现实的.
我们可以利用实幂级数的收敛半径计算公式来求复幂级数的收敛半径.


\subsection{收敛半径的计算}

由计算实幂级数收敛半径的\noun{比值法}\footnote{又名\emph{达朗贝尔公式}.}可得:
\begin{theorem}
  若极限
  \[
    r=\lim_{n\ra\infty}\abs{\frac{c_{n+1}}{c_n}}
  \]
  存在或为 $+\infty$, 则 $\sumf0 c_nz^n$ 的收敛半径为
  \[
    R=\dfrac1r.
  \]
  若 $r=0$ 或 $+\infty$, 则相应地 $R=+\infty$ 或 $0$.
\end{theorem}

若将上述结论中的 $r$ 换成 \alert{$r=\liml_{n\ra\infty}\sqrt[n]{\abs{c_n}}$} 也成立, 叫做\noun{根式法}\footnote{
  又名\emph{柯西公式}.
  一般情形下, 我们有\emph{柯西-阿达马公式}: 
  $r=\liml_{k\ra\infty}\max\limits_{n\ge k}\sqrt[n]{\abs{c_n}}\in\BR\cup\{+\infty\}$, 即该数列的上极限.
}.

\begin{example}
  求幂级数 $\sumf1 \frac{(z-1)^n}n$ 的收敛半径, 并讨论 $z=0$ 和 $2$ 的情形.
\end{example}

\begin{solution}
  由
  \[
     \lim_{n\ra\infty}\abs{\frac{c_{n+1}}{c_n}}
    =\lim_{n\ra\infty}\frac n{n+1}=1
  \]
  可知收敛半径为 $1$.
  当 $z=2$ 时,
  \[
    \sumf1 \frac{(z-1)^n}n=\sumf1 \frac1n
  \]
  为调和级数, 从而发散.
  当 $z=0$ 时,
  \[
    \sumf1 \frac{(z-1)^n}n=\sumf1 \frac{(-1)^n}n
  \]
  为交错级数, 从而收敛.
\end{solution}

事实上, \alert{收敛圆周上既可能处处收敛, 也可能处处发散, 也可能既有收敛的点也有发散的点}.\footnote{
  例如 $\sumf1 z^n$ 在收敛圆周 $|z|=1$ 内处处发散; $\sumf1 \dfrac{z^n}{n^2}$ 在收敛圆周 $|z|=1$ 内处处绝对收敛; $\sumf1 \dfrac{(-1)^{[\sqrt n]}z^n}{n}$ 在收敛圆周 $|z|=1$ 内处处条件收敛, 这里 $[\alpha]$ 表示不超过 $\alpha$ 的最大整数.
}

\begin{example}
  求幂级数 $\sumf0 \cos(\ii n)z^n$ 的收敛半径.
\end{example}

\begin{solution}
  我们有
  \[
    c_n=\cos(\ii n)=\dfrac{\ee^n+\ee^{-n}}2.
  \]
  由
  \[
     \lim_{n\ra\infty}\abs{\frac{c_{n+1}}{c_n}}
    =\lim_{n\ra\infty}\frac{\ee^{n+1}+\ee^{-n-1}}{\ee^n+\ee^{-n}}
    =\ee\lim_{n\ra\infty}\frac{1+\ee^{-2n-2}}{1+\ee^{-2n}}
    =\ee
  \]
  可知收敛半径为 $\dfrac1\ee$.
\end{solution}

\begin{exercise}
  幂级数 $\sumf0 (1+\ii)^nz^n$ 的收敛半径为\fillblank[2cm]{}.
\end{exercise}


\subsection{幂级数的运算性质}

\begin{theorem}
  设幂级数
  \[
    f(z)=\sumf0 a_nz^n,\quad
    g(z)=\sumf0 b_nz^n
  \]
  的收敛半径分别为 $R_1,R_2$.
  那么当 $|z|<R=\min\{R_1,R_2\}$ 时,
  \[
    (f\pm g)(z)=\sumf0 (a_n\pm b_n)z^n,\quad
    (fg)(z)=\sumf0 \biggl(\sum_{k=0}^na_kb_{n-k}\biggr)z^n.
  \]
\end{theorem}

设 $c_n=a_n+b_n$.
当 $R_1\neq R_2$ 时, $\sumf0 c_n z^n$ 的收敛半径为 $\min\{R_1,R_2\}$.
这是因为当 $|z|$ 位于 $R_1,R_2$ 之间时(不妨设 $R_1>R_2$), 若 $\sumf0 c_n z^n$ 收敛, 则由 $\sumf0 a_n z^n$ 收敛可知   $\pm\sumf0 b_n z^n=\sumf0 (c_n-a_n) z^n$ 收敛, 这与 $|z|>R_2$ 矛盾.

但是若 $R_1=R_2$, 则 $\sumf0 (a_n\pm b_n)z^n$ 的收敛半径可以比 $R_1$ 大. 我们只需取一收敛半径大于 $R_1$ 的幂级数 $\sumf0 c_nz^n$, 并令 $b_n=c_n-a_n$, 则 $\sumf0 a_nz^n,\sumf0 b_nz^n$ 的收敛半径相同, 而 $\sumf0 (a_n+b_n)z^n$ 的收敛半径更大.

\begin{theorem}
  若幂级数 $\sumf0 c_nz^n$ 的收敛半径为 $R$, 则在 $|z|<R$ 上:
  \begin{enuma}
    \item 它的和函数 $f(z)=\sumf0 c_nz^n$ 解析,
    \item $f'(z)=\sumf1 nc_nz^{n-1}$,
    \item $\dint_0^zf(z)\d z=\sumf0 \frac{c_n}{n+1}z^{n+1}$.
  \end{enuma}
\end{theorem}

也就是说, \alert{在收敛圆内, 幂级数的和函数解析, 且可以逐项求导, 逐项积分}.

\begin{example}
  把函数 $\dfrac1{z-b}$ 表成形如 $\sumf0 c_n(z-a)^n$ 的幂级数, 其中 $a\neq b$.
\end{example}

\begin{solution}
  注意到
  \[
     \frac1{z-b}
    =\frac1{(z-a)-(b-a)}
    =\frac1{a-b}\cdot\frac1{1-\dfrac{z-a}{b-a}}.
  \]
  当 $|z-a|<|b-a|$ 时,
  \[
     \frac1{z-b}
    =\frac1{a-b}\sumf0 \Bigl(\frac{z-a}{b-a}\Bigr)^n,
  \]
  即
  \[
     \frac1{z-b}
    =-\sumf0 \frac{(z-a)^n}{(b-a)^{n+1}},
      \quad|z-a|<|b-a|.
  \]
\end{solution}

\begin{example}
  求幂级数 $\sumf1 (2^n-1)z^{n-1}$ 的收敛半径与和函数.
\end{example}

\begin{solution}
  由
  \[
     \lim_{n\ra\infty}\abs{\frac{c_{n+1}}{c_n}}
    =\lim_{n\ra\infty}\frac{2^{n+1}-1}{2^n-1}
    =2\lim_{n\ra\infty}\frac{1-2^{-n-1}}{1-2^{-n}}
    =2
  \]
  可知收敛半径为 $\dfrac12$.
  当 $|z|<\dfrac12$ 时, $|2z|<1$, 从而
  \begin{align*}
     \sumf1 (2^n-1)z^{n-1}&
    =\sumf1 2^n z^{n-1}-\sumf1 z^{n-1}\\&
    =\frac2{1-2z}-\frac1{1-z}
    =\frac1{(1-2z)(1-z)}.
  \end{align*}
\end{solution}

\begin{example}
  求幂级数 $\sumf0 (n+1)z^n$ 的收敛半径与和函数.
\end{example}

\begin{solution}
  由
  \[
     \lim_{n\ra\infty}\abs{\frac{c_{n+1}}{c_n}}
    =\lim_{n\ra\infty}\frac{n+1}n
    =\lim_{n\ra\infty}\bigl(1+\frac1n\bigr)
    =1
  \]
  可知收敛半径为 $1$.
  当 $|z|<1$ 时,
  \[
     \sumf0 z^{n+1}
    =\frac z{1-z}
    =-1-\frac1{z-1},
  \]
  因此
  \[
     \sumf0 (n+1)z^n
    =\Bigl(-\frac1{z-1}\Bigr)'
    =\frac1{(z-1)^2},
      \quad |z|<1.
  \]
\end{solution}

\begin{exercise}
  求幂级数 $\sumf1 \dfrac{z^n}n$ 的收敛半径与和函数.
\end{exercise}

\begin{example}
  求 $\doint_{|z|=\half} \sumf{-1} z^n\d z$.
\end{example}

\begin{solution}
  由于 $\sumf0 z^n$ 在 $|z|<1$ 收敛, 它的和函数解析, 因此
  \begin{align*}
     \oint_{|z|=\half} \sumf{-1} z^n\d z&
    =\oint_{|z|=\half}\frac1z\d z
      +\oint_{|z|=\half} \sumf0 z^n\d z\\&
    =2\cpi\ii +0
    =2\cpi\ii.
  \end{align*}
\end{solution}



\section{泰勒级数}

我们知道, 幂级数在它的收敛圆域内的和函数是一个解析函数.
反过来, 解析函数是不是也一定可以在解析点展开成幂级数呢? 也就是说是否存在泰勒级数展开. 答案是肯定的.

在高等数学中我们知道, 一个实变量函数即使在一点附近无限次可导, 它的泰勒级数也未必收敛到原函数.
例如
\[
  f(x)=\begin{cases}
    \ee^{-1/x^2},&x\neq 0,\\
    0,&x=0.
  \end{cases}
\]
它处处可导, 但是它在 $0$ 处的各阶导数都是 $0$.
因此它的泰勒级数是 $0$, 余项恒为 $f(x)$.
除 $0$ 外它的泰勒级数均不收敛到原函数.

而即使是泰勒级数能收敛到原函数的情形, 它成立的范围也很难从函数本身读出.
例如
\[
  \dfrac1{1+x}=1-x+x^2-x^3+\cdots
\]
成立的范围是 $|x|<1$, 这可以从该函数在 $x=-1$ 处无定义看出.
而
\[
  \dfrac1{1+x^2}=1-x^2+x^4-x^6+\cdots
\]
成立的范围也是 $|x|<1$, 但这个函数却处处任意阶可导.
为什么它的麦克劳林展开成立的最大区间也是 $(-1,1)$ 呢?
这些问题在本节可以得到回答.


\subsection{泰勒展开的形式与性质}

\begin{figure}[!htb]
  \centering
  \begin{tikzpicture}
    \filldraw [
      rotate=10,
      yshift=-5mm,
      cstcurve,
      main,
      cstfill1,
      decoration = {
        markings,
        mark = at position .5 with {
          \node[right]{$D$};
       }
     },
      postaction={decorate},
      domain=0:360,
      samples=500,
    ] plot ({3*cos(\x)+.2*cos(2*\x)-.3*cos(3*\x)-.2}, {1.8*sin(\x)+.2*sin(2*\x)});
    \draw[cstdash,third] (0,0) circle (1.3);
    \draw[cstcurve,second] (0,0) circle (1)
      node[above] {$z_0$};
    \draw[cstra,second] (0,0)--(-.6,-.8)
      node[pos=.5,above] {$r$};
    \fill[cstdot,second] (0,0) circle;
    \fill[cstdot,third] (.4,-.4) circle
      node[below] {$z$};
    \fill[cstdot] (0.8,-0.6) circle
      node[above] {$\zeta$};
  \end{tikzpicture}
  \caption{函数在解析点附近的泰勒展开}
\end{figure}

设函数 $f(z)$ 在区域 $D$ 内解析, $z_0\in D$ 到 $D$ 边界的距离为 $d$ (可以为 $+\infty$).
在 $D$ 内作一圆周 $K:|\zeta-z_0|=r<d$.
对于 $K$ 内部一点 $z$, $|z-z_0|<|\zeta-z_0|=r$.
由 $\dfrac1{\zeta-z}$ 幂级数展开的部分和可得
\[
   \frac1{\zeta-z}-\frac1{\zeta-z}\biggl(\frac{z-z_0}{\zeta-z_0}\biggr)^N
  =\frac1{\zeta-z_0}\cdot\frac{1-\biggl(\dfrac{z-z_0}{\zeta-z_0}\biggr)^N}{1-\dfrac{z-z_0}{\zeta-z_0}}
  =\sum_{n=0}^{N-1}\frac{(z-z_0)^n}{(\zeta-z_0)^{n+1}}.
\]
由\thmCIH 可知
\begin{align*}
   f(z)&
  =\frac1{2\cpi\ii}\oint_K \frac{f(\zeta)}{\zeta-z}\d\zeta\\&
  =R_N(z)+\sum_{n=0}^{N-1}\biggl(\frac1{2\cpi\ii}\oint_K
      \frac{f(\zeta)}{(\zeta-z_0)^{n+1}}\d\zeta
    \biggr)(z-z_0)^n\\&
  =R_N(z)+\sum_{n=0}^{N-1}\frac{f^{(n)}(z_0)}{n!}(z-z_0)^n,
\end{align*}
其中
\[
   R_N(z)
  =\frac1{2\cpi\ii}\oint_K\frac{f(\zeta)}{\zeta-z}\cdot\biggl(\frac{z-z_0}{\zeta-z_0}\biggr)^N\d\zeta.
\]

由于 $f(\zeta)$ 在 $D\supseteq K$ 内解析, 从而在 $K$ 上连续且有界.
设对任意 $\zeta\in K$, $|f(\zeta)|\le M$,
则由\thmGrowUp 和 $\abs{z-z_0}<\abs{\zeta-z_0}$ 可知当 $N\ra\infty$ 时,
\begin{align*}
   |R_N(z)|&
  \le\frac 1{2\cpi}\oint_K\abs{\frac{f(\zeta)}{\zeta-z}\cdot\biggl(\frac{z-z_0}{\zeta-z_0}\biggr)^N}\d s\\&
  \le\frac 1{2\cpi}\cdot \frac M{r-|z-z_0|}\cdot\abs{\frac{z-z_0}{\zeta-z_0}}^N\cdot 2\cpi r\ra 0.
\end{align*}
由此得到\noun{泰勒展开}:

\begin{theorem}
  若 $f(z)$ 在 $z_0$ 处解析, 则
  \[
    f(z)=\sumf0 \frac{f^{(n)}(z_0)}{n!}(z-z_0)^n,
      \quad |z-z_0|<d,
  \]
  其中 $d$ 是 $z_0$ 到最近的 $f(z)$ 奇点的距离.\footnotemark
\end{theorem}
\footnotetext{
  若 $f(z)$ 在复平面内处处解析, 则 $d=+\infty$.
  若 $f(z)$ 有奇点, 则一定存在距离解析点 $z_0$ 距离最近的奇点.
  不妨设 $z_0=0$.
  假设不存在这样的奇点, 则存在 $r\ge 0$ 使得 $f(z)$ 在 $|z|\le r$ 内解析, 且存在一串奇点 $z_1,z_2,\cdots$, 使得 $|z_k|$ 严格单调减趋于 $r$.
  由\emph{魏尔斯特拉斯聚点定理}, 存在 $\{z_n\}_{n\ge 1}$ 的一个收敛的子数列.
  设该子数列的极限为 $c$, 则 $c$ 是 $f(z)$ 的奇点且 $|c|=r$, 这与假设矛盾.
}

称在 $z_0=0$ 处的泰勒展开为\noun{麦克劳林展开}.

由于幂级数和函数在收敛圆域内解析, 因此 $f(z)$ 的泰勒展开成立的圆域不包含 $f(z)$ 的奇点.
由此可知, 解析函数在 $z_0$ 处\alert{泰勒展开成立的最大圆域半径 是 $z_0$ 到最近奇点的距离} $d$.

需要注意的是, 泰勒级数的收敛半径可能大于 $d$, 而且泰勒展开也可能对于一些点 $|z-z_0|\ge d$ 成立.
不过对于有理函数(分子分母没有公共零点), 其泰勒展开的收敛半径的确就是 $d$, 更多分析见 \ref{ssec:taylor-expansion-radius}.

现在我们来看本节开头提出的问题. 尽管函数
\[
  f(z)=\begin{cases}
    \ee^{-1/z^2},&z\neq 0,\\
    0,&z=0
  \end{cases}
\]
在 $z=x$ 沿实轴方向趋于 $0$ 时极限是零, 但当 $z=y\ii\ra0$ 时, $f(z)=\ee^{1/y^2}\ra\infty$, 因此 $0$ 是奇点, $f(z)$ 无法在该点处展开成幂级数.

对于函数 $f(z)=\dfrac1{1+z^2}$ 而言, 它的的奇点为 $\pm\ii$.
所以它的麦克劳林展开成立的最大半径是 $d=1$.
这就解释了为什么对应的实变量函数 $f(x)=\dfrac1{1+x^2}$ 的麦克劳林展开成立的最大开区间是 $(-1,1)$.


\subsection{泰勒展开的计算方法}

\begin{example}
  \label{exam:exp-taylor-expansion}
  由于
  \[
    (\ee^z)^{(n)}\big|_{z=0}=\ee^z\big|_{z=0}=1,
  \]
  因此
  \[
     \ee^z=\sumf0 \frac{z^n}{n!}
    =1+z+\frac{z^2}{2!}+\frac{z^3}{3!}+\cdots
  \]
\end{example}

这也表明, \ref{ssec:exponential-function} 中复指数函数可以通过右侧的幂级数定义得到.

\begin{example}
  由
  \[
    (\cos z)^{(n)}=\cos\Bigl(z+\frac{n\cpi}2\Bigr)
  \]
  可知
  \[
    (\cos z)^{(2n+1)}\big|_{z=0}=0,\quad 
    (\cos z)^{(2n)}\big|_{z=0}=(-1)^n,
  \]
  因此
  \[
     \cos z
    =\sumf0 (-1)^n\frac{z^{2n}}{(2n)!}
    =1-\frac{z^2}{2!}+\frac{z^4}{4!}-\frac{z^6}{6!}+\cdots
  \]
\end{example}

若 $f(z)$ 在 $z_0$ 附近展开为 $\sumf0 c_n(z-z_0)^n$,
则由幂级数的逐项求导性质可知
\[
   f^{(m)}(z_0)
  =\sum_{n=m}^\infty c_n n(n-1)\cdots(n-m+1)(z-z_0)^{n-m}
    \Big|_{z=z_0}
  =m!c_m.
\]
所以\alert{解析函数的幂级数展开是唯一的}.
因此解析函数的泰勒展开不仅可以通过直接求各阶导数得到, 也可以\alert{利用幂级数的运算法则得到}.

\begin{example}
  由 $\ee^z$ 的泰勒展开可得
  \begin{align*}
     \sin z&
    =\frac{\ee^{\ii z}-\ee^{-\ii z}}{2\ii}
    =\sumf0 \frac{(\ii z)^n-(-\ii z)^n}{2\ii\cdot n!}\\&
    =\sumf0 (-1)^n\frac{z^{2n+1}}{(2n+1)!}
    =z-\frac{z^3}{3!}+\frac{z^5}{5!}-\frac{z^7}{7!}+\cdots
  \end{align*}
\end{example}

可以发现, 奇函数的麦克劳林展开只有奇数幂次项, 没有偶数幂次项, 而偶函数的麦克劳林展开只有偶数幂次项, 没有奇数幂次项.

\begin{example}
  函数 $\ln(1+z)$ 在去掉射线 $z=x\le-1$ 的区域内解析, 因此它在 $|z|<1$ 内解析, 且此时
  \[
     \bigl(\ln(1+z)\bigr)'
    =\frac1{1+z}
    =\sumf0 (-1)^nz^n.
  \]
  逐项积分得到
  \[
     \ln(1+z)
    =\sumf0 \frac{(-1)^nz^{n+1}}{n+1}
    =\sumf1 \frac{(-1)^{n+1}z^n}{n},\quad|z|<1.
  \]
  这里注意由于 $\ln(1+z)$ 在零处取值为零, 因此逐项积分后没有常数项.
\end{example}

\begin{figure}[!htb]
  \centering
  \begin{tikzpicture}
    \def\r{1.3}
    \def\u{.8}
    \def\a{2.1}
    \def\b{1.6}
    \fill[cstfille1] (-\a,-\b) rectangle (\a,\b);
    \cutline{-\u-\cutwd}{0}{\r-\cutwd}{180}{main};
    \filldraw[cstcurve,fifth,cstfill5] (0,0) circle (\u);
    \draw[cstaxis] (-2.4,0)--(2.4,0);
    \draw[cstaxis] (0,-2)--(0,2);
  \end{tikzpicture}
  \caption{$\ln(z+1)$ 和 $(1+z)^a$ 主值泰勒展开成立的最大圆域, $a$ 不是正整数}
\end{figure}

\begin{example}
  \label{exam:power-taylor-series}
  设函数 $f(z)$ 为幂函数 $(1+z)^a$ 的主值 $\ee^{a\ln(1+z)}$. 它在去掉射线 $z=x\le -1$ 的区域内解析.
  我们有(均取主值)
  \[
    f'(z)=a(1+z)^{a-1},\quad
    f''(z)=a (a-1)(1+z)^{a-2},
  \]
  一般地,
  \[
    f^{(n)}(z)=a(a-1)\cdots(a-n+1)(1+z)^{a-n}.
  \]
  因此
  \[
    f^{(n)}(0)=a(a-1)\cdots(a-n+1),
  \]
  \begin{align*}
     (1+z)^a&
    =\sumf0 \frac{a(a-1)\cdots(a-n+1)}{n!}z^n\\&
    =1+a z+\frac{a(a-1)}2z^2+\frac{a(a-1)(a-2)}{3!}z^3+\cdots,\quad |z|<1.
  \end{align*}
\end{example}

当 $a=m$ 是正整数时, 上述麦克劳林展开成立的范围为整个复平面.
此时展开式中幂次大于 $m$ 的项系数为零, 从而得到牛顿二项式展开
\[
   (1+z)^m
  =\sum_{n=0}^m \rmC_m^n z^n
  =\rmC_m^0+\rmC_m^1 z+\cdots+\rmC_m^m z^m,
\]
其中 $\rmC_m^n=\dfrac{m!}{n!(m-n)!}$ 是组合数.\footnote{
  为了简便, 也可对任意复数 $a$ 记组合数 $\rmC_a^n=\dfrac{a(a-1)\cdots(a-n+1)}{n!}$.
  此外, 也有用记号 $\Bigl(\genfrac{}{}{0pt}{0}an\Bigr)$ 表示组合数的.
}

\begin{example}
  求函数 $\dfrac1{3z-2}$ 的麦克劳林展开.
\end{example}

\begin{solution}
  由于 $\dfrac1{3z-2}$ 的奇点为 $z=\dfrac23$, 因此它在 $|z|<\dfrac23$ 内解析.
  此时
  \begin{align*}
     \frac1{3z-2}&
    =-\half\cdot\frac1{1-\dfrac{3z}2}
    =-\half\sumf0 \Bigl(\frac{3z}2\Bigr)^n\\&
    =-\sumf0 \frac{3^n}{2^{n+1}}z^n,
      \quad|z|<\frac23.
  \end{align*}
\end{solution}

\begin{example}
  将 $\dfrac1{(1+z)^2}$ 展开成 $z$ 的幂级数.
\end{example}

\begin{solution}[解法一]
  由\thmref{例}{exam:power-taylor-series} 中幂函数的展开可知, 当 $|z|<1$ 时,
  \[
    (1+z)^{-2}=\sumf0 \frac{(-2)(-3)\cdots(-1-n)}{n!}z^n
    =\sumf0 (-1)^n(n+1)z^n.
  \]
\end{solution}

\begin{solution}[解法二]
  由于 $\dfrac1{(1+z)^2}$ 的奇点为 $z=-1$, 因此它在 $|z|<1$ 内解析.
  由于
  \[
     \frac1{1+z}=1-z+z^2-z^3+\cdots
    =\sumf0 (-1)^nz^n,
  \]
  因此
  \[
     \frac1{(1+z)^2}
    =-\Bigl(\frac1{1+z}\Bigr)'
    =-\sumf1 (-1)^n nz^{n-1}
    =\sumf0 (-1)^n (n+1)z^n,\quad |z|<1.
  \]
\end{solution}

如 \ref{ssec:application-of-derivative}所言, 若一个有理函数的奇点均可求出, 则可以将它写成一个多项式与一些部分分式之和.
而部分分式可以展开为
\[
   \frac1{(\lambda-z)^k}
  =\sumf0 \frac{(n+k-1)\cdots(n+2)(n+1)}{(k-1)!}\lambda^{-(n+k)}z^n,\quad |z|<|\lambda|.
\]
由此可得到该有理函数的泰勒展开.

\begin{exercise}
  求 $\dfrac1{1-3z+2z^2}$ 的麦克劳林展开.
\end{exercise}
\smallskip

解析函数的泰勒展开还说明了幂级数的和函数无论怎样扩充定义域, 它在收敛圆周上一定有奇点.
设幂级数 $\sumf0c_n(z-z_0)^n$ 的收敛半径为 $R>0$, 且在 $|z|<R$ 时和函数等于函数 $f(z)$.
注意到 $f(z)$ 在 $|z-z_0|<d$ 上可以展开为幂级数, 其中 $d$ 是 $z_0$ 到 $f(z)$ 最近奇点的距离.
由泰勒展开的唯一性可知该幂级数就是 $\sumf0c_n(z-z_0)^n$, 从而 $R=d$, $f(z)$ 一定在 $|z|=R$ 上有奇点.


\subsection{泰勒展开在级数中的应用\optional}

解析函数的泰勒展开可以帮助我们计算级数的和.

\begin{example}
  计算级数 $\sumf0 \frac{(8\ii)^n}{n!}$ 的和.
\end{example}

\begin{solution}
  由 $\ee^z$ 的泰勒展开可知
  \[
     \sumf0 \frac{(8\ii)^n}{n!}
    =\ee^{8\ii}
    =\cos 8+\ii \sin 8.
  \]
\end{solution}

但是对于幂级数收敛圆周上的收敛点, 如何计算相应的和呢?
我们有如下定理:\footnote{%
  \begin{tikzpicture}[overlay,xshift=.85\textwidth,yshift=-2.8\baselineskip]
    \def\r{1}
    \def\t{50}
    \def\d{2*\r*cos(\t)}
    \begin{scope}[rotate=30]
      \coordinate (A) at (\r,0);
      \coordinate (B) at ({\r-\d*cos(\t)},{\d*sin(\t)});
      \coordinate (C) at ({\r-\d*cos(\t)},{-\d*sin(\t)});
      \filldraw[thick,cstfill1] ($(A)!.5!(B)$)--(A)--($(A)!.5!(C)$)--(0,0)--cycle;
      \draw[cstdash] (0,0)--(A);
      \draw[thick] (B)--(A)--(C);
      \draw[cstcurve,main] (0,0) circle (\r);
      \fill[cstdot,fourth] (A) circle node[right] {$a$};
    \end{scope}
    \draw[cstaxis] (-1.5,0)--(1.5,0);
    \draw[cstaxis] (0,-1.3)--(0,1.5);
  \end{tikzpicture}%
  相应地, 前述\thmAF 被称为\emph{阿贝尔第一定理}. \thmAS 结论的一般\\
  形式为: 对任意 $0\le \theta<\dfrac\cpi2$,
  \begin{align*}
    \lim_{\substack{z\ra a,|z|<R\\-\theta\le\arg (1-z/a)\le\theta}}f(z)=f(a),\qquad\qquad\qquad&&&
  \end{align*}
  也就是说, $z$ 在右图阴影区域内趋于 $a$.
}

\begin{theorem}[阿贝尔第二定理]
  \label{thm:abel-second}
  设幂级数 $f(z)=\sumf0 c_nz^n$ 的收敛半径为 $R$.
  若 $f(z)$ 在收敛圆周 $|z|=R$ 上一点 $z=a$ 处收敛, 则
  \[
    \lim_{t\ra 1^-}f(ta)=f(a).
  \]
\end{theorem}

注意一般情形下, 即使极限 $\liml_{t\ra 1^-}f(ta)$ 存在也不能保证幂级数在 $a$ 处一定收敛.
例如 $f(z)=\sumf0 (-1)^nz^n$ 在 $1$ 处的左极限存在且为 $\dfrac12$, 但该幂级数在 $1$ 处发散.

\begin{proof}
  设幂级数 $g(z)=f(az)$, 则 $g(z)$ 的收敛半径为 $1$, 我们只需证明 $\liml_{t\ra 1^-} g(t)=g(1)$.
  因此我们可不妨设 $R=1,a=1$.

  对于 $0<t<1$, 令
  \[
    S_N=\suml_{n=0}^N c_n,\quad 
    T_N=\suml_{n=1}^N c_n(t^n-1).
  \]
  对任意 $\varepsilon>0$, 存在正整数 $M$ 使得当 $N>M$ 时, $|S_N-f(1)|<\varepsilon$.
  于是
  \begin{align*}
     T_N&
    =\sum_{n=0}^M c_n(t^n-1)+\sum_{n=M+1}^N(S_n-S_{n-1})(t^n-1)\\&
    =\sum_{n=0}^M c_n(t^n-1)+\sum_{n=M+1}^NS_n(t^n-1)-\sum_{n=M}^{N-1}S_n(t^{n+1}-1)\\&
    =\sum_{n=0}^M c_n(t^n-1)+S_M(1-t^M)+S_N(t^N-1)+(1-t)\sum_{n=M}^{N-1}S_nt^n\\&
    =\sum_{n=0}^M c_n(t^n-t^M)+S_N(t^N-1)+(t^M-t^N)f(1)+(1-t)\sum_{n=M}^{N-1}\bigl(S_n-f(1)\bigr)t^n
  \end{align*}
  满足
  \[
    |T_N|\le \Bigl|\sum_{n=0}^M c_n(t^n-t^M)+\bigl(S_N-f(1)\bigr)(t^N-1)\Bigr|+(1-t^M)|f(1)|+(t^M-t^N)\varepsilon.
  \]
  注意到
  \[
    \liml_{N\ra\infty} S_N=f(1),\quad \liml_{N\ra\infty} T_N=f(t)-f(1).
  \]
  令 $N\ra\infty$, 我们得到
  \[
    |f(t)-f(1)|\le \Bigl|\sum_{n=0}^M c_n(t^n-t^M)\Bigr|+(1-t^M)|f(1)|+t^M \varepsilon.
  \]
  令 $t\ra1^-$ 并由 $\varepsilon$ 的任意性可知 $\lim\limits_{t\ra 1^-}f(t)=f(1)$.
\end{proof}

\begin{example}
  计算级数 $\sumf0 \dfrac{(-1)^n}{3n+1}$ 的和.
\end{example}

\begin{solution}
  该级数是交错级数, 因此收敛.
  设
  \[
    \omega=\ee^{\frac{2\cpi\ii}3}=\frac{-1+\sqrt 3i}2
  \]
  是三次单位根, 则 $\omega^2=\ov\omega$, $1+\omega+\omega^2=0$.
  当 $|z|<1$ 时,
  \[
    \ln(1+z)=\sumf1 \frac{(-1)^{n+1}z^n}{n}.
  \]
  为了消去 $3n$ 和 $3n+2$ 幂次的项, 考虑
  \begin{align*}
    f(z)&=\ln(1+z)+\omega^2\ln(1+\omega z)+\omega\ln(1+\omega^2 z)\\
    &=\sumf1 \frac{(-1)^{n+1}(1+\omega^{n+2}+\omega^{2n+4})z^n}{n}
    =3\sumf0 \frac{(-1)^n z^{3n+1}}{3n+1}.
  \end{align*}
  显然 $f(z)$ 在 $z=1$ 处连续, 因此由\thmAS 可得
  \[
     \sumf1 \dfrac{(-1)^n}{3n+1}
    =\frac13 f(1)
    =\frac13\bigl(\ln2+\omega^2\ln(1+\omega)+\omega\ln(1+\omega^2)\bigr)
    =\frac13\ln2+\frac{\sqrt3}9\cpi.
  \]
\end{solution}


\subsection{泰勒展开成立的范围\optional}
\label{ssec:taylor-expansion-radius}

本节中我们来研究泰勒展开成立的最大圆域半径 $d$ 与泰勒级数的收敛半径 $R$ 的关系, 其中 $d$ 等于 $z_0$ 与 $f(z)$ 最近奇点的距离.

\begin{example}
  函数
  \[
    f(z)=\begin{cases}
      \ee^z,&z\neq 1;\\
      0,&z=1
    \end{cases}
  \]
  的麦克劳林展开为
  \[
    f(z)=\sumf0 \dfrac{z^n}{n!},\quad |z|<d=1,
  \]
  而右侧幂级数
  \[
    g(z)=\sumf0 \dfrac{z^n}{n!}=\ee^z
  \]
  的收敛半径为 $R=+\infty$.
  可以看出
  \[
    \lim_{z\ra 1} f(z)=\lim_{z\ra 1}g(z)=g(1)=\ee
  \]
  不等于 $f(1)$.
\end{example}

\begin{example}
  设
  \[
    f(z)=\ln(z-1-\ii),\quad 
    g(z)=\ln\Bigl(1-\frac z{1+\ii}\Bigr)+\ln(-1-\ii).
  \]
  不难知道 $f(z)$ 和 $g(z)$ 的实部相等, 虚部相等或相差 $2\cpi$.
  分情况讨论可知
  \begin{align*}
    f(z)=\begin{cases}
      g(z),&\text{若}\ \arg(z-1-\ii)\in \bigl(-\cpi,\dfrac\cpi4\bigr];\\
      g(z)+2\cpi\ii,&\text{若}\ \arg(z-1-\ii)\in \bigl(\dfrac\cpi4,\cpi\bigr].
    \end{cases}
  \end{align*}

  \begin{figure}[!hbt]
    \centering
    \begin{tikzpicture}
      \filldraw[cstcurve,fourth,fill=fourth!15] (0,0) circle ({sqrt(2)});
      \filldraw[cstcurve,main,cstfill1] (0,0) circle (1);
      \draw[cstdash,second] (1,1)--(-3,1)
        node[below right] {$f=g$}
        node[above right] {$f\neq g$};
      \draw[cstra,third] (0,0)--node[above left,shift={(.2,0)}] {$1$}({cos(35)},{sin(35)});
      \draw[cstra,third] (0,0)--({sqrt(2)*cos(-20)},{sqrt(2)*sin(-20)}) node[right,shift={(-.1,0)}] {$\sqrt2$};
      \draw[cstdash,second] (1,1)--(2.3,2.3);
      \fill[cstdot,third] (1,1) circle
        node[right,shift={(.1,0)}] {$1+\ii$};
      \draw[cstaxis] (-3,0)--(3,0);
      \draw[cstaxis] (0,-2)--(0,2.5);
    \end{tikzpicture}
    \caption{$f(z)$ 与 $g(z)$ 在不同区域的表现}
  \end{figure}

  函数 $f(z)$ 在 $\Re(z-1-\ii)\le 0$ 之外解析, 因此其麦克劳林展开成立的半径为 $1$.
  当 $|z|<d=1$ 时, 由对数函数的麦克劳林展开可知
  \[
    f(z)=g(z)=\ln(-1-\ii)-\sumf1 \frac1{n(1+\ii)^n}z^n,
  \]
  而右侧幂级数的收敛半径为 $\sqrt2$, 和函数为 $g(z)$.
  当 $z$ 属于圆域 $D=\set{z:|z|<\sqrt2}$ 的子集
  \[
    D\cap\set{z: \arg(z-1-\ii)\in\bigl(\frac\cpi4,\cpi\bigr]}
    =D\cap\set{z: \Im z\ge1}
  \]
  时, $f(z)=g(z)+2\cpi\ii $. 因此
  \[
    \liml_{\substack{z\ra \ii\\|z|<1}} f(z)=g(\ii)=-\cpi\ii
  \]
  不等于 $f(\ii)$.
\end{example}


可以看出在这两个例子中, 之所以泰勒级数的收敛半径 $R>d$, 是因为 $f(z)$ 在 $|z-z_0|=d$ 上的奇点是可以``消去''的, 只需要在奇点的一个邻域内将 $f(z)$ 换成泰勒级数的和函数 $g(z)$, $f(z)$ 便可以``解析延拓''到更大的区域.
若奇点 $z=a$ 无法通过上述方式消除, 则一定有 $R=d$.

\begin{theorem}
  设 $a$ 是函数 $f(z)$ 距离解析点 $z_0$ 最近的奇点, $d=|a-z_0|$.
  若极限
  \[
    \liml_{\substack{z\ra a\\|z-z_0|<d}} f(z)
  \]
  不存在, 则 $f(z)$ 在 $z_0$ 处泰勒级数的收敛半径等于 $d$.
\end{theorem}

\begin{example}
  设 $f(z)=\dfrac{p(z)}{q(z)}$ 是有理函数, 且分子分母没有公共零点.
  对于 $q(z)$ 的零点 $z=a$, 我们有 $\liml_{z\ra a}f(z)=\infty$. 因此 $f(z)$ 在解析点 $z_0$ 处泰勒级数的收敛半径就是 $z_0$ 到最近奇点的距离 $d$.
\end{example}

\begin{example}
  设 $f(z)=\ee^{\frac1z}$.
  由于 $\liml_{x\ra 0^+}f(x)=\infty$, 因此 $f(z)$ 在 $z_0\neq 0$ 处泰勒级数的收敛半径就是 $z_0$ 到奇点 $0$ 的距离 $d=|z_0|$.
\end{example}



\section{洛朗级数}

\subsection{双边幂级数}

若函数 $f(z)$ 在 $z_0$ 处解析, 则 $f(z)$ 可以展开成 $z-z_0$ 的幂级数.
若 $f(z)$ 在 $z_0$ 处不解析呢?
此时 $f(z)$ 一定不能展开成 $z-z_0$ 的幂级数, 然而它却可能可以展开为\noun{双边幂级数}\footnote{
  非负幂次部分叫作它的\emph{解析部分}, 负幂次部分叫作它的\emph{主要部分}.
}
\[
   \sumff c_n(z-z_0)^n
  =\underbrace{\sumf1 c_{-n}(z-z_0)^{-n}}_{\text{\normalsize \noun{负幂次部分}}}
    +\underbrace{\sumf0 c_n(z-z_0)^n}_{\text{\normalsize \noun{非负幂次部分}}}.
\]

为了保证双边幂级数的收敛范围有一个好的性质以便于我们使用, 我们对它的敛散性作如下定义:
\begin{definition}
  若双边幂级数的非负幂次部分和负幂次部分作为函数项级数都收敛, 则我们称这个双边幂级数\noun{收敛}. 否则称之为\noun{发散}.
\end{definition}

注意双边幂级数的敛散性不能像幂级数那样通过部分和形成的数列的极限来定义,
因为使用不同的部分和选取方式会影响到其敛散性和极限值.

现在我们来研究双边幂级数的敛散性.
设 $\sumf0 c_n(z-z_0)^n$ 的收敛半径为 $R_2$, 则它在 $|z-z_0|<R_2$ 内收敛, 在 $|z-z_0|>R_2$ 内发散.

对于负幂次部分, 令 $\zeta=\dfrac1{z-z_0}$, 则负幂次部分是 $\zeta$ 的一个幂级数 $\sumf1 c_{-n}\zeta^n$.
设该幂级数的收敛半径为 $R$, 则它在 $|\zeta|<R$ 内收敛, 在 $|\zeta|>R$ 内发散.
设 $R_1=\dfrac1R$, 则负幂次部分在 $|z-z_0|>R_1$ 内收敛, 在 $|z-z_0|<R_1$ 内发散.

\begin{enuma}
  \item 若 $R_1>R_2$, 则该双边幂级数处处不收敛.
  \item 若 $R_1=R_2$, 则该双边幂级数只在圆周 $|z-z_0|=R_1$ 上可能有收敛的点. 此时没有收敛域.
  \item 若 $R_1<R_2$, 则该双边幂级数在 $R_1<|z-z_0|<R_2$ 内收敛, 在 $|z-z_0|<R_1$ 和 $>R_2$ 内发散, 在圆周 $|z-z_0|=R_1$ 和 $R_2$ 上既可能发散也可能收敛.
\end{enuma}
因此\alert{双边幂级数的收敛域为圆环域 $R_1<|z-z_0|<R_2$}.

当 $R_1=0$ 或 $R_2=+\infty$ 时, 圆环域的形状会有所不同.
\begin{figure}[!htb]
  \centering
  \begin{tikzpicture}
    \begin{scope}[xshift=-40mm]
      \filldraw[cstcurve,draw=main,cstfill1] (0,0) circle (1.5);
      \coordinate (A) at ({1.5*cos(35)},{1.5*sin(35)});
      \draw[second,cstra] (0,0)--(A)
        node[pos=.4,below right] {$R_2$};
      \filldraw[cstdote,draw=main] (0,0) circle
        node[left,third] {$z_0$}
        node[below=15mm] {$0<|z-z_0|<R_2$};
    \end{scope}
    \begin{scope}
      \fill[cstfille1] (0,0) circle (1.5);
      \filldraw[cstcurve,fill=white,draw=main] (0,0) circle (1); 
      \coordinate (A) at ({cos(35)},{sin(35)});
      \draw[second,cstra] (0,0)--(A)
        node[pos=.4,below right] {$R_1$};
      \fill[cstdot,fill=second] (0,0) circle
        node[left,third] {$z_0$}
        node[below=15mm] {$R_1<|z-z_0|<+\infty$};
    \end{scope}
    \begin{scope}[xshift=40mm]
      \fill[cstfille1] (0,0) circle (1.5);
      \filldraw[cstdote,draw=main] (0,0) circle
        node[left=2pt,third,fill=white,inner sep=0pt] {$z_0$}
        node[below=15mm] {$0<|z-z_0|<+\infty$};
    \end{scope}
  \end{tikzpicture}
  \caption{特殊的圆环域}
\end{figure}

双边幂级数的非负幂次部分和负幂次部分在收敛圆环域内都收敛, 并注意到 $\zeta=\dfrac1{z-z_0}$ 关于 $z$ 解析.
因此它们的和函数都解析, 且可以逐项求导、逐项积分.
从而\alert{双边幂级数的和函数是解析的, 且可以逐项求导、逐项积分}.

\begin{example}
  求双边幂级数
  \[
    \sumf1 \frac{2^n}{z^n}+\sumf0 \frac{z^n}{(2+\ii)^n}
  \]
  的收敛域及和函数.
\end{example}

\begin{solution}
  不难知道, 非负幂次部分收敛域为 $|z|<|2+\ii|=\sqrt5$, 负幂次部分收敛域为 $|z|>|2|=2$. 
  因此该双边幂级数的收敛域为 $2<|z|<\sqrt5$.
  此时
  \[
     \sumf1 \frac{2^n}{z^n}+\sumf0 \frac{z^n}{(2+\ii)^n}
    =\frac{\dfrac2z}{1-\dfrac2z}+\frac1{1-\dfrac z{2+\ii}}
    =\frac{-\ii z}{(z-2)(z-2-\ii)}.
  \]
\end{solution}


\subsection{洛朗展开的形式}

双边幂级数的和函数是其收敛圆环域内的解析函数.
反过来, 在圆环域内解析的函数也一定能展开为双边幂级数, 被称为\noun{洛朗级数}.
例如 $f(z)=\dfrac1{z(1-z)}$ 在 $z=0,1$ 以外解析.
在圆环域 $0<|z|<1$ 内,
\[f(z)=\frac1z+\frac1{1-z}=\frac1z+1+z+z^2+z^3+\cdots\]
在圆环域 $1<|z|<+\infty$ 内,
\[f(z)=\frac1z-\frac1z\cdot\frac1{1-\dfrac1z}=-\frac1{z^2}-\frac1{z^3}-\frac1{z^4}-\cdots\]

现在我们来证明洛朗级数的存在性并得到洛朗展开式.
设 $f(z)$ 在圆环域 $R_1<|z-z_0|<R_2$ 内解析.
设 $R_1<r<R<R_2$, 
\[
  \color{main}{K_1:|z-z_0|=r},\quad \color{fifth}{K_2:|z-z_0|=R}
\]
是该圆环域内的两个圆周.
对于 $r<|z-z_0|<R$, 由\thmCCC 和\thmCIH,
\[
   f(z)
  =\frac1{2\cpi\ii}\oint_{K_2}\frac{f(\zeta)}{\zeta-z}\d\zeta
  -\frac1{2\cpi\ii}\oint_{K_1}\frac{f(\zeta)}{\zeta-z}\d\zeta.
\]

\begin{figure}[!htb]
  \centering
  \begin{tikzpicture}
    \filldraw[cstcurve,fifth,cstfill3] (0,0) circle (1.6);
    \filldraw[cstcurve,main,fill=white] (0,0) circle (.8);
    \node[third,inner sep=1pt] (Z) at (-1.3,-.3) {$\zeta$};
    \coordinate (Z1) at ({1.6*cos(225)},{1.6*sin(225)});
    \coordinate (Z2) at ({.8*cos(225)},{.8*sin(225)});
    \draw[cstra,fifth] (Z)--(Z1);
    \draw[cstra,main] (Z)--(Z2);
    \fill[cstdot,fifth] (Z1) circle;
    \fill[cstdot,main] (Z2) circle;
    \draw[fifth,cstra] (0,0)--({1.6*cos(35)},{1.6*sin(35)})
      node[right] {$K_2$}
      node[pos=.75,below] {$R$};
    \draw[main,cstra] (0,0)--({.8*cos(290)},{.8*sin(290)})
      node[below] {$K_1$}
      node[pos=.4,right,shift={(-.05,0)}] {$r$};
    \fill[cstdot,third] (0,0) circle;
  \end{tikzpicture}
  \caption{函数在圆环域内的洛朗展开}
\end{figure}

和泰勒级数的推导类似,
\[
   \frac1{2\cpi\ii}\oint_{K_2}\frac{f(\zeta)}{\zeta-z}\d\zeta
  =\sumf0 \biggl(\frac1{2\cpi\ii}
    \oint_{K_2}\frac{f(\zeta)}{(\zeta-z_0)^{n+1}}\d\zeta
  \biggr)(z-z_0)^n
\]
可以表达为幂级数的形式.
类似地, $K_1$ 上的积分部分可由
\[
   \frac1{z-\zeta}-\frac1{z-\zeta}\Bigl(\frac{\zeta-z_0}{z-z_0}\Bigr)^N
  =\frac1{z-z_0}\cdot\frac{1-\Bigl(\dfrac{\zeta-z_0}{z-z_0}\Bigr)^N}{1-\dfrac{\zeta-z_0}{z-z_0}}
  =\sum_{n=0}^{N-1}\frac{(\zeta-z_0)^n}{(z-z_0)^{n+1}}
\]
得到
\[
  -\frac1{2\cpi\ii}\oint_{K_1}\frac{f(\zeta)}{\zeta-z}\d\zeta
  =R_N+\sum_{n=1}^N\biggl(\frac1{2\cpi\ii}\oint_{K_1}
    \frac{f(\zeta)\d\zeta}{(\zeta-z_0)^{-n+1}}
   \biggr)(z-z_0)^{-n},
\]
其中
\[
  R_N(z)=\frac1{2\cpi\ii}\oint_{K_1}\frac{f(\zeta)}{z-\zeta}\cdot\Bigl(\frac{\zeta-z_0}{z-z_0}\Bigr)^{N-1}\d\zeta.
\]
由于 $f(\zeta)$ 在 $D\supseteq K_1$ 内解析, 从而在 $K_1$ 上连续且有界.
设对任意 $\zeta\in K$, $|f(\zeta)|\le M$, 则由\thmGrowUp 和 $|\zeta-z_0|=r<|z-z_0|$ 可知当 $N\ra \infty$ 时,
\begin{align*}
  |R_N(z)|&\le\frac 1{2\cpi}\oint_{K_1}\abs{\frac{f(\zeta)}{z-\zeta}\cdot\Bigl(\frac{\zeta-z_0}{z-z_0}\Bigr)^N}\d s\\
  &\le\frac 1{2\cpi}\cdot\frac M{|z-z_0|-r}\cdot\abs{\frac{\zeta-z_0}{z-z_0}}^N\cdot 2\cpi r\ra 0.
\end{align*}
故
\[
  f(z)=
    \sumf0 \biggl(\frac1{2\cpi\ii}\oint_{K_2}\frac{f(\zeta)}{(\zeta-z_0)^{n+1}}\d\zeta\biggr)(z-z_0)^n
    +\sumf1 \biggl(\frac1{2\cpi\ii}\oint_{K_1}\frac{f(\zeta)}{(\zeta-z_0)^{-n+1}}\d\zeta\biggr)(z-z_0)^{-n},
\]
其中 $r<|z-z_0|<R$.
由\thmCCC, $K_1,K_2$ 可以换成任意一条在圆环域中内部包含 $z_0$ 的闭路 $C$.
由此我们得到\noun{洛朗展开}:

\begin{theorem}
  \label{thm:laurent-expansion}
  若 $f(z)$ 在圆环域 $R_1<|z-z_0|<R_2$ 内解析, 则 $f(z)$ 可以在该圆环域内展开成双边幂级数
  \[
    f(z)=\sumff \biggl(\frac1{2\cpi\ii}\oint_C \frac{f(\zeta)}{(\zeta-z_0)^{n+1}}\d\zeta\biggr)(z-z_0)^n,
  \]
  其中 $C$ 是该圆环域中内部包含 $z_0$ 的闭路.
\end{theorem}

注意这里和泰勒展开不同, 系数不能表达为 $f(z)$ 的高阶导数形式, 因为 $f(z)$ 在 $C$ 的内部不一定解析.


\subsection{洛朗展开的计算方法}

设在圆环域 $R_1<|z-z_0|<R_2$ 内的解析函数 $f(z)$ 可以表达为双边幂级数
\[
  f(z)=\sumff c_n(z-z_0)^n.
\]
逐项积分并由\thmref{定理}{thm:closed-power-integral} 得到
\[
   \oint_C \frac{f(\zeta)\d\zeta}{(\zeta-z_0)^{m+1}}
  =\sum_{n=-\infty}^\infty c_n\oint_C (\zeta-z_0)^{n-m-1}\d\zeta
  =2\cpi\ii c_m.
\]
从而 $c_m$ 就是洛朗展开的系数.
因此 $f(z)$ 在一固定圆环域内的\alert{双边幂级数展开是唯一的, 它一定是洛朗级数}.
注意, 解析函数在不同圆环域内的洛朗展开可能是不同的, 这和双边幂级数展开的唯一性无关.

若用积分来计算洛朗级数的系数, 过程往往较为繁琐.
因此我们通常利用双边幂级数展开的唯一性, 通过\alert{使用双边幂级数的代数、求导、求积分运算}来得到洛朗级数.

\begin{example}
  将 $f(z)=\dfrac{\ee^z-1}{z^2}$ 展开为以 $0$ 为中心的洛朗级数.
\end{example}

\begin{solution}
  由于 $f(z)$ 在 $0$ 以外处处解析, 因此它可以在 $0<|z|<+\infty$ 内展开成洛朗级数:
  \[
     \frac{\ee^z-1}{z^2}
    =\frac1{z^2}\Bigl(z+\frac{z^2}{2!}+\frac{z^3}{3!}+\cdots\Bigr)
    =\frac1z+\frac1{2!}+\frac{z}{3!}+\cdots
    =\frac1z+\sumf0 \frac1{(n+2)!}z^n.
  \]
\end{solution}

\begin{example}
  在下列圆环域中把 $f(z)=\dfrac1{(z-1)(z-2)}$ 展开为洛朗级数:
  \begin{subexample}(3)
    \item $0<|z|<1$;
    \item $1<|z|<2$;
    \item $2<|z|<+\infty$.
  \end{subexample}
\end{example}

\begin{solution}
  由于 $f(z)$ 的奇点为 $z=1,2$, 因此在这些圆环域内 $f(z)$ 都可以展开为洛朗级数.
  注意到
  \[
    f(z)=\frac1{z-2}-\frac1{z-1},
  \]
  我们可以根据 $|z|$ 的范围来将其展开成等比级数.
  \begin{enumr}
    \item 由于 $|z|<1,\abs{\dfrac z2}<1$, 因此
      \begin{align*}
         f(z)&
        =-\frac1{2-z}+\frac1{1-z}
        =-\half\cdot\frac1{1-\dfrac z2}+\frac1{1-z}
        =-\half\sumf0 \Bigl(\frac z2\Bigr)^n+\sumf0 z^n\\&
        =\sumf0 \Bigl(1-\frac1{2^{n+1}}\Bigr)z^n
        =\frac12+\frac34z+\frac78z^2+\cdots
      \end{align*}
    \item 由于 $\abs{\dfrac 1z}<1,\abs{\dfrac z2}<1$, 因此
      \begin{align*}
         f(z)&
        =\frac1{1-z}-\frac1{2-z}
        =-\frac1z\cdot\frac1{1-\dfrac1z}-\half\cdot\frac1{1-\dfrac z2}\\&
        =-\frac1z\sumf0 \Bigl(\frac1z\Bigr)^n-\half\sumf0 \Bigl(\frac z2\Bigr)^n
        =-\sumf1 \frac1{z^n}-\sumf0 \frac1{2^{n+1}}z^n\\&
        =\cdots-\frac1{z^2}-\frac1z-\half -\frac14z-\frac18z^2-\cdots
      \end{align*}
    \item 由于 $\abs{\dfrac 1z}<1,\abs{\dfrac 2z}<1$, 因此
      \begin{align*}
         f(z)&
        =\frac1{1-z}-\frac1{2-z}
        =-\frac1z\cdot\frac1{1-\dfrac1z}+\frac1z\cdot\frac1{1-\dfrac2z}\\&
        =-\frac1z\sumf0 \Bigl(\frac1z\Bigr)^n+\frac1z\sumf0 \Bigl(\frac2z\Bigr)^n
        =\sumf0 \frac{2^n-1}{z^{n+1}}
        =\frac1{z^2}+\frac3{z^3}+\frac7{z^4}+\cdots
      \end{align*}
  \end{enumr}
\end{solution}

\begin{example}
  将函数 $f(z)=\dfrac{z+1}{(z-1)^2}$ 在圆环域 $0<|z|<1$ 内展开成洛朗级数.
\end{example}

\begin{solution}[解法一]
  由于
  \[
     f(z)
    =\frac{z-1+2}{(z-1)^2}
    =\frac1{z-1}+\frac{2}{(z-1)^2}
    =-\frac1{1-z}+2\Bigl(\frac1{1-z}\Bigr)',
  \]
  因此当 $0<|z|<1$ 时,
  \begin{align*}
      f(z)&
    =-\sumf0 z^n+2\Bigl(\sumf0 z^n\Bigr)'
    =-\sumf0 z^n+2\sumf1 nz^{n-1}\\&
    =-\sumf0 z^n+2\sumf0 (n+1)z^n
    =\sumf0 (2n+1)z^n.
  \end{align*}
\end{solution}

\begin{solution}[解法二]
  由于 $0<|z|<1$ 时,
  \[
     \frac{1}{(z-1)^2}
    =\Bigl(\frac1{1-z}\Bigr)'
    =\Bigl(\sumf0 z^n\Bigr)'
    =\sumf1 nz^{n-1},
  \]
  因此
  \[
     f(z)
    =z\sumf1 nz^{n-1}+\sumf1 nz^{n-1}
    =\sumf1 nz^n+\sumf0 (n+1)z^n
    =\sumf0 (2n+1)z^n.
  \]
\end{solution}

可以看出, 有理函数的洛朗展开通常需要将其分解成部分分式 $\dfrac1{(z-a)^m}$ 的线性组合.
根据 $z$ 所处的圆环域, 选取 $\dfrac{z-z_0}{a-z_0}$ 和 $\dfrac{a-z_0}{z-z_0}$ 中模小于 $1$ 的数作为公比来求得 $\dfrac1{z-a}$ 的洛朗展开.
然后对其求 $m-1$ 阶导, 最后合并相同幂次项的系数.

洛朗展开的一些特点可以帮助我们检验计算的正确性.
\begin{enuma}
  \item 若 $f(z)$ 在 $|z-z_0|<R_2$ 内解析, 则 $f(z)$ 可以展开为泰勒级数. 由唯一性可知泰勒级数等于洛朗级数, 因此此时洛朗展开一定没有负幂次项.
  \item 若有理函数(分子分母没有公共零点) 在圆周 $|z-z_0|=R_1>0$ 和 $|z-z_0|=R_2>0$ 上都有奇点, 则在圆环域 $R_1<|z-z_0|<R_2$ 上的洛朗展开一定有无穷多负幂次和无穷多正幂次项.
  \item 有理函数在解析区域 $0<|z-z_0|<r$ 的洛朗展开最多只有有限多负幂次项, 且最低负幂次是 $z-z_0$ 在分母因式分解中出现的次数; 在解析区域 $R<|z-z_0|<+\infty$ 的洛朗展开最多只有有限多正幂次项, 且最高正幂次是分子次数减去分母次数.
\end{enuma}

更多有关结论可阅读 \ref{ssec:rational-function-expansion}.

\begin{exercise}
  将函数 $f(z)=\dfrac{z+1}{(z-1)^2}$ 在圆环域 $1<|z|<+\infty$ 内展开成洛朗级数.
\end{exercise}

注意到当 $n=-1$ 时, 洛朗展开的系数
\[
  c_{-1}=\frac1{2\cpi\ii}\oint_C f(\zeta)\d\zeta,
\]
因此洛朗展开可以用来帮助计算函数的积分.

\begin{example}
  求 $\doint_{|z|=3}\frac1{z(z+1)^2}\d z$.
\end{example}

\begin{solution}
  注意到闭路 $|z|=3$ 落在 $1<|z+1|<+\infty$ 内, 我们在这个圆环域内求 $f(z)=\dfrac1{z(z+1)^2}$ 的洛朗展开.
  \begin{align*}
     f(z)&
    =\frac1{z(z+1)^2}=\frac1{(z+1)^3}\cdot\frac1{1-\dfrac1{z+1}}\\&
    =\frac1{(z+1)^3}\sumf0 \frac1{(z+1)^n}
    =\sumf3 \frac1{(z+1)^n}
  \end{align*}
  故
  \[
    \oint_C f(z)\d z=2\cpi\ii  c_{-1}=0.
  \]
\end{solution}

\begin{figure}[!hbt]
  \centering
  \begin{minipage}{.48\textwidth}
    \centering
    \begin{tikzpicture}
      \begin{scope}[scale=.6]
        \fill[cstfille1] circle (3.5);
        \fill[white] (-1,0) circle (1);
        \draw[cstcurve,main] (-1,0) circle (1);
        \draw[
          cstcurve,
          fourth,
          decoration={
            markings,
            mark=at position .125 with {
              \arrow[rotate=-7]{Stealth}
            }
          },
          postaction={decorate}] circle (3);
        \draw[cstaxis] (-4,0)--(4,0);
        \draw[cstaxis] (0,-4)--(0,4);
        \coordinate (A) at (-1,0);
      \end{scope}
      \fill[cstdot,fourth] (A) circle node[below] {$-1$};
      \fill[cstdot,fourth] circle;
    \end{tikzpicture}
    \caption{$\dfrac1{z(z+1)^2}$ 解析的圆环域}
  \end{minipage}
  \begin{minipage}{.48\textwidth}
    \centering
    \begin{tikzpicture}
      \begin{scope}[scale=.8]
        \filldraw[cstcurve,main,cstfill1] (0,0) circle ({sqrt(pi)});
        \draw[cstaxis] (-3,0)--(3,0);
        \draw[cstaxis] (0,-2.5)--(0,2.5);
        \foreach \i in {0,1,2}
          \coordinate (A\i) at ({(\i-1)*sqrt(pi)},0);
      \end{scope}
      \fill[cstdot,fourth] (A0) circle node[below left] {$-\sqrt\cpi$};
      \filldraw[cstdote,fourth,fill=white] (A1) circle;
      \fill[cstdot,fourth] (A2) circle node[below right] {$\sqrt\cpi$};
      \draw[
        cstcurve,
        fourth,
        decoration={
          markings,
          mark=at position .125 with {
            \arrow[rotate=-7]{Stealth}
          }
        },
        postaction={decorate}] circle (1);
    \end{tikzpicture}
    \caption{$\dfrac z{\sin z^2}$ 解析的圆环域}
  \end{minipage}
\end{figure}

\begin{example}
  求 $\doint_{|z|=1}\frac{z}{\sin z^2}\d z$.
\end{example}

可以看出, 该积分无法使用\thmCIH 直接计算.

\begin{solution}
  注意到闭路 $|z|=1$ 落在 $0<|z|<\sqrt \cpi$ 内, 我们在这个圆环域内求 $f(z)=\dfrac{z}{\sin z^2}$ 的洛朗展开:
  \[
     f(z)
    =\frac{z}{\sin z^2}
    =\frac{z}{z^2-\dfrac{z^6}{3!}+\dfrac{z^{10}}{5!}+\cdots}
    =\frac1z+\frac{z^3}6+\cdots
  \]
  故
  \[
    \oint_C f(z)\d z=2\cpi\ii  c_{-1}=2\cpi\ii .
  \]
\end{solution}

每次使用洛朗级数来计算积分略显繁琐, 因为实际上我们只需要 $c_{-1}$ 而并不需要其它系数.
针对不同的奇点特点, 我们的确有一些计算方法来直接得到 $c_{-1}$ 而不用求出整个洛朗展开, 这也引出了留数的概念.


\subsection{有理函数的泰勒展开和洛朗展开\optional}
\label{ssec:rational-function-expansion}

在讨论有理函数的泰勒展开和洛朗展开之前, 我们先将 \ref{ssec:taylor-expansion-radius}的结论推广到洛朗展开的情形.

\begin{theorem}
  设 $f(z)$ 在圆环域 $D:R_1<|z-z_0|<R_2$ 内解析, 其中 $0<R_1<R_2<+\infty$.
  \begin{enuma}
    \item 若 $f(z)$ 在 $|z-z_0|=R_1$ 有满足极限
    \[
      \liml_{\substack{z\ra a\\ z\in D}} f(z)
    \]
    不存在的奇点 $a$, 则 $f(z)$ 在圆环域 $R_1<|z-z_0|<R_2$ 内的洛朗展开有无穷多负幂次项.
    \label{enum:laurent-expansion-infinite-negative}
    \item 若 $f(z)$ 在 $|z-z_0|=R_2$ 有满足极限
    \[
      \liml_{\substack{z\ra a\\z\in D}} f(z)
    \]
    不存在的奇点 $a$, 则 $f(z)$ 在圆环域 $R_1<|z-z_0|<R_2$ 内的洛朗展开有无穷多正幂次项.
  \end{enuma}
\end{theorem}

\begin{proof}
  设 $f(z)$ 的洛朗展开为
  \[
    f(z)=\sumff c_n(z-z_0)^n.
  \]
  \begin{enuma}
    \item 若 $f(z)$ 的洛朗展开
    只有有限多个负幂次项, 且最低幂次为 $-m$, 则幂级数
    \[
        \sum_{n=-m}^\infty c_n(z-z_0)^{n+m}
      =\sumf0 c_{n-m}(z-z_0)^n
    \]
    的和函数 $g(z)$ 在 $|z-z_0|<R_2$ 内解析, 从而
    \[
      \liml_{\substack{z\ra a\\z\in D}} f(z)
      =\liml_{\substack{z\ra a\\z\in D}} \frac{g(z)}{(z-z_0)^m}
      =\frac{g(a)}{(a-z_0)^m}
    \]
    存在.
    这与假设矛盾, 因此 $f(z)$ 的洛朗展开有无穷多负幂次项.
    \item 考虑 $h(z)=f\Bigl(\dfrac1z+z_0\Bigr)$, 则 $h(z)$ 在圆环域 $\dfrac 1{R_2}<|z|<\dfrac 1{R_1}$ 内满足 \ref{enum:laurent-expansion-infinite-negative} 中条件, 从而其洛朗展开
    \[
      h(z)=\sumff c_nz^{-n}
    \]
    有无穷多负幂次项, 即 $f(z)$ 的洛朗展开有无穷多正幂次项.
    \qedhere
  \end{enuma}
\end{proof}

\begin{example}
  设 $f(z)=\dfrac{p(z)}{q(z)}$ 是有理函数, 且分子分母没有公共零点.
  若 $f(z)$ 在圆环域 $0<R_1<|z-z_0|<R_2<+\infty$ 内解析, 且在圆周 $|z-z_0|=R_1$ 和 $|z-z_0|=R_2$ 上都有奇点, 则 $f(z)$ 在圆环域 $R_1<|z-z_0|<R_2$ 内的洛朗展开有无穷多负幂次项和无穷多正幂次项.
\end{example}

不仅如此, 有理函数 $f(z)$ 在不同的圆环域内的洛朗展开系数还具有较为简单的关系.
根据 \ref{ssec:application-of-derivative}的讨论不难知道, $f(z)$ 可以拆分成
\[
  f(z)=g(z)+\sum_{j=1}^k \sum_{r=1}^{m_j} \frac{c_{j,r}}{(1-z/\lambda_j)^r},
\]
其中 $g(z)$ 是一个只有有限项的双边幂级数, $\lambda_j$ 为 $q(z)$ 的 $m_j$ 重非零零点.


\subsubsection{泰勒展开和幂级数}

假设 $q(0)\neq 0$.
此时 $g(z)$ 是多项式, $f(z)$ 有麦克劳林展开.
由
\[
  \frac1{1-z}=1+z+z^2+\cdots,\quad |z|<1,
\]
逐项求 $k-1$ 阶导数, 并将 $z$ 换成 $z/\lambda$ 得到
\[
   \frac1{(1-z/\lambda)^k}
  =\sumf0 \frac{(n+k-1)\cdots(n+2)(n+1)}{(k-1)!}\lambda^{-n}z^n,\quad |z|<|\lambda|.
\]
因此若 $f(z)$ 的麦克劳林展开为
\[
  f(z)=\sumf0 c_n z^n,\quad |z|<\min\{|\lambda_1|,\cdots,|\lambda_k|\},
\]
则除去至多有限项外, 麦克劳林展开的系数为
\[
  c_n=p_1(n)\lambda_1^{-n}+\cdots+p_k(n)\lambda_k^{-n},
\]
其中
\[
  p_j(n)=\sum_{r=1}^{m_j} (n+r-1)\cdots(n+2)(n+1)c_{j,r}
\]
是 $n$ 的 $m_j-1$ 次多项式.

反过来, 若一个幂级数 $\sumf0 c_nz^n$ 系数具有上述形式, 则它的和函数必然是有理函数, 且分母为 
\[
  q(z)=(z-\lambda_1)^{m_1}\cdots (z-\lambda_k)^{m_k},
\]
其中 $m_j$ 为多项式 $p_j$ 的次数 $+1$.


\subsubsection{洛朗展开}

由
\[
  \frac1{1-z}=-\Bigl(\frac1z+\frac1{z^2}+\frac1{z^3}+\cdots\Bigr),\quad |z|<1,
\]
逐项求 $k-1$ 阶导数, 并将 $z$ 换成 $z/\lambda$ 得到
\[
    \frac1{(1-z/\lambda)^k}
   =-\sum_{n<0} \frac{(n+k-1)\cdots(n+2)(n+1)}{(k-1)!}\lambda^{-n}z^n,\quad |z|>|\lambda|.
\]
因此 $f(z)$ 在其解析的圆环域 $r<|z|<R$ 内的洛朗展开为
\[
  f(z)=g(z)+\sum_{n\ge 0} \sum_{|\lambda_j|\ge R} p_j(n)\lambda_j^{-n}z^n
  -\sum_{n<0} \sum_{0<|\lambda_j|\le r} p_j(n)\lambda_j^{-n}z^n.
\]

由于 $p_j(n)$ 的形式可以从 $f(z)$ 在任一圆环域内的洛朗展开读出, 因此我们有如下结论.
  
\begin{theorem}
  \label{thm:rational-function-laurent}
  设有理函数 $f(z)$ 在其解析的圆环域 $r<|z|<R$ 内有前述形式的洛朗展开
  \[
    f(z)=g(z)+\sum_{n\ge 0} a_n z^n-\sum_{n<0} b_n z^n.
  \]
  \vspace{-\baselineskip}
  \begin{enuma}
    \item $f(z)$ 在其解析的另一圆环域 $0\le r'<|z|<R'\le r$ 内有洛朗展开
      \[
        f(z)=g(z)+\sum_{n\ge 0} (a_n+b_n-c_n) z^n-\sum_{n<0} c_n z^n,
      \]
      其中 $c_n$ 是 $b_n$ 通项形式中满足 $0<|\lambda_j|\le r'$ 的奇点 $\lambda_j$ 所对应的项.
    \item $f(z)$ 在其解析的另一圆环域 $R\le r'<|z|<R'\le+\infty$ 内有洛朗展开
    \[
      f(z)=g(z)+\sum_{n\ge 0} c_n z^n-\sum_{n<0} (a_n+b_n-c_n) z^n,
    \]
    其中 $c_n$ 是 $a_n$ 通项形式中满足 $|\lambda_j|\ge R'$ 的奇点 $\lambda_j$ 所对应的项.
  \end{enuma}
\end{theorem}
  
该定理给出的有理函数在不同圆环域内洛朗展开系数联系的结论, 可以用来帮助计算有理函数的洛朗展开. 
更多结论和例子参见\cite{YuanZhang2024}.

特别地, 当 $r'=0$ 或 $R'=+\infty$ 时, 可以得到如下推论. 

\begin{corollary}
  \label{cor:rational-function-laurent}
  设有理函数 $f(z)$ 在其解析的圆环域 $r<|z|<R$ 内有前述形式的洛朗展开
  \[
    f(z)=g(z)+\sum_{n\ge 0} a_n z^n-\sum_{n<0} b_n z^n.
  \]
  那么 $f(z)$ 有洛朗展开
  \begin{align*}
    f(z)&=g(z)+\sum_{n\ge 0} (a_n+b_n) z^n,\quad 0<|z|<\delta;\\
    f(z)&=g(z)-\sum_{n<0} (a_n+b_n) z^n,\quad |z|>X,
  \end{align*} 
  其中 $\delta$ 是 $f(z)$ 非零奇点的最小模, $X$ 是 $f(z)$ 奇点的最大模.
\end{corollary}

以上结论中将 $z$ 换成 $z-z_0$ 自然也成立.

\begin{example}
  将
  \[
    f(z)=\dfrac{z^3}{(z-1)(z-2)}
  \]
  在各个以 $0$ 为圆心的圆环域内展开为洛朗级数. 
\end{example}

\begin{solutionenum}
  \item 设 $0<|z|<1$, 则
  \[
      f(z)
    =z+3+\frac1{1-z}-\frac8{2-z}
    =3+z+\sum_{n\ge 0}(1-2^{-n+2})z^n
    =\sum_{n\ge 2}(1-2^{-n+2})z^n.
  \]
  \item 设 $1<|z|<2$. 由\thmref{定理}{thm:rational-function-laurent}, $c_n$ 是 $a_n=1-2^{-n+2}$ 中 $|\lambda|\ge 2$ 对应的项, 即 $c_n=-2^{-n+2}$. 从而
  \[
      f(z)
    =3+z-\sum_{n\ge0}2^{-n+2}z^n-\sum_{n\le-1}z^n
    =-\sum_{n\ge 2}2^{-n+2}z^n-\sum_{n\le 1}z^n.
  \]
  \item 设 $|z|>2$. 由\thmref{定理}{thm:rational-function-laurent} 或\thmref{推论}{cor:rational-function-laurent} 得到
  \[
      f(z)
    =3+z-\sum_{n\le -1}(1-2^{-n+2})z^n
    =-\sum_{n\le1}(1-2^{-n+2})z^n.
  \]
\end{solutionenum}

\begin{example}
  将 $f(z)=\dfrac1{(z-\eta)^2}+\dfrac1{z-\xi}$ 在各个以 $0$ 为圆心的圆环域内展开为洛朗级数, 其中 $|\eta|>|\xi|>0$. 
\end{example}

\begin{solutionenum}
  \item 设 $0<|z|<|\xi|$, 则
  \[
      f(z)
    =\frac1{\eta}\Bigl(\frac1{1-z/\eta}\Bigr)'-\frac1\xi\cdot\frac1{1-z/\xi}
    =\sum_{n\ge 0}\Bigl(\frac{n+1}{\eta^{n+2}}-\frac1{\xi^{n+1}}\Bigr)z^n.
  \]
  \item 设 $|\xi|<|z|<|\eta|$. 根据\thmref{定理}{thm:rational-function-laurent}, $c_n$ 是 $a_n=\dfrac{n+1}{\eta^{n+2}}-\dfrac1{\xi^{n+1}}$ 中 $|\lambda|\ge |\eta|$ 对应的项, 即 $c_n=\dfrac{n+1}{\eta^{n+2}}$. 从而
  \[
      f(z)
    =\sum_{n\ge 0}\frac{n+1}{\eta^{n+2}}z^n+\sum_{n\le -1}\frac1{\xi^{n+1}}z^n.
  \]
  \item 设 $|z|>2$. 由\thmref{定理}{thm:rational-function-laurent} 或\thmref{推论}{cor:rational-function-laurent} 得到
  \[
      f(z)
    =-\sum_{n\le -1}\Bigl(\frac{n+1}{\eta^{n+2}}-\frac1{\xi^{n+1}}\Bigr)z^n.
  \]
\end{solutionenum}



\section{孤立奇点}

我们根据奇点附近洛朗展开的特点来对其进行分类, 以便后面分类计算留数.

\subsection{孤立奇点的类型}

\begin{example}
  考虑函数 $f(z)=\dfrac1{\sin(1/z)}$.
  显然 $0$ 和 $z_k=\dfrac1{k\cpi}$ 是奇点, $k$ 是非零整数.
  因为 $\liml_{k\ra+\infty} z_k=0$, 所以 $0$ 的任何一个去心邻域内都有奇点.
  此时无法选取一个圆环域 $0<|z|<\delta$ 作 $f(z)$ 的洛朗展开, 我们不考虑这类奇点.
\end{example}

\begin{figure}[!hbt]
  \centering
  \begin{tikzpicture}
    \def\s{6}
    \foreach \i in {1,2,...,10}{
      \coordinate (A\i) at ({\s/\i/pi},0);
      \coordinate (B\i) at ({-\s/\i/pi},0);
      \draw[fifth] (A\i)--++(0,.2);
      \draw[fifth] (B\i)--++(0,.2);
    }
    \draw (A1) node[below] {$z_1$};
    \draw (A2) node[below] {$z_2$};
    \draw (B1) node[below] {$z_{-1}$};
    \draw (B2) node[below] {$z_{-2}$};
    \draw node[below right,inner sep=1pt] {$0$};
    \fill[fifth] (-.2,0) rectangle (.2,.2);
    \draw[cstaxis] (0,-1)--(0,1);
    \draw[cstaxis] (-{\s/2.5},0)--({\s/2.5},0);
  \end{tikzpicture}
  \caption{$\dfrac1{\sin(1/z)}$ 的奇点分布}
\end{figure}

\begin{definition}
  若 $z_0$ 是 $f(z)$ 的一个奇点, 且 $z_0$ 的某个邻域内没有其它奇点, 则称 $z_0$ 是 $f(z)$ 的一个\noun{孤立奇点}.
\end{definition}

\begin{exampleenum}
  \item $z=0$ 是 $\ee^{\frac1z},\dfrac{\sin z}z$ 的孤立奇点.
  \item $z=-1$ 是 $\dfrac1{z(z+1)}$ 的孤立奇点.
  \smallskip
  \item $z=0$ 不是 $\dfrac1{\sin(1/z)}$ 的孤立奇点.
\end{exampleenum}
\smallskip

若 $f(z)$ 只有有限多个奇点, 则这些奇点都是孤立奇点.

若 $f(z)$ 在孤立奇点 $z_0$ 的去心邻域 $0<|z-z_0|<\delta$ 内解析, 则 $f(z)$ 可以在该邻域作洛朗展开.
根据该洛朗级数负幂次部分的项数, 我们将孤立奇点分为三类:

\begin{table}[!htb]
  \centering
  \begin{tabular}{ccc}
    \topcolorrule
      \bf 孤立奇点类型&
      \bf 洛朗级数特点&
      $\liml_{z\ra z_0}f(z)$\\
    \topcolorrule
      可去奇点&
      没有负幂次部分&
      存在\\
    \midcolorrule
      $m$ 阶极点&
      \makecell{负幂次部分只有有限项非零\\最低次为 $-m$ 次}&
      $\infty$\\
    \midcolorrule
      本性奇点&
      负幂次部分有无限项非零&
      不存在且不为 $\infty$\\
    \bottomcolorrule
  \end{tabular}
  \caption{孤立奇点的分类}
\end{table}


\subsubsection{可去奇点}

\begin{definition}
  若 $f(z)$ 在孤立奇点 $z_0$ 去心邻域的洛朗展开没有负幂次部分, 即
  \[f(z)=c_0+c_1(z-z_0)+c_2(z-z_0)^2+\cdots,\quad 0<|z-z_0|<\delta,\]
  是幂级数, 则称 $z_0$ 是 $f(z)$ 的\noun{可去奇点}.
\end{definition}

设 $g(z)$ 为右侧幂级数的和函数, 则 $g(z)$ 在 $|z-z_0|<\delta$ 内解析,
且除 $z_0$ 外 $f(z)=g(z)$.
通过补充或修改定义 $f(z_0)=g(z_0)=c_0$, 可使得 $f(z)$ 也在 $z_0$ 处解析.
这就是``可去''的含义.\footnote{
  修改后的函数自然不再是原来的函数, 但在很多时候, 比如计算 $f(z)$ 绕闭路的积分时, 这种修改并不影响我们计算结果.
}

\begin{theorem}
  \label{thm:test-removable}
  若 $z_0$ 是 $f(z)$ 的孤立奇点, 则下列结论等价:
  \begin{enuma}
    \item $z_0$ 是 $f(z)$ 的可去奇点;
    \label{enum:removable}
    \item $\liml_{z\ra z_0}f(z)$ 存在;
    \label{enum:finite-limit}
    \item $\liml_{z\ra z_0}(z-z_0)f(z)=0$.
    \label{enum:zero-limit}
  \end{enuma}
\end{theorem}

\begin{proof}
  \ref{enum:removable}$\implies$\ref{enum:finite-limit}$\implies$\ref{enum:zero-limit} 是容易的.
  假设 $\liml_{z\ra z_0}(z-z_0)f(z)=0$.
  对任意 $\varepsilon>0$, 存在 $\delta>0$ 使得当 $0<|z-z_0|<\delta$ 时, $|(z-z_0)f(z)|<\varepsilon$.
  取闭路
  \[
    C:|z-z_0|=r<\min\{1,\delta\}.
  \]
  由\thmGrowUp 可知 $f(z)$ 在 $z_0$ 去心邻域洛朗展开的系数 $c_n$ 满足
  \[
     |c_n|
    =\abs{\frac1{2\cpi\ii}\oint_C \frac{f(\zeta)}{(\zeta-z_0)^{n+1}}\d z}
    \le \frac1{2\cpi} \cdot\frac{\varepsilon}{r^{n+2}}\cdot 2\cpi r
    =\varepsilon r^{-n-1}.
  \]
  当 $n$ 是负整数时, 我们得到 $|c_n|\le \varepsilon$.
  再由 $\varepsilon$ 的任意性得到 $c_n=0$.
  因此 $z_0$ 是 $f(z)$ 的可去奇点.
\end{proof}

\begin{exampleenum}
  \item 函数
  \[
    f(z)=\frac{\sin z}z=1-\dfrac{z^2}{3!}+\dfrac{z^4}{5!}+\cdots
  \]
  在孤立奇点 $0$ 处的洛朗展开没有负幂次项, 因此 $0$ 是可去奇点.
  也可以从 $\liml_{z\ra0}zf(z)=\sin 0=0$ 看出.
  \item 函数
  \[
    f(z)=\frac{\ee^z-1}z=1+\dfrac z{2!}+\dfrac{z^2}{3!}+\cdots
  \]
  在孤立奇点 $0$ 处的洛朗展开没有负幂次项, 因此 $0$ 是可去奇点.
  也可以从 $\liml_{z\ra0}zf(z)=\ee^0-1=0$ 看出.
\end{exampleenum}

在这两个例子中, 我们也可以使用洛必达法则得到 $\liml_{z\ra 0}f(z)=1$ 来说明 $0$ 是可去奇点.


\subsubsection{极点}

\begin{definition}
  若 $f(z)$ 在孤立奇点 $z_0$ 去心邻域的洛朗展开负幂次部分只有有限多项非零, 且非零的最低幂次项是 $-m\le-1$ 次, 即
  \[
    f(z)=c_{-m}(z-z_0)^{-m}+\cdots+c_0+c_1(z-z_0)+\cdots,\quad  0<|z-z_0|<\delta,
  \]
  其中 $c_{-m}\neq 0$, 则称 $z_0$ 是 $f(z)$ 的 \nouns{$m$ 阶极点}{极点}.\footnotemark
\end{definition}
\footnotetext{也叫 \emph{$m$ 级极点}.}

设
\[
  g(z)=c_{-m}+c_{-m+1}(z-z_0)+c_{-m+2}(z-z_0)^2+\cdots,
\]
则 $g(z)$ 在 $z_0$ 处解析且非零, 而且
\[
  f(z)=\dfrac{g(z)}{(z-z_0)^m},\quad 0<|z-z_0|<\delta.
\]
由此可以得到:

\begin{theorem}
  \label{thm:test-pole}
  \begin{enuma}
    \item $z_0$ 是 $f(z)$ 的 $m$ 阶极点当且仅当 $\liml_{z\ra z_0}(z-z_0)^mf(z)$ 存在且非零.
    \item $z_0$ 是 $f(z)$ 的极点当且仅当 $\liml_{z\ra z_0}f(z)=\infty$.
  \end{enuma}
\end{theorem}

\begin{proofenuma}
  \item 证明和\thmref{定理}{thm:test-removable} 的证明类似, 这里省略.
  \item 若 $z_0$ 是 $f(z)$ 的 $m$ 阶极点, 则
  \[
      \lim_{z\ra z_0}\frac1{f(z)}
    =\lim_{z\ra z_0}\frac{(z-z_0)^m}{g(z)}
    =\frac{0}{c_{-m}}=0.
  \]
  因此 $\liml_{z\ra z_0}f(z)=\infty$.
  反之, 若 $\liml_{z\ra z_0}f(z)=\infty$, 则
  \[
      \lim_{z\ra z_0}(z-z_0)\cdot\frac1{(z-z_0)f(z)}
    =\lim_{z\ra z_0}\frac1{f(z)}=0.
  \]
  由\thmref{定理}{thm:test-removable} 可知 $z_0$ 是 $\dfrac 1{(z-z_0)f(z)}$ 的可去奇点.
  设 $\dfrac 1{(z-z_0)f(z)}$ 在 $0<|z-z_0|<\delta$ 内的洛朗展开为
  \[
    \frac1{(z-z_0)f(z)}=a_m(z-z_0)^m+a_{m+1}(z-z_0)^{m+1}+\cdots,
  \]
  其中 $m\ge0,a_m\neq 0$.
  那么
  \[
    \lim_{z\ra z_0}(z-z_0)^{m+1} f(z)=\frac1{a_m}\neq 0,
  \]
  从而 $z_0$ 是 $f(z)$ 的 $m+1$ 阶极点.\qedhere
\end{proofenuma}

\begin{example}
  设
  \[
    f(z)=\frac{\sin z}{z^2(z+2)^3}.
  \]
  由于
  \[
     \lim_{z\ra -2}(z+2)^3f(z)
    =\lim_{z\ra -2}\frac{\sin z}{z^2}
    =-\frac14\sin 2\neq0,
  \]
  因此 $-2$ 是三阶极点.
  由于
  \[
     \lim_{z\ra 0}zf(z)
    =\lim_{z\ra 0}\frac{\sin z}z\cdot\frac1{(z+2)^3}
    =\frac18\neq0,
  \]
  因此 $0$ 是一阶极点.
\end{example}

\begin{exercise}
  求 $f(z)=\dfrac1{z^3-z^2-z+1}$ 的奇点, 并指出极点的阶.
\end{exercise}


\subsubsection{本性奇点}

\begin{definition}
  若 $f(z)$ 在孤立奇点 $z_0$ 的去心邻域的洛朗展开负幂次部分有无限多项非零, 则称 $z_0$ 是 $f(z)$ 的\noun{本性奇点}.
\end{definition}

\begin{example}
  由
  \[
    \ee^{\frac1z}=1+\frac1z+\frac1{2z^2}+\cdots
  \]
  可知 $0$ 是 $\ee^{\frac1z}$ 的本性奇点.
  若 $f(z)$ 在复平面内处处解析, 且 $f(z)$ 不是多项式, 由其泰勒展开可知 $0$ 是 $f\Bigl(\dfrac1z\Bigr)$ 的本性奇点.
\end{example}

由\thmref{定理}{thm:test-removable} 和 \ref{thm:test-pole} 可得:
\begin{theorem}
  $z_0$ 是 $f(z)$ 的本性奇点当且仅当 $\liml_{z\ra z_0}f(z)$ 不存在也不是 $\infty$.
\end{theorem}

事实上我们有\noun{皮卡大定理}: 对于本性奇点 $z_0$ 的任何一个去心邻域, $f(z)$ 的像取遍所有复数, 至多有一个取不到.
例如 $\ee^{\frac1z}$ 在 $0$ 的任何一个去心邻域内都能取到所有非零复数.


\subsection{零点与极点}

可去奇点的性质比较简单, 而本性奇点的性质又较为复杂, 因此我们主要关心的是极点的情形.
我们来研究极点与零点的联系, 并给出极点的阶的一种计算方法.

\begin{definition}
  \label{def:zero-order}
  若 $f(z)$ 在解析点 $z_0$ 处的泰勒级数非零的最低幂次项是 $m\ge1$ 次, 即
  \[
    f(z)=c_m(z-z_0)^m+c_{m+1}(z-z_0)^{m+1}+\cdots,\ 0<|z-z_0|<\delta,
  \]
  其中 $c_m\neq 0$, 则称 $z_0$ 是 $f(z)$ 的 \nouns{$m$ 阶零点}{零点}.\footnotemark
\end{definition}
\footnotetext{也叫 \emph{$m$ 级零点}.}

此时 $f(z)=(z-z_0)^mg(z)$, $g(z)$ 在 $z_0$ 处解析且 $g(z_0)\neq 0$.

根据泰勒展开的系数与高阶导数的关系可得:

\begin{theorem}
  设 $f(z)$ 在 $z_0$ 处解析.
  那么 $z_0$ 是 $m$ 阶零点当且仅当
  \[
    f(z_0)=f'(z_0)=\cdots=f^{(m-1)}(z_0)=0,\quad
    f^{(m)}(z_0)\neq 0.
  \]
\end{theorem}

\begin{exampleenum}
  \item 函数 $f(z)=z(z-1)^3$ 有一阶零点 $0$ 和三阶零点 $1$.
  \item 设 $f(z)=\sin z-z$.
  由于
  \[
    f(z)=\frac{z^3}{3!}-\frac{z^5}{5!}+\cdots
  \]
  因此 $0$ 是三阶零点.
  \item 若 $z_0$ 是 $f(z)$ 的 $m$ 阶零点, 则 $z_0$ 是 $f(z)^k$ 的 $km$ 阶零点.
  \item 若 $0$ 是 $f(z)$ 的 $m$ 阶零点, 则 $0$ 是 $f(z^k)$ 的 $km$ 阶零点.
\end{exampleenum}

\begin{theorem}
  \label{thm:zero-isolated}
  非零的解析函数的零点总是孤立的.
\end{theorem}

\begin{proof}
  设 $f(z)$ 是区域 $D$ 上的非零解析函数, $z_0\in D$ 是 $f(z)$ 的一个零点.
  由于 $f(z)$ 不恒为零, 因此可设 $z_0$ 是 $f(z)$ 的 $m$ 阶零点, 从而在 $z_0$ 的一个邻域内 $f(z)=(z-z_0)^m g(z)$, $g(z)$ 在 $z_0$ 处解析且非零.
  于是存在 $z_0$ 的一个去心邻域, 使得在这个去心邻域内 $g(z)\neq0$, 从而 $f(z)\neq 0$.
\end{proof}

若解析函数 $f_1(z)$ 和 $f_2(z)$ 满足 $f_1(z_n)=f_2(z_n)$, 其中 $\{z_n\}_{n\ge1}$ 是一收敛数列, 则所有的 $z_n$ 以及该数列极限都是 $f_1-f_2$ 的零点, 这迫使 $f_1\equiv f_2$.
由此可知, 一旦我们知道了解析函数在一个收敛数列上的所有值, 这个解析函数本身就被唯一决定了.
这也说明了 \ref{ssec:exponential-function}中指数函数的\hyperref[enum:exp-expansion]{解析延拓}定义和其它定义等价.

下面我们给出分式的奇点和分子分母零点的联系.

\begin{theorem}[可去奇点和极点判定方法]
  设 $z_0$ 是 $f(z)$ 的 $m$ 阶零点, $g(z)$ 的 $n$ 阶零点.
  那么
  \begin{enuma}
    \item $z_0$ 是 $f(z)g(z)$ 的 $m+n$ 阶零点;
    \item 若 $m\ge n$, 则 $z_0$ 是 $\dfrac{f(z)}{g(z)}$ 的可去奇点;
    \item 若 $m<n$, 则 $z_0$ 是 $\dfrac{f(z)}{g(z)}$ 的 $n-m$ 阶极点.
  \end{enuma}
\end{theorem}

若 $z_0$ 是 $f(z)$ 的解析点但不是零点, 我们可以取 $m=0$, 结论依然成立.

\begin{proof}
  由题设知存在解析函数 $f_0(z),g_0(z)$ 满足在 $z_0$ 的一个邻域内
  \[
    f(z)=(z-z_0)^mf_0(z),\quad 
    g(z)=(z-z_0)^ng_0(z),
  \]
  且 $f_0(z_0)\neq0,g_0(z_0)\neq 0$.
  从而 $f_0(z)g_0(z)$ 在 $z_0$ 处解析且非零.
  由
  \[
    f(z)g(z)=(z-z_0)^{m+n}f_0(z)g_0(z),\quad
    \frac{f(z)}{g(z)}=(z-z_0)^{m-n}\frac{f_0(z)}{g_0(z)}
  \]
  可知命题成立.
\end{proof}

\begin{example}
  $z=0$ 是下列函数的几阶极点?
  \begin{subexample}(2)
    \item $f(z)=\dfrac{\ee^z-1}{z^2}$;
    \item $f(z)=\dfrac{(\ee^z-1)^3z^2}{\sin z^7}$.
  \end{subexample}
\end{example}

\begin{solutionenum}
  \item 由于
  \[
    \ee^z-1=z+\frac{z^2}{2!}+\cdots
  \]
  所以 $0$ 是 $\ee^z-1$ 的一阶零点, $f(z)$ 的一阶极点.
  \item 由于 $(\sin z)'|_{z=0}=\cos 0=1$, 所以 $0$ 是 $\sin z$ 的一阶零点, $\sin z^7$ 的 $7$ 阶零点.
  由于 $0$ 是 $(\ee^z-1)^3z^2$ 的 $5$ 阶零点, 因此 $0$ 是 $f(z)$ 的二阶极点.
\end{solutionenum}

\begin{exercise}
  求 $f(z)=\dfrac{(z-5)\sin z}{(z-1)^2z^2(z+1)^3}$ 的奇点以及极点的阶.
\end{exercise}


\subsection{孤立奇点 \texorpdfstring{$\infty$}{∞} 的分类\optional}

当我们把 $\infty$ 添加到复平面使其变成扩充复平面后, 从几何上看它变成了一个球面.
这样的一个球面是一种封闭的曲面, 它具有某些整体性质.

当我们需要计算一个闭路上函数的积分的时候,
我们需要研究闭路内部每一个奇点处的洛朗展开,
从而得到相应的小闭路上的积分.
若在这个闭路内部的奇点比较多, 而外部的奇点比较少时, 这样计算就不太方便.
此时若通过变量替换 $z=\dfrac1t$, 转而研究闭路外部奇点处的洛朗展开, 便可减少所需考虑的奇点个数, 从而降低所需的计算量.
因此我们需要研究函数在 $\infty$ 的行为.

\begin{definition}
  若函数 $f(z)$ 在 $\infty$ 的去心邻域 $R<|z|<+\infty$ 内没有奇点, 则称 $\infty$ 是 $f(z)$ 的\noun{孤立奇点}.
\end{definition}

设 $g(t)=f\Bigl(\dfrac1t\Bigr)$, 则 $g(t)$ 在圆环域 $0<|t|<\dfrac1R$ 内解析, $0$ 是它的孤立奇点.
研究 $f(z)$ 在 $\infty$ 的性质可以转为研究 $g(t)$ 在 $0$ 的性质.

\begin{definition}
  若 $0$ 是 $g(t)$ 的可去奇点 ($m$ 阶极点, 或本性奇点), 则称 $\infty$ 是 $f(z)$ 的\noun{可去奇点} (\nouns{$m$ 阶极点}{极点}, 或\noun{本性奇点}).
\end{definition}

设 $f(z)$ 在圆环域 $R<|z|<+\infty$ 内的洛朗展开为
\[
  f(z)=\cdots+\frac{c_{-2}}{z^2}+\frac{c_{-1}}{z}+c_0+c_1z+c_2z^2+\cdots
\]
则 $g(t)$ 在圆环域 $0<|t|<\dfrac1R$ 内的洛朗展开为
\[
  g(t)=\cdots+\frac{c_2}{t^2}+\frac{c_1}t+c_0+c_{-1}t+c_{-2}t^2+\cdots
\]
由此得到 $\infty$ 的奇点类型和 $f(z)$ 在 $\infty$ 的去心邻域洛朗展开的关系.

\begin{table}[!htb]
  \centering
  \begin{tabular}{ccc}
    \topcolorrule
      \bf $\infty$ 类型&
      \bf 洛朗级数特点&
      $\lim\limits_{z\ra\infty}f(z)$\\ 
    \topcolorrule
      可去奇点&
      没有正幂次部分&
      存在\\
    \midcolorrule
      $m$ 阶极点&
      \makecell{正幂次部分只有有限项非零\\最高次为 $m$ 次}&
      $\infty$\\
    \midcolorrule
      本性奇点&
      正幂次部分有无限项非零&
      不存在且不为 $\infty$\\ 
    \bottomcolorrule
  \end{tabular}
  \caption{孤立奇点 $\infty$ 的分类}
\end{table}

\begin{exampleenum}
  \item 设 $f(z)=\dfrac z{z+1}$. \smallskip
  由 $\liml_{z\ra\infty}f(z)=1$ 可知 $\infty$ 是可去奇点.事实上此时 $f(z)$ 在 $1<|z|<+\infty$ 内的洛朗展开为
  \[
      f(z)=\frac{1}{1+\dfrac1z}
    =1-\frac1z+\frac1{z^2}-\frac1{z^3}+\cdots
  \]
  \item 函数 $f(z)=z^2+\dfrac\ii z$ 含有正次幂项且最高次为 $2$, 因此 $\infty$ 是 $2$ 阶极点.
  \smallskip

  一般地, 若 $f(z)=\dfrac{p(z)}{q(z)}$ 是有理函数, $p,q$ 次数分别为 $m,n$, 则当 $m>n$ 时, $\infty$ 是 $f(z)$ 的 $m-n$ 阶极点;
  当 $m\le n$ 时, $\infty$ 是 $f(z)$ 的可去奇点.
  特别地, $\infty$ 是 $n\ge1$ 次多项式的 $n$ 阶极点.
\end{exampleenum}

\begin{example}
  函数 
  \[
    \sin z=z-\frac{z^3}{3!}+\frac{z^5}{5!}-\frac{z^7}{7!}+\cdots
  \]
  含有无限多正次幂项, 因此 $\infty$ 是 $\sin z$ 的本性奇点.
  一般地, 若 $f(z)$ 在复平面内处处解析, 且 $f(z)$ 不是多项式, 则 $\infty$ 是它的本性奇点.
\end{example}

\begin{example}
  求函数
  \[
    f(z)=\dfrac{(z^2-1)(z-2)^3}{(\sin{\cpi z})^3}
  \]
  在扩充复平面内的奇点和奇点类型, 并指出极点的阶.
\end{example}

\begin{solutionenum}
  \item 整数 $z=k\neq \pm1,2$ 是 $\sin{\cpi z}$ 的一阶零点, 因此是 $f(z)$ 的三阶极点.
  \item $z=\pm1$ 是 $z^2-1$ 的一阶零点, 因此是 $f(z)$ 的二阶极点.
  \item $z=2$ 是 $(z-2)^3$ 的三阶零点, 因此是 $f(z)$ 的可去奇点.
  \item 由于奇点 $1,2,3,\cdots\ra \infty$, 因此 $\infty$ 不是孤立奇点.
\end{solutionenum}

\begin{exercise}
  求函数
  \[
    f(z)=\dfrac{z^2+4\cpi^2}{z^3(\ee^z-1)}
  \]
  在扩充复平面内的奇点和奇点类型, 并指出极点的阶.
\end{exercise}

\begin{example}[代数学基本定理]
  \label{exam:algebraic-basic-theorem}
  证明非常数复系数多项式 $p(z)$ 总有复零点.\footnotemark
\end{example}
\footnotetext{
  该定理最先由高斯于1799年严格证明.
  参考 \cite[第25章1节]{Kline1990b}.
}

\begin{proof}
  假设多项式 $p(z)$ 没有复零点, 则 $f(z)=\dfrac1{p(z)}$ 在复平面内处处解析, 从而 $f(z)$ 在 $0$ 处可以展开为幂级数.
  由于 $\infty$ 是 $p(z)$ 的极点, $\liml_{z\ra\infty}p(z)=\infty$.
  因此 $\liml_{z\ra\infty}f(z)=0$, $\infty$ 是 $f(z)$ 的可去奇点.
  这意味着 $f(z)$ 在 $0$ 处的洛朗展开没有正幂次项.
  二者结合可知 $f(z)$ 只能是常数, 矛盾!
\end{proof}



\psection{本章小结}

本章所需掌握的知识点如下:
\begin{conclusion}
  \item 会判断简单的复数项级数 $\sumf1z_n$ 的敛散性.
  \begin{conclusion}
    \item 若实部和虚部级数至少有一个发散, 则原级数发散; 若二者都绝对收敛, 则原级数发散; 其它情形原级数条件收敛.
    \item 若 $\liml_{n\ra\infty} z_n=0$ 不成立, 则级数发散.
    \item 若 $\lambda=\liml_{n\ra\infty}\abs{\dfrac{z_{n+1}}{z_n}}$ 存在或 $\lambda=\liml_{n\ra\infty}\sqrt[n]{|z_n|}$ 存在, 则当 $\lambda<1$ 时级数绝对收敛; 当 $\lambda<1$ 时级数发散; 当 $\lambda=1$ 则都有可能.
  \end{conclusion}
  \item 能熟练使用比值法和根式法计算幂级数 $\sumf0c_n(z-z_0)^n$ 的收敛半径.
  \begin{conclusion}
    \item 幂级数的收敛区域是一个圆域.
    \item 若 $r=\liml_{n\ra\infty}\abs{\dfrac{c_{n+1}}{c_n}}$ 存在(或为 $+\infty$)或 $r=\liml_{n\ra\infty}\sqrt[n]{|c_n|}$ 存在(或为 $+\infty$), 则收敛圆域的半径, 也就是收敛半径 $R=\dfrac1r$.
    \item 幂级数在其收敛圆周上的敛散性各种情况都有可能.
  \end{conclusion}
  \item 熟知泰勒展开和洛朗展开成立的条件、形式、常见性质.
  \begin{conclusion}
    \item $f(z)$ 在解析的圆域内可以展开为幂级数, 在解析的圆环域内可以展开为双边幂级数.
    \item 泰勒展开形式为 $f(z)=\sumf0 \dfrac{f^{(n)}(z_0)}{n!}(z-z_0)^n$.
    \item 洛朗展开形式为 $f(z)=\sumff \Bigl(\oint_C \dfrac{f(z)}{(z-z_0)^{n+1}}\d z\Bigr)(z-z_0)^n$, 其中 $C$ 为圆环域内任意一条内部包含 $z_0$ 的闭路.
  \end{conclusion}
  \item 掌握简单函数的泰勒展开和洛朗展开. 特别地, 要掌握有理函数情形的计算方法.
  \begin{conclusion}
    \item 幂级数在其收敛圆域内、双边幂级数在其收敛圆环域内的和函数是解析函数, 且可以进行各种代数运算、逐项求导、逐项积分.
    \item 解析函数的幂级数展开和双边幂级数展开是唯一的, 所以我们可以通过对简单函数的展开进行各种运算来得到解析函数函数的泰勒展开或洛朗展开.
    \item 有理函数的展开可以通过拆分为部分分式之和进行展开来计算.
  \end{conclusion}
  \item 会判断简单的奇点分类, 会利用零点的阶判断分式的极点的阶.
  \begin{conclusion}
    \item 通过在奇点 $z_0$ 去心邻域内洛朗展开的特点来判断.
    \item 若 $\liml_{z\ra z_0} f(z)$ 存在(为 $\infty$, 或既不存在也不是 $\infty$), 则 $z_0$ 是可去奇点(极点, 或本性奇点).
    \item 若 $\liml_{z\ra z_0} (z-z_0)f(z)=0$, 则 $z_0$ 是可去奇点.
    \item 若 $\liml_{z\ra z_0} (z-z_0)^mf(z)$ 存在且非零, 则 $z_0$ 是 $m$ 阶极点.
    \item 若 $z_0$ 分别是 $f(z),g(z)$ 的 $m,n$ 阶零点, 则当 $m\ge n$ 时 $z_0$ 是 $\dfrac{f(z)}{g(z)}$ 的可去奇点; 当 $m<n$ 时 $z_0$ 是 $\dfrac{f(z)}{g(z)}$ 的 $n-m$ 阶极点.
  \end{conclusion}
\end{conclusion}

本章中不易理解和易错的知识点包括:
\begin{enuma}
  \item 混淆比值法及根式法中的 $r$ 和幂级数的收敛半径 $R$, 二者互为倒数.
  \item 混淆泰勒展开成立的最大圆域半径和泰勒级数的收敛半径, 二者并不总是相等.
  \item 在计算洛朗展开时, 错误地选取模大于 $1$ 的数作为公比来展开成等比级数. 应当根据圆环域的范围来选择公比.
\end{enuma}



\psection{本章作业}

\begin{homework}
  \item 单选题.
  \begin{homework}
    \item 若级数 $\sumf0 z_n$ 条件收敛, 则下列选项不可能成立的是\fillbrace{}.
      \begin{exchoice}(2)
        \item 实部级数 $\sumf0 x_n$ 条件收敛
        \item 实部级数 $\sumf0 x_n$ 绝对收敛
        \item 虚部级数 $\sumf0 y_n$ 条件收敛
        \item 虚部级数 $\sumf0 y_n$ 发散
      \end{exchoice}
    \item 设 $z_n\neq0$ 且级数 $\sumf0 z_n$ 绝对收敛, 则下列选项不可能成立的是\fillbrace{}.
      \begin{exchoice}(2)
        \item $\liml_{n\ra\infty}\abs{\dfrac{z_{n+1}}{z_n}}<1$
        \item $\liml_{n\ra\infty}\abs{\dfrac{z_{n+1}}{z_n}}=1$
        \item $\liml_{n\ra\infty}\sqrt[n]{|z_n|}=1$
        \item $\liml_{n\ra\infty}\sqrt[n]{|z_n|}>1$
      \end{exchoice}
    \item 以下表述正确的是\fillbrace{}.
      \begin{exchoice}
        \item 幂级数总在它的收敛圆周内处处收敛
        \item 幂级数的和函数在收敛圆周内可能有奇点
        \item 幂级数在它的收敛圆周上可能处处绝对收敛
        \item 任一在 $z_0$ 处可导的函数一定可以在 $z_0$ 的邻域内展开成泰勒级数
      \end{exchoice}
    \item 幂级数在其收敛圆周上\fillbrace{}.
      \begin{exchoice}(2)
        \item 一定处处绝对收敛
        \item 一定处处条件收敛
        \item 一定有发散的点
        \item 可能处处收敛也可能有发散的点
      \end{exchoice}
    \item 若级数 $\sumf0 a_n(z-1)^n$ 在点 $z=3$ 发散, 则\fillbrace{}.
      \begin{exchoice}(2)
        \item 在点 $z=-1$ 收敛
        \item 在点 $z=-3$ 发散
        \item 在点 $z=2$ 收敛
        \item 以上都不对
      \end{exchoice}
    \item 函数 $f(z)=\dfrac{z-1}{z^2-z-2}$ 不能在\fillbrace{}内作洛朗展开.
      \begin{exchoice}(2)
        \item $0<|z|<2$
        \item $2<|z|<4$
        \item $0<|z+1|<2$
        \item $1<|z+1|<3$
      \end{exchoice}
    \item 若 $z_0$ 是 $f(z)$ 的一阶极点, $g(z)$ 的一阶零点, 则 $z_0$ 是 $f(z)^3g(z)^2$ 的\fillbrace{}.
      \begin{exchoice}(4)
        \item 一阶极点
        \item 一阶零点
        \item 可去奇点
        \item 三阶极点
      \end{exchoice}
    \item 若 $z_0$ 是 $f(z)$ 的二阶零点, $g(z)$ 的一阶零点, 则 $z_0$ 是 $\dfrac{f(z)}{g(z)}$ 的\fillbrace{}.
      \begin{exchoice}(4)
        \item 一阶极点
        \item 一阶零点
        \item 可去奇点
        \item 三阶极点
      \end{exchoice}
  \end{homework}
  \item 填空题.
  \begin{homework}
    \item 函数 $f(z)=\dfrac{\ee^{1/z}}{z+1}$ 在 $z_0=i$ 处的泰勒展开成立的最大圆域是 $|z-i|<$\fillblank{}.
    \item 函数 $f(z)=\dfrac{\ee^z-1}{z}$ 在 $z_0=0$ 处的泰勒级数的收敛半径为\fillblank{}.
    \item 若函数 $f(z)=\dfrac1{(z+5)\sin z}$ 可以在圆环域 $0<|z|<R$ 内作洛朗展开, 则 $R$ 的最大值为\fillblank{}.
    \item 若函数 $f(z)=\dfrac{\ln(z+2)}{z+\ii}$ 可以在圆环域 $r<|z|<2$ 内作洛朗展开, 则 $r$ 的最小值为\fillblank{}.
    \item $0$ 是 $(\cos z+\ch z-2)^2$ 的\fillblank{}阶零点.
    \item $0$ 是 $\dfrac1{(\sin z+\sh z-2z)^2}$ 的\fillblank{}阶极点.
  \end{homework}
  \item 解答题.
  \begin{homework}
    \item 判断下列级数绝对收敛、条件收敛还是发散:
      \begin{subhomework}(2)
        \item $\sumf2 \frac{\ii^n}{\ln n}$;
        \item $\sumf0 \frac{(6+5i)^n}{8^n}$;
        \item $\sumf1 \frac{n^2}{5^n}(1+2i)^n$;
        \item $\sumf1 \frac{\ii^n}{n}$;
        \item $\sumf1 \Bigl(\frac{1}{\ln (\ii n)}\Bigr)^n$;
        \item $\sumf0 \frac{(1+\ii)^n}{5^n}$;
        \item $\sumf0 \frac{\cos(\ii n)}{2^n}$;
        \item $\sumf1 \Bigl(\frac1n+\frac{(-1)^n\ii}{\sqrt n}\Bigr)$.
      \end{subhomework}
    \item 计算下列幂级数的收敛半径.
    \begin{subhomework}(2)
      \item $\sumf1 (\ii z)^n$;
      \item $\sumf1 \frac{z^n}{(1+\ii)^n}$;
      \item $\sumf1 \frac{n!}{n^n}z^n$;
      \item $\sumf1 \frac1n(z-1)^n$;
      \item $\sumf0 (1+\ii)^n(z+\ii)^n$;
      \item $\sumf1 \ee^{\frac{\cpi\ii}n}z^n$;
      \item $\sumf1 \Bigl(\frac{z-2}{\ln{\ii n}}\Bigr)^n$;
      \item $\sumf1 \frac{z^n}{n^p}$, 其中 $p$ 是正整数;
      \item $\sumf1 (n+a^n)z^n$, 其中 $a$ 是正实数;
      \item $\sumf1 (a^n+b^n)z^n$.
    \end{subhomework}
    \item 把下列各函数在 $z_0$ 处展开成幂级数, 并指出它们的收敛半径:
    \begin{subhomework}(2)
      \item $\dfrac1{(1+z^2)^2}, z_0=0$;
      \item $\dfrac{1}{(z-1)(z-2)}, z_0=0$;
      \item $\ee^z\cos z, z_0=0$;
      \item $f(z)=\dfrac{1}{z-2}+\ee^{-z}, z_0=0$;
      \item $\dfrac{z}{(z+1)(z+2)}, z_0=2$;
      \smallskip
      \item $\dfrac1{z^2}, z_0=-1$;
      \item $\arctan z=-\dfrac\ii2\ln\dfrac{1+\ii z}{1-\ii z}, z_0=0$;
      \item $\sqrt{z+2}$ 的主值, $z_0=-1$.
    \end{subhomework}
    \item 将下列函数在 $0<|z|<1$ 和 $1<|z|<2$ 内展开成洛朗级数.
      \begin{subhomework}(2)
        \item $\dfrac{\sin z}{z^2}$;
        \item $\dfrac{\ee^{\frac1z}}{z-1}$;
        \item $\dfrac{z+1}{z^2(z-1)}$;
        \item $\dfrac1{(1-z)(z-2)}$;
        \item $\dfrac{z+1}{(z-1)^2}$;
        \item $\dfrac{2}{z(z+2)}$.
      \end{subhomework}
    \item 将下列函数在 $0<|z-1|<1$ 和 $2<|z-1|<+\infty$ 内展开成洛朗级数.
      \begin{subhomework}(2)
        \item $\dfrac{\ee^z}{z^2}$;
        \item $\dfrac{z+1}{(z-1)(z-2)}$;
        \item $\dfrac{1}{z^3+z^2}$;
        \item $\dfrac{z}{z^2-3z+2}$.
      \end{subhomework}
    \item 若 $C$ 为正向圆周 $|z|=3$, 求积分 $\doint_Cf(z)\d z$ 的值, 其中 $f(z)$ 为:
      \begin{subhomework}(2)
        \item $\dfrac1{z(z+2)}$;
        \item $\dfrac{e^{\frac1z}}{z-1}$.
      \end{subhomework}
    \item 求数列
      \[
        a_0=a_1=1,\quad a_{n+2}=2a_{n+1}-2a_n
      \]
      的通项公式.
      提示: 先求出 $\sumf0 a_n z^n$ 的和函数.
    \item 指出下述推理的错误之处: 我们有
      \begin{align*}
        \frac{z}{1-z}&=z+z^2+z^3+z^4+\cdots,\\
        \frac{z}{z-1}&=1+\frac1z+\frac1{z^2}+\frac1{z^3}+\cdots.
      \end{align*}
      又因为 $\dfrac{z}{1-z}+\dfrac{z}{z-1}=0$, 因此
      \[
        \cdots+\frac1{z^3}+\frac1{z^2}+\frac1z+1+z+z^2+z^3+z^4+\cdots=0.
      \]
    \item 下列函数有哪些奇点? 若是极点, 请指出它的阶:
      \begin{subhomework}(3)
        \item $\dfrac1{(z-2)^3(z^2+1)^2}$;
        \item $\dfrac{\cos z-1}{z^3}$;
        \item $\dfrac1{z^3+z^2-z-1}$;
        \item $\dfrac z{(1+z^2)(1+\ee^{\cpi z})}$;
        \item $\dfrac1{\ee^{z-1}}$;
        \item $\dfrac1{z^2(\ee^z-1)}$;
        \item $\dfrac{z^6}{1+z^4}$;
        \item $\dfrac1{\sin z^2}$;
        \item $\dfrac{z\sin z^3}{\ln(1-z^4)}$;
        \item $\dfrac{\sin z}{(z-\cpi)^2}$;
        \item $\dfrac{(\ee^z-1)^2z^3}{\sin z^8}$;
        \item $\dfrac{z-\cpi}{(\sin z)^3}$.
      \end{subhomework}
    \item 设解析函数 $f(z)$ 满足 $f(\zeta z)=\zeta^m f(z)$, 其中 $\zeta=\ee^{\frac{2\cpi\ii}n}$ 是 $n$ 次单位根.
      证明: $f(z)$ 的麦克劳林展开只有 $nk+m$ 次项, $k\in\BZ$.
    \item 证明: 若 $z_0$ 是 $f(z)$ 的 $m>1$ 阶零点, 则 $z_0$ 是 $f^{(k)}(z)$ 的 $m-k$ 阶零点, 其中 $1\le k<m$.
    \item 证明下列幂级数的收敛半径相同:
      \begin{subhomework}(3)
        \item $\sumf0 c_nz^n$;
        \item $\sumf0 \dfrac{c_n}{n+1}z^{n+1}$;
        \item $\sumf1 nc_nz^{n-1}$.
      \end{subhomework}
    \item 证明: 若级数 $\sumf0 z_n$ 绝对收敛, 则 $\sumf0 z_n^2$ 也绝对收敛.
    \item 设 $a$ 是 $\varphi(z)$ 和 $\psi(z)$ 的 $m$ 阶和 $n$ 阶极点. 证明:
      \begin{subhomework}
        \item 当 $m>n$ 时, $a$ 是 $\varphi(z)+\psi(z)$ 的 $m$ 阶极点;
        \item 当 $m=n$ 时, $a$ 是 $\varphi(z)+\psi(z)$ 的 $\le m$ 阶极点或可去奇点.
      \end{subhomework}
    \item 设函数 $f(z)$ 在扩充复平面上只有有限多个奇点, 这些奇点都是极点且 $\liml_{z\ra\infty}f(z)\neq0$.
      证明 $f(z)$ 在扩充复平面上所有零点的阶之和与所有极点的阶之和相等. 提示: $f(z)$ 一定是有理函数.
    \item \optionalex 计算如下级数的和.
    \begin{subhomework}(2)
      \item $\sumf0 \dfrac1{(3n+1)!}$;
      \item $\sumf0 \dfrac{(-1)^n}{2n+1}$.
    \end{subhomework}
    \item \optionalex 已知有理函数 $f(z)=\dfrac{6}{z^2(z-1)^3}$ 在 $0<|z|<1$ 内的洛朗展开为
      \[
        f(z)=-\sum_{n\ge 0}(n^3+12n^2+47n+60)z^n,
      \]
      求它在 $1<|z|<+\infty$ 内的洛朗展开.
    \item \optionalex 已知有理函数 $f(z)=\dfrac{120}{(z-1)(z^2-4)(z^2-9)}$ 在 $0<|z|<1$ 内的洛朗展开为
      \[
        f(z)=\sum_{n\ge 0}\Bigl(-5+\frac2{(-2)^{n+1}}+\frac6{2^{n+1}}-\frac1{(-3)^{n+1}}-\frac2{3^{n+1}}\Bigr)z^n,
      \]
      求它在其它解析的圆环域内的洛朗展开.
  \end{homework}
\end{homework}

% 
% 
% \begin{example}
% 求幂级数 $\sumf1 \frac{z^n}{n^p}$ 的收敛半径并讨论在收敛圆周上的情形, 其中 $p\in\BR$.
% \end{example}
% 
% \begin{solution}
% 由 $\liml_{n\ra\infty}\abs{\frac{c_{n+1}}{c_n}}=\lim_{n\ra\infty}\left(\frac n{n+1}\right)^p=1$ 可知收敛半径为 $1$.
% {设 $|z|=1$.
% \begin{itemize}
% \item 若 $p>1$, $\sumf1 \abs{\frac{z^n}{n^p}}=\sumf1 \frac1{n^p}$ 收敛,
% {原级数在收敛圆周内处处绝对收敛.}
% \item 若 $p\le 0$, $\abs{\dfrac{z^n}{n^p}}=\dfrac1{n^p}\not\ra 0$,
% {原级数在收敛圆周内处处发散.}
% \end{itemize}}
% \end{solution}
% \end{frame}

% 
% 
% 回忆\emph{狄利克雷判别法}: 若 $\set{a_n}_{n\ge 1}$ 部分和有界, 实数项数列 $\set{b_n}_{n\ge 1}$ 单调趋于 $0$, 则 $\sumf1 a_nb_n$ 收敛.

% 
% \begin{solution}
% \begin{itemize}
% \item 若 $0<p\le1$, $\sumf1 \frac1{n^p}$ 发散, 
% {而在收敛圆周上其它点 $z\neq1$ 处,
% \[|z+z^2+\cdots+z^n|=\abs{\frac{z(1-z^n)}{1-z}}
% \le\frac{2}{|1-z|}\]
% 有界, 数列 $\set{n^{-p}}_{n\ge 1}$ 单调趋于 $0$,}
% {因此 $\sumf1 \frac{z^n}{n^p}$ 收敛.}
% {故该级数在 $z=1$ 发散, 在收敛圆周上其它点收敛.}
% \end{itemize}
% \end{solution}
% \end{frame}




% \begin{theorem}
% 	设幂级数
% 	\[
% 		f(z)=\sumf0 a_nz^n,|z|<R,
% 	\]
% 	设函数 $\varphi(z)$ 在集合 $D$ 上满足 $|\varphi(z)|<R$.
% 	那么当 $z\in D$ 时,
% 	\[f[\varphi(z)]=\sumf0 a_n[\varphi(z)]^n.\]}
% \end{theorem}

% 设 $f(x)=\dfrac1x$, 那么
% \[
%   f(m)\le \int_{m-1/2}^{m+1/2}f(x)\d x
%   \le \frac12\bigl(f(m-\dfrac12)+f(m+\dfrac12)\bigr).
% \]
% \begin{center}
%   \begin{tikzpicture}
%     \draw[cstaxis] (-.5,0)--(3.5,0);
%     \draw[cstaxis] (0,-.5)--(0,3);
%     \fill[cstfill1] (1,0)--(1,1)--(2,.5)--(2,0)--cycle;
%     \fill[white] plot[domain=1:2,smooth cycle] (\x,{1/\x});
%     \draw[cstcurve,main,smooth,domain=.4:3] plot (\x,{1/\x});
%     \draw[cstcurve,fourth] (1,0)--(1,1);
%     \draw[cstcurve,fourth] (2,0)--(2,.5);
%     \draw[second] (1,1)--(2,.5);
%     \draw[second] (1,{8/9})--(2,{4/9});
%     \fill[cstdote,third] (1,1) circle;
%     \fill[cstdote,third] (2,.5) circle;
%     \fill[cstdote,third] (1.5,{2/3}) circle;
%   \end{tikzpicture}
% \end{center}
% 设 $S=f(n+1)+\cdots+f(2n)$, 则
% \begin{align*}
%   S&\le \int_{n+1/2}^{2n+1/2} f(x)\d x=\ln\frac{4n+1}{2n+1},\\
%   \ln 2&=\int_{n}^{2n}f(x)\d x
%   \le S+\frac12 f(n)-\frac12f(2n)=S+\frac1{4n},
% \end{align*}
% 于是由
% \[
%   -\frac14\le (S-\ln 2)n\le n\ln\frac{4n+1}{4n+2}
% \]
% 和夹逼准则可得 $\liml_{n\ra\infty}(S-\ln 2)n=-\dfrac14$.

% 
\chapter{留数}
\section{孤立奇点}

\subsection{孤立奇点的类型}

我们先根据奇点附近洛朗展开的形式来对其进行分类, 以便于分类计算留数.

\begin{example}
		考虑函数 $f(z)=\dfrac1{\sin(1/z)}$, 显然 $0,z_k=\dfrac1{k\pi}$ 是奇点, $k$ 是非零整数.
	{因为 $\lim\limits_{k\to+\infty} z_k=0$, 所以 $0$ 的任何一个去心邻域内都有奇点.此时无法选取一个圆环域 $0<|z|<\delta$ 作 $f(z)$ 的洛朗展开, 因此我们不考虑这类奇点.
	}
	{
	\begin{center}
		\begin{tikzpicture}[framed]
			\draw[cstcurve,third] (0,0) circle (1.3);
			\fill[cstdot] (0,0) circle;
			\fill[cstdot,second] (2,0) circle;
			\fill[cstdot,main] (1,0) circle;
			\fill[cstdot,second] (0.6667,0) circle;
			\fill[cstdot,main] (0.5,0) circle;
			\fill[cstdot,second] (0.4,0) circle;
			\draw
				(-0.3,0) node {$0$}
				(2,-0.3) node[second] {$z_1$}
				(1,0.3) node[main] {$z_2$}
				(0.6667,-0.3) node[second] {$z_3$}
				(0.5,0.3) node[main] {$z_4$}
				(0.4,-0.3) node[second] {$z_k$};
		\end{tikzpicture}
	\end{center}}
\end{example}

\begin{definition}
	如果 $z_0$ 是 $f(z)$ 的一个奇点, 且 $z_0$ 的某个邻域内没有其它奇点, 则称 $z_0$ 是 $f(z)$ 的一个\emph{孤立奇点}.
\end{definition}

\begin{example}
	\begin{itemize}
		\item $z=0$ 是 $e^{\frac1z},\dfrac{\sin z}z$ 的孤立奇点.
		\item $z=-1$ 是 $\dfrac1{z(z+1)}$ 的孤立奇点.
		\item $z=0$ 不是 $\dfrac1{\sin(1/z)}$ 的孤立奇点.
	\end{itemize}
\end{example}

若 $f(z)$ 只有有限多个奇点, 则这些奇点都是孤立奇点.

如果 $f(z)$ 在孤立奇点 $z_0$ 的去心邻域 $0<|z-z_0|<\delta$ 内解析, 则可以作 $f(z)$ 的洛朗展开.
根据该洛朗级数主要部分的项数, 我们可以将孤立奇点分为三种:
% \begin{center}
% 	\defaultrowcolors
% 	\Newcommand\arraystretch{1.4}
% 	\begin{tabular}{|c|c|c|}\hline
% 		孤立奇点类型&洛朗级数特点&$\lim\limits_{z\to z_0}f(z)$\\\hline
% 		可去奇点&没有主要部分&存在且有限\\\hline
% 		&主要部分只有有限项非零&\\
% 		\multirow{-2}*{$m$ 阶极点}&最低次为 $-m$ 次&\multirow{-2}*{$\infty$}\\\hline
% 		本性奇点&主要部分有无限项非零&不存在且不为 $\infty$\\\hline
% 	\end{tabular}
% \end{center}

\begin{definition}
	若 $f(z)$ 在孤立奇点 $z_0$ 的去心邻域的洛朗级数没有主要部分, 即
	\[f(z)=c_0+c_1(z-z_0)+c_2(z-z_0)^2+\cdots,\quad 0<|z-z_0|<\delta,\]
	是幂级数, 则称 $z_0$ 是 $f(z)$ 的\emph{可去奇点}.
\end{definition}

设 $g(z)$ 为右侧幂级数的和函数, 则 $g(z)$ 在 $|z-z_0|<\delta$ 上解析,
且除 $z_0$ 外 $f(z)=g(z)$.
通过补充或修改定义 $f(z_0)=g(z_0)=c_0$, 可使得 $f(z)$ 也在 $z_0$ 解析.
这就是``可去''的含义.

\begin{theorem}
	\begin{tabular}{rl}
		$z_0$ 是 $f(z)$ 的可去奇点
		&$\iff\lim\limits_{z\to z_0}f(z)$ 存在且有限\\
		&$\iff\lim\limits_{z\to z_0}(z-z_0)f(z)=0$.
	\end{tabular}
\end{theorem}

\begin{example}
		\[f(z)=\frac{\sin z}z=1-\dfrac{z^2}{3!}+\dfrac{z^4}{5!}+\cdots\]
		没有负幂次项, 因此 $0$ 是可去奇点.

	{也可以从 $\lim\limits_{z\to0}zf(z)=\sin 0=0$ 看出.
	}
\end{example}

\begin{example}
		\[f(z)=\frac{e^z-1}z=1+\dfrac z{2!}+\dfrac{z^2}{3!}+\cdots\]
		没有负幂次项, 因此 $0$ 是可去奇点.

	{也可以从 $\lim\limits_{z\to0}zf(z)=e^0-1=0$ 看出.
	}
\end{example}

\begin{definition}
	若 $f(z)$ 在孤立奇点 $z_0$ 的去心邻域的洛朗级数主要部分有无限多项非零, 则称 $z_0$ 是 $f(z)$ 的\emph{本性奇点}.
\end{definition}

\begin{example}
	由于 $\displaystyle e^{\frac1z}=1+\frac1z+\frac1{2z^2}+\cdots$, 因此 $0$ 是本性奇点.
\end{example}

\begin{theorem}
	$z_0$ 是 $f(z)$ 的本性奇点 $\iff\lim\limits_{z\to z_0}f(z)$ 不存在也不是 $\infty$.
\end{theorem}

事实上我们有\emph{皮卡大定理}: 对于本性奇点 $z_0$ 的任何一个去心邻域, $f(z)$ 的像取遍所有复数, 至多有一个取不到.

可去奇点的性质比较简单, 而本性奇点的性质又较为复杂, 因此我们主要关心的是极点的情形.

\begin{definition}
	如果 $f(z)$ 在孤立奇点 $z_0$ 的去心邻域的洛朗级数主要部分只有有限多项非零, 即
	\[f(z)=\frac{c_{-m}}{(z-z_0)^m}+\cdots+c_0+c_1(z-z_0)+\cdots,\ 0<|z-z_0|<\delta,\]
	其中 $c_{-m}\neq 0,m\ge 1$, 则称 $z_0$ 是 $f(z)$ 的 \emph{$m$ 阶极点}或 \emph{$m$ 级极点}.
\end{definition}

令
\[g(z)=c_{-m}+c_{-m+1}(z-z_0)+c_{-m+2}(z-z_0)^2+\cdots,\]
则 $g(z)$ 在 $z_0$ 解析且非零,
且
\[f(z)=\dfrac{g(z)}{(z-z_0)^m},0<|z-z_0|<\delta.\]

\begin{theorem}
	\begin{enumerate}
		\item $z_0$ 是 $f(z)$ 的 $m$ 阶极点 $\iff\lim\limits_{z\to z_0}(z-z_0)^mf(z)$ 存在且非零.
		\item $z_0$ 是 $f(z)$ 的极点 $\iff\lim\limits_{z\to z_0}f(z)=\infty$.
	\end{enumerate}
\end{theorem}

\begin{example}
		$f(z)=\dfrac{3z+2}{z^2(z+2)}$,
	{由于 $\lim\limits_{z\to 0}z^2f(z)=1$, 因此 $0$ 是二阶极点.同理 $-2$ 是一阶极点.
	}
\end{example}

\begin{exercise}
	求 $f(z)=\dfrac1{z^3-z^2-z+1}$ 的奇点, 并指出极点的阶.
\end{exercise}

\begin{answer}
	$-1$ 是一阶极点, $1$ 是二阶极点.
\end{answer}

\subsection{零点与极点}

我们来研究极点与零点的联系, 并给出极点的阶的计算方法.
\begin{definition}
	如果 $f(z)$ 在解析点 $z_0$ 处的泰勒级数最低次项幂次是 $m\ge1$, 即
	\[f(z)=c_m(z-z_0)^m+c_{m+1}(z-z_0)^{m+1}+\cdots,\ 0<|z-z_0|<\delta,\]
	其中 $c_m\neq 0$, 则称 $z_0$ 是 $f(z)$ 的 \emph{$m$ 阶零点}.
\end{definition}

此时 $f(z)=(z-z_0)^mg(z)$, $g(z)$ 在 $z_0$ 解析且 $g(z_0)\neq 0$.

\begin{theorem}
	设 $f(z)$ 在 $z_0$ 解析.
	$z_0$ 是 $m$ 阶零点当且仅当
	\[f(z_0)=f'(z_0)=\cdots=f^{(m-1)}(z_0)=0,\quad
	f^{(m)}(z_0)\neq 0.\]
\end{theorem}

\begin{example}
		$f(z)=z(z-1)^3$
	{有一阶零点 $0$ 和三阶零点 $1$.
	}
\end{example}

\begin{example}
		$f(z)=\sin z-z$.
	{由于
		\[f(z)=\frac{z^3}{3!}-\frac{z^5}{5!}+\cdots\]
		因此 $0$ 是三阶零点.
	}
\end{example}

\begin{theorem}
非零的解析函数的零点总是孤立的.
\end{theorem}

\begin{proof}
	设 $f(z)$ 是区域 $D$ 上的非零解析函数, $z_0\in D$ 是 $f(z)$ 的一个零点.
{%
	由于 $f(z)$ 不恒为零, 因此存在 $m\ge 1$ 使得在 $z_0$ 的一个邻域内 $f(z)=(z-z_0)^m g(z)$, $g(z)$ 在 $z_0$ 处解析且非零.
}

{
	对于 $\varepsilon=\dfrac12|g(z_0)|$, 存在 $\delta>0$ 使得当 $z\in \Uc(z_0,\delta)\subseteq D$ 时, $|g(z)-g(z_0)|<\varepsilon$.
}%
{%
	从而 $g(z)\neq0$, $f(z)\neq 0$.\qedhere
}
\end{proof}

由此可知, 一旦我们知道了解析函数在一串有极限的数列上的值, 这个解析函数本身就被唯一决定了.

为了统一地研究零点和极点, 我们引入下述记号.
设 $z_0$ 是 $f(z)$ 的可去奇点、极点或解析点.
记 $\ord(f,z_0)$ 为 $f(z)$ 在 $z_0$ 的洛朗展开的最低次项幂次.

不难看出,
\begin{enumerate}
	\item 如果 $\ord(f,z_0)\ge0$, 则 $z_0$ 是可去奇点或解析点.
	\item 如果 $\ord(f,z_0)=m>0$, 则 $z_0$ 是可去奇点或 $m$ 阶零点.
	\item 如果 $\ord(f,z_0)=-m<0$, 则 $z_0$ 是 $m$ 阶极点.
\end{enumerate}

\begin{alertblock}{可去奇点和极点判定方法}
	如果 $\ord(f,z_0)=m,\ord(g,z_0)=n$, 那么
	\[\ord\left(\frac fg,z_0\right)=m-n,\quad\ord(fg,z_0)=m+n.\]
\end{alertblock}

\begin{proof}
		设 $f_0(z)$ 为幂级数 $(z-z_0)^{-m}f(z)$ 的和函数, $g_0(z)$ 为幂级数 $(z-z_0)^{-n}g(z)$ 的和函数,
	{则 $f_0(z),g_0(z)$ 在 $z_0$ 解析且非零.
	}%

	{因此 $\dfrac{f_0(z)}{g_0(z)},f_0(z)g_0(z)$ 在 $z_0$ 解析且非零.由
		\[\frac{f(z)}{g(z)}=(z-z_0)^{m-n}\frac{f_0(z)}{g_0(z)},\quad
		f(z)g(z)=(z-z_0)^{m+n}f_0(z)g_0(z)\]
		可知命题成立.\qedhere
	}
\end{proof}

\begin{corollary}
	设 $z_0$ 是 $f(z)$ 的 $m$ 阶零点, 是 $g(z)$ 的 $n$ 阶零点.
	\begin{enumerate}
		\item 若 $m\ge n$, 则 $z_0$ 是 $\dfrac{f(z)}{g(z)}$ 的可去奇点.
		\item 若 $m<n$ 时, 则 $z_0$ 是 $\dfrac{f(z)}{g(z)}$ 的 $n-m$ 阶极点.
	\end{enumerate}
\end{corollary}

\begin{example}
	单选题: (2021年B卷) $z=0$ 是函数 $f(z)=\dfrac{e^z-1}{z^2}$ 的\fillbrace{{A}} 阶极点.
	\xx{$1$}{$2$}{$3$}{$4$}
\end{example}

\begin{solution}
		由于 $e^z-1=z+\dfrac{z^2}{2!}+\cdots$, 所以 $0$ 是 $e^z-1$ 的一阶零点.

	{因此 $\ord(f,0)=1-2=-1$, $0$ 是一阶极点.
	}
\end{solution}

\begin{example}
	$z=0$ 是 $f(z)=\dfrac{(e^z-1)^3z^2}{\sin z^7}$ 的几阶极点?
\end{example}

\begin{solution}
		由于 $(\sin z)'(0)=\cos 0=1$, 所以 $0$ 是 $\sin z$ 的一阶零点.

	{因此 $\ord(f,0)=3+2-7=-2$, $0$ 是二阶极点.
	}
\end{solution}

\begin{exercise}
	求 $f(z)=\dfrac{(z-5)\sin z}{(z-1)^2z^2(z+1)^3}$ 的奇点.
\end{exercise}

\begin{answer}
	$1$ 是二阶极点, $0$ 是一阶极点, $-1$ 是三阶极点.
\end{answer}

\subsection{函数在 \texorpdfstring{$\infty$}{∞} 的性态}

当我们把复平面扩充成闭复平面后, 从几何上看它变成了一个球面.
这样的一个球面是一种封闭的曲面, 它具有某些整体性质.

当我们需要计算一个闭路上函数的积分的时候,
我们需要研究闭路内部每一个奇点处的洛朗展开,
从而得到相应的小闭路上的积分.
如果在这个闭路内部的奇点比较多, 而外部的奇点比较少时, 这样计算就不太方便.
此时如果通过变量替换 $z=\dfrac1t$, 转而研究闭路外部奇点处的洛朗展开,
便可减少所需考虑的奇点个数, 从而降低所需的计算量.
因此我们需要研究函数在 $\infty$ 的性态.

\begin{definition}
	如果函数 $f(z)$ 在 $\infty$ 的去心邻域 $R<|z|<+\infty$ 内没有奇点, 则称 $\infty$ 是 $f(z)$ 的\emph{孤立奇点}.
\end{definition}

设 $g(t)=f\left(\dfrac1t\right)$, 则研究 $f(z)$ 在 $\infty$ 的性质可以转为研究 $g(t)$ 在 $0$ 的性质.
$g(t)$ 在圆环域 $0<|t|<\dfrac1R$ 上解析, $0$ 是它的孤立奇点.

\begin{definition}
	如果 $0$ 是 $g(t)$ 的可去奇点 ($m$ 阶极点、本性奇点), 则称 $\infty$ 是 $f(z)$ 的\emph{可去奇点 ($m$ 阶极点、本性奇点).}
\end{definition}

设 $f(z)$ 在圆环域 $R<|z|<+\infty$ 的洛朗展开为
\[f(z)=\cdots+\frac{c_{-2}}{z^2}+\frac{c_{-1}}{z}+c_0+c_1z+c_2z^2+\cdots\]
则 $g(t)$ 在圆环域 $0<|t|<\dfrac1R$ 的洛朗展开为
\[g(t)=\cdots+\frac{c_2}{t^2}+\frac{c_1}t+c_0+c_{-1}t+c_{-2}t^2+\cdots\]
% \begin{center}
% 	\defaultrowcolors
% 	\Newcommand\arraystretch{1.2}
% 	\begin{tabular}{|c|c|c|}\hline
% 		$\infty$ 类型&洛朗级数特点&$\lim\limits_{z\to\infty}f(z)$\\\hline
% 		可去奇点&没有正幂次部分&存在且有限\\\hline
% 		&正幂次部分只有有限项非零&\\
% 		\multirow{-2}*{$m$ 阶极点}&最高次为 $m$ 次&\multirow{-2}*{$\infty$}\\\hline
% 		本性奇点&正幂次部分有无限项非零&	不存在且不为 $\infty$\\\hline
% 	\end{tabular}
% \end{center}

\begin{example}
		$f(z)=\dfrac z{z+1}$.
	{由 $\lim\limits_{z\to\infty}f(z)=1$ 可知 $\infty$ 是可去奇点.事实上此时 $f(z)$ 在 $1<|z|<+\infty$ 内的洛朗展开为
		\[f(z)=\frac{1}{1+\dfrac1z}=1-\frac1z+\frac1{z^2}-\frac1{z^3}+\cdots\]
	}
\end{example}

\begin{example}
		函数 $f(z)=z^2+\dfrac1z$
	{含有正次幂项且最高次为 $2$, 因此 $\infty$ 是 $2$ 阶极点.
	}
\end{example}

\begin{example}
		设 $p(z)$ 是 $n\ge1$ 次多项式,
	{则 $\infty$ 是 $p(z)$ 的 $n$ 阶极点.
	}
\end{example}

\begin{example}
		函数 
		\[\sin z=z-\frac{z^3}{3!}+\frac{z^5}{5!}-\frac{z^7}{7!}+\cdots\]
	{含有无限多正次幂项, 因此 $\infty$ 是本性奇点.
	}

	{事实上, 如果函数 $f(z)$ 在复平面上处处解析, 且 $f(z)$ 不是多项式, 则 $\infty$ 是它的本性奇点.
	}
\end{example}

\begin{example}
	函数 $f(z)=\dfrac{(z^2-1)(z-2)^3}{(\sin{\pi z})^3}$ 在扩充复平面内有哪些什么类型的奇点, 并指出极点的阶.
\end{example}

\begin{solution}
	\begin{itemize}
		\item 整数 $z=k\neq \pm1,2$ 是 $\sin{\pi z}$ 的一阶零点, 因此是 $f(z)$ 的三阶极点.
		\item $z=\pm1$ 是 $z^2-1$ 的一阶零点, 因此是 $f(z)$ 的二阶极点.
		\item $z=2$ 是 $(z-2)^3$ 的三阶零点, 因此是 $f(z)$ 的可去奇点.
		\item 由于奇点 $1,2,3,\cdots\to \infty$, 因此 $\infty$ 不是孤立奇点.
	\end{itemize}
\end{solution}

\begin{exercise}
	函数 $f(z)=\dfrac{z^2+4\pi^2}{z^3(e^z-1)}$ 在扩充复平面内有哪些什么类型的奇点, 并指出极点的阶.
\end{exercise}

\begin{answer}
	\begin{itemize}
		\item $z=2k\pi i$ 是一阶极点, $k\neq 0,\pm1$.
		\item $z=0$ 是四阶极点.
		\item $z=\pm 2\pi i$ 是可去奇点.
		\item $z=\infty$ 不是孤立奇点.
	\end{itemize}
\end{answer}

\begin{example}
	证明非常数复系数多项式 $p(z)$ 总有复零点.
\end{example}

\begin{proof}
	假设多项式 $p(z)$ 没有复零点, 那么 $f(z)=\dfrac1{p(z)}$ 在复平面上处处解析, 
{%
	从而 $f(z)$ 在 $0$ 处可以展开为幂级数.
}

{%
	由于 $\infty$ 是 $p(z)$ 的极点, $\lim\limits_{z\to\infty}p(z)=\infty$.
}%
{%
	因此 $\lim\limits_{z\to\infty}f(z)=0$, $\infty$ 是 $f(z)$ 的可去奇点.
}%
{%
	这意味着 $f(z)$ 在 $0$ 处的洛朗展开没有正幂次项.
}%
{%
	二者结合可知 $f(z)$ 只能是常数, 矛盾!\qedhere
}
\end{proof}

设 $z_1$ 是 $n$ 次多项式 $p(z)$ 的零点, 则 $\dfrac{p(z)}{z-z_1}$ 是 $n-1$ 次多项式.
归纳可知, $p(z)$ 可以分解为 $p(z)=(z-z_1)\cdots(z-z_n)$.

\section{留数}

\subsection{留数定理}

\begin{definition}
	设 $z_0$ 为 $f(z)$ 的孤立奇点, $f(z)$ 在它的某个去心邻域内的洛朗展开为
		\[f(z)=\cdots+\frac{c_{-1}}{z-z_0}+c_0+c_1(z-z_0)+\cdots.\]
	称
		\[\Res[f(z),z_0]:=c_{-1}=\frac1{2\pi i}\oint_Cf(z)\diff z\]
	为函数 \emph{$f(z)$ 在 $z_0$ 的留数}, 其中 $C$ 为该去心邻域中绕 $z_0$ 的一条闭路.
\end{definition}

可以看出, 知道留数之后可以用来计算积分.

\begin{block}{留数定理}
	若 $f(z)$ 在闭路 $C$ 上解析, 在 $C$ 内部的奇点为 $z_1,z_2,\dots,z_n$, 则
	\[\oint_Cf(z)\diff z=2\pi i\sum_{k=1}^n\Res[f(z),z_k].\]
\end{block}

\begin{proof}
	\begin{center}
		\begin{tikzpicture}[scale=.8]
			\filldraw[cstcurve,rounded corners=0.4cm,main,cstfill2] (-3,-1) rectangle (3,1);
			\draw[cstcurve,main,cstwra] (3,-0.2)--(3,0.1);
			\filldraw[cstcurve,white,draw=second] (-2,0.2) circle (0.5);
			\draw[cstcurve,second,cstwra] (-2,0.7) arc(90:135:0.5);
			\fill[cstdot] (-2,0.2) circle;
			\filldraw[cstcurve,white,draw=second] (0.2,-0.2) circle (0.5);
			\draw[cstcurve,second,cstwra] (0.2,0.3)arc(90:135:0.5);
			\fill[cstdot] (0.2,-0.2) circle;
			\filldraw[cstcurve,white,draw=second] (2,0.2) circle (0.5);
			\draw[cstcurve,second,cstwra] (2,0.7)arc(90:135:0.5);
			\fill[cstdot] (2,0.2) circle;
			\draw
				(3.4,0) node[main] {$C$}
				(-1.3,-0.2) node[second] {$C_1$}
				(0.2,0.5) node[second] {$C_2$}
				(2.4,-0.5) node[second] {$C_3$}
				(-2,-0.1) node {$z_1$}
				(0.2,-0.5) node {$z_2$}
				(2,-0.1) node {$z_3$};
		\end{tikzpicture}
	\end{center}

	{由复闭路定理,
	\[\oint_Cf(z)\diff z=\sum_{k=1}^n\oint_{C_k}f(z)\diff z
	=2\pi i\sum_{k=1}^n\Res[f(z),z_k].\qedhere\]}
\end{proof}

\subsection{留数的计算方法}

若 $z_0$ 为 $f(z)$ 的可去奇点, 则显然 $\Res[f(z),z_0]=0$.

\begin{example}
		$f(z)=\dfrac{z^3(e^z-1)^2}{\sin z^4}$.
	{由于 $\ord(f,0)=3+2-4=1$, $z=0$ 是 $f(z)$ 的可去奇点,因此 \[\Res[f(z),0]=0.\]
	}
\end{example}

若 $z_0$ 为 $f(z)$ 的本性奇点, 一般只能从定义计算.

\begin{example}
	$f(z)=z^4\sin\dfrac1z$.
{%
	由于
	\[f(z)=z^4\sum_{n=0}^\infty(-1)^n\frac{z^{-2n-1}}{(2n+1)!}
	=z^3-\frac z{3!}+\frac1{5!z}+\cdots\]
}
{%
	因此 \[\Res[f(z),0]=\frac1{120}.\]
}
\end{example}

设 $z_0$ 为 $f(z)$ 的极点.

\begin{alertblock}{极点留数计算公式 I}
	如果 $z_0$ 是 $\le m$ 阶极点或可去奇点, 那么
	\[\Res[f(z),z_0]=\frac1{(m-1)!}\lim_{z\to z_0}\frac{\diff^{m-1}}{\diff z^{m-1}}[(z-z_0)^mf(z)].\]
\end{alertblock}

\begin{alertblock}{极点留数计算公式 II}
	如果 $z_0$ 是一阶极点或可去奇点, 那么
	\[\Res[f(z),z_0]=\lim\limits_{z\to z_0}(z-z_0)f(z).\]
\end{alertblock}

\begin{proof}
	设
	\begin{align*}
		f(z)&=c_{-m}(z-z_0)^{-m}+\cdots+c_{-1}(z-z_0)^{-1}+c_0+\cdots,\\
		g(z)&=c_{-m}+\cdots+c_{-1}(z-z_0)^{m-1}+c_0(z-z_0)^m+\cdots,
	\end{align*}
{%
	则 $g(z)=(z-z_0)^mf(z)$.
}%
{%
	由泰勒展开系数与函数导数的关系可知
	\[\Res[f(z),z_0]=c_{-1}=\dfrac1{(m-1)!}g^{(m-1)}(z_0).\qedhere\]
}
\end{proof}

\begin{example}
	求 $\Res\left[\dfrac{e^z}{z^n},0\right]$.
\end{example}

\begin{solution}
		显然 $0$ 是 $n$ 阶极点,
	{\begin{align*}
			\Res\left[\frac{e^z}{z^n},0\right]&=\frac1{(n-1)!}\lim_{z\to0}(e^z)^{(n-1)}\\
			&{=\frac1{(n-1)!}\lim_{z\to0}e^z=\frac1{(n-1)!}.}
		\end{align*}
	}
\end{solution}

\begin{example}
	求 $\Res\left[\dfrac{z-\sin z}{z^6},0\right]$.
\end{example}

\begin{solution}
		因为 $z=0$ 是 $z-\sin z$ 的三阶零点,
	{所以是 $\dfrac{z-\sin z}{z^6}$ 的三阶极点.如果用公式
		\[\Res\left[\frac{z-\sin z}{z^6},0\right]
		=\frac1{2!}\lim_{z\to0}\left(\frac{z-\sin z}{z^3}\right)''\]
		计算会很繁琐.
	}

	{
	\begin{align*}
		\Res\left[\frac{z-\sin z}{z^6},0\right]&=\frac1{5!}\lim_{z\to0}(z-\sin z)^{(5)}\\
		&{=\frac1{5!}\lim_{z\to0}(-\cos z)=-\frac1{120}.}
	\end{align*}}
\end{solution}

\begin{exercise}
		求 $\Res\left[\dfrac{e^z-1}{z^5},0\right]=$\fillblank[2cm][2mm]{{$\dfrac1{24}$}}.
\end{exercise}

\begin{alertblock}{极点留数计算公式 III}
	设 $P(z),Q(z)$ 在 $z_0$ 解析且 $z_0$ 是 $Q$ 的一阶零点, 则
	\[\Res\left[\frac{P(z)}{Q(z)},z_0\right]=\frac{P(z_0)}{Q'(z_0)}.\]
\end{alertblock}

\begin{proof}
		不难看出 $z_0$ 是 $f(z)=\dfrac{P(z)}{Q(z)}$ 的一阶极点或可去奇点.
	{因此
		\begin{align*}
			&\peq\Res[f(z),z_0]=\lim_{z\to z_0}(z-z_0)f(z)\\
			&{=\lim_{z\to z_0}\frac{P(z)}{\dfrac{Q(z)-Q(z_0)}{z-z_0}}}
			{=\frac{P(z_0)}{\lim\limits_{z\to z_0}\dfrac{Q(z)-Q(z_0)}{z-z_0}}=\frac{P(z_0)}{Q'(z_0)}.\qedhere}
		\end{align*}
	}
\end{proof}

\begin{example}
	求 $\Res\left[\dfrac{z}{z^8-1},\dfrac{1+i}{\sqrt2}\right]$.
\end{example}

\begin{solution}
		由于 $z=\dfrac{1+i}{\sqrt2}$ 是分母的 $1$ 阶零点,
	{因此
		\[\Res\left[\frac z{z^8-1},\frac{1+i}{\sqrt2}\right]
		=\frac z{(z^8-1)'}\Big|_{z=\frac{1+i}{\sqrt2}}
		=\frac z{8z^7}\Big|_{z=\frac{1+i}{\sqrt2}}
		=\frac i8.\]
	}
\end{solution}


\begin{example}
	计算积分 $\displaystyle\oint_{|z|=2}\frac{e^z}{z(z-1)^2}\diff z$.
\end{example}

\begin{solution}
		$f(z)=\dfrac{e^z}{z(z-1)^2}$ 在 $|z|<2$ 内有奇点 $z=0,1$.
	{\begin{align*}
		\Res[f(z),0]&=\lim_{z\to0}\frac{e^z}{(z-1)^2}=1,\\
		{\Res[f(z),1]}&{=\lim_{z\to1}\left(\frac{e^z}z\right)'=\lim_{z\to1}\frac{e^z(z-1)}{z^2}=0,}
	\end{align*}\[\oint_{|z|=2}\frac{e^z}{z(z-1)^2}\diff z
		=2\pi i\bigl[\Res[f(z),0]+\Res[f(z),1]\bigr]
		=2\pi i.\]
	}
\end{solution}

\subsection{在 \texorpdfstring{$\infty$}{∞} 的留数*}

\begin{definition}
	设 $\infty$ 为 $f(z)$ 的孤立奇点, $f(z)$ 在某个 $R<|z|<+\infty$ 内的洛朗展开为
	\[f(z)=\cdots+c_{-1}z^{-1}+c_0+c_1z+\cdots\]
	称
	\[\Res[f(z),\infty]:=-c_{-1}=\frac1{2\pi i}\oint_{C^-}f(z)\diff z\]
	为函数 \emph{$f(z)$ 在 $\infty$ 的留数}, 其中 $C$ 为该圆环域中绕 $0$ 的一条闭路.
\end{definition}

由于
\[f\left(\frac1z\right)\frac1{z^2}=\cdots+\frac{c_1}{z^3}+\frac{c_0}{z^2}+\frac{c_{-1}}z+c_{-2}+\cdots\]
因此 
	\[\Res[f(z),\infty]=-\Res\left[f\left(\frac1z\right)\frac1{z^2},0\right].\]

需要注意的是, 和普通复数不同, \emph{即便 $\infty$ 是可去奇点, 也不意味着 $\Res[f(z),\infty]=0$}.

\begin{theorem}
	如果 $f(z)$ 只有有限个奇点, 那么 $f(z)$ 在\emph{扩充复平面内各奇点处的留数之和为 $0$}.
\end{theorem}

\begin{proof}
	设闭路 $C$ 内部包含除 $\infty$ 外所有奇点 $z_1,\dots,z_n$.
	由留数定理
		\[-2\pi i\Res[f(z),\infty]=\oint_C f(z)\diff z=2\pi i\sum_{k=1}^n\Res[f(z),z_k].\]
	故 $\suml_{k=1}^n \Res[f(z),z_k]+\Res[f(z),\infty]=0$.
\end{proof}

\begin{example}
	求 $\displaystyle\oint_{|z|=2}f(z)\diff z$, 其中 $f(z)=\dfrac{\sin(1/z)}{(z+i)^{10}(z-1)(z-3)}$.
\end{example}

\begin{solution}
	$f(z)$ 在 $|z|>2$ 内只有奇点 $3,\infty$.
{
	\[\Res[f(z),3]=\lim_{z\to3}(z-3)f(z)=\frac1{2(3+i)^{10}}\sin\frac13.\]
}
	\begin{align*}
		\Res[f(z),\infty]&=-\Res\left[f\left(\frac1z\right)\frac1{z^2},0\right]\\
		&=-\Res\left[\frac{z^{10}\sin z}{(1+iz)^{10}(1-z)(1-3z)},0\right]=0.
	\end{align*}
{
	\begin{align*}
		\oint_{|z|=2}f(z)\diff z
		&=2\pi i\bigl[\Res[f(z),-i]+\Res[f(z),1]+\Res[f(z),0]\bigr]\\
		&{=-2\pi i\bigl[\Res[f(z),3]+\Res[f(z),\infty]\bigr]=-\frac{\pi i}{(3+i)^{10}}\sin\frac13.}
	\end{align*}
}
\end{solution}

\begin{center}
	\begin{tikzpicture}[node distance=25pt]
		\node at (4.55,0.3) [cstnode2]	(end1){求出 $f(z)$ 的原函数 $F(z)$ 得到 $\displaystyle\int_Cf(z)\diff z=F(b)-F(a)$};
		\node at (0,2.2) [cstnode]	(isanalytic){$f(z)$ 解析};
		\node at (4.5,2.2) [cstnode2,align=center]	(end2){设曲线方程为 $z(t)$, 则\\
		积分$\displaystyle=\int_a^bf(z)z'(t)\diff t$\\
		可能需要分段计算};
		\node at (9,2.2) [cstnode2,align=center]	(end4){用闭路外的孤立\\奇点的留数计算, \\最后取负号*};
		\node at (0,4.5) [cstnode]	(isclosed){是闭路};
		\node at (4.5,4.5) [cstnode]	(issingle){只有孤立奇点};
		\node at (9,4.5) [cstnode,align=center]	(ismany){闭路内部和外部\\孤立奇点数量};
		\node at (1.5,6.5) [cstnode1]  (integral){定积分 $\displaystyle\int_Cf(z)\diff z$ 的计算};
		\node at (9,6.5) [cstnode2,align=center]	(end3){用闭路内的孤立\\奇点的留数计算};

		\draw[cstnarrow,third] (integral.-160) -- (isclosed);
		\draw[cstnarrow,third] (isanalytic) -- node[left]{是} (end1.173);
		\draw[cstnarrow,third] (isclosed) -- node[above]{是} (issingle);
		\draw[cstnarrow,third] (issingle) -- node[above]{是} (ismany);
		\draw[cstnarrow,third] (isclosed) -- node[left]{否} (isanalytic);
		\draw[cstnarrow,third] (issingle) -- node[left]{否} (end2);
		\draw[cstnarrow,third] (isanalytic) -- node[above]{否} (end2);
		\draw[cstnarrow,third] (ismany) -- node[left]{闭路内的少} (end3);
		\draw[cstnarrow,third] (ismany) -- node[left]{闭路外的少} (end4);
	\end{tikzpicture}
\end{center}

在求有理函数的洛朗展开, 以及之后在求有理函数的拉普拉斯逆变换时, 我们需要将一个有理函数表达为分母只有一个零点的有理函数之和.
例如:
\[\frac{z-3}{(z+1)(z-1)^2}=\frac1{z-1}-\frac1{(z-1)^2}-\frac1{z+1}.\]
我们可以用待定系数法计算, 不过有时候使用留数会更为简便.

\begin{solution}
		设 $\displaystyle f(z)=\frac{z-3}{(z+1)(z-1)^2}=\frac a{z-1}+\frac b{(z-1)^2}+\frac c{z+1}$,
	{则
		\begin{align*}
			a&=\Res[f(z),1]=\left(\frac{z-3}{z+1}\right)'\Big|_{z=1}=\frac 4{(z+1)^2}\Big|_{z=1}=1,\\
			b&=\Res[(z-1)f(z),1]=\frac{z-3}{z+1}\Big|_{z=1}=-1,\\
			c&=\Res[f(z),-1]=\frac{z-3}{(z-1)^2}\Big|_{z=-1}=-1.
		\end{align*}故 $\displaystyle f(z)=\frac1{z-1}-\frac1{(z-1)^2}-\frac1{z+1}$.
	}
\end{solution}

\section{留数在定积分的应用*}

\subsection{正弦余弦的有理函数的积分}

本节中我们将对若干种在实变中难以计算的定积分和广义积分使用复变函数和留数的技巧进行计算.
本节内容不作考试要求.

考虑 $\displaystyle\int_0^{2\pi} R(\cos\theta,\sin\theta)\diff\theta$, 其中 $R$ 是一个有理函数.
令 $z=e^{i\theta}$, 则 $\diff z=iz\diff\theta$,
\[\cos\theta=\half\left(z+\frac1z\right)=\frac{z^2+1}{2z},\quad
\sin\theta=\frac1{2i}\left(z-\frac1z\right)=\frac{z^2-1}{2iz},\]
	\[\int_0^{2\pi} R(\cos\theta,\sin\theta)\diff\theta
	=\oint_{|z|=1} R\left(\frac{z^2+1}{2z},\frac{z^2-1}{2iz}\right)\frac1{iz}\diff z.\]
由于被积函数是一个有理函数, 它的积分可以由 $|z|<1$ 内奇点留数得到.

\begin{example}
	求 $\displaystyle\int_0^{2\pi}\frac{\sin^2\theta}{5-3\cos\theta}\diff\theta$.
\end{example}

\begin{solution}
	令 $z=e^{i\theta}$, 则 $\diff z=iz\diff\theta$,
	{
		\[\cos\theta=\half\left(z+\frac1z\right)=\frac{z^2+1}{2z},\qquad
		\sin\theta=\frac1{2i}\left(z-\frac1z\right)=\frac{z^2-1}{2iz},\]

		\begin{align*}
			\int_0^{2\pi}\frac{\sin^2\theta}{5-3\cos\theta}\diff\theta
			&=\oint_{|z|=1}\frac{(z^2-1)^2}{-4z^2}\cdot\frac1{5-3\frac{z^2+1}{2z}}\cdot\frac{\diff z}{iz}\\
			&=-\frac i6\oint_{|z|=1}\frac{(z^2-1)^2}{z^2(z-3)(z-\frac13)}\diff z.
		\end{align*}
	}
	设 $f(z)=\dfrac{(z^2-1)^2}{z^2(z-3)(z-\frac13)}$,
	{则
		\[\Res[f(z),0]=\frac{10}3,\quad\Res[f(z),\frac13]=-\frac83,\]
	}

	{
		\[
			\int_0^{2\pi}\frac{\sin^2\theta}{5-3\cos\theta}\diff\theta
			=-\frac i6\cdot 2\pi i\Bigl[\Res[f(z),0]+\Res[f(z),\frac13]\Bigr]
			=\frac{2\pi}9.
		\]
	}
\end{solution}

\subsection{有理函数的广义积分}

考虑 $\displaystyle\int_{-\infty}^{+\infty}R(x)\diff x$, 其中 $R(x)$ 是一个有理函数, 分母比分子至少高 $2$ 次, 且分母没有实根.
我们先考虑 $\displaystyle\int_{-r}^rR(x)\diff x$.
设 $f(z)=R(z),C=C_r+[-r,r]$ 如下图所示, 使得上半平面内 $f(z)$ 的奇点均在 $C$ 内,
则
\[2\pi i\sum_{\Im a>0}\Res[f(z),a]=\oint_Cf(z)\diff z=\int_{-r}^rR(x)\diff x+\int_{C_r}f(z)\diff z.\]

\begin{center}
	\begin{tikzpicture}[framed]
		\draw[cstaxis] (-2,0)--(2,0);
		\draw[cstaxis] (0,-0.2)--(0,2);
		\draw[cstcurve,main] (-1.5,0) arc(180:0:1.5);
		\draw[cstcurve,main,cstla] (-1.2,0.9) arc(135:130:1.5);
		\draw[cstcurve,second] (-1.5,0)--(1.5,0);
		\draw[cstcurve,second,cstra] (-1.5,0)--(-0.5,0);
		\draw
			(-1.5,-0.3) node[second] {$-r$}
			(1.5,-0.3) node[second] {$r$}
			(1.3,1.2) node[main] {$C_r$};
	\end{tikzpicture}
\end{center}

由于 $P(x)$ 分母次数比分子至少高 $2$ 次,
当 $r\to+\infty$ 时,
\[\abs{\int_{C_r}f(z)\diff z}\le \pi r\max_{|z|=r}|f(z)|
=\pi \max_{|z|=r}|zf(z)|\to 0.\]
故
	\[\int_{-\infty}^{+\infty}R(x)\diff x=2\pi i\sum_{\Im a>0}\Res[R(z),a].\]

\begin{example}
	求 $\displaystyle\int_{-\infty}^{+\infty}\frac{\diff x}{(x^2+a^2)^3},a>0$.
\end{example}

\begin{solution}
	$f(z)=\dfrac1{(z^2+a^2)^3}$ 在上半平面内的奇点为 $ai$.

	{
		\begin{align*}
		\Res[f(z),ai]&=\frac1{2!}\lim_{z\to ai}\left[\frac1{(z+ai)^3}\right]''\\
		&=\half\lim_{z\to ai}\frac{12}{(z+ai)^5}=\frac{3}{16a^5i},
		\end{align*}故
		\[\int_{-\infty}^{+\infty}\frac{\diff x}{(x^2+a^2)^3}
	=2\pi i\Res[f(z),ai]=\frac{3\pi}{8a^5}.\]
	}
\end{solution}

\subsection{有理函数与三角函数之积的广义积分}

考虑 $\displaystyle\int_{-\infty}^{+\infty}R(x)\cos{\lambda x}\diff x$, $\displaystyle\int_{-\infty}^{+\infty}R(x)\sin{\lambda x}\diff x$, 其中 $R(x)$ 是一个有理函数, 分母比分子至少高 $2$ 次, 且分母没有实根.
和前一种情形类似, 我们有
	\[\int_{-\infty}^{+\infty}R(x)e^{i\lambda x}\diff x
	=2\pi i\sum_{\Im a>0}\Res[R(z)e^{i\lambda z},a],\]
因此所求积分分别为它的实部和虚部.


\begin{example}
	求 $\displaystyle\int_{-\infty}^{+\infty}\frac{\cos x\diff x}{(x^2+a^2)^2}, a>0$.
\end{example}

\begin{solution}
	$f(z)=\dfrac{e^{iz}}{(z^2+a^2)^2}$ 在上半平面内的奇点为 $ai$,
	{
		\[\Res[f(z),ai]=\lim_{z\to ai}\left[\frac{e^{iz}}{(z+ai)^2}\right]'=-\frac{e^{-a}(a+1)i}{4a^3}.\]故
		$\displaystyle\qquad \int_{-\infty}^{+\infty}\frac{e^{ix}\diff x}{(x^2+a^2)^2}=2\pi i \Res[f(z),ai]=\frac{\pi e^{-a}(a+1)}{2a^3}$,
		\[\int_{-\infty}^{+\infty}\frac{\cos x\diff x}{(x^2+a^2)^2}=\frac{\pi e^{-a}(a+1)}{2a^3}.\]
	}
\end{solution}

\subsection{其它例子}

最后我们再来看一个例子.

\begin{example}
	求积分 $I=\displaystyle\int_0^{+\infty}\frac{x^p}{x(x+1)}\diff x,0<p<1$.
\end{example}

\begin{solution}
	\[I=\int_0^{+\infty}\frac{x^p}{x(x+1)}\diff x\xto{\text{令}\ x=e^t}\int_{-\infty}^{+\infty}\frac{e^{pt}}{e^t+1}\diff t.\]
	{考虑 $f(z)=\dfrac{e^{pz}}{e^z+1}$ 在如下闭路 $C$ 上的积分.
	\begin{center}
		\begin{tikzpicture}[framed,scale=.8]
			\draw[cstaxis] (-3,0)--(3,0);
			\draw[cstaxis] (0,0)--(0,1.7);
			\draw[cstcurve,second] (-1.5,0) rectangle (1.5,1.2);
			\draw[cstcurve,main] (-1.5,0)--(-1.5,1.2);
			\draw[cstcurve,main] (1.5,0)--(1.5,1.2);
			\draw[cstcurve,second,cstra] (-1.5,0)--(-0.5,0);
			\draw[cstcurve,second,cstra] (0,1.2)--(-0.5,1.2);
			\draw[cstcurve,main,cstra] (1.5,0)--(1.5,0.9);
			\draw[cstcurve,main,cstra] (-1.5,1.2)--(-1.5,0.3);
			\draw
				(1.8,0.3) node {$R$}
				(-2,0.3) node {$-R$}
				(0.4,0.8) node {$2\pi i$}
				(1.2,0.4) node[main] {$C_1$}
				(-1.1,0.5) node[main] {$C_2$}
				(-0.4,0.8) node[second] {$l$};
		\end{tikzpicture}
	\end{center}}

		由于 $l:z=t+2\pi i,-R\le t\le R$,
	{因此
		\[\int_l f(z)\diff z
		=\int_R^{-R}\frac{e^{2p\pi i}\cdot e^{pt}}{e^t+1}\diff t
		=-e^{2p\pi i}\int_{-R}^Rf(t)\diff t.\]由于 $C_1:z=R+it,0\le t\le 2\pi$,因此
		\[\abs{\int_{C_1}f(z)\diff z}\le \frac{e^{(p+1)R}}{e^R-1}\cdot 2\pi\to 0\quad(R\to+\infty).\]同理
		\[\abs{\int_{C_2}f(z)\diff z}\le \frac{e^{-(p+1)R}}{1-e^{-R}}\cdot 2\pi\to 0\quad(R\to+\infty).\]
	}

	由于
	\[\Res[f(z),\pi i]
	=\frac{e^{pz}}{(e^z+1)'}\bigg|_{z=\pi i}=-e^{p\pi i},\]
	{因此
		\begin{align*}
		&\left(\int_{-R}^R+\int_l+\int_{C_1}+\int_{C_2}\right)f(z)\diff z\\
		=&\oint_Cf(z)\diff z=2\pi i\Res[f(z),\pi i]=-2\pi ie^{p\pi i},
		\end{align*}令 $R\to+\infty$,则
		\begin{align*}
		&(1-e^{2p\pi i})I=-2\pi ie^{p\pi i},\quad
		{I=\frac{2\pi i}{e^{p\pi i}-e^{-p\pi i}}=\frac{\pi}{\sin p\pi}.}
		\end{align*}
	}
\end{solution}



% \chapter{保形映射}
\label{chapter:6}

保形映射是指通过构造一个函数把边界不规则、不易研究的区域映射到边界规则、易研究的区域.
我们将从解析函数在导数非零处的几何意义出发, 得到共性映射的定义.
然后讨论分式线性映射、幂函数、指数函数和儒可夫斯基函数所对应的映射特点, 并利用它们得到一些情形下区域的保形映射.
最后介绍保形映射在平面向量场的应用.



\section{保形映射的概念}

\subsection{导数的几何意义}


在\thmref{例}{exam:orthogonal-curve} 中我们提到, 对于解析函数 $f(z)$, 当 $f'(z_0)\neq 0$ 时, 经过 $z_0$ 的两条曲线 $C_1,C_2$ 的夹角和它们的像 $f(C_1),f(C_2)$ 在 $f(z_0)$ 处的夹角总是相同的.
大致来说, 这是因为 $\d f=f'(z_0)\d z$, 从复数乘法的几何意义可知, 局部上 $w=f(z)$ 把 $z_0$ 附近的点以 $z_0$ 为中心放缩 $\abs{f'(z_0)}$ 倍并逆时针旋转 $\arg{f'(z_0)}$.
我们来严格表述这一性质.

\begin{figure}[H]
  \centering
  \begin{tikzpicture}
    \draw[cstcurve,cstnra,third] (-1,0)-- node[above] {$w=f(z)$} (1,0);
    \draw (-4,-2) node[below] {$z$ 平面};
    \draw (4,-2) node[below] {$w$ 平面};
    \begin{scope}[shift={(-6,-2)}]
      \draw[cstaxis] (0,0)--(4,0);
      \draw[cstaxis] (0,0)--(0,4);
      \coordinate (B) at (1.5,1.5);
      \coordinate (A) at (3,3);
      \coordinate (C) at (1,2.366);
      \draw[fourth,thick,cstra] pic [cstfill4,draw=fourth, "$\theta$", angle eccentricity=1.5,] {angle=A--B--C};
      \begin{scope}[main]
        \draw[thick] (.75,.75)--(3,3);
        \draw[cstcurve] (.9,.6) to[bend left=35](3.75,2.25);
      \end{scope}
      \begin{scope}[second,rotate around={75:(1.5,1.5)}]
        \draw[thick] (.75,.75)--(3,3);
        \draw[cstcurve] (.9,.6) to[bend left=35](3.75,2.25);
      \end{scope}
    \end{scope}
    \begin{scope}[shift={(2,-2)}]
      \draw[cstaxis] (0,0)--(4,0);
      \draw[cstaxis] (0,0)--(0,4);
      \begin{scope}[rotate=-20,scale=.8,shift={(.5,.6)}]
        \coordinate (B) at (1.5,1.5);
        \coordinate (A) at (3,3);
        \coordinate (C) at (1,2.366);
        \draw[fourth,thick,cstra] pic [cstfill4,draw=fourth, "$\theta$", angle eccentricity=1.5] {angle=A--B--C};
        \begin{scope}[main]
          \draw[thick] (.75,.75)--(3,3);
          \draw[cstcurve] (.9,.6) to[bend left=35](3.75,2.25);
        \end{scope}
        \begin{scope}[second,rotate around={75:(1.5,1.5)}]
          \draw[thick] (.75,.75)--(3,3);
          \draw[cstcurve] (.9,.6) to[bend left=35](3.75,2.25);
        \end{scope}
      \end{scope}
    \end{scope}
  \end{tikzpicture}
  \caption{解析函数的保角性}
\end{figure}

设
\[
  C:z=z(t)=x(t)+\ii y(t),\ a\le t\le b
\]
是区域 $D$ 内过 $z_0=z(t_0)$ 的一条有向曲线.
若 $z'(t_0)\neq 0$, 则 $C$ 在 $z_0$ 处的切向量为 $z'(t_0)$ 对应的向量, 即 $\bigl(x'(t_0),y'(t_0)\bigr)$, 切线方程为
\[
  \ell: z=z(t_0)+z'(t_0)s,\ s\in\BR.
\]

设 $w=f(z)$ 是区域 $D$ 上的解析函数, 且 $f'(z_0)\neq 0,z_0\in D$.
通过映射 $w=f(z)$ 作用之后, $C$ 的像 $C'$ 的参数方程为
\[
  C':w=f(t)=f\bigl(z(t)\bigr),\ a\le t\le b,
\]
它在 $w_0=f(z_0)$ 处的切向量为 $\dfrac{\d w}{\d t}=f'(z_0) z'(t_0)$, 切线方程为
\[
  \ell': z=f(t_0)+f'(z_0) z'(t_0)s,\ s\in\BR.
\]
由此可知, $f(z)$ 将切线的角度逆时针旋转了 $\arg f'(z_0)$.
自然地, 若在 $z$ 平面内有两条不同的经过 $z_0$ 的曲线 $C_1,C_2$, 它们的夹角与它们在 $w$ 平面内的像 $C_1',C_2'$ 的夹角是相同.
这种性质被称为\nouns{保角性}\index{baojiaoxing@保角性}.
不仅如此, 由 $\d f=f'(z_0)\d z$ 可知 $w=f(z)$ 还将 $z_0$ 附近的点做了伸缩, 伸缩率为 $\abs{f'(z_0)}$.
\footnote{
  若 $f(z)$ 在 $z_0$ 处解析且 $f'(z_0)=0$, 则 $w=f(z)$ 一定不保角.
  设 $z_0$ 是 $f(z)$ 的 $m$ 阶零点, $a=m!f^{(m)}(z_0)\neq 0$, 则
  \[
    f(z_0+r\ee^{\ii\theta})=f(z_0)+ar^m\ee^{m\ii\theta}+o(r^m).
  \]
  于是经过 $z_0$ 的曲线 $C$ 的像 $C'$ 的切线可以选择 $s=r^m$ 为参数, 切线方程为 $\ell':z=f(z_0)+a\ee^{m\ii\theta}s$.
  从而 $w=f(z)$ 将曲线夹角放大为 $m$ 倍.
} 
% \begin{marker}
  \alert{保角性不仅要保持夹角不变, 也要保持夹角的方向.}
% \end{marker}

\begin{theorem}
  \label{thm:analytic-nonzero-coformal}
  设 $w=f(z)$ 在 $z_0$ 处解析且 $f'(z_0)\neq0$, 那么 $w=f(z)$ 在 $z_0$ 处满足如下性质:
  \begin{enuma}
    \item 保角性: 经过 $z_0$ 的两条曲线的夹角和它们在 $w=f(z)$ 下的像的夹角大小和方向都相同;
    \item 伸缩率不变性: 连接 $z_0$ 与其附近点的曲线, 经过映射后被伸缩了 $\abs{f'(z_0)}$ 倍, 这个倍率与具体曲线无关.\footnotemark
  \end{enuma}
\end{theorem}
\footnotetext{
  ``附近''并不是严格的数学概念, 这里实际上就是指 $\abs{\delt w}=\abs{f'(z_0)}\cdot\abs{\delt z}+o(\delt z)$.
}

反过来, 若 $w=f(z)$ 在 $z_0$ 附近有上述特点, 设它将经过 $z_0$ 的水平曲线逆时针旋转 $\theta$ 并放缩 $r$ 倍, 则对于充分小的 $\abs{\delt z}$, 有 $\delt w=r\ee^{\ii\theta}\delt z+o(\delt z)$.
令 $\delt z\ra 0$, 则我们得到 $f(z)$ 在 $z_0$ 处可微且 $f'(z_0)=r\ee^{\ii\theta}$.
\footnote{
  若 $f(z)$ 在 $z_0$ 处可导但不解析, 且 $f'(z_0)=0$, 映射 $w=f(z)$ 有可能在 $z_0$ 处保持夹角(伸缩率为 $0$).
  例如 $f(z)=z\abs{z}$, 即 $f(r\ee^{\ii\theta})=r^2\ee^{\ii\theta}$ 满足 $f'(0)=0$ 且在 $0$ 处保持夹角.
  不过这不是我们关心的情形.
}


\subsection{保形映射的定义}

\begin{definition}
  \begin{enuma}
    \item 若 $w=f(z)$ 在 $z_0$ 处解析, 并具有保角性和伸缩率不变性, 则称 $w=f(z)$ 是 $z_0$ 处的\nouns{保角映射}\index{baojiaoyingshe@保角映射}.
    \item 若 $w=f(z)$ 在区域 $D$ 内处处都是保角映射, 则称 $w=f(z)$ 是 $D$ 内的\nouns{保角映射}\index{baojiaoyingshe@保角映射}.
  \end{enuma}
\end{definition}

由\thmref{定理}{thm:analytic-nonzero-coformal} 立即得知:
\begin{theorem}
  若 $f(z)$ 在区域 $D$ 内解析且导数处处非零, 则它是区域 $D$ 内的保角映射.
\end{theorem}

很多时候, 我们需要把一个复杂区域上的问题通过保角映射化为简单区域上的问题, 例如单位圆域 $\BD:\abs{z}\le1$\index{0d@$\BD$} 或上半平面 $\BH:\Im z>0$\index{0hh@$\BH$}.
因此我们要求所使用的映射是一一对应.

\begin{definition}
  若 $w=f(z)$ 是区域 $D$ 内的一一保角映射, 则称 $f(z)$ 是 $D$ 内的\nouns{保形映射}\index{baoxingyingshe@保形映射}.\footnotemark
\end{definition}
\footnotetext{
  也叫\emph{保形变换}\index{baoxingbianhuan@保形变换}或\emph{共形映射}\index{gongxingyingshe@共形映射}.
}

显然保形映射的逆以及复合也是保形映射.
我们常常利用这一点, 使用多个简单的保形映射将一个区域逐步简化.

若 $w=f(z)$ 是区域 $D$ 内的保形映射, 则它一定是 $D$ 内的解析函数, 且导数处处非零.
注意 $w=f(z)$ 在 $D$ 的边界上可以导数为 $0$ 甚至无定义.

若我们不要求 $w=f(z)$ 保持夹角的方向, 它可以不是解析函数.
例如 $w=\ov z$ 也保持夹角大小和伸缩率不变, 但是它改变了夹角的方向.
% \begin{marker}
  \alert{本书中考虑的保形映射都是保持夹角方向的.}
% \end{marker}


\subsection{扩充复平面上的保形映射}

我们来讨论扩充复平面 $\BC^*$ 上的保形映射.
本章中我们做如下约定:
% \begin{marker}
  \begin{enuma}
    \item 将 $\liml_{z\ra a}f(z)=\infty$ 简记为 $f(a)=\infty$, 将 $\liml_{z\ra \infty}f(z)=A\in\BC^*$ 简记为 $f(\infty)=A$;
    \item 所有的复变函数都看作是 $D\subseteq\BC^*$ 到 $\BC^*$ 的映射.
  \end{enuma}
% \end{marker}

若曲线
\[
  C: z=z(t),\ a<t<b
\]
满足 $\liml_{t\ra a^+}z(t)=\infty$ 或 $\liml_{t\ra b^-}z(t)=\infty$, 则称 $C$ \nouns{延伸至 $\infty$}\index{yanshenzhiinfty@延伸至 $\infty$}.
此时我们把 $\infty$ 也看作是 $C$ 上的一个点.
这里的 $a,b$ 也可以为 $-\infty,+\infty$.

\begin{definition}
  \begin{enuma}
    \item 若 $f(z_0)=\infty$ 且 $\dfrac1w=\dfrac1{f(z)}$ 是 $z_0$ 处的保角映射, 则称 $w=f(z)$ 是 \nouns{$z_0$ 处的保角映射}\index{baojiaoyingshe@保角映射}.
    \item 若 $f(\infty)=w_0$ 且 $w=f\Bigl(\dfrac1t\Bigr)$ 是 $t=0$ 处的保角映射, 则称 $w=f(z)$ 是 \nouns{$\infty$ 处的保角映射}\index{baojiaoyingshe@保角映射}.
    \item 若 $f(\infty)=\infty$ 且 $\dfrac1w=\dfrac1{f(1/t)}$ 是 $t=0$ 处的保角映射, 则称 $w=f(z)$ 是 \nouns{$\infty$ 处的保角映射}\index{baojiaoyingshe@保角映射}.
  \end{enuma}
\end{definition}

对于复球面上的相交曲线, 也可以定义夹角和保形映射.
这些概念通过球极投影和上述对应方式与我们在通常复平面上定义的概念是一致的.

\begin{example}
  \label{exam:linear-transform}
  线性函数 $w=f(z)=az+b,a\neq 0$ 是整个复平面内的保形映射.
  它实际上可以分解成旋转、相似和平移三个映射的复合.
  由于 $f(\infty)=\infty$, 且
  \[
    \dfrac1w=\dfrac1{f(z)}=\dfrac1{f(1/t)}=\dfrac{t}{a+bt}.
  \]
  在 $t=0$ 处解析, 因此 $w=f(z)$ 是 $\infty$ 处的保角映射.
  显然 $w=f(z)$ 是一一的, 故 $w=f(z)$ 是扩充复平面上的保形映射.
\end{example}

\begin{example}
  \label{exam:inverse-transform}
  倒数映射 $w=f(z)=\dfrac1z$ 在任意非零点的导数 $f'(z)=-\dfrac1{z^2}$ 非零, 因此在这些点处是保形映射.
  由于 $f(0)=\infty$, 且 $\dfrac1w=z$ 在 $z=0$ 处是保角映射, 因此 $w=f(z)$ 是 $0$ 处的保角映射.
  由于 $f(\infty)=0$, 且 $w=f\Bigl(\dfrac1t\Bigr)=t$ 在 $t=0$ 处是保角映射, 因此 $w=f(z)$ 是 $\infty$ 处的保角映射.
  显然 $w=f(z)$ 是一一的, 故 $w=f(z)$ 是扩充复平面上的保形映射.
\end{example}

\begin{exercise}
  以下映射中\fillbrace{}是 $0$ 处的保角映射.
  \begin{exchoice}(4)
    \item $w=\ov z$
    \item $w=z^2$
    \item $w=\ln z$
    \item $w=\dfrac1z$
  \end{exchoice}
\end{exercise}

对于扩充复平面上的单连通区域, 我们有如下定理:
\footnote{
  扩充复平面内区域的定义和复平面内区域的定义类似, 只不过需要额外考虑 $\infty$ 的情形.
  若区域 $D$ 内任意一条闭路的内部或外部(包含 $\infty$)都完全属于该区域, 则称该区域为\emph{单连通区域}\index{quyu@区域!danliantongquyu@单连通区域}.
  例如圆周外部(包含 $\infty$)是扩充复平面内的单连通区域.
  从复球面的角度来看更明显.
  对于单连通区域 $D\neq \BC^*$, 通过对应复球面上一点 $z_0\notin D$ 的球极投影, $D$ 可以对应到复平面的一个单连通区域.
}

\begin{theorem}[黎曼映射定理]
  \index{limanyingshedingli@黎曼映射定理}
  \label{thm:riemann-mapping}
  若 $D\neq\BC^*$ 是扩充复平面中的非空单连通区域, 且 $D$ 不是扩充复平面去掉一个点, 则存在 $D$ 到单位圆域的保形映射 $w=f(z)$.
  对于任意常数 $\theta_0$ 和 $D$ 中任意一点 $z_0$, 若要求 $f(z_0)=0,\arg f'(z_0)=\theta_0$, 则存在唯一一个这样的 $f$.
\end{theorem}

由保形映射是一一对应, 且它的逆也是保形映射可得:
\begin{corollary}
  若 $D_1\neq\BC^*,D_2\neq\BC^*$ 是扩充复平面中的非空单连通区域, 且它们不是扩充复平面去掉一个点, 则存在 $D_1$ 到 $D_2$ 的保形映射 $w=f(z)$.
  对于任意常数 $\theta_0$ 和 $D$ 中任意一点 $z_0$, 若要求 $f(z_0)=0,\arg f'(z_0)=\theta_0$, 则存在唯一一个这样的 $f$.
\end{corollary}

不过, 黎曼映射定理只表明存在这样的映射, 具体映射的构造往往需要使用各种函数组合得到.



\section{分式线性映射}

形如
\[
  w=f(z)=\dfrac{az+b}{cz+d},\quad ad-bc\neq 0
\]
的映射叫作\nouns{分式线性映射}\index{fenshixianxingyingshe@分式线性映射}\index{0fz=azbczd@$f(z)=\dfrac{az+b}{cz+d}$\smallskip}\footnote{
  也叫\emph{莫比乌斯映射}\index{mobiwusiyingshe@莫比乌斯映射}.
}.
它是保形映射中比较简单也比较容易研究的一种映射.
不难看出, 若 $ad-bc=0$, 则 $w$ 是一常数, 这不是保形映射也不是我们感兴趣的.

分式线性映射的逆和复合仍然是分式线性映射.
设
\[
  f(z)=\frac{az+b}{cz+d},\quad
  g(z)=\frac{a'z+b'}{c'z+d'},
\]
则
\[
  f^{-1}(z)=\frac{dz-c}{-bz+a},\quad
  (f\circ g)(z)=\frac{(aa'+bc')z+(ab'+bd')}{(ca'+dc')z+(cb'+dd')}.
\]

\subsection{分式线性映射的性质}

\subsubsection{保形性}

我们将分式线性映射改写为如下形式:
\[
   w
  =\frac{az+b}{cz+d}
  =-\frac{ad-bc}{c^2}\cdot\frac1{z+d/c}+\frac ac.
\]
于是它可由三类比较简单的分式线性映射复合得到:
\begin{enuma}
  \item 平移映射 $w=z+b$;
  \item 相似映射 $w=az$;\smallskip
  \item 倒数映射 $w=\dfrac1z$.
\end{enuma}
\smallskip
由\thmref{例}{exam:linear-transform} 和\thmref{例}{exam:inverse-transform}, 我们立即得到:

\begin{theorem}
  分式线性映射是扩充复平面内的保形映射.
\end{theorem}


\subsubsection{保圆性}

由于保形映射具有保角性和伸缩率不变性, 因此它在局部保持图形的形状不变.
不过这种不变只是在一定的程度下的近似.
而对于分式线性映射, 它总把直线和圆周映成直线或圆周.
我们也可以将直线视为扩充复平面中经过 $\infty$ 的半径无限大的圆周.\footnote{
  球极投影总将复球面中的圆周映成复平面中的圆周或直线(含 $\infty$).
}

\begin{theorem}
  分式线性映射把圆周或直线映成圆周或直线.
\end{theorem}

\begin{proof}
  显然平移映射和相似映射把圆周映成圆周, 直线映成直线.
  对于倒数映射 $w=\dfrac1z$, 注意到圆周和直线的方程都可以表示为
  \[
    A(x^2+y^2)+Bx+Cy+D=0,
  \]
  其中 $B^2+C^2-4AD>0$. 由
  \[
      z
    =\frac1w
    =\frac{u}{u^2+v^2}-\ii \frac{v}{u^2+v^2}
  \]
  得到
  \[
    A+Bu-Cv+D(u^2+v^2)=0.
  \]
  因此倒数映射把圆周或直线映成圆周或直线, 命题得证.
\end{proof}

不难看出, 分式线性映射 $w=f(z)$ 将圆周或直线 $C$ 映成直线, 当且仅当 $C$ 上的一点的像为 $\infty$, 即 $-\dfrac dc\in C$.
它也总将直线映成经过点 $\dfrac ac$ 的圆周或直线.


\subsubsection{保对称点}

若 $z_1,z_2$ 的垂直平分线是直线 $\ell$, 或者它们重合且都在直线 $\ell$ 上, 则称 $z_1,z_2$ 关于直线 $\ell$ \nouns{对称}\index{duicheng@对称}.
对于圆周我们也可以定义对称点.
设 $C:\abs{z-z_0}=R$ 是一圆周, 若 $z_1,z_2$ 在从 $z_0$ 出发的同一条射线上, 且 $\abs{z_1-z_0}\cdot\abs{z_2-z_0}=R^2$, 则称 $z_1,z_2$ 关于圆周 $C$ \nouns{对称}\index{duicheng@对称}.
圆周上的点总和其自身对称, 除此之外, 关于圆周对称的点一定是一个在圆周内, 一个在圆周外.
我们还规定 $z_0$ 和 $\infty$ 关于 $C$ 对称.

\begin{lemma}
  \label{lem:symmetry-circle}
  点 $z_1,z_2$ 关于圆周 $C:\abs{z-z_0}=R$ 对称, 当且仅当经过 $z_1,z_2$ 的任一圆周和 $C$ 正交, 即在交点处切线垂直.
\end{lemma}

\begin{proof}
  不妨设 $z_1=-1,z_2=1$.
  那么经过 $z_1,z_2$ 的圆周 $\Gamma$ 的圆心可设为 $\gamma=\lambda\ii,\lambda\in\BR$, 半径为 $r=\sqrt{1+\lambda^2}$.
  注意 $C$ 和 $\Gamma$ 正交等价于
  \begin{equation}
    \label{eq:orth-symmetry-points}
     \abs{z_0-\lambda\ii}^2
    =R^2+r^2
    =R^2+1+\lambda^2,\quad
     \abs{z_0}^2+\lambda(z_0-\ov{z_0})\ii
    =R^2+1.
  \end{equation}

  若经过 $z_1,z_2$ 的所有圆周 $\Gamma$ 都与 $C$ 正交, 则上式对任意 $\lambda\in\BR$ 均成立, 从而 $z_0=\ov{z_0}$, $z_0=\pm\sqrt{R^2+1}$ 为实数.
  \[
    \abs{z_1-z_0}\cdot\abs{z_2-z_0}=\abs{z_0^2-1}=R^2
  \]
  且 $z_1,z_2$ 在 $z_0$ 同侧, 即 $z_1,z_2$ 关于 $C$ 对称.

  反过来, 若 $z_1,z_2$ 关于 $C$ 对称, 则 $z_0\in\BR$ 满足
  \[
    R^2=(z_1-z_0)(z_2-z_0)=z_0^2-1.
  \]
  于是 \ref{eq:orth-symmetry-points} 恒成立, 即 $C$ 和 $\Gamma$ 正交.
\end{proof}

\begin{figure}[H]
  \centering
  \begin{tikzpicture}
    \def\a{2.86}
    \def\b{.84}
    \def\s{.5}
    \begin{scope}[scale=.5,rotate=56.31]
      \coordinate (P);
      \coordinate (Z) at (-4,0);
      \coordinate (O) at (0,-3);
      \coordinate (Z1) at ({-36/13},{-24/13});
      \coordinate (Z2) at (0,{-78/13});
      \coordinate (Z12) at ($(Z1)!.5!(Z2)$);
      \begin{scope}[cstcurve,third]
        \draw[main] (Z) circle (4);
        \draw[second] (O) circle (3);
        \draw[cstdash] (Z)--(O)--(P)--cycle;
        \draw (P)++(-\s,0)--++(0,-\s)--++(\s,0);
      \end{scope}
      \draw[cstaxis] (Z)--($(Z1)!1.5!(Z2)$);
      \draw[cstaxis] ($(Z12)!-1.5!(O)$)--($(Z12)!3.5!(O)$);
    \end{scope}
    \fill[cstdot,main] (Z) circle node[below] {$z_0$};
    \fill[cstdot,fourth] (Z1) circle node[below left] {$z_1$};
    \fill[cstdot,fourth] (Z2) circle node[below right] {$z_2$};
    \fill[cstdot,second] (O) circle node[right] {$\gamma$};
  \end{tikzpicture}
  \caption{过对称点圆周与原圆周正交}
\end{figure}

把上述命题的圆周换成圆周或直线也成立.

\begin{theorem}
  分式线性映射具有保对称点性, 即若 $z_1,z_2$ 关于圆周或直线 $C$ 对称, 则 $f(z_1),f(z_2)$ 关于 $C$ 在 $w=f(z)$ 下的像对称.
\end{theorem}

\begin{proof}
  过 $w_1=f(z_1),w_2=f(z_2)$ 作圆周或直线 $\Gamma'$.
  由保圆性可知 $\Gamma'$ 的原像 $\Gamma$ 也是圆周或直线.
  由\thmref{引理}{lem:symmetry-circle} 可知 $C$ 和 $\Gamma$ 正交, 再由保角性可知 $C$ 的像 $C'$ 和 $\Gamma'$ 正交.
  最后由\thmref{引理}{lem:symmetry-circle} 可知 $f(z_1),f(z_2)$ 关于 $C'$ 对称.
\end{proof}


\subsubsection{约束条件}

尽管分式线性映射的表达式中有四个参数, 但把它们都换成非零常数倍之后对应的是同一个分式线性映射, 所以应当只需要三个约束条件就可以确定分式线性映射.
下述定理说明的确如此.

\begin{theorem}
  \label{thm:three-points-determine-fractal-transform}
  设 $z_1,z_2,z_3$ 是 $z$ 平面三个不同的点, $w_1,w_2,w_3$ 是 $w$ 平面三个不同的点.
  那么存在唯一一个分式线性映射 $w=f(z)$, 使得 $f(z_k)=w_k,\ k=1,2,3$, 它满足
  \[
     \frac{w-w_1}{w-w_2}\cdot\frac{w_3-w_2}{w_3-w_1}
    =\frac{z-z_1}{z-z_2}\cdot\frac{z_3-z_2}{z_3-z_1}.
  \]
\end{theorem}

\begin{proof}
  显然该等式确定了一个满足题设的分式线性映射, 下证唯一性.
  设 $w=\dfrac{az+b}{cz+d}$ 是满足题设的一个分式线性映射, 则
  \[
     w-w_k
    =\frac{az+b}{cz+d}-\frac{az_k+b}{cz_k+d}
    =\frac{(ad-bc)(z-z_k)}{(cz+d)(cz_k+d)}.
  \]
  于是对于 $k=1,2$, 有
  \[
    w_3-w_k=\frac{(ad-bc)(z_3-z_k)}{(cz_3+d)(cz_k+d)}.
  \]
  从而
  \[
    \frac{w-w_k}{w_3-w_k}
   =\frac{z-z_k}{z_3-z_k}\cdot\frac{cz_3+d}{cz+d}.
  \]
  分别代入 $k=1,2$ 并将两式相除即可得到定理中的等式.
\end{proof}

实际上, 等式右侧的分式线性映射分别将 $z_1,z_2,z_3$ 映成 $0,\infty,1$, 代入后解得左侧的 $w$ 把右侧的 $0,\infty,1$ 映成 $w_1,w_2,w_3$.

\begin{corollary}
  若分式线性映射 $w=f(z)$ 将 $z_1,z_2$ 映成 $w_1,w_2$, 则存在常数 $k$ 使得
  \[
    \frac{w-w_1}{w-w_2}=k\frac{z-z_1}{z-z_2}.
  \]
  特别地, 若 $f(a)=0,f(b)=\infty$, 则 $w=k\dfrac{z-a}{z-b}$.
\end{corollary}

在 $z$ 平面和 $w$ 平面分别给定圆周 $C,C'$.
那么我们可以利用\thmref{定理}{thm:three-points-determine-fractal-transform} 构造出分式线性映射 $w=f(z)$ 将 $C$ 映成 $C'$, 只需要分别在 $C,C'$ 上任取三个不同的点即可.
由于 $w=f(z)$ 是扩充复平面内的连续一一对应, 因此 $C$ 内部的连续曲线在该映射下的像还是一条连续曲线, 且与 $C'$ 没有交点.
这意味着该映射把 $C$ 的内部映射为 $C'$ 的内部或外部.

\begin{figure}[H]
  \centering
  \begin{tikzpicture}
    \begin{scope}[shift={(-4,0)}]
      \coordinate (A) at (.84,-1.2);
      \coordinate (B) at (0,-1.2);
      \coordinate (C) at (-.36,0);
      \draw[thick,fourth] (A)--(B)--(C);
      \draw[fourth,thick,cstla] pic [cstfill4,draw=fourth, "$\theta$", angle eccentricity=1.7, angle radius=3mm] {angle=A--B--C};
      \draw[cstcurve] (-.72,-.36) to[bend left=30] (0,-1.2);
      \fill[cstdot,third] (-.72,-.36) circle node[above] {$z_0$};
      \fill[cstdot,main] (0,-1.2) circle node[below] {$z_1$};
      \fill[cstdot,main] (.96,.72) circle node[right] {$z_2$};
      \fill[cstdot,main] (-.96,.72) circle node[left] {$z_3$};
      \draw[cstcurve,main] (0,0) circle (1.2);
    \end{scope}
    \begin{scope}[shift={(.5,0)},rotate=60]
      \coordinate (A) at (1.05,-1.5);
      \coordinate (B) at (0,-1.5);
      \coordinate (C) at (-.45,0);
      \draw[thick,fourth] (A)--(B)--(C);
      \draw[fourth,thick,cstla] pic [cstfill4,draw=fourth, "$\theta$", angle eccentricity=1.7, angle radius=3mm] {angle=A--B--C};
      \draw[cstcurve] (-.9,-.45) to[bend left=30] (0,-1.5);
      \fill[cstdot,third] (-.9,-.45) circle node[left] {$w_0$};
      \fill[cstdot,main] (0,-1.5) circle node[right] {$w_1$};
      \fill[cstdot,main] (1.2,.9) circle node[above] {$w_2$};
      \fill[cstdot,main] (-1.2,.9) circle node[left] {$w_3$};
      \draw[cstcurve,main] (0,0) circle (1.5);
    \end{scope}
    \begin{scope}[shift={(.5,0)},rotate=90]
      \coordinate (A) at (-1.05,-1.5);
      \coordinate (B) at (0,-1.5);
      \coordinate (C) at (.45,-3);
      \draw[thick,fourth] (A)--(B)--(C);
      \draw[fourth,thick,cstla] pic [cstfill4,draw=fourth, "$\theta$", angle eccentricity=1.7, angle radius=3mm] {angle=A--B--C};
      \draw[cstcurve] (.9,-2.55) to[bend left=30] (0,-1.5);
      \fill[cstdot,third] (.9,-2.55) circle node[above] {$w_0$};
      \fill[cstdot,main] (0,-1.5) circle node[left] {$w_1$};
      \fill[cstdot,main] (1.2,.9) circle node[left] {$w_3$};
      \fill[cstdot,main] (-1.2,.9) circle node[below] {$w_2$};
      \draw[cstcurve,main] (0,0) circle (1.5);
    \end{scope}
  \end{tikzpicture}
  \caption{分式线性映射将圆周内部映成圆周内部或外部}
\end{figure}

若希望该映射把 $C$ 的内部映成 $C'$ 的内部, 我们只需要在 $C$ 上逆时针依次选择 $z_1,z_2,z_3$, 在 $C'$ 上也逆时针依次选择 $w_1,w_2,w_3$ 即可.
这是因为对于 $C$ 内部任意一点 $z_0$, 连接 $z_0,z_1$ 的曲线 $\Gamma$ 到逆时针圆弧 $\warc{z_1z_2}$ 一定是顺时针旋转角度 $\theta<\cpi$.
由于分式线性映射具有保角性, 因此 $\Gamma$ 的像到逆时针圆弧 $\warc{w_1w_2}$ 也是顺时针旋转角度 $\theta$, 从而 $f(z_0)$ 只能在 $C'$ 内部.
若希望分式线性映射将 $C$ 的内部映射为 $C'$ 的外部, 只需顺时针依次选择 $w_1,w_2,w_3$.
若 $C$ 或 $C'$ 是直线, 只需将圆周内部替换为三个点顺序所指方向的左侧即可.


\subsection{分式线性映射举例}

\begin{example}
  分式线性映射
  \[
    w=f(z)=\dfrac{z+\ii}{z-\ii}
  \]
  将中心分别在 $\pm\dfrac{\sqrt3}3$, 半径为 $\dfrac{2\sqrt3}3$ 的圆弧围成的区域 $D$ 映成什么区域 $D'$?
\end{example}

\begin{figure}[H]
  \centering
  \begin{tikzpicture}
    \draw[cstcurve,cstnra,third] (-1,0)-- node[above] {$w=f(z)$} (1,0);
    \begin{scope}[shift={(-4,0)}]
      \filldraw[cstcurve,main,cstfill1] (0,1) arc(120:240:1.155)
        arc(-60:60:1.155) --cycle;
      \draw[cstaxis] (-2,0)--(2,0);
      \draw[cstaxis] (0,-2)--(0,2);
      \draw[cstdash] 
        (.577,0)--
        (0,1) --
        (-.577,0) node[below left] {$-\frac{\sqrt3}3$}--
        (0,-1) --cycle;
      \coordinate (A) at (-1.732,0);
      \coordinate (O) at (0,1);
      \coordinate (B) at (1.732,0);
      \draw[thick,cstra] pic [draw=main] {angle=A--O--B};
      \draw[thick,main] ($(A)!.5!(O)$)--(O)--($(B)!.5!(O)$);
      \fill[cstdot,third] (0,-1) circle node[below right] {$z_1$};
      \fill[cstdot,third] (.577,0) circle node[below right] {$z_2$};
      \fill[cstdot,third] (0,1) circle node[above right] {$z_3$};
    \end{scope}
    \begin{scope}[shift={(4,0)}]
      \coordinate (A) at (-1,1.732);
      \coordinate (O) at (0,0);
      \coordinate (B) at (-1,-1.732);
      \fill[cstcurve,cstfille5] (A) arc(120:240:2)--(O)--cycle;
      \draw[fifth,thick,cstra] pic [cstfill5,draw=fifth] {angle=A--O--B};
      \draw[cstcurve,fifth] (A)--(O)--(B);
      \draw[cstdash,fifth] (-1,1.732)--(-1.3,{1.3*sqrt(3)});
      \draw[cstaxis] (-2,0)--(2,0);
      \draw[cstaxis] (0,-2)--(0,2);
      \fill[cstdot,third] (0,0) circle node[above right] {$w_1=0$};
      \fill[cstdot,third] (-.5,.866) circle node[above right,inner sep=1pt] {$w_2$};
      \fill[cstdot,third] (-1.3,{1.3*sqrt(3)}) circle node[left] {$w_3 =\infty$};
    \end{scope}
  \end{tikzpicture}
  \caption{两圆弧围成的区域到角形区域的映射}
\end{figure}

\begin{solution}
  由于
  \[
    f(\ii)=\infty,\quad
    f(-\ii)=0,\quad
    f\Bigl(\dfrac{\sqrt3}3\Bigr)=-\frac12+\frac{\sqrt3}2\ii,
    \quad
    f\Bigl(-\dfrac{\sqrt3}3\Bigr)=-\frac12-\frac{\sqrt3}2\ii,
  \]
  因此它将两段圆弧映成两条经过 $0,-\dfrac12\pm\dfrac{\sqrt3}2\ii,\infty$ 的射线, 即 $\arg w=\pm\dfrac{2\cpi}3$.
  由于
  \[
    z_1=-\ii,\quad z_2=\frac{\sqrt3}3,\quad z_3=\ii
  \]
  在圆弧上逆时针排列, $D'$ 在它们的像
  \[
    w_1=0,\quad w_2=-\frac12+\frac{\sqrt3}2\ii,\quad w_3=\infty
  \]
  所指方向的左侧, 因此 $D'$ 为射线 $\arg w=\pm\dfrac{2\cpi}3$ 所夹成的夹角为 $\dfrac{2\cpi}3$ 的角形区域
  \[
    D':\dfrac{2\cpi}3<\Arg w<\dfrac{4\cpi}3.
  \]
\end{solution}

若要求将特定区域映成另一区域的分式线性映射, 我们可分别在它们的边界上选择三个点得到相应映射.
需要注意选择的点的次序.

\begin{example}
  \label{exam:imz-open-disk}
  求将上半平面 $\BH:\Im z>0$ 映成单位圆域 $\BD:\abs{w}<1$ 的分式线性映射 $w=f(z)$.
\end{example}

\begin{figure}[H]
  \centering
  \begin{tikzpicture}
    \draw[cstcurve,cstnra,third] (-1,0)-- node[above] {$w=f(z)$} (1,0);
    \def\w{2}
    \begin{scope}[shift={(-4,0)}]
      \fill[cstcurve,main,cstfille1] (-\w,\w) rectangle (\w,0);
      \draw[cstaxis,cstcurve,main] (-\w,0)--(\w,0);
      \draw[cstaxis] (0,-\w)--(0,\w);
      \fill[cstdot,third] (-1,0) circle node[below] {$z_1$};
      \fill[cstdot,third] (0,0) circle node[below left] {$z_2$};
      \fill[cstdot,third] (1,0) circle node[below] {$z_3$};
      \draw (\w,\w) node[main,below left] {$\BH$};
    \end{scope}
    \begin{scope}[shift={(4,0)}]
      \filldraw[cstcurve,fifth,cstfill5] (0,0) circle (1);
      \draw[cstaxis] (-\w,0)--(\w,0);
      \draw[cstaxis] (0,-\w)--(0,\w);
      \fill[cstdot,fourth] (0,0) circle;
      \fill[cstdot,third] (0,-1) circle node[below left] {$w_1$};
      \fill[cstdot,third] (1,0) circle node[above right] {$w_2$};
      \fill[cstdot,third] (0,1) circle node[above left] {$w_3$};
      \draw (.7,.7) node[fifth,below left] {$\BD$};
    \end{scope}
  \end{tikzpicture}
  \caption{$\BH$ 到 $\BD$ 的映射}
\end{figure}

\begin{solution}
  我们在直线 $\Im z=0$ 上自左往右选择 $z_1=-1,z_2=0,z_3=1$, 并在单位圆周 $\abs{z}=1$ 上逆时针选择 $w_1=-\ii,w_2=1,w_3=\ii$.
  由\thmref{定理}{thm:three-points-determine-fractal-transform} 可得
  \[
     \frac{w+\ii}{w-1}\cdot\frac{\ii-1}{\ii+\ii}
    =\frac{z+1}{z-0}\cdot\frac{1+1}{1-0}.
  \]
  解得
  \[
    w=-\dfrac{z-\ii}{z+\ii}.
  \]
\end{solution}

若我们换一组点, 可能会得到不同的分式线性映射.
若 $f_1,f_2$ 是两个这样的分式线性映射, 则 $f_2\circ f_1^{-1}$ 是 $\BD$ 到自身的分式线性映射.

\begin{example}
  求单位圆域 $\BD$ 到自身的所有分式线性映射 $w=f(z)$.
\end{example}

\begin{solution}
  设 $w=f(z)=\dfrac{az+b}{cz+d}$. \smallskip 
  若 $\alpha=f^{-1}(0)$, 则 $\abs{\alpha}<1$ .
  由于分式线性映射保对称点, 因此 $f$ 将 $\alpha$ 的对称点 $\dfrac1{~\ov\alpha~}$ 映成 $0$ 的对称点 $\infty$.
  这意味着
  \[
    w=k\frac{z-\alpha}{z-1/\ov\alpha}=k'\frac{z-\alpha}{1-\ov\alpha z}.
  \]
  由于 $f(1)=k'\dfrac{1-\alpha}{1-\ov\alpha}$ 的模为 $1$, 因此 $\abs{k'}=1$.
  设 $k'=\ee^{\ii\theta}$, 则
  \[
    w=\ee^{\ii\theta}\frac{z-\alpha}{1-\ov\alpha z},\quad \abs{\alpha}<1.
  \]
  不难知道该映射将单位圆周映成单位圆周, 且 $\abs{f(0)}<1$.
  因此这就是 $\BD$ 到自身的所有分式线性映射.
\end{solution}

事实上, 单位圆域到自身的保形映射一定是分式线性映射, 见\thmref{定理}{thm:riemann-mapping}.
由此也不难知道\thmref{例}{exam:imz-open-disk} 中所有可能的分式线性映射是
\begin{equation}
  \label{eq:coformal-h-to-d}
  w=\ee^{\ii\theta}\dfrac{z+\ov\beta}{z+\beta},\quad\Im\beta>0.
\end{equation}

\begin{example}
  求将右半平面 $D:\Re z>0$ 映成圆域 $D':\abs{w-2}<2$ 的分式线性映射 $w=f(z)$, 使得 $f(0)=0,f(1)=2$.
\end{example}

\begin{figure}[H]
  \centering
  \begin{tikzpicture}
    \draw[cstcurve,cstnra,third] (-1,0)-- node[above] {$w=f(z)$} (1,0);
    \def\w{2}
    \begin{scope}[shift={(-4,0)},scale=.8]
      \fill[cstcurve,main,cstfille1] (0,\w) rectangle (\w,-\w);
      \draw[cstaxis,cstcurve,main] (0,-\w)--(0,\w);
      \draw[cstaxis] (-\w,0)--(\w,0);
      \coordinate (Z1) at (1,0);
      \coordinate (Z2) at (0,0);
      \coordinate (Z3) at (-1,0);
      \draw (\w,\w) node[main,below left] {$D$};
    \end{scope}
    \begin{scope}[shift={(4,0)},scale=.64]
      \filldraw[cstcurve,fifth,cstfill5] (2,0) circle (2);
      \coordinate (W1) at (2,0);
      \coordinate (W2) at (0,0);
      \coordinate (W3) at (-2.5,0);
      \draw[cstaxis] (0,-2.5)--(0,2.5);
      \draw[cstaxis] (0,0)--(5,0);
      \draw[cstdash] (0,0)--(W3);
      \draw (2,2) node[fifth,below] {$D'$};
    \end{scope}
    \begin{scope}[cstdot,third]
      \fill (Z1) circle node[below] {$z_1$};
      \fill (Z2) circle node[below right] {$z_2$};
      \fill (Z3) circle node[below] {$z_3$};
      \fill (W1) circle node[below] {$w_1$};
      \fill (W2) circle node[above left] {$w_2$};
      \fill (W3) circle node[below] {$w_3 =\infty$};
    \end{scope}
  \end{tikzpicture}
  \caption{右半平面到圆域的映射}
\end{figure}

\begin{solution}[解法一]
  通过 $s=\ii z$ 可以将 $D$ 映成 $\BH$, 再根据\eqref{eq:coformal-h-to-d}, 映射
  \[
    t=\ee^{\ii\theta}\dfrac{s+\ov\beta}{s+\beta},\quad\Im\beta>0
  \]
  将 $\BH$ 映成 $\BD$.
  最后映射 $w=2+2t$ 将 $\BD$ 映成 $D'$.
  将三者复合得到
  \[
    w=2+2\ee^{\ii\theta}\dfrac{\ii z+\ov\beta}{\ii z+\beta}.
  \]
  由 $f(0)=0$ 可知 $\ee^{\ii\theta}=-\dfrac{\beta}{~\ov\beta~}$, 于是
  \[
    w=2-2\frac{\beta}{~\ov\beta~}\cdot\frac{\ii z+\ov\beta}{\ii z+\beta}
     =2\Bigl(1-\frac{\beta}{~\ov\beta~}\Bigr)\cdot\frac{z}{z-\ii\beta}.
  \]
  再由 $f(1)=2$ 可得 $\beta=\ii$. 故 $w=\dfrac{4z}{z+1}$.
\end{solution}

\begin{solution}[解法二]
  由于 $1,-1$ 关于直线 $\Re z=0$ 对称, 因此 $w=f(-1)$ 为 $2$ 关于 $\abs{w-2}=2$ 的对称点, 即 $\infty$.
  由于 $f(0)=0$, 因此存在 $k$ 使得 $w=\dfrac{kz}{z+1}$.
  再由 $f(1)=2$ 可得 $k=4$.
  故
  \[
    w=\dfrac{4z}{z+1}.
  \]
\end{solution}

\begin{exercise}
  求将右半平面 $D:\Re z>0$ 映成圆周外部 $D':\abs{w-2}>2$ 的分式线性映射 $w=f(z)$, 使得 $f(0)=0,f(-1)=2$.
\end{exercise}

对于两个圆周或直线围成的区域, 我们需要将交点映射为交点, 然后再选择第三个点.

\begin{example}
  求将角形区域 $D:-\dfrac\cpi4<\arg z<\dfrac{\cpi}4$ 映成单位右半圆域 $D':\abs{w}<1,\Re w>0$ 的分式线性映射 $w=f(z)$.
\end{example}

\begin{figure}[H]
  \centering
  \begin{tikzpicture}
    \draw[cstcurve,cstnra,third] (-1,0)-- node[above] {$w=f(z)$} (1,0);
    \begin{scope}[shift={(-4,0)}]
      \def\u{2}
      \fill[cstcurve,main,cstfille1] (1.414,1.414) arc(45:-45:\u) --(0,0)--cycle;
      \coordinate (A) at (1,-1);
      \coordinate (O) at (0,0);
      \coordinate (B) at (1,1);
      \draw[thick,cstra] pic [cstfill1,draw=main] {angle=A--O--B};
      \draw[cstcurve,main] (1.414,1.414)--(0,0)--(1.414,-1.414);
      \draw[cstaxis] (-\u,0)--(\u,0);
      \draw[cstaxis] (0,-2)--(0,2);
      \draw[cstdash,main] (1.414,-1.414)--(2,-2);
      \coordinate (Z1) at (0,0);
      \coordinate (Z2) at (1,-1);
      \coordinate (Z3) at (\u,-\u);
      \draw ({\u*.7},{\u*.3}) node[main] {$D$};
    \end{scope}
    \begin{scope}[shift={(4,0)}]
      \filldraw[cstcurve,fifth,cstfill5] (0,1) arc(90:-90:1)--cycle;
      \coordinate (A) at (0,.75);
      \coordinate (O) at (0,1);
      \coordinate (B) at (.75,1);
      \draw[thick,cstra] pic [draw=fourth] {angle=A--O--B};
      \draw[cstaxis] (0,-2)--(0,2);
      \draw[cstaxis] (-2,0)--(2,0);
      \draw[thick,fourth] (A)--(O)--(B);
      \coordinate (W1) at (0,1);
      \coordinate (W2) at (0,0);
      \coordinate (W3) at (0,-1);
      \draw (.6,.3) node[fifth] {$D'$};
    \end{scope}
    \begin{scope}[cstdot,third]
      \fill (Z1) circle node[below left] {$z_1$};
      \fill (Z2) circle node[below left] {$z_2$};
      \fill (Z3) circle node[right] {$z_3 =\infty$};
      \fill (W1) circle node[below left] {$w_1$};
      \fill (W2) circle node[below left] {$w_2$};
      \fill (W3) circle node[below left] {$w_3$};
    \end{scope}
  \end{tikzpicture}
  \caption{角形区域映成半圆盘}
\end{figure}

\begin{solution}
  该映射将圆周或直线的交点映成交点, 不妨设 $f(0)=\ii,f(\infty)=-\ii$.
  于是可设
  \[
    w=-\ii\dfrac{z-k}{z+k}.
  \]
  为了保证 $w=f(z)$ 将 $\arg z=-\dfrac\cpi4$ 所在直线的右上侧映成 $\Re w=0$ 的右侧, 我们选择 $f(1-\ii)=0$.
  于是
  \[
    0=f(1-\ii)=-\ii\frac{1-\ii-k}{1-\ii+k},\quad
    k=1-\ii,
  \]
  故 $w=-\ii\dfrac{z-1+\ii}{z+1-\ii}$.
\end{solution}

\begin{example}
  求将偏心圆环域 $D:\abs{z}>1,\abs{z-1}<\dfrac{3\sqrt2}2$ 映成圆环域 $D':1<\abs{w}<2$ 的分式线性映射 $w=f(z)$.
\end{example}

\begin{figure}[H]
  \centering
  \begin{tikzpicture}
    \draw[cstcurve,cstnra,third] (-1,0)-- node[above] {$w=f(z)$} (1,0);
    \def\u{.8}
    \begin{scope}[shift={(-5,0)}]
      \filldraw[cstcurve,main,cstfill1] (\u,0) circle ({\u*2.121});
      \filldraw[cstcurve,main,fill=white] (0,0) circle (\u);
      \draw[cstaxis] (0,-2.2)--(0,2.2);
      \draw[cstaxis] (-2,0)--(3,0);
      \fill[cstdot,main] (0,0) circle;
      \fill[cstdot,third] (\u,0) circle node[below right] {$z_3$};
      \fill[cstdot,third] (-{\u*.5},0) circle node[below] {$z_2$};
      \fill[cstdot,third] (-{\u*2},0) circle node[below] {$z_1$};
      \draw ({\u*2.414},{\u*1.414}) node[main,below left] {$D$};
    \end{scope}
    \begin{scope}[shift={(4.5,0)}]
      \filldraw[cstcurve,fifth,cstfill5] (0,0) circle ({\u*2});
      \filldraw[cstcurve,fifth,fill=white] (0,0) circle (\u);
      \fill[cstdot,fifth] (0,0) circle;
      \draw[cstaxis] (0,-2.2)--(0,2.2);
      \draw[cstaxis] (-{\u*2},0)--(2,0);
      \draw[cstdash] (-{\u*2},0)--(-2.5,0);
      \fill[cstdot,third] (-2.5,0) circle node[below] {$w_1=\infty$};
      \fill[cstdot,third] (0,0) circle node[below left] {$w_2$};
      \fill[cstdot,third] (\u,0) circle node[below right] {$w_3$};
      \draw ({\u*1.414},{\u*1.414}) node[fifth,below left] {$D'$};
    \end{scope}
  \end{tikzpicture}
  \caption{偏心圆环域到圆环域的映射}
\end{figure}

\begin{solution}
  我们首先求偏心圆环域 $D$ 边界两个圆周的公共对称点 $z_1,z_2$.
  由于它们和两圆圆心都共线, 因此它们位于两圆圆心连线上, 也就是说 $z_1,z_2$ 是实数.
  不仅如此, 它们一定在两圆圆心的同一侧, 从而
  \[
    (z_1-0)\cdot (z_2-0)=1,\quad
    (z_1-1)\cdot(z_2-1)=\frac92.
  \]
  整理得到 $z_1+z_2=-\dfrac{5}2$.
  于是 $z_1,z_2$ 是一元二次方程 $z^2+\dfrac52z+1=0$ 的两个解, 即 $-\dfrac12$ 和 $-2$.
  注意到圆环域 $D'$ 边界两个同心圆的公共对称点只能是圆心 $0$ 和无穷远点 $\infty$.
  因此 $-\dfrac12$ 和 $-2$ 的像就是 $0$ 和 $\infty$.

  若 $f\Bigl(-\dfrac12\Bigr)=0$, $f(-2)=\infty$, 则 $w=k\dfrac{2z+1}{z+2}$. \smallskip
  由于 $-\dfrac12$ 在 $D$ 内圆内部, 它的像 $0$ 在 $D'$ 内圆内部, 因此 $w=f(z)$ 将 $D$ 内圆内部映到 $D'$ 内圆内部, 从而将内圆 $\abs{z}=1$ 映到内圆 $\abs{w}=1$.
  设 $f(1)=\ee^{\ii\theta}$, 则
  \[
    w=\ee^{\ii\theta}\dfrac{2z+1}{z+2}.
  \]
  同理, 若 $f\Bigl(-\dfrac12\Bigr)=0$, $f(-2)=\infty$, 则 
  \[
    w=\ee^{\ii\theta}\dfrac{2z+4}{2z+1}.
  \]
\end{solution}


\subsection{分式线性映射的其它表现\optional}

分式线性映射的复合和逆变换的形式与矩阵的乘法和逆的形式很相像.
事实上这并非偶然.
我们将分式线性映射 $w=f(z)=\dfrac{az+b}{cz+d}$ 对应到矩阵 $\bfA=\begin{pmatrix}
  a&b\\c&d
\end{pmatrix}$.
显然 $\bfA$ 和它的常数倍对应同一个分式线性映射.
对于 $z\in\BC^*$, 定义复直线
\[
  V_z=\begin{cases}
    \{(zt,t)^\rmT\mid t\in\BC\},&z\in\BC,\\
    \{(t,0)^\rmT\mid t\in\BC\},&z=\infty,
  \end{cases}
\]
它是 $\BC^2$ 中的一维线性子空间.
由于
\begin{align*}
   \bfA(zt,t)^\rmT&
  =\begin{pmatrix}
    a&b\\
    c&d
  \end{pmatrix}\begin{pmatrix}
    zt\\
    t
  \end{pmatrix}
  =\begin{pmatrix}
    (az+b)t\\
    (cz+d)t
  \end{pmatrix}\in V_{f(z)},\\
   \bfA(t,0)^\rmT&
  =\begin{pmatrix}
    a&b\\
    c&d
  \end{pmatrix}\begin{pmatrix}
    t\\
    0
  \end{pmatrix}
  =\begin{pmatrix}
    at\\
    ct
  \end{pmatrix}\in V_{f(\infty)},
\end{align*}
因此 $\bfA$ 将 $V_z$ 映成 $V_{f(z)}$.
若分式线性映射 $f,g$ 分别对应矩阵 $\bfA,\bfB$, 则 $f\circ g$ 就是 $\bfA\bfB$ 所对应的分式线性映射, $f^{-1}$ 就是 $\bfA^{-1}$ 所对应的分式线性映射.

\begin{example}
  设 $f(z)=\dfrac{\ii z+1}{\ii z-1}$.
  证明 $f\circ f\circ f$ 是恒等映射.
\end{example}

\begin{proof}
  $f$ 对应的矩阵 $\bfA=\begin{pmatrix}
    \ii&1\\
    \ii&-1
  \end{pmatrix}$.
  由于
  \[
    \bfA^2=\begin{pmatrix}
      -1+\ii&-1+\ii\\
      -1-\ii&1+\ii
    \end{pmatrix},\quad
    \bfA^3=\begin{pmatrix}
      -2-2\ii&\\
      &-2-2\ii
    \end{pmatrix},
  \]
  因此 $f\circ f\circ f$ 是恒等映射.
\end{proof}

由于分式线性映射是扩充复平面 $\BC^*$ 到自身的变换, 因此它也诱导了复球面 $S$ 上的变换.
如果 $\bfA$ 满足 $\bfA{\ov\bfA}^\rmT=\bfE$, 我们称之为\nouns{酉矩阵}\index{youjuzhen@酉矩阵}.
此时它具有形式
\[
  \bfA=\begin{pmatrix}
    z_1&-z_2\\
    \ov{z_2}&\ov{z_1}
  \end{pmatrix},
\]
其中 $\abs{z_1}^2+\abs{z_2}^2=1$.
它所对应的分式线性映射 $f$ 诱导了 $S$ 上的旋转变换.
为了使该分式线性映射与其对应的四元数 $q=z_1+z_2\jj$ 表示的旋转一致, 我们重新选择 $S$ 所在空间的 $x,y,z$ 轴如\ref{fig:xyz-complex-sphere} 所示.
那么球极投影 \ref{eq:polar-project} 变为
\begin{align*}
  \varphi:\BC^*&\lra S\\
  x+y\ii&\lto\biggl(\frac{\abs{z}^2-1}{\abs{z}^2+1},\frac{2y}{\abs{z}^2+1},-\frac{2x}{\abs{z}^2+1}\biggr),\\
  \frac{c-b\ii}{a-1}&\lto (a,b,c).
\end{align*}

\begin{figure}[H]
  \centering
  \begin{tikzpicture}
    \begin{scope}[scale=1.2]
      \def\a{.7}
      \fill[cstfill1,scale=.7] (-3.65,-.804)--(-1.85,.804)--(3.65,.804)--(1.85,-.804)--cycle;
      \filldraw[cstcurve,cstfill,fill opacity=.5] (0,0) circle (1);
      \draw[cstdash,main] (0,0) circle (1 and 0.3);
      \coordinate (N) at (0,1);
      \draw[cstdash] (0,0)--(N)
        node[above left] {$N$};
      \draw[cstaxis] (N)--(0,1.8)
        node[above] {$x$ 轴};
      \draw[cstdash] (0,0)--(1,0);
      \draw[cstaxis] (1,0)--(2,0)
        node[right] {虚轴 ($y$ 轴)};
      \draw[cstdash] (0,0)--({-.8*\a},{-.9*\a});
      \draw[cstaxis] ({-.8*\a},{-.9*\a})--(-.96,-1.08)
        node[left] {实轴};
      \draw[cstdash] (0,0)--({.8*\a},{.9*\a});
      \draw[cstaxis] ({.8*\a},{.9*\a})--(.96,1.08)
        node[right] {$z$ 轴};
    \end{scope}
    \fill[cstdot] (N) circle;
  \end{tikzpicture}
  \caption{复球面所在空间的坐标系}
  \label{fig:xyz-complex-sphere}
\end{figure}

特别地,
\[
  \varphi(\infty)=(1,0,0),\quad
  \varphi(\ii)=(0,1,0),\quad
  \varphi(-1)=(0,0,1).
\]
此时, 由\eqref{eq:rotation-quaternion} 定义的 $q$ 对应的 $\BR^3$ 上的旋转变换 $\Phi_q(\bfa)=q\bfa q^{-1}$ 和 $w=f(z)$ 诱导的 $S$ 上的旋转变换 $F=\varphi\circ f\circ\varphi^{-1}$ 是相同的.

\begin{exampleenum}
  \item 设 $q=\cos\dfrac\theta2+\ii\sin\dfrac\theta 2$.
  那么 $z_1=\ee^{\frac{\ii\theta}2}, z_2=0$, 它对应的分式线性映射为
  \[
    w=f(z)=\ee^{\ii\theta}z.
  \]
  我们有
  \begin{align*}
      F(1,0,0)&
    =\varphi\bigl(f(\infty)\bigr)
    =(1,0,0),\\
      F(0,1,0)&
    =\varphi\bigl(f(\ii)\bigr)
    =(0,\cos\theta,\sin\theta),\\
      F(0,0,1)&
    =\varphi\bigl(f(-1)\bigr)
    =(0,-\sin\theta,\cos\theta),
  \end{align*}
  即 $F$ 是绕 $x$ 轴逆时针旋转 $\theta$ 的变换.
  \item 设 $q=\cos\dfrac\theta2+\jj\sin\dfrac\theta 2$.
  那么 $z_1=\cos\dfrac\theta2, z_2=\sin\dfrac\theta2$, 它对应的分式线性映射为
  \[
    w=f(z)=\frac{\cos \frac\theta2 \cdot z-\sin\frac\theta2}{\sin\frac\theta2 \cdot z+\cos\frac\theta2}.
  \]
  我们有
  \begin{align*}
      F(1,0,0)&
    =\varphi\bigl(f(\infty)\bigr)
    =(\cos\theta,0,-\sin\theta),\\
      F(0,1,0)&
    =\varphi\bigl(f(\ii)\bigr)
    =(0,1,0),\\
      F(0,0,1)&
    =\varphi\bigl(f(-1)\bigr)
    =(\sin\theta,0,\cos\theta),
  \end{align*}
  即 $F$ 是绕 $y$ 轴逆时针旋转 $\theta$ 的变换.
  \item 设 $q=\cos\dfrac\theta2+\kk\sin\dfrac\theta 2$.
  那么 $z_1=\cos\dfrac\theta2, z_2=\ii\sin\dfrac\theta2$, 它对应的分式线性映射为
  \[
    w=f(z)=\frac{\cos \frac\theta2 \cdot z-\ii \sin\frac\theta2}{-\ii\sin\frac\theta2 \cdot z+\cos\frac\theta2}.
  \]
  我们有
  \begin{align*}
      F(1,0,0)&
    =\varphi\bigl(f(\infty)\bigr)
    =(\cos\theta,\sin\theta,0),\\
      F(0,1,0)&
    =\varphi\bigl(f(\ii)\bigr)
    =(-\sin\theta,\cos\theta,0),\\
      F(0,0,1)&
    =\varphi\bigl(f(-1)\bigr)
    =(0,0,1),
  \end{align*}
  即 $F$ 是绕 $z$ 轴逆时针旋转 $\theta$ 的变换.
\end{exampleenum}

由于任一旋转均可分解为上述三类旋转的复合, 任一模为 $1$ 的四元数也可以分解为上述三类四元数的乘积, 因此所有酉矩阵对应的分式线性映射就对应 $S$ 的所有旋转.
设 $\abs{z_1}^2+\abs{z_2}^2=1$.
我们将上述讨论的对应总结成\ref{tab:fractal-transform-display}.

\begin{table}[H]
  \centering
  \tabcolsep=3mm
  \begin{tabular}{ccc}
    \topcolorrule
      \bf 表达形式&
      \bf 作用的集合&
      \bf 表现\\
    \topcolorrule
      $w=f(z)=\dfrac{z_1z-z_2}{\ov{z_2}z+\ov{z_1}}$&
      扩充复平面 $\BC^*\ni z$&
      分式线性映射\\
    \midcolorrule
      $w=f(z)=\dfrac{z_1z-z_2}{\ov{z_2}z+\ov{z_1}}$&
      复球面 $S\ni (a,b,c)$&
      旋转\\
    \midcolorrule
      酉矩阵 $\bfA=\begin{pmatrix}
        z_1&-z_2\\
        \ov{z_2}&\ov{z_1}
      \end{pmatrix}$&
      $\BC^2$ 的所有一维子空间 $V_z$&
      线性变换\\
    \midcolorrule
      四元数 $q=z_1+z_2\jj$&
      三维向量空间 $\BR^3\ni c\ii+b\jj-a\kk$&
      旋转 $\Phi_q$\\
    \bottomcolorrule
  \end{tabular}
  \caption{分式线性映射的不同表现}
  \label{tab:fractal-transform-display}
\end{table}



\section{初等函数对应的映射}

本节中我们将研究在保形映射中常用的初等函数对应的映射.

\subsubsection{幂函数}

\begin{figure}[H]
  \centering
  \begin{tikzpicture}
    \draw[cstcurve,cstnra,third] (-1,0)-- node[above] {$w=z^n$} (1,0);
    \def\r{1.5}
    \begin{scope}[shift={({-2-\r},0)}]
      \def\a{10}
      \def\b{60}
      \draw[cstaxis] (-\r,0)--(\r,0);
      \coordinate (O);
      \coordinate (A) at ({\r*cos(\a)},{\r*sin(\a)});
      \coordinate (B) at ({\r*cos(\b)},{\r*sin(\b)});
      \fill[cstcurve,main,cstfille1] (A) arc(\a:\b:\r) --(O)--cycle;
      \draw[thick,cstra] pic [cstfill1,draw=main] {angle=A--O--B};
      \draw[cstcurve,main] (A)--(O)--(B);
      \draw[cstaxis] (0,-\r)--(0,\r);
    \end{scope}
    \begin{scope}[shift={({2+\r},0)}]
      \def\a{30}
      \def\b{180}
      \draw[cstaxis] (-\r,0)--(\r,0);
      \coordinate (O);
      \coordinate (A) at ({\r*cos(\a)},{\r*sin(\a)});
      \coordinate (B) at ({\r*cos(\b)},{\r*sin(\b)});
      \fill[cstcurve,fifth,cstfille5] (A) arc(\a:\b:\r) --(O)--cycle;
      \draw[thick,cstra] pic [cstfill5,draw=fifth] {angle=A--O--B};
      \draw[cstcurve,fifth] (A)--(O)--(B);
      \draw[cstaxis] (0,-\r)--(0,\r);
    \end{scope}
    \begin{scope}[shift={(0,-3.5)}]
      \draw[cstcurve,cstnra,third] (-1,0)-- node[above] {$w=z^n$} (1,0);
      \begin{scope}[shift={({-2-\r},0)}]
        \def\b{120}
        \draw[thick] (-\r,0)--(0,0);
        \coordinate (O);
        \coordinate (A) at (\r,0);
        \coordinate (B) at ({\r*cos(\b)},{\r*sin(\b)});
        \fill[cstcurve,main,cstfille1] (A) arc(0:\b:\r) --(O)--cycle;
        \draw[thick,cstra] pic [cstfill1,draw=main] {angle=A--O--B};
        \draw[cstcurve,main] (O)--(B);
        \draw[cstaxis,cstcurve,main] (0,0)--(\r,0);
        \draw[cstaxis] (0,-\r)--(0,\r);
      \end{scope}
      \begin{scope}[shift={({2+\r},0)}]
        \coordinate (O);
        \fill[cstcurve,fifth,cstfille5] (-\r,-\r) rectangle (\r,\r);
        \cutline{0}{0}{\r}{0}{fifth};
        \draw[cstaxis] (0,-\r)--(0,\r);
        \draw[cstaxis] (-\r,0)--(\r,0);
      \end{scope}
    \end{scope}
  \end{tikzpicture}
  \caption{幂函数在角形区域上的映射}
\end{figure}

设 $w=f(z)=z^n$, $n\ge2$ 为正整数.
这个函数是处处解析的, 导数为 $f'(z)=nz^{n-1}$.
因此在 $z\neq0$ 处它是保形映射.
对于 $z=0$, 不难看出该映射将圆周 $\abs{z}=r$ 映成圆周 $\abs{w}=r^n$, 将射线 $\Arg z=\theta$ 映成 $\Arg w=n\theta$.
因此它将角形区域 $\theta_1<\Arg z<\theta_2$ 映成角形区域 $n\theta_1<\Arg z<n\theta_2$, 且在这两个区域上是一一对应的, 其中 $0<\theta_2-\theta_1<\dfrac{2\cpi}n$.
所以 $w=z^n$ 在 $z=0$ 处没有保角性.

幂函数 $w=z^n$ 将角形区域 $0<\Arg z<\dfrac{2\cpi}n$ 映成去掉正实轴和零的复平面 $0<\Arg z<2\cpi$, 其中 $\Arg z=0$ 被映射为 $w$ 平面正实轴的``上沿岸'', $\Arg z=\dfrac{2\cpi}n$ 被映射为 $w$ 平面正实轴的``下沿岸''.
该映射在这两个区域上是一一对应的.

对于正实数幂次情形的主值, 我们也有类似的结论.

\begin{example}
  求将角形区域 $D:\dfrac\cpi4<\arg z<\dfrac{3\cpi}4$ 映成 $\BD$ 的映射 $w=f(z)$.
\end{example}

\begin{solution}
  由于 $D$ 的边界是两条射线, 不可能通过分式线性映射映成单位圆周, 所以不可能单纯使用分式线性映射将 $D$ 映成 $\BD$.
  我们考虑将 $D$ 的边界映成一条直线, 那么幂函数 $s=z^2$ 就可以做到这一点, 它将 $D$ 映成左半平面 $D_s:\Re z<0$.
  再由\thmref{例}{exam:imz-open-disk} 可知, 映射 $w=\dfrac{1+s}{1-s}$ 将 $D_s$ 映成 $\BD$.
  故
  \[
    w=\frac{1+z^2}{1-z^2}
  \]
  就是一个符合要求的映射.
\end{solution}

\begin{figure}[H]
  \centering
  \begin{tikzpicture}
    \draw[cstcurve,cstnra,third] (-1,0)-- node[above] {$w=f(z)$} (1,0);
    \draw[cstcurve,cstnra] (-3.5,-1)--node[below left] {$s=z^2$}(-2,-2.5);
    \draw[cstcurve,cstnra] (1.5,-2.5)--node[below right] {$w=\dfrac{1+s}{1-s}$}(3,-1);
    \def\w{1.5}
    \begin{scope}[shift={(-4,0)}]
      \def\a{45}
      \def\b{135}
      \draw[cstaxis] (-\w,0)--(\w,0);
      \coordinate (O);
      \coordinate (A) at ({\w*cos(\a)},{\w*sin(\a)});
      \coordinate (B) at ({\w*cos(\b)},{\w*sin(\b)});
      \fill[cstcurve,main,cstfille1] (A) arc(\a:\b:\w) --(O)--cycle;
      \draw ({\w*.25},{\w*.5}) node[main] {$D$};
      \draw[cstcurve,main] (A)--(O)--(B);
      \draw[cstaxis] (0,-\w)--(0,\w);
    \end{scope}
    \begin{scope}[shift={(4,0)}]
      \def\r{1}
      \filldraw[cstcurve,fifth,cstfill5] (0,0) circle (\r);
      \draw[cstaxis] (0,-\w)--(0,\w);
      \draw[cstaxis] (-\w,0)--(\w,0);
      \fill[cstdot,fifth] (0,0) circle;
      \draw (.5,.5) node[fifth] {$\BD$};
    \end{scope}
    \begin{scope}[shift={(0,-3)}]
      \fill[cstcurve,second,cstfille2] (0,\w) rectangle (-\w,-\w);
      \draw (-\w,-\w) node[second,above right] {$D_s$};
      \draw[cstaxis,cstcurve,second] (0,-\w)--(0,\w);
      \draw[cstaxis] (-\w,0)--(\w,0);
    \end{scope}
  \end{tikzpicture}
  \caption{角形区域到 $\BD$ 的映射}
\end{figure}

\begin{example}
  求将带割痕的右半平面 $D:\Re z>0, z\notin[0,1]$ 映成右半平面 $D':\Re w>0$ 的映射 $w=f(z)$.
\end{example}

\begin{figure}[H]
  \centering
  \begin{tikzpicture}
    \draw[cstcurve,cstnra,third] (-1,0)-- node[above] {$w=f(z)$} (1,0);
    \draw[cstcurve,cstnra] (-4.5,-1.5) to[bend right=30] node[left] {$s=z^2$} (-4.5,-3.5);
    \draw[cstcurve,cstnra] (-.8,-3.5)-- node[above] {$t=s-1$} (.8,-3.5);
    \draw[cstcurve,cstnra] (4.5,-3.5) to[bend right=30] node[right] {$w=\sqrt{t}$}(4.5,-1.5);
    \def\w{1.5}
    \begin{scope}[shift={(-4,0)}]
      \fill[cstcurve,main,cstfille1] (0,-\w) rectangle (\w,\w);
      \draw (\w,\w) node[main,below left] {$D$};
      \draw[cstaxis,cstcurve,main] (0,-\w)--(0,\w);
      \cutline{.6}{0}{.6}{180}{main};
      \draw[cstaxis] (-\w,0)--(\w,0);
    \end{scope}
    \begin{scope}[shift={(4,0)}]
      \fill[cstcurve,fifth,cstfille5] (0,-\w) rectangle (\w,\w);
      \draw (\w,\w) node[fifth,below left] {$D'$};
      \draw[cstaxis,cstcurve,fifth] (0,-\w)--(0,\w);
      \draw[cstaxis] (-\w,0)--(\w,0);
    \end{scope}
    \begin{scope}[shift={(-2.5,-3.5)}]
      \coordinate (A) at (0,\w);
      \coordinate (B) at (0,{-\w});
      \fill[cstfille2] (-\w,-\w) rectangle (\w,\w);
      \draw (\w,\w) node[second,below left] {$D_s$};
      \cutline{.6}{0}{\w+.6}{180}{second};
      \draw[cstaxis] (0,-\w)--(0,\w);
      \draw[cstaxis] (-\w,0)--(\w,0);
    \end{scope}
    \begin{scope}[shift={(2.5,-3.5)}]
      \def\w{1.5}
      \coordinate (A) at (0,{\w});
      \coordinate (B) at (0,{-\w});
      \fill[cstfille2] (-\w,-\w) rectangle (\w,\w);
      \draw (\w,\w) node[second,below left] {$D_t$};
      \cutline{0}{0}{\w}{180}{second};
      \draw[cstaxis] (0,-\w)--(0,\w);
      \draw[cstaxis] (-\w,0)--(\w,0);
    \end{scope}
  \end{tikzpicture}
  \caption{带割痕的右半平面到右半平面的映射}
\end{figure}

\begin{solution}
  处理割痕的核心想法是将割痕两侧与虚轴的夹角展平.
  首先 $s=z^2$ 将 $D$ 映成复平面带割痕 $z\le 1$ 的区域 $D_s$.
  然后 $t=s-1$ 将 $D_s$ 映成复平面带割痕 $z\le 0$ 的区域 $D_t$.
  最后 $w=\sqrt t$ (取主值) 将 $D_t$ 映成右半平面.
  所以
  \[
    w=\sqrt{z^2-1}
  \]
  就是一个符合要求的映射.
\end{solution}


\begin{exercise}
  通过哪些映射的复合可以将带正实轴割痕的单位圆域 $D$ 映成上半平面 $\BH$?
\end{exercise}

\subsection{指数函数}

设 $w=f(z)=\ee^z$ 为指数函数.
由于 $w'=\ee^z$ 处处非零, 因此它是整个复平面上的保形映射.
指数函数将直线族 $\Re z=c$ 映成圆周族 $\abs{w}=\ee^c$, 将直线族 $\Im z=c$ 映成射线族 $\Arg w=c$.
所以它将水平带状区域 $\theta_1<\Im z<\theta_2$ 映成角形区域 $\theta_1<\Arg w<\theta_2$, 其中 $0<\theta_2-\theta_1<2\cpi$.
特别地, 它将 $0<\Im z<2\cpi$ 映成去掉正实轴和零的复平面.

若我们想求反过来的映射, 则选取对数函数.

\begin{figure}[H]
  \centering
  \begin{tikzpicture}
    \draw[cstcurve,cstnra,third] (-1,0)-- node[above] {$w=\ee^z$} (1,0);
    \def\w{1.5}
    \begin{scope}[shift={({-2-\w},0)}]
      \def\a{.2}
      \def\b{.8}
      \draw[cstaxis] (-\w,0)--(\w,0);
      \fill[cstfille1] (-\w,\a) rectangle(\w,\b);
      \draw[cstcurve,main] (-\w,\a) rectangle(\w,\a);
      \draw[cstcurve,main] (-\w,\b) rectangle(\w,\b);
      \draw[cstaxis] (0,-\w)--(0,\w);
    \end{scope}
    \begin{scope}[shift={({2+\w},0)}]
      \def\a{36}
      \def\b{144}
      \draw[cstaxis] (-\w,0)--(\w,0);
      \coordinate (O);
      \coordinate (A) at ({\w*cos(\a)},{\w*sin(\a)});
      \coordinate (B) at ({\w*cos(\b)},{\w*sin(\b)});
      \fill[cstcurve,fifth,cstfille5] (A) arc(\a:\b:\w) --(O)--cycle;
      \draw[thick,cstra] pic [cstfill5,draw=fifth] {angle=A--O--B};
      \draw[cstcurve,fifth] (A)--(O)--(B);
      \draw[cstaxis] (0,-\w)--(0,\w);
    \end{scope}
    \begin{scope}[shift={(0,{-2-\w})}]
      \draw[cstcurve,cstnra,third] (-1,0)-- node[above] {$w=\ee^z$} (1,0);
      \begin{scope}[shift={({-2-\w},0)}]
        \def\a{-1}
        \def\b{1}
        \fill[cstfille1] (-\w,\a) rectangle(\w,\b);
        \draw[cstcurve,main] (-\w,\a) rectangle(\w,\a);
        \draw[cstcurve,main] (-\w,\b) rectangle(\w,\b);
        \draw[cstaxis] (0,-\w)--(0,\w);
        \draw[cstaxis] (-\w,0)--(\w,0);
      \end{scope}
      \begin{scope}[shift={({2+\w},0)}]
        \coordinate (O);
        \fill[cstcurve,fifth,cstfille5] (-\w,-\w) rectangle (\w,\w);
        \cutline{0}{0}{\w}{180}{fifth};
        \draw[cstaxis] (0,-\w)--(0,\w);
        \draw[cstaxis] (-\w,0)--(\w,0);
      \end{scope}
    \end{scope}
  \end{tikzpicture}
  \caption{指数函数在带状区域上的映射}
\end{figure}

\begin{example}
  求将竖直带状区域 $D:1<\Re z<2$ 映成单位圆域 $\BD$ 的映射 $w=f(z)$.
\end{example}

\begin{solution}
  我们先将竖直带状区域映成宽度为 $\cpi$ 的水平带状区域, 然后利用指数函数将其映成半平面, 最后再映成圆域.
  由于 $s=\cpi\ii z$ 将 $D$ 映成
  \[
    D_s:\cpi<\Im s<2\cpi,
  \]
  $t=-\ee^s$ 将 $D_s$ 映成上半平面 $\BH$, $w=\dfrac{t-\ii}{t+\ii}$ 将 $\BH$ 映成 $\BD$, 因此
  \[
    w=\frac{\ee^{\cpi\ii z}+\ii}{\ee^{\cpi\ii z}-\ii}
  \]
  将 $D$ 映成 $\BD$.
\end{solution}

\begin{figure}[H]
  \centering
  \begin{tikzpicture}
    \def\w{1.5}
    \draw[cstcurve,cstnra,third] (-1,0)-- node[above] {$w=f(z)$} (1,0);
    \draw[cstcurve,cstnra] (-5,-1.5) to[bend right=30] node[left] {$s=\cpi\ii z$}(-4,-3);
    \draw[cstcurve,cstnra] (-1,-4) to[bend right=30] node[below] {$t=-\ee^s$}(1,-4);
    \draw[cstcurve,cstnra] (4,-3) to[bend right=30] node[right] {$w=\dfrac{t-\ii}{t+\ii}$}(5,-1.5);
    \begin{scope}[shift={(-4,0)}]
      \def\a{.5}
      \def\b{1}
      \fill[cstfille1] (\a,-\w) rectangle(\b,\w);
      \draw[cstcurve,main] (\a,-\w) rectangle(\a,\w);
      \draw[cstcurve,main] (\b,-\w) rectangle(\b,\w);
      \draw[cstaxis] (-\w,0)--(\w,0);
      \draw[cstaxis] (0,-\w)--(0,\w);
      \draw ({(\a+\b)*.5},\w) node[main,below] {$D$};
    \end{scope}
    \begin{scope}[shift={(4,0)}]
      \def\r{1}
      \filldraw[cstcurve,fifth,cstfill5] circle (\r);
      \draw[cstaxis] (0,-\w)--(0,\w);
      \draw[cstaxis] (-\w,0)--(\w,0);
      \draw (.7,.7) node[fifth,below left] {$\BD$};
    \end{scope}
    \begin{scope}[shift={(-2,-3.5)}]
      \def\a{.65}
      \def\b{1.3}
      \fill[cstfille2] (-\w,\a) rectangle(\w,\b);
      \draw[cstcurve,second] (-\w,\a) rectangle(\w,\a);
      \draw[cstcurve,second] (-\w,\b) rectangle(\w,\b);
      \draw[cstaxis] (-\w,0)--(\w,0);
      \draw[cstaxis] (0,-\w)--(0,\w);
      \draw (\w,{(\a+\b)*.5}) node[second,left] {$D_s$};
    \end{scope}
    \begin{scope}[shift={(2,-3.5)}]
      \fill[cstcurve,second,cstfille2] (-\w,\w) rectangle (\w,0);
      \draw[cstaxis,cstcurve,second] (-\w,0)--(\w,0);
      \draw[cstaxis] (0,-\w)--(0,\w);
      \draw (\w,\w) node[second,below left] {$\BH$};
    \end{scope}
  \end{tikzpicture}
  \caption{竖直带状区域到 $\BD$ 的映射}
\end{figure}

\begin{example}
  求将带割痕的水平带状区域 $D:0<\Re z<2a, z-a\ii\notin(-\infty,b]$ 映成水平带状区域 $D':0<\Re z<2a$ 的映射 $w=f(z)$.
\end{example}

\begin{figure}[H]
  \centering
  \begin{tikzpicture}
    \def\w{1.5}
    \draw[cstcurve,cstnra,third] (-1,0)-- node[above] {$w=f(z)$} (1,0);
    \draw[cstcurve,cstnra] (-5,-1) to[bend right=30] node[left] {$s=\ee^{\frac\cpi a z}$}(-5,-3);
    \draw[cstcurve,cstnra] (-1,-3.5)--node[above] {$t=s+\ee^{\frac{b\cpi}a}$}(1,-3.5);
    \draw[cstcurve,cstnra] (5,-3) to[bend right=30] node[right] {$w=\dfrac a\cpi\ln t$}(5,-1);
    \begin{scope}[shift={(-4,0)}]
      \def\b{1.2}
      \fill[cstfille1] (-\w,0) rectangle(\w,\b);
      \draw[cstcurve,main] (-\w,0) rectangle(\w,0);
      \draw[cstcurve,main] (-\w,\b) rectangle(\w,\b);
      \cutline{.3*\w}{.5*\b}{1.3*\w}{180}{main};
      \draw[cstaxis] (0,-\w)--(0,\w);
      \draw[cstaxis] (-\w,0)--(\w,0);
      \draw (\w,\b) node[main,below left] {$D$};
    \end{scope}
    \begin{scope}[shift={(4,0)}]
      \def\b{1.2}
      \fill[cstfille5] (-\w,0) rectangle(\w,\b);
      \draw[cstcurve,fifth] (-\w,\b)--(\w,\b);
      \draw[cstaxis,cstcurve,fifth] (-\w,0)--(\w,0);
      \draw[cstaxis] (0,-\w)--(0,\w);
      \draw (\w,\b) node[fifth,below left] {$D'$};
    \end{scope}
    \begin{scope}[shift={(-3,-3.5)}]
      \fill[cstfille2] (-\w,-\w) rectangle(\w,\w);
      \cutline{-.3*\w}{0}{1.3*\w}{0}{second};
      \draw[cstaxis] (-\w,0)--(\w,0);
      \draw[cstaxis] (0,-\w)--(0,\w);
      \draw (\w,\w) node[second,below left] {$D_s$};
    \end{scope}
    \begin{scope}[shift={(3,-3.5)}]
      \fill[cstfille2] (-\w,-\w) rectangle(\w,\w);
      \cutline{0}{0}{\w}{0}{second};
      \draw[cstaxis] (-\w,0)--(\w,0);
      \draw[cstaxis] (0,-\w)--(0,\w);
      \draw (\w,\w) node[second,below left] {$D_t$};
    \end{scope}
  \end{tikzpicture}
  \caption{带割痕的水平带状区域到水平带状区域的映射}
\end{figure}

\begin{solution}
  我们先通过指数映射将带割痕的水平带状区域映成带割痕的复平面, 其中割痕的像落在实轴上.
  然后将割痕移动为正实轴, 最后利用对数函数再映成水平带状区域.
  由于 $s=\ee^{\frac\cpi a z}$ 将 $D$ 映成
  \[
    D_s: s\notin[-\ee^{\frac{b\cpi}a},+\infty),
  \]
  $t=s+\ee^{\frac{b\cpi}a}$ 将 $D_s$ 映成
  \[
    D_t: t\notin[0,+\infty),
  \]
  $w=\dfrac a\cpi\ln t$ 将 $D_t$ 映成 $D'$, 因此
  \[
    w=\frac a\cpi\ln\bigl(\ee^{\frac\cpi a z}+\ee^{\frac{b\cpi}a}\bigr)
  \]
  将 $D$ 映成 $D'$.
\end{solution}


\subsection{儒可夫斯基函数}

称函数
\[
  w=f(z)=\frac12\Bigl(z+\frac{a^2}z\Bigr),\quad a>0,
\]
为\nouns{儒可夫斯基函数}\index{rukefusijihanshu@儒可夫斯基函数}\index{0fz12za2z@$f(z)=\dfrac12\Bigl(z+\dfrac{a^2}z\Bigr)\smallskip$}.
它在 $0$ 以外处处解析, $0$ 是它的一个一阶极点.
由于
\[
  f'(z)=\dfrac12\Bigl(1-\dfrac{a^2}{z^2}\Bigr),
\]
因此它在 $z\neq 0,\pm a$ 处是保形映射.

不难知道 $f(z_1)=f(z_2)$ 当且仅当 $z_1z_2=a^2$.
所以 $w=f(z)$ 将圆域 $\abs{z}<a$ 一一地映射到它的像, 且和 $\abs{z}>a$ 的像相同.
若 $z=a\ee^{\ii\theta}$, 则 $w=a\cos\theta\in[-a,a]$.
所以它将圆周 $\abs{z}=a$ 映成直线段 $[-a,a]$.
反过来, 对于 $w=a\cos\theta\in[-a,a]$, 根据上述讨论, 它的原像只能是 $a\ee^{\pm\ii\theta}$.
所以 $w=f(z)$ 将\alert{圆域 $\abs{z}<a$ 映成扩充复平面去掉割痕 $[-a,a]$}.

\begin{figure}[H]
  \centering
  \begin{tikzpicture}
    \def\a{.5}
    \def\b{1.5}
    \draw[cstcurve,cstnra,third] (-1,0)-- node[above] {$w=\ee^z$} (1,0);
    \begin{scope}[shift={(-4,0)}]
      \filldraw[cstcurve,main,fill=main!55] circle (\a*\b*\b*\b);
      \filldraw[cstcurve,main,fill=main!35] circle (\a*\b*\b);
      \filldraw[cstcurve,main,fill=main!15] circle (\a*\b);
      \filldraw[cstcurve,main,fill=white] circle (\a);
      \draw[cstaxis] (0,-2)--(0,2);
      \draw[cstaxis] (-2,0)--(2,0);
      \draw[thick,main,cstra] (0,0)--({\a*cos(30)},{\a*sin(30)});
    \end{scope}
    \begin{scope}[shift={(4,0)}]
      \coordinate (O);
      \filldraw[cstcurve,fifth,fill=fifth!55] circle ({.5*\b*\b*\b+.5/\b/\b/\b} and {.5*\b*\b*\b-.5/\b/\b/\b});
      \filldraw[cstcurve,fifth,fill=fifth!35] circle ({.5*\b*\b+.5/\b/\b} and {.5*\b*\b-.5/\b/\b});
      \filldraw[cstcurve,fifth,fill=fifth!15] circle ({.5*\b+.5/\b} and {.5*\b-.5/\b});
      \draw[cstaxis] (-2.3,0)--(2.3,0);
      \draw[cstaxis] (0,-2)--(0,2);
      \draw[cstcurve,fifth] (-1,0)--(1,0);
      \fill[cstdot,fifth] (1,0) circle;
      \fill[cstdot,fifth] (-1,0) circle;
    \end{scope}
  \end{tikzpicture}
  \caption{儒可夫斯基函数在圆环域上的映射}
\end{figure}

我们来看其它半径的圆 $\abs{z}=r\neq a$ 的像. 
设 $s=\dfrac{a^2}r$, $z=r\ee^{\ii\theta}$, 则
\[
  w=f(z)=\frac12(r\ee^{\ii\theta}+s\ee^{-\ii\theta})
  =\frac{r+s}2 \cos \theta+\frac{r-s}2\ii\sin\theta.
\]
这是椭圆的参数方程, 所以圆周 $\abs{z}=r\neq a$ 和 $\abs{z}=\dfrac{a^2}r$ 被映成一个椭圆.
对于 $\theta\in(0,\cpi)$, 当 $r<a$ 时, $r<s$, 从而 $w$ 在下半平面.
因此圆周 $\abs{z}=r<a$ 上半部分被映成椭圆的下半部分.
所以 $w=f(z)$ 将\alert{圆域 $\abs{z}<a$ 的下半部分或圆域 $\abs{z}>a$ 的上半部分映成上半平面 $\BH$}.

\begin{example}
  求上半圆域 $D: \abs{z}<1,\Im z>0$ 映成单位圆域 $\BD$ 的映射 $w=f(z)$.
\end{example}

\begin{figure}[H]
  \centering
  \begin{tikzpicture}
    \def\w{1.5}
    \draw[cstcurve,cstnra,third] (-1,0)-- node[above] {$w=f(z)$} (1,0);
    \draw[cstcurve,cstnra] (-3,-1)--node[below left] {$s=-\dfrac12\Bigl(z+\dfrac1z\Bigr)$}(-2,-2);
    \draw[cstcurve,cstnra] (2,-2)--node[below right] {$w=\dfrac{s-\ii}{s+\ii}$}(3,-1);
    \begin{scope}[shift={(-4,0)}]
      \draw[cstaxis] (-\w,0)--(\w,0);
      \filldraw[cstcurve,main,cstfill1] (1,0) arc(0:180:1) --cycle;
      \draw[cstaxis] (0,-\w)--(0,\w);
      \draw (.7,.7) node[main,below left] {$D$};
    \end{scope}
    \begin{scope}[shift={(4,0)}]
      \filldraw[cstcurve,fifth,cstfill5] (0,0) circle (1);
      \draw[cstaxis] (-\w,0)--(\w,0);
      \draw[cstaxis] (0,-\w)--(0,\w);
      \fill[cstdot,fourth] (0,0) circle;
      \draw (.7,.7) node[fifth,below left] {$\BD$};
    \end{scope}
    \begin{scope}[shift={(0,-2.5)}]
      \fill[cstcurve,second,cstfille2] (-\w,\w) rectangle (\w,0);
      \draw[cstaxis,cstcurve,second] (-\w,0)--(\w,0);
      \draw[cstaxis] (0,-\w)--(0,\w);
      \draw (\w,\w) node[second,below left] {$\BH$};
    \end{scope}
  \end{tikzpicture}
  \caption{上半圆域到 $\BD$ 的映射}
\end{figure}

\begin{solution}
  注意到
  \[
    s=-\frac12\Bigl(z+\frac1z\Bigr)
  \]
  将 $D$ 映成 $\BH$, 分式线性映射
  \[
    w=\frac{s-\ii}{s+\ii}
  \]
  将 $\BH$ 映成 $\BD$.
  因此
  \[
    w=\frac{s-\ii}{s+\ii}
    =\frac{z^2+2\ii z+1}{z^2-2\ii z+1}
  \]
  将 $D$ 映成 $\BD$.
\end{solution}

\begin{example}
  求上将半平面去掉半圆 $\abs{z}\le 1,\Im z>0$ 和射线 $x=0,y\ge 2$ 的区域 $D$ 映成上半平面 $\BH$ 的映射 $w=f(z)$.
\end{example}

\begin{figure}[H]
  \centering
  \begin{tikzpicture}
    \def\w{1.5}
    \def\r{.5}
    \draw[cstcurve,cstnra,third] (-1,0)-- node[above] {$w=f(z)$} (1,0);
    \draw[cstcurve,cstnra] (-3.5,-.7)--node[left,shift={(0,.1)}] {$s=\dfrac12\Bigl(z+\dfrac1z\Bigr)$}(-4.5,-2.3);
    \draw[cstcurve,cstnra] (-3.7,-4)--node[below,shift={(-.1,0)}] {$t=s^2$}(-1.7,-4);
    \draw[cstcurve,cstnra] (1.7,-4)--node[below] {$p=1+\dfrac9{16t}$}(3.7,-4);
    \draw[cstcurve,cstnra] (4.5,-2.3)--node[right] {$w=\sqrt p$}(3.5,-.7);
    \begin{scope}[shift={(-3,0)}]
      \fill[cstfille1] (-\w,0)--
        (-\r,0) arc (180:0:\r)--
        (\w,0)--(\w,\w)--(-\w,\w)--cycle;
      \draw[cstaxis] (-\w,0)--(\w,0);
      \draw[cstcurve,main] (-\w,0)--(-\r,0) arc (180:0:\r)--(\w,0);
      \draw[cstaxis,cstcurve,main] (\r,0)--(\w,0);
      \cutline{0}{2*\r}{\w-2*\r}{90}{main};
      \draw[cstaxis] (0,-\w)--(0,\w);
      \draw (\w,\w) node[main,below left] {$D$};
    \end{scope}
    \begin{scope}[shift={(-5.4,-4)}]
      \fill[cstfille2] (-\w,0) rectangle (\w,\w);
      \cutline{0}{.75*\r}{\w-.75*\r}{90}{second};
      \draw[cstaxis] (0,-\w)--(0,\w);
      \draw[cstaxis,cstcurve,second] (-\w,0)--(\w,0);
      \draw (\w,\w) node[second,below left] {$D_s$};
    \end{scope}
    \begin{scope}[shift={(0,-4)}]
      \fill[cstfille2] (-\w,-\w) rectangle (\w,\w);
      \cutline{0}{0}{\w}{0}{second};
      \cutline{-9/16*\w}{0}{7/16*\w}{180}{second};
      \draw[cstaxis] (-\w,0)--(\w,0);
      \draw[cstaxis] (0,-\w)--(0,\w);
      \draw (\w,\w) node[second,below left] {$D_t$};
    \end{scope}
    \begin{scope}[shift={(5.4,-4)}]
      \fill[cstfille2] (-\w,-\w) rectangle (\w,\w);
      \cutline{0}{0}{\w}{0}{second};
      \draw[cstaxis] (-\w,0)--(\w,0);
      \draw[cstaxis] (0,-\w)--(0,\w);
      \draw (\w,\w) node[second,below left] {$D_p$};
    \end{scope}
    \begin{scope}[shift={(3,0)}]
      \fill[cstfille5] (-\w,0) rectangle (\w,\w);
      \draw[cstaxis] (-\w,0)--(\w,0);
      \draw[cstcurve,fifth] (-\w,0)--(\w,0);
      \draw[cstaxis] (0,-\w)--(0,\w);
      \draw (\w,\w) node[fifth,below left] {$\BH$};
    \end{scope}
  \end{tikzpicture}
  \caption{去掉半圆的带割痕上半平面到 $\BH$ 的映射}
\end{figure}

\begin{solutionenum}
  \item 儒可夫斯基映射
  \[
    s=\frac12\Bigl(z+\frac1z\Bigr)
  \]
  将上半单位圆周映成线段 $[-1,1]$, 将 $\BH$ 在单位圆外部的区域映成 $\BH$, 所以 $D$ 被映成带割痕的上半平面
  \[
    D_s: z\in\BH, z\notin\bigsetm{z}{\Re z=0,\Im z\ge \dfrac34}.
  \]
  \item 映射
  \[
    t=s^2
  \]
  将 $D_s$ 映成带两条割痕的复平面
  \[
    D_t: z\notin (-\infty,-\dfrac9{16}]\cup[0,+\infty).
  \]
  \item 由于两条射线交于 $\infty$, 通过分式线性映射可以把它们映成一条射线.
  取
  \begin{alignat*}{3}
    z_1&=-\dfrac9{16},\quad&
    z_2&=0,&
    z_3&=\infty,\\
    w_1&=0,&
    w_2&=\infty,\quad&
    w_3&=1,
  \end{alignat*}
  则对应分式线性映射为
  \[
    p=\frac{t+\dfrac9{16}}t=1+\frac9{16t},
  \]
  它将 $D_t$ 映成 $D_p:z\notin[0,+\infty)$.
  \item 最后, $w=\sqrt p$ 将 $D_p$ 映成 $\BH$.
  故
  \[
    w=\sqrt{1+\frac9{4(z+1/z)^2}}
    =\frac{\sqrt{4z^4+17z^2+4}}{2(z^2+1)}.
  \]
\end{solutionenum}

此外, 儒可夫斯基映射还可以与指数函数复合得到正弦、余弦、双曲正弦、双曲余弦等映射.

\begin{figure}[H]
  \centering
  \begin{tikzpicture}
    \def\w{1.5}
    \def\u{.8}
    \draw[cstcurve,cstnra,third] (-1,0)-- node[above=1mm] {$w=\ch z$} (1,0);
    \begin{scope}[shift={(-3,0)}]
      \fill[cstfille1] (0,0) rectangle (\w,\u);
      \draw[cstaxis] (0,-\w)--(0,\w);
      \draw[cstcurve,main] (\w,0)--(0,0)--(0,\u)--(\w,\u);
      \draw[thick] (-\w,0)--(0,0);
      \draw[cstaxis,cstcurve,main] (0,0)--(\w,0);
      \draw (\w,\u) node[main,below left] {$D$};
      \draw (0,1) node[left] {$\cpi$};
    \end{scope}
    \begin{scope}[shift={(3,0)}]
      \fill[cstfille2] (-\w,0) rectangle (\w,\w);
      \draw[cstaxis] (0,-\w)--(0,\w);
      \draw[cstaxis,cstcurve,second] (-\w,0)--(\w,0);
      \draw (\w,\w) node[second,below left] {$\BH$};
    \end{scope}
  \end{tikzpicture}
  \caption{半个带状区域到 $\BH$ 的映射}
\end{figure}

\begin{exercise}
  正弦函数 $w=\sin z$ 将半个带状区域 $D:-\dfrac\cpi2<\Re z<\dfrac\cpi 2, \Im z<0$ 映成什么?
\end{exercise}



\section{保形映射在标量场的应用 \optional}

用保形映射可以求出很多平面场的分布.
我们来看几个例子.

\begin{example}
  两块半无穷大的金属板连成一块无穷大的板, 连接处绝缘.
  设两部分的电势分别为 $v_1$ 和 $v_2$, 求金属板上的电势分布.
\end{example}

\begin{figure}[H]
  \centering
  \begin{tikzpicture}
    \draw[cstcurve,cstnra,third] (-1,0)--(1,0)
      node[midway, above] {$w=f(z)$};
    \def\w{2}
    \def\d{.2}
    \begin{scope}[shift={(-4,0)}]
      \draw[decorate,decoration={brace,amplitude=5},cstfill5] (\w,0)--(0,0);
      \draw[decorate,decoration={brace,amplitude=5},cstfill2] (0,0)--(-\w,0);
      \fill[cstfille1] (-\w,0) rectangle (\w,\w);
      \draw[cstaxis] (0,-\w)--(0,\w);
      \draw[cstcurve,second] (-\w,0)--(0,0);
      \draw[cstaxis,cstcurve,fifth] (0,0)--(\w,0);
      \draw ({.5*\w},0) node[below=2mm,fifth] {$v_1$}
        ({-.5*\w},0) node[below=2mm,second] {$v_2$};
      \foreach \i in {2,4,6,8}{
        \draw[cstaxis,main] ({\d*\i},0) arc (0:180:{\d*\i});
      }
      \draw (\w,\w) node[main,below left] {$\BH$};
    \end{scope}
    \begin{scope}[shift={(4,0)}]
      \def\a{.6}
      \def\b{1.4}
      \fill[cstfill1] (-\w,\a) rectangle (\w,\b);
      \draw[cstcurve,fifth] (-\w,\a)--(\w,\a);
      \draw[cstcurve,second] (-\w,\b)--(\w,\b);
      \draw[cstaxis] (-2,0)--(2,0);
      \draw[cstaxis] (0,-2)--(0,2);
      \draw (-\w,\a) node[left,fifth] {$v_1$}
        (-\w,\b) node[left,second] {$v_2$};
      \foreach \i in {-7,-5,...,7}{
        \draw[cstaxis,main] ({\d*\i},\a)--({\d*\i},\b);
      }
      \draw (\w,{\a*.5+\b*.5}) node[main,left] {$D$};
    \end{scope}
  \end{tikzpicture}
  \caption{电流分布}
\end{figure}

\begin{solution}
  由于金属板是无限长的, 所以在垂直于金属板和连接线的平面上, 场的分布情况完全相同, 因此这个静电场是一个平面场.
  以金属板在该平面的投影为实轴, 连接处为原点建立坐标系.
  设 $v_1,v_2$ 分别为正半实轴和负半实轴的电势.

  若保形映射 $w=f(z)$ 将上半平面 $\BH$ 映射为水平带状区域 
  \[
    D:v_1<\Im w<v_2,
  \]
  且将正半实轴和负半实轴分别映成 $\Im w=v_1$ 和 $\Im w=v_2$, 则从 $D$ 上的电势分布 $v=\Im w$ 就可以得到 $\BH$ 上的电势分布.
  映射
  \[
    s=\ln z
  \]
  将 $\BH$ 映成水平带状区域 $0<\Im s<\cpi$, 然后映射
  \[
    w=\dfrac{v_2-v_1}\cpi s+v_1
  \]
  将其映成水平带状区域 $v_1<\Im w<v_2$.
  所以
  \[
    w=\dfrac{v_2-v_1}\cpi \ln z+\ii v_1,
  \]
  $\BH$ 的电势分布为
  \[
    v=\Im\Bigl(\dfrac{v_2-v_1}\cpi \ln z+\ii v_1\Bigr)
    =\dfrac{v_2-v_1}\cpi \arg z+v_1.
  \]
  由对称性可知金属板外的电势分布为
  \[
    v=\dfrac{v_2-v_1}\cpi \abs{\arg z}+v_1.
  \]
\end{solution}

\begin{example}
  有一个圆形薄金属板, 上下用两个热绝缘材料完全包裹.
  若金属板边界上两个半圆周上的温度分别为 $T_1,T_2$, 求金属板上的温度分布.
\end{example}

\begin{figure}[H]
  \centering
  \begin{tikzpicture}
    \draw[cstcurve,cstnra,third] (-1,0)-- node[above] {$w=f(z)$} (1,0);
    \def\gd{5}
    \def\w{1.7}
    \begin{scope}[shift={(-4,0)}]
      \def\r{1.6}
      \fill[cstcurve,cstfill1] circle (\r);
      \draw[cstcurve,second] (\r,0) arc (0:-180:\r);
      \draw[cstcurve,fifth] (\r,0) arc (0:180:\r);
      \draw ({\r*.707},{\r*.707}) node[main,above right] {$\BD$};
      \draw[thick] (-2,0)--(-\r,0);
      \draw[cstaxis] (\r,0)--(2,0);
      \draw[cstaxis] (0,-2)--(0,2);
      \foreach \i in {22.5,45,67.5}{
        \draw[cstdash, main] (-\r,0) arc ({90+\i}:{90-\i}:{\r/sin(\i)});
        \draw[cstdash, main] (-\r,0) arc ({-90-\i}:{-90+\i}:{\r/sin(\i)});
      }
      \draw[cstdash, main] (-\r,0)--(\r,0);
      \draw
        node[above left=\r,fifth] {$T_1$}
        node[below left=\r,second] {$T_2$};
      \filldraw[cstdote] (\r,0) circle;
      \filldraw[cstdote] (-\r,0) circle;
    \end{scope}
    \begin{scope}[shift={(4,0)}]
      \def\a{.25}
      \def\b{1.4}
      \fill[cstfill1] (-\w,\a) rectangle (\w,\b);
      \draw[cstcurve,fifth] (-\w,\a)--(\w,\a);
      \draw[cstcurve,second] (-\w,\b)--(\w,\b);
      \draw[cstaxis] (-2,0)--(2,0);
      \draw[cstaxis] (0,-2)--(0,2);
      \draw (-\w,\a) node[fifth,left] {$T_1$}
        (-\w,\b) node[second,left] {$T_2$};
      \foreach \i in {1,2,...,7}{
        \draw[cstdash, main] (-\w,{\a+\i/8*(\b-\a)})--(\w,{\a+\i/8*(\b-\a)});
      }
      \draw (\w,\b) node[main,below right] {$D$};
    \end{scope}
  \end{tikzpicture}
  \caption{等温线}
\end{figure}

\begin{solution}
  由于金属板上下热绝缘, 因此热流被严格限制在金属板内, 这个温度分布是一个平面场.
  设金属板的半径为 $1$, 以金属板的圆心为原点建立直角坐标系, 使得上下半圆周的温度分别为 $T_1,T_2$.

  若保形映射 $w=f(z)$ 将单位圆域 $\BD$ 映射为水平带状区域 
  \[
    D:T_1<\Im w<T_2,
  \]
  且将上下半圆周分别映成 $\Im w=T_1$ 和 $\Im w=T_2$, 则从 $D$ 上的温度分布 $T=\Im w$ 就可以得到 $\BD$ 的温度分布.
  映射
  \[
    s=\ii\dfrac{1-z}{1+z}
  \]
  将 $\BD$ 映成 $\BH$, 然后映射
  \[
    w=\dfrac{T_2-T_1}\cpi \ln s+T_1
  \]
  将其映成水平带状区域 $D$.
  所以
  \[
    w=\dfrac{T_2-T_1}\cpi \ln \Bigl(\ii\frac{1-z}{1+z}\Bigr)+\ii T_1,
  \]
  $\BD$ 的温度分布为
  \[
    T=\Im\biggl(\dfrac{T_2-T_1}\cpi \ln \Bigl(\ii\frac{1-z}{1+z}\Bigr)+\ii T_1\biggr)
    =T_1+\frac{T_2-T_1}\cpi \arccot\frac{2y}{1-x^2-y^2}.
  \]
\end{solution}


\newpage
\startwidepage
\psection{本章小结}

本章所需掌握的知识点如下:
\begin{conclusion}
  \item 理解保形映射的定义.
  \begin{conclusion}
    \item 区域上的保形映射是一一的保角映射.
    \item 点 $z$ 处的保角映射是指经过 $z$ 的两条曲线的夹角和它们的像的夹角是相同的, 且角度的方向也相同; 而且 $z$ 附近的点具有相同的伸缩率.
    \item 区域上的保角映射就是导数非零的解析函数.
    \item 与 $\infty$ 有关的保角映射需要通过倒数映射将其转化为 $0$ 来讨论.
  \end{conclusion}
  \item 掌握分式线性映射的性质和构造方式.
  \begin{conclusion}
    \item 分式线性映射将圆周或直线 $C$ 映成圆周或直线 $C'$.
    \item 分式线性映射将关于 $C$ 的对称点映成关于 $C'$ 的对称点.
    \item 分式线性映射将圆域或半平面映成圆域或半平面.
    \item 平面上三个不同的点以及它们的像可以唯一确定一个分式线性映射:
    \[
       \frac{w-w_1}{w-w_2}\cdot\frac{w_3-w_2}{w_3-w_1}
      =\frac{z-z_1}{z-z_2}\cdot\frac{z_3-z_2}{z_3-z_1}.
    \]
    \item 构造分式线性映射选择三个点时, 先选择圆周或直线的交点, 或者关于圆周或直线的对称, 最后再选择其它边界点, 并根据区域所处位置确定三个点的顺序.
  \end{conclusion}
  \item 掌握幂函数、指数函数和儒可夫斯基函数所对应的映射特点.
  \begin{conclusion}
    \item 幂函数可以将角形区域映成角形区域, 并改变它的夹角大小.
    特别地, 半平面可以看作夹角为 $\cpi$ 的角形区域, 带射线割痕的复平面可以看作夹角为 $2\cpi$ 的角形区域.
    \item 指数函数可以将带状区域映成角形区域, 对数函数则是反过来.
    \item 儒可夫斯基函数可以将下半圆域映成 $\BH$, 也可以把上半平面挖去上半圆周映成 $\BH$.
  \end{conclusion}
  \item 会利用上述函数构造简单的单连通区域间的保形映射.
  \begin{conclusion}
    \item 单连通区域之间总存在共性映射, 但一般情形往往很难构造.
    \item 不同的单连通区域之间可以通过上半平面 $\BH$ 或单位圆域 $\BD$ 为中介来构造映射.
  \end{conclusion}
\end{conclusion}

本章中不易理解的概念和难点包括:
\begin{enuma}
  \item 保角映射和保形映射的差异: 保形映射是一一的保角映射, 这是为了保证区域间的映射可逆.
  \item 与 $\infty$ 相关的保形映射: 作变量替换 $t=\dfrac1z$ 或 $s=\dfrac1w$ 来将其转化为普通复数情形的保形映射.
  \item 利用\thmref{定理}{thm:three-points-determine-fractal-transform} 构造分式线性映射时, 需要根据区域在圆周内部还是外部、直线的哪一侧来确定三点的顺序.
  \item 带割痕的区域的处理, 需要使用幂函数将割痕与边界合并.
  \item 儒可夫斯基函数在处理带半圆边界的半平面中的应用.
\end{enuma}


\psection{本章作业}
\begin{homework}
  \item 单选题.
  \begin{homework}
    \item 保形映射总能保持区域内图形的\fillbrace{}.
    \begin{exchoice}(4)
      \item 面积
      \item 周长
      \item 夹角和方向
      \item 对称性
    \end{exchoice}
    \item 下列命题正确的是\fillbrace{}.
    \begin{exchoice}(1)
      \item 若 $w=f(z)$ 是区域 $D$ 内的解析函数, 且导数处处非零, 则它是 $D$ 上的保形映射.
      \item 若 $w=f(z)$ 是区域 $D$ 内的解析函数, 且导数处处非零, 则它是 $D$ 上的一一对应.
      \item 若 $w=f(z)$ 是区域 $D$ 内的解析函数, 且导数处处非零, 则它是 $D$ 上的保角映射.
      \item 若 $w=f(z)$ 在 $z_0$ 处解析且 $f'(z_0)\neq0$, 则它是 $z_0$ 处的保形映射.
    \end{exchoice}
    \item 分式线性映射总是\fillbrace{}.
    \begin{exchoice}(1)
      \item 将圆和直线都映射为圆
      \item 将圆和直线都映射为直线
      \item 将圆映射为圆或直线
      \item 将关于直线对称的点映射为关于直线对称的点
    \end{exchoice}
    \item 若\fillbrace{}, 且分式线性映射 $w=f(z)$ 将 $z_1,z_2,z_3$ 映成 $w_1,w_2,w_3$, 则它将单位圆映成上半平面.
    \begin{exchoice}(1)
      \item $z_1=1,z_2=\ii,z_3=-\ii$, $w_1=-1,w_2=0,w_3=\infty$
      \item $z_1=1,z_2=-\ii,z_3=-1$, $w_1=-1,w_2=0,w_3=1$
      \item $z_1=1,z_2=\ii,z_3=-1$, $w_1=1,w_2=0,w_3=\infty$
      \item $z_1=1,z_2=-1,z_3=-\ii$, $w_1=1,w_2=0,w_3=-1$
    \end{exchoice}
    \item \fillbrace{}可以将角形区域映成角形区域.
    \begin{exchoice}(2)
      \item 分式线性映射
      \item 幂函数
      \item 指数函数
      \item 儒可夫斯基函数
    \end{exchoice}
    \item 儒可夫斯基函数可以将以原点为圆心的圆周映成\fillbrace{}.
    \begin{exchoice}(4)
      \item 直线段
      \item 圆周
      \item 椭圆
      \item 直线段或椭圆
    \end{exchoice}
  \end{homework}
  \item 填空题.
  \begin{homework}
    \item 若分式线性映射 $w=f(z)$ 将上半平面映成单位圆域, 且 $f(\ii)=2$, 则 $f(-\ii)=$\fillblank[7mm]{}.
    \item 若分式线性映射 $w=f(z)$ 将单位圆域映成右半平面, 且 $f(0)=1$, 则 $f(\infty)=$\fillblank[7mm]{}.
    \item 若 $w=f(z)$ 将角形区域 $0<\arg z<\dfrac\cpi6$ 共形映射到上半平面, 则 $f(z)$ 可以是\fillblank{}.
    \item 若 $w=f(z)$ 将水平带状区域 $0<\Im z<\cpi$ 共形映射到上半平面, 则 $f(z)$ 可以是\fillblank{}.
  \end{homework}
  \item 计算题.
  \begin{homework}
    \item 求以下区域在相应映射下的像.
    \begin{subhomework}(2)
      \item $\Re z<0, w=(1-\ii)z+\ii$;
      \item $\Im z>0, w=z-1-\ii$;
      \item $0<\Re z<1, w=\dfrac1{z-1}$;
      \item $\abs{z-1}<1, w=\dfrac z{z+1}$;
      \item $\dfrac\cpi2<\arg z<\cpi, w=\sqrt z$ (主值);
      \item $1<\Re z<2, w=\ee^z$;
      \item $\abs{z}>1, \Im z>0, w=z+\dfrac 2z$;
      \item $\abs{z}<1, \Im z<0, w=z+\dfrac 2z$.
    \end{subhomework}
    \item 求将 $-1,\infty,\ii$ 分别映成下列各点的分式线性映射.
    \begin{subhomework}(4)
      \item $\ii,1,1+\ii$;
      \item $1,\ii,\infty$;
      \item $0,\infty,1$;
      \item $\infty,-1,\ii$.
    \end{subhomework}
    \item 求扩充复平面上 $\abs{z}>1$ 到单位圆域 $\BD$ 且满足
    \[
      w\Bigl(\frac12\Bigr)=0,\quad
      \arg w'\Bigl(\frac12\Bigr)=\cpi.
    \]
    的保形映射 $w=f(z)$.
    \item 求 $1<\abs{z}<2$ 到 $\abs{w}>1, \abs{w+1}<\dfrac{4\sqrt3}3$ 的保形映射 $w=f(z)$.
    \item 求将 $D$ 映成 $D'$ 的分式线性映射.
    \begin{subhomework}(2)
      \item $D=\BD, D': \abs{w-1}<1$;
      \item $D: \abs{z+\ii}>1, D': \Im z<1$;
      \item $D: \Re z>-1, D': \abs{z+1}<2$;
      \item $D: \Re z+\Im z<2, D'=\BD$.
    \end{subhomework}
    \item 求将 $D$ 映成上半平面的保形映射.
    \begin{subhomework}(2)
      \item $D: \Im z>1, \abs{z}<2$;
      \item $D: \abs{z}>2, \abs{z-\sqrt2}<\sqrt 2$;
      \item $D: \abs{z}<1, \dfrac\cpi4<\arg z<\dfrac \cpi2$;
      \smallskip
      \item $D:\abs{z}>2, 0<\Arg z<\dfrac{3\cpi}2$;
      \item $D: \abs{z}<1, z\notin[0,1]$;
      \item $D: \abs{z}<2, \abs{z-1}>1$;
      \item $D: \Re z>0, \abs{z}<1$;
      % \item $D: a<\Re z<b$;
      \item $D: \Re z>0, 0<\Im z<1$.
    \end{subhomework}
    \item 求将 $D$ 映成上半平面的保形映射.
    \begin{subhomework}(1)
      \item $D$ 为带有割痕 $x=0,0\le y\le a$ 的上半平面;
      \item $D$ 为带有割痕 $x=0,y\le -1$ 的单位圆周外部 $\abs{z}>1$;
      \item $D$ 为带有割痕 $\arg z=\dfrac\cpi4, \abs{z}<2$ 的角形区域 $0<\arg z<\dfrac\cpi2$;
    \end{subhomework}
    \item 求上半平面到自身的所有分式线性映射.
    \item \optionalex 设 $w=f(z)=\dfrac{az+b}{cz+d}$ 是分式线性映射, 且 $f\circ f\circ f$ 是恒等映射. 求 $a,b,c,d$ 所需满足的条件.
    % a=d 或 $ad-bc=(a+d)^2$
    \item \optionalex 有一个半圆形薄金属板, 上下用两个热绝缘材料完全包裹. 若在半圆周上的温度为 $T_1$, 边界直径上的温度为 $T_2$, 求金属板上的温度分布.
    \item \optionalex 一块无限大的 $\dfrac14$ 圆形金属板, 圆心处由绝缘体隔开, 两条垂直的无限长半径上的电势分别为 $v_1,v_2$, 求金属板上的电势分布.
  \end{homework}
\end{homework}
\finishwidepage
  



% 
\addcontentsline{toc}{chapter}{练习答案}
\chapter*{练习答案}
\setcounter{chapter}{0}
\stepcounter{chapter}\setcounter{exer}{0}

\exans $-4$.
\exans $1$.
\exans $-\ov z$.
\exans $(x_1x_2-y_1y_2)^2+(x_1y_2+x_2y_1)^2$ 或 $(x_1x_2+y_1y_2)^2+(x_1y_2-x_2y_1)^2$.
\exans $\dfrac{7-4i}5$.

\exans 仅当 $z$ 不是负实数和 $0$ 时它才成立.

\exans $z_1,\dots,z_n$ 中的非零元辐角相等.
\exans $\displaystyle z=2\sqrt3\left(\cos\frac{-\pi}3+i\sin\frac{-\pi}3\right)=2\sqrt3e^{-\frac{\pi i}3}$, 写成 $\dfrac{5\pi}3$ 也可以.

\exans $-2^{2022}$.
\exans $\pm\dfrac{\sqrt3+i}2,\pm i,\pm\dfrac{\sqrt3-i}2$.

\exans 双曲线 $x^2-y^2=\dfrac12$ 和双曲线 $xy=\dfrac14$.
\exans
\begin{enumerate}
	\item 上半平面对应的闭区域为 $\Im z\ge0$.
	\item 下半平面对应的闭区域为 $\Im z\le0$.
	\item 左半平面对应的闭区域为 $\Re z\le0$.
	\item 右半平面对应的闭区域为 $\Re z\ge0$.
	\item 竖直带状区域对应的闭区域为 $x_1\le\Re z\le x_2$.
	\item 水平带状区域对应的闭区域为 $y_1\le\Im z\le y_2$.
	\item 角状区域对应的闭区域为 $\alpha_1\le \arg z\le \alpha_2$ 以及原点. 如果 $\alpha_1=-\pi,\alpha_2=\pi$, 则为 $\BC$.
	\item 圆环域对应的闭区域为 $r\le|z|\le R$.
\end{enumerate}
\exans 整个复平面.

\exans
\begin{enumerate}
	\item $\Re z$ 的定义域为 $\BC$, 值域为 $\BR$.
	\item $\arg z$ 的定义域为 $\{z\in\BC\mid z\neq 0\}$, 值域为 $(-\pi,\pi]$.
	\item $|z|$ 的定义域为 $\BC$, 值域为 $\{x\in\BR\mid x\ge0\}$.
	\item 当 $n>0$ 时, $z^n$ 的定义域为 $\BC$, 值域为 $\BC$.
	当 $n<0$ 时, $z^n$ 的定义域为 $\{z\in\BC\mid z\neq 0\}$, 值域为 $\{z\in\BC\mid z\neq 0\}$.
	\item $\dfrac{z+1}{z^2+1}$ 的定义域为 $\{z\in\BC\mid z\neq \pm i\}$, 值域为 $\BC$.
\end{enumerate}


\stepcounter{chapter}\setcounter{exer}{0}


\exans 处处不可导.
\exans C. 因为邻域也是一个区域.
\exans A. 根据C-R方程可知对于A, $u_x(0)=2\neq v_y(0)=3$. 对于BD, 各个偏导数在 $0$ 处取值都是 $0$. C则是处处都可导.

\exans $\ln 2-\dfrac{2\pi i}3$.
\exans $\ln 3$.





% \appendix
% 
\chapter{同调代数初步}\label{chap:homological_algebra}




该附录包含了该课程所需要的同调代数方面的内容, 其中每一节应当安排在正文相同序号的章之前. 诱导模和导出函子可以安排在第三章之前.

\section{模}\label{sec:modules}
\subsection{模和模同态}\label{subsec:modules_and_homomorphisms}
设 $(M,+)$ 是交换群, 记 $\End(M)$ 为 $M$ 的自同态全体, $\Aut(M)$ 为 $M$ 的自同构全体, 则 $(\End(M),+,\circ)$ 在加法和复合意义下构成环, 它的单位群为 $\Aut(M)$.

\begin{definition}{模}{module}
设 $R$ 是(含幺)交换环, 称环同态 $\rho:R\to \End(M)$ 为 $R$ \noun{模}, 或简称 $M$ 是 $R$ 模\footnote{如果将 $\End(M)$ 上的乘法定义为 $fg=g\circ f$, 则这样的环同态被称为右 $R$ 模, 原来的环同态则被称为左 $R$ 模. 如果 $M$ 既是左模又是右模且 $(ra)s=r(as)$, 则称之为双模.}. 对于 $r\in R, a\in M$, 我们记 $ra$ 或 $r.a=\rho(r)(a)$.
\end{definition}

\begin{definition}{群模}{group module}
设 $G$ 是群, 称群同态 $\rho:G\to \Aut(M)$ 为 $G$ \noun{模}, 或简称 $M$ 是 $G$ 模\footnote{类似地我们有右 $G$ 模和双模.}. 对于 $s\in G, a\in M$, 我们记 $sa$ 或 $s.a=\rho(s)(a)$. 注意 $M$ 是 $G$ 模等价于 $M$ 是 $\BZ[G]$ 模.
\end{definition}

\begin{remark}
如果 $M$ 是一个乘法群, 我们通常将 $R$ 或 $G$ 的作用记为 $a^r$ 这种形式. 
\end{remark}

\begin{example}
(1) 交换群 $G$ 是自然的 $\End(G)$ 模.

(2) $\BZ$ 模就是交换群.

(3) 如果环 $A\subseteq B$, 则 $B$ 可视为自然 $A$ 模.

(4) 只有一个元素的群自然是 $R$ 模, 称之为零模.
\end{example}

\begin{definition}{模同态}{morphism of modules}
设 $M,N$ 为两个 $R$ 模. 如果群同态 $f:M\to N$ 满足 $f(ra)=rf(a),\forall r\in R$, 则称之为\noun{模同态}. 如果 $f$ 是群的单同态, 满同态, 同构, 则称之为模的\noun{单同态}, \noun{满同态}, \noun{同构}, 记为 $M\inj N, M\surj N, M\cong N$. 记 $\Hom_R(M,N)$ 为 $M$ 到 $N$ 的模同态全体.
\end{definition}

\begin{definition}{子模}{submodule}
如果 $N$ 是 $M$ 的子群, 且 $ra\in N,\forall r\in R,a\in N$, 则称 $N$ 是 $M$ 的\noun{子模}. 显然任意多个子模的交仍然是子模. $M$ 有限多个子模的元素之和也形成 $M$ 的子模. $M$ 中包含其子集 $S$ 最小的子模称为由 $S$ \noun{生成的子模}.

如果存在 $a\in M$ 使得 $M=Ra$, 即 $M$ 由 $\set{a}$ 生成, 称之为\noun{循环模}. 如果存在有限集 $S\subseteq M$ 使得 $S$ 生成 $M$, 则称之为\noun{有限生成模}.
\end{definition}

\begin{proposition}{}{}
设 $N$ 是 $M$ 的子模, $M/N=\set{x+N\mid x\in M}$ 为其商群. 定义 $r(a+N)=ra+N$, 则 $M/N$ 是 $R$ 模, 称为\noun{商模}.
\end{proposition}
\begin{proof}
易证.
\end{proof}

\begin{definition}{零化子}{annihilator}
对于 $a\in M$, 定义 
  \[\Ann(a)=\set{r\in R\mid ra=0},\]
  \[\Ann(M)=\set{r\in R\mid rM=0}\]
为 $a$ 和 $M$ 的\noun{零化子}, 则它们是 $R$ 的左理想. 如果 $\Ann(a)$ 非零, 称 $a$ 为\noun{扭元}. 如果 $M$ 所有元素都是扭元, 称之为\noun{扭模}.
\end{definition}

\begin{example}
(1) 群同态就是 $\BZ$ 模同态; 群同构就是 $\BZ$ 模同构.

(2) 有限生成 $\BZ$ 模就是有限生成交换群.

(3) 域 $F$ 上的模就是 $F$ 上的向量空间, 有限生成 $F$ 模就是有限维 $F$ 向量空间.

(4) 环 $R$ 的左理想是 $R$ 的子模.
\end{example}

\begin{exercise}
设 $A\subseteq B\subseteq C$ 是整环. 如果 $B$ 是有限生成 $A$ 模, $C$ 是有限生成 $B$ 模, 则 $C$ 是有限生成 $A$ 模.
\end{exercise}

\begin{proposition}{中山引理}{}
设 $R$ 是交换环, $\fa$ 为它的一个理想, 且 $\fa$ 是所有极大理想的子集.
如果有限生成 $R$ 模 $M$ 和它的子模 $N$ 满足 $M=N+\fa M$, 则 $M=N$. 特别地, 如果 $R$ 是局部环, $\fa$ 为其唯一极大理想时该命题成立. 特别地, 如果 $M=\fa M$, 则 $M=0$.
\end{proposition}
\begin{proof}
由于 $M/N=I\cdot M/N$, 因此我们不妨设 $N=0$. 设 $a_1,\dots,a_n$ 是 $M$ 的一组生成元, 则存在 $A\in M_n(\fa)$ 使得
  \[\begin{pmatrix}
    a_1\\\vdots\\ a_n
  \end{pmatrix}=A\begin{pmatrix}
    a_1\\\vdots\\ a_n.
  \end{pmatrix}\]
于是
  \[(I_n-A)^*(I_n-A)\begin{pmatrix}
    a_1\\\vdots\\ a_n
  \end{pmatrix}=\mathbf{0},\]
即 $\det(I_n-A) a_i=0$. 而 $\det(I_n-A)$ 展开后除了对角元以外都属于理想 $\fa$, 因此 $\det(I_n-A)=1+a,a\in\fa$. 如果 $1+a$ 不是单位, 则存在极大理想 $\fm$ 包含它, 从而 $1=(1+a)-a\in\fm$, 矛盾! 所以 $1+a\in R^\times$, $a_i=0,M=0$.
\end{proof}


\subsection{直和和自由模}
\label{subsec:direct_sum_and_free_modules}

\begin{definition}{直积和直和}{direct product and direct sum}
设 $M_i,i\in I$ 是一族 $R$ 模. 定义 $\prod\limits_{i\in I} M_i$, $\bigoplus\limits_{i\in I} M_i$ 为群的直积和直和, 则它们有自然的 $R$ 模结构, $R$ 通过在每个分量作用. 称之为模的\noun{直积}和\noun{直和}.
\end{definition}

\begin{definition}{自由模}{free module}
如果存在一族元素 $a_i \in M, i\in I$ 使得 $M=\bigoplus\limits_{i\in I} Ra_i$, 且 $Ra_i\cong R$, 则称之为\noun{自由模}. 换言之, $M\cong \bigoplus\limits_{i\in I} R$.
\end{definition}

\begin{proposition}{}{}
主理想整环 $R$上的有限生成模一定同构于
  \[M\cong R^{\oplus r}\oplus \bigoplus_i R/\fa_i,\]
其中 $\fa_i$ 是 $R$ 的非零理想.
\end{proposition}
\begin{proof}
略.
\end{proof}

\begin{definition}{秩}{rank}
称 $r=\rank M$ 为 $M$ 的秩.
\end{definition}

\begin{example}
设 $A,B$ 为 $R$ 模, 令 $A\otimes B$ 为形如 $a\otimes b, a\in A, b\in B$ 的对象生成的交换群, 其中 $ra\otimes b=a\otimes rb, \forall r\in R$. 换言之,
  \[A\otimes_R B=\pair{(a,b)\mid a\in A,b\in B}/\sim,\]
其中 $(ra,b)\sim (a,rb)$. $A\otimes B$ 可以自然地看成 $R$ 模, 称之为 $A$ 和 $B$ 的\noun{张量积}. 我们有
  \[A\otimes B\cong B\otimes A,\]
  \[(A\otimes B)\otimes C\cong A\otimes(B\otimes C)\cong A\otimes B\otimes C,\]
  \[(A\oplus B)\otimes C\cong (A\otimes C)\oplus(B\otimes C),\]
  \[A\otimes R\cong A.\]
\end{example}

\subsection{诱导模}

\begin{definition}{诱导模}{induced module}
如果 $H\leqslant G$ 是一个子群, 则对于任意 $H$ 模 $B$,
  \[A=\Ind_G^H B:=\BZ[G]\otimes_{\BZ[H]} B\] 
是一个 $G$ 模, 称为\noun{诱导模}. 这里 $\BZ[H]$ 在 $\BZ[G]$ 通过右乘 $h^{-1}$ 作用, $G$ 在 $A$ 通过左乘 $g$ 作用.

另一种看法是将诱导模看成全体函数 $f:G\to B$, 其中 $f(gh)=f(g)^h,\forall h\in H$. 然后 $G$ 的作用是 $f^\sigma(x)=f(\sigma^{-1} x)$. 当 $(G:H)$ 有限时这二者是同构的. 显然 $B=\BZ[H]\otimes_{\BZ[H]} B$ 是 $A$ 的一个 $H$ 子模, 且
  \[\Ind_G^H B=\bigoplus_{\sigma H\in G/H} B^\sigma\]
是 $B$ 模同构, 这里 $\sigma$ 取遍左陪集 $G/H$ 的一组代表元.
\end{definition}


\section{范畴}

\subsection{范畴与函子}\label{subsec:categories_and_functors}

\begin{definition}{范畴}{category}
\noun{范畴} $\cC$ 由如下三个要素构成:
\begin{itemize}
\item 一个类 $\Obj\cC$, 其中的元素 $A\in\Obj\cC$ (或简记为 $A\in\cC$) 被称为\noun{对象};
\item 对于任意对象 $A,B$, 存在集合 $\Hom(A,B)$, 其中的元素 $u$ 被称为 $A$ 到 $B$ 的\noun{态射}, 记为 $u:A\to B$; 不同的有序对 $(A,B)$ 对应的态射不同;
\item 对于任意对象 $A,B,C$, 存在映射
  \[\Hom(A,B)\times\Hom(B,C)\to \Hom(A,C).\]
称 $(v,u)$ 的像为二者的复合, 记为 $u\circ v$ 或 $uv$.
\end{itemize}
这些要素需要满足
\begin{itemize}
\item 结合律: 对于 $u:A\to B, v:B\to C, w:C\to D$, $w\circ(v\circ u)=(w\circ v)\circ u$;
\item 对于任意 $A\in\cC$, 存在 $\id_A\in\Hom(A,A)$ 使得对任意 $u:A\to B$, $u\circ\id_A=u$; 对任意 $v:B\to A$, $\id_A\circ v=v$.
\end{itemize}
\end{definition}

\begin{example}
(1) 范畴的对象并不要求是一个具体的集合, 态射也不要求是集合间的映射, 尽管从主流集合论出发包括自然数, 实数等均为视为集合. 设 $(I,\le)$ 是一个偏序集, 对于 $i,j\in I$, 当 $i\le j$ 时, $\Hom(i,j)$ 为单点集; 否则 $\Hom(n,m)$ 为空. 这样便构造了一个范畴. 例如 $(\BN^+,\le)$, $(\BN^+,\mid)$, 拓扑空间开集关于包含关系等, 都可以构成范畴.

(2) 范畴对象构成的一般是一个类而不是集合. 例如全体集合关于集合间的映射构成的范畴 $\cSets$ 的对象全体, 即全体集合, 就不是一个集合(为什么).

(3) 其它例子包括: 全体群关于群同态构成范畴 $\cGroups$; 全体交换群关于群同态构成范畴 $\cAb$; 全体环关于环同态构成范畴 $\cRings$; 环 $R$ 上全体模关于模同态构成范畴 $\cMod/R$; 域 $k$ 上全体线性空间关于线性映射构成范畴 $\cVect/k$ 等.
\end{example}

\begin{definition}{对偶范畴}{dual category}
设 $\cA$ 是一个范畴, 定义其\noun{对偶范畴} $\cA^\op$:
\begin{itemize}
\item $\Obj \cA^\op=\Obj \cA$;
\item $\Hom_{\cA^\op}(A,B)=\Hom_{\cA}(B,A)$.
\end{itemize}
\end{definition}

\begin{example}
设 $(I,\le)$ 是一个偏序集, 则 $(I,\ge)$ 也是一个偏序集, 它们对应的范畴构成对偶范畴.
\end{example}

\begin{definition}{函子}{functor}
范畴 $\cA$ 到范畴 $\cB$ 间的\nouns{(共变)函子}{函子!共变函子} $\CF$ 由如下要素构成:
\begin{itemize}
\item 对于任意 $A\in\cA$, 有 $\CF(A)\in\cB$;
\item 对于任意 $\cA$ 中态射 $u:A_1\to A_2$, 有 $\CF(u):\CF(A_1)\to\CF(A_2)$,
\end{itemize}
且满足
\begin{itemize}
\item $\CF(\id_A)=\id_{\CF(A)}$;
\item $\CF(u\circ v)=\CF(u)\circ \CF(v)$.
\end{itemize}
\end{definition}

\begin{definition}{反变函子}{contravariant functor}
范畴 $\cA$ 到范畴 $\cB$ 间的\nouns{反变函子}{函子!反变函子} $\CF$ 由如下要素构成:
\begin{itemize}
\item 对于任意 $A\in\cA$, 有 $\CF(A)\in\cB$;
\item 对于任意 $\cA$ 中态射 $u:A_1\to A_2$, 有 $\CF(u):\CF(A_2)\to\CF(A_1)$,
\end{itemize}
且满足
\begin{itemize}
\item $\CF(\id_A)=\id_{\CF(A)}$;
\item $\CF(u\circ v)=\CF(v)\circ \CF(u)$.
\end{itemize}
这等价于共变函子 $\CF:\cA^\op\to\cB$.
\end{definition}

\begin{example}
(1) $\id_\cA:\cA\to \cA$ 将范畴 $\cA$ 的所有对象和映射保持不变, 它显然是一个函子, 称为\noun{恒等函子}.

(2) 设 $k$ 是一个域对于任意集合 $S$, 定义 $\CF(S)$ 为以 $S$ 为基的 $k$ 上线性空间, 则 $\CF:\cSets\to\cVect/k$ 是一个函子. $\CF$ 在态射上怎么作用?

(3) 对于任意群 $G$, 定义 $\CF(G)$ 为其对应的集合, 则 $\CF:\cGroups\to\cSets$ 是一个函子, 称之为\noun{遗忘函子}. 同理我们有遗忘函子 $\cMod/R\to\cAb$ 等.

(4) 设 $\cA$ 是一个范畴, $A,M,N\in\cC$. 定义 $\Hom(A,-)(M)=\Hom(A,M)$, 则 $\Hom(A,-):\cA\to\cSets$ 是一个函子, 其中对于 $u:M\to N$, 
  \[\fct{\Hom(A,-)(u):\Hom(A,M)}{\Hom(A,N)}{v}{u\circ v.}\]

(5) 设 $\cA$ 是一个范畴, $A,M,N\in\cC$. 定义 $\Hom(-,A)(M)=\Hom(M,A)$, 则 $\Hom(-,A):\cA\to\cSets$ 是一个反变函子, 其中对于 $u:M\to N$, 
  \[\fct{\Hom(-,A)(u):\Hom(N,A)}{\Hom(M,A)}{v}{v\circ u.}\]
我们称 $\Hom(A,-),\Hom(-,A)$ 为 \nouns{$\Hom$ 函子}{Hom 函子@$\Hom$ 函子}.

(6) 设 $H$ 是 $G$ 的一个子群, 则 $\Ind_G^H:\cMod/H\to\cMod/G$ 和 $\Res_G^H:\cMod/G\to\cMod/H$ 是函子, 其中 $\Res_G^H(M)=M$. 它们互为\noun{伴随}, 即 
  \[\Hom_G(\Ind_G^H M,N)=\Hom_H(M,\Res_G^H N).\]

(7) 设 $G$ 是一个群, $G^\ab=G/[G,G]$ 为其极大阿贝尔商, 则 $(~)^\ab:\cGroups\to \cAb$ 是一个函子.
\end{example}

\begin{definition}{范畴的同构}{isomorphism of categories}
设 $u:A\to B$. 如果存在 $v:B\to A$ 使得 $v\circ u=\id_A,u\circ v=\id_B$, 则称 $u$ 是\noun{同构}.
\end{definition}

\begin{definition}{自然变换与范畴等价}{natural transformation and equivalence of categories}
设 $\CF,\CG:\cA\to \cB$ 是两个函子. 称 $f$ 为 $\CF$ 到 $\CG$ 的\noun{自然变换}, 如果对于任意 $A\in\cA$, 存在 $f_A:\CF(A)\to\CG(A)$, 且满足对任意态射 $u:A_1\to B_2$,
  \[\xymatrix{
\CF(A_1)\ar[r]^{\CF(u)}\ar[d]_{f_{A_1}}&\CF(A_2)\ar[d]^{f_{A_2}}\\
\CG(A_1)\ar[r]^{\CG(u)}&\CG(A_2)
}\]
交换. 特别地, 我们有自然变换 $\id_\CF:\CF\to \CF$, 其中 $(\id_\CF)_A=\id_{\CF(A)}$. 如果 $\cA$ 是一个\noun{小范畴}, 即 $\Obj \cA$ 是一个集合, 则 $\cA\to\cB$ 间的函子以及函子的自然变换构成范畴 $\cFunc(\cA,\cB)$. 对于反变函子, 我们也可以类似定义自然变换. 

如果存在自然变换 $f:\CF\to\CG,g:\CG\to \CF$ 使得 $g\circ f=\id_\CF$, $f\circ g=\id_\CG$, 则称 $\CF$ 和 $\CG$ \noun{同构}. 换言之, $\CF,\CG$ 在 $\cFunc(\cA,\cB)$ 中同构. 这也等价于对任意 $A\in\cA$, $f_A:\CF(A)\ra\CG(A)$ 是同构.

如果存在 $\CF:\cA\to \cB$, $\CG:\cB\to \cA$ 使得 $\CG\circ \CF$ 和 $\id_\cA$ 同构, $\CF\circ \CG$ 和 $\id_\cB$ 同构, 则称 $\CF,\CG$ 诱导了 $\cA$ 和 $\cB$ 的\noun{范畴等价}. 这不同于范畴同构, 后者是指范畴的对象的态射完全一一对应, 即 $\CF\circ\CG=\id_\cB,\CG\circ\CF=\id_\cA$. 但是范畴等价意味着两个范畴的对象在同构意义下是一一对应的, 特别地, 二者的\noun{骨架范畴}是同构的, 其中骨架范畴是指范畴的每个对象的同构等价类中只选取一个对象.
\end{definition}





\subsection{加性范畴}\label{subsec:additive_category}

范畴论中大量概念都是通过\noun{泛性质}来定义的.

\begin{definition}{始对象}{}
如果范畴 $\cA$ 中的对象 $I$ 满足:
\begin{itemize}
\item 对于任意对象 $A$, $\Hom(I,A)=\set{i_A}$ 是单点集;
\item 对于任意态射 $u:A\to B$, $u\circ i_A=i_B$,
\end{itemize}
则称 $I$ 为 $\cA$ 的\noun{始对象}.
\end{definition}

\begin{definition}{终对象}{}
如果范畴 $\cA$ 中的对象 $F$ 满足:
\begin{itemize}
\item 对于任意对象 $A$, $\Hom(A,F)=\set{j_A}$ 是单点集;
\item 对于任意态射 $u:A\to B$, $j_B\circ u=j_A$,
\end{itemize}
则称 $F$ 为 $\cA$ 的\noun{终对象}.
\end{definition}

\begin{definition}{零对象}{}
如果一个对象既是始对象也是终对象, 称之为\noun{零对象}, 通常记为 $0$, 并记 $\Hom(A,0)=\set{0},\Hom(0,A)=\set{0}$.
\end{definition}

\begin{proposition}{}{}
始对象在同构意义下是唯一的; 终对象在同构意义下是唯一的.
\end{proposition}
\begin{proof}
设 $I,I'$ 是始对象, 则 $\Hom(I,I')=\set{i_{I'}}, \Hom(I',I)=\set{i'_I}$, 因此 $i_{I'}\circ i'_I:I\to I$. 由于 $\Hom(I,I)=\set{\id_I}$, 因此 $i_{I'}\circ i'_I=\id_I$. 同理 $i'_I\circ i_{I'}=\id_{I'}$, 所以 $i_{I'}:I\to I'$ 是同构. 类似地, 终对象在同构意义下也是唯一的.
\end{proof}

设 $A_i,i\in I$ 是范畴 $\cA$ 中的一族对象.

\begin{definition}{直和}{direct sum}
如果对象 $A$ 以及一族态射 $\alpha_i:A_i\to A$, 满足对于任意对象 $M$ 和一族态射 $u_i:A_i\to M$, 存在唯一的 $v:A\to M$ 使得下图交换
  \[\xymatrix{
A_i\ar[d]_{\alpha_i}\ar[rd]^{\forall u_i}&\\
A\ar@{.>}_{\exists ! v}[r]&M
}\]
则称 $(A,\alpha_i)$ 为 $A_i$ 的\noun{直和}, 记为 $\oplus_i A_i$.
\end{definition}

\begin{definition}{直积}{direct product}
如果对象 $A$ 以及一族态射 $\beta_i:A\to A_i$, 满足对于任意对象 $M$ 和一族态射 $u_i:M\to A_i$, 存在唯一的 $v:M\to A$ 使得下图交换
  \[\xymatrix{
M\ar[rd]_{u_i}\ar@{.>}^{\exists ! v}[r]&A\ar[d]^{\beta_i}\\
&A_i
}\]
则称 $(A,\beta_i)$ 为 $A_i$ 的\noun{直积}, 记为 $\prod_i A_i$.
\end{definition}

\begin{proposition}{}{}
直和和直积是同构意义下唯一的.
\end{proposition}
\begin{proof}
易证.
\end{proof}

对于 $\cAb,\cMod/R$ 等范畴, 我们可以发现 $\Hom(A,B)$ 均构成交换群且有有限直和, 有限直积, 核, 像等概念. 由此出发, 我们可以定义加性范畴和阿贝尔范畴.

\begin{definition}{加性范畴}{additive category}
如果范畴 $\cC$ 满足
\begin{itemize}
\item 对于任意对象 $A,B,C$, $\Hom(A,B)$ 具有交换群结构, 且态射复合
  \[\Hom(A,B)\times\Hom(B,C)\to \Hom(A,C)\]
是双线性的;
\item 存在零对象 $0$;
\item 对于任意对象 $A,B$, 存在直和 $A\oplus B$ 和直积 $A\times B$,
\end{itemize}
我们称之为\noun{加性范畴}.
\end{definition}

\begin{proposition}{}{}
对于加性范畴的对象 $A,B$, 我们有同构 $A\oplus B\simto A\times B$.
\end{proposition}
\begin{proof}
考虑 $\id_A:A\to A,0:A\to B$, 存在 $(\id_A,0):A\to A\times B$. 同理存在 $(0,\id_B):B\to A\times B$. 因此存在态射 $i:A\oplus B\to A\times B$, 使得下图交换
  \[\xymatrix{
    A\ar[rd]^{(\id_A,0)}\ar[d]_{\alpha_A}&\\
    A\oplus B\ar[r]^i&A\times B\\
    B\ar[u]^{\alpha_B}\ar[ru]_{(0,\id_B)}
  }\]
容易验证 $\alpha_A\circ \beta_A+\alpha_B\circ\alpha_A:A\times B\to A\oplus B$ 是它的逆.
\end{proof}

\begin{definition}{加性函子}{additive functor}
如果加性范畴间的函子 $\CF:\cA\to \cB$ 满足
\begin{itemize}
\item $\CF(0)=0$;
\item 自然态射 $\CF(A_1)\oplus\CF(A_2)\to \CF(A_1\oplus A_2)$ 是同构,
\end{itemize}
称之为\noun{加性函子}. 这等价于对任意 $A,B$, $\CF:\Hom(A,B)\to\Hom\bigl(\CF(A),\CF(B)\bigr)$ 是群同态.
\end{definition}

\begin{example}
(1) 加性范畴的对偶仍然是加性的.

(2) $\cAb$ 是加性范畴, 其上的 $\Hom$ 函子是加性函子.
\end{example}


\subsection{阿贝尔范畴}\label{subsec:abelian_category}

设 $u:A\to B$ 是加性范畴 $\cA$ 上的一个态射. 

\begin{definition}{核}{kernel}如果对象 $C$ 和态射 $i:C\to B$ 满足对于任意对象 $M$ 和态射 $v: M\to A$, 若 $u\circ v=0$, 则存在唯一的态射 $w:M\to C$ 使得下图交换
  \[\xymatrix{
M\ar@{.>}[d]_{\exists! w}\ar[rd]_{\forall v} \ar[rrd]^0&&\\
C\ar[r]_i &A\ar[r]_u &B
}\]
则称 $(C,i)$ 为 $u$ 的\noun{核}, 记为 $\ker u$. 若 $\ker u=0$, 称 $u$ 为\noun{单态射}.
\end{definition}

\begin{definition}{余核}{cokernel}
如果对象 $D$ 和态射 $j:B\to D$ 满足对于任意对象 $M$ 和态射 $v: B\to M$, 若 $v\circ u=0$, 则存在唯一的态射 $w:D\to M$ 使得下图交换
  \[\xymatrix{
A\ar[r]^u \ar[rrd]_0&B\ar[rd]^{\forall v}\ar[r]^j&D\ar@{.>}[d]^{\exists! w}\\
&&M\\
}\]
则称 $(D,j)$ 为 $u$ 的\noun{余核}, 记为 $\coker u$. 若 $\coker u=0$, 称 $u$ 为\noun{满态射}.
\end{definition}

\begin{definition}{像和余像}{image and coimage}
称余核的核 $\ker(\coker u)$ 为 $u$ 的\noun{像} $\im u$;
称核的余核 $\coker(\ker u)$ 为 $u$ 的\noun{余像} $\coim u$.
\end{definition}

我们将它们对应的对象记为 $\Ker,\Coker,\Im,\CoIm$.

\begin{definition}{阿贝尔范畴}{abelian category}
如果加性范畴 $\cA$ 满足
\begin{itemize}
\item 任意态射均有核和余核;
\item 对于任意态射 $u:A\to B$, 自然映射 $\CoIm u\to \Im u$ 是同构,
\end{itemize}
则称 $\cA$ 为\noun{阿贝尔范畴}. 这等价于既满又单的态射是同构.
\end{definition}

\begin{example}
(1) 阿贝尔范畴的对偶仍然是阿贝尔的.

(2) $\cAb$, $\cMod/R$ 是阿贝尔范畴.

(3) (Mitchell 嵌入定理) 任何一个小阿贝尔范畴 $\cA$ 可正合嵌入为一个模范畴 $\cMod/R$ 的全子范畴, 即存在函子 $\CF:\cA\to \cMod/R$, 使得 $\CF$ 诱导了 $\Obj \cA\inj \Obj\cMod/R$, $\Hom_\cA(A,B)=\Hom_{\cMod/R}\bigl(\CF(A),\CF(B)\bigr),$ 且保持核和余核.
\end{example}

\subsection{正合列}\label{subsec:exact_sequence}

\begin{definition}{正合}{exact}
设 $\cA$ 为阿贝尔范畴, $A,B,C\in \cA$. 称 $A\sto{u} B\sto{v} C$ \noun{正合}, 如果自然映射 $\Ker v\simeq \Im u$ 是同构. 由于它们都可以看成是 $B$ 的子对象 (存在到 $B$ 的单态射), 此时 $\Ker v=\Im u$.
由此可知
  \[0\ra A \sto{u}B\sto{v}C\ra 0\]
正合当且仅当 $\Ker u=0, \Im u=\Ker v,\Im v=C$, 这样的序列被称为\noun{短正合列}.
\end{definition}

\begin{proposition}{蛇形引理}{}
考虑阿贝尔范畴 $\cA$ 中的交换图
  \[\xymatrix{
    &A\ar[r]\ar[d]^{\alpha}&B\ar[r]\ar[d]^\beta&C\ar[r]\ar[d]^\gamma&0\\
    0\ar[r]&A'\ar[r]&B\ar[r]&C
  }\]
其中每行都是正合的, 则存在唯一的态射
  \[\delta:\Ker \gamma\to \Coker\alpha\]
使得下图交换
  \[\xymatrix{
    B\times_C\Ker\gamma\ar[r]\ar[d]&\Ker \gamma\ar[d]^\delta\\
    A'\ar[r]&\Coker \alpha
  }\]
其中左竖直态射由 $\beta$ 诱导, 而且我们有正合列
  \[\Ker\alpha\to\Ker\beta\to\Ker\gamma\sto{\delta}\Coker\alpha\to\Coker\beta\to\Coker\gamma.\]
\end{proposition}

对于模范畴情形, 我们可以直接验证.

\begin{corollary}{五引理}{}
考虑交换图表
  \[\xymatrix{
    A^1\ar[r]\ar[d]^{u^1}&A^2\ar[r]\ar[d]^{u^2}&
    A^3\ar[r]\ar[d]^{u^3}&A^4\ar[r]\ar[d]^{u^4}&
    A^5\ar[d]^{u^5}\\
    B^1\ar[r]&B^2\ar[r]&B^3\ar[r]&B^4\ar[r]&B^5,
  }\]
其中每行都正合. 如果 $u^1,u^2,u^4,u^5$ 是同构, 则 $u^3$ 也是同构.
\end{corollary}


\subsection{正向极限和逆向极限}

\begin{definition}{正向极限}{direct limit}
设 $I$ 是一个偏序集.
对于范畴 $\cA$ 中的一族对象 $A_i,i\in I$, 以及 $i
\le j$ 时态射 $\alpha_{ij}:A_i\to A_j$, 如果对象 $A$ 以及一族态射 $\alpha_i:A_i\to A$, 满足对任意 $i
le j$, $\alpha_j\circ\alpha_{ij}=\alpha_i$, 以及对于任意对象 $M$ 和一族态射 $u_i:A_i\to M$, 如果 $u_j\circ \alpha_{ij}=u_i$, 存在唯一的 $v:A\to M$ 使得下图交换
  \[\xymatrix{
A_i\ar[r]^{\alpha_{ij}}\ar[rd]_{\alpha_i}\ar@/_/[rdd]_{\forall u_i}&A_j\ar[d]_{\alpha_j}\ar@/^/[dd]^{u_j}\\
&A\ar@{.>}_{\exists ! v}[d]\\
&M
}\]
则称 $(A,\alpha_i)$ 为 $A_i$ 的\noun{正向极限}, 记为 $\ilim_i A_i$.
\end{definition}

\begin{definition}{逆向极限}{inverse limit}
对于范畴 $\cA$ 中的一族对象 $A_i,i\in I$, 以及 $i
\le j$ 时态射 $\alpha_{ij}:A_i\to A_j$, 如果对象 $A$ 以及一族态射 $\alpha_i:A_i\to A$, 满足对任意 $i
le j$, $\alpha_j\circ\alpha_{ij}=\alpha_i$, 以及对于任意对象 $M$ 和一族态射 $u_i:A_i\to M$, 如果 $u_j\circ \alpha_{ij}=u_i$, 存在唯一的 $v:A\to M$ 使得下图交换
  \[\xymatrix{
M\ar@/_/[dd]_{\forall u_i}\ar@/^/[rdd]^{\forall u_j}\ar@{.>}[d]^{\exists ! v}&\\
A\ar[d]^{\alpha_i}\ar[rd]^{\alpha_j}&\\
A_i\ar[r]_{\alpha_{ij}}&A_j
}\]
则称 $(A,\alpha_i)$ 为 $A_i$ 的\noun{逆向极限}, 记为 $\plim_i A_i$.
\end{definition}

\begin{proposition}{}{}
正向极限和逆向极限是同构意义下唯一的.
\end{proposition}
\begin{proof}
易证.
\end{proof}

我们考虑模范畴情形. 对于正向系 $A_i,i\in I$, 设 $A=\ilim_i A_i$, 则存在映射 $u:\bigoplus_i A_i\to A$. 考虑 $\bigoplus_i A_i$ 中由 $a_j-u_{ij}(a_i)$ 生成的子模 $M$, 则
  \[\ilim_i A_i=\frac{\bigoplus_i A_i}{M},\]
所以正向极限是直和的商模. 同理, 设 $B=\plim_i A_i$, 则存在映射 $u:B\to \prod_i A_i$. 考虑 $\prod_i A_i$ 中由满足 $a_j=u_{ij}(a_i)$ 的元素 $(a_i)_i$ 全体 $N$, 则 $N$ 是 $\prod_i A_i$ 的子模, 它就是 $\plim_i A_i$.

\subsection{复形}
设 $\cA$ 是一个阿贝尔范畴. $\cA$ 上的\noun{复形} $L=L^\bullet$ 是指一族对象 $L^i,i\in\BZ$, 以及态射 $d=d^i:L^i\ra L^{i+1}$, 使得 $d\circ d=0$. 我们记为
  \[L=(\cdots\ra L^i\ra L^{i+1}\ra\cdots).\]
其中 $d$ 被称为 $L$ 的\noun{微分}, $L^i$ 被称为 $i$ 次分量. 复形的\noun{态射} $u:L\to M$ 是指一族 $u^i:L^i\to M^i$, 使得 $d_M\circ u^i=u^{i+1}\circ d_L$. $\cA$ 上复形全体构成阿贝尔范畴 $\cC(\cA)$.

定义
  \[Z^iL=\Ker d^i:L^i\to L^{i+1},\quad B^iL=\Im d^{i-1}:L^{i-1}\to L^i,\]
  \[\rmH^i=Z^i/B^i,\]
为 $L$ 的\noun{循环}, \noun{边界}, \noun{上同调}.

\begin{definition}{拟同构}{quasi-isomorphism}
设 $u:L\to M$ 是复形的态射. 
如果 $\rmH^i(u):\rmH^iL\to \rmH^iM$ 是同构, $\forall i$, 则称 $u$ 是\noun{拟同构}.
\end{definition}

显然任意 $A\in \cA$ 可以看做 $0$ 处是 $A$, 其它地方是 $0$ 的复形.

\begin{definition}{解出}{resolution}
设 $A\in\cA$, $L,M\in\cC(\cA)$. 称 $u:L\to E$ 是一个\noun{左解出}, 如果 $L^i=0,i>0$. 这等价于给出正合列
  \[\cdots\ra L^2\ra L^1\ra L^0\ra A\ra 0.\]
类似地, 称 $u:A\to M$ 是一个\noun{右解出}, 如果 $M^i=0,i<0$. 这等价于给出正合列
  \[0\ra A\ra M^0\ra M^1\ra M^2\ra\cdots.\]
\end{definition}


\subsection{导出函子}

\begin{definition}{导出函子}{derived functor}
设 $\CF:\cA\to\cB$ 是加性范畴间的加性函子. 如果对于任意正合列
  \[0\ra A_1\ra A_2\ra A_3\ra 0,\]
序列
  \[0\ra\CF(A_1)\ra\CF(A_2)\ra \CF(A_3)\]
(或 $\CF(A_1)\ra\CF(A_2)\ra \CF(A_3)\ra 0$) 也正合, 则称 $\CF$ 是\noun{左正合}(或\noun{右正合}). 如果 $\CF$ 既左正合也右正合, 则称其\noun{正合}. 对于反变函子 $\CG:\cA\to \cB$, 我们称其左正合(或右正合)是指其对应的共变函子 $\CG^\op:\cA^\op\to\cB$ 左正合(或右正合).
\end{definition}

\begin{example}
设 $M\in\cMod/R$. 函子 $\Hom(M,-):\cMod/R\to\cMod/R$ 是左正合的. 设
  \[\xymatrix@1{0\ar[r]&A\ar[r]^u&B\ar[r]^v&C}\]
正合, 则
  \[\xymatrix@1{0\ar[r]&\Hom(M,A)\ar[r]&\Hom(M,B)\ar[r]&\Hom(M,C)}\]
正合. 显然该序列构成复形. 设 $f\in \Hom(M,A)$ 使得 $u\circ f=0$, 由于 $u$ 是单射, 因此 $f=0$. 设 $g\in \Hom(M,B)$ 使得 $v\circ g=0$, 对任意 $m\in M$, $g(m)\in \Ker v=\Im u$, 因此存在唯一的 $a\in A$ 使得 $u(a)=g(m)$. 定义 $h:M\to A,h(m)=a$, 则容易看出 $h$ 是模同态且 $u\circ h=g$. 

类似地, 反变函子 $\Hom(-,M):\cMod/R\to\cMod/R$ 左正合. 
\end{example}

\begin{example}
设 $M\in\cMod/R$, 则函子 $M\otimes-:\cMod/R\to\cMod/R$ 是右正合的.
\end{example}

\begin{definition}{内射和投射}{injective and projective}
设 $\cA$ 是阿贝尔范畴. 如果 $\Hom(-,M)$ 正合, 我们称 $M$ 是\noun{内射}的. 我们称 $\cA$ 有足够多的内射对象, 是指对任意 $L\in\cA$, 存在内射 $L'\in \cA$ 和单态射 $L\to L'$.

如果 $\Hom(M,-)$ 正合, 我们称 $M$ 是\noun{投射}的. 我们称 $\cA$ 有足够多的投射对象, 是指对任意 $L\in\cA$, 存在投射 $L'\in \cA$ 和满态射 $L'\to L$.
\end{definition}

设 $\cA$ 是有足够多的内射对象的阿贝尔范畴. 对于任意 $A\in\cA$, 存在内射 $I^0$ 和单态射 $A\to I^0$. 对其余核进行同样的操作 $\Coker(A\to I^0)\to I^1$, 反复操作下去, 我们便可得到 $A$ 的一个内射右解出
  \[0\ra A\ra I^0\ra I^1\ra \cdots.\]
对于左正合函子 $\CF:\cA\to \cB$, 复形
  \[0\ra \CF(I^0)\ra \CF(I^1)\ra \cdots\]
的上同调 $R^i\CF(A)$ 称为 $\CF$ 的\noun{右导出函子} $R^i\CF:\cA\to \cB$. 显然 $R^0\CF=\CF$.

类似地, 设 $\cA$ 是有足够多的投射对象的阿贝尔范畴. 对于任意 $A\in\cA$, 存在投射左解出
  \[\cdots\ra P^1\ra P^0\ra A\ra 0.\]
对于右正合函子 $\CF:\cA\to \cB$, 复形
  \[\cdots\ra \CF(P^1)\ra \CF(P^0)\ra 0\]
的同调 $L^i\CF(A):=\rmH^{-i}\bigl(\CF(P^\bullet)\bigr)$ 称为 $\CF$ 的\noun{左导出函子} $L^i\CF:\cA\to \cB$. 显然 $L^0\CF=\CF$.

对于反变函子, 考虑其对应的共变函子即可.

\section{群的上同调}

\subsection{上同调群}

设 $A$ 是一个 $G$ 模, 定义 $\CF(A)=A^G$ 为 $A$ 中被 $G$ 固定的部分, 则这诱导了 $G$ 模范畴到交换群范畴的一个函子. $\CF$ 是左正合的, 即如果
  \[0\ra A\ra B\ra C\]
是 $G$ 模正合列($A\ra B$ 是单射, $A\ra B$ 的像等于 $B\ra C$ 的核), 则
  \[0\ra \CF(A)\ra \CF(B)\ra \CF(C)\]
是交换群的正合列.

\begin{exercise}
证明 $A\mapsto A^G$ 是左正合的.
\end{exercise}

基于范畴的一般理论, $\CF$ 有所谓\noun{右导出函子} $\rmH^i(G,-)=R^i\CF$, 它们可以通过下述方式得到. 我们可以构造 $\BZ$ 的左解出序列
  \[\cdots\ra P_2\ra P_1\ra P_0\ra \BZ\ra 0,\]
其中 $P_i$ 都是自由 $G$ 模. 于是 $K^i=\Hom_G(P_i,A)$ 构成余链复形
  \[0\ra K_0\ra K_1\to K_2\to K_3\to\cdots,\]
即连续的两个映射的复合是 $0$, 定义
  \[\rmH^q(G,A)=\rmH^q(K)=\frac{\Ker(K^q\to K^{q+1})}{\Im(K^{q-1}\to K^q)}.\]
实际上, 我们可以取 $P_i=\BZ[G\times \cdots\times G]$, 其中一共有 $i+1$ 个 $G$, $G$ 通过对角作用, 即
  \[s.(g_0,\dots,g_i)=(sg_0,\dots,sg_i).\]
映射为
  \[d(g_0,\dots,g_1)=\sum_{j=0}^i (-1)^j(g_0,\dots,\hat g_j,\dots,g_i),\]
其中 $\hat g_j$ 表示去除该项. 特别地 $d:P_0\to\BZ$ 为 $d(g_0)=1$.

\begin{exercise}
验证 $\cdots\ra P_2\ra P_1\ra P_0\ra \BZ\ra 0$ 是正合的.
\end{exercise}

于是 $K^i=\Hom_G(P_i,A)$ 可以看成 $G\times\cdots\times G$ 上满足
  \[h(s.g_0,\dots,s.g_i)=s.h(g_0,\dots,g_i)\]
的函数全体. 由此也可以看出 $h$ 完全由函数
  \[f(g_1,\dots,g_i)=h(1,g_1,g_1g_2,\dots,g_1\dots g_i)\]
确定. 通过这种非齐次的表达式, $d$ 变为了
  \[\begin{split}
  df(g_1,\dots,g_{i+1})=&g_1.f(g_2,\dots,g_{i+1})\\
  &+\sum_{j=1}^{i}(-1)^j f(g_1,\dots,g_jg_{j+1},\dots,g_{i+1})\\
  &+(-1)^{i+1}f(g_1,\dots,g_i).
  \end{split}\]

特别地, $1$ 余循环 $\Ker(K^1\to K^2)$ 由满足
  \[f(gg')=g.f(g')+f(g)\]
的函数构成, $1$ 余边界 $\Im(K^0\to K^1)$ 由 $f(g)=g.a-a$ 形式的函数构成. 显然, 如果 $G$ 的作用是平凡的, 则 $\rmH^1(G,A)=\Hom(G,A)$. 

\begin{exercise}
$2$ 余循环满足什么条件?
\end{exercise}

由导出函子的性质, 我们有
  \[0\ra A\ra B\ra C\ra 0\]
正合, 则
  \[\cdots \ra \rmH^q(G,B)\ra \rmH^q(G,C)\sto{\delta} \rmH^{q+1}(G,A)\ra \rmH^{q+1}(G,B)\ra\cdots\]
正合, 其中 $\delta$ 被称为\noun{连接映射}.

\subsection{同调群}
设 $A$ 是一个 $G$ 模, $DA$ 为 $A$ 中 $s.a-a,s\in G$ 生成的子模, 考虑 $\CF(A)=A_G:=A/DA$, 它是 $A$ 被 $G$ 作用平凡的极大商.

\begin{exercise}
证明 $A\mapsto A_G$ 是右正合的.
\end{exercise}

基于范畴的一般理论, $\CF$ 有所谓\noun{左导出函子} $\rmH_i(G,-)=L^i\CF$, 它们可以通过下述方式得到. 类似地, 我们可以构造 $\BZ$ 的左解出序列
  \[\cdots\ra P_2\ra P_1\ra P_0\ra \BZ\ra 0,\]
其中 $P_i=\BZ[G\times \cdots\times G]$. 于是 $\rmH_q(G,A)$ 为链复形
  \[\cdots\ra P_2\otimes_G A\ra P_1\otimes_G A\ra P_0\otimes_G A\ra 0\]
其中的元素可视为函数 $x(g_1,\dots,g_q)$. 类似地, $d$ 为
  \[\begin{split}
dx(g_1,\dots,g_{q-1})=&\sum_{g\in G}g^{-1}.f(g,g_1,\dots,g_{q-1})\\
&+\sum_{j=1}^{q-1}(-1)^j \sum_{g\in G}x(g_1,\dots,g_jg,g^{-1},,\dots,g_{q-1})\\
&+(-1)^{q}f(g_1,\dots,g_{q-1},q).
\end{split}\]
我们有类似的长正合列.

若 $A=\BZ$, $G$ 为平凡作用, 则 $\rmH_1(G,\BZ)=G^\ab$. 实际上, 设 $\pi:\BZ[G]\to\BZ$ 为\noun{增广映射}, 即 $\sum n_g g\mapsto \sum n_g$. 令 $I_G$ 为其核, 即\noun{增广理想}, 它由 $g-1$ 生成. 由定义, $\rmH_0(G,A)=A/I_GA$. 考虑
  \[0\ra I_G\ra \BZ[G]\sto{\pi}\BZ\ra 0.\]
我们有 $\rmH_0(G,I_G)=I_G/I_G^2$ 且其在 $\rmH_0(G,\BZ[G])$ 中的像为 $0$. 而 $\BZ[G]$ 是自由模, 它的同调为 $0$, 因此同调的上正合列诱导了同构
  \[d:\rmH_1(G,Z)\to \rmH_0(G,I_G)=I_G/I_G^2.\]
容易验证 $s\mapsto s-1$ 诱导了同构 $G^\ab\simeq I_G/I_G^2$. 因此 $\rmH_1(G,\BZ)=G^\ab$.

由导出函子的性质, 我们有
  \[0\ra A\ra B\ra C\ra 0\]
正合, 则
  \[\cdots \ra \rmH_{q+1}(G,B)\ra \rmH_{q+1}(G,C)\sto{\delta} \rmH_{q}(G,A)\ra \rmH_{q}(G,B)\ra\cdots\]
正合, 其中 $\delta$ 被称为\noun{连接映射}.

\subsection{泰特上同调}
我们希望将群的上同调和同调统一起来. 设 $G$ 有限群. 记
  \[\bfN=\sum_{g\in G} g\in\BZ[G]\]
为它的\noun{范数},
  \[I_G=\pair{g-1\mid g\in G}\subseteq \BZ[G]\]
为\noun{增广理想}. $\bfN$ 在 $A$ 上的作用满足
  \[I_GA\subseteq A^{\bfN=0}=\ker \bfN,\quad \bfN A=\im\bfN\subseteq A^G.\]
定义\noun{泰特上同调}
  \[\begin{split}
\wh\rmH^n(G,A)&=\rmH^n(G,A),\quad n\ge 1\\
\wh\rmH^0(G,A)&=A^G/\bfN A,\\
\wh\rmH^{-1}(G,A)&=A^{\bfN=1}/I_G A,\\
\wh\rmH^{-n}(G,A)&=\rmH_{n-1}(G,A),\quad n\ge 2
\end{split}\]
则对于正合列
  \[0\ra A\ra B\ra C\ra 0,\]
我们有长正合列
  \[\cdots\ra \wh\rmH^{q-1}(G,C)\ra \wh\rmH^q(G,A)\ra \wh\rmH^q(G,B)\ra \wh\rmH^q(G,C)\ra\wh\rmH^{q+1}(G,A)\ra\cdots\]
后文中我们将简记 $\rmH^n=\wh \rmH^n,n\in\BZ$.

\subsection{埃尔布朗商}
为了计算类域的上同调, 我们需要埃尔布朗商.
设 $G$ 有限群, $A$ 是 $G$ 模, 则
  \[\begin{split}
\rmH^0(G,A)&=A^G/\bfN A,\\
\rmH^{-1}(G,A)&=A^{\bfN=1}/I_G A,\\
\rmH^1(G,A)&=Z^1/B^1,
\end{split}\]
其中
  \[Z^1:=\set{f:G\to A\mid f(gh)=f(g)^h f(h)},\]
  \[B^1:=\set{f_a:G\to A\mid f_a(g)=a^{g-1},a\in A}.\]

\begin{proposition}{}{}
如果 $G=\pair{\sigma}$ 是循环群, 则 $\rmH^1(G,A)=\rmH^{-1}(G,A)$.
对于 $G$ 模的正合列
  \[\xymatrix{1\ar[r] &A\ar[r]^i &B\ar[r]^j &C\ar[r] &1}\]
我们有正合六边形
  \[\xymatrix{
    & \rmH^0(G,A)\ar[r]^{f_1} &\rmH^0(G,B)\ar[rd]^{f_2} & \\
    \rmH^{-1}(G,C)\ar[ru]^{f_6}&&&\rmH^0(G,C)\ar[ld]^{f_3}\\
    & \rmH^{-1}(G,B)\ar[lu]^{f_5}& \rmH^{-1}(G,A)\ar[l]_{f_4}&
  }\]
\end{proposition}
该命题可以利用此情形下泰特上同调和复形
  \[\xymatrix{\cdots\ar[r]^{\sigma-1}&\BZ[G]\ar[r]^\bfN&\BZ[G]\ar[r]^{\sigma-1}&\BZ[G]\ar[r]^\bfN&\BZ[G]\ar[r]^{\sigma-1}&\dots}\]
的上同调一致得到, 见~\cite[\S 8.4]{Serre1979}. 由此可知 $\rmH^n(G,A)$ 只与 $n$ 的奇偶性有关, 从而由上同调的长正合列得到该命题.
也可以直接证明, 见~\cite[Proposition~4.3.7, Proposition~4.7.1]{Neukirch1999}, 其中 $f_3(c)=\bigl(j^{-1}(c)\bigr)^{\sigma-1}, f_6(c)=\bfN\bigl(j^{-1}(c)\bigr).$

\begin{exercise}
验证 $f_3,f_6$ 是良定义的, 并由此证明该命题.
\end{exercise}

\begin{definition}{埃尔布朗商}{Herbrand quotient}
定义 
  \[h(G,A)=\frac{\#\rmH^0(G,A)}{\#\rmH^{-1}(G,A)}\]
为 $A$ 的\noun{埃尔布朗商}. 这里它只在两个上同调都有限的情形才有定义.
\end{definition}

由正合六边形,
  \[0\ra \Im f_6\to \rmH^0(G,A)\ra \Im f_1\ra 0\]
正合, 因此 $\#\rmH^0(G,A)=\#\Im f_6\cdot\#\Im f_1$. 类似地, 对其它上同调也有这样的形式, 因此
  \[h(G,B)=h(G,A)h(G,C).\]

\begin{exercise}
证明有限模的埃尔布朗商是 $1$.
\end{exercise}

\begin{proposition}{}{cohomology_of_induced_modules}
如果 $G$ 是有限循环群, 则
  \[\rmH^i(G,\Ind^H_GB)\cong \rmH^i(H,B).\]
\end{proposition}
\begin{proof}
设 $A=\Ind_G^H B$. 设 $R$ 是 $G/H$ 的一组代表元. 考虑 $H$ 模同态
  \[\pi:A\to B,\quad f\mapsto f(1),\]
  \[\nu:A\to B,\quad f\mapsto \prod_{\tau\in R}f(\tau).\]
容易看出
  \[s:B\to A,\quad b\mapsto f_b(h)=\begin{cases}
    b^h, &\text{如果 }h\in H,\\
    1,   &\text{如果 }h\notin H
  \end{cases}\]
满足 $\pi\circ s=\nu\circ s=\id$. 我们还有
  \[\pi\circ \bfN_G=\bfN_H\circ v.\]
很明显, $\pi$ 诱导了同构 $A^G\to B^H$, 而且 
	\[\pi(\bfN_GA)=\bfN_H(\nu A)\subseteq \bfN_HB,\ 
	\bfN_H B=\bfN_H(\nu s B)=\pi\bigl(\bfN_G(sB)\bigr)\subseteq \pi(\bfN_G A).\]
因此 $\rmH^0(G,A)=\rmH^0(H,B)$. $i=-1$ 情形留作习题.
\end{proof}

\begin{exercise}
证明 $G$ 是有限循环群 时, $\rmH^{-1}(G,\Ind^H_GB)\cong \rmH^{-1}(H,B)$.
\end{exercise}

\begin{exercise}
如果 $G$ 是有限群, $H$ 是正规子群, 则 $\rmH^1(G,\Ind^H_GB)\cong \rmH^1(H,B)$.
\end{exercise}







% \backmatter
% \printbibliography[heading=bibintoc, title=\ebibname]


\phantomsection\addcontentsline{toc}{chapter}{参考文献}
\begin{thebibliography}{Car43}

\bibitem[Car43]{Cardano1643}
吉罗拉莫·卡尔达诺, 王宪生 译.
{\em 我的生平}.
浙江大学出版社, 2021.\\
译自: 
Girolamo Cardano, {\em De Vita Propria}, Paris, 1643.

\bibitem[Car63]{Cardano1663}
Gerolamo Cardano, Sydney Henry Gould (Translator),
{\em The Book on Games of Chance}, Dover Publications,2015.\\
译自: 
Gerolamo Cardano, {\em Liber de Zudo Aleae}, Paris, 1663.


\bibitem[Kli90]{Kline1990}
莫里斯·克莱因, 张理京, 张锦炎, 江泽涵 译.
{\em 古今数学思想(第一册)}.
上海科学技术出版社, 2013.\\
译自: 
Morris Kline, {\em Mathematical Thought from Ancient to Modern Times}, New York, NY, 1990.


\bibitem[Cot14]{Cotes1714}
{\em Phil. Trans.}, 29, 5--45, 1714.

\bibitem[BJMR75]{BergJulianMinesRichman1975}
Gordon O. Berg, W. Julian, R. Mines, Fred Richman,
{\em The constructive Jordan curve theorem},
Rocky Mountain Journal of Mathematics, 5 (2):225--236,
1975.

\end{thebibliography}

% \chapter*{中外人名对照表}
%\addcontentsline{toc}{chapter}{中外人名对照表}
\markboth{中外人名对照表}{中外人名对照表}
\begin{center}
\begin{tabular}{lr}
阿贝尔&Niels Henrik Abel, 1802--1829\\
阿基米德&\cmu{Ἀρχιμήδης}, 公元前 287--前 212\\
埃尔布朗&Jacques Herbrand, 1908--1931\\
艾森斯坦&Ferdinand Gotthold Max Eisenstein, 1823--1852\\
奥斯特洛斯基&\cmu{Олександр Маркович Островський}, 1893--1986\\
贝克&Alan Baker, 1939--2018\\
伯奇&Bryan John Birch, 1931--~~~~\\
泊松&Siméon Denis Poisson, 1781--1840\\
戴德金&Julius Wilhelm Richard Dedekind, 1831--1916\\
狄利克雷&Johann Peter Gustav Lejeune Dirichlet, 1805--1859\\
法尔廷斯&Gerd Faltings, 1954--~~~~\\
方丹&Jean-Marc Fontaine, 1944--2019\\
费马&Pierre de Fermat, 1601--1665\\
傅里叶&Jean-Baptiste Joseph Fourier, 1768--1830\\
弗罗贝尼乌斯&Ferdinand Georg Frobenius, 1849--1917\\
伽罗瓦&Évariste Galois, 1811-1832\\
高斯&Johann Carl Friedrich Gauß, 1777--1855\\
谷山丰&谷山豊, 1927--1953\\
哈尔&Alfréd Haar, 1885--1933\\
哈塞&Helmut Hasse, 1898--1979\\
豪斯多夫&Felix Hausdorff, 1868--1942\\
赫克&Erich Hecke, 1887--1947\\
亨泽尔&Kurt Hensel, 1861--1941\\
怀尔斯&Sir Andrew John Wiles, 1953--~~~~\\
克拉斯纳&Marc Krasner, 1912--1985\\
克鲁尔&Wolfgang Krull, 1899--1971\\
克罗内克&Leopold Kronecker, 1823--1891\\
柯西&Augustin-Louis Cauchy, 1789--1857\\
库默尔&Ernst Eduard Kummer, 1810--1893\\
莱布尼茨&Gottfried Wilhelm Freiherr von Leibniz, 1646--1716\\
勒让德&Adrien-Marie Legendre, 1752--1833\\
黎曼&Georg Friedrich Bernhard Riemann, 1826--1866\\
卢宾&Jonathan Darby Lubin, 1936--~~~~\\
闵可夫斯基&Hermann Minkowski, 1864--1909\\
默比乌斯&August Ferdinand Möbius, 1790--1868
\end{tabular}
\end{center}
\begin{center}
\begin{tabular}{lr}
牛顿&Sir Isaac Newton, 1643--1727\\
诺特&Amalie Emmy Noether, 1882--1935\\
欧拉&Leonhardus Eulerus, 1707--1783\\
庞特里亚金&\cmu{Лев Семёнович Понтря́гин}, 1908--1988\\
佩尔&John Pell, 1611--1685\\
切博塔廖夫&\cmu{Мико́ла Григо́рович Чеботарьо́в}, 1894--1947\\
塞尔&Jean-Pierre Serre, 1926--~~~~\\
施瓦兹&Laurent-Moïse Schwartz, 1915--2002\\
斯温纳顿-戴尔&Sir Henry Peter Francis Swinnerton-Dyer, 1927--2018\\
沙法列维奇&\cmu{И́горь Ростисла́вович Шафаре́вич}, 1923--2017\\
泰勒&Brook Taylor, 1685--1731\\
泰特&John Torrence Tate, 1925--2019\\
泰希米勒&Paul Julius Oswald Teichm\"uller, 1913--1943\\
韦伯&Wilhelm Eduard Weber, 1804--1891\\
魏尔斯特拉斯&Karl Theodor Wilhelm Weierstraß, 1815--1897\\
维特&Ernst Witt, 1911--1991\\
韦伊&André Weil, 1906--1998\\
希尔伯特&David Hilbert, 1862--1943\\
伯努利&Jacques Bernoulli, 1654--1705\\
岩泽健吉&岩澤健吉, 1917--1998\\
志村五郎&志村五郎, 1930--2019
\end{tabular}
\end{center}


\newpage
%\phantomsection
\addcontentsline{toc}{chapter}{索引}
\chapter*{索\qquad 引}
\begingroup  
% \let\clearpage\relax % 取消 clearpage 的效果,防止分页  
\begin{multicols}{2} % 开始两栏布局  
\printindex % 打印索引  
\end{multicols}  
\endgroup  
% \chapter*{参考答案}

\ansno{1.1.1} $-4$.
\ansno{1.1.2} $1$.
\ansno{1.1.3} $-\ov z$.

\ansno{1.2.1} $\displaystyle z=2\sqrt3\left(\cos\frac{-\pi}3+i\sin\frac{-\pi}3\right)=2\sqrt3e^{-\frac{\pi i}3}$, 写成 $\dfrac{5\pi}3$ 也可以.

\ansno{1.3.1} $-2^{2022}$.
\ansno{1.3.2} $\pm\dfrac{\sqrt3+i}2,\pm i,\pm\dfrac{\sqrt3-i}2$.

\ansno{1.4.1} 双曲线 $x^2-y^2=\dfrac12$ 和双曲线 $xy=\dfrac14$.
\ansno{1.4.2}
\begin{enumerate}
	\item 上半平面对应的闭区域为 $\Im z\ge0$.
	\item 下半平面对应的闭区域为 $\Im z\le0$.
	\item 左半平面对应的闭区域为 $\Re z\le0$.
	\item 右半平面对应的闭区域为 $\Re z\ge0$.
	\item 竖直带状区域对应的闭区域为 $x_1\le\Re z\le x_2$.
	\item 水平带状区域对应的闭区域为 $y_1\le\Im z\le y_2$.
	\item 角状区域对应的闭区域为 $\alpha_1\le \arg z\le \alpha_2$ 以及原点. 如果 $\alpha_1=-\pi,\alpha_2=\pi$, 则为 $\BC$.
	\item 圆环域对应的闭区域为 $r\le|z|\le R$.
\end{enumerate}
\ansno{1.4.3} 整个复平面.

\ansno{1.5.1}
\begin{enumerate}
	\item $\Re z$ 的定义域为 $\BC$, 值域为 $\BR$.
	\item $\arg z$ 的定义域为 $\{z\in\BC\mid z\neq 0\}$, 值域为 $(-\pi,\pi]$.
	\item $|z|$ 的定义域为 $\BC$, 值域为 $\{x\in\BR\mid x\ge0\}$.
	\item 当 $n>0$ 时, $z^n$ 的定义域为 $\BC$, 值域为 $\BC$.
	当 $n\le 0$ 时, $z^n$ 的定义域为 $\{z\in\BC\mid z\neq 0\}$, 值域为 $\{z\in\BC\mid z\neq 0\}$.
	\item $\dfrac{z+1}{z^2+1}$ 的定义域为 $\{z\in\BC\mid z\neq \pm i\}$, 值域为 $\BC$.
\end{enumerate}


\ansno{2.1.1} 处处不可导.
\ansno{2.1.2} A. 因为解析要求在 $z_0$ 的一个邻域内都可导才行.
\ansno{2.2.1} A. 根据C-R方程可知对于A, $u_x(0)=2\neq v_y(0)=3$. 对于BD, 各个偏导数在 $0$ 处取值都是 $0$. C则是处处都可导.

\ansno{2.3.1} $\ln 2-\dfrac{2\pi i}3$.
\ansno{2.3.2} $\ln 3$.



\end{document}

