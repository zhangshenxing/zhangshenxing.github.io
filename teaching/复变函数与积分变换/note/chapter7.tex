\chapter{积分变换}
\label{chapter:7}

积分变换是现代科学与工程中广泛使用的数学工具.
我们将从练离散的周期函数傅里叶级数出发, 得到连续的一般函数傅里叶变换.
然后我们将介绍傅里叶变换的各种性质, 以及它与傅里叶级数的联系.
最后我们介绍它在积分和级数的计算, 以及在微分方程中的应用.

傅里叶变换对函数要求较高, 为了解决该问题, 考虑函数与指数衰减函数乘积的傅里叶变换, 并由此得到拉普拉斯变换.
从拉普拉斯变换我们可以得到解析函数, 从而我们可以利用复变函数理论来研究.
我们将研究它的各种性质, 以及在微分方程中的应用.

\section{傅里叶变换}

\subsection{积分变换的引入}

在学习指数和对数的时候, 我们了解到利用对数可以将乘除、幂次转化为加减、乘除.
\begin{example}
  计算 $12345\times 67890$.
\end{example}

\begin{solution}
  通过查对数表得到
  \[
    \ln 12345\approx 9.4210,\qquad\ln 67890\approx 11.1256.
  \]
  将二者相加并通过反查对数表得到原值
  \[
    12345\times 67890\approx \exp(20.5466)\approx 8.3806\times 10^8.
  \]
\end{solution}

而对于函数而言, 我们常常要解函数的方程.

\begin{example}
  解微分方程
  \[
    \begin{cases}
      y''+y=t,&\\
      y(0)=y'(0)=0.&
    \end{cases}
  \]
\end{example}

我们希望能找到一种函数的\alert{变换 $\msl$}, 使得它可以把函数 $y$ 的微分和积分运算变成 $\msl[y]$ 的代数运算.
求得 $\msl[y]$ 之后再通过\alert{反变换 $\msl^{-1}$} 得到原来的函数 $y$.
这个变换最常见的就是我们将要介绍的傅里叶变换和拉普拉斯变换.


\subsection{傅里叶级数}

我们先回顾下傅里叶级数.
考虑定义在 $(-\infty,+\infty)$ 上周期为 $T>0$ 的函数 $f(t)$.
我们知道函数
\[
  1,\ \sin{\omega t},\ \cos{\omega t},\ 
  \sin{2\omega t},\ \cos{2\omega t},\ \cdots
\]
都是周期为 $T$ 的函数, 其中 $\omega=\dfrac{2\cpi}T$.
类似于线性组合的概念, 我们希望将 $f$ 表达为上述函数的可数线性组合\footnote{
  通常的线性组合是指有限项的求和.
  若向量
  \[
    \bfv=\liml_{N\ra+\infty} \sum_{n=1}^N\lambda_n \bfv_n,
  \]
  则称 $\bfv$ 是 $\bfv_1,\bfv_2,\cdots$ 的\emph{可数线性组合}.
  注意这里会涉及到该线性空间内收敛的含义.
  不过我们只考虑函数, 收敛就是指无穷级数收敛.

  类似于线性空间的一组基, 定义\emph{绍德尔基}为一组线性无关的向量, 使得任一向量都可以被这组向量可数线性组合.
}.

\begin{theorem}
  若 $f(t)$ 在区间 $(-\dfrac T2,\dfrac T2)$ 上满足\noun{狄利克雷条件}:
  \begin{enumpar}
    \item 间断点只有有限多个, 且均为第一类间断点;
    \item 只有有限个极值点,
  \end{enumpar}\par\noindent
  则 $f(t)$ 可以表达为级数
  \[
    f(t)=\frac{a_0}2+\sumf1 \left(a_n\cos n\omega t+b_n \sin n\omega t\right).
  \]
  当 $t$ 是间断点时, 等式左侧需改为 $\dfrac{f(t+)+f(t-)}2$.
\end{theorem}
上述级数被称为 $f(t)$ 的\noun{傅里叶级数}, 我们来将其改写为复指数形式.
由
\[
  \cos x=\frac{\ee^{\ii x}+\ee^{-\ii x}}2,\quad \sin x=\frac{\ee^{\ii x}-\ee^{-\ii x}}{2\ii }
\]
可知 $f(t)$ 的傅里叶级数可以改写为函数 $\ee^{\ii n\omega t}$ 的可数线性组合
\[
  f(t)=\sumff c_n\ee^{\ii n\omega t}.
\]
现在我们来计算这个可数线性组合的系数.

对于定义在 $(-\infty,+\infty)$ 上周期为 $T>0$ 的\alert{复值}函数 $f,g$, 定义内积\footnote{
  可以看出, 积分范围可以换成任意长度为 $T$ 的区间.
  为了避免和线性泛函记号冲突, 这里我们用圆括号表示内积.
}
\[
  (f,g):=\frac1T\intT f(t)\ov g(t)\d t.
\]
那么
\[
  \bigl(\ee^{\ii m\omega t},\ee^{\ii n\omega t}\bigr)
  =\frac1T\intT \ee^{\ii (m-n)\omega t}\d t
  =\begin{cases}
    1,&m=n\\
    0,&m\neq n.
  \end{cases}
\]
所以 $\set{\ee^{\ii n\omega t}\mid n\in\BZ}$ 全体构成一组标准正交基. 于是
\[
  c_n=\bigl(f,\ee^{\ii n\omega t}\bigr)
  =\frac 1T\intT f(t)\ee^{-\ii n\omega t} \d t,
\]
我们得到周期函数\noun{傅里叶级数的复指数形式}:
\[
  f(t)=\frac 1T\sumff \biggl(\intT f(\tau)\ee^{-\ii n\omega\tau} \d\tau\biggr) \ee^{\ii n\omega t}.
\]

不难知道,
\[
  (f,f)
  =\sum_{m=-\infty}^{+\infty}\sumff 
    c_m\ov{c_n}\bigl(\ee^{\ii m\omega t},\ee^{\ii n\omega t}\bigr)
  =\sumff |c_n|^2,
\]
此即勾股定理.

\begin{theorem}[帕塞瓦尔恒等式]
  对于傅里叶级数 $f(t)=\dfrac 1T\sumff c_n \ee^{\ii n\omega t}$, 有
  \[
     \frac1{T}\intT |f(t)|^2\d t
    =\sumff |c_n|^2.
  \]
\end{theorem}

\begin{example}
  计算 $\sumf1 \dfrac1{n^2}$.
\end{example}

这个问题我们在\thmref{例}{exam:bessel-question} 中使用留数计算过, 现在我们用傅里叶级数的性质来计算它.
我们需要找一个函数的傅里叶系数与 $\dfrac1{n}$ 有关的函数.

\begin{figure}[!htb]
  \centering
  \begin{tikzpicture}
    \draw[cstaxis] (-4,0)--(4,0);
    \draw[cstaxis] (0,-2)--(0,2);
    \draw[cstcurve,main] (-3,-1)--(-1,1);
    \draw[cstcurve,main] (-1,-1)--(1,1);
    \draw[cstcurve,main] (1,-1)--(3,1);
    \draw[cstdash] (-3,-1)--(-3,0);
    \draw[cstdash] (-1,-1)--(-1,1);
    \draw[cstdash] (1,-1)--(1,1);
    \draw[cstdash] (3,0)--(3,1);
    \draw (0,0) node[below right] {$0$};
    \draw (2,0) node[below right] {$2\cpi$};
    \draw (-2,0) node[above left] {$-2\cpi$};
  \end{tikzpicture}
  \caption{函数 $f(t)$ 的图像}
\end{figure}

\begin{solution}
  设 $f(t)$ 是一个周期为 $2\cpi$ 的函数, 且 $f(t)=t,t\in[-\cpi,\cpi)$.
  当 $n\neq0$ 时,
  \[
    c_n=\frac1{2\cpi}\int_{-\cpi}^\cpi t\ee^{-\ii nt}\d t
    =\frac1{2\cpi}\Bigl(\frac{\ii t}n+\frac1{n^2}\Bigr)\ee^{-\ii nt}\Big|_{-\cpi}^\cpi
    =\frac{(-1)^n\ii}n.
  \]
  当 $n=0$ 时,
  \[
    c_0=\frac1{2\cpi}\int_{-\cpi}^\cpi t\d t
    =\frac1{2\cpi}\cdot\frac{t^2}2\Big|_{-\cpi}^\cpi
    =0.
  \]
  因此
  \[
     \sum_{n\neq 0}\frac1{n^2}
    =(f,f)
    =\frac1{2\cpi}\int_{-\cpi}^{\cpi}t^2\d t
    =\frac{\cpi^2}3,
    \qquad
    \sumf1 \frac1{n^2}=\frac{\cpi^2}6.
  \]
\end{solution}


\subsection{傅里叶变换}

\subsubsection{傅里叶积分公式}

对于一般的函数 $f(t)$, 它未必是周期的.
此时它无法像前面的情形一样, 表达成可数多个函数 $\set{\ee^{\ii n\omega t}\mid n\in\BZ}$ 的可数线性组合, 而是所有的 $\set{\ee^{\ii \omega t},\omega\in(-\infty,+\infty)}$ 的某种``线性组合''.
由于我们有不可数无穷多个这样的函数, 因此这种``线性组合的系数''应当是无穷小, 求和自然需要换成积分.
若记 $\ee^{\ii \omega t}$ 的``系数''为函数 $\dfrac1{2\cpi}F(\omega)$, 则
\[
  f(t)=\frac1{2\cpi}\intff F(\omega) \ee^{\ii \omega t}\d t.
\]

我们来从傅里叶级数形式地推导出函数 $F(\omega)$ 的形式.
考虑 $f(t)$ 在 $\biggl(-\dfrac T2,\dfrac T2\biggr)$ 上的限制, 并向两边扩展成一个周期函数 $f_T(t)$.
设
\[
  \omega_n=n\omega,\qquad \delt\omega_n=\omega_n-\omega_{n-1}=\omega,
\]
则
\begin{align*}
  f(t)&=\lim_{T\ra +\infty}f_T(t)\\
  &=\lim_{T\ra+\infty} \frac 1T \sumff 
    \biggl(\intT f(\tau)\ee^{-\ii \omega_n\tau}\d\tau\biggr)
  \ee^{\ii \omega_n t}\\
  &=\frac1{2\cpi}\lim_{\delt\omega_n\ra 0}\sumff 
    \biggl(\intT f(\tau)\ee^{-\ii \omega_n\tau} \d \tau\biggr)\ee^{\ii \omega_n t}
  \delt\omega_n\\
  &=\frac1{2\cpi}\intff 
    \biggl(\intff f(\tau) \ee^{-\ii \omega\tau} \d \tau\biggr)\ee^{\ii \omega t}
  \d\omega.
\end{align*}
这个公式被称为\noun{傅里叶积分公式}.
注意这个过程并不严谨, 只是大致陈述从傅里叶级数到傅里叶变换的过程, 其结论的严格表述如下:

\begin{theorem}[\nount{傅里叶积分定理}]
  若 $f(t)$ 在 $(-\infty,+\infty)$ 上绝对可积, 且在任一有限区间上满足狄利克雷条件, 则
  \begin{equation}\label{eq:fourier-integral}
    f(t)=\frac1{2\cpi}\intff 
    \biggl(\intff f(\tau) \ee^{-\ii \omega\tau} \d \tau\biggr)\ee^{\ii \omega t}
    \d\omega.
  \end{equation}
  对于 $f(t)$ 的间断点左边需要改成 $\dfrac{f(t+)+f(t-)}2$.
\end{theorem}

\begin{definition}
  若 $f(t)$ 满足傅里叶积分定理的条件, 则称
  \[
    F(\omega)=\intff f(t) \ee^{-\ii \omega t} \d t
  \]
  为 $f(t)$ 的\noun{傅里叶变换}, 记作 \noun{$\msf[f]$}.
  称
  \[
    f(t)=\frac1{2\cpi} \intff F(\omega) \ee^{\ii \omega t} \d\omega
  \]
  为 $F(\omega)$ 的\noun{傅里叶逆变换}, 记作 \noun{$\msf^{-1}[f]$}.
\end{definition}

若 $f(t)$ 是复值函数, 也可定义其傅里叶变换, 只要其实部和虚部都满足傅里叶积分定理的条件即可.

也可以把 $f(t),F(\omega)$ 叫作一个\noun{傅里叶变换对}, $f(t),F(\omega)$ 分别叫作\noun{原函数}和\noun{象函数}.
若 $f(t)$ 表示随时间 $t$ 变化的函数, 则 $F(\omega)$ 表示的是频率 $\omega$ 的函数, 所以傅里叶变换是\alert{时域到频域的转换}.

从定义出发, 不难发现傅里叶变换满足:
\begin{theorem}[对称性质]\label{thm:symmetry-property}
  若 $\msf[f(t)]=g(\omega)$, 则 $\msf[g(t)]=2\cpi f(-\omega)$.
  对于 $f$ 的不连续点, 右侧值需要修改为左右极限平均值.
\end{theorem}

\begin{proof}
  由\thmFI 可知
  \[
    f(t)=\frac1{2\cpi}\intff g(\omega)\ee^{\ii\omega t}\d \omega,
  \]
  从而
  \[
    2\cpi f(-t)=\intff g(\omega)\ee^{-\ii\omega t}\d \omega.
  \]
  交换变量 $t$ 和 $\omega$ 即得该定理.
\end{proof}


\subsubsection{傅里叶积分公式的变化形式}

\thmFI 有一些变化形式.
例如:
\begin{align*}
  f(t)&=\frac1{2\cpi} \intff \biggl(\intff f(\tau) \ee^{-\ii \omega\tau} \d\tau\biggr) \ee^{\ii \omega t} \d\omega\\
  &=\frac1{2\cpi} \intff \intff f(\tau) \ee^{\ii \omega(t-\tau)}\d\tau \d\omega\\
  &=\frac1{2\cpi} \intff \biggl(
    \underbrace{\intff f(\tau)\cos \omega(t-\tau) \d\tau}_{\text{\small$\omega$ 的偶函数}}
    +\ii \underbrace{\intff f(\tau)\sin \omega(t-\tau) \d\tau}_{\text{\small$\omega$ 的奇函数}}
    \biggr) \d\omega\\
  &=\frac1\cpi\intf\biggl(\intff f(\tau) \cos \omega(t-\tau)\d\tau\biggr)\d\omega.
\end{align*}
于是得到\noun{傅里叶积分公式的三角形式}:
\[
  f(t)=\frac1\cpi\intf\biggl(\intff f(\tau) \cos \omega(t-\tau)\d\tau\biggr)\d\omega.
\]

若 $f(t)$ 是偶函数, 则 $f(t)\cos{\omega t}$ 是 $t$ 的偶函数, $f(t)\sin{\omega t}$ 是 $t$ 的奇函数,
\[
  F(\omega)=\intff f(t)\ee^{-\ii \omega t}\d t
  =2\intf f(t)\cos{\omega t}\d t
\]
也是偶函数, 从而得到\noun{傅里叶余弦变换}和\noun{傅里叶余弦积分公式}:
\[
  f(t)=\frac2\cpi\intf F_c(\omega)\cos\omega t\d\omega,\quad\text{其中}\ 
  F_c(\omega)=\intf f(t)\cos\omega t\d t.
\]
类似地, 若 $f(t)$ 是奇函数, 则有\noun{傅里叶正弦变换}和\noun{傅里叶正弦积分公式}:
\[
  f(t)=\frac2\cpi\intf F_s(\omega)\sin\omega t\d\omega,\quad\text{其中}\ 
  F_s(\omega)=\intf f(\tau)\sin \omega\tau\d\tau.
\]


\subsubsection{傅里叶变换计算举例}

\begin{example}\label{exam:square-pulse}
  求矩形脉冲函数 $f(t)=
    \begin{cases}
      1/2, & |t|\le 1,\\
      0, & |t|>1
    \end{cases}$
  的傅里叶变换.
\end{example}

\begin{solution}
  由于 $f(t)$ 是偶函数, 因此
  \begin{align*}
    F(\omega)&=\msf[f(t)]
    =\intff f(t)\ee^{-\ii \omega t}\d t
    =2\intf f(t)\cos{\omega t}\d t\\
    &=\int_0^1\cos{\omega t}\d t
    =\frac{\sin \omega}{\omega}.
  \end{align*}
\end{solution}

\begin{figure}[!htb]
  \centering
  \begin{tikzpicture}
    \begin{scope}
      \draw[cstaxis] (-2,0)--(2,0);
      \draw[cstaxis] (0,-1)--(0,2);
      \draw[cstdash](-1,0)--(-1,1);
      \draw[cstdash](1,0)--(1,1);
      \draw[main,cstcurve](-1,1)--(1,1);
      \draw (2,0) node[above] {$t$};
      \draw (0,2) node[right] {$f(t)$};
    \end{scope}
    \begin{scope}[xshift=6cm]
      \draw[cstaxis] (-2.5,0)--(2.5,0);
      \draw[cstaxis] (0,-1)--(0,2);
      \draw[cstcurve,second,domain=1:700,smooth] plot ({\x*0.003}, {1.5*sin(\x)*180/\x/pi});
      \draw[cstcurve,second,domain=1:700,smooth] plot ({-\x*0.003}, {1.5*sin(\x)*180/\x/pi});
      \draw (2.5,0) node[above] {$\omega$};
      \draw (0,2) node[right] {$F(\omega)$};
    \end{scope}
  \end{tikzpicture}
  \caption{矩形脉冲与 $\sinc$ 函数}
\end{figure}

我们将它的傅里叶变换称为 \noun{$\sinc$ 函数} $\sinc(x)=\dfrac{\sin x}x$.
注意到它是偶函数, 由\thmFI 可知
\[
  f(t)=\msf^{-1}[F(\omega)]=\frac1{2\cpi}\intff F(\omega)\ee^{\ii \omega t}\d\omega
  =\frac1\cpi\intf\frac{\sin\omega\cos\omega t}{\omega}\d \omega.
\]
当 $t=\pm1$ 时, 左侧应替换为 $\dfrac{f(t+)+f(t-)}2=\dfrac14$.
由此可得
\[
  \intf\frac{\sin \omega\cos\omega t}\omega\d\omega=\begin{cases}
    \cpi/2,&|t|<1,\\
    \cpi/4,&|t|=1,\\
    0,&|t|>1.
  \end{cases}
\]
特别地, 我们可以得到狄利克雷积分 $\intf\frac{\sin\omega}\omega\d\omega=\frac\cpi2$.

由\thmFSym 可知
\[
  \msf[\sinc(t)]=\begin{cases}
    \cpi,&|\omega|<1,\\
    \cpi/2,&|\omega|=1,\\
    0,&|\omega|>1.
  \end{cases}
\]

\begin{example}\label{exam:double-pulse}
  求双脉冲函数 $f(t)=
    \begin{cases}
      1,&t\in(0,1),\\
      -1,&t\in(-1,0),\\
      0,&\text{其它情形}
    \end{cases}$
  的傅里叶变换.
\end{example}

\begin{solution}
  由于 $f(t)$ 是奇函数, 因此
  \begin{align*}
    F(\omega)&=\msf[f(t)]
    =\intff f(t)\ee^{-\ii \omega t}\d t
    =2\ii \intf f(t)\sin{\omega t}\d t\\
    &{=2\ii \int_0^1\sin{\omega t}\d t}
    {=-\frac{2\ii (1-\cos\omega)}\omega.}
  \end{align*}
\end{solution}

\begin{figure}[!htb]
  \centering
  \begin{tikzpicture}
    \begin{scope}
      \draw[cstaxis] (-2,0)--(2,0);
      \draw[cstaxis] (0,-1.5)--(0,2);
      \draw[cstdash](-1,0)--(-1,-1);
      \draw[cstdash](1,0)--(1,1);
      \draw[main,cstcurve](-1,-1)--(0,-1);
      \draw[main,cstcurve](0,1)--(1,1);
      \draw (2,0) node[above] {$t$};
      \draw (0,2) node[right] {$f(t)$};
    \end{scope}
    \begin{scope}[xshift=6cm]
      \draw[cstaxis] (-2.5,0)--(2.5,0);
      \draw[cstaxis] (0,-1.5)--(0,2);
      \draw[cstcurve,second,domain=1:540,smooth] plot ({\x*pi/800}, {-1.5*(1-cos(\x))*180/\x/pi});
      \draw[cstcurve,second,domain=1:540,smooth] plot ({-\x*pi/800}, {1.5*(1-cos(\x))*180/\x/pi});
      \draw (2.5,0) node[above] {$\omega$};
      \draw (0,2) node[right] {$\ii F(\omega)$};
    \end{scope}
  \end{tikzpicture}
  \caption{双脉冲与它的傅里叶变换}
\end{figure}

与\thmref{例}{exam:square-pulse} 类似可得
\[
  \intf\frac{(1-\cos\omega)\sin\omega t}\omega\d\omega
  =\begin{cases}
    \cpi/2,&0<t<1,\\
    \cpi/4,&t=1,\\
    0,&t>1.
  \end{cases}
\]

\begin{example}\label{exam:fourier-transform-exponential-decay}
  求指数衰减函数 $f(t)=\begin{cases}
    0,&t<0,\\
    \ee^{-\beta t},&t\ge 0
  \end{cases}$ 的傅里叶变换, 其中 $\beta>0$.
\end{example}

\begin{solution}
  由定义可知
  \begin{align*}
    F(\omega)&=\msf[f(t)]=\intff f(t)\ee^{-\ii \omega t}\d t
    =\intf\ee^{-\beta t}\ee^{-\ii \omega t}\d t\\
    &=\intf\ee^{-(\beta+\ii \omega)t}\d t
    =-\frac1{\beta+\ii \omega}\ee^{-(\beta+\ii \omega)t}\Big|_0^{+\infty}
    =\frac1{\beta+\ii \omega}.
  \end{align*}
  这里, 当 $t\ra +\infty$ 时, $|\ee^{(-\beta+\ii \omega)t}|=\ee^{-\beta t}\ra 0$.
\end{solution}

\begin{figure}[!htb]
  \centering
  \begin{tikzpicture}
    \begin{scope}
      \draw[cstaxis] (-.5,0)--(3.5,0);
      \draw[cstaxis] (0,-.5)--(0,2.5);
      \draw[cstcurve,main,domain=0:25,smooth] plot ({.1*\x}, {2*exp(-.1*\x)});
      \draw (3.5,0) node[above] {$t$};
      \draw (0,2.5) node[right] {$f(t)$};
    \end{scope}
    \begin{scope}[xshift=8cm]
      \draw[cstaxis] (-2.5,0)--(2.5,0);
      \draw[cstaxis] (0,-.5)--(0,2.5);
      \draw[cstcurve,second,domain=-10:10,smooth] plot ({.2*\x}, {2/(1+.1*\x*\x)});
      \draw (2.5,0) node[above] {$\omega$};
      \draw (0,2.5) node[right] {$\abs{F(\omega)}$};
    \end{scope}
  \end{tikzpicture}
  \caption{指数衰减函数与它的傅里叶变换}
\end{figure}

不难看出, 这里实际上只要求 $\Re\beta>0$.

由\thmFI 可知
\begin{align*}
    f(t)
  &=\frac1{2\cpi}\intff \frac1{\beta+\ii \omega}\ee^{\ii \omega t}\d \omega
   =\frac1{2\cpi}\intff \frac{\beta\cos\omega t+\omega\sin\omega t}{\beta^2+\omega^2}\d \omega\\
  &=\frac1{\cpi}\intf \frac{\beta\cos\omega t+\omega\sin\omega t}{\beta^2+\omega^2}\d \omega,
\end{align*}
当 $t=0$ 时, 左侧应替换为 $\dfrac{f(0+)+f(0-)}2=\dfrac12$.
由此可得
\[
  \intf\frac{\beta\cos\omega t+\omega\sin\omega t}{\beta^2+\omega^2}\d\omega
  =\begin{cases}
    0,&t<0,\\
    \cpi/2,&t=0,\\
    \cpi \ee^{-\beta t},&t>0.
  \end{cases}
\]

\begin{example}
  求钟形脉冲函数 $f(t)=\ee^{-\beta t^2}$ 的傅里叶变换, 其中 $\beta>0$.
\end{example}

\begin{solution}
  由定义和 \ref{eq:exp-cos-integral} 可知
  \begin{align*}
    F(\omega)&=\msf[f(t)]=\intff f(t)\ee^{-\ii \omega t}\d t
    =\intff \ee^{-\beta t^2}\ee^{-\ii \omega t}\d t\\
    &=2\intf \ee^{-\beta t^2}\cos \omega t\d t
    =\sqrt{\frac\cpi\beta}\ee^{-\frac{\omega^2}{4\beta}}.
  \end{align*}
\end{solution}


\subsection{狄拉克 \texorpdfstring{$\dirac$}{δ} 函数}

傅里叶变换存在的条件是比较苛刻的.
例如常值函数 $f(t)=1$ 在 $(-\infty,+\infty)$ 上不是绝对可积的, 所以它没有傅里叶变换, 这很影响我们使用傅里叶变换.
为此我们引入广义函数的概念.

设 $\mss$ 是由一些性质较好的函数构成的线性空间, 例如我们可以取所有光滑函数 $f(t)$, 且满足对任意非负整数 $m,n$, $\liml_{t\ra\infty}f^{(m)}(t)t^n=0$.
设 $\lambda(t)$ 是局部可积缓增函数, 也就是说, 它在任一有限区间上可积, 且存在 $n$ 使得 $\liml_{t\ra\infty}\lambda(t)t^{-n}=0$.
那么对任意 $f(t)\in\mss$, $\lambda(t)f(t)$ 是绝对可积的, 从而我们可以定义一个线性映射
\begin{align*}
  \mss&\lra \BR\\
  f(t)&\mapsto \pair{\lambda,f}:=\intff \lambda(t)f(t)\d t.
\end{align*}
反过来, 这个线性映射也``几乎处处''确定了 $\lambda(t)$ 本身.\footnote{
  ``几乎处处''是指若修改 $\lambda(t)$ 在一个零测集上的函数值, 则对应的线性映射不会改变.
  集合 $S$ 是一个\emph{零测集}, 若对任意 $\varepsilon>0$, $S$ 可以被包含在至多可数个总长度为 $\varepsilon$ 的区间中.
}

\noun{广义函数}\footnote{也叫\emph{分布}.}就是指一个线性映射 $\msl:\mss\ra \BR$.
为了和普通函数类比, 常常也将广义函数写成上述积分或配对形式:
\[
  \msl(f)=\intff \lambda(t)f(t)\d t=\pair{\lambda,f}.
\]
注意这里并不是真正的积分, 只是一种记号上的约定, $\lambda(t)$ 也并不是一个真正的函数.
自然地, 局部可积缓增函数 $\lambda(t)$ 就是一个广义函数.

\begin{definition}
  狄拉克 \noun{$\dirac$ 函数}\footnotemark 是指广义函数
  \[
    \pair{\dirac,f}=\intff \dirac(t)f(t)\d t=f(0).
  \]
\end{definition}
\footnotetext{又名\emph{单位脉冲函数}.}

可以证明, 不存在局部可积缓增函数 $\lambda(t)$ 使得 $\pair{\lambda,f}=f(0)$. 也就是说, $\dirac$ 函数不可能是真正的函数.

设
\[
  \dirac_\varepsilon(t)=\begin{cases}
    1/\varepsilon,&0\le t\le \varepsilon,\\
    0,&\text{其它情形,}
  \end{cases}
\]
则
\[
  \pair{\dirac_\varepsilon,f}=\frac1\varepsilon\int_0^\varepsilon f(t)\d t=f(\xi),\quad \xi\in(0,\varepsilon).
\]
当 $\varepsilon\ra0$ 时, 右侧就趋于 $f(0)$.
因此 $\dirac$ 可以看成 $\dirac_\varepsilon$ 的``弱极限''\footnote{
  若 $\liml_{\varepsilon\ra 0}\pair{\lambda_\varepsilon,f}=\pair{\lambda,f}$, 则称 $\lambda$ 是 $\lambda_n$ 当 $\varepsilon\ra 0$ 时的\emph{弱极限}.
}.

通常用长度为 $1$ 的向上有向线段来表示它, 而用长度为 $k$ 的向上有向线段表示 $k\dirac(t)$, 长度为 $k$ 的向下有向线段表示 $-k\dirac(t)$.

\begin{figure}[!htb]
  \centering
  \begin{tikzpicture}
    \draw[cstaxis] (-1,0)--(1,0);
    \draw[cstaxis] (0,-0.5)--(0,2);
    \draw (0,1.5) node[left] {$1$}
      (1,0) node[above] {$t$}
      (0,0) node[below left] {$O$}
      (0,1.5) node[right] {$\dirac(t)$};
    \draw[main,cstcurve,cstra](0,0)--(0,1.5);
  \end{tikzpicture}
  \caption{$\dirac$ 函数}
\end{figure}

对于广义函数 $\lambda$, 仿照函数积分的性质, 我们定义
\begin{align*}
  \intff \lambda(t-t_0)f(t)\d t
  &:=\intff \lambda(t)\cdot f(t+t_0)\d t,\\
  \intff \lambda(at)f(t)\d t
  &:=\intff \lambda(t)\cdot\frac1{|a|}f\bigl(\frac ta\bigr)\d t,\\
  \intff \lambda'(t)f(t)\d t
  &:=-\intff \lambda(t)f'(t)\d t.
\end{align*}
显然 $\intff \dirac(t-t_0)f(t)\d t=f(t_0)$.

\begin{theorem}
  $\dirac$ 函数满足如下性质:
  \begin{enumpar}
    \item \noun{筛选性质}: $f(t)\dirac(t-t_0)=f(t_0)\dirac(t-t_0)$.
    \item $\dirac(at)=\dfrac1{|a|}\dirac(t)$. 特别地, $\dirac(t)=\dirac(-t)$.
    \item $\pair{\dirac^{(n)},f}=(-1)^nf^{(n)}(0)$.
    \item $u'(t)=\dirac(t)$, 其中 $u(t)=\begin{cases}1,&t\ge0,\\0,&t<0\end{cases}$ 是\noun{单位阶跃函数}\footnotemark.
  \end{enumpar}
\end{theorem}
\footnotetext{又名\emph{赫维赛德函数}.}

\begin{proof}
  \begin{enumnopar}
    \item 这是因为
    \[
      \intff \dirac(t-t_0)f(t)\cdot g(t)\d t=f(t_0)g(t_0)
      =\intff \dirac(t-t_0)f(t_0)\cdot g(t)\d t.
    \]
    \item 这是因为
    \[
      \intff \dirac(at)f(t)\d t
      =\intff \dirac(t)\cdot \frac1{|a|}f\bigl(\frac ta\bigr)\d t
      =\frac1{|a|}f(0).
    \]
    \item 这是因为
    \begin{align*}
      \intff \dirac^{(n)}(t)f(t)\d t
      &=-\intff \dirac^{(n-1)}(t)f'(t)\d t
      =\cdots\\
      &=(-1)^n\intff \dirac(t)f^{(n)}(t)\d t
      =(-1)^nf^{(n)}(0).
    \end{align*}
    \item 由于 $\liml_{t\ra\infty}f(t)=0$, 因此
    \[
      \intff u'(t)f(t)\d t
      =-\intff u(t)f'(t)\d t
      =-\intf f'(t)\d t
      =-f(t)\big|_0^{+\infty}
      =f(0).\qedhere
    \]
  \end{enumnopar}
\end{proof}

$\dirac$ 函数并非通常的函数, 自然也无法定义傅里叶变换.
对于 $f(t)\in\mss$ 以及普通函数 $\lambda$, 我们发现
\begin{align*}
  \pair{\msf[\lambda],f}
  =\intff\intff \lambda(t)\ee^{-\ii \omega t}\d t f(\omega)\d \omega
  =\intff\lambda(t) \intff \ee^{-\ii \omega t}f(\omega)\d \omega\d t 
  =\pair{\lambda,\msf[f]}.
\end{align*}
从而我们可以按如下方式定义广义函数的\noun{广义傅里叶变换}:
\[
  \pair{\msf[\lambda],f}=\pair{\lambda,\msf[f]}.
\]
于是
\[
  \pair{\msf[\dirac],f}
  =\pair{\dirac,\msf[f]}
  =\msf[f](0)
  =\intff f(t)\d t=\pair{1,f},
\]
即 $\msf[\dirac]=1$.
我们也可以直接从 $\dirac$ 函数的定义得到
\[
  \msf[\dirac(t)]
  =\intff \dirac(t)\ee^{-\ii \omega t}\d t
  =\ee^{-\ii \omega t}\big|_{t=0}
  =1.
\]
同理可得其傅里叶逆变换.

\begin{theorem}
  \[
    \msf[\dirac(t)]=1,\qquad
    \msf^{-1}[\dirac(\omega)]=\dfrac1{2\cpi}.
  \]
\end{theorem}

\begin{example}\label{exam:fourier-transform-ut}
  证明: $\msf[u(t)]=\dfrac1{\ii \omega}+\cpi\dirac(\omega)$.
\end{example}

\begin{proof}
  由定义可知
  \[
    \msf^{-1}\biggl(\frac1{\ii \omega}\biggr)
    =\frac1{2\cpi}\intff \frac{\ee^{\ii \omega t}}{\ii \omega} \d\omega
    =\frac1\cpi\intf\frac{\sin\omega t}{\omega} \d\omega.
  \]
  通过变量替换, 不难从 $\intf\frac{\sin\omega}\omega \d\omega=\dfrac\cpi2$ 得到
  \[
    \intf\frac{\sin\omega t}\omega \d\omega=\frac\cpi2\sgn(t).
  \]
  故
  \[
    \msf^{-1} \biggl(\frac1{\ii \omega}+\cpi\dirac(\omega)\biggr)
    =\half\sgn(t)+\half =u(t)\quad (t\neq 0).\qedhere
  \]
\end{proof}


\subsection{傅里叶变换的性质}

我们不可能也没必要每次都对需要变换的函数从定义出发计算傅里叶变换. 
通过研究傅里叶变换的性质, 结合常见函数的傅里叶变换, 我们可以得到很多情形的傅里叶变换.

本节中总记 $\msf[f]=F,\msf[g]=G$.
由于积分本身具有线性性质, 因此积分变换也都满足线性性质.

\begin{theorem}[线性性质]
  对于任意常数 $\alpha$ 和 $\beta$, 有
  \[
    \msf[\alpha f+\beta g]=\alpha F+\beta G,\quad
    \msf^{-1}[\alpha F+\beta G]=\alpha f+\beta g.
  \]
\end{theorem}


\subsubsection{位移性质}

\begin{theorem}[位移性质]
  \[\msf[f(t-t_0)]=\ee^{-\ii \omega t_0}F(\omega),\quad
  \msf^{-1}[F(\omega-\omega_0)]=\ee^{\ii \omega_0 t}f(t).\]
\end{theorem}

由于傅里叶变换反映的是各个频率的``线性组合系数'', 而 $\ee^{\ii \omega_0 t}f(t)$ 是 $f(t)$ 各个频率增加 $\omega_0$, 因此 $\ee^{\ii \omega_0 t}f(t)$ 的傅里叶变换图像自然就是 $f(t)$ 的傅里叶变换图像向右平移 $\omega_0$.

\begin{proof}
  由变量替换可知
  \begin{align*}
    \msf[f(t-t_0)]&=\intff f(t-t_0)\ee^{-\ii \omega t}\d t\\
    &=\intff f(t)\ee^{-\ii \omega (t+t_0)}\d t=\ee^{-\ii \omega t_0}F(\omega).
  \end{align*}
  逆变换情形类似可得.
\end{proof}

由此可得
\[
  \msf[\dirac(t-t_0)]=\ee^{-\ii \omega t_0},\quad
  \msf^{-1}[\dirac(\omega-\omega_0)]=\dfrac1{2\cpi}\ee^{\ii \omega_0 t}.
\]

\begin{example}
  求 $\sin{\omega_0 t}$ 的傅里叶变换.
\end{example}

\begin{solution}
  由于
  \[
    \msf[\ee^{\ii \omega_0t}]=2\cpi\dirac(\omega-\omega_0),
  \]
  因此
  \begin{align*}
    \msf[\sin{\omega_0 t}]&=\frac1{2\ii }\bigl(\msf[\ee^{\ii \omega_0t}]-\msf[\ee^{-\ii \omega_0t}]\bigr)\\
    &=\frac1{2\ii }\bigl(2\cpi\dirac(\omega-\omega_0)-2\cpi\dirac(\omega+\omega_0)\bigr)
    =\ii \cpi\bigl(\dirac(\omega+\omega_0)-\dirac(\omega-\omega_0)\bigr).
  \end{align*}
\end{solution}

同理可得
\[
  \msf[\cos{\omega_0 t}]
  =\cpi\bigl(\dirac(\omega+\omega_0)+\dirac(\omega-\omega_0)\bigr).
\]
我们作此汇总:
\begin{fifth}{与 $\dirac$ 函数有关的傅里叶变换汇总}
  \begin{enumpar}
    \item $\msf[\dirac(t)]=1$, $\msf[\dirac(t-t_0)]=\ee^{-\ii \omega t_0}$;
    \item $\msf[1]=2\cpi\dirac(\omega)$, $\msf[\ee^{\ii \omega_0 t}]=2\cpi\dirac(\omega-\omega_0)$;
    \item $\msf[\sin \omega_0t]=\ii \cpi[\dirac(\omega+\omega_0)-\dirac(\omega-\omega_0)]$;
    \item $\msf[\cos{\omega_0 t}]=\cpi[\dirac(\omega+\omega_0)+\dirac(\omega-\omega_0)]$;
    \item $\msf[u(t)]=\dfrac1{\ii \omega}+\cpi\dirac(\omega)$.
  \end{enumpar}
\end{fifth}


% \begin{main}{常见傅里叶变换汇总 II}
% 	\[\msf[u(t)\ee^{-\beta t}]=\dfrac1{\beta+\ii \omega},\quad
% 	\msf[\ee^{-\beta t^2}]=\sqrt{\dfrac\cpi\beta}\ee^{-\omega^2/(4\beta)},\]
% 	\[\msf[u(t)]=\dfrac1{\ii \omega}+\cpi\dirac(\omega).\]
% \end{main}





\begin{exercise}
  求 $\ii\cos{\omega_0 t}+\sin{\omega_0 t}$ 的傅里叶变换.
\end{exercise}


\subsubsection{微分性质和积分性质}

\begin{theorem}[微分性质]
  \label{thm:fourier-differential-property}
  若 $f(t)$ 在 $(-\infty,+\infty)$ 至多有有限多个点处不可导, 且 $\liml_{t\ra\infty}f(t)=0$, 则\footnotemark
  \[
    \msf[f'(t)]=\ii \omega F(\omega),\qquad
    \msf^{-1}[F'(\omega)]=-\ii tf(t).
  \]
  一般地, 若对于 $k=0,1,\cdots,n-1$, $f^{(k)}(t)$ 均在 $(-\infty,+\infty)$ 至多有有限多个点处不可导, 且 $\liml_{t\ra\infty}f^{(k)}(t)=0$, 则
  \[
    \msf[f^{(n)}(t)]=(\ii \omega)^n F(\omega),
    \msf^{-1}[F^{(n)}(\omega)]=(-\ii t)^nf(t).
  \]
\end{theorem}
\footnotetext{
  一般地, 若 $\liml_{t\ra+\infty}f(t)=A$, $\liml_{t\ra-\infty}f(t)=B$, 则 $g(t)=f(t)-A-(B-A)u(t)$ 满足定理条件.
  由于
  \[
    G(\omega)
    =F(\omega)-2A\cpi\dirac(\omega)-(B-A)\bigl(\frac1{\ii \omega}+\cpi\dirac(\omega)\bigr)
    =F(\omega)-(A+B)\cpi\dirac(\omega)-(B-A)\frac1{\ii \omega},
  \]
  因此
  \begin{align*}
    \msf[f'(t)]
    &=\msf[g'(t)]+(B-A)\msf[\dirac(t)]
    =\ii \omega G(\omega)+(B-A)
    =\ii \omega \bigl(F(\omega)-(A+B)\cpi\dirac(\omega)\bigr).
  \end{align*}
}

\begin{proof}
  由分部积分可知
  \begin{align*}
    \msf[f']
    &=\intff f'(t)\ee^{-\ii \omega t}\d t
    =f(t)\ee^{-\ii \omega t}\Big|_{-\infty}^{+\infty}-\intff f(t)\frac{\d \ee^{-\ii \omega t}}{\d t}\d t\\
    &=\ii \omega \intff f(t)\ee^{-\ii \omega t}\d t
    =\ii \omega F(\omega),
  \end{align*}
  这里, 当 $t\ra \infty$ 时, $\abs{f(t)\ee^{-\ii\omega t}}=|f(t)|\ra 0$.
  逆变换情形类似可得.
  $n$ 阶导数情形由归纳可得.
\end{proof}

由\thmFDif 自然得到函数乘多项式之后的傅里叶变换:
\begin{align*}
  \msf[tf(t)]&=\ii F'(\omega),&
  \msf^{-1}[\omega F(\omega)]&=-\ii f'(t),\\
  \msf[t^kf(t)]&=\ii ^kF^{(k)}(\omega),&
  \msf^{-1}[\omega^kF(\omega)]&=(-\ii )^kf^{(k)}(t).
\end{align*}

\begin{example}\label{exam:fourier-transform-power-rational}
  求 $t^k \ee^{-\beta t}u(t)$ 的傅里叶变换, 其中 $\beta>0$.
\end{example}

\begin{solution}
  由于
  \[
    \msf[\ee^{-\beta t}u(t)]=\frac{1}{\beta+\ii \omega},
  \]
  因此
  \[
    \msf[t^k\ee^{-\beta t}u(t)]=\ii ^k\left(\frac1{\beta+\ii \omega}\right)^{(k)}
    =\frac{k!}{(\beta+\ii \omega)^{k+1}}.
  \]
\end{solution}

\begin{theorem}[积分性质]\label{thm:fourier-integral-property}
  我们有
  \[
    \msf\Bigl(\int_{-\infty}^t f(\tau)\d\tau\Bigr)
    =\frac1{\ii \omega}F(\omega)+\cpi F(0)\dirac(\omega),
  \]
  其中 $F(0)=\intff f(t)\d t$.
\end{theorem}

\begin{proof}
  设
  \[
    g(t)=\int_{-\infty}^t f(\tau)\d \tau-F(0)u(t),
  \]
  则 $g'(t)=f(t)-F(0)\dirac(t)$.
  由\thmFDif 可知 $\msf[g'(t)]=\ii \omega G(\omega)$, 从而
  \[
    G(\omega)=\frac1{\ii \omega}\msf[g'(t)]
    =\frac1{\ii \omega}\bigl(F(\omega)-F(0)\bigr).
  \]
  再由\thmref{例}{exam:fourier-transform-ut} 可得
  \begin{align*}
    \msf\Bigl(\int_{-\infty}^t f(\tau)\d\tau\Bigr)
    &=G(\omega)+F(0)\msf[u(t)]
    =\frac1{\ii \omega}\bigl(F(\omega)-F(0)\bigr)+F(0)\bigl(\frac1{\ii \omega}+\cpi\dirac(\omega)\bigr)\\
    &=\frac1{\ii \omega}F(\omega)+\cpi F(0)\dirac(\omega).\qedhere
  \end{align*}
\end{proof}

傅里叶变换的微分和积分性质可以将微积分方程转化为象函数的代数方程, 从而可用于解微积分方程.


\subsubsection{相似性质}

由变量替换易得:
\begin{theorem}[相似性质]\label{thm:fourier-similar-property}
  \[
    \msf[f(at)]=\frac1{|a|}F\bigl(\frac\omega a\bigr),\quad
    \msf^{-1}[F(a\omega)]=\frac1{|a|}f\bigl(\frac t a\bigr).
  \]
\end{theorem}

由\thmref{例}{exam:fourier-transform-power-rational} 和\thmFSim 不难知道
\begin{equation}
  \label{eq:fourier-inverse-transform-rational-function}
  \msf^{-1}\Bigl(\frac1{(\omega-\omega_0)^k}\Bigr)
  =\begin{cases}
    \ii\dfrac{(\ii t)^{k-1}}{(k-1)!}\ee^{\ii \omega_0 t} u(t),
    &\Im\omega_0>0;\\
    -\ii\dfrac{(\ii t)^{k-1}}{(k-1)!}\ee^{\ii \omega_0 t} u(-t),
    &\Im\omega_0<0.
  \end{cases}
\end{equation}
由\thmFSym 可知,
\begin{equation}
  \label{eq:fourier-transform-rational-function}
  \msf\Bigl(\frac1{(t-t_0)^k}\Bigr)
  =\begin{cases}
    -2\cpi\ii\dfrac{(-\ii \omega)^{k-1}}{(k-1)!}\ee^{-\ii \omega t_0} u(\omega),
    &\Im t_0<0;\\
    2\cpi\ii\dfrac{(-\ii \omega)^{k-1}}{(k-1)!}\ee^{-\ii \omega t_0} u(-\omega),
    &\Im t_0>0.
  \end{cases}
\end{equation}
以上公式右侧在 $0$ 处的值都需要为左右极限的平均值.
设 $f(z)$ 是有理函数, 分母的次数大于分子的次数, 且分母没有实根.
那么 $f(z)$ 可以拆分为部分分式之和, 从而可求得其傅里叶变换和傅里叶逆变换.


\subsubsection{周期函数的傅里叶变换}

\begin{example}\label{exam:fourier-transform-period-function}
  设 $\msf[f(t)]=F(\omega)$. 求
  \[
    g(t)=\sumff f(t+nT)
  \]
  的傅里叶变换.
\end{example}

\begin{solution}
  显然 $g(t)$ 是一个周期为 $T$ 的函数.
  设 $\omega_0=\dfrac{2\cpi}T$.
  由于 $g(t)$ 的傅里叶展开系数为
  \[
    c_n=\frac1T\intT f(t)\ee^{-\ii n\omega_0 t}\d t
    =\frac 1TF(n\omega_0),
  \]
  因此
  \[
    g(t)=\frac1T\sumff F(n\omega_0)\ee^{\ii n\omega_0t},
  \]
  从而
  \[
    G(\omega)=\msf[g(t)]
    =\frac{2\cpi}T\sumff F(n\omega_0)\dirac(n-n\omega_0).
  \]
\end{solution}

由此可知, $f(t)$ 的周期扩展 $f_T(t)$ 的傅里叶变换是一系列脉冲函数的组合, 而各个脉冲的强度正是由 $F(\omega)$ 所决定.

\begin{figure}[!htb]
  \centering
  \begin{tikzpicture}[yscale=1.8]
    \begin{scope}[xscale=.4]
      \draw[cstaxis] (-6.5,0)--(6.5,0)
        node[above] {$t$};
      \draw[cstaxis] (0,-.5)--(0,2)
        node[right] {$g(t)$};
      \begin{scope}[cstdash]
        \draw (-5,0)--(-5,1);
        \draw (-3,0)--(-3,1);
        \draw (-1,0)--(-1,1);
        \draw (1,0)--(1,1);
        \draw (3,0)--(3,1);
        \draw (5,0)--(5,1);
      \end{scope}
      \draw[main,cstcurve](-5,1)--(-3,1);
      \draw[main,cstcurve](-1,1)--(1,1);
      \draw[main,cstcurve](3,1)--(5,1);
    \end{scope}
    \begin{scope}[xshift=7cm,xscale=.3]
      \draw[cstaxis] (-10.5,0)--(10.5,0)
        node[above] {$\omega$};
      \draw[cstaxis] (0,-.5)--(0,2)
        node[right] {$G(\omega)$};
      \draw[thick,domain=1:800,smooth] plot ({\x/90}, {sin(\x)*90/\x});
      \draw[thick,domain=1:800,smooth] plot ({-\x/90}, {sin(\x)*90/\x});
      \begin{scope}[cstcurve,cstra,second]
        \draw (0,0)--(0,{pi/2});
        \draw (1,0)--(1,1);
        \draw (-1,0)--(-1,1);
        \draw (3,0)--(3,-.333);
        \draw (-3,0)--(-3,-.333);
        \draw (5,0)--(5,.2);
        \draw (-5,0)--(-5,.2);
      \end{scope}
    \end{scope}
  \end{tikzpicture}
  \caption{周期函数的傅里叶变换}
\end{figure}


\subsection{傅里叶变换的应用}

\subsubsection{乘积定理以及应用}

\begin{theorem}[乘积定理]
  若 $\msf[f(t)]=F(\omega),\msf[g(t)]=G(\omega)$, 则
  \[
    \intff f(t)\ov{g(t)}\d t
    =\frac1{2\cpi}\intff F(\omega)\ov{G(\omega)}\d \omega.
  \]
  特别地,
  \[
    \intff |f(t)|^2\d t
    =\frac1{2\cpi}\intff |F(\omega)|^2\d \omega.
  \]
\end{theorem}

此即 $(F,G)=2\cpi(f,g)$.
也就是说, $\dfrac1{\sqrt{2\cpi}}\msf$ 是正交变换.

\begin{proof}
  由
  \begin{align*}
    \intff f(t)\ov{g(t)}\d t
    &=\frac1{2\cpi}\intff \intff F(\omega)\ee^{\ii \omega t}\d \omega \ov{g(t)}\d t\\
    &=\frac1{2\cpi}\intff F(\omega)\ov{\intff g(t)\ee^{-\ii \omega t}\d t}\d \omega\\
    &=\frac1{2\cpi}\intff F(\omega)\ov{G(\omega)}\d \omega
  \end{align*}
  得到.
\end{proof}

\begin{example}
  计算 $\intff \frac{\sin^4 x}{x^2}\d x$.
\end{example}

\begin{solution}
  由\thmref{例}{exam:double-pulse} 可知
  \[
    f(t)=\begin{cases}
      1, &t\in(0,1),\\
      -1, &t\in(-1,0),\\
      0,&\text{其它情形.}
    \end{cases}
  \]
  的傅里叶变换为
  \[
    F(\omega)=-2\ii\frac{1-\cos \omega}{\omega}.
  \]
  因此
  \begin{align*}
    \intff \frac{\sin^4 x}{x^2}\d x
    &=\frac12\intff \Bigl(\frac{1-\cos 2x}{2x}\Bigr)^2\d (2x)
    =\frac18\intff |F(\omega)|^2\d \omega\\
    &=\frac18\cdot2\cpi\intff |f(t)|^2\d t
    =\frac\cpi 4\int_{-1}^1 \d t
    =\frac\cpi2.
  \end{align*}
\end{solution}


\subsubsection{泊松求和公式及其应用}

\begin{theorem}[泊松求和公式]\label{thm:poisson-summation-formula}
  若 $\msf[f(t)]=F(\omega)$, 则
  \[
    \sumff f(n)=\sumff F(2\cpi n).
  \]
  若 $n$ 是 $f(t)$ 的间断点, 则 $f(n)$ 需要修改为 $\dfrac{f(n+)+f(n-)}2$.
\end{theorem}

\begin{proof}
  设
  \[
    g(t)=\sumff f(t+n).
  \]
  由\thmref{例}{exam:fourier-transform-period-function} 计算过程可知
  \[
    g(t)=\sumff F(2\cpi n)\ee^{2\cpi \ii  n t}.
  \]
  令 $t=0$ 即得.
\end{proof}

\begin{example}
  计算 $\sumff \dfrac1{n^2+1}$.
\end{example}

\begin{solution}
  由 \ref{eq:fourier-transform-rational-function} 可知 $f(t)=\dfrac1{t^2+1}$ 的傅里叶变换为
  \begin{align*}
    \msf[f(t)]
    =\frac\ii2\msf\Bigl[\frac1{t+\ii}-\frac1{t-\ii}\Bigr]
    =\frac\ii2\bigl(-2\cpi\ii \ee^{-\omega} u(\omega)-2\cpi\ii \ee^{\omega} u(-\omega)\bigr)
    =\cpi \ee^{-|\omega|}.
  \end{align*}
  这里注意第二个等式右侧在 $\omega=0$ 处的值需要修改为 $\cpi$.
  由\thmFPoi 可得
  \[
     \sumff f(n)
    =\sumff F(2\cpi n)
    =\cpi+2\cpi\sumf1 \ee^{-2\cpi n}
    =\cpi+2\cpi\frac{\ee^{-2\cpi}}{1-\ee^{-2\cpi}}
    =\frac\cpi{\tanh\cpi}.
  \]
\end{solution}

\begin{exercise}
  计算 $\sumff \dfrac1{n^2+n+1}$.
\end{exercise}
% $\dfrac{2\cpi}{\sqrt3}\cdot\dfrac{\ee^{\sqrt3\cpi}-1}{\ee^{\sqrt3\cpi}+1}
% =\dfrac{2\cpi}{\sqrt3}\tanh\dfrac{\sqrt3\cpi}2$.

% \sumff \dfrac1{n^2+a^2}=\cpi /a\tanh(\cpi a)
% \sumff \dfrac1{(n+1/2)^2+a^2}=\cpi \tanh(\cpi a)/a


\subsubsection{卷积及其应用}

\begin{definition}
  函数 $f_1(t)$ 和 $f_2(t)$ 的\noun{卷积}是指
  \[
    (f_1\ast f_2)(t)=\intff f_1(\tau)f_2(t-\tau)\d \tau.
  \]
\end{definition}

卷积满足如下重要性质:

\begin{theorem}[卷积定理]\label{thm:convolution-theorem}
  若 $f_1,f_2$ 的傅里叶变换分别为 $F_1,F_2$, 则
  \[
    \msf[f_1\ast f_2]=F_1\cdot F_2,\qquad
    \msf^{-1}[F_1\ast F_2]=\frac1{2\cpi}f_1\cdot f_2.
  \]
\end{theorem}

\begin{proof}
  我们有
  \begin{align*}
    \msf[f_1\ast f_2]&=\intff \intff f_1(\tau)f_2(t-\tau)\d \tau \cdot \ee^{-\ii \omega t}\d t\\
    &=\intff \intff f_1(\tau)\ee^{-\ii \omega \tau}\cdot f_2(t-\tau)\ee^{-\ii \omega (t-\tau)}\d t\d \tau\\
    &=\intff \intff f_1(\tau)\ee^{-\ii \omega \tau}\cdot f_2(t)\ee^{-\ii \omega t}\d t\d \tau\\
    &=\intff f_1(\tau)\ee^{-\ii \omega \tau}\d\tau\intff f_2(t)\ee^{-\ii \omega t}\d t
    =\msf[f_1]\msf[f_2].
  \end{align*}
\end{proof}

以下卷积的性质可由卷积的定义直接证明:
\begin{enumpar}
  \item $f_1\ast f_2=f_2\ast f_1,\ (f_1\ast f_2)\ast f_3=f_1\ast(f_2\ast f_3)$;
  \item $f_1\ast(f_2+f_3)=f_1\ast f_2+f_1\ast f_3$;
  \item $f\ast\dirac=f$;
  \item $(f_1\ast f_2)'=f_1'\ast f_2=f_1\ast f_2'$.
\end{enumpar}\par\noindent
不过, 利用\thmFConv 可以更容易地理解这些等式.
因为这些等式的傅里叶变换实际上是在说象函数满足乘法交换律、结合律、分配律, 且 $F\cdot 1=F$,
\[
  \ii\omega (F_1F_2)=(\ii\omega F_1)F_2
  =F_1(\ii\omega F_2).
\]
这里我们用到了 $\msf[\dirac]=1$ 以及\thmFDif.

\begin{example}
  设 $f_1(t)=u(t),f_2(t)=\ee^{-t}u(t)$. 求 $f_1\ast f_2$.
\end{example}

\begin{solution}
  由定义可知
  \[
     (f_1\ast f_2)(t)
    =\intff f_2(\tau)f_1(t-\tau)\d \tau
    =\intf \ee^{-\tau}u(t-\tau)\d \tau.
  \]
  当 $t<0$ 时, $(f_1\ast f_2)(t)=0$.
  当 $t\ge0$ 时, 
  \[
     (f_1\ast f_2)(t)
    =\int_0^t \ee^{-\tau}\d \tau
    =1-\ee^{-t}.
  \]
  故 $(f_1\ast f_2)(t)=(1-\ee^{-t})u(t)$.
\end{solution}

我们来看卷积在广义积分中的一个应用.

\begin{example}
  计算积分 $I=\intff \frac{\sin \omega}{\omega}\cdot \frac{\sin (\omega/3)}{\omega/3}\d\omega$.
\end{example}

\begin{solution}
  设 $F(\omega)=\sinc(\omega)$, $G(\omega)=\sinc(\omega/3)$, 则由傅里叶逆变换的定义可知:
  \[
     \msf^{-1}[FG](0)
    =\frac1{2\cpi}\intff F(\omega)G(\omega)\ee^{\ii \omega t}\d\omega\Big|_{\omega=0}
    =\frac1{2\cpi}I.
  \]
  由\thmref{例}{exam:square-pulse} 我们知道
  \[
    f(t)=\msf^{-1}[F(\omega)]=\begin{cases}
      1/2, & |t|<1,\\
      1/4, & |t|=1,\\
      0, & |t|>1.
    \end{cases}
  \]
  再由\thmFSim 可知
  \[
    g(t)=\msf^{-1}[G(\omega)]=3f(3t).
  \]
  由\thmFConv 可知
  \[
     \msf^{-1}[FG](0)
    =(f\ast g)(0)
    =\intff f(-t)g(t)\d t
    =\half\int_{-1}^1 g(t)\d t
    =\half,
  \]
  故 $I=2\cpi\msf^{-1}[FG](0)=\cpi$.
\end{solution}

由卷积的定义可以看出, 函数 $g(t)$ 与函数
\[
  f_\varepsilon(t)=\begin{cases}
    1/2\varepsilon, & |t|<\varepsilon,\\
    1/4\varepsilon, & |t|=\varepsilon,\\
    0, & |t|>\varepsilon
  \end{cases}
\]
的卷积在 $t$ 处的值
\[
   (g\ast f_\varepsilon)(t)
  =k\int_{-\varepsilon}^{\varepsilon} g(t-\tau)\d \tau
  =k\int_{t-\varepsilon}^{t+\varepsilon} g(\tau)\d \tau
\]
就是 $g(t)$ 在区间 $[t-\varepsilon,t+\varepsilon]$ 上的平均值.
当 $\varepsilon$ 很小时, 与 $f_\varepsilon$ 的卷积没有对 $g(t)$ 的形状作很大改变, 但却让 $g(t)$ 变得更``平滑''了.
当然, 根据我们的需要可以选择不同的函数与其作卷积, 以对齐进行平滑、锐化或边缘检测等操作.
由此可知卷积在信号和图像处理等邻域有着重要作用.

\begin{figure}[!htb]
  \centering
  \begin{tikzpicture}[xscale=.65]
    \begin{scope}
      \draw[cstaxis] (-2,0)--(2,0)
        node[above] {$t$};
      \draw[cstaxis] (0,-.5)--(0,2)
        node[right] {$g$};
      \draw[cstdash] (-1,0)--(-1,1);
      \draw[cstdash] (1,0)--(1,1);
      \draw[main,cstcurve](-1,1)--(1,1);
    \end{scope}
    \begin{scope}[xshift=7cm]
      \draw[cstaxis] (-2.5,0)--(2.5,0)
        node[above] {$t$};
      \draw[cstaxis] (0,-.5)--(0,2)
        node[right] {$g\ast f$};
      \draw[main,cstcurve](-1.5,0)--(-0.5,1)--(0.5,1)--(1.5,0);
    \end{scope}
    \begin{scope}[xshift=15cm]
      \draw[cstaxis] (-3,0)--(3,0)
        node[above] {$t$};
      \draw[cstaxis] (0,-.5)--(0,2)
        node[right] {$g\ast f\ast f$};
      \draw[cstcurve,main,domain=-10:10,smooth] plot ({\x*0.1}, {1-.5*\x*\x*0.01});
      \draw[cstcurve,main,domain=0:10,smooth] plot ({2-\x*0.1}, {.5*\x*\x*0.01});
      \draw[cstcurve,main,domain=0:10,smooth] plot ({-2+\x*0.1}, {.5*\x*\x*0.01});
    \end{scope}
  \end{tikzpicture}
  \caption{卷积将函数变平滑}
\end{figure}



\subsubsection{利用傅里叶变换解函数方程}

利用积分变换解含微分或积分的方程的步骤如下: 对原方程两边同时作积分变换, 将原函数 $y$ 的方程转化为象函数 $Y$ 的代数方程.
解出 $Y$ 并对齐应用积分逆变换得到原函数 $y$.\footnote{
  事实上在使用\thmFDif 时, 我们对 $f(t)$ 本身是有要求的.
  在实际应用中, 我们往往先忽略这种要求而利用该性质求得方程的解, 然后有必要的话再验证解的合理性.
}

\begin{figure}[!htb]
  \centering
  \begin{tikzpicture}[
    node distance=40pt,
    align=center,
    minimum height=30pt,
    minimum width=60pt
  ]
    \node[cstnode] (a){含微分或\\积分的方程};
    \node[cstnode,right=120pt of a] (b){象函数的\\代数方程};
    \node[cstnode,below=of a] (c){原函数\\即方程的解};
    \node[cstnode,below=of b] (d){象函数};
    \draw[cstra,cstcurve,main] (a)--node[above,third]{傅里叶变换 $\msf$}(b);
    \draw[cstra,cstcurve,main] (d)--node[below,third]{傅里叶逆变换 $\msf^{-1}$}(c);
    \draw[cstra,cstcurve,main] (b)--(d);
    \draw[cstnra,fourth] (a)--(c);
  \end{tikzpicture}
  \caption{使用积分变换解函数方程的步骤}
\end{figure}

\begin{example}
  解方程 $y'(t)-\dint_{-\infty}^t y(\tau)\d \tau=2\dirac(t)$.
\end{example}

\begin{solution}
  设 $\msf[y]=Y$.
  两边同时作傅里叶变换并由\thmFDif 和\thmFInt 得到
  \[
    \ii \omega Y(\omega)-\frac1{\ii \omega}Y(\omega)
    -\cpi Y(0)\dirac(\omega)=2,
  \]
  \begin{align*}
     Y(\omega)
    =-\frac{\ii \omega}{1+\omega^2}\bigl(2+\cpi Y(0)\dirac(\omega)\bigr)
    =-\frac{2\ii \omega}{1+\omega^2}
    =-\ii\Bigl(\frac1{\omega-\ii}+\frac1{\omega+\ii}\Bigr).
  \end{align*}
  这里我们利用了 $\dirac$ 函数的筛选性质.
  再由 \ref{eq:fourier-inverse-transform-rational-function} 可知
  \[
     y(t)
    =-\ii\msf^{-1}\biggl(\frac1{\omega-\ii}+\frac1{\omega+\ii}\biggr)
    =-\ii\bigl(\ii \ee^{-t}u(t)-\ii \ee^tu(-t)\bigr)
    =\sgn(t)\ee^{-|t|}.
  \]
\end{solution}

傅里叶变换也可以用来解偏微分方程.
我们来利用它解一维波动方程的初值问题.

\begin{example}
  解方程
  \[
     \dfrac{\partial^2 u}{\partial t^2}
    =\dfrac{\partial^2u}{\partial x^2},\qquad
     u(x,0)=\cos x,\quad
     \dfrac{\partial u}{\partial t}(x,0)=\sin x.
  \]
  其中 $u(x,t)$ 是一个二元实函数.
\end{example}

\begin{solution}
  记 $U(\omega,t)=\msf[u(x,t)]$ 为 $u$ 关于 $x$ 的傅里叶变换, 那么
  \[
    \msf\biggl[\frac{\partial^2u}{\partial x^2}\biggr]
   =(\ii\omega)^2 U
   =-\omega^2 U,\quad
    \msf\biggl[\frac{\partial^2u}{\partial t^2}\biggr]
   =\frac{\partial^2 U}{\partial t^2},
  \]
  对原方程组等式两边同时作傅里叶变换, 我们得到
  \[\begin{cases}
     \dfrac{\partial^2 U}{\partial t^2}
    =-\omega^2 U,\\
     U(\omega,0)
    =\msf[\cos x]
    =\cpi\bigl(\dirac(\omega+1)+\dirac(\omega-1)\bigr),\\
     \dfrac{\partial U}{\partial t}(\omega,0)
    =\msf[\sin x]
    =\cpi\ii\bigl(\dirac(\omega+1)-\dirac(\omega-1)\bigr).
  \end{cases}\]
  该方程是 $t$ 的二阶常系数齐次常微分方程, 其通解为
  \[
    U(\omega,t)=c_1\sin\omega t+c_{2}\cos\omega t.
  \]
  代入初值条件可知
  \[
     c_1
    =\frac{\cpi\ii}{\omega}\bigl(\dirac(\omega+1)-\dirac(\omega-1)\bigr),\quad
     c_2
    =\cpi\bigl(\dirac(\omega+1)+\dirac(\omega-1)\bigr).
  \]
  由 $\dirac$ 函数的筛选性质可得
  \begin{align*}
     U(\omega,t)&
    =\Bigl(\cpi\cos\omega t+\frac{\cpi\ii}\omega\sin\omega t\Bigr)\dirac(\omega+1)
    +\Bigl(\cpi\cos\omega t-\frac{\cpi\ii}\omega\sin\omega t\Bigr)\dirac(\omega-1)\\&
    =(\cpi\cos t+\cpi\ii\sin t)\dirac(\omega+1)
    +(\cpi\cos t-\cpi\ii\sin t)\dirac(\omega-1).
  \end{align*}
  因此
  \begin{align*}
     u(x,t)&
    =\msf^{-1}[U(\omega,t)]\\&
    =\frac1{2\cpi}\bigl(
       (\cpi\cos t+\cpi\ii\sin t)\ee^{-\ii x}
      +(\cpi\cos t-\cpi\ii\sin t)\ee^{\ii x}
    \bigr)\\&
    =\cos t\cos x+\sin t\sin x\\&
    =\cos\left(t-x\right).
  \end{align*}
\end{solution}

\begin{example}
  \label{exam:second-ode}
  解方程 $y''(t)-y(t)=0$.
\end{example}

\begin{solution}
  设 $\msf[y]=Y$, 则
  \[
    \msf[y''(t)-y(t)]=[(\ii \omega)^2-1]Y(\omega)=0,
  \]
  \[Y(\omega)=0,\quad y(t)=\msf^{-1}[Y(\omega)]=0.\]
\end{solution}

显然这是不对的, 该方程的解应该是 $y(t)=C_1\ee^t+C_2\ee^{-t}$.
原因在于使用傅里叶变换解函数方程只能得到存在傅里叶变换的函数, 而像 $\ee^t,\ee^{-t}$ 这样的函数并没有傅里叶变换.\footnote{
  即使考虑广义函数它们也没有傅里叶变换, 因为它们不是缓增的.
}
为了解决这个问题, 我们需要一个对函数限制更少的积分变换来解决此类方程.





\section{拉普拉斯变换}

\subsection{拉普拉斯变换}

傅里叶变换对函数要求过高, 这使得在很多时候无法应用它, 或者要引入复杂的广义函数.
对于一般的 $f(t)$, 为了让它绝对可积, 我们考虑它与指数衰减函数的乘积
\[
  f(t)u(t)\ee^{-\beta t},\qquad\beta>0.
\]
它的傅里叶变换为
\[
   \msf\bigl[f(t)u(t)\ee^{-\beta t}\bigr]
  =\intf f(t)\ee^{-(\beta+\ii \omega)t}\d t
  =\intf f(t)\ee^{-st}\d t,
\]
其中 $s=\beta+\ii \omega$.
这样的积分在我们遇到的多数情形都是存在的, 只要选择充分大的 $\beta=\Re s$.

\begin{theorem}[拉普拉斯变换存在定理]
  若定义在 $[0,+\infty)$ 上的函数 $f(t)$ 满足:
  \begin{enumpar}
    \item 在任一有限区间上至多只有有限多间断点;
    \item 存在 $M,c$ 使得 $|f(t)|\le M\ee^{ct}$,
  \end{enumpar}\par\noindent
  则 $\msf\bigl[f(t)u(t)\ee^{-\beta t}\bigr]$ 在 $\Re s>c$ 上存在且为解析函数.
\end{theorem}

\begin{definition}
  若 $f(t)$ 满足拉普拉斯变换存在定理的条件, 则称
  \[
    F(s)=\intf f(t)\ee^{-st}\d t
  \]
  为 $f(t)$ 的\noun{拉普拉斯变换}, 记作 \noun{$\msl[f(t)]$}.
  称 $f(t)$ 为 $F(s)$ 的\noun{拉普拉斯逆变换}, 记作 \noun{$\msl^{-1}[F(s)]$}.
\end{definition}

虽然我们限定了函数只定义在 $t\ge 0$ 处, 但很多时候这不影响我们使用.
例如, 在物理学或工程学中, 很多时候我们只需要考虑系统自某个时间点开始之后的行为.

\begin{example}
  求拉普拉斯变换 $\msl[\ee^{kt}]$.
\end{example}

\begin{solution}
  由定义可知
  \[
     \msl[\ee^{kt}]
    =\intf\ee^{kt}\ee^{-st}\d t
    =\intf\ee^{-(s-k)t}\d t
    =-\frac1{s-k}\ee^{-(s-k)t}\Big|_0^{+\infty}
    =\frac1{s-k},
  \]
  其中 $\Re s>\Re k$.
  于是得到 $\msl[\ee^{kt}]=\dfrac1{s-k}$.
\end{solution}

特别地, $\msl[1]=\dfrac1s$.

\begin{example}
  求拉普拉斯变换 $\msl[t^m]$, 其中 $m$ 是正整数.
\end{example}

\begin{solution}
  由分部积分可知
  \[
     \msl[t^m]
    =\intf t^m\ee^{-st}\d t
    =-\frac{t^m\ee^{-st}}s\Big|_0^{+\infty}+\intf\frac{\ee^{-st}}s\cdot mt^{m-1}\d t
    =\frac ms\msl[t^{m-1}].
  \]
  归纳可知
  \[
    \msl[t^m]=\dfrac{m!}{s^m}\msl[1]{=\dfrac{m!}{s^{m+1}}}.
  \]
\end{solution}

事实上, 对于任意实数 $m>-1$, 均有 $\msl[t^m]=\dfrac{\Gamma(m+1)}{s^{m+1}}$, 其中 $\Gamma$ 函数定义为
\[
  \Gamma(m)=\intf \ee^{-t}t^{m-1}\d t.
\]

\subsection{拉普拉斯变换的性质}

和傅里叶变换类似, 拉普拉斯变换也有着各种性质.
本节中总记 $\msl[f]=F,\msl[g]=G$.

\begin{theorem}[线性性质]
  对于任意常数 $\alpha$ 和 $\beta$, 有
  \[
    \msl[\alpha f+\beta g]=\alpha F+\beta G,\quad
    \msl^{-1}[\alpha F+\beta G]=\alpha f+\beta g.
  \]
\end{theorem}

\begin{example}
  求函数 $\sin{kt}$ 的拉普拉斯变换.
\end{example}

\begin{solution}
  由于
  \[
    \msl[\ee^{kt}]=\frac1{s-k},
  \]
  因此
  \[
     \msl[\sin{kt}]
    =\frac{\msl[\ee^{\ii kt}]-\msl[\ee^{-\ii kt}]}{2\ii }
    =\frac1{2\ii }\left(\frac1{s-\ii k}-\frac1{s+\ii k}\right)
    =\frac k{s^2+k^2}.
  \]
\end{solution}

同理可得
\[
  \msl[\cos kt]=\frac{s}{s^2+k^2}.
\]

\begin{exercise}
  求 $\sin{kt}\cos{kt}$ 的拉普拉斯变换.
\end{exercise}

\begin{theorem}[位移性质]
  \label{thm:laplace-shift-property}
  若 $\msl[f(t)]=F(s)$, 则 
  \[
    \msl[\ee^{s_0 t}f(t)]=F(s-s_0).
  \]
\end{theorem}

\begin{proof}
  我们有
  \begin{align*}
     \msl[\ee^{s_0 t}f(t)]
    =\intf \ee^{s_0 t}f(t)\ee^{-st}\d t
    =\intf f(t)\ee^{-(s-s_0) t}\d t
     =F(s-s_0).\qedhere
  \end{align*}
\end{proof}

\begin{example}
  求函数 $t^m\ee^{kt}$ 的拉普拉斯变换, 其中 $t$ 是正整数.
\end{example}

\begin{solution}
  由于 $\msl[t^m]=\dfrac{m!}{s^{m+1}}$, 因此由\thmLShift 可知
  \[
    \msl[t^m\ee^{kt}]=\frac{m!}{(s-k)^{m+1}}.
  \]
\end{solution}

设 $F(s)$ 是有理函数, 且分母的次数大于分子的次数.
那么 $F(s)$ 可以拆分为部分分式之和, 从而可求得其拉普拉斯逆变换.

\begin{theorem}[延迟性质]
  若 $f(t)$ 满足当 $t<0$ 时, $f(t)=0$, 则
  \[
    \msl[f(t-t_0)]=\ee^{-s t_0}F(s).
  \]
\end{theorem}

\begin{proof}
  由变量替换可知
  \begin{align*}
      \msl[f(t-t_0)]
    &=\intf f(t-t_0)\ee^{-s t}\d t
     =\int_{-t_0}^{+\infty} f(t)\ee^{-s(t+t_0)}\d t\\
    &=\intf f(t)\ee^{-s(t+t_0)}\d t
     =\ee^{-s t_0}F(s).\qedhere
  \end{align*}
\end{proof}

\begin{theorem}[微分性质]
  \label{thm:laplace-differential-property}
  若 $\msl[f(t)]=F(s)$, 则 $\msl[f'(t)]=sF(s)-f(0)$.
  由此可知
  \[
    \msl[f''(t)]=s^2F(s)-sf(0)-f'(0).
  \]
  一般地,
  \[
    \msl[f^{(n)}(t)]=s^nF(s)-s^{n-1}f(0)-\cdots-sf^{(n-2)}(0)-f^{(n)}(0).
  \]
\end{theorem}

\begin{proof}
  由分部积分可知
  \begin{align*}
     \msl[f'(t)]
    &=\intf f'(t)\ee^{-st}\d t
    =f(t)\ee^{-st}\Big|_0^{+\infty}-\intf f(t)\frac{\d \ee^{-st}}{\d t}\d t\\
    &=-f(0)+s\intf f(t)\ee^{-st}\d t
    =sF(s)-f(0),
  \end{align*}
  这里, 由于 $|f(t)|\le M\ee^{ct}$, 因此当 $\Re s>c$ 时, $\liml_{t\ra+\infty} f(t)\ee^{-st}=0$.
  
  将 $f(t)$ 换成 $f'(t)$ 得到
  \[
    \msl[f''(t)]=s\msl[f'(t)]-f'(0)
    =s\bigl(sF(s)-f(0)\bigr)-f'(0)
    =s^2F(s)-sf(0)-f'(0).
  \]
  $n$ 阶导数情形由归纳可得.
\end{proof}

\begin{theorem}[乘多项式性质]
  \label{thm:laplace-multiply-polynomial-property}
  若 $\msl[f(t)]=F(s)$, 则 $\msl[tf(t)]=-F'(s)$.
  一般地, $\msl[t^nf(t)]=(-1)^n F^{(n)}(s)$.
\end{theorem}

\begin{proof}
  对
  \[
    F(s)=\intf f(t)\ee^{-st}\d t
  \]
  两边同时对 $s$ 求导得到
  \[
    F'(s)=\intf -tf(t)\ee^{-st}\d t.
  \]
  于是命题得证. 一般情况归纳可得.
\end{proof}

\begin{theorem}[积分性质]
  \label{thm:laplace-integral-property}
  若 $\msl[f(t)]=F(s)$, 则
  \[
    \msl\Bigl[\int_0^t f(\tau)\d\tau\Bigr]
    =\frac1{s}F(s).
  \]
\end{theorem}

\begin{proof}
  设
  \[
    g(t)=\int_0^t f(\tau)\d \tau,
  \]
  则 $g'(t)=f(t), g(0)=0$.
  由\thmLDif 可知 $F(s)=\msl[f(t)]=\msl[g'(t)]=s\msl[g(t)]$, 从而命题得证.
\end{proof}

拉普拉斯变换的微分和积分性质可以将微积分方程转化为象函数的代数方程, 从而可用于解微积分方程.

我们作此汇总:
\begin{fifth}{与有理函数有关的拉普拉斯变换汇总}
  \begin{enumpar}
    \item $\displaystyle\msl[1]=\frac1s$, 
    $\displaystyle \msl[\ee^{kt}]=\frac1{s-k}$;
    \item $\displaystyle\msl[t^m]=\frac{m!}{s^{m+1}}$, 
    $\displaystyle\msl[t^m\ee^{kt}]=\frac{m!}{(s-k)^{m+1}}$;
    \item $\displaystyle\msl[\sin kt]=\frac{k}{s^2+k^2}$, 
    $\displaystyle\msl[\ee^{at}\sin kt]=\frac{k}{(s-a)^2+k^2}$;
    \item $\displaystyle\msl[\cos kt]=\frac{s}{s^2+k^2}$,
    $\displaystyle\msl[\ee^{at}\cos kt]=\frac{s-a}{(s-a)^2+k^2}$.
  \end{enumpar}
\end{fifth}

由变量替换易得:
\begin{theorem}[相似性质]\label{thm:laplace-similar-property}
  对于 $a>0$, 
  \[
    \msl[f(at)]=\frac1a F\bigl(\frac sa\bigr).
  \]
\end{theorem}

由于在拉普拉斯变换中, 我们考虑的函数在 $t<0$ 时都是零.
此时函数的卷积变成了
\[
   f_1(t)\ast f_2(t)
  =\int_0^t f_1(\tau)f_2(t-\tau)\d \tau,\quad t\ge 0,
\]
且我们有如下的卷积定理.

\begin{theorem}[卷积定理]
  若 $f_1(t),f_2(t)$ 的拉普拉斯变换分别为 $F_1(s),F_2(s)$, 则
  \[
    \msl[f_1(t)\ast f_2(t)]=F_1(s)\cdot F_2(s).
  \]
\end{theorem}



\subsection{拉普拉斯逆变换}

由于拉普拉斯变换来自傅里叶变换, 因此其逆变换也具有和傅里叶变换类似的形式.
设
\[
   F(s)
  =\msl[f(t)]
  =\msf[f(t)u(t)\ee^{-\beta t}],
\]
则
\[
   f(t)u(t)\ee^{-\beta t}
  =\frac1{2\cpi}\intff F(s)\ee^{\ii \omega t}\d \omega.
\]
由于 $s=\beta+\ii\omega$, 因此 $\d s=\ii\d \omega$,
\[
   f(t)u(t)
  =\frac1{2\cpi\ii}\int_{\beta-\ii\infty}^{\beta+\ii\infty} F(s)\ee^{(\beta+\ii\omega)t}\d(\beta+\ii\omega)
  =\frac1{2\cpi\ii}\int_{\beta-\ii\infty}^{\beta+\ii\infty} F(s)\ee^{st}\d s.
\]
由于拉普拉斯变换只关心 $f(t)$ 在 $t\ge 0$ 的取值, 因此
\[
   f(t)
  =\frac1{2\cpi\ii}\int_{\beta-\ii\infty}^{\beta+\ii\infty} F(s)\ee^{st}\d s.
\]

当 $F(s)$ 满足一定条件时, 我们可以使用留数计算它的拉普拉斯逆变换.

\begin{theorem}
  若 $F(s)$ 只有有限多个奇点 $s_1,\cdots,s_k$, 且 $\liml_{s\ra\infty}F(s)=0$, 则
  \[
    \msl^{-1}[F(s)]
    =\sum_{k=1}^n \Res\bigl(F(s)\ee^{st},s_k\bigr).
  \]
\end{theorem}

\begin{figure}[!htb]
  \centering
  \begin{tikzpicture}
    \draw[cstaxis] (-3,0)--(2,0);
    \draw[cstaxis] (0,-3)--(0,3);
    \draw[
      cstcurve,
      second,
      decoration={
        markings,
        mark = at position .75 with {
          \arrow{Straight Barb}
          \node[right]{$\ell$};
        }
      },
      postaction={decorate}
    ] (.5,-2)--(.5,2);
    \draw[
      cstcurve,
      main,
      decoration={
        markings,
        mark = at position .35 with {
          \arrow{Straight Barb}
          \node[above left]{$C_R$};
        }
      },
      postaction={decorate}
    ] (.5,2) arc (90:270:2);
    \draw[cstra] (.5,0)--node[below]{$R$}({.5-2*cos(35)},{-2*sin(35)});
    \fill[cstdot,third] (-1,-.5) circle;
    \fill[cstdot,third] (-.2,-1.2) circle;
    \fill[cstdot,third] (-.6,1) circle;
    \node[below right] at (.5,0) {$\beta$};
  \end{tikzpicture}
  \caption{左半圆形闭路}
  \label{fig:left-half-circle-contour}
\end{figure}

\begin{proof}
  选择如\ref{fig:left-half-circle-contour} 所示的闭路 $C=C_R+\ell$, 其中 $C_R$ 是以 $R$ 为半径, $\beta$ 为圆心的左半圆, $\ell$ 是从 $\beta-\ii R$ 到 $\beta+\ii R$ 的直线段.
  选择合适的 $\beta$ 和充分大的 $R$ 使得 $F(s)$ 的所有奇点均在 $C$ 的内部.
  设 $F(s)$ 在 $C_R$ 上满足 $|F(s)|\le M_R$.
  作变量替换 $s=\beta+iR\ee^{\ii\theta}\in C_R,\theta\in[0,\cpi]$, 则由\thmGrowUp,
  \[
     \abs{\int_{C_R} F(s)\ee^{st}\d s}
    \le M_R \int_0^\cpi \ee^{(\beta-R\sin \theta)t} \cdot R\d \theta
    =2RM_R\int_0^{\cpi/2}\ee^{(\beta-R\sin \theta)t}\d \theta.
  \]
  注意到当 $\theta\in[0,\cpi/2]$ 时, $\sin\theta\ge \dfrac{2\theta}\cpi$, 因此
  \[
     \abs{\int_{C_R} F(s)\ee^{st}\d s}
    \le2RM_R\int_0^{\cpi/2}\ee^{(\beta-\frac{2R}{\cpi}\theta)t}\d \theta.
    =\frac1tM_R\cpi\ee^{\beta t}(1-\ee^{-Rt}).
  \]
  由 $\liml_{R\ra+\infty} M_R=0$ 可得
  \[
    \lim_{R\ra+\infty} \int_{C_R} F(s)\ee^{st}\d s=0.
  \]

  由于 $\ee^{st}$ 处处解析, 从而 $F(s)\ee^{st}$ 的奇点也是 $s_1,\cdots,s_k$.
  由\thmRes 可知
  \[
    \int_{C_R} F(s)\ee^{st}\d s+\int_\ell F(s)\ee^{st}\d s
    =\int_C F(s)\ee^{st}\d s
    =2\cpi\ii\sum_{k=1}^n\Res\bigl(F(s)\ee^{st},s_k\bigr).
  \]
  再令 $R\ra+\infty$ 得到
  \[
     \msl^{-1}[F(s)]
    =\frac1{2\cpi\ii}\int_{\beta-\ii\infty}^{\beta+\ii\infty} F(s)\ee^{st}\d s
    =\sum_{k=1}^n\Res\bigl(F(s)\ee^{st},s_k\bigr).
    \qedhere
  \]
\end{proof}

\begin{example}
  求 $F(s)=\dfrac1{s(s-1)^2}$ 的拉普拉斯逆变换.
\end{example}

\begin{solution}
  由
  \begin{align*}
     \Res[F(s)\ee^{st},0]
    &=\frac{\ee^{st}}{(s-1)^2}\Big|_{s=0}
     =1,\\
     \Res[F(s)\ee^{st},1]
    &=\left(\frac{\ee^{st}}s\right)'\Big|_{s=1}
     =\frac{t\ee^{st}s-\ee^{st}}{s^2}\Big|_{s=1}
     =(t-1)\ee^t,
  \end{align*}
  可知 $\msl^{-1}[F(s)]=1+(t-1)\ee^t$.
\end{solution}

我们也可以直接利用常见函数的拉普拉斯变换来计算逆变换.

\begin{solution}[另解]
  设
  \[
    F(s)=\frac as+\frac b{s-1}+\frac c{(s-1)^2},
  \]
  则
  \begin{align*}
    a&=\lim_{s\ra0}sF(s)
      =1,\\
    b&=\lim_{s\ra1}\bigl((s-1)^2F(s)\bigr)'
      =\lim_{s\ra1}\Bigl(-\frac1{s^2}\Bigr)
      =-1,\\
    c&=\lim_{s\ra1}(s-1)^2F(s)
      =1.
  \end{align*}
  故
  \[
     \msl^{-1}[F(s)]
    =\msl^{-1}\biggl[\frac1s-\frac1{s-1}+\frac1{(s-1)^2}\biggr]
    =1-\ee^t+t\ee^t.
  \]
\end{solution}


\subsection{拉普拉斯变换的应用}

和傅里叶变换类似, 拉普拉斯变换也可用来解微分和积分方程, 步骤是类似的.

\begin{example}
  解微分方程
  \[
    \begin{cases}
      y''+2y=\sin t,\\
      y(0)=0,\quad y'(0)=2.
    \end{cases}
  \]
\end{example}

\begin{solution}
  设 $\msl[y]=Y$, 则由\thmLDif
  \[
     \msl[y'']
    =s^2Y-sy(0)-y'(0)
    =s^2Y-2.
  \]
  对原方程两边同时作拉普拉斯变换得到
  \[
     s^2Y-2+2Y
    =\msl[\sin t]
    =\frac{1}{s^2+1},
  \]
  解得
  \[
     Y(s)
    =\frac{2}{s^2+2}+\frac{1}{(s^2+1)(s^2+2)}
    =\frac{1}{s^2+1}+\frac{1}{s^2+2},
  \]
  从而
  \[
     y(t)
    =\msl^{-1}\biggl[\frac{1}{s^2+1}\biggr]+\msl^{-1}\biggl[\frac{1}{s^2+2}\biggr]
    =\sin t+\frac{\sqrt 2}2\sin(\sqrt 2 t).
  \]
\end{solution}

\begin{example}
  解微分方程
  \[
    \begin{cases}
      x'(t)+2x(t)+2y(t)=10\ee^{2t}, \\
      -2x(t)+y'(t)+3y(t)=13\ee^{2t}, \\
      x(0)=1,y(0)=3.
    \end{cases}
  \]
\end{example}

\begin{solution}
  设 $\msl[x]=X,\msl[y]=Y$.
  对原方程组两边同时作拉普拉斯变换得到
  \[
    \begin{cases}
      sX-1+2X+2Y=\dfrac{10}{s-2},\\[2\itemsep]
      -2X+sY-3+3Y=\dfrac{13}{s-2}.
    \end{cases}
  \]
  解得
  \[
    X(s)=\frac1{s-2},\qquad 
    Y(s)=\frac3{s-2},
  \]
  从而
  \[
     x(t)
    =\msl^{-1}\biggl[\frac1{s-2}\biggr]
    =\ee^{2t},\qquad
     y(t)
    =\msl^{-1}\biggl[\frac3{s-2}\biggr]
    =3\ee^{2t}.
  \]
\end{solution}

\begin{example}
  解微分方程 $y''(t)-y(t)=0$.
\end{example}

\begin{solution}
  设 $a=y(0),b=y'(0),\msl[y]=Y$, 则由\thmLDif
  \[
    \msl[y'']=s^2Y-as-b.
  \]
  对原方程两边同时作拉普拉斯变换得到
  \[
    s^2Y-as-b-Y=0,
  \]
  解得
  \[
     Y(s)
    =\frac{as+b}{s^2-1}
    =\frac{a+b}2\cdot\frac1{s-1}+\frac{a-b}2\cdot\frac1{s+1},
  \]
  从而
  \[
     y(t)
    =\msl^{-1}[Y(s)]
    =\Res[Y(s)\ee^{st},1]+\Res[Y(s)\ee^{st},-1]
    =\frac{a+b}2\ee^t+\frac{a-b}2\ee^{-t}.
  \]
\end{solution}

这样, 我们便用拉普拉斯变换解决了傅里叶变换不能解决的方程\thmref{例}{exam:second-ode}.




\begin{exercise}
单选题: 下列不是傅里叶变换对的是\fillbrace{}.
\begin{exchoice}(2)
  \item $\dirac(t),1$
  \item $\ee^{\ii \omega_0t},2\cpi\dirac(\omega-\omega_0)$
  \item $\sin \omega_0t, \cpi[\dirac(\omega-\omega_0)+\dirac(\omega+\omega_0)]$
  \item $1,2\cpi\dirac(\omega)$
\end{exchoice}
\end{exercise}


\begin{exercise}
填空题: $\dirac(t-t_0)$ 的傅里叶变换为 $F(\omega)=$\fillblank{$\ee^{-\ii \omega t_0}$}.
\end{exercise}


\begin{exercise}
填空题: $F(\omega)=\dirac(\omega+2)$ 的傅里叶逆变换为 $f(t)=$\fillblank[2cm][1mm]{$\frac1{2\cpi}\ee^{-2\ii t}$}.
\end{exercise}


\begin{exercise}
填空题: $F(\omega)=2\cpi\dirac(\omega)$ 的傅里叶逆变换为 $f(t)=$\fillblank{$1$}.
\end{exercise}


\begin{exercise}
填空题: $f(t)=\sin t+\ii \cos t$ 的傅里叶变换为\fillblank[3cm]{$2\cpi \ii \dirac(\omega+1)$}.
\end{exercise}


\begin{exercise}
用拉普拉斯变换解微分方程
\[\begin{cases}y''+4y'+3y=\ee^{-t},&\\y(0)=y'(0)=1.\end{cases}\]
\end{exercise}
\begin{solution}
设 $\msl[y]=Y$, 则
\[\msl[y']=sY-y(0)=sY-1,\]
\[\msl[y'']=s^2Y-sy(0)-y'(0)=s^2Y-s-1,\]
因此
\[s^2Y-s-1+4(sY-1)+3Y=\msl[\ee^{-t}]=\frac{1}{s+1},\]
\[Y(s)=\frac{1}{2(s+1)^2}+\frac{7}{4(s+1)}-\frac3{4(s+3)},\]
\[y(t)=\msl^{-1}[Y(s)]=\frac12 t\ee^{-t}+\frac74 \ee^{-t}-\frac34 \ee^{-3t}. \qedhere\]
\end{solution}


\begin{exercise}
用拉普拉斯变换解微分方程
\[\begin{cases}y''+y=t,&\\y(0)=1,\quad y'(0)=-2.\end{cases}\]
\end{exercise}
\begin{solution}
设 $\msl[y]=Y$, 则
\[\msl[y'']=s^2Y-sy(0)-y'(0)=s^2Y-s+2,\]
因此
\[s^2Y-s+2+Y=\msl[t]=\frac{1}{s^2},\]
\[Y(s)=\frac1{s^2}-\frac3{s^2+1}+\frac s{s^2+1},\]
\[y(t)=\msl^{-1}[Y(s)]=t-3\sin t+\cos t. \qedhere\]
\end{solution}


\begin{exercise}
用拉普拉斯变换解微分方程
\[\begin{cases}y''+4y'+3y=\ee^t,&\\y(0)=0,\quad y'(0)=2.\end{cases}\]
\end{exercise}
\begin{solution}
设 $\msl[y]=Y$, 则
\[\msl[y']=sY-y(0)=sY,\]
\[\msl[y'']=s^2Y-sy(0)-y'(0)=s^2Y-2,\]
因此
\[s^2Y-2+4sY+3Y=\msl[\ee^t]=\frac{1}{s-1},\]
\[Y(s)=\frac{1}{8(s-1)}+\frac{3}{4(s+1)}-\frac7{8(s+3)},\]
\[y(t)=\msl^{-1}[Y(s)]=\frac18\ee^t+\frac34 \ee^{-t}-\frac78 \ee^{-3t}. \qedhere\]
\end{solution}


\begin{exercise}
用拉普拉斯变换解微分方程
\[\begin{cases}y''-3y'+2y=2\ee^{-t},&\\y(0)=2,\quad y'(0)=-1.\end{cases}\]
\end{exercise}
\begin{solution}
设 $\msl[y]=Y$, 则
\[\msl[y']=sY-y(0)=sY-2,\]
\[\msl[y'']=s^2Y-sy(0)-y'(0)=s^2Y-2s+1,\]
因此
\[s^2Y-2s+1-3(sY-2)+2Y=\msl[2\ee^{-t}]=\frac{2}{s+1},\]
\[Y(s)=\frac{4}{s-1}+\frac{1}{3(s+1)}-\frac7{3(s-2)},\]
\[y(t)=\msl^{-1}[Y(s)]=4\ee^t+\frac13 \ee^{-t}-\frac73\ee^{2t}. \qedhere\]
\end{solution}


\begin{exercise}
用拉普拉斯变换解微分方程
\[\begin{cases}y''(t)+4y(t)=3\cos t,& \\y(0)=y'(0)=0.&\end{cases}\]
\end{exercise}
\begin{solution}
设 $\msl[y]=Y$, 则
\[\msl[y']=s^2Y-sy(0)-y'(0)=s^2Y,\]
因此
\[s^2Y+4Y=\frac{3s}{s^2+1},\]
\[Y(s)=\frac{3s}{(s^2+1)(s^2+4)}=\frac{s}{s^2+1}-\frac{s}{s^2+4},\]
\[y(t)=\msl^{-1}[Y(s)]=\cos t-\cos(2t).\qedhere\]
\end{solution}




\section*{扩展阅读}
该部分作业不需要交, 有兴趣的同学可以做完后交到本人邮箱.
\begin{exercise}
对于正奇数 $k$, 设 $F_k(\omega)=\dfrac{\sin(\omega/k)}{\omega/k}$, 设
\[I_k=\intff F_1(\omega)F_3(\omega)\cdots F_k(\omega)\d\omega.\]
则
\[I_1=I_3=I_5=\cdots=I_{13}=\cpi,\quad
I_{15}=\frac{467 807 924 713 440 738 696 537 864 469}{467 807 924 720 320 453 655 260 875 000}\cpi.\]
这是为什么呢?

  根据 
\[f(t)=\msf^{-1}[F_1]=\begin{cases}
  1/2, & |t|<1,\\
  1/4, & |t|=1,\\
  0, & |t|>1
\end{cases}\]
得到 $\msf^{-1}[F_k]=f_k(t):=kf(kt)$.

  注意到函数 $g(t)$ 和 $f_k(t)$ 卷积之后在 $t$ 处的值相当于 $g(t)$ 在 $\Bigl[t-\dfrac1k,t+\dfrac1k\Bigr]$ 上取平均值.
由此证明 $f_1(t)\ast f_3(t)\ast \cdots\ast f_k(t)$ 在 $|t|<1-\dfrac13-\dfrac15-\cdots-\dfrac1k$ 上取值为 $\dfrac12$.

  根据
\[\frac1{2\cpi}I_k=\msf^{-1}[F_1F_3\cdots F_k](0)=(f_1\ast f_3\ast\cdots\ast f_k)(0)\]
解释上述现象.
更多细节可见\cite{3B1Bb}.
\end{exercise}



最后我们来看卷积对波尔文积分现象的解释.
对于正整数 $k$, 设 $F_k(\omega)=\sinc(\omega/k)$.
那么积分序列
\[
  I_k=\intff F_1(\omega)F_3(\omega)\cdots F_k(\omega)\d\omega
\]
的前几项为
\[
  I_1=I_3=I_5=\cdots=I_{13}=\cpi,\quad
  I_{15}=\frac{467 807 924 713 440 738 696 537 864 469}{467 807 924 720 320 453 655 260 875 000}\cpi.
\]
为什么会发生这种奇怪的现象呢?
由\thmFSim 可知 $f_k(t):=\msf^{-1}[F_k(\omega)]=kf(kt)$.
% 由卷积的定义可以看出, 函数 $g(t)$ 和 $f_k(t)$ 的卷积在 $t$ 处的值
% \[
% 	 (g\ast f_k)(t)
% 	=k\int_{-1/k}^{1/k} g(t-\tau)\d \tau
% 	=k\int_{t-1/k}^{t+1/k} g(\tau)\d \tau
% \]
% 就是 $g(t)$ 在区间 $[t-1/k,t+1/k]$ 上的平均值.
注意到 $f_1(t)=f(t)$ 在 $(-1,1)$ 上恒为 $1/2$, 因此 $f_1\ast f_3$ 在 $(-1+1/3,1-1/3)$ 上恒为 $1/2$.
依次递推下去可知, $f_1(t)\ast f_3(t)\ast \cdots\ast f_k(t)$ 在
\[
  |t|<1-\dfrac13-\dfrac15-\cdots-\dfrac1k
\]
上取值为 $\dfrac12$.
对于 $k=1,3\cdots,13$, 上述不等式右侧大于零, 从而
\[
  I_k=2\cpi (f_1\ast f_3\ast \cdots\ast f_k)(0)=\cpi.
\]
而对于 $k\ge 15$, 上述不等式右侧小于零, 从而 $I_k=\cpi$ 不再成立.
更多细节可见\cite{3B1Bb}.
% 更多细节可见: \url{https://www.bilibili.com/video/BV18e4y1u7BH/}


% \item 利用拉普拉斯变换解微分方程
% $\begin{cases}
% y''(t)-y'(t)=\ee^t,&\\
% y(0)=y'(0)=0.&
% \end{cases}$


% \item 利用拉氏变换解微分方程 
% \[\begin{cases}
% y''+2y'+y=t\ee^t&\\
% y(0)=y'(0)=0.&
% \end{cases}\]






% \item 函数 $\sin t+\ii \cos t$ 的傅里叶变换为\fillblank{}.
% \item 函数 $\ee^{\ii t}$ 的傅里叶变换为\fillblank{}.
% \item 函数 $f(t)=\cos(3t)$ 的傅里叶变换为 $F(\omega)=$\fillblank{}.
% \item 常值函数 $F(\omega)=2$ 的傅里叶逆变换为 $f(t)=$\fillblank{}.
% \item 用拉普拉斯变换求解微分方程初值问题
% \[\begin{cases}
% y''(t)+2y(t)=\sin t,&\\
% y(0)=0,\quad y'(0)=2.
% \end{cases}\]
% \item 用拉普拉斯变换求解微分方程初值问题
% \[\begin{cases}
% y''(t)+2y'(t)=8\ee^{2t},&\\
% y(0)=0,\quad y'(0)=2.
% \end{cases}\]
% \item 设 $f(z)=\dfrac{\ee^z}{(z-\cpi\ii)(z-2\cpi\ii)^2}$. 求 $f(z)$ 在有限复平面内的奇点以及 $\displaystyle\oint_{|z|=8}f(z)\d z$.
% \item 用拉普拉斯变换求解微分方程初值问题
% \[\begin{cases}
% y''(t)+4y(t)=3\cos t,&\\
% y(0)=1,\quad y'(0)=2.
% \end{cases}\]
% \item 用拉普拉斯变换求解微分方程初值问题
% \[\begin{cases}
% y''(t)-4y(t)=3\ee^t,&\\
% y(0)=0,\quad y'(0)=1.
% \end{cases}\]
