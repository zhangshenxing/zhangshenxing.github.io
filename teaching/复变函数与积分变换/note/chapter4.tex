
\chapter{级数}
\section{复数项级数}

\subsection{复数项级数}

复数域上的级数与实数域上的级数并无本质差别.

\begin{definition}
	\begin{itemize}
		\item 设 $\{z_n\}_{n\ge1}$ 是复数列. 表达式 $\suml_{n=1}^\infty z_n$ 称为复数项\emph{无穷级数}.
		\item 称 $s_n:=z_1+z_2+\cdots+z_n$ 为该级数的\emph{部分和}.
		\item 如果部分和数列 $\set{s_n}_{n\ge 1}$ 极限存在, 则称 $\suml_{n=1}^\infty z_n$ \emph{收敛}, 并记 $\suml_{n=1}^\infty z_n=\lim\limits_{n\to\infty}s_n$ 为它的\emph{和}. 否则称该级数\emph{发散}.
	\end{itemize}
\end{definition}

如果 $\suml_{n=1}^\infty z_n=A$ 收敛, 则 $z_n=s_n-s_{n-1}\to A-A=0$.
因此 \emph{$z_n\to0$ 是 $\suml_{n=1}^\infty z_n$ 收敛的必要条件}.

\begin{theorem}
	$\suml_{n=1}^\infty z_n=a+bi$ 当且仅当 $\suml_{n=1}^\infty x_n=a,\suml_{n=1}^\infty y_n=b$.
\end{theorem}

\begin{proof}
	设部分和
	\[\sigma_n=x_1+x_2+\cdots+x_n,\quad
		\tau_n=y_1+y_2+\cdots+y_n.\]
	{则
		\[s_n=z_1+z_2+\cdots+z_n=\sigma_n+i\tau_n.\]由复数列的敛散性判定条件可知
		\[\lim_{n\to\infty}s_n=a+bi\iff	\lim_{n\to\infty}\sigma_n=a,\quad \lim_{n\to\infty}\tau_n=b.\]于是命题得证.\qedhere}
\end{proof}

\begin{theorem}
	如果实数项级数
	\[\sum_{n=1}^\infty|z_n|=|z_1|+|z_2|+\cdots\]
	收敛, 则 $\suml_{n=1}^\infty z_n$ 也收敛, 且 $\abs{\suml_{n=1}^\infty z_n}\le\suml_{n=1}^\infty |z_n|$.
\end{theorem}

\begin{proof}
	因为 $|x_n|,|y_n|\le|z_n|$, 由比较判别法可知实数项级数 $\suml_{n=1}^\infty x_n$, $\suml_{n=1}^\infty y_n$ 绝对收敛, 从而收敛.
	{故 $\suml_{n=1}^\infty z_n$ 也收敛.}

	{由三角不等式可知
		$\displaystyle\abs{\sum_{k=1}^n z_k}\le \sum_{k=1}^n|z_k|$.两边同时取极限即得级数的不等式关系
		\[\abs{\sum_{n=1}^\infty z_n}=\abs{\lim_{n\to\infty}\sum_{k=1}^n z_k}=
		\lim_{n\to\infty}\abs{\sum_{k=1}^n z_k}\le\lim_{n\to\infty}\sum_{k=1}^n|z_k|=\sum_{n=1}^\infty |z_n|,\]其中第二个等式是因为绝对值函数 $|z|$ 连续.\qedhere}
\end{proof}

\subsection{绝对收敛和条件收敛}

\begin{definition}
		\begin{enumerate}
		\item 如果级数 $\suml_{n=1}^\infty |z_n|$ 收敛, 则称 $\suml_{n=1}^\infty z_n$ \emph{绝对收敛}.
		\item 称收敛但不绝对收敛的级数\emph{条件收敛}.
	\end{enumerate}
\end{definition}

\begin{theorem}
	$\suml_{n=1}^\infty z_n$ 绝对收敛当且仅当它的实部和虚部级数都绝对收敛.
\end{theorem}

\begin{proof}
	必要性由前一定理的证明已经知道,
	{充分性由 $|z_n|\le|x_n|+|y_n|$ 可得.\qedhere}
\end{proof}

\begin{tikzpicture}[overlay]
	\draw[decorate,decoration={brace,amplitude=8},thick,main] (2.56,-4.25)--(2.56,-1.33);
	\draw[decorate,decoration={brace,amplitude=8},thick,second] (4.52,-0.34)--(10.33,-0.34);
	\draw (7.57,0.2) node[second] {实部级数}
	(1.9,-3) node[align=center,main] {虚\\部\\级\\数};
\end{tikzpicture}
% \begin{center}
% 	\begin{tabular}{|c|c|c|c|}
% 		\hline
% 		&~~发散~~&条件收敛&绝对收敛\\\hline
% 		发散&发散&发散&发散\\\hline
% 		条件收敛&发散&\cellcolor{lightg} 条件收敛
% 			&\cellcolor{lightg} 条件收敛\\\hline
% 		绝对收敛&发散&\cellcolor{lightg}条件收敛
% 			&\cellcolor{lightr}绝对收敛\\\hline
% 	\end{tabular}
% \end{center}

绝对收敛的复级数各项可以任意重排次序而不改变其绝对收敛性, 且不改变其和.

一般的级数重排有限项不改变其敛散性与和, 但如果重排无限项则可能会改变其敛散性与和.

想一想: 什么时候 $\abs{\suml_{n=1}^\infty z_n}=\suml_{n=1}^\infty|z_n|$?

\begin{example}
	级数 $\displaystyle\sum_{n=1}^\infty\frac{1+i^n}n$ 发散、条件收敛、还是绝对收敛?
\end{example}

\begin{solution}
	由于实部级数
	\[\sum_{n=1}^\infty x_n=
	1+\frac13+\frac24+\frac15+\frac17+\frac28+\cdots>\sum_{n=1}^\infty\frac1{2n-1}\]
	发散, 所以该级数发散.
\end{solution}

它的虚部级数是一个交错级数, 从而是条件收敛的.

\begin{example}
	级数 $\displaystyle\sum_{n=1}^\infty\dfrac{i^n}n$ 发散、条件收敛、还是绝对收敛?
\end{example}

\begin{solution}
	因为它的实部和虚部级数
	\[\sum_{n=1}^\infty x_n=-\half +\frac14+\frac16-\frac18+\cdots\]
{
	\[\sum_{n=1}^\infty y_n=1-\frac13+\frac15-\frac17+\cdots\]
	均条件收敛,
}{所以原级数条件收敛.}
\end{solution}

\begin{exercise}
	级数 $\displaystyle\sum_{n=1}^\infty\left[\frac{(-1)^n}n+\frac i{2^n}\right]$ 发散、条件收敛、还是绝对收敛?
\end{exercise}

由正项级数的判别法可以得到:
设
\begin{enumerate}
	\item \emph{达朗贝尔判别法(比值法)}: $\lambda=\displaystyle\lim_{n\to\infty}\abs{\frac{z_{n+1}}{z_n}}$ (假设存在);
	\item \emph{柯西判别法(根式法)}: $\lambda=\displaystyle\lim_{n\to\infty}\sqrt[n]{\abs{z_n}}$ (假设存在);
	\item \emph{柯西-阿达马判别法}: $\lambda=\displaystyle\ov{\lim_{n\to\infty}}\sqrt[n]{\abs{z_n}}$ (子数列中极限的最大值).
\end{enumerate}

\begin{itemize}
	\item 当 $\lambda<1$ 时, $\suml_{n=0}^\infty z_n$ 绝对收敛.
	\item 当 $\lambda>1$ 时, $\suml_{n=0}^\infty z_n$ 发散.
	\item 当 $\lambda=1$ 时, 无法使用该方法判断敛散性.
\end{itemize}
其证明是通过将该级数与相应的等比级数做比较得到的.

\begin{example}
	级数 $\displaystyle\sum_{n=0}^\infty\frac{(8i)^n}{n!}$ 发散、条件收敛、还是绝对收敛?
\end{example}

\begin{solution}
	因为 $\displaystyle\lim_{n\to\infty}\abs{\frac{z_{n+1}}{z_n}}=\lim_{n\to\infty}\abs{\dfrac{8}{n+1}}=0$, 所以该级数绝对收敛.
\end{solution}

实际上, 它的实部和虚部级数分别为
\[1-\frac{8^2}{2!}+\frac{8^4}{4!}-\cdots=\cos 8,\quad
8-\frac{8^3}{3!}+\frac{8^5}{5!}-\cdots=\sin 8,\]
因此
\[\sum_{n=0}^\infty\frac{(8i)^n}{n!}=\cos 8+i\sin 8=e^{8i}.\]

\section{幂级数}

\subsection{幂级数的收敛域}

复变函数级数与实变量函数级数也是类似的.

\begin{definition}
	\begin{itemize}
		\item 设 $\{f_n(z)\}_{n\ge 1}$ 是一个复变函数列, 其中每一项都在区域 $D$ 上有定义.
		表达式 $\suml_{n=1}^\infty f_n(z)$ 称为\emph{复变函数项级数}.
		\item 对于 $z_0\in D$, 如果级数 $\suml_{n=1}^\infty f_n(z_0)$ 收敛, 则称 \emph{$\suml_{n=1}^\infty f_n(z)$ 在 $z_0$ 处收敛}, 相应级数的值称为它的\emph{和}.
		\item 如果 $\suml_{n=1}^\infty f_n(z)$ 在 $D$ 上处处收敛, 则它的和是一个函数, 称为\emph{和函数}.
		\item 称形如 $\suml_{n=0}^\infty c_n(z-a)^n$ 的函数项级数为\emph{幂级数}.
	\end{itemize}
\end{definition}

我们只需要考虑 $a=0$ 情形的幂级数, 因为二者的收敛范围与和函数只是差一个平移.

\begin{theorem}{阿贝尔定理}
	\begin{enumerate}
		\item 如果 $\suml_{n=0}^\infty c_nz^n$ 在 $z_0\neq 0$ 处收敛, 那么对任意 $|z|<|z_0|$ 的 $z$, 该级数必绝对收敛.
		\item 如果 $\suml_{n=0}^\infty c_nz^n$ 在 $z_0\neq 0$ 处发散, 那么对任意 $|z|>|z_0|$ 的 $z$, 该级数必发散.
	\end{enumerate}
\end{theorem}

\begin{proof}
	{\enumnum1 因为级数收敛, 所以 $\lim\limits_{n\to\infty}c_n z_0^n=0$.故存在 $M$ 使得 $|c_nz_0^n|<M$.对于 $|z|<|z_0|$,
		\[\sum_{n=0}^\infty|c_nz^n|=\sum_{n=0}^\infty|c_nz_0^n|\cdot\abs{\frac z{z_0}}^n
		{\le M\sum_{n=0}^\infty\abs{\frac z{z_0}}^n
		=\frac{M}{1-\abs{\dfrac z{z_0}}}.}\]所以级数在 $z$ 处绝对收敛.\enumnum2是\enumnum1的逆否命题.\qedhere}
\end{proof}

设 $R$ 是实幂级数 $\suml_{n=0}^\infty|c_n|x^n$ 的收敛半径.
\begin{itemize}
	\item 如果 $R=+\infty$, 由阿贝尔定理可知 $\suml_{n=0}^\infty c_nz^n$ 处处绝对收敛.
	\item 如果 $0<R<+\infty$, 那么 $\suml_{n=0}^\infty c_nz^n$ 在 $|z|<R$ 上绝对收敛, 在 $|z|>R$ 上发散.
	\item 如果 $R=0$, 那么 $\suml_{n=0}^\infty c_nz^n$ 仅在 $z=0$ 处收敛, 对任意 $z\neq 0$ 都发散.
\end{itemize}
我们称 $R$ 为该幂级数的\emph{收敛半径}.

\begin{center}
	\begin{tikzpicture}
		\filldraw[cstcurve,main,cstfill2] (0,0) circle (1.2);
		\fill[cstdot] (0,0) circle;
		\draw[cstcurve,cstra] (0,0)--(0.96,0.72);
		\draw[cstcurve,cstra] (-0.96,0.72)--(-2,0.72);
		\draw
			(0.6,0.1) node {$R$}
			(0,-0.4) node[second] {绝对收敛}
			(2.5,-0.4) node[second] {发散}
			(-3,0.72) node[main] {都有可能};
	\end{tikzpicture}
\end{center}

\begin{example}
	求幂级数 $\suml_{n=0}^\infty z^n=1+z+z^2+\cdots$ 的收敛半径与和函数.
\end{example}

\begin{solution}
	如果幂级数收敛, 则由 $z^n\to0$ 可知 $|z|<1$.
	{当 $|z|<1$ 时, 和函数为
		\[\lim_{n\to\infty}s_n=\lim_{n\to\infty}\frac{1-z^{n+1}}{1-z}=\frac1{1-z}.\]因此收敛半径为 $1$.}
\end{solution}

\subsection{收敛半径的计算}

由正项级数的相应判别法容易得到公式 $R=\dfrac1r$, 其中
\begin{enumerate}
	\item \emph{达朗贝尔公式(比值法)}: $r=\displaystyle\lim_{n\to\infty}\abs{\frac{c_{n+1}}{c_n}}$ (假设存在);
	\item \emph{柯西公式(根式法)}: $r=\displaystyle\lim_{n\to\infty}\sqrt[n]{|c_n|}$ (假设存在);
	\item \emph{柯西-阿达马公式}: $r=\displaystyle\ov{\lim_{n\to\infty}} \sqrt[n]{|c_n|}$.
\end{enumerate}
如果 $r=0$ 或 $+\infty$, 则 $R=+\infty$ 或 $0$.

\begin{example}
	求幂级数 $\displaystyle\sum_{n=1}^\infty\frac{(z-1)^n}n$ 的收敛半径, 并讨论 $z=0,2$ 的情形.
\end{example}

\begin{solution}
	由 $\displaystyle\lim_{n\to\infty}\abs{\frac{c_{n+1}}{c_n}}=\lim_{n\to\infty}\frac n{n+1}=1$ 可知收敛半径为 $1$.

	{当 $z=2$ 时, $\displaystyle\sum_{n=1}^\infty\frac{(z-1)^n}n=\sum_{n=1}^\infty\frac1n$ 发散.}

	{当 $z=0$ 时, $\displaystyle\sum_{n=1}^\infty\frac{(z-1)^n}n=\sum_{n=1}^\infty\frac{(-1)^n}n$ 收敛.}
\end{solution}

事实上, \emph{收敛圆周上既可能处处收敛, 也可能处处发散, 也可能既有收敛的点也有发散的点}.

\begin{example}
	求幂级数 $\suml_{n=0}^\infty\cos(in)z^n$ 的收敛半径.
\end{example}

\begin{solution}
	我们有 $c_n=\cos(in)=\dfrac{e^n+e^{-n}}2$.
	{由
		\[\lim_{n\to\infty}\abs{\frac{c_{n+1}}{c_n}}=\lim_{n\to\infty}\frac{e^{n+1}+e^{-n-1}}{e^n+e^{-n}}=e\lim_{n\to\infty}\frac{1+e^{-2n-2}}{1+e^{-2n}}=e\]
	可知收敛半径为 $\dfrac1e$.}
\end{solution}

\begin{exercise}
	幂级数 $\suml_{n=0}^\infty(1+i)^nz^n$ 的收敛半径为\fillblank[2cm][2mm]{{$\dfrac{\sqrt2}2$}}.
\end{exercise}

% 
% 
% \begin{example}
% 求幂级数 $\displaystyle\sum_{n=1}^\infty\frac{z^n}{n^p}$ 的收敛半径并讨论在收敛圆周上的情形, 其中 $p\in\BR$.
% \end{example}
% 
% \begin{solution}
% 由 $\displaystyle\lim_{n\to\infty}\abs{\frac{c_{n+1}}{c_n}}=\lim_{n\to\infty}\left(\frac n{n+1}\right)^p=1$ 可知收敛半径为 $1$.
% {设 $|z|=1$.
% \begin{itemize}
% \item 若 $p>1$, $\displaystyle\sum_{n=1}^\infty\abs{\frac{z^n}{n^p}}=\sum_{n=1}^\infty\frac1{n^p}$ 收敛,
% {原级数在收敛圆周上处处绝对收敛.}
% \item 若 $p\le 0$, $\abs{\dfrac{z^n}{n^p}}=\dfrac1{n^p}\not\to0$,
% {原级数在收敛圆周上处处发散.}
% \end{itemize}}
% \end{solution}
% \end{frame}

% 
% 
% 回忆\emph{狄利克雷判别法}: 若 $\set{a_n}_{n\ge 1}$ 部分和有界, 实数项数列 $\set{b_n}_{n\ge 1}$ 单调趋于 $0$, 则 $\suml_{n=1}^\infty a_nb_n$ 收敛.

% 
% \begin{solution}
% \begin{itemize}
% \item 若 $0<p\le1$, $\displaystyle\sum_{n=1}^\infty\frac1{n^p}$ 发散, 
% {而在收敛圆周上其它点 $z\neq1$ 处,
% \[|z+z^2+\cdots+z^n|=\abs{\frac{z(1-z^n)}{1-z}}
% \le\frac{2}{|1-z|}\]
% 有界, 数列 $\set{n^{-p}}_{n\ge 1}$ 单调趋于 $0$,}
% {因此 $\displaystyle\sum_{n=1}^\infty\frac{z^n}{n^p}$ 收敛.}
% {故该级数在 $z=1$ 发散, 在收敛圆周上其它点收敛.}
% \end{itemize}
% \end{solution}
% \end{frame}

\subsection{幂级数的运算性质}

\begin{theorem}
	设幂级数
	\[f(z)=\sum_{n=0}^\infty a_nz^n,|z|<R_1,\quad
	g(z)=\sum_{n=0}^\infty b_nz^n,|z|<R_2.\]
	{那么当 $|z|<R=\min\{R_1,R_2\}$ 时,
	\[(f\pm g)(z)=\sum_{n=0}^\infty (a_n\pm b_n)z^n,\quad
	(fg)(z)=\sum_{n=0}^\infty\left(\sum_{k=0}^na_kb_{n-k}\right)z^n.\]}
\end{theorem}

当 $f,g$ 的收敛半径相同时, $f\pm g$ 或 $fg$ 的收敛半径可以比 $f,g$ 的大.

\begin{theorem}
	设幂级数
	\[f(z)=\sum_{n=0}^\infty a_nz^n,|z|<R,\]
	设函数 $\varphi(z)$ 在 $|z|<r$ 上解析且 $|\varphi(z)|<R$, 
	{那么当 $|z|<r$ 时,
	\[f[\varphi(z)]=\sum_{n=0}^\infty a_n[\varphi(z)]^n.\]}
\end{theorem}

\begin{theorem}
	设幂级数 $\suml_{n=0}^\infty c_nz^n$ 的收敛半径为 $R$, 则在 $|z|<R$ 上:
	\begin{enumerate}
		\item 它的和函数 $f(z)=\suml_{n=0}^\infty c_nz^n$ 解析,
		\item $f'(z)=\suml_{n=1}^\infty nc_nz^{n-1}$,
		\item $\displaystyle\int_0^zf(z)\diff z=\sum_{n=0}^\infty \frac{c_n}{n+1}z^{n+1}$.
	\end{enumerate}
\end{theorem}

也就是说, \emph{在收敛圆内, 幂级数的和函数解析, 且可以逐项求导, 逐项积分}.

由于和函数在 $|z|>R$ 上没有定义, 因此我们不能谈和函数在 $|z|=R$ 上的解析性.

如果函数 $g(z)$ 在该幂级数收敛的点处和 $f(z)$ 均相同, 则 $g(z)$ \emph{一定在收敛圆周上有奇点}.
这是因为一旦 $g(z)$ 在收敛圆周上处处解析, 该和函数就可以在一个半径更大的圆域上作泰勒展开.

\begin{example}
	把函数 $\dfrac1{z-b}$ 表成形如 $\suml_{n=0}^\infty c_n(z-a)^n$ 的幂级数, 其中 $a\neq b$.
\end{example}

\begin{solution}
	\[\frac1{z-b}=\frac1{(z-a)-(b-a)}
	{=\frac1{a-b}\cdot\frac1{1-\dfrac{z-a}{b-a}}.}\]
	{当 $|z-a|<|b-a|$ 时,$\displaystyle\frac1{z-b}=\frac1{a-b}\sum_{n=0}^\infty\left(\frac{z-a}{b-a}\right)^n$,即
	\[\frac1{z-b}=-\sum_{n=0}^\infty\frac{(z-a)^n}{(b-a)^{n+1}},\quad|z-a|<|b-a|.\]}
\end{solution}

\begin{example}
	求幂级数 $\suml_{n=1}^\infty(2^n-1)z^{n-1}$ 的收敛半径与和函数.
\end{example}

\begin{solution}
	由 $\displaystyle\lim_{n\to\infty}\abs{\frac{c_{n+1}}{c_n}}=\lim_{n\to\infty}\frac{2^{n+1}-1}{2^n-1}=2$ 可知收敛半径为 $\dfrac12$.
	{当 $|z|<\dfrac12$ 时, $|2z|<1$.}
	{从而
	\begin{align*}
	\sum_{n=1}^\infty(2^n-1)z^{n-1}&=\sum_{n=1}^\infty 2^n z^{n-1}-\sum_{n=1}^\infty z^{n-1}\\
	&{=\frac2{1-2z}-\frac1{1-z}=\frac1{(1-2z)(1-z)}.}
	\end{align*}}
\end{solution}

\begin{example}
	求幂级数 $\suml_{n=0}^\infty(n+1)z^n$ 的收敛半径与和函数.
\end{example}

\begin{solution}
	由 $\displaystyle\lim_{n\to\infty}\abs{\frac{c_{n+1}}{c_n}}=\lim_{n\to\infty}\frac{n+1}n=1$ 可知收敛半径为 $1$.
{当 $|z|<1$ 时,
	\[\int_0^z\sum_{n=0}^\infty(n+1)z^n\diff z=\sum_{n=0}^\infty z^{n+1}=\frac z{1-z}=-1-\frac1{z-1},\]
}{因此
	\[\sum_{n=0}^\infty(n+1)z^n=\left(-\frac1{z-1}\right)'=\frac1{(z-1)^2},\quad |z|<1.\]}
\end{solution}

通过对
\[1+\lambda z+\lambda^2 z^2+\cdots=\dfrac1{1-\lambda z}\]
两边求 $k$ 阶导数可得
\[\sum_{n=0}^{\infty}(n+k)\cdots(n+2)(n+1)\lambda^n z^n=\frac{k!}{(1-\lambda z)^{k+1}}.\]
因此如果 $p(n)$ 是次数为 $m-1$ 的多项式, 那么
\[\sum_{n=0}^\infty p(n)\lambda^n z^n=\frac{P(z)}{(1-\lambda z)^{m}},\]
其中 $P$ 是多项式.

一般地, 如果幂级数的系数形如
\[c_n=p_1(n)\lambda_1^n+\cdots+p_k(n)\lambda_k^n,\]
则和函数一定是形如
\[\sum_{n=0}^{\infty}c_nz^n
=\frac{P(z)}{(1-\lambda_1z)^{m_1}\cdots(1-\lambda_kz)^{m_k}}\]
的有理函数,	其中 $m_j=\deg p_j+1$.
反过来这样的分式展开成幂级数的系数也一定有上述形式, 至多有有限多项例外.
这可以帮助我们进行计算的验证.
\begin{exercise}
	求幂级数 $\suml_{n=1}^\infty\dfrac{z^n}n$ 的收敛半径与和函数.
\end{exercise}

\begin{example}
	求 $\displaystyle\oint_{|z|=\half }\left(\sum_{n=-1}^\infty z^n\right)\diff z$.
\end{example}

\begin{solution}
	由于 $\suml_{n=0}^\infty z^n$ 在 $|z|<1$ 收敛,
{它的和函数解析.
}{因此
	\begin{align*}
	\oint_{|z|=\half }\left(\sum_{n=-1}^\infty z^n\right)\diff z
	&=\oint_{|z|=\half }\frac1z\diff z+\oint_{|z|=\half }\left(\sum_{n=0}^\infty z^n\right)\diff z\\
	&{=2\pi i+0=2\pi i.}
\end{align*}}
\end{solution}

\section{泰勒级数}

\subsection{泰勒展开的形式与性质}

我们知道, 幂级数在它的收敛域内的和函数是一个解析函数.
反过来, 解析函数是不是也一定可以在一点展开成幂级数呢? 也就是说是否存在\emph{泰勒级数}展开?

在实变函数中我们知道, 一个函数即使在一点附近无限次可导, 它的泰勒级数也未必收敛到原函数.
例如
\[f(x)=\begin{cases}
e^{-x^{-2}},&x\neq 0,\\
0,&x=0.\end{cases}\]
它处处可导, 但是它在 $0$ 处的各阶导数都是 $0$.
因此它的泰勒级数是 $0$, 余项恒为 $f(x)$.
除 $0$ 外它的泰勒级数均不收敛到原函数.

而即使是泰勒级数能收敛到原函数的情形, 它成立的范围也很难从函数本身读出.
例如
\[\dfrac1{1+x}=1-x+x^2-x^3+\cdots,\quad|x|<1.\]
这可以从 $x=-1$ 是奇点看出.
而
\[\dfrac1{1+x^2}=1-x^2+x^4-x^6+\cdots,\quad|x|<1\]
却并没有奇点.
为什么它的麦克劳林级数成立的开区间也是 $(-1,1)$?
这个问题在本节可以得到回答.

设函数 $f(z)$ 在区域 $D$ 解析, $z_0\in D$.
设 $|z-z_0|$ 小于 $z_0$ 到 $D$ 边界的距离 $d$,
则存在 $|z-z_0|<r<d$.
设 $K:|\zeta-z_0|=r$, 则 $K$ 和它的内部包含在 $D$ 中.
由于 $\abs{\dfrac{z-z_0}{\zeta-z_0}}<1$, 因此
\[\frac1{\zeta-z}=\frac1{\zeta-z_0}\cdot\frac1{1-\dfrac{z-z_0}{\zeta-z_0}}=\sum_{n=0}^\infty\frac{(z-z_0)^n}{(\zeta-z_0)^{n+1}}.\]


\begin{center}
	\begin{tikzpicture}
		\filldraw[cstcurve,main,smooth,cstfill2] plot coordinates {(-2.25,0) (-1.5,-0.75) (0,-1.5) (1.05,-1.5) (1.35,0) (0,1.2) (-1.5,1.2) (-2.25,0)};
		\draw[cstdash,third] (0,0) circle (1);
		\draw[cstcurve,second] (0,0) circle (0.8);
		\fill[cstdot,main] (0,0) circle;
		\fill[cstdot,third] (0.4,0.4) circle;
		\fill[cstdot] (0.6,-0.5) circle;
		\draw[cstcurve,cstra,main] (0,0)--(-0.48,-0.64);
		\draw
			(-0.25,0) node[main] {$z_0$}
			(0.2,0.4) node[third] {$z$}
			(0,-0.5) node[main] {$r$}
			(0.4,-0.3) node {$\zeta$};
	\end{tikzpicture}
\end{center}

故
\begin{align*}
	f(z)&=\frac1{2\pi i}\oint_K \frac{f(\zeta)}{\zeta-z}\diff\zeta
	{=\frac1{2\pi i}\oint_K f(\zeta)\sum_{n=0}^\infty\frac{(z-z_0)^n}{(\zeta-z_0)^{n+1}}\diff\zeta}\\
	&{=
	\sum_{n=0}^{N-1}\left[\frac1{2\pi i}\oint_K\frac{f(\zeta)\diff\zeta}{(\zeta-z_0)^{n+1}}\right](z-z_0)^n+R_N(z),}\\
	&{=
	\sum_{n=0}^{N-1}\frac{f^{(n)}(z_0)}{n!}(z-z_0)^n+R_N(z),}
\end{align*}
其中
\begin{align*}
	R_N(z)&=\frac1{2\pi i}\oint_Kf(\zeta)\left[\sum_{n=N}^\infty\frac{(z-z_0)^n}{(\zeta-z_0)^{n+1}}\right]\diff\zeta\\
	&=\frac1{2\pi i}\oint_K\frac{f(\zeta)}{\zeta-z}\cdot\left(\frac{z-z_0}{\zeta-z_0}\right)^N\diff\zeta.
\end{align*}

由于 $f(\zeta)$ 在 $D\supseteq K$ 上解析, 从而在 $K$ 上连续且有界.
设 $|f(\zeta)|\le M,\zeta\in K$,
那么
\begin{align*}
	|R_N(z)|&\le\frac 1{2\pi}\oint_K\abs{\frac{f(\zeta)}{\zeta-z}\cdot\left(\frac{z-z_0}{\zeta-z_0}\right)^N}\diff s\\
	&{\le\frac 1{2\pi}\cdot \frac M{r-|z-z_0|}\cdot\abs{\frac{z-z_0}{\zeta-z_0}}^N\cdot 2\pi r\to 0\quad (N\to\infty).}
\end{align*}
故
\[f(z)=\sum_{n=0}^\infty\frac{f^{(n)}(z_0)}{n!}(z-z_0)^n,\quad |z-z_0|<d.\]

由于幂级数在收敛半径内的和函数是解析的, 因此解析函数的泰勒展开成立的圆域不包含奇点.
由此可知, 解析函数在 $z_0$ 处\emph{泰勒展开成立的圆域的最大半径是 $z_0$ 到最近奇点的距离}.

需要注意的是, 泰勒级数的收敛半径是有可能比这个半径更大的,
而且泰勒展开等式也可能在这个圆域之外的点成立.
例如 
\[f(z)=\begin{cases}
	e^z,&z\neq 1;\\
	0,&z=1
\end{cases}\]
的麦克劳林展开为 $f(z)=\suml_{n=0}^\infty \dfrac{z^n}{n!},\quad |z|<1$.

现在我们来看 $f(z)=\dfrac1{1+z^2}$.
它的奇点为 $\pm i$, 所以它的麦克劳林展开成立的半径是 $1$.
这就解释了为什么函数 $f(x)=\dfrac1{1+x^2}$ 的麦克劳林展开成立的开区间是 $(-1,1)$.

\subsection{泰勒展开的计算方法}

若 $f(z)$ 在 $z_0$ 附近展开为 $\suml_{n=0}^\infty c_n(z-z_0)^n$,
则由幂级数的逐项求导性质可知
\[f^{(n)}(z_0)=\sum_{k=n}^\infty c_k k(k-1)\cdots(k-n+1)(z-z_0)^{k-n}\Big|_{z=z_0}=n!c_n.\]
所以\emph{解析函数的幂级数展开是唯一的}.

因此解析函数的泰勒展开不仅可以\emph{直接求出各阶导数得到}, 也可以\emph{利用幂级数的运算法则得到}.

\begin{example}
	由于 $(e^z)^{(n)}(0)=e^z|_{z=0}=1$, 因此
	\[e^z=1+z+\frac{z^2}{2!}+\frac{z^3}{3!}+\cdots=\sum_{n=0}^\infty\frac{z^n}{n!},\quad\forall z.\]
\end{example}

\begin{example}
	由于 $\displaystyle(\cos z)^{(n)}=\cos\left(z+\dfrac{n\pi}2\right)$,
		\[(\cos z)^{(2n+1)}(0)=0,\quad (\cos z)^{(2n)}(0)=(-1)^n,\]
	因此
		\[\cos z=1-\frac{z^2}{2!}+\frac{z^4}{4!}-\frac{z^6}{6!}+\cdots=\sum_{n=0}^\infty(-1)^n\frac{z^{2n}}{(2n)!},\quad\forall z.\]
\end{example}

\begin{example}
	由 $e^z$ 的泰勒展开可得
	\begin{align*}
		\sin z&=\frac{e^{iz}-e^{-iz}}{2i}
		{=\sum_{n=0}^\infty\frac{(iz)^n-(-iz)^n}{2i\cdot n!}}\\
		&=z-\frac{z^3}{3!}+\frac{z^5}{5!}-\cdots
		=\sum_{n=0}^\infty(-1)^n\frac{z^{2n+1}}{(2n+1)!},\quad\forall z.
	\end{align*}
		这里, 因为 $\sin z$ 是奇函数, 所以它的麦克劳林展开只有奇数幂次项, 没有偶数幂次项.
\end{example}

\begin{example}
求对数函数的主值 $\ln(1+z)$ 的麦克劳林展开.
\end{example}
\begin{solution}
由于 $\ln(1+z)$ 在去掉射线 $z=x\le-1$ 的区域内解析,
{因此它在 $|z|<1$ 内解析,
}{且
	\[[\ln(1+z)]'=\frac1{1+z}=\sum_{n=0}^\infty(-1)^nz^n,\quad|z|<1.\]
}{逐项积分得到
\begin{align*}
\ln(1+z)&=\int_0^z\frac1{1+\zeta}\diff\zeta
	{=\int_0^z\sum_{n=0}^\infty(-1)^n\zeta^n\diff\zeta}\\
&{=\sum_{n=0}^\infty\frac{(-1)^nz^{n+1}}{n+1}}
	{=\sum_{n=1}^\infty\frac{(-1)^{n+1}z^n}{n},\quad|z|<1.}
\end{align*}}
\end{solution}

\begin{example}
	函数 $f(z)=(1+z)^\alpha$ 的主值为 $\exp\bigl[\alpha\ln(1+z)\bigr]$. 它在去掉射线 $z=x\le -1$ 的区域内解析. 由于
	\begin{align*}
		f^{(n)}(0)&=\alpha(\alpha-1)\cdots(\alpha-n+1)\exp\bigl[(\alpha-n)\ln(1+z)\bigr]\Big|_{z=0}\\
		&=\alpha(\alpha-1)\cdots(\alpha-n+1).
	\end{align*}
	因此
	\begin{align*}
		(1+z)^\alpha&=1+\alpha z+\frac{\alpha(\alpha-1)}2z^2+\frac{\alpha(\alpha-1)(\alpha-2)}{3!}z^3+\cdots\\
		&=\sum_{n=0}^\infty\frac{\alpha(\alpha-1)\cdots(\alpha-n+1)}{n!}z^n,\quad |z|<1.
	\end{align*}
\end{example}

当 $\alpha=n$ 是正整数时, 上述麦克劳林展开的 $>n$ 幂次项系数为零, 
从而
	\[(1+z)^n=\sum_{k=0}^n \mathrm{C}_n^k z^k,\]
此即牛顿二项式展开.

\begin{example}
	将 $\dfrac1{(1+z)^2}$ 展开成 $z$ 的幂级数.
\end{example}

\begin{solution}
	由幂函数展开可知当 $|z|<1$ 时,
	\[(1+z)^{-2}=\sum_{n=0}^\infty \frac{(-2)(-3)\cdots(-1-n)}{n!}z^n
	{=\sum_{n=0}^\infty (-1)^n(n+1)z^n.}\]
\end{solution}

\begin{solution}{另解}
	由于 $\dfrac1{(1+z)^2}$ 的奇点为 $z=-1$, 因此它在 $|z|<1$ 内解析.
	{由于
		\[\frac1{1+z}=1-z+z^2-z^3+\cdots=\sum_{n=0}^\infty (-1)^nz^n,\]因此
		\[
		\frac1{(1+z)^2}=-\left(\frac1{1+z}\right)'
		{=-\sum_{n=1}^\infty(-1)^n nz^{n-1}}
		{=\sum_{n=0}^\infty(-1)^n (n+1)z^n,\quad |z|<1.}
		\]
	}
\end{solution}
一般地, 我们有
\[\frac1{(1-\lambda z)^k}=\sum_{n=0}^\infty(n+k-1)\cdots(n+2)(n+1)z^n,\ |z|<\frac1{|\lambda|}.\]

\begin{example}
	求 $\dfrac1{3z-2}$ 的麦克劳林展开.
\end{example}

\begin{solution}
	由于 $\dfrac1{3z-2}$ 的奇点为 $z=\dfrac23$, 因此它在 $|z|<\dfrac23$ 内解析.
	{此时
	\begin{align*}
		\frac1{3z-2}&=-\half\cdot\frac1{1-\dfrac{3z}2}
			{=-\half\sum_{n=0}^\infty\left(\frac{3z}2\right)^n}\\
		&{=-\sum_{n=0}^\infty\frac{3^n}{2^{n+1}}z^n,\quad|z|<\frac23.}
	\end{align*}}
\end{solution}

\begin{exercise}
	求 $\dfrac1{1-3z+2z^2}$ 的麦克劳林展开.
\end{exercise}

有理函数展开为真分式形式若用待定系数法总略显繁琐, 我们现介绍一种方法.
设
\[f(z)=\dfrac{1}{(z-1)(z-2)(z+2)}=\frac{a}{z-1}+\frac{b}{z-2}+\frac{c}{z+2},\]
\[a=\frac1{2\pi i} \oint_{|z-1|=0.1}f(z)\diff z=\dfrac{1}{(z-2)(z+2)}\Big|_{z=1}=-\frac13.\]
类似可得 $b=\dfrac14,c=\dfrac1{12}$.

对于分母有重根的情形, 例如
\[f(z)=\dfrac{1}{(z-1)^2(z-2)^3}=\frac{a}{z-1}+\frac{b}{(z-1)^2}+\frac{c}{z-2}+\frac{d}{(z-2)^2}+\frac{e}{(z-2)^3},\]
\[a=\frac1{2\pi i} \oint_{|z-1|=0.1}f(z)\diff z=\left(\dfrac{1}{(z-2)^3}\right)'\Big|_{z=1}=-3,\]
\[b=\frac1{2\pi i} \oint_{|z-1|=0.1}(z-1)f(z)\diff z=\dfrac{1}{(z-2)^3}\Big|_{z=1}=-1.\]
类似可得 $c=3,d=-2,e=1$, 只是我们需要计算高阶导数.
因此
\[f(z)=\sum_{n=0}^\infty\left(3-(n+1)+\frac{-\frac32-\frac12(n+1)-\frac18(n+2)(n+1)}{2^n}\right)z^n,\quad |z|<1.\]

\section{洛朗级数}

\subsection{双边幂级数}

如果解析函数 $f(z)$ 在 $z_0$ 处解析, 那么在 $z_0$ 处可以展开成泰勒级数.
如果 $f(z)$ 在 $z_0$ 处不解析呢?
此时 $f(z)$ 一定不能展开成 $z-z_0$ 的幂级数,
然而它却可能可以展开为\emph{双边幂级数}
	\[\sum_{n=-\infty}^\infty c_n(z-z_0)^n
		=\underbrace{\sum_{n=1}^\infty c_{-n}(z-z_0)^{-n}}_{\text{\normalsize 负幂次部分}}
		+\underbrace{\sum_{n=0}^\infty c_n(z-z_0)^n}_{\text{\normalsize 非负幂次部分}}.\]
例如
	\[\frac1{z^2(1-z)}=\frac1{z^2}+\frac1z+1+z+z^2+\cdots,\quad 0<|z|<1.\]

为了保证双边幂级数的收敛范围有一个好的性质以便于我们使用, 我们对它的敛散性作如下定义:
\begin{definition}
	如果双边幂级数的非负幂次部分和负幂次部分作为函数项级数都收敛, 则我们称这个双边幂级数\emph{收敛}. 否则我们称之为\emph{发散}.
\end{definition}

注意双边幂级数的敛散性不能像幂级数那样通过部分和形成的数列的极限来定义,
因为使用不同的部分和选取方式会影响到极限的数值.

设 $\suml_{n=0}^\infty c_n(z-z_0)^n$ 的收敛半径为 $R_2$, 则它在 $|z-z_0|<R_2$ 内收敛, 在 $|z-z_0|>R_2$ 内发散.

对于负幂次部分, 令 $\zeta=\frac1{z-z_0}$, 那么负幂次部分是 $\zeta$ 的一个幂级数 $\suml_{n=1}^\infty c_{-n}\zeta^n$.
设该幂级数的收敛半径为 $R$, 则它在 $|\zeta|<R$ 内收敛, 在 $|\zeta|>R$ 内发散.
设 $R_1:=\frac1R$, 则 $\suml_{n=1}^\infty c_{-n}(z-z_0)^{-n}$ 在 $|z-z_0|>R_1$ 内收敛, 在 $|z-z_0|<R_1$ 内发散.

\begin{enumerate}
	\item 如果 $R_1>R_2$, 则该双边幂级数处处不收敛.
	\item 如果 $R_1=R_2$, 则该双边幂级数只在圆周 $|z-z_0|=R_1$ 上可能有收敛的点.
		此时没有收敛域.
	\item 如果 $R_1<R_2$, 则该双边幂级数在 $R_1<|z-z_0|<R_2$ 内收敛, 在 $|z-z_0|<R_1$ 或 $>R_2$ 内发散, 在圆周 $|z-z_0|=R_1$ 或 $R_2$ 上既可能发散也可能收敛.
\end{enumerate}

因此\emph{双边幂级数的收敛域为圆环域 $R_1<|z-z_0|<R_2$}.

当 $R_1=0$ 或 $R_2=+\infty$ 时, 圆环域的形状会有所不同.
\begin{center}
	\begin{tikzpicture}
		\filldraw[cstcurve,draw=main,cstfill2] (0,0) circle (1.5);
		\filldraw[cstdote,draw=main] (0,0) circle;
		\draw[second,cstra,thick] (0.048,0.036)--(1.2,0.9);
		\draw
			(0.7,0.2) node[second] {$R_2$}
			(-0.3,0) node[main] {$z_0$}
			(0,-2) node {$0<|z-z_0|<R_2$};

		\fill[cstfille2] (2.5,-1.5) rectangle (5.5,1.5);
		\filldraw[cstcurve,fill=white,draw=main] (4,0) circle (1); 
		\fill[cstdot,second] (4,0) circle;
		\draw[second,cstra,thick] (4,0)--(4.6,-0.8);
		\draw
			(4.1,-0.6) node[second] {$R_1$}
			(3.7,0) node[main] {$z_0$}
			(4,-2) node {$R_1<|z-z_0|<+\infty$};

		\fill[cstfille2] (6.5,-1.5) rectangle (9.5,1.5);
		\filldraw[cstdote,draw=main] (8,0) circle;
		\draw
			(7.7,0) node[main] {$z_0$}
			(8,-2) node {$0<|z-z_0|<+\infty$};
	\end{tikzpicture}
\end{center}

双边幂级数的非负幂次部分和负幂次部分在收敛圆环域内都收敛,
因此它们的和函数都解析($\zeta=\dfrac1{z-z_0}$ 关于 $z$ 解析), 且可以逐项求导、逐项积分.
从而\emph{双边幂级数的和函数也是解析的, 且可以逐项求导、逐项积分}.

\begin{example}
	求双边幂级数 $\displaystyle\sum_{n=1}^\infty\frac{2^n}{z^n}+\sum_{n=0}^\infty\frac{z^n}{(2+i)^n}$ 的收敛域与和函数.
\end{example}

\begin{solution}
	非负幂次部分收敛域为 $|z|<|2+i|=\sqrt5$, 负幂次部分收敛域为 $|z|>|2|=2$.
	{因此该双边幂级数的收敛域为 $2<|z|<\sqrt5$.此时
		\[\sum_{n=1}^\infty\frac{2^n}{z^n}+\sum_{n=0}^\infty\frac{z^n}{(2+i)^n}
		=\frac{\dfrac 2z}{1-\dfrac 2z}+\frac1{1-\dfrac z{2+i}}
		=\frac{-iz}{(z-2)(z-2-i)}.\]}
\end{solution}

\subsection{洛朗展开的形式}

反过来, 在圆环域内解析的函数也一定能展开为双边幂级数, 被称为\emph{洛朗级数}.

例如 $f(z)=\dfrac1{z(1-z)}$ 在 $z=0,1$ 以外解析.
在圆环域 $0<|z|<1$ 内,
\[f(z)=\frac1z+\frac1{1-z}=\frac1z+1+z+z^2+z^3+\cdots\]
在圆环域 $1<|z|<+\infty$ 内,
\[f(z)=\frac1z-\frac1z\cdot\frac1{1-\dfrac1z}=-\frac1{z^2}-\frac1{z^3}-\frac1{z^4}-\cdots\]

现在我们来证明洛朗级数的存在性并得到洛朗展开式.
设 $f(z)$ 在圆环域 $R_1<|z-z_0|<R_2$ 内解析.
设
	\[K_1:|z-z_0|=r,\quad K_2:|z-z_0|=R,\quad R_1<r<R<R_2.\]
是该圆环域内的两个圆周. 
对于 $r<|z-z_0|<R$, 由柯西积分公式,
\[f(z)=\frac1{2\pi i}
	\oint_{K_2}\frac{f(\zeta)}{\zeta-z}\diff\zeta
	-\frac1{2\pi i}\oint_{K_1}\frac{f(\zeta)}{\zeta-z}\diff\zeta.\]

\begin{center}
	\begin{tikzpicture}
		\filldraw[cstcurve,second,cstfill2] (0,0) circle (1.5);
		\filldraw[cstcurve,main,fill=white] (0,0) circle (1);
		\fill[cstdot,main] (-1,0) circle;
		\fill[cstdot,second] (-1.48,-0.3) circle;
		\fill[cstdot,black] (0,1.25) circle;
		\draw[second,cstra,thick] (0,0)--(1.2,0.9);
		\draw[main,cstra,thick] (0,0)--(0.6,-0.8);
		\draw
			(-1.25,-0.15) node {$\zeta$}
			(0.5,0.1) node[second] {$R$}
			(0.1,-0.5) node[main] {$r$}
			(-0.3,0) node {$z_0$}
			(0.8,-1) node[main] {$K_1$}
			(1.4,1.2) node[second] {$K_2$}
			(-0.3,1.25) node {$z$};
		\fill[cstdot,black] (0,0) circle;
	\end{tikzpicture}
\end{center}

和泰勒级数的推导类似,
	\[\frac1{2\pi i}\oint_{K_2}\frac{f(\zeta)}{\zeta-z}\diff\zeta
	=\sum_{n=0}^\infty\left[\frac1{2\pi i}\oint_{K_2}\frac{f(\zeta)\diff\zeta}{(\zeta-z_0)^{n+1}}\right](z-z_0)^n\]
可以表达为幂级数的形式.
对于 $\zeta\in K_1$, 由 $\abs{\dfrac{\zeta-z_0}{z-z_0}}<1$ 可得
	\[-\frac1{\zeta-z}=\frac1{z-z_0}\cdot\frac1{1-\dfrac{\zeta-z_0}{z-z_0}}=\sum_{n=1}^\infty\frac{(z-z_0)^{-n}}{(\zeta-z_0)^{-n+1}},\]
	\[-\frac1{2\pi i}\oint_{K_1}\frac{f(\zeta)}{\zeta-z}\diff\zeta=\frac1{2\pi i}\oint_{K_1}f(\zeta)\sum_{n=1}^\infty\frac{(z-z_0)^{-n}}{(\zeta-z_0)^{-n+1}}\diff\zeta.\]

令
\begin{align*}
	R_N(z)&=\frac1{2\pi i}\oint_{K_1}f(\zeta)\sum_{n=N}^\infty\frac{(z-z_0)^{-n}}{(\zeta-z_0)^{-n+1}}\diff\zeta\\
	&=\frac1{2\pi i}\oint_{K_1}\frac{f(\zeta)}{z-\zeta}\cdot\left(\frac{\zeta-z_0}{z-z_0}\right)^{N-1}\diff\zeta.
\end{align*}
由于 $f(\zeta)$ 在 $D\supseteq K_1$ 上解析, 从而在 $K_1$ 上连续且有界.
设 $|f(\zeta)|\le M,\zeta\in K_1$,
那么
\begin{align*}
|R_N(z)|&\le\frac 1{2\pi}\oint_{K_1}\abs{\frac{f(\zeta)}{z-\zeta}\cdot\left(\frac{\zeta-z_0}{z-z_0}\right)^{N-1}}\diff s\\
&{\le\frac 1{2\pi}\cdot\frac M{|z-z_0|-r}\cdot\abs{\frac{\zeta-z_0}{z-z_0}}^{N-1}\cdot 2\pi r\to 0\quad (N\to\infty).}
\end{align*}

故
\begin{align*}
f(z)&=\sum_{n=0}^\infty\left[\frac1{2\pi i}\oint_{K_2}\frac{f(\zeta)\diff\zeta}{(\zeta-z_0)^{n+1}}\right](z-z_0)^n\\
&\qquad+\sum_{n=1}^\infty \left[\frac1{2\pi i}\oint_{K_1}\frac{f(\zeta)\diff\zeta}{(\zeta-z_0)^{-n+1}}\right](z-z_0)^{-n},
\end{align*}
其中 $r<|z-z_0|<R$.
由复合闭路定理, $K_1,K_2$ 可以换成任意一条在圆环域内绕 $z_0$ 的闭路 $C$.
从而我们得到 \emph{$f(z)$ 在以 $z_0$ 为圆心的圆环域的洛朗展开}
	\[f(z)=\sum_{n=-\infty}^\infty\left[\frac1{2\pi i}\oint_C\frac{f(\zeta)\diff\zeta}{(\zeta-z_0)^{n+1}}\right](z-z_0)^n,\]
其中 $R_1<|z-z_0|<R_2$.

\subsection{洛朗展开的计算方法}

我们称 $f(z)$ 洛朗展开的非负幂次部分为它的\emph{解析部分}, 负幂次部分为它的\emph{主要部分}.

设在圆环域 $R_1<|z-z_0|<R_2$ 内的解析函数 $f(z)$ 可以表达为双边幂级数
\[f(z)=\sum_{n=-\infty}^\infty c_n(z-z_0)^n,\]
则逐项积分得到
\[\oint_C\frac{f(\zeta)\diff\zeta}{(\zeta-z_0)^{n+1}}=\sum_{k=-\infty}^\infty c_k\oint_C(\zeta-z_0)^{k-n-1}\diff\zeta=2\pi i c_n.\]
因此 $f(z)$ 在圆环域内的\emph{双边幂级数展开是唯一的, 它就是洛朗级数}.

\begin{example}
	将 $f(z)=\dfrac{e^z-1}{z^2}$ 展开为以 $0$ 为中心的洛朗级数.
\end{example}

由洛朗级数的唯一性, 我们可以从 $e^z$ 的泰勒展开通过代数运算来得到洛朗级数.
这种做法比直接计算积分更简便.
因此我们一般\emph{不用直接法}, 而是\emph{用双边幂级数的代数、求导、求积分运算}来得到洛朗级数.

\begin{solution}
	\[\frac{e^z-1}{z^2}=\frac1{z^2}\left(z+\frac{z^2}{2!}+\frac{z^3}{3!}+\cdots\right)
	{=\frac1z+\sum_{n=0}^\infty \frac1{(n+2)!}z^n,}\]
	{其中 $0<|z|<+\infty$.}
\end{solution}

\begin{example}
	在下列圆环域中把 $f(z)=\dfrac1{(z-1)(z-2)}$ 展开为洛朗级数.

	{\enumnum1 $0<|z|<1$, \enumnum2 $1<|z|<2$, \enumnum3 $2<|z|<+\infty$.}
\end{example}

\begin{solution}
	由于 $f(z)$ 的奇点为 $z=1,2$, 因此在这些圆环域内 $f(z)$ 都可以展开为洛朗级数.
	{注意到
		\[f(z)=\frac1{z-2}-\frac1{z-1},\]
		因此我们可以根据 $|z|$ 的范围来将其展开成等比级数.}

	\enumnum1 由于 $|z|<1,\abs{\dfrac z2}<1$,
	{因此
		\begin{align*}
			f(z)&=-\frac1{2-z}+\frac1{1-z}
			{=-\half\cdot\frac1{1-\dfrac z2}+\frac1{1-z}}\\
			&{=-\half\sum_{n=0}^\infty\left(\frac z2\right)^n+\sum_{n=0}^\infty z^n}
			{=\sum_{n=0}^\infty\left(1-\frac1{2^{n+1}}\right)z^n}\\
			&{=\half +\frac34z+\frac78z^2+\cdots}
		\end{align*}}

	\enumnum2 由于 $\abs{\dfrac1z}<1,\abs{\dfrac z2}<1$, 
	{因此
		\begin{align*}
			f(z)&=\frac1{1-z}-\frac1{2-z}
			{=-\frac1z\cdot\frac1{1-\dfrac1z}-\half\cdot\frac1{1-\dfrac z2}}\\
			&{=-\frac1z\sum_{n=0}^\infty \left(\frac1z\right)^n-\half\sum_{n=0}^\infty\left(\frac z2\right)^n}
			{=-\sum_{n=1}^{\infty}\frac1{z^n}-\sum_{n=0}^\infty\frac1{2^{n+1}}z^n}\\
			&{=\cdots-\frac1{z^2}-\frac1z-\half -\frac14z-\frac18z^2-\cdots}
		\end{align*}}

	\enumnum3 由于 $\abs{\dfrac1z}<1,\abs{\dfrac2z}<1$, {因此
		\begin{align*}
			f(z)&=\frac1{1-z}-\frac1{2-z}
				{=-\frac1z\cdot\frac1{1-\dfrac1z}+\frac1z\cdot\frac1{1-\dfrac2z}}\\
			&{=-\frac1z\sum_{n=0}^\infty \left(\frac1z\right)^n+\frac1z\sum_{n=0}^\infty\left(\frac2z\right)^n}
				{=\sum_{n=0}^\infty\frac{2^n-1}{z^{n+1}}}\\
			&{=\frac1{z^2}+\frac3{z^3}+\frac7{z^4}+\cdots}
		\end{align*}}
\end{solution}

洛朗展开的一些特点可以帮助我们检验计算的正确性.
\begin{itemize}
	\item  若 $f(z)$ 在 $|z-z_0|<R_2$ 内\emph{解析},
		则 $f(z)$ 可以展开为泰勒级数.
		由唯一性可知泰勒级数等于洛朗级数,
		因此\emph{此时洛朗展开一定没有负幂次项}.
	\item 若 $f(z)$ 在圆周 $|z-z_0|=R_1,R_2>0$ 上\emph{有奇点}, 则在圆环域 $R_1<|z-z_0|<R_2$ 上的洛朗展开\emph{一定有无穷多负幂次和无穷多正幂次项}.
	\item 有理函数在 $0<|z-z_0|<r$ 洛朗展开\emph{最多只有有限多负幂次项}, 在 $R<|z-z_0|<+\infty$ 洛朗展开\emph{最多只有有限多正幂次项}.
\end{itemize}

如果有理函数 $f(z)$ 在圆环域 $r<|z|<R$ 内解析, 那么它的洛朗展开一定形如
\[f(z)=P(z)+\sum_{n\ge 0}a_n z^n+\sum_{n<0}b_n z^n,\]
其中 $P(z)$ 只有有限多项, 
$a_n$ 是形如 $p(n)\lambda^{-n}$ 的线性组合, $|\lambda|\ge R$ 是奇点, $\deg p+1$ 是 $\lambda$ 在 $f(z)$ 出现的重数; 而 $b_n$ 则是 $|\lambda|\le r$ 的那个奇点对应的组合.

不仅如此, 在不同的圆环域不同圆环域上的洛朗展开{\itshape 形式地相减}, 系数会有\emph{共同的通项形式}.

例如在 $0<|z|<1$ 内,
\[f(z)=\frac{120}{(z-1)(z^2-4)(z^2-9)}=\sum_{n\ge 0}\left(-5+\frac2{(-2)^{n+1}}+\frac6{2^{n+1}}-\frac1{(-3)^{n+1}}-\frac2{3^{n+1}}\right)z^n.\]
那么在 $2<|z|<3$ 内洛朗展开的系数就是上面的每个系数(不论正负 $n$)减去一个共同形式的项, 使得其非负幂次通项对应那些奇点 $|\lambda|\ge 3$, 而负幂次通项对应那些奇点 $|\lambda|\le2$.
故
\[f(z)=\sum_{n\ge 0}\left(-\frac1{(-3)^{n+1}}-\frac2{3^{n+1}}\right)z^n+\sum_{n\le-1}\left(5-\frac2{(-2)^{n+1}}-\frac6{2^{n+1}}\right)z^n.\]
其证明可见\emph{\href{https://zhangshenxing.gitee.io/teaching/publications/袁志杰张神星2023 复变函数在不同圆环域内洛朗展开的联系.pdf}{复变函数在不同圆环域内洛朗展开的联系}}一文.

\begin{example}
	将函数 $f(z)=\dfrac{z+1}{(z-1)^2}$ 在圆环域 $0<|z|<1$ 内展开成洛朗级数.
\end{example}

\begin{solution}
	\[f(z)=\frac{z-1+2}{(z-1)^2}=\frac1{z-1}+\frac{2}{(z-1)^2}=-\frac1{1-z}+2\left(\frac1{1-z}\right)'\]
	{因此当 $0<|z|<1$ 时,
		\[f(z)=-\sum_{n=0}^\infty z^n+2\left(\sum_{n=0}^\infty z^n\right)'=-\sum_{n=0}^\infty z^n+2\sum_{n=1}^\infty nz^{n-1}
		=\sum_{n=0}^\infty(2n+1)z^n.\]}
\end{solution}

\begin{exercise}
	将函数 $f(z)=\dfrac{z+1}{(z-1)^2}$ 在圆环域 $1<|z|<+\infty$ 内展开成洛朗级数.
\end{exercise}

利用幂函数的泰勒展开可以得到
\[\frac1{(1-\lambda z)^k}=-\sum_{n<0}(n+k-1)\cdots(n+2)(n+1)z^n,\ |z|>\frac1{|\lambda|}.\]
实际上求和范围可以改成 $n\le -k$.

注意到当 $n=-1$ 时, 洛朗级数的系数
\[c_{-1}=\frac1{2\pi i}\oint_C f(\zeta)\diff\zeta,\]
因此洛朗展开可以用来帮助计算函数的积分,
这便引出了\emph{留数}的概念.


\begin{example}
	求 $\displaystyle\oint_{|z|=3}\frac1{z(z+1)^2}\diff z$.
\end{example}

\begin{solution}
	注意到闭路 $|z|=3$ 落在 $1<|z+1|<+\infty$ 内.
	{我们在这个圆环域内求 $f(z)=\dfrac1{z(z+1)^2}$ 的洛朗展开.
		\begin{align*}
		f(z)&=\frac1{z(z+1)^2}=\frac1{(z+1)^3}\cdot\frac1{1-\dfrac1{z+1}}\\
		&{=\frac1{(z+1)^3}\sum_{n=0}^\infty\frac1{(z+1)^n}}
		{=\sum_{n=0}^{\infty}\frac1{(z+1)^{n+3}}}
		\end{align*}故
		$\displaystyle\oint_Cf(z)\diff z=2\pi i c_{-1}=0$.}
\end{solution}
% 

\begin{example}
	求 $\displaystyle\oint_{|z|=1}\frac{z}{\sin z^2}\diff z$.
\end{example}

\begin{solution}
	注意到闭路 $|z|=1$ 落在 $0<|z|<\sqrt \pi$ 内.
	{我们在这个圆环域内求 $f(z)=\dfrac{z}{\sin z^2}$ 的洛朗展开.
		\[
		f(z)=\frac{z}{\sin z^2}=
		\frac{z}{z^2-\dfrac{z^6}{3!}+\dfrac{z^{10}}{5!}+\cdots}
		{=\frac1z+\frac{z^3}6+\cdots}
		\]故
		$\displaystyle\oint_Cf(z)\diff z=2\pi i c_{-1}=2\pi i$.}
\end{solution}




\sectionExerciseAnswer
\exans1} $-\dfrac12+\dfrac i2$.
\exans1} \enumnum1 $0$; \enumnum2 $0$.
\exans5} $v(x,y)=2x^3+3x^2y-6xy^2-y^3+C$.

% \begin{answer}
% 	当且仅当非零的 $z_n$ 的辐角全都相同时成立.
% \end{answer}


% \begin{answer}
% 	实部级数条件收敛, 虚部级数绝对收敛, 所以该级数条件收敛.
% \end{answer}

% \begin{answer}
% 	收敛半径为 $1$, 和函数为 $-\ln(1-z)$.
% \end{answer}

% \begin{answer}
% \[
% 	\frac1{1-3z+2z^2}=\frac{2}{1-2z}-\frac1{1-z}
% 	=\sum_{n=0}^\infty(2^{n+1}-1)z^n,\quad |z|<\half .
% \]
% \end{answer}

% \begin{answer}
% 	\[f(z)=\sum_{n=1}^\infty \frac{2n-1}{z^n}.\]
% \end{answer}