\chapter{练习答案}

\begin{multicols}{2}

\setcounter{chapter}{0}

\begin{exerciseanswer}
  \item $-4$.
  \item $\pm(\sqrt3+\sqrt2\ii)$.
  \item $1$.
  \item 分别是 $-\ov z$ 和 $-z$.
  \item $\dfrac{7-4\ii}5$.
  \item 仅当 $z$ 不是负实数和 $0$ 时它才成立, 即 $\arg z\in(-\cpi,\cpi)$.
  \item $z_1,z_2,\cdots,z_n$ 中的非零元辐角都相等.
  \item $\displaystyle z=2\sqrt3\Bigl(\cos\bigl(-\frac\cpi3\bigr)+\ii\sin\bigl(-\frac\cpi3\bigr)\Bigr)=2\sqrt3\ee^{-\frac{\cpi\ii}3}$, 辐角写成 $\dfrac{5\cpi}3$ 也可以.
  \item $4+\ii$ 或 $2-\ii$.
  \item $2^{12}=4096$.
  \item $\pm\dfrac{\sqrt3+\ii}2,\pm \ii,\pm\dfrac{\sqrt3-\ii}2$.
  \item 双曲线 $x^2-y^2=\dfrac12$ 和双曲线 $xy=\dfrac14$.
  \item \delspace
    \begin{enumnopar}[(i)]
      \item 上半平面对应的闭区域为 $\Im z\ge0$.
      \item 下半平面对应的闭区域为 $\Im z\le0$.
      \item 左半平面对应的闭区域为 $\Re z\le0$.
      \item 右半平面对应的闭区域为 $\Re z\ge0$.
      \item 竖直带状区域对应的闭区域为 $x_1\le\Re z\le x_2$.
      \item 水平带状区域对应的闭区域为 $y_1\le\Im z\le y_2$.
      \item 角形区域对应的闭区域为 $\alpha_1\le \Arg z\le \alpha_2$ 以及原点.
      \item 圆域对应的闭区域为 $|z|\le R$.
      \item 圆环域对应的闭区域为 $r\le|z|\le R$.
    \end{enumnopar}
  \item C
  \item \delspace
  \begin{enumnopar}[(i)]
    \item $\Re z$ 的定义域为 $\BC$, 值域为 $\BR$.
    \item $\arg z$ 的定义域为 $\{z\in\BC\mid z\neq 0\}$, 值域为 $(-\cpi,\cpi]$.
    \item 当 $n>0$ 时, $z^n$ 的定义域为 $\BC$, 值域为 $\BC$.
    当 $n<0$ 时, $z^n$ 的定义域为 $\{z\in\BC\mid z\neq 0\}$, 值域为 $\{z\in\BC\mid z\neq 0\}$.
    \item $\dfrac{z+\ii}{z^2+1}$ 的定义域为 $\{z\in\BC\mid z\neq \pm i\}$, 值域为 $\BC$.
  \end{enumnopar}
  \item $-\dfrac dc$.
  \item 收敛到 $0$.
\end{exerciseanswer}

\begin{exerciseanswer}
  \item 处处不可导.
  \item A
  \item D
  \item A
  \item $\ln 2-\dfrac{2\cpi\ii}3$.
  \item $\ln 3$.
\end{exerciseanswer}

\begin{exerciseanswer}
  \item $-\dfrac12+\dfrac\ii2$.
  \item \delspace\begin{enuminline}[(i)]
    \item $0$.
    \item $0$.
    \end{enuminline}
  \item $\cpi\ii$.
  \item $\sin1-\cos 1$.
  \item $-2\cpi\ii$.
  \item $2\cpi\ii$.
  \item $2uv+C,C\in\BR$.
\end{exerciseanswer}

\begin{exerciseanswer}
  \item 当且仅当非零的 $z_n$ 的辐角全都相同时成立.
  \item 条件收敛.
  \item $\dfrac{\sqrt2}2$.
  \item 收敛半径为 $1$, 和函数为 $-\ln(1-z)$.
  \item $\sumf0(2^{n+1}-1)z^n,\quad |z|<\dfrac12$.
  \item $\sumf1 \dfrac{2n-1}{z^n}$.
  \item $-1$ 是一阶极点, $1$ 是二阶极点.
  \item $1$ 是二阶极点, $0$ 是一阶极点, $-1$ 是三阶极点.
  \item \begin{itemize}
    \item $z=2k\cpi\ii $ 是一阶极点, $k\neq 0,\pm1$.
    \item $z=0$ 是四阶极点.
    \item $z=\pm 2\cpi\ii $ 是可去奇点.
    \item $z=\infty$ 不是孤立奇点.
  \end{itemize}
\end{exerciseanswer}




\begin{solution}
  \begin{enuma}
    \item 在 $1<|z|<2$ 内根据定理~\ref{thm:rational}, $c_n$ 是 $a_n$ 中 $|\lambda|\ge 2$ 对应的项, 即
  \[c_n=\frac2{(-2)^{n+1}}+\frac6{2^{n+1}}-\frac1{(-3)^{n+1}}-\frac2{3^{n+1}}.\]
  从而在 $1<|z|<2$ 内, 
  \[f(z)=\sum_{n\ge 0}\left(\frac2{(-2)^{n+1}}+\frac6{2^{n+1}}-\frac1{(-3)^{n+1}}-\frac2{3^{n+1}}\right)z^n+\sum_{n\le-1}5z^n.\]
      \item 在 $2<|z|<3$ 内根据定理~\ref{thm:rational}, $c_n$ 是 $a_n$ 中 $|\lambda|\ge 3$ 对应的项, 即
  \[c_n=-\frac1{(-3)^{n+1}}-\frac2{3^{n+1}}.\]
  从而在 $2<|z|<3$ 内, 
  \[f(z)=\sum_{n\ge 0}\left(-\frac1{(-3)^{n+1}}-\frac2{3^{n+1}}\right)z^n+\sum_{n\le-1}\left(5-\frac2{(-2)^{n+1}}-\frac6{2^{n+1}}\right)z^n.\]
    \item 由定理~\ref{thm:rational} 或推论~\ref{cor:1} 直接得到在 $|z|>3$ 内, 
  \[f(z)=\sum_{n\le-1}\left(5-\frac2{(-2)^{n+1}}-\frac6{2^{n+1}}+\frac1{(-3)^{n+1}}+\frac2{3^{n+1}}\right)z^n.\]    
  \end{enuma}
\end{solution}
  
\end{multicols}