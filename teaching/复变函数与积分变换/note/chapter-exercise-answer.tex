\chapter{练习答案}

\setcounter{chapter}{0}
\stepcounter{chapter}
\setcounter{exer}{0}

\exans $-4$.

\exans $\pm(\sqrt3+\sqrt2\ii)$.

\exans $1$.

\exans 分别是 $-\ov z$ 和 $-z$.

\exans $\dfrac{7-4\ii}5$.

\exans 仅当 $z$ 不是负实数和 $0$ 时它才成立, 即 $\arg z\in(-\cpi,\cpi)$.

\exans $z_1,z_2,\cdots,z_n$ 中的非零元辐角都相等.

\exans $\displaystyle z=2\sqrt3\Bigl(\cos\bigl(-\frac\cpi3\bigr)+\ii\sin\bigl(-\frac\cpi3\bigr)\Bigr)=2\sqrt3\ee^{-\frac{\cpi\ii}3}$, 辐角写成 $\dfrac{5\cpi}3$ 也可以.

\exans $4+\ii$ 或 $2-\ii$.

\exans $2^{12}=4096$.

\exans $\pm\dfrac{\sqrt3+\ii}2,\pm \ii,\pm\dfrac{\sqrt3-\ii}2$.

\exans 双曲线 $x^2-y^2=\dfrac12$ 和双曲线 $xy=\dfrac14$.

\exans \delspace
\begin{enumnopar}[(i)]
  \item 上半平面对应的闭区域为 $\Im z\ge0$.
  \item 下半平面对应的闭区域为 $\Im z\le0$.
  \item 左半平面对应的闭区域为 $\Re z\le0$.
  \item 右半平面对应的闭区域为 $\Re z\ge0$.
  \item 竖直带状区域对应的闭区域为 $x_1\le\Re z\le x_2$.
  \item 水平带状区域对应的闭区域为 $y_1\le\Im z\le y_2$.
  \item 角形区域对应的闭区域为 $\alpha_1\le \Arg z\le \alpha_2$ 以及原点.
  \item 圆域对应的闭区域为 $|z|\le R$.
  \item 圆环域对应的闭区域为 $r\le|z|\le R$.
\end{enumnopar}

\exans C

\exans \delspace
\begin{enumnopar}[(i)]
  \item $\Re z$ 的定义域为 $\BC$, 值域为 $\BR$.
  \item $\arg z$ 的定义域为 $\{z\in\BC\mid z\neq 0\}$, 值域为 $(-\cpi,\cpi]$.
  \item 当 $n>0$ 时, $z^n$ 的定义域为 $\BC$, 值域为 $\BC$.
  当 $n<0$ 时, $z^n$ 的定义域为 $\{z\in\BC\mid z\neq 0\}$, 值域为 $\{z\in\BC\mid z\neq 0\}$.
  \item $\dfrac{z+\ii}{z^2+1}$ 的定义域为 $\{z\in\BC\mid z\neq \pm i\}$, 值域为 $\BC$.
\end{enumnopar}

\exans $-\dfrac dc$.

\exans 收敛到 $0$.



\stepcounter{chapter}
\setcounter{exer}{0}

\exans 处处不可导.

\exans A

\exans D

\exans A

\exans $\ln 2-\dfrac{2\cpi\ii}3$.

\exans $\ln 3$.



\stepcounter{chapter}
\setcounter{exer}{0}

\exans $-\dfrac12+\dfrac\ii2$.

\exans \delspace\begin{enuminline}[(i)]
  \item $0$.
  \item $0$.
\end{enuminline}

\exans $\cpi\ii$.

\exans $\sin1-\cos 1$.

\exans $-2\cpi\ii$.

\exans $0$.

\exans $2uv+C$.


\stepcounter{chapter}
\setcounter{exer}{0}









\begin{solution}
  \begin{enumpar}
    \item 在 $1<|z|<2$ 内根据定理~\ref{thm:rational}, $c_n$ 是 $a_n$ 中 $|\lambda|\ge 2$ 对应的项, 即
  \[c_n=\frac2{(-2)^{n+1}}+\frac6{2^{n+1}}-\frac1{(-3)^{n+1}}-\frac2{3^{n+1}}.\]
  从而在 $1<|z|<2$ 内, 
  \[f(z)=\sum_{n\ge 0}\left(\frac2{(-2)^{n+1}}+\frac6{2^{n+1}}-\frac1{(-3)^{n+1}}-\frac2{3^{n+1}}\right)z^n+\sum_{n\le-1}5z^n.\]
      \item 在 $2<|z|<3$ 内根据定理~\ref{thm:rational}, $c_n$ 是 $a_n$ 中 $|\lambda|\ge 3$ 对应的项, 即
  \[c_n=-\frac1{(-3)^{n+1}}-\frac2{3^{n+1}}.\]
  从而在 $2<|z|<3$ 内, 
  \[f(z)=\sum_{n\ge 0}\left(-\frac1{(-3)^{n+1}}-\frac2{3^{n+1}}\right)z^n+\sum_{n\le-1}\left(5-\frac2{(-2)^{n+1}}-\frac6{2^{n+1}}\right)z^n.\]
    \item 由定理~\ref{thm:rational} 或推论~\ref{cor:1} 直接得到在 $|z|>3$ 内, 
  \[f(z)=\sum_{n\le-1}\left(5-\frac2{(-2)^{n+1}}-\frac6{2^{n+1}}+\frac1{(-3)^{n+1}}+\frac2{3^{n+1}}\right)z^n.\]    
  \end{enumpar}
\end{solution}
  