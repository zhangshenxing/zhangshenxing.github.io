
\addcontentsline{toc}{chapter}{练习答案}
\chapter*{练习答案}
\setcounter{chapter}{0}
\stepcounter{chapter}\setcounter{exer}{0}

\exans $-4$.
\exans $1$.
\exans $-\ov z$.
\exans $(x_1x_2-y_1y_2)^2+(x_1y_2+x_2y_1)^2$ 或 $(x_1x_2+y_1y_2)^2+(x_1y_2-x_2y_1)^2$.
\exans $\dfrac{7-4i}5$.

\exans 仅当 $z$ 不是负实数和 $0$ 时它才成立.

\exans $z_1,\dots,z_n$ 中的非零元辐角相等.
\exans $\displaystyle z=2\sqrt3\left(\cos\frac{-\pi}3+i\sin\frac{-\pi}3\right)=2\sqrt3e^{-\frac{\pi i}3}$, 写成 $\dfrac{5\pi}3$ 也可以.

\exans $-2^{2022}$.
\exans $\pm\dfrac{\sqrt3+i}2,\pm i,\pm\dfrac{\sqrt3-i}2$.

\exans 双曲线 $x^2-y^2=\dfrac12$ 和双曲线 $xy=\dfrac14$.
\exans
\begin{enumerate}
	\item 上半平面对应的闭区域为 $\Im z\ge0$.
	\item 下半平面对应的闭区域为 $\Im z\le0$.
	\item 左半平面对应的闭区域为 $\Re z\le0$.
	\item 右半平面对应的闭区域为 $\Re z\ge0$.
	\item 竖直带状区域对应的闭区域为 $x_1\le\Re z\le x_2$.
	\item 水平带状区域对应的闭区域为 $y_1\le\Im z\le y_2$.
	\item 角状区域对应的闭区域为 $\alpha_1\le \arg z\le \alpha_2$ 以及原点. 如果 $\alpha_1=-\pi,\alpha_2=\pi$, 则为 $\BC$.
	\item 圆环域对应的闭区域为 $r\le|z|\le R$.
\end{enumerate}
\exans 整个复平面.

\exans
\begin{enumerate}
	\item $\Re z$ 的定义域为 $\BC$, 值域为 $\BR$.
	\item $\arg z$ 的定义域为 $\{z\in\BC\mid z\neq 0\}$, 值域为 $(-\pi,\pi]$.
	\item $|z|$ 的定义域为 $\BC$, 值域为 $\{x\in\BR\mid x\ge0\}$.
	\item 当 $n>0$ 时, $z^n$ 的定义域为 $\BC$, 值域为 $\BC$.
	当 $n<0$ 时, $z^n$ 的定义域为 $\{z\in\BC\mid z\neq 0\}$, 值域为 $\{z\in\BC\mid z\neq 0\}$.
	\item $\dfrac{z+1}{z^2+1}$ 的定义域为 $\{z\in\BC\mid z\neq \pm i\}$, 值域为 $\BC$.
\end{enumerate}


\stepcounter{chapter}\setcounter{exer}{0}


\exans 处处不可导.
\exans C. 因为邻域也是一个区域.
\exans A. 根据C-R方程可知对于A, $u_x(0)=2\neq v_y(0)=3$. 对于BD, 各个偏导数在 $0$ 处取值都是 $0$. C则是处处都可导.

\exans $\ln 2-\dfrac{2\pi i}3$.
\exans $\ln 3$.



