\part{积分变换}


\begin{frame}[<*>]{本章作业(选做)}
\begin{itemize}
\item \emph{1.2} \textbf{1}, \textbf{10}, {11}(1)(3)(5)
\item \emph{2.2} \textbf{1}(1)(5)(6)(9), \textbf{6}(1)(6)(8)
\item \emph{2.4} \textbf{3}(2)(4)
\item \emph{2.5} \textbf{1}(5), \textbf{2}(1), \textbf{5}(1)
\end{itemize}
\end{frame}


\begin{frame}{积分变换的引入}
\onslide<+->
在学习指数和对数的时候, 我们了解到利用对数可以将乘除、幂次转化为加减、乘除.
\begin{example}
计算 $12345\times 67890$.
\end{example}
\begin{solution}
通过查对数表得到
\[\ln 12345\approx 9.4210,\qquad\ln 67890\approx 11.1256.\]
\onslide<+->
将二者相加并通过反查对数表得到原值
\[12345\times 67890\approx \exp(20.5466)\approx 8.3806\times 10^8.\]
\end{solution}
\end{frame}


\begin{frame}{积分变换的引入}
\onslide<+->
而对于函数而言, 我们常常要解函数的微积分方程.
\begin{example}
解微分方程
	\[\begin{cases}
		y''+y=t	,& \\
		y(0)=y'(0)=0.&
	\end{cases}\]
\end{example}
\begin{solution}
\indent
我们希望能找到一种函数\emph{变换 $\msl$}, 使得它可以把函数的微分和积分变成代数运算, 计算之后通过\emph{反变换 $\msl^{-1}$} 求得原来的解.

\indent
\onslide<+->
这个变换最常见的就是我们将要介绍的\emph{傅里叶变换}和\emph{拉普拉斯变换}.
\end{solution}
\end{frame}

