\section{函数的极限}

\subsection{函数极限的定义}
\begin{frame}{函数在 $+\infty$ 的极限}
	\onslide<+->
	我们仿造数列的极限来定义 $x\ra+\infty$ 时 $f(x)$ 的极限.
	\onslide<+->
	回忆数列的 $\varepsilon$-$N$ 语言:
	\begin{center}
		\alert{$\forall\varepsilon>0, \exists N$ 使得当 $n>N$ 时, 有 $\abs{a_n-a}<\varepsilon$.}
	\end{center}
	\onslide<+->
	\begin{definition}
		设 $x>M$ 时函数 $f(x)$ 有定义.
		\onslide<+->{如果存在常数 $A$ 满足:
		\begin{center}
			\alert{$\forall\varepsilon>0, \exists X$ 使得当 $x>X$ 时, 有 $\abs{f(x)-A}<\varepsilon$,}
		\end{center}
		}\onslide<+->{则称 $A$ 为 \emph{$f(x)$ 当 $x\ra+\infty$ 时的极限}, 记为 \emph{$\lim\limits_{x\ra+\infty}f(x)=A$} 或 $f(x)\ra A (x\ra+\infty)$.}
		\onslide<+->{\begin{tikzpicture}[overlay]
			\draw[semithick,third] (1.3,1.3) rectangle (10.5,2);
			\draw (11.6,1.65) node[third] {$\varepsilon$-$X$ 语言};
		\end{tikzpicture}}
	\end{definition}
\end{frame}


\begin{frame}{函数在 $-\infty$ 的极限}
	\onslide<+->
	从图像上看, 就是函数在 $(X,+\infty)$ 上的图像被夹在直线 $y=A\pm\varepsilon$ 之间.
	\begin{center}
		\begin{tikzpicture}[framed]
			\draw[cstaxis] (-0.5,0)--(9,0);
			\draw[cstaxis] (0,-0.4)--(0,1.7);
			\begin{scope}[visible on=<1>]
				\draw[cstdash,fourth] (0,1)--(9,1);
				\draw (-0.25,1) node {$A$};
			\end{scope}
			\begin{scope}[visible on=<2->]
				\draw[cstdash,fifth] (0,1.2)--(9,1.2);
				\draw[cstdash,fifth] (0,0.8)--(9,0.8);
				\draw[cstdash,third] (4,0)--(4,1.2);
				\draw (-0.6,1.2) node {$A+\varepsilon$};
				\draw (-0.6,0.8) node {$A-\varepsilon$};
				\draw (4,-0.25) node {$X$};
			\end{scope}
			\draw[cstcurve,main,smooth,domain=1.2:9.5] plot({\x-1},{1+3*sin(\x*240)/(1+\x*\x)});
		\end{tikzpicture}
	\end{center}

	\onslide<+->\onslide<+->
	仿造上述定义, 我们有:
	\onslide<+->
	\begin{definition}
		设 $x<-M$ 时函数 $f(x)$ 有定义.
		\onslide<+->{如果存在常数 $A$ 满足:
			\begin{center}
				\alert{$\forall\varepsilon>0, \exists X$ 使得当 $x<-X$ 时, 有 $\abs{f(x)-A}<\varepsilon$,}
			\end{center}
		}\onslide<+->{则称 $A$ 为 \emph{$f(x)$ 当 $x\ra-\infty$ 时的极限}, 记为 \emph{$\lim\limits_{x\ra-\infty}f(x)=A$}.}
	\end{definition}
\end{frame}


\begin{frame}{函数在 $\infty$ 的极限}
	\onslide<+->
	\begin{definition}
		设 $|x|>M$ 时函数 $f(x)$ 有定义.
		\onslide<+->{如果存在常数 $A$ 满足:
			\begin{center}
				\alert{$\forall\varepsilon>0, \exists X$ 使得当 $|x|>X$ 时, 有 $\abs{f(x)-A}<\varepsilon$,}
			\end{center}
		}\onslide<+->{则称 $A$ 为 \emph{$f(x)$ 当 $x\ra\infty$ 时的极限}, 记为 \emph{$\lim\limits_{x\ra\infty}f(x)=A$}.}
	\end{definition}
	\onslide<+->
	注意, 函数极限中需要分清 $x\ra\infty, x\ra+\infty, x\ra-\infty$, 而数列情形只有 $n\ra\infty$, 因为 $n$ 是正整数.

	\onslide<+->
	类似于数列极限与子数列极限的关系, 我们有
	\onslide<+->
	\begin{theorem}
		$\lim\limits_{x\ra\infty}f(x)=A \iff \lim\limits_{x\ra+\infty}f(x)=\lim\limits_{x\ra-\infty}f(x)=A$.
	\end{theorem}
\end{frame}


\begin{frame}{函数在一点的极限}
	\onslide<+->
	类似地, 当 $x$ 越来越接近 $x_0$ 时,
	\onslide<+->
	如果函数值 $f(x)$ 越来越接近常数 $A$, 则 $A$ 就是 $x\to x_0$ 时的极限.
	\onslide<+->
	\begin{center}
		\begin{tikzpicture}[framed]
			\draw[cstaxis] (-0.5,0)--(10,0);
			\draw[cstaxis] (0,-0.4)--(0,3.4);
			\draw[cstcurve,main,smooth,domain=0.02:1.6] plot({4+2*\x},{2+1.4*\x*sin(1/\x*180)});
			\draw[cstcurve,main,smooth,domain=0.02:1.6] plot({4-2*\x},{2+1.4*\x*sin(1/\x*180)});
			\begin{scope}[visible on=<3>]
				\draw[cstdash,fourth] (0,2)--(10,2);
				\draw (-0.25,2) node {$A$};
				\draw (4,-0.25) node {$x_0$};
			\end{scope}
			\begin{scope}[visible on=<4->]
				\draw[cstdash,fifth] (0,2.4)--(10,2.4);
				\draw[cstdash,fifth] (0,1.6)--(10,1.6);
				\draw[cstdash,third] (3.25,0)--(3.25,2.4);
				\draw[cstdash,third] (4.75,0)--(4.75,2.4);
				\draw (-0.6,2.4) node {$A+\varepsilon$};
				\draw (-0.6,1.6) node {$A-\varepsilon$};
				\draw (4.75,-0.35) node {$x_0+\delta$};
				\draw (3.25,-0.35) node {$x_0-\delta$};
			\end{scope}
			\filldraw[cstdote] (4,2) circle;
			\fill[cstdot,main] (4,1) circle;
		\end{tikzpicture}
	\end{center}
\end{frame}


\begin{frame}{函数在一点的极限}
	\onslide<+->
	为了陈述方便, 我们引入去心邻域的概念.
	\onslide<+->
	\begin{definition}
	设 $\delta>0$. $x_0$ 的\emph{去心 $\delta$ 邻域}是指
		\[\Uc(x_0,\delta)=\set{x: 0<|x-x_0|<\delta}=(x_0-\delta,x_0)\cup(x_0,x_0+\delta).\]
	\end{definition}
	\onslide<+->
	\begin{definition}
		设函数 $f(x)$ 在 $x_0$ 的某个去心邻域内有定义.
		\onslide<+->{如果存在常数 $A$ 满足
		\begin{center}
			\alert{$\forall\varepsilon>0, \exists\delta>0$ 使得当 $x\in\Uc(x_0,\delta)$ 时, 有 $|f(x)-A|<\varepsilon$,}
		\end{center}
		}\onslide<+->{则称 $A$ 为 $f(x)$ 当 $x\ra x_0$ 时的极限, 记为 $\lim\limits_{x\ra x_0}f(x)=A$ 或 $f(x)\ra A(x\ra x_0)$.
		}\onslide<+->{\begin{tikzpicture}[overlay]
			\draw[semithick,third] (1.8,1.2) rectangle (12.4,1.9);
			\draw (13.4,1.55) node[third] {$\varepsilon$-$\delta$ 语言};
		\end{tikzpicture}}
	\end{definition}
\end{frame}


\begin{frame}{函数在一点的单侧极限}
	\onslide<+->
	类似地可以定义单侧极限:
	\onslide<+->
	\begin{align*}
		 \lim\limits_{x\ra x_0^+}f(x)=A&\iff\alert{\forall\varepsilon>0, \exists\delta>0\;\text{使得当}\;x\in(x_0,x_0+\delta)\;\text{时, 有}\;|f(x)-A|<\varepsilon}.\\
		\lim\limits_{x\ra x_0^-}f(x)=A&\iff\alert{\forall\varepsilon>0, \exists\delta>0\;\text{使得当}\;x\in(x_0-\delta,x_0)\;\text{时, 有}\;|f(x)-A|<\varepsilon}.
	\end{align*}
	\onslide<+->
	同样地, 我们有:
	\begin{theorem}
		$\lim\limits_{x\ra x_0}f(x)=A\iff\lim\limits_{x\ra x_0^+}f(x)=\lim\limits_{x\ra x_0^-}f(x)=A$.
	\end{theorem}
	\onslide<+->
	为了简便, 我们记 \alert{$f(x_0^+)=\lim\limits_{x\ra x_0^+}f(x)$, $f(x_0^-)=\lim\limits_{x\ra x_0^-}f(x)$}.
	\onslide<+->
	注意它们和 $f(x_0)$ 并无关系, $f$ 甚至可以在 $x_0$ 处无定义.
\end{frame}


\subsection{函数极限的证明}
\begin{frame}{例题: 函数在无穷远的极限}
	\onslide<+->
	\begin{example}
		证明 $\lim\limits_{x\ra\infty}\dfrac1x=0$.
	\end{example}
	\onslide<+->
	\begin{analysis}
		和数列极限类似, 这种问题的证明通常也分为两步:
		\begin{itemize}
			\item 估计 $|f(x)-A|$, 得到它和 $|x-x_0|<\delta$ 或 $|x|>X$ 的不等式关系. 从而求得 $\delta$ 或 $X$. 这个过程中可以进行适当的放缩.
			\item 将 $\delta$ 或 $X$ 代入极限的定义中.
		\end{itemize}
		\onslide<+->{对于本题, 从 $\Bigl|\dfrac1x-0\Bigr|<\varepsilon$ 解得 $|x|>\dfrac1\varepsilon$.}
	\end{analysis}
	\onslide<+->
	\begin{proof}
		$\forall\varepsilon>0$, 令 $X=\dfrac1\varepsilon$.
		\onslide<+->{当 $|x|>X$ 时, 有 $\Bigl|\dfrac1x-0\Bigr|=\dfrac1{|x|}<\varepsilon$.
		}\onslide<+->{所以 $\lim\limits_{x\ra\infty}\dfrac1x=0$.\qedhere}
	\end{proof}
\end{frame}


\begin{frame}{例题: 函数在无穷远的单侧极限}
	\onslide<+->
	\begin{example}
		证明 $a>1$ 时, $\lim\limits_{x\ra-\infty}a^x=0$.
	\end{example}
	\onslide<+->
	\begin{analysis}
		由于 $a>1$ 时, $\log_ax$ 是单调递增的.
		\onslide<+->{因此 $|a^x-0|=a^x<\varepsilon \iff x<\log_a \varepsilon$.}
	\end{analysis}
	\onslide<+->
	\begin{proof}
		$\forall\varepsilon>0$, 令 $X=-\log_a\varepsilon$.
		\onslide<+->{当 $x<-X$ 时, 有 $|a^x-0|=a^x<\varepsilon$.}
		
		\onslide<+->{所以 $\lim\limits_{x\ra-\infty}a^x=0$.\qedhere}
	\end{proof}
\end{frame}


\begin{frame}{例题: 函数在一点的极限}\small
	\onslide<+->
	\begin{example}
		证明 $\lim\limits_{x\ra x_0}(ax+b)=ax_0+b$.
	\end{example}
	\onslide<+->
	\begin{analysis}
		$|(ax+b)-(ax_0+b)|=a\cdot|x-x_0|<\varepsilon$, 因此我们可以取 $\delta=\varepsilon/a$.
		\onslide<+->{注意我们需要单独考虑 $a=0$ 的情形.}
	\end{analysis}
	\onslide<+->
	\begin{proof*}
		我们有 $|(ax+b)-(ax_0+b)|=a\cdot|x-x_0|$.
		\onslide<+->{如果 $a=0$, $\forall\varepsilon>0$, 令 $\delta=1$.
		}\onslide<+->{当 $0<|x-x_0|<\delta$ 时, 有 $|(ax+b)-(ax_0+b)|=0<\varepsilon$.}

		\onslide<+->{如果 $a\neq 0$, $\forall\varepsilon>0$, 令 $\delta=\dfrac\varepsilon a$.
		}\onslide<+->{当 $0<|x-x_0|<\delta$ 时, 有
			\[|(ax+b)-(ax_0+b)|=a\cdot|x-x_0|<a\delta=\varepsilon.\]
		}\onslide<+->{所以 $\lim\limits_{x\ra x_0}(ax+b)=ax_0+b$.\qedhere}
	\end{proof*}
\end{frame}


\begin{frame}{例题: 线性函数在一点的极限}
	\onslide<+->
	\begin{example}
		证明 $\lim\limits_{x\ra x_0}\sin x=\sin x_0$.
	\end{example}
	\onslide<+->
	与三角函数有关的放缩往往要用到和差化积公式
	\[\sin x-\sin y=2\sin\frac{x-y}2\cos\frac{x+y}2,\quad
	\cos x-\cos y=-2\sin\frac{x+y}2\sin\frac{x-y}2,\]
	然后将不含 $x-x_0$ 的项放缩到 $1$;
	\onslide<+->
	以及三角函数基本不等式
	\[\left|\sin x\right|\le|x|, \forall x;\qquad|x|\le\left|\tan x\right|, \forall x \in\bigl(-\frac\pi2,\frac\pi2\bigr).\]
\end{frame}


\begin{frame}{例题: 三角函数在一点的极限}
	\onslide<+->
	\begin{proof*}
		我们有
		\[\left|\sin x-\sin x_0\right|=\Bigl|2\sin\frac{x-x_0}2\cos\frac{x+x_0}2\Bigr|
			\le2\Bigl|\sin\frac{x-x_0}2\Bigr|
			\le2\Bigl|\frac{x-x_0}2\Bigr|=|x-x_0|.\]

		\onslide<+->{$\forall\varepsilon>0$, 令 $\delta=\varepsilon$.
		}\onslide<+->{当 $0<|x-x_0|<\delta$ 时, 有
		\[\left|\sin x-\sin x_0\right|\le|x-x_0|<\delta=\varepsilon.\]
		}\onslide<+->{所以 $\lim\limits_{x\ra x_0}\sin x=\sin x_0$.\qedhere}
	\end{proof*}
\end{frame}


\begin{frame}{例题: 函数在无穷远的极限}
	\onslide<+->
	\begin{example}
		证明 $\lim\limits_{x\ra\infty}\arctan x$ 不存在.
	\end{example}
	\onslide<+->
	\begin{analysis*}
		从图像上可以看出 $\lim\limits_{x\ra\pm\infty}\arctan x=\pm\dfrac\pi2$.

		\onslide<+->{我们想要使用 $|x|$ 来控制 $\Bigl|\arctan x-\dfrac\pi2\Bigr|$, 不过这个形式不容易估计.
		}\onslide<+->{令 $t=\dfrac\pi2-\arctan x$, 则问题变成了 $|x|=\Bigl|\tan\Bigl(\dfrac\pi2-t\Bigr)\Bigr|=\dfrac1{\left|\tan t\right|}$ 和 $|t|$ 的关系.
		}\onslide<+->{而我们有 $|t|\le |\tan t|$.}

		\onslide<+->{我们还需要估计 $t$ 的范围.
		}\onslide<+->{由于我们考虑的是 $x\ra+\infty$, 不妨设 $x>0$,
		}\onslide<+->{那么
		\[\arctan x \in\Bigl(0,\frac\pi2\Bigr),\qquad t=\frac\pi2-\arctan x\in\Bigl(0,\frac\pi2\Bigr).\]}
		\vspace{-\baselineskip}
	\end{analysis*}
\end{frame}


\begin{frame}{例题: 函数在无穷远的极限}
	\begin{proof*}
		我们来证明 $\lim\limits_{x\ra+\infty}\arctan x=\dfrac\pi2$.
		\onslide<+->{当 $x>0$ 时, $0<\dfrac\pi2-\arctan x<\dfrac\pi2$.
		}\onslide<+->{因此
			\[\Bigl|\frac\pi2-\arctan x\Bigr|
			\le\Bigl|\tan\Bigl(\frac\pi2-\arctan x\Bigr)\Bigr|
			=\frac1{\left|\tan(\arctan x)\right|}=\frac1{|x|}.\]
		}\onslide<+->{$\forall\varepsilon>0$, 令 $X=\dfrac1\varepsilon>0$.
		}\onslide<+->{当 $x>X$ 时, 有
		\[\Bigl|\frac\pi2-\arctan x\Bigr|\le\frac1{|x|}<\frac1X=\varepsilon.\]
		}\onslide<+->{所以 $\lim\limits_{x\ra+\infty}\arctan x=\dfrac\pi2$.}

		\onslide<+->{类似可证, $\lim\limits_{x\ra-\infty}\arctan x=-\dfrac\pi2$.
		}\onslide<+->{因此 $\lim\limits_{x\ra\infty}\arctan x$ 不存在.\qedhere}
	\end{proof*}
\end{frame}


\begin{frame}{例题: 有理函数在一点的极限}\small
	\beqskip{2pt}
	\onslide<+->
	\begin{example}
		证明 $\lim\limits_{x\ra2}\dfrac{x^2-4}{x-2}=4$.
	\end{example}
	\onslide<+->
	\begin{analysis}
		这种极限是 $\dfrac fg$ 型, 其中 $f\ra0, g\ra0$.
		\onslide<+->{我们称之为 \emph{$\dfrac00$ 型不定式}.
		}\onslide<+->{它的极限可能存在, 可能不存在.
		}\onslide<+->{这种一般要去掉公因式, 将其变为定式.}
	\end{analysis}
	\onslide<+->
	\begin{proof}
		$\displaystyle \biggl|\frac{x^2-4}{x-2}-4\biggr|=|x+2-4|=|x-2|$.
		\onslide<+->{$\forall\varepsilon>0$, 令 $\delta=\varepsilon$.
		}\onslide<+->{当 $0<|x-2|<\delta$ 时, 有
		\[\biggl|\frac{x^2-4}{x-2}-4\biggr|=|x-2|<\varepsilon.\]
		}\onslide<+->{所以 $\lim\limits_{x\ra2}\dfrac{x^2-4}{x-2}=4$.\qedhere}
	\end{proof}
	\endgroup
\end{frame}


\begin{frame}{例题: 分段函数求待定参数}
	\onslide<+->
	\begin{example}
		如果函数 $f(x)=\begin{cases}
			a\sin x,&x<\pi/2;\\
			x+b,&x>\pi/2
		\end{cases}$ 满足 $\lim\limits_{x\ra\pi/2}f(x)=1$, 求 $a,b$.
	\end{example}
	\onslide<+->
	\begin{analysis*}
		本题是典型的由分段函数性质求待定参数的问题, 我们后续会经常遇到.
		\onslide<+->{由于一点处极限等价于两侧极限都存在且为 $1$, 因此我们会得到两个等式, 从而可以解出两个未知参数.}

		\onslide<+->{由于 $f(x)$ 的两个分段都是我们已经求过极限的函数, 因此我们可以直接用前面已经证明的结论.}
	\end{analysis*}
	\onslide<+->
	\begin{solution}
		由于 $\displaystyle f\Bigl[\bigl(\frac\pi2\bigr)^+\Bigr]=\frac\pi2+b,\ 
		f\Bigl[\bigl(\frac\pi2\bigr)^-\Bigr]=a\sin\frac\pi2=a$,
		\onslide<+->{因此 $a=1, b=1-\dfrac\pi2$.}
	\end{solution}
\end{frame}


\begin{frame}{例题: 取整函数的极限}
	\beqskip{7pt}
	\onslide<+->
	\begin{example}
		对于哪些 $x_0$, $\lim\limits_{x\ra x_0}[x]$ 存在.
	\end{example}
	\onslide<+->
	\begin{analysis*}
		与 $[x]$ 有关的问题往往需要用到两个不等式
		\begin{center}
			\alert{$[x]\le x<x+1$} 或 \alert{$x-1<[x]\le x$}.
		\end{center}

		\onslide<+->{从 $[x]$ 的图像上可以看出 $x_0\in\BZ$ 时左右极限不相等, 从而极限不存在.
		}\onslide<+->{解答时, 我们取 $x_0$ 的 $\delta=1/2$ 邻域, 则在这个邻域的左右各自半边内, $[x]$ 是常值函数, 从而得到单侧极限.}

		\onslide<+->{当 $x_0\notin \BZ$ 时, 我们同样希望取一个小邻域使得 $[x]$ 是常值函数.
		}\onslide<+->{这需要 $\delta$ 不超过 $x_0$ 和两边的最近的整数的距离.
		}\onslide<+->{所以
		\[\delta=\min\set{x_0-[x_0], [x_0]+1-x_0}.\]}
		\vspace{-\baselineskip}
	\end{analysis*}
	\endgroup
\end{frame}


\begin{frame}{例题: 取整函数的极限}
	\onslide<+->
	\begin{solution*}
		如果 $x_0\in\BZ$, 则
		\begin{itemize}
			\item 当 $x\in(x_0, x_0+1/2)$ 时, $[x]=x_0$, 所以 $\lim\limits_{	x\ra x_0^+}[x]=x_0$;
			\item 当 $x\in(x_0-1/2, x_0)$ 时, $[x]=x_0-1$, 所以 $\lim\limits_{	x\ra x_0^-}[x]=x_0-1$.
		\end{itemize}
		\onslide<+->{因此 $\lim\limits_{x\ra x_0}[x]$ 不存在.}

		\onslide<+->{如果 $x_0\notin\BZ$, 令 $\delta=\min\set{x_0-[x_0], [x_0]+1-x_0}>0$.
		}\onslide<+->{name 
		\[[x_0]\le x_0-\delta<x_0+\delta\le[x_0]+1.\]
		}\onslide<+->{当 $0<|x-x_0|<\delta$ 时, 有 $x_0-\delta<x<x_0+\delta$,
		}\onslide<+->{从而 $[x_0]<x<[x_0]+1$, $[x]=x_0$.
		}\onslide<+->{因此 $\lim\limits_{x\ra x_0}[x]=x_0$.}

		\onslide<+->{故当且仅当 $x_0\notin\BZ$ 时, $\lim\limits_{x\ra x_0}[x]$ 存在.}
	\end{solution*}
\end{frame}
