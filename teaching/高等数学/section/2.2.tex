\section{函数的极限}

\subsection{函数极限的定义}
\begin{frame}{函数在 $+\infty$ 的极限}
	\onslide<+->
	和数列情形类似, 我们可以定义 $x\ra+\infty$ 时 $f(x)$ 的极限.
	\onslide<+->
	回忆数列的 $\varepsilon$-$N$ 语言:
	\begin{center}
		\alert{$\forall\varepsilon>0, \exists N$ 使得当 $n>N$ 时, 有 $\abs{a_n-a}<\varepsilon$.}
	\end{center}
	\onslide<+->
	\begin{definition}
		设函数 $f(x)$ 的定义包含某个 $(M,+\infty)$, 即在 $x$ 充分大时有定义, $A$ 为一常数.
		\onslide<+->{如果
		\begin{center}
			\alert{$\forall\varepsilon>0, \exists X$ 使得当 $x>X$ 时, 有 $\abs{f(x)-A}<\varepsilon$,}
		\end{center}
		}\onslide<+->{则称 $A$ 为 \emph{$f(x)$ 当 $x\ra+\infty$ 时的极限}, 记为 \emph{$\lim\limits_{x\ra+\infty}f(x)=A$} 或 $f(x)\ra A (x\ra+\infty)$.}
		\onslide<+->{\begin{tikzpicture}[overlay,xshift=106mm,yshift=-19mm]
			\draw[semithick,dcolorc] (-9.3,3.2) rectangle (-0.1,3.9);
			\draw (1,3.55) node[dcolorc] {$\varepsilon$-$X$ 语言};
		\end{tikzpicture}}
	\end{definition}
	\onslide<+->
	\begin{center}
		\begin{tikzpicture}
			\draw[cstaxis] (-0.5,0)--(10,0);
			\draw[cstaxis] (0,-0.5)--(0,1.5);
			\draw[cstdash,dcolord] (0,1)--(10,1);
			\draw[domain=1:10]  plot(\x,{sin \x/\x});
		\end{tikzpicture}
	\end{center}
\end{frame}


\begin{frame}{函数在 $\infty$ 的极限}
	\onslide<+->
	从图像上看, 就是函数在 $(X,+\infty)$ 的限制的图像被夹在直线 $y=A\pm\varepsilon$ 之间.
	\onslide<+->
	我们将红字部分称为 \alert{$\varepsilon$-$X$} 语言.
	\onslide<+->
	仿造上述定义, 我们有:
	\onslide<+->
	\begin{definition}
		设函数 $f(x)$ 当 $-x$ 充分大时有定义 (即存在 $M$ 使得 $f(x)$ 在 $(-\infty,-M)$ 上有定义), $A$ 为常数.
		\onslide<+->{如果
			\begin{center}
				\alert{$\forall\varepsilon>0, \exists X$ 使得当 $x<-X$ 时, 有 $\abs{f(x)-A}<\varepsilon$,}
			\end{center}
		}\onslide<+->{则称 $A$ 为 \emph{$f(x)$ 当 $x\ra-\infty$ 时的极限}, 记为 \emph{$\lim\limits_{x\ra-\infty}f(x)=A$} 或 $f(x)\ra A (x\ra-\infty)$.}
	\end{definition}
	\onslide<+->
	\begin{definition}
		设函数 $f(x)$ 当 $|x|$ 充分大时有定义 (即存在 $M$ 使得 $f(x)$ 在 $(-\infty,-M)\cup(M,+\infty)$ 上有定义), $A$ 为常数.
		\onslide<+->{如果
			\begin{center}
				\alert{$\forall\varepsilon>0, \exists X$ 使得当 $|x|>X$ 时, 有 $\abs{f(x)-A}<\varepsilon$,}
			\end{center}
		}\onslide<+->{则称 $A$ 为 \emph{$f(x)$ 当 $x\ra\infty$ 时的极限}, 记为 \emph{$\lim\limits_{x\ra\infty}f(x)=A$} 或 $f(x)\ra A (x\ra\infty)$.}
	\end{definition}
	\onslide<+->
	注意, 函数极限中需要分清 $x\ra\infty, x\ra+\infty, x\ra-\infty$, 而数列情形只有 $n\ra\infty$, 因为 $n$ 是正整数.
\end{frame}


% \begin{frame}{}
% 	\onslide<+->
% 	类似于数列极限与子数列极限的关系, 我们有
% 	\onslide<+->
% 	\begin{theorem}
% 		$\lim\limits_{x\ra\infty}f(x)=A \iff \lim\limits_{x\ra+\infty}f(x)=\lim\limits_{x\ra-\infty}f(x)=A$.
% 	\end{theorem}
% 	\onslide<+->
% 	当 y \ra \infty 时, x=x0+ 1
% y \ra x0 . 因此如果存在 𝛿>0 使得函数 $f(x)$ 在 x0 的去
% 心 𝛿 邻域
% 𝑈
% ∘
% x0 , 𝛿=x0-𝛿 , x0 \cup x0 , x0+ 𝛿
% 	上有定义, 则 \lim\limits_
% x\rax0
% f(x) 应当定义为 \lim\limits_
% y\ra\inftyf x0+ 1
% y . 于是我们得到下述定义:
% 	定义 设函数 $f(x)$ 在 x0 的某个去心邻域内有定义, A 为常数. 如果
% \forall\varepsilon>0, \exists 𝛿>0 使得当 x \in 𝑈
% ∘
% x0 , 𝛿 时, 有 $f(x)$-A<\varepsilon  ,
% 	则称 A 为 $f(x)$ 当 x\ra x0 时的极限, 记为 \lim\limits_
% x\rax0
% f(x)=A 或 f(x) \ra A x\ra x0 .
% \end{frame}


% \begin{frame}
% 	从几何角度来看, 就是函数在 𝑈
% ∘
% x0 , 𝛿 上的限制的图像被夹在直线 y=A \pm \varepsilon  之
% 间.
% 	类似地可以定义单侧极限
% 	\lim\limits_
% x\rax0
% + $f(x)$=A ⇔ \forall\varepsilon>0, \exists 𝛿>0 使得当 x \in x0 , x0+ 𝛿 时, 有 $f(x)$-A<\varepsilon  .
% 	\lim\limits_
% x\rax0
% - $f(x)$=A ⇔ \forall\varepsilon>0, \exists 𝛿>0 使得当 x \in x0-𝛿 , 0 时, 有 $f(x)$-A<\varepsilon  .
% 	我们将红字部分称为 \varepsilon -𝛿 语言.
% 	\begin{theorem}
% \lim\limits_
% \end{theorem}
% x\rax0
% f x=A ⇔ \lim\limits_
% x\rax0
% + $f(x)$=\lim\limits_
% x\rax0
% - $f(x)$=A.
% 	如果 \lim\limits_
% x\rax0
% + $f(x)$=A, 我们记 f x0
% +=A. 如果 \lim\limits_
% x\rax0
% - $f(x)$=A, 我们记 f x0
% -=A.
% \end{frame}


% \begin{frame}
% 	\begin{example}
% \begin{proof}
% \end{example}
% \lim\limits_
% \end{proof}
% x\ra\infty
% 1
% x=0.
% 	分析 和数列极限类似, 这种问题的证明通常也分为两部分
% 	先估计 $f(x)$-A , 得到它和 x-x0<𝛿 或 x>X 的不等式关系,
% 从而求得 𝛿 或 X. 这个过程中可以进行适当的放缩.
% 	将 𝛿 或 X 代入极限的定义中.
% 	1
% x-0=1
% x<\varepsilon , x>1
% \varepsilon  , 因此我们可以取 X=1
% \varepsilon  .
% 	\begin{proof}
% \forall\varepsilon>0, 令 X=1
% \end{proof}
% \varepsilon  . 当 x>X 时, 有 1
% x-0=1
% x<\varepsilon . 所以 \lim\limits_
% x\ra\infty
% 1
% x =
% 0.
% \end{frame}


% \begin{frame}
% 	\begin{example}
% \begin{proof}
% \end{example}
% a>1 时, \lim\limits_
% \end{proof}
% x\ra-\inftya x=0.
% 	分析 由于 a>1 时, \log a x 是单调递增的. 因此 a x-0=a x<\varepsilon , x <
% \log a \varepsilon .
% 	\begin{proof}
% \forall\varepsilon>0, 令 X=- \log a \varepsilon . 当 x<-X 时, 有 a x-0=a x<\varepsilon . 所以
% \end{proof}
% \lim\limits_
% x\ra-\inftya x=0.
% 	\begin{example}
% \begin{proof}
% \end{example}
% \lim\limits_
% \end{proof}
% x\rax0
% ax+ b=ax0+ b.
% 	分析 ax+ b-ax0+ b=a \cdot x-x0<\varepsilon , 因此我们可以取 𝛿=\varepsilon 
% a .
% 注意我们需要单独考虑 a=0 的情形.
% \end{frame}


% \begin{frame}
% 	\begin{proof}
% 我们有 ax+ b-ax0+ b=a \cdot x-x0 .
% \end{proof}
% 	如果 a=0, \forall\varepsilon>0, 令 𝛿=1. 当 0<x-x0<𝛿 时, 有
% ax+ b-ax0+ b=a \cdot x-x0=0<\varepsilon .
% 	如果 a\neq 0, \forall\varepsilon>0, 令 𝛿=\varepsilon 
% a. 当 0<x-x0<𝛿 时, 有
% ax+ b-ax0+ b=a \cdot x-x0<a 𝛿=\varepsilon .
% 	所以 \lim\limits_
% x\rax0
% ax+ b=ax0+ b.
% 	注记 当 $f(x)$=A 时, 我们可以取任一正数作为 𝛿 , 只要 𝑈
% ∘
% x, 𝛿 包含在该
% 函数的定义域范围内即可.
% \end{frame}


% \begin{frame}
% 	\begin{example}
% \begin{proof}
% \end{example}
% \lim\limits_
% \end{proof}
% x\rax0
% sin x=sin x0 .
% 	与三角函数有关的放缩往往要用到和差化积公式
% sin x-sin y=2 sin x-y
% 2 cos x+ y
% 2 , cos x-cos y=-2 sin x+ y
% 2 sin x-y
% 2 ,
% 	然后将不含 x-x0 的项放缩到 1; 以及三角函数基本不等式
% sin x ≤ x , \forall x \in-\infty,+\infty , x ≤ tan x , \forall x \in-\pi
% 2 , \pi
% 2 .
% 	\begin{proof}
% 我们有 sin x-sin x0=2 sin x-x0
% \end{proof}
% 2 cos x+x0
% 2 ≤ 2 sin x-x0
% 2 ≤ 2 x-x0
% 2=x-x0 .
% 	\forall\varepsilon>0, 令 𝛿=\varepsilon  . 当 0<x-x0<𝛿 时, 有
% sin x-sin x0 ≤ x-x0<𝛿=\varepsilon  .
% 	所以 \lim\limits_
% x\rax0
% sin x=sin x0 .
% \end{frame}


% \begin{frame}
% 	\begin{example}
% \begin{proof}
% \end{example}
% \lim\limits_
% \end{proof}
% x\ra\infty arctan x 不存在.
% 	分析 从图像上可以看出 \lim\limits_
% x\ra+\infty arctan x=\pi
% 2 . 我们想要使用 x 来控制
% arctan x-\pi
% 2 . 不过这个形式不容易估计, 我们令 t=arctan x, 则问题变
% 成了 t-\pi
% 2 和 tan t 的关系. 再令 s=arctan x-\pi
% 2 , 则 s ≤ tan s .
% 	不过这个不等式并不总是对的, 我们需要估计 s=arctan x-\pi
% 2 的范围.
% 由于我们考虑的是 x\ra+\infty, 不妨设 x>0, 那么 arctan x \in 0, \pi
% 2 , s =
% arctan x-\pi
% 2 \in-\pi
% 2 , 0 .
% \end{frame}


% \begin{frame}
% 	\begin{proof}
% 我们来证明 \lim\limits_
% \end{proof}
% x\ra+\infty arctan x=\pi
% 2 . 当 x>0 时,arctan x-\pi
% 2 \in-\pi
% 2 , 0 .
% 因此 arctan x-\pi
% 2 ≤ tan \pi
% 2-arctan x=1
% tan arctan x=1
% x.
% 	\forall\varepsilon>0, 令 X=1
% \varepsilon>0. 当 x>X 时, 有
% arctan x-\pi
% 2 ≤ 1
% x<1
% X=\varepsilon .
% 	所以 \lim\limits_
% x\ra+\infty arctan x=\pi
% 2 .
% 	同理, \lim\limits_
% x\ra-\infty arctan x=- \pi
% 2 . 因此 \lim\limits_
% x\ra\infty arctan x 不存在.



% • 例 证明 \lim\limits_
% 𝑥→2
% 𝑥 2 −4
% 𝑥 −2=4.
% • 分析 这种极限是 𝑓
% 𝑔 型, 其中 𝑓 → 0, 𝑔 → 0. 我们称之为 0
% 0 型不定式. 它的极限可
% 能存在, 可能不存在. 这种一般要去掉公因式, 将其变为定式.
% • 证明 𝑥 2 −4
% 𝑥−2 − 4=𝑥+ 2 − 4=𝑥 − 2 .
% • ∀\varepsilon>0, 令 𝛿=\varepsilon  .当 0<𝑥 − 2<𝛿 时, 有
% 𝑥 2 − 4
% 𝑥 − 2 − 4=𝑥 − 2<\varepsilon .
% • 所以 \lim\limits_
% 𝑥→2
% 𝑥 2 −4
% 𝑥−2=4.


% • 例 如果函数 𝑓 𝑥=ቐ𝑎 sin 𝑥 , 𝑥<𝜋
% 2
% 𝑥+ b, 𝑥>𝜋
% 2
% 满足 \lim\limits_
% 𝑥→𝜋
% 2
% 𝑓(𝑥)=1, 求 𝑎, b.
% • 分析 这种是典型的由分段函数性质求待定参数的问题, 我们后续会经常遇
% 到. 由于一点处极限等价于两侧极限都存在且为 1, 因此我们会得到两个
% 等式, 从而可以解出两个未知参数.
% • 由于 𝑓 𝑥 的两个分段都是我们已经求过极限的函数, 因此我们可以直接
% 用前面已经证明的结论.
% • 解 由于 \lim\limits_
% 𝑥→ 𝜋
% 2
% + 𝑓 𝑥=𝜋
% 2+ b, \lim\limits_
% 𝑥→ 𝜋
% 2
% − 𝑓 𝑥=𝑎, 因此 𝑎=1, b=1 − 𝜋
% 2 .



% • 例 对于哪些 𝑥0 , \lim\limits_
% 𝑥→𝑥0
% [𝑥] 存在.
% • 分析 与 𝑥 有关的问题往往需要用到两个不等式,
% 𝑥 ≤ 𝑥<𝑥+ 1 或 𝑥 − 1<𝑥 ≤ 𝑥.
% • 我们回忆 𝑥 的图像. 从图像上可以看出 𝑥0 ∈ ℤ 时左右极限不相等, 从而
% 该点极限不存在. 解答时, 我们取一个很小的邻域, 使得在这个邻域的左右
% 各自半边内, 该函数是常值函数, 从而得到单侧极限.
% • 当 𝑥0 ∉ ℤ 时, 我们同样希望取一个小邻域使得 𝑥 是常值函数. 这需要 𝛿
% 不超过 𝑥0 和两边的最近的整数的距离. 所以
% 𝛿=min{𝑥 − 𝑥0 , 𝑥0+ 1 − 𝑥} .



% • 解 如果 𝑥0 ∈ ℤ, 则
% • 当 𝑥 ∈ 𝑥0 , 𝑥0+ 1
% 2 时, 𝑥=𝑥0 , 所以 \lim\limits_
% 𝑥→𝑥0
% + 𝑥=𝑥0 ;
% • 当 𝑥 ∈ 𝑥0 − 1
% 2 , 𝑥0 时, 𝑥=𝑥0 − 1, 所以 \lim\limits_
% 𝑥→𝑥0
% − 𝑥=𝑥0 − 1.
% • 因此 \lim\limits_
% 𝑥→𝑥0
% 𝑥 不存在.
% • 如果 𝑥0 ∉ ℤ, 令 𝛿=min 𝑥0 − 𝑥0 , 𝑥0+ 1 − 𝑥0>0. 于是
% 𝑥0 ≤ 𝑥0 − 𝛿<𝑥0+ 𝛿 ≤ 𝑥0+ 1.
% • 当 0<𝑥 − 𝑥0<𝛿 时, 有 𝑥0 − 𝛿<𝑥<𝑥0+ 𝛿, 从而 𝑥0<𝑥<𝑥0+
% 1, 𝑥=𝑥0 . 因此 \lim\limits_
% 𝑥→𝑥0
% 𝑥=𝑥0 .
% • 故当且仅当 𝑥0 ∉ ℤ 时, \lim\limits_
% 𝑥→𝑥0
% [𝑥] 存在.



% • 现在我们可以严格定义渐近线了. 我们知道 𝑎𝑥+ b𝑦+ 𝑐=0 当 𝑎, b 不全
% 为零时表示一条直线, 无论是 𝑦=𝑘𝑥+ b 还是 𝑥=𝑎 都可以统一为这种
% 形式.
% • 点 𝑃=(𝑥, 𝑦) 到这条直线的距离是 𝑎𝑥+b𝑦+𝑐
% 𝑎2+b2 . 点 𝑃 趋向于无穷远可以表
% 达为 𝑥 2+ 𝑦 2 →+∞.
% • 定义 对于曲线 𝐶 , 如果 \lim\limits_ 𝑎𝑥+ b𝑦+ 𝑐=0, 其中 𝑥 2+ 𝑦 2 → ∞,
% 𝑥, 𝑦 ∈ 𝐶 , 则称直线 𝑎𝑥+ b𝑦+ 𝑐=0 是曲线 𝐶 的一条渐近线.
% • 用 \varepsilon-X 语言表达就是:
% ∀\varepsilon>0, \exists X 使得当 𝑥 2+ 𝑦 2>X, 𝑥, 𝑦 ∈ 𝐶 时, 有 𝑎𝑥+ b𝑦+ 𝑐<\varepsilon .



% • 渐近线可分为: 水平渐近线、竖直渐近线和斜渐近线三种, 相应的判定方
% 法等价于
% • \lim\limits_
% 𝑥→+∞ 𝑓 𝑥 − 𝑘𝑥+ b=0 或 \lim\limits_
% 𝑥→−∞ 𝑓 𝑥 − 𝑘𝑥+ b=0, 直线 𝑦 =
% 𝑘𝑥+ b;
% • \lim\limits_
% 𝑥→𝑎−
% 1
% 𝑓 𝑥=0 或 \lim\limits_
% 𝑥→𝑎+
% 1
% 𝑓 𝑥=0, 直线 𝑥=𝑎.
% • 注意两个单侧极限有一个存在即可.
% • 函数的渐近线指的就是它的图像的渐近线.
% • 例 𝑓 𝑥=1
% 𝑥−1+ 𝑥, \lim\limits_
% 𝑥→∞ 𝑓 𝑥 − 𝑥=0, \lim\limits_
% 𝑥→1
% 1
% 𝑓 𝑥=0, 因此 𝑦=𝑥, 𝑥=1 是
% 它的全部渐近线.



