\begin{frame}{课程安排}
	\onslide<+->
	本课程共16周80课时(2023/02/20$\sim$2023/06/09).

	\onslide<+->
	前八周上课时间是周一四, 后八周上课时间是周一三四.
	\begin{figure}[htbp]
		\centering
		\begin{minipage}[t]{0.48\textwidth}
			\centering
			\onslide<3->{\includegraphics[height=4cm]{image-qq.png}

			\emph{课程QQ群}\\
			群号: \textbf{561271638}\\
			答案\textbf{034Y01}}
		\end{minipage}
		\begin{minipage}[t]{0.48\textwidth}
			\centering
			\onslide<4->{\includegraphics[height=4cm]{image-book.jpg}
		
			\emph{教材}\\
			朱士信~唐烁主编\\
			《高等数学》上册}
		\end{minipage}
	\end{figure}
\end{frame}


\begin{frame}{课程内容I}
	\begin{center}
		\begin{tikzpicture}[
			small mindmap,
			every node/.style={concept, circular drop shadow,execute at begin node=\hskip0pt},
			root concept/.append style={concept color=black, fill=white, line width=1ex, text=black},
			font=\scriptsize, text=white,
			childa/.style={concept color=red,faded/.style={concept color=red!50}},
			childb/.style={concept color=blue,faded/.style={concept color=blue!50}},
			childc/.style={concept color=orange,faded/.style={concept color=orange!50}},
			grow cyclic,
			level 1/.append style={level distance=3.5cm,font=\small},
			level 2/.append style={level distance=2.5cm,sibling angle=45,font=\scriptsize}]
			\node [root concept] {函数和极限} % root
			child [grow=180, childa] { node {函数}
				child[grow=-130] { node {定义域和值域} }
				child[grow=-90] { node {有界、单调、奇偶、周期} }
				child[grow=-50] { node {初等函数} }
			}
			child [grow=-90, level distance=3cm, childb] { node {极限}
				child[grow=-165,level distance=4.5cm] { node {7种极限的定义} }
				child[grow=-145,level distance=3.3cm] { node {极限的性质} }
				child[grow=-113] { node {极限的代数复合运算} }
				child[grow=-67] { node {不定式无穷小} }
				child[grow=-35,level distance=3.3cm] { node {单调有界收敛准则} }
				child[grow=-15,level distance=4.5cm] { node {夹逼准则} }
			}
			child [grow=0, childc] { node {连续}
				child[grow=-130] { node {分段函数连续性} }
				child[grow=-90] { node {间断点} }
				child[grow=-50] { node {最值、介值、零点定理} }
			};
		\end{tikzpicture}
	\end{center}
\end{frame}


\begin{frame}{课程内容II}
	\begin{center}
		\begin{tikzpicture}[
			small mindmap,
			every node/.style={concept, circular drop shadow,execute at begin node=\hskip0pt},
			root concept/.append style={concept color=black, fill=white, line width=1ex, text=black},
			font=\scriptsize, text=white,
			childa/.style={concept color=blue,faded/.style={concept color=blue!50}},
			childb/.style={concept color=green!50!black,faded/.style={concept color=green!50!black!50}},
			grow cyclic,
			level 1/.append style={level distance=3.5cm,font=\small},
			level 2/.append style={level distance=2.7cm,sibling angle=45,font=\scriptsize}]
			\node [root concept] {导数及应用} % root
			child [grow=225, childa] { node {导数}
				child[grow=120] { node {运算法则} }
				child[grow=160] { node {隐函数导数} }
				child[grow=200] { node {参变量函数导数} }
				child[grow=240] { node {高阶导数} }
				child[grow=280] { node {微分} }
			}
			child [grow=-45, childb] { node {一元函数微分学}
				child[grow=80] { node {中值定理} }
				child[grow=40] { node {洛必达法则} }
				child[grow=0] { node {泰勒展开} }
				child[grow=-40] { node {单调性、最值、极值} }
				child[grow=-80] { node {凹凸性、拐点} }
				child[grow=-120] { node {在等式和不等式的应用} }
			};
		\end{tikzpicture}
	\end{center}
\end{frame}

\begin{frame}{课程学习方法}
	\begin{center}
		\begin{tikzpicture}[node distance=25pt]
			\node[cstnodeg,align=center] (1) at (0,2)  {\emph{课前}\\预习课本};
			\node[cstnodeg,align=center] (2) at (3,0)  {\emph{课上}\\认真听课\\记好笔记};
			\node[cstnodeg,align=center] (3) at (0,-2) {\emph{课后}\\过一遍教材\\与课上知识点};
			\node[cstnodeg,align=center] (4) at (-3,0) {\emph{作业}\\检测学\\习效果};
			\draw[cstnarrow,dcolorc] (1.east) to[bend left] (2.north);
			\draw[cstnarrow,dcolorc] (2.south) to[bend left] (3.east);
			\draw[cstnarrow,dcolorc] (3.west) to[bend left] (4.south);
			\draw[cstnarrow,dcolorc] (4.north) to[bend left] (1.west);
		\end{tikzpicture}
	\end{center}
\end{frame}

