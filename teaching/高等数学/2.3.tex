
\end{frame}


\begin{frame}
2.3 极限的性质
	与数列极限类似, 对于函数的六种极限我们均有:
	定理(唯一性) 如果 \lim\limits_ $f(x)$ 存在, 则必唯一.
	定理(局部有界性) 如果 \lim\limits_ $f(x)$ 存在, 则 $f(x)$ 局部有界.
	所谓的局部, 是指极限定义中 $f(x)$-A<\varepsilon  成立的某个区间范围. 以
x\ra+\infty 为例, 局部有界是指存在 X 使得 f | X,+\infty 有界.
	定理(局部保号性) 如果 \lim\limits_ $f(x)$=a>0, 则 $f(x)$ 局部大于 0.
	推论 如果 \lim\limits_ $f(x)$ 存在且局部非负, 则极限非负.
	推论 如果 \lim\limits_ $f(x)$ , \lim\limits_g(x) 存在且局部 $f(x)$ \ge g(x), 则 \lim\limits_ $f(x)$ \ge
\lim\limits_g(x).
\end{frame}


\begin{frame}
	极限的四则运算性质
	\begin{theorem}
设 \lim\limits_ $f(x)$=A, \lim\limits_ g x=𝐵.
\end{theorem}
	(1) \lim\limits_ f \pm g x=A \pm 𝐵.
	(2) \lim\limits_ fg x=A𝐵.
	(3) 当 𝐵\neq 0 时, \lim\limits_ f
g x=A
𝐵 .
	\begin{proof}
我们只证明 x\ra x0 的情形, 其它情形类似.
\end{proof}
	(1) \forall\varepsilon>0, \exists 𝛿1>0 使得当 0<x-x0<𝛿1 时, 有 $f(x)$-A<\varepsilon 
2;
	\exists 𝛿2>0 使得当 0<x-x0<𝛿2 时, 有 g x-𝐵<\varepsilon 
2.
	因此, 当 0<x-x0<𝛿=min 𝛿1 , 𝛿2 时, 有
f \pm g x-A \pm 𝐵=$f(x)$-A \pm g x-𝐵 ≤ $f(x)$-A+ g x-𝐵<\varepsilon  .
	所以 \lim\limits_
x\rax0
f \pm g x=A \pm 𝐵 .
\end{frame}


\begin{frame}
	(2) 分析 $f(x)$ g x-A𝐵=$f(x)$ g x-𝐵+ 𝐵 $f(x)$-A
≤ $f(x)$ \cdot g x-𝐵+ 𝐵 \cdot $f(x)$-A .
	由局部有界性, \exists 𝛿1 , M>0 使得当 0<x-x0<𝛿1 时, 有 $f(x)$ ≤ M.
	\forall\varepsilon>0, \exists 𝛿2>0 使得当 0<x-x0<𝛿2 时, 有 $f(x)$-A<\varepsilon 
2 𝐵+1 ;
	\exists 𝛿3>0 使得当 0<x-x0<𝛿3 时, 有 g x-𝐵<\varepsilon 
2M.
	因此, 当 0<x-x0<𝛿=min 𝛿1 , 𝛿2 , 𝛿3 时, 有
f x g x-A𝐵 ≤ M \cdot g x-𝐵+ 𝐵 \cdot $f(x)$-A
≤ M \cdot \varepsilon 
2M+ 𝐵 \cdot \varepsilon 
2 𝐵+ 1<\varepsilon .
	所以 \lim\limits_
x\rax0
fg x=A𝐵.
\end{frame}


\begin{frame}
	(3) 我们只需要\begin{proof}
1
\end{proof}
g \ra 1
𝐵 然后利用(2). 这需要对 1
g 在 x0 附近进行估计
使得其有个非零下界.
	对于 \varepsilon =𝐵
2>0, \exists 𝛿1>0 使得当 0<x-x0<𝛿1 时, 有 g x-𝐵 ≤
\varepsilon =𝐵
2 . 于是
g x \ge 𝐵-g x-𝐵 \ge 𝐵
2 ,
1
g x-1
𝐵=g x-𝐵
𝐵 \cdot g x ≤ 2
𝐵 2 g x-𝐵 .
\end{frame}


\begin{frame}
	\forall\varepsilon>0, \exists 𝛿2>0 使得当 0<x-x0<𝛿2 时, 有 g x-𝐵<𝐵 2
2 \varepsilon .
	因此当 0<x-x0<𝛿=min 𝛿1 , 𝛿2 时,
1
g x-1
𝐵=g x-𝐵
𝐵 \cdot g x ≤ 2
𝐵 2 g x-𝐵<\varepsilon .
	所以 \lim\limits_
x\rax0
1
g x=1
𝐵. 由(2)可知 \lim\limits_
x\rax0
f(x)
g x=\lim\limits_
x\rax0
f x \cdot \lim\limits_
x\rax0
1
g x=A
𝐵.
	推论 设 \lim\limits_ $f(x)$=A, 则
	(1) \lim\limits_ Cf x=CA.
	(2) \lim\limits_ $f(x)$ 𝑘=A𝑘 , 𝑘 \in ℕ+ .
\end{frame}


\begin{frame}
	数列与函数的关系
	设 an 是一个数列. 定义 $f(x)$=a x , 它是 1,+\infty 上的一个分段函数.
	如果 \lim\limits_
n\ra\inftyan=a, 则 \forall\varepsilon>0, \exists N 使得当 n>N 时, 有 an-a<\varepsilon  . 因此当 x>
N+ 1 时, x>N, $f(x)$-a<\varepsilon  , 所以 \lim\limits_
x\ra+\inftyf x=a
.
	反过来也成立, 因此 \lim\limits_
n\ra\inftyan=\lim\limits_
x\ra+\inftyf x . 从而与函数极限 \lim\limits_
x\ra+\infty 相关的概念和
性质可以移植到数列上.
	\begin{theorem}
设 \lim\limits_
\end{theorem}
n\ra\infty an=a, \lim\limits_
n\ra\infty bn=b, 则
	\lim\limits_
n\ra\infty an \pm bn=a \pm b, \lim\limits_
n\ra\infty an bn=ab, \lim\limits_
n\ra\infty
an
bn
= a
b b\neq 0 ,
	\lim\limits_
n\ra\infty C an=Ca, \lim\limits_
n\ra\infty an
𝑘=a𝑘 , 𝑘 \in ℕ+ .
\end{frame}


\begin{frame}
	\begin{example}
对于多项式 𝑃(x ), 有 \lim\limits_
\end{example}
x\rax0
𝑃 x=𝑃 x0 .
	\begin{example}
对于多项式 𝑃 x , 𝑄 x , 如果 𝑄 x0\neq 0, 则 \lim\limits_
\end{example}
x\rax0
𝑃 x
𝑄 x=𝑃 x0
𝑄 x0
.
	\begin{example}
求 \lim\limits_
\end{example}
x\ra1
x 2 -1
x 3 -1.
	分析 该极限是 0
0 型不定式.
	解 由于 x\ra 1 时, x-1\neq 0, 因此 \lim\limits_
x\ra1
x 2 -1
x 3 -1=\lim\limits_
x\ra1
x+1
x 2+x+1=2
3.
	可以看出, 利用极限的四则运算性质, 我们可以得到很多的极限. 下面我们介绍
极限的函数复合运算性质. 我们在未来会看到, 在研究函数连续性和可导性时,
也会通过研究简单函数、四则运算性质和函数复合运算性质来研究复杂函数的
相应性质.
\end{frame}


\begin{frame}
极限的复合运算性质
	\begin{theorem}
设 \lim\limits_
\end{theorem}
x\rax0
𝜑 x=a, \lim\limits_
𝑢\raa f (𝑢)=A. 若函数 𝑢=𝜑 x 在点 x0 的某个去心邻域内不等
于 a, 则
\lim\limits_
x\rax0
f 𝜑 x=\lim\limits_
𝑢\raa f (𝑢)=A.
	\begin{proof}
由 \lim\limits_
\end{proof}
𝑢\raa f (𝑢)=A 知 \forall\varepsilon>0, \exists 𝜂>0 使得当 0<𝑢-a<𝜂 时, 有 f 𝑢-A<\varepsilon  .
	由 \lim\limits_
x\rax0
𝜑 x=a 知 \exists 𝛿1>0 使得当 0<x-x0<𝛿1 时, 有 𝜑 x-a<𝜂 .
	设 𝑢=𝜑 x 在 𝑈
∘
x0 , 𝛿2 上不等于 a, 则当 0<x-x0<𝛿=min 𝛿1 , 𝛿2 时, 有
0<𝜑 x-a<𝜂 , f 𝜑 x-A<\varepsilon  .
	所以 \lim\limits_
x\rax0
f 𝜑 x=A.
	注记 如果 f a=\lim\limits_
𝑢\raa f (𝑢)=A, 则可以不要求 𝑢=𝜑 x 在点 x0 的某个去心邻域内
不等于 a.
\end{frame}


\begin{frame}
	\begin{example}
求 \lim\limits_
\end{example}
x\ra1
3-x- 1+x
x 2+x-2 .
	该极限是 0
0 型不定式, 分子分母代入均为 0. 利用 a-b=a-b
a+ b 可将极限为
0 的根式化为有理形式, 这种技巧我们会常常使用.
	解 我们先\begin{proof}
\lim\limits_
\end{proof}
x\rax0
x=x0 (x0>0). 注意到 x-x0=x-x0
x+ x0
.
	\forall\varepsilon>0, 令 𝛿=min x0
2 , x0 \varepsilon  . 当 0<x-x0<𝛿 时, 有
x+ x0>x0 , x-x0=x-x0
x+ x0
< x0 \varepsilon 
x0
= \varepsilon  .
	所以 \lim\limits_
x\rax0
x=x0 .



• 现在
\lim\limits_
𝑥→1
3 − 𝑥 − 1+ 𝑥
𝑥 2+ 𝑥 − 2=\lim\limits_
𝑥→1
3 − 𝑥 − 1+ 𝑥
𝑥 − 1 𝑥+ 2 3 − 𝑥+ 1+ 𝑥
=
\lim\limits_
𝑥→1 −2
𝑥+ 2 3 − 𝑥+ 1+ 𝑥=−2
3 2+ 2=− 2
6 .
• 其中最后一步使用了极限的复合函数运算以及 \lim\limits_
𝑥→𝑥0
𝑥=𝑥0 .


