
\end{frame}


\begin{frame}
2.3 极限的性质
	与数列极限类似, 对于函数的六种极限我们均有:
	定理(唯一性) 如果 lim $f(x)$ 存在, 则必唯一.
	定理(局部有界性) 如果 lim $f(x)$ 存在, 则 $f(x)$ 局部有界.
	所谓的局部, 是指极限定义中 $f(x)$ - 𝐴 < 𝜀 成立的某个区间范围. 以
x \ra +\infty 为例, 局部有界是指存在 𝑋 使得 f | 𝑋,+\infty 有界.
	定理(局部保号性) 如果 lim $f(x)$=a>0, 则 $f(x)$ 局部大于 0.
	推论 如果 lim $f(x)$ 存在且局部非负, 则极限非负.
	推论 如果 lim $f(x)$ , limg(x) 存在且局部 $f(x)$ ≥ g(x), 则 lim $f(x)$ ≥
limg(x).
\end{frame}


\begin{frame}
	极限的四则运算性质
	\begin{theorem}
设 lim $f(x)$=𝐴, lim g x=𝐵.
\end{theorem}
	(1) lim f \pm g x=𝐴 \pm 𝐵.
	(2) lim fg x=𝐴𝐵.
	(3) 当 𝐵\neq 0 时, lim f
g x=𝐴
𝐵 .
	\begin{proof}
我们只证明 x \ra x0 的情形, 其它情形类似.
\end{proof}
	(1) \forall 𝜀>0, ∃𝛿1>0 使得当 0 < x - x0 < 𝛿1 时, 有 $f(x)$ - 𝐴 < 𝜀
2;
	∃𝛿2>0 使得当 0 < x - x0 < 𝛿2 时, 有 g x - 𝐵 < 𝜀
2.
	因此, 当 0 < x - x0 < 𝛿=min 𝛿1 , 𝛿2 时, 有
f \pm g x - 𝐴 \pm 𝐵=$f(x)$ - 𝐴 \pm g x - 𝐵 ≤ $f(x)$ - 𝐴 + g x - 𝐵 < 𝜀 .
	所以 lim
x\rax0
f \pm g x=𝐴 \pm 𝐵 .
\end{frame}


\begin{frame}
	(2) 分析 $f(x)$ g x - 𝐴𝐵=$f(x)$ g x - 𝐵 + 𝐵 $f(x)$ - 𝐴
≤ $f(x)$ ⋅ g x - 𝐵 + 𝐵 ⋅ $f(x)$ - 𝐴 .
	由局部有界性, ∃𝛿1 , 𝑀>0 使得当 0 < x - x0 < 𝛿1 时, 有 $f(x)$ ≤ 𝑀.
	\forall 𝜀>0, ∃𝛿2>0 使得当 0 < x - x0 < 𝛿2 时, 有 $f(x)$ - 𝐴 < 𝜀
2 𝐵 +1 ;
	∃𝛿3>0 使得当 0 < x - x0 < 𝛿3 时, 有 g x - 𝐵 < 𝜀
2𝑀.
	因此, 当 0 < x - x0 < 𝛿=min 𝛿1 , 𝛿2 , 𝛿3 时, 有
f x g x - 𝐴𝐵 ≤ 𝑀 ⋅ g x - 𝐵 + 𝐵 ⋅ $f(x)$ - 𝐴
≤ 𝑀 ⋅ 𝜀
2𝑀 + 𝐵 ⋅ 𝜀
2 𝐵 + 1 < 𝜀.
	所以 lim
x\rax0
fg x=𝐴𝐵.
\end{frame}


\begin{frame}
	(3) 我们只需要\begin{proof}
1
\end{proof}
g \ra 1
𝐵 然后利用(2). 这需要对 1
g 在 x0 附近进行估计
使得其有个非零下界.
	对于 𝜀=𝐵
2>0, ∃𝛿1>0 使得当 0 < x - x0 < 𝛿1 时, 有 g x - 𝐵 ≤
𝜀=𝐵
2 . 于是
g x ≥ 𝐵 - g x - 𝐵 ≥ 𝐵
2 ,
1
g x - 1
𝐵=g x - 𝐵
𝐵 ⋅ g x ≤ 2
𝐵 2 g x - 𝐵 .
\end{frame}


\begin{frame}
	\forall 𝜀>0, ∃𝛿2>0 使得当 0 < x - x0 < 𝛿2 时, 有 g x - 𝐵 < 𝐵 2
2 𝜀.
	因此当 0 < x - x0 < 𝛿=min 𝛿1 , 𝛿2 时,
1
g x - 1
𝐵=g x - 𝐵
𝐵 ⋅ g x ≤ 2
𝐵 2 g x - 𝐵 < 𝜀.
	所以 lim
x\rax0
1
g x=1
𝐵. 由(2)可知 lim
x\rax0
f(x)
g x=lim
x\rax0
f x ⋅ lim
x\rax0
1
g x=𝐴
𝐵.
	推论 设 lim $f(x)$=𝐴, 则
	(1) lim Cf x=C𝐴.
	(2) lim $f(x)$ 𝑘=𝐴𝑘 , 𝑘 \in ℕ+ .
\end{frame}


\begin{frame}
	数列与函数的关系
	设 an 是一个数列. 定义 $f(x)$=a x , 它是 1, +\infty 上的一个分段函数.
	如果 lim
n\ra\inftyan=a, 则 \forall 𝜀>0, ∃𝑁 使得当 n>𝑁 时, 有 an - a < 𝜀 . 因此当 x >
𝑁 + 1 时, x>𝑁, $f(x)$ - a < 𝜀 , 所以 lim
x\ra+\inftyf x=a
.
	反过来也成立, 因此 lim
n\ra\inftyan=lim
x\ra+\inftyf x . 从而与函数极限 lim
x\ra+\infty 相关的概念和
性质可以移植到数列上.
	\begin{theorem}
设 lim
\end{theorem}
n\ra\infty an=a, lim
n\ra\infty 𝑏n=𝑏, 则
	lim
n\ra\infty an \pm 𝑏n=a \pm 𝑏, lim
n\ra\infty an 𝑏n=a𝑏, lim
n\ra\infty
an
𝑏n
= a
𝑏 𝑏\neq 0 ,
	lim
n\ra\infty C an=Ca, lim
n\ra\infty an
𝑘=a𝑘 , 𝑘 \in ℕ+ .
\end{frame}


\begin{frame}
	\begin{example}
对于多项式 𝑃(x ), 有 lim
\end{example}
x\rax0
𝑃 x=𝑃 x0 .
	\begin{example}
对于多项式 𝑃 x , 𝑄 x , 如果 𝑄 x0\neq 0, 则 lim
\end{example}
x\rax0
𝑃 x
𝑄 x=𝑃 x0
𝑄 x0
.
	\begin{example}
求 lim
\end{example}
x\ra1
x 2 -1
x 3 -1.
	分析 该极限是 0
0 型不定式.
	解 由于 x \ra 1 时, x - 1\neq 0, 因此 lim
x\ra1
x 2 -1
x 3 -1=lim
x\ra1
x +1
x 2 +x+1=2
3.
	可以看出, 利用极限的四则运算性质, 我们可以得到很多的极限. 下面我们介绍
极限的函数复合运算性质. 我们在未来会看到, 在研究函数连续性和可导性时,
也会通过研究简单函数、四则运算性质和函数复合运算性质来研究复杂函数的
相应性质.
\end{frame}


\begin{frame}
极限的复合运算性质
	\begin{theorem}
设 lim
\end{theorem}
x \rax0
𝜑 x=a, lim
𝑢\raa f (𝑢)=𝐴. 若函数 𝑢=𝜑 x 在点 x0 的某个去心邻域内不等
于 a, 则
lim
x \rax0
f 𝜑 x=lim
𝑢\raa f (𝑢)=𝐴.
	\begin{proof}
由 lim
\end{proof}
𝑢\raa f (𝑢)=𝐴 知 \forall 𝜀>0, ∃𝜂>0 使得当 0 < 𝑢 - a < 𝜂 时, 有 f 𝑢 - 𝐴 < 𝜀 .
	由 lim
x\rax0
𝜑 x=a 知 ∃𝛿1>0 使得当 0 < x - x0 < 𝛿1 时, 有 𝜑 x - a < 𝜂 .
	设 𝑢=𝜑 x 在 𝑈
∘
x0 , 𝛿2 上不等于 a, 则当 0 < x - x0 < 𝛿=min 𝛿1 , 𝛿2 时, 有
0 < 𝜑 x - a < 𝜂 , f 𝜑 x - 𝐴 < 𝜀 .
	所以 lim
x\rax0
f 𝜑 x=𝐴.
	注记 如果 f a=lim
𝑢\raa f (𝑢)=𝐴, 则可以不要求 𝑢=𝜑 x 在点 x0 的某个去心邻域内
不等于 a.
\end{frame}


\begin{frame}
	\begin{example}
求 lim
\end{example}
x\ra1
3-x- 1+x
x 2 +x-2 .
	该极限是 0
0 型不定式, 分子分母代入均为 0. 利用 a - 𝑏=a-𝑏
a+ 𝑏 可将极限为
0 的根式化为有理形式, 这种技巧我们会常常使用.
	解 我们先\begin{proof}
lim
\end{proof}
x\rax0
x=x0 (x0>0). 注意到 x - x0=x-x0
x+ x0
.
	\forall 𝜀>0, 令 𝛿=min x0
2 , x0 𝜀 . 当 0 < x - x0 < 𝛿 时, 有
x + x0>x0 , x - x0=x - x0
x + x0
< x0 𝜀
x0
= 𝜀 .
	所以 lim
x\rax0
x=x0 .
