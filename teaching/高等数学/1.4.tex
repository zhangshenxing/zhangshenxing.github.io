\section{一些常用不等式和等式}

\subsection{三角函数的基本不等式}

\begin{frame}{三角函数的基本不等式}
	\onslide<+->
	\begin{alertblock}{三角函数的基本不等式}
		当 $x\in\left(0,\dfrac\pi2\right)$ 时, 有 $\sin x<x<\tan x$.
	\end{alertblock}
	\onslide<+->
	\begin{proof}
		作半径为 $1$ 的圆, 如图所示.
		\onslide<+->{\begin{tikzpicture}[overlay,xshift=4.5cm,yshift=-3.4cm]
				\coordinate (A) at (1,0);
				\coordinate (B) at (0,0);
				\coordinate (C) at (1,{sqrt 3});
				\draw[thick,dcolorb] (A)--(B)--(C);
				\draw[dcolorb,thick] pic [draw, "$x$",angle eccentricity=2,angle radius=2mm] {angle};
				\draw[cstcurve,dcolora] (A)--(B)--(C)--cycle;
				\draw[cstcurve,dcolora] (A)--(0.5,{0.5*sqrt 3})--cycle;
				\draw[cstcurve,dcolorb] (B) circle (1);
				\draw
					(1.2,0) node {$A$}
					(-0.2,-0.2) node {$O$}
					(0.45,1.2) node {$B$}
					(1.2,1.73) node {$C$}
					(0.5,-0.2) node {$1$};
			\end{tikzpicture}}
		\onslide<+->{由
		\[\text{$\triangle OAB$ 的面积 $<$ 扇形 $OAB$ 的面积 $<$ 直角 $\triangle OAC$ 的面积}\]
		}\onslide<+->{可知
		\[\half\sin x<\half x<\half \tan x.\]
		}\onslide<+->{从而命题得证.\qedhere}
		\vspace{1cm}
	\end{proof}
\end{frame}


\begin{frame}{三角函数的基本不等式}
	\onslide<+->
	\begin{block}{结论}
		$\sin x\begin{cases}
			<x,&x>0,\\
			=x,&x=0,\\
			>x,&x<0.		
		\end{cases}$
		从而有 $\abs{\sin x}\le |x|$.
	\end{block}
	\onslide<+->
	\begin{proof*}
		当 $0<x<\dfrac\pi2$ 时, 由前一结论有 $\sin x<x$.
		\onslide<+->{当 $x\ge\dfrac\pi2$ 时,
			\[\sin x\le 1<\frac\pi2\le x.\]
		}\onslide<+->{故 $f(x)=x-\sin x>0, \forall x>0$.}
	
		\onslide<+->{注意到 $f(x)$ 是奇函数, 因此 $x<0$ 时, $f(x)=-f(-x)<0$.\qedhere}
	\end{proof*}
\end{frame}


\subsection{算术不等式}
\begin{frame}{均值不等式}
	\onslide<+->
	\begin{alertblock}{均值不等式}
		对任意 $n$ 个正数 $a_1, a_2, \dots, a_n$, 有
		\[\sqrt[n]{a_1 a_2\cdots a_n}\le\frac{a_1+a_2+\cdots+a_n}n.\]
		等式成立当且仅当所有 $a_i$ 均相等.
	\end{alertblock}
	\onslide<+->
	其中不等式的左侧被称为\emph{几何平均数}, 右侧被称为\emph{算术平均数}.
\end{frame}


\begin{frame}{均值不等式}
	\onslide<+->
	\begin{proof*}
		我们用数学归纳法来证明.
		\onslide<+->{$n=1$ 时显然成立.
		}\onslide<+->{假设对于 $n-1\ge1$ 个数成立. 我们不妨设 $a_n\ge a_i, 1\le i\le n$.
		}
	
		\onslide<+->{令 $x=\dfrac{a_1+\cdots+a_{n-1}}{n-1}$, 则由归纳假设 $x^{n-1}\ge a_1\cdots a_{n-1}$.
		}\onslide<+->{由于 $a_n\ge x$, 故
		\begin{align*}
			\left(\frac{a_1+a_2+\cdots+a_n}n\right)^n
			&=\left(\frac{(n-1)x+a_n}n\right)^n
			 =\left(x+\frac{a_n-x}n\right)^n\\
			&\ge x^n+\rmC_n^1 x^{n-1}\left(\frac{a_n-x}n\right)
			 =x^{n-1}a_n\\
			&\ge a_1\cdots a_{n-1}a_n.
		\end{align*}
		}\onslide<+->{从而命题对 $n$ 成立.
		}\onslide<+->{等式成立当且仅当 $a_n=x=\sqrt[n-1]{a_1\cdots a_{n-1}}$.
		}\onslide<+->{由归纳假设可知此时所有 $a_i$ 均相等.\qedhere}
	\end{proof*}
\end{frame}


\begin{frame}{柯西不等式}
	\beqskip{5pt}
	\onslide<+->
	\begin{theorem}
	对任意 $2n$ 个实数 $a_1, a_2, \dots, a_n, b_1, b_2, \dots, b_n$, 有
	\[\sum_{i=1}^na_i^2\cdot \sum_{i=1}^n b_i^2
	\ge\left(\sum_{i=1}^n a_ib_i\right)^2.\]
	等号成立当且仅当 $b_i$ 全为零或者 $\dfrac{a_1}{b_1}=\cdots=\dfrac{a_n}{b_n}$.
	\end{theorem}
	\onslide<+->
	\begin{proof}
		若 $b_i$ 全为 $0$, 显然成立.
		\onslide<+->{假设 $b_i$ 不全为零, 由于
		\[\sum_{i=1}^n(b_i x-a_i)^2
		=\left(\sum_{i=1}^n b_i^2\right)x^2-2\left(\sum_{i=1}^n a_ib_i\right)x+\left(\sum_{i=1}^n a_i^2\right)\ge0\]
		恒成立,
		}\onslide<+->{因此它的判别式 $\Delta\le 0$, 于是得到柯西不等式.
		}\onslide<+->{等式成立即 $\Delta=0$, 这等价于该方程有解 $x$, 于是 $a_i=xb_i$.\qedhere}
	\end{proof}
	\endgroup
\end{frame}


\begin{frame}{柯西不等式}
	\onslide<+->
	\begin{corollary}
		对任意 $n$ 个正实数 $x_1, x_2, \dots, x_n$, 有
		\[\sqrt{\frac{x_1^2+\cdots+x_n^2}n}\ge \frac{x_1+\cdots+x_n}n\ge \frac1{\frac1{x_1}+\cdots+\frac1{x_n}}.\]
		\vspace{0.5cm}
	\begin{tikzpicture}[overlay,yshift=0.4cm,xshift=1.4cm]
		\draw[decorate,thick,decoration={brace,amplitude=5},dcolorb,visible on=<4->] (2.8,0.1)--(0,0.1);
		\draw[dcolorb,visible on=<4->] (1.4,-0.4) node {平方平均数};
		\draw[decorate,thick,decoration={brace,amplitude=5},dcolorb,visible on=<5->] (8.8,0)--(6.2,0);
		\draw[dcolorb,visible on=<5->] (7.5,-0.5) node {调和平均数};
	\end{tikzpicture}
	\end{corollary}
	\onslide<+->
	\begin{proof}
		令 $a_i=x_i, b_i=1$ 可得第一个不等式.
		\onslide<+->{令 $a_i=\sqrt{x_i}, b_i=\dfrac1{\sqrt{x_i}}$ 可得第二个不等式.\qedhere}
	\end{proof}
\end{frame}


\subsection{三角函数有关等式}
\begin{frame}{积化和差与和差化积公式}
	\begin{block}{积化和差公式}
		\begin{itemize}
			\item $\sin x\sin y=-\dfrac12[\cos(x+y)-\cos(x-y)]$,
			\item $\sin x\cos y=\dfrac12[\sin(x+y)+\sin(x-y)]$,
			\item $\cos x\cos y=\dfrac12[\cos(x+y)+\cos(x-y)]$.
		\end{itemize}
	\end{block}
	\onslide<+->
	\begin{block}{和差化积公式}
		\begin{itemize}
			\item $\sin x\pm \sin y=2\sin\dfrac{x\pm y}2\cos\dfrac{x\mp y}2$,
			\item $\cos x+\cos y=2\cos\dfrac{x+y}2\cos\dfrac{x-y}2$,
			\item $\cos x-\cos y=-2\sin\dfrac{x+y}2\sin\dfrac{x-y}2$.
		\end{itemize}
	\end{block}
\end{frame}


\begin{frame}{倍角公式与万能公式}
	\onslide<+->
	\begin{block}{倍角公式}
		\begin{itemize}
			\item $\sin(2x)=2\sin x\cos x$,
			\item $\cos(2x)=\cos^2x-\sin^2x=2\cos^2x-1=1-2\sin^2x$,
			\item $\tan(2x)=\dfrac{2\tan x}{1-\tan^2x}$.
		\end{itemize}
	\end{block}
	\onslide<+->
	\begin{block}{万能公式}
		令 $t=\tan\dfrac x2$, 则
		\[\sin x=\frac{2t}{1+t^2},\quad
			\cos x=\frac{1-t^2}{1+t^2},\quad
			\tan x=\frac{2t}{1-t^2}.\]
	\end{block}
	\onslide<+->
	万能公式可将 $x$ 的三角函数转化为 $\tan \dfrac x2$ 的有理函数.
\end{frame}

\subsection{其它常见公式}

\begin{frame}{数列求和公式}
	\onslide<+->
	\begin{block}{等差数列求和公式}
		设 $x_n=an+b$, 则
		\[x_1+x_2+\cdots+x_n=\frac{n(x_1+x_n)}2=a\frac{n(n+1)}2+bn.\]
	\end{block}
	\onslide<+->
	\begin{block}{等比数列求和公式}
		设 $x_n=x_1 q^{n-1}$, 则
		\[x_1+x_2+\cdots+x_n=\begin{cases}
			\dfrac{x_1 q^{n-1}}{q-1},& q\neq 1,\\
			nx_1, & q=1.
		\end{cases}\]
	\end{block}
	\onslide<+->
	由此可知
	\[x^n-y^n=(x-y)(x^{n-1}+x^{n-2}y+\cdots+xy^{n-2}+y^{n-1}).\]
\end{frame}


\begin{frame}{自然数的幂次和}
	\onslide<+->
	\begin{block}{自然数的幂次和}
		\begin{itemize}
			\item $1+2+3+\cdots+n=\dfrac{n(n+1)}2$,
			\item $1^2+2^2+3^2+\cdots+n^2=\dfrac{n(n+1)(2n+1)}6$,
			\item $1^3+2^3+3^3+\cdots+n^3=\left(\dfrac{n(n+1)}2\right)^2$.
		\end{itemize}
	\end{block}
\end{frame}


\begin{frame}{拆分技巧}
	\onslide<+->
	\begin{example}
		\begin{align*}
			\sum_{k=1}^n\frac1{k(k+3)}
			&=\frac13\sum_{k=1}^n\left(\frac1k-\frac1{k+3}\right)\\
			&=\frac13\left(1+\frac12+\frac13-\frac1{n+1}-\frac1{n+2}-\frac1{n+3}\right).
		\end{align*}
	\end{example}
	\onslide<+->
	\begin{example}
		\begin{align*}
			\sum_{k=1}^n\frac1{\sqrt{k+2}+\sqrt k}
			&=\sum_{k=1}^n \frac{\sqrt{k+2}-\sqrt k}2\\
			&=\frac{\sqrt{n+2}+\sqrt{n+1}-\sqrt 2-1}2.
		\end{align*}
	\end{example}
\end{frame}


\begin{frame}{二项式展开}
	\onslide<+->
	\begin{block}{二项式展开}
		\[(x+y)^n=\sum_{k=0}^n \rmC_n^k x^ky^{n-k},\qquad
		\text{其中}\ \rmC_n^k=\frac{n!}{k!(n-k)!}.\]	
	\end{block}
	\onslide<+->
	令 $x=y=1$, 则
	\[\sum_{k=0}^n \rmC_n^k=2^n.\]
	\onslide<+->
	令 $x=-1,y=1$, 则
	\[\sum_{k=0}^n (-1)^k\rmC_n^k=\rmC_n^0-\rmC_n^1+\cdots+(-1)^n\rmC_n^n=0.\]
\end{frame}
