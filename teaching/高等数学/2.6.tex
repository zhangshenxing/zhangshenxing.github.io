\end{frame}


\begin{frame}2.6 函的连续性•在客观世界中, 很多现象都是变化例如气温升降植物生长等. 那么何用语言来刻画呢?直理解就自量距离近时值也不会相差太远定义设y=f(x)点x0某个邻域内有果limx\rax0f(x)=fx0, 则称y=f(x)x0处我们可以引入增概念记Δx=x-x0, x\rax0即指Δx\ra0.•Δy=fx-fx0, y=f(x)x0limΔx\ra0Δy=0.
\end{frame}


\begin{frame}•和极限一样, 连续也有单侧的概念:设存在𝛿>0使得y=f(x)x0,x0+𝛿上定义. 如果fx0+=limx\rax0+f(x)=fx0, 则称函y=f(x)点x0处右𝛿>0y=f(x)x0-𝛿,x0fx0-=limx\rax0-f(x)=fx0, y=f(x)x0左理y=fxx0当且仅y=fxx0既又即fx0+=fx0-=fx0.•例由于limx\ra0+x=0=0, 因此x0fx满足fx≤x_2. -x_2≤fx≤x_2夹逼准可知limx\ra0fx=0=f0, 从而f(x)0
\end{frame}


\begin{frame}•定义如果函y=fx在开区间(a,𝑏)内的每一个点都连续, 则称y=fx(a,𝑏)或y=fx是a,𝑏上.y=fx(a,𝑏)且a处左𝑏右y=fx[a,𝑏]y=fxa,𝑏类似地我们可以半闭性从图像看条不断曲线例由于limx\rax0sinx=sinx0对任意实x0均成立因此sinx-\infty,+\infty
数
学(下)•定理设函f1(x)在(a,𝑏)内连续, f2x𝑏,𝑐. 则fx=ቐf1x,a<x<𝑏f𝑏,x=𝑏f2x,𝑏<x<𝑐(a,𝑐)当且仅f1𝑏-=f𝑏=f2𝑏+.•证明x0\ina,𝑏时存f1x0+=f1x0-=f1(x)从而fx0+=fx0-=f(x)fxx0处同fxx0\in𝑏,𝑐x0=𝑏f𝑏-=f1𝑏-,f𝑏+=f2𝑏+因此该命题成立
数
学(下)•推论设函f1(x)在(a,𝑏]上连续, f2x[𝑏,𝑐). 则fx=ቊf1x,a<x<𝑏f2x,𝑏≤x<𝑐(a,𝑐)内当且仅f1𝑏=f2𝑏这两个命题常用于判断分段点处的性例fx=sgnx=ቐ-1,x<00,x=01,x>0. 由f0-=-1,f0+=1,f0=0, 因此fx0既不左又右其它都fx=x=ቊ-x,x<0x,x≥0. -x|x=0=xx=0=0, fx
\end{frame}


\begin{frame}•例fx=ቊx_2+1,x≥13-x,x<1. 由于x_2+1|x=1=2=3-x|x=1, 因此fx处连续.当
a取何值时, 函fx=൝1-cos2xtan2x,x<0a+x,x≥0在x=0解f0-=limx\ra0-1-cos2xtan2x=limx\ra0-122x_2x_2=2=f0=f0+=a,a=2.
\end{frame}


\begin{frame}•例设fx=൝sinxx,x\neq01,x=0, 则f0+=f0-=limx\ra0sinxx=1=f(0), 因此fx处连续. 从这个子可以看出, 函sinxx尽管在0没有定义却通过补充来得到一0的不并非所点都样做称fx为它间断也该三种情形会产生:•(1) f(x0)无意;2) limx\rax0f(x)存3) f(x0)且limx\rax0f(x)但limx\rax0fx\neqfx0.
\end{frame}


\begin{frame}•我们将间断点分为两类.第一x0函fx的, 且limx\rax0-f(x)limx\rax0+f(x)存在•(1) 可去: 若limx\rax0-fx=limx\rax0+f(x)即limx\rax0f(x)但limx\rax0fx\neqfx0或fx0不例如0是sinxx该情形通过补充修改fxx0处fx0=limx\rax0fx其连续性除这唯种调整后使2) 跳跃limx\rax0-fx\neqlimx\rax0+f(x)此时limx\rax0f(x)0sgn(x)
\end{frame}


\begin{frame}•第二类间断点limx\rax0-f(x)和limx\rax0+f(x)至少有一个不存在, 包括无穷: 例如fx=1xlimx\ra0fx=\infty.•振荡fx=sin1x0附近-1,1之限次.求函fx=xsinx的型果是可去补充定义使连续解当sinx\neq0, 即x\neq𝑘\pi,𝑘\inℤ时fx总这由极四则运算法得到x=0limx\ra0fx=1知0f0=1以x=𝑘\pi\neq0于limx\ra𝑘\pifx=\infty, 因此𝑘\pi
\end{frame}


\begin{frame}•例讨论函fx=limn\ra\infty1+x_1+x_2n的连续性.
解求极限得
fx=1+x,x<11,x=10,x=-10,x>1=ቐ1+x,x<11,x=10,x>1或x≤-1.•当x=-1时, 由于f-1-=f-1+=f-1=0, 因此f(x)在x=-1处x=1f1-=2,f1+=0, 1是跳跃间断点可见列不一定还x
y
𝑂
y=f(x)
\end{frame}


\begin{frame}•例求函fx=1exx-1-1的间断点和类型.解当exx-1-1=0时,xx-1=0,x=0. 因此为0,1.•由于
limx\ra0fx=\infty, 0是第二limx\ra1+exx-1=\infty,limx\ra1-exx-1=0,•f1+=0\neqf1-=-1, 1跳跃
\end{frame}


\begin{frame}•例函fx=(e1x+e)tanxx(e1x-e)在-\pi,\pi上的第一类间断点是x=( )(A)0 B)1 C)-\pi2(D)\pi2•解容易看出fx-\pi,\pi为0,1,\pm\pi2.•由于limx\ra1fx=\infty, limx\ra\pm\pi2f(x)=\infty, 因此1,\pm\pi2二.limx\ra0+e1x=+\infty, limx\ra0-e1x=0, limx\ra0+fx=1\neqlimx\ra0-fx=-1, 故0跳跃,选A).
\end{frame}


\begin{frame}•连续函的运算定理设fx,g(x)在x0处均, 则
fx\pmgx,fxgx,fxgxgx0\neq0•x0. 这由概念和极限可得例证明tanx其义域内于sinx,cosx-\infty,+\infty上因此tanx=sinxcosx同cotx如果fx区间𝐼x单调且它反𝐼y=f𝐼x也
\end{frame}


\begin{frame}•我们首先来证明𝐼y也是一段区间. 不妨设fx单调递增y1=fx_1<y<y2=fx_2,x𝑖\in𝐼x, 说y\in𝐼y令a1=x_1,𝑏1=x_2. 若fan+𝑏n2≥yan+1=an,𝑏n+1=an+𝑏n2; 否则an+1=an+𝑏n2,𝑏n+1=𝑏n可以归纳地fan+1≤y≤f𝑏n+1.•现在由fan≤fan+1知an≤an+1≤y故an有上界列从而极限同理𝑏n且0<𝑏n-an=21-nx_2-x_1零因此二者相记为xfan≤f(x)≤f𝑏nlimn\ra\inftyfan=limn\ra\inftyf𝑏n这迫使fx=y显然x\in𝐼xy\in𝐼y
\end{frame}


\begin{frame}•设y0=fx0在𝐼y内部. \forall 𝜀>0,∃𝜀1\in0,𝜀2s.t.𝑈x0,2𝜀1⊆𝐼xy1=fx0-𝜀1,y2=fx0+𝜀1,𝛿=miny2-y0,y0-y1.•\forall y\in𝑈y0,𝛿,y1<y<y2, 从而x0-𝜀1<f-1y<x0+𝜀1. 因此f-1y0处连续若𝐼x为半开闭区间或, 端点也可类似证明例由知arcsinx,arccosx,arctanx,arccotx相应上都是的定理如果奇/偶函fxa,𝑏a>0它-𝑏,-afx[0,a)则-a,a断集合总关于原对称
\end{frame}


\begin{frame}•例证明f(x)=ax(a>1)连续. 设x0是任一实\forall 𝜀>0, 令𝛿=\loga1+𝜀a-x0>0.•\forall x\inx0,x0+𝛿,ax-ax0=ax0ax-x0-1<ax0a𝛿-1=𝜀,limx\rax0ax=ax0.•\forall x\inx0-𝛿,x0,ax0-ax=ax01-ax-x0<ax01-a-𝛿=𝜀1+𝜀a-x0<𝜀从而limx\rax0ax=ax0, 即ax在x0处0<a<1时, limx\rax0ax=limy\ra-x01ay=1a-x0=ax0. 因此指函都的由于单调对0,+\infty上双曲shx,chx,thx也再根据性(chx需要限制0,+\infty)可知反
\end{frame}


\begin{frame}•定理设𝑢=𝜑x, 若函y=f(𝑢)在点𝑢0=a=limx\rax0𝜑x处连续则limx\rax0f𝜑x=flimx\rax0𝜑x=f(a).•证明这由极限的复合性质和义得到. 推论𝑢=𝜑xx0y=f(𝑢)𝑢0=𝜑x0f𝜑xx0例幂y=x\mu0,+\infty内是因为y=x\mu=e\mulnx而对指都我们知道域\mu取值有关奇/偶称可每种情形
\end{frame}


\begin{frame}•结论一切初等函在其有定义的区间内都是连续. 如果fxx0处, 则limx\rax0fx=f(x0).•所以这里我们可看出性用来计算极限例limx\ra1arccos2-x_1+x由于它因此
limx\ra1arccos2-x_1+x=อarccos2-x_1+xx=1=arccos12=\pi3.
\end{frame}


\begin{frame}•例limx\ra0ln5-sin2xx.0不在它的定义域内, 但是
limx\ra05-sin2xx=5-2=3, 于limx\ra0ln5-sin2xx=ln3.
•例limx\ra13-x-1+xx_2+x-2=limx\ra11x+2x-1⋅3-x-1-x3-x+1+x=limx\ra1-2x+23-x+1+x=-23⋅2+2=-26.
\end{frame}


\begin{frame}•例limx\ra0ln1+xx=limx\ra0ln1+x_1x=lnlimx\ra01+x_1x=lne=1.
•例limx\ra0ax-1x=lim𝑡\ra0𝑡\loga1+𝑡𝑡=ax-1\ra0=lna⋅lim𝑡\ra0𝑡ln1+𝑡=lna.limx\ra01+x𝛼-1x. 令𝑡=1+x𝛼-1\ra0, 则ln(1+𝑡)=𝛼ln(1+x),1+x𝛼-1x=𝑡x=𝑡ln1+𝑡⋅𝛼ln1+xx\ra𝛼.故limx\ra01+x𝛼-1x=𝛼
\end{frame}


\begin{frame}•由此我们又得到了一些等价无穷小,
包括前面的有: x\ra0时sinx∼tanx~arcsinx∼arctanx~x,1-cosx=12x_2,
ex-1∼x∼ln1+x,ax-1∼xlnaa>0,a\neq1,1+x𝛼-1∼𝛼x𝛼\neq0•在运用可以将关系中x及极限过程x起换成任意函. 例如sinx-1x+1∼x-1x+1x-1x+1\ra0, 即sinx-1x+1∼x-1x+1x\ra1.
\end{frame}


\begin{frame}•例设a1=xcosx-1,a2=xln1+3x,a3=3x+1-1. 当x\ra0+时, 这些无穷小量按照从低阶到高的排序是( )(A) a1,a2,a3(B) a2,a3,a1(C) a2,a1,a3(D) a3,a2,a1•x\ra0+a1∼x⋅-12x=-12x_2, a2=x⋅3x=x56,a3∼13x故选B).•limx\ra0arctanx_2e2x-1ln1-x分析我们观察发现它00型不定式于可以用等价替换.解limx\ra0arctanx_2e2x-1ln1-x=limx\ra0x_22x(-x)=-12.
\end{frame}


\begin{frame}•例limx\ra0ln(cosx)2x-1ln1-x.分析我们观察发现它是00型不定式, 其中子两层函的复合从外往里逐使用等价无穷小替换解当x\ra0时cosx\ra1.•由于ln1+x∼x(x\ra0), 因此lncosx∼cosx-1~-12x_2x\ra0.•2x-1∼xln2,ln1-x∼-x可知
limx\ra0ln(cosx)2x-1ln1-x=limx\ra0-12x_2xln2⋅-x=12ln2.
\end{frame}


\begin{frame}•例若limx\ra0sinxex-acosx-𝑏=5, 则a=________, 𝑏=________.•解由于limx\ra0sinxcosx-𝑏=0, 因此
limx\ra0ex-a=limx\ra0sinxcosx-𝑏limx\ra0sinxex-acosx-𝑏=0,•a=1. 是limx\ra0sinxex-1=1,limx\ra0cosx-𝑏=1-𝑏=5,𝑏=-4.
\end{frame}


\begin{frame}•例limn\ra\inftyn+n-n.分析这是\infty-\infty型不定式, 我们用变量替换𝑡=1n\ra0. 解limn\ra\inftyn+n-n=lim𝑡\ra01𝑡+1𝑡-1𝑡=lim𝑡\ra01+𝑡-1𝑡=lim𝑡\ra011+𝑡+1=12.
\end{frame}


\begin{frame}•定理如果lim𝑢x=1,lim𝑣x=\infty, 则lim𝑢x𝑣x=elim𝑢x-1𝑣x.证明由于ln𝑢x∼𝑢x-1, 因此
limln𝑢x⋅𝑣x=lim𝑢x-1𝑣x.ex是连续函, lim𝑢x𝑣x=elimln𝑢x⋅𝑣x=elim𝑢x-1𝑣x.注可知,1\infty型不式总以化为0⋅\infty.例limn\ra\infty1+𝜆nn=elimn\ra\infty𝜆n⋅n=e𝜆limx\ra01-sin5x_2x=elimx\ra0-2sin5xx=elimx\ra0-10xx=e-10.•limx\ra01+x_2tanx=elimx\ra02xtanx=elimx\ra02xx=e2.
\end{frame}


\begin{frame}•我们来讨论幂指函的极限与各自关系. 假设𝑢x>0.
𝐥𝐢𝐦𝒖
𝐥𝐢𝐦𝒗
𝐥𝐢𝐦𝒖𝒗
𝐴≥0(存在)
𝐵(𝐴𝐵1(\infty都有可能(1\infty型不定式)
𝐴>1(+\infty
+\infty
0<𝐴<1(-\infty
+\infty
𝐴>1(-\infty
0
0<𝐴<1(+\infty
0
0(+\infty
0
0(-\infty
+\infty
+\infty
𝐵>0(或+\infty
+\infty
+\infty
𝐵<0(-\infty
0
+\infty
0(
\end{frame}


\begin{frame}•有限闭区间上连续函的性质在界a,𝑏许多重要, 这些理论和实践中都作用. 我们将介绍几个常见从何角度非直观但严格证明则需较深定义设fx𝐼(内)如果
∃x0\in𝐼使得当x\in𝐼时fx≤fx0(或fx≥fx0),•称fx𝐼最大值fx0(小fx0), 记做maxx\in𝐼fxminx\in𝐼fx即
fx0=maxx\in𝐼fxfx0=minx\in𝐼fx
\end{frame}


\begin{frame}•注意supx\in𝐼f(x)和maxx\in𝐼fx的不同. maxx\in𝐼fx如果存在, 则必定是某一点函值但supx\in𝐼f(x)有可能并即使supx\in𝐼f(x)\neq+\infty.•最大或小唯x0x0却未例fx=sinxx=2𝑘\pi+\pi2处取得1, x=2𝑘\pi-\pi2-1, 其中𝑘\inℤ
\end{frame}


\begin{frame}•定理(最值)如果函fx在有限闭区间a,𝑏上连续, 则fxa,𝑏一大和小.
例图的fxx_1处取得fx_1=minx\in[a,𝑏]fxx_2fx_2=maxx\in[a,𝑏]fxx
y
𝑂
y=f(x)
ax_1x_2𝑏
\end{frame}


\begin{frame}•推论(有界定理)如果函fx在限闭区间a,𝑏上连续, 则fxa,𝑏.对于开内的或不这些结一成立例fx=tanx-\pi2,\pi2但无也最大值小fx=x_2-\infty,\infty为0.•fx=൝1x,-1≤x≤1,x\neq0;0,x=0[-1,1]
\end{frame}


\begin{frame}•定理(介值)如果函fx在有限闭区间a,𝑏上连续, fa\neqf𝑏C于fa,f𝑏之则∃𝜉\in(a,𝑏)使得f𝜉=C.
推论fxa,𝑏设𝑀=fx_1为最大𝑚=fx_2小𝑀>𝑚𝑀>C>𝑚∃𝜉x_1,x_2f𝜉=Cfxa,𝑏它的域也是x
y
𝑂
y=f(x)
a𝑏
f𝑏
fa
𝑀
𝑚
x_1
x_2
C
𝜉1𝜉2𝜉3
\end{frame}


\begin{frame}•推论(零点定理)如果函fx在有限闭区间a,𝑏上连续, 且faf𝑏<0, 则∃𝜉\ina,𝑏使得f𝜉=0.•从图像看的曲线两个端分别位于x轴侧该与x至少一交. 这是fx说fx=0开a,𝑏内根更常见情形faf𝑏≤0, ∃𝜉\ina,𝑏f𝜉=0.
x
y
𝑂
y=f(x)
a𝑏
f𝑏
fa
𝜉1𝜉2𝜉3
\end{frame}


\begin{frame}•例证明方程x=cosx在0,\pi2内有解. 令fx=x-cosx, 则它0,\pi2上连续且f0=-1<0,f\pi2=\pi2>0. 由零点定理知∃𝜉\in0,\pi2使得f𝜉=0, 即fx0,\pi2于fx是单调递增函因此这个唯一的我们可以用计算器快速求近似值随便从如x0=1开始xn+1=cosxn设x=cosxxn+1-x=cosxn-cosx=-2sinxn+x_2sinxn-x_2≤2⋅xn-x_2=xn-x.x0>x归纳-1nxn-x>0. 而{x_2n},x_2n-1界列极限a,𝑏推公式两边同时取a=cos𝑏,𝑏=cosa和差化积a-𝑏≤2sina-𝑏2≤a-𝑏,a=𝑏,a=cosaa=x
\end{frame}


\begin{frame}•例证明方程x3-4x_2+1=0在0,1内有解. 令fx=x3-4x_2+1, 则它0,1上连续, f0=1>0,f1=-2<0. 由零点定理知∃𝜉\in(0,1)使得f(𝜉)=0.•对于函fx我们可以利用二分法来逼近的不妨设fx满足fa<0,f𝑏>0. 记a0=a,𝑏0=𝑏如果fa+𝑏2≤0, a1=a+𝑏2,𝑏1=𝑏; fa+𝑏2>0, a1=a,𝑏1=a+𝑏2.•递归地构造处一串单增界列an和减𝑏n因此者极限存之差an-𝑏n趋相同这个是fx
\end{frame}


\begin{frame}•例设函fx在0,1上连续, f0+f1=1. 证明0,1至少存一点𝜉使得f(𝜉)=𝜉. 分析这种问题般条件为个满足某些性质然后等式我们从结论构造辅助𝐹将其变成找𝐹零的利用验(1) 𝐹闭区间[a,𝑏]; 2) 𝐹a𝐹𝑏≤0.•当号立时还需要单独讨令𝐹x=fx-x则它0,1且
𝐹1=f1-1=-f0=-𝐹0.•如果f0=0, 取𝜉=0即可f0\neq0,𝐹1𝐹0=-f02<0, 而由定理知∃𝜉\in(0,1)𝐹𝜉=0, f(𝜉)=𝜉
\end{frame}


\begin{frame}•例设函fx在0,2a上连续, f0=f(2a). 证明存𝜉\in0,a使得f(a+𝜉)=f(𝜉).•令𝐹x=fa+x-f(x)则它0,a且
𝐹a=f2a-fa=f0-fa=-𝐹0.•如果𝐹0=0, 取𝜉=0即可𝐹0\neq0,𝐹a𝐹0=-𝐹02<0, 从而由零点定理知∃𝜉\in(0,a)𝐹𝜉=0, f(a+𝜉)=f𝜉
\end{frame}


\begin{frame}•例一名游客去黄山二日. 第天早上8:00从脚出发, 经过6个小时到达顶沿着的路线恰好也花了证明这两内在某相同间点地令fx为自x距离gxx则和f6=g0\neq0. 另方面f0=g6=0.•设𝐹x=fx-g(x)它0,6连续𝐹0=-g0,𝐹6=f6=g(0). 因此𝐹0𝐹6=-g02<0, 而由零定理知存𝜉\in(0,6)使得𝐹(𝜉)=𝜉𝜉后想象该且进度完全致那么然会遇
\end{frame}


\begin{frame}综合训练•例求limx\ra0+x_1x, 其中[x]表示不超过x的最大整. 解我们利用有界函乘以无穷小仍然是由于0≤1x-1x<1, 且limx\ra0+x=0, 因此limx\ra0+x_1x-1x=limx\ra0+1-x_1x=0. 从而limx\ra0+x_1x=1.•另夹逼准则x≤x<x+1, x-1<x≤x当x>0时x_1x-1<x_1x≤x⋅1x即1-x<x_1x≤1.•limx\ra0+1-x=limx\ra0+1=1, 可知limx\ra0+x_1x=1.
\end{frame}


\begin{frame}•例求limx\ra0x_2+cosx_1sin2x. 解注意到这是1\infty型不定式limx\ra0x_2+cosx_1sin2x=elimx\ra0x_2+cosx-1sin2x由于limx\ra0x_2+cosx-1sin2x=limx\ra0x_2-2sin2x_2x_2=1-limx\ra02sin2x_2x_2=12, 因此原=e12.•红色等能直接代入sinx∼x
\end{frame}


\begin{frame}•例求limx\ra\infty2\piarctanx-x_2+1x.解
limx\ra\infty2\piarctanx-x_2+1x=2\pilimx\ra\inftyarctanx-limx\ra\inftyx_2+1x=2\pi⋅\pi2-1=0.•正确的limx\ra+\infty2\piarctanx-x_2+1x=2\pilimx\ra+\inftyarctanx-limx\ra+\inftyx_2+1x=2\pi⋅\pi2-1=0
limx\ra-\infty2\piarctanx-x_2+1x=2\pilimx\ra-\inftyarctanx-limx\ra-\inftyx_2+1x=2\pi⋅-\pi2+1=0.•因此limx\ra\infty2\piarctanx-x_2+1x=0.•注遇到arctanx,e1x,x,x这些函时, 要当心是否区分左右极限XX
\end{frame}


\begin{frame}•例设fx=ax3+𝑏x_2+𝑐xx_2-1满足limx\ra\inftyfx=2,limx\ra1fx=𝑑. 求常a,𝑏,𝑐,𝑑的值解由limx\ra\inftyfx=2可知a=0,𝑏=2.•limx\ra12x_2+𝑐x=limx\ra1x_2-1⋅limx\ra12x_2+𝑐xx_2-1=0⋅𝑑=02+𝑐=0, 因此𝑐=-2, 𝑑=limx\ra1fx=limx\ra12x_2-2xx_2-1=limx\ra12xx+1=1.•lim𝑢x𝑣x型极限, 当𝑢x\raC\neq0而𝑣x\ra0时它趋于无穷limx\ra12x_2+𝑐xx_2-1存在其分子必定为0.
\end{frame}


\begin{frame}•例设列an满足0<an<1,an+11-an>14. 证明an收敛, 并求它的极限.我们利用单调有界准则由于题目中是关an+1因此猜测an+1≥an解均值不等式an1-an≤14, an1-an<an+11-an,an<an+1. 从而an存在为𝐴an+11-an>14两边取可得𝐴1-𝐴≥14,𝐴-122≤0,𝐴=12. limn\ra\inftyan=12.
\end{frame}


\begin{frame}•例方程xex-cosx=0在0,+\infty内有几个实根?我们想要用零点定理, 需将函限闭区间上. 解当x>1时xex>e因此xex-cosx=01,+\infty没设fx=xex-cosx由于f0=-1<0,f1=e-cos1>0, fx[0,1]x,ex0,+\infty非负且单调递增xexcosx0,\pi减fx0,1从而fx0,1只一综所述xex-cosx=00,+\infty1
\end{frame}


\begin{frame}•例求所有满足fx+y=fx+f(y)的连续函f:ℝ\raℝ. 解令x=0, 则可知f0=0.•y=𝑘x, f𝑘+1x=fx+f𝑘x,f𝑘+1x-f𝑘x=f(x).•对于正整nfnx=fnx-f0=Σn𝑘=1f𝑘x-f𝑘-1x=nf(x)y=-xf0=fx+f-xf-x=-f(x)从而任意nfnx=nfxx=𝑚nf𝑚=fn⋅𝑚n=nf𝑚n,f𝑚n=f𝑚n=𝑚f1n即理x,fx=xf(1).•实x存在列xn\rax是fx=limn\ra\inftyfxn=limn\ra\inftyxnf1=f1x因此fx=𝑘x𝑘为一
\end{frame}


\begin{frame}•例求所有满足fx+y=fx⋅f(y)的连续函f:ℝ\raℝ. 解令x=y, 则f2x=fx_2,fx=fx_22≥0.•若存在x使得fx=0, fy=fx⋅fy-x=0, f=0.•f\neq0, 对任意x,fx>0. gx=lnf(x)gx+y=gx+g(y)且g因此gx=𝑘xfx=e𝑘xa=e𝑘fx=axfx=0或axa>0.
\end{frame}


\begin{frame}•例求所有满足f2x=fx的连续函f:ℝ\raℝ. 解显然f2nx=fx, n为任意正整对于afa2n=fa由a2n\ra0, 是f0=limn\ra\inftyfa2n=f(a)从而f=C常值若只要f在0以外定义则存非f设gx=f2x,x>0, gx+1=gx我们可取gx=sin2\pix,fx=sin2\pi\log2x