
\end{frame}


\begin{frame}
第 二 章 极 限 和 连 续
\end{frame}


\begin{frame}
2.1 数列的极限
	\begin{example}
我们知道, 双曲线 x 2
\end{example}
a2 - y2
𝑏2=1 的图像有两条渐近线 x
a \pm y
𝑏=0.
	所谓的渐近线, 指的是曲线上一点 𝑀 沿曲线趋近于无穷远或某个间断点
时, 如果 𝑀 到一条直线的距离无限趋近于零,那么这条直线称为这条曲
线的渐近线。
	在这个定义中, 为了严格地描述“趋近”的含义, 我们需要引入极限的概
念.
	\begin{example}
一辆汽车沿着直线行驶, 它的瞬时速率 𝑣 定义为 Δ𝑠
\end{example}
Δ𝑡 , 其中 Δ𝑠 为一小段
时间 Δ𝑡 内的位移, Δ𝑡 需要充分小. 严格地描述它需要引入极限的概念.
\end{frame}


\begin{frame}
	\begin{example}
我国古代数学家刘徽为了计算圆周率 \pi, 采用无限逼近的思
\end{example}
想建立了割圆法.
	计算单位圆内接正六边形的面积 𝐴1=3 sin \pi
3=3 3
2 .
	计算单位圆内接正12边形的面积 𝐴2=6 sin \pi
6=3.
	计算单位圆内接正24边形的面积 𝐴3=12 sin \pi
12=3 6 - 2 .
	计算单位圆内接正3 ⋅ 2n 边形的面积 𝐴n=3 ⋅ 2n-1 sin \pi
3⋅2n-1 .
	如此下去, 这个面积越来越接近圆的面积 \pi. 其中正弦值可以通
过半角公式计算得到.
	为什么这样下去会越来越趋近于 \pi 呢? 这也需要用到极限的概
念.
\end{frame}


\begin{frame}
	数列极限的定义
	极限的非严格定义 对于一个数的过程和实数 a, 如果对于任意的正数 𝜀 >
0, 均存在某个过程的截断, 在这个截断之后, 这个过程和 a 相差不超过 𝜀.
	我们先来看数列的极限. 所谓(无穷)数列, 就是指按一定规则排列的无穷多
个实数 a1 , a2 , ... , an , ...
	记为 {an }, an 被称为第 n 项, 用于描述所有项的式子 an=f n 被称为
它的通项.
	注意和集合不同, 这里 a𝑖 有顺序, 而且可以有相同的.
	数列等价于一个函数 f: ℕ+=1,2,3, ... \ra ℝ.
\end{frame}


\begin{frame}
	不同的数列当 n \ra \infty 时具有不同的表现行为.
	1
2n : 1
2 , 1
4 , 1
8 , 1
16 , 1
32 ... 递减地越来越接近 0
	n : 1,2,3,4,5, ... 无限增大
	3 ⋅ 2n-1 sin \pi
3⋅2n-1 : 3 3
2 , 3,3 6 - 2 , 24 sin \pi
24 , ... 递增地越来越接近 \pi
	1 + -1 n
n : 0, 3
2 , 2
3 , 5
4 , 4
5 , 7
6 , 6
7 , ... 交错地越来越接近 1.
	-1 n + 1
n : 0, 3
2 , - 2
3 , 5
4 , - 4
5 , 7
6 , - 6
7 , ... 交错地分别越来越接近 1 和 -1.
\end{frame}


\begin{frame}
	定义 设有数列 an 和常数 a. 如果
\forall 𝜀>0, ∃𝑁 使得当 n>𝑁 时, 有 an - a < 𝜀,
	则称 a 为 an 当 n \ra \infty 时的极限, 记为 lim
n\ra\infty an=a 或 an \ra a n \ra \infty .
	如果不存在这样的常数 a, 则称该数列发散(没有极限, 不收敛).
	我们将红字部分称为 𝜀-𝑁 语言.
	𝜀 的任意性保证了 an 和 a 可以任意接近. 由于 𝜀 小的情形可以推出更大
的 𝜀 成立, 因此我们实际只需要考虑很接近 0 的 𝜀.
	而 𝑁=𝑁𝜀 则与 𝜀 相对应, 不同的 𝜀 可能对应不同的 𝑁𝜀 . 该数值可以换成
任何一个比它大的数值, 所以我们只关心它的存在性, 而不关心它具体的
数值. 所以我们可以根据需要假设 𝑁>0 或者 𝑁 是正整数.
\end{frame}


\begin{frame}
	\begin{example}
证明当 q < 1 时 lim
\end{example}
n\ra\infty qn=0.
	分析 这种问题的证明通常分为两步:
	估计 an - a , 得到它和 n 的不等式关系, 从而求得 𝑁=𝑁𝜀 . 这个过
程中可以进行适当的放缩.
	将上述 𝑁 代入极限的定义中.
	\begin{proof}
qn - 0=q n < 𝜀, n>\log q 𝜀 .
\end{proof}
	\forall 𝜀>0, 令 𝑁=\log q 𝜀. 当 n>𝑁 时, 有 qn - 0=q n < 𝜀. 所以
lim
n\ra\infty qn=0.
	对于其它情形, 我们只需替换红字部分.
\end{frame}


\begin{frame}
	\begin{example}
\begin{proof}
\end{example}
lim
\end{proof}
n\ra\infty
sin n
n=0.
	\begin{proof}
我们有 sin n
\end{proof}
n - 0 ≤ 1
n.
	\forall 𝜀>0, 令 𝑁=1
𝜀 . 当 n>𝑁 时, 有
sin n
n - 0 ≤ 1
n < 𝜀.
	所以 lim
n\ra\infty
sin n
n=0.
\end{frame}


\begin{frame}
	\begin{example}
\begin{proof}
\end{example}
lim
\end{proof}
n\ra\infty
2n2 +2n-4
n2 -8=2.
	\begin{proof}
我们有 2n2 +2n-4
\end{proof}
n2 -8 - 2=2n+12
n2 -8 . 若 n ≥ 12, 则
2n + 12
n2 - 8 ≤ 3n
n2 - n=3
n - 1 .
	\forall 𝜀>0, 令 𝑁=max 1 + 3
𝜀 , 12 . 当 n>𝑁 时, 有
2n2 + 2n - 4
n2 - 8 - 2 ≤ 3
n - 1 < 𝜀.
所以 lim
n\ra\infty
2n2 +2n-4
n2 -8=0.
\end{frame}


\begin{frame}
	\begin{example}
"极限 lim
\end{example}
n\ra\infty an=a 存在" 的充要条件是 "\forall 𝜀>0, ( )".
	(A) 必有无穷多项 an 满足 an - a < 𝜀
	(B) 所有项 an 满足 an - a < 𝜀
	(C) 只有有限项 an 满足 an - a ≥ 𝜀
	(D) 可能有无穷多项 an 满足 an - a ≥ 𝜀
	解 \forall 𝜀>0, 存在正整数 𝑁 使得当 n>𝑁 时, 有 an - a < 𝜀. 这等价于至
多只有有限项 a1 , ... , a𝑁 满足 an - a ≥ 𝜀. 故选 C, 而 BD 均不正确.
	对于 A , 反\begin{example}
an=-1 n , a=1.
\end{example}
