\section{数列的极限}

\subsection{极限的引入}
% \begin{frame}{极限的引入}
% 	\onslide<+->
% 	\begin{example*}
% 		我们知道, 双曲线 $\dfrac{x^2}{a^2}-\dfrac{y^2}{b^2}=1$ 的图像有两条渐近线 $\dfrac xa\pm\dfrac yb=0$.

% 		\onslide<+->{所谓的渐近线, 指的是当曲线上一点 $M$ 沿曲线\alert{趋近}于无穷远或某个间断点时, 如果 $M$ 到一条直线的距离无限\alert{趋近}于零, 那么这条直线称为这条曲线的\emph{渐近线}.
% 		}\onslide<+->{在这个定义中, 为了严格地描述``趋近''的含义, 我们需要引入极限的概念.}
% 	\end{example*}
% 	\onslide<+->
% 	\begin{example*}
% 		一辆汽车沿着直线行驶, 它的\alert{瞬时速率} $v$ 定义为 $\dfrac{\Delta s}{\Delta t}$, 其中 $\Delta s$ 为一小段时间 $\Delta t$ 内的位移, $\Delta t$ 需要\alert{充分小}.
		
% 		\onslide<+->{严格地描述它需要引入极限的概念.}
% \end{example*}
% \end{frame}


% \begin{frame}{极限的引入}\small
% 	\begin{example}
% 		我国古代数学家刘徽为了计算圆周率 $\pi$, 采用\alert{无限逼近}的思想建立了割圆法.
% 		\begin{tikzpicture}[overlay,xshift=-0.5cm,yshift=-1.7cm,scale=1.5]
% 			\draw[cstcurve,dcolora] (0,0) circle (1);
% 			\foreach \j in {1,2,3,...,6}{
% 				\filldraw[cstcurve,dcolorb,cstfill,visible on=<2>] (0,0)--({cos (60*\j)},{sin (60*\j)})--({cos(60*(\j+1))},{sin(60*(\j+1))})--cycle;
% 			}
% 			\foreach \j in {1,2,3,...,12}{
% 				\filldraw[thick,dcolorb,cstfill,visible on=<3>] (0,0)--({cos (30*\j)},{sin (30*\j)})--({cos(30*(\j+1))},{sin(30*(\j+1))})--cycle;
% 			}
% 			\foreach \j in {1,2,3,...,24}{
% 				\filldraw[semithick,dcolorb,cstfill,visible on=<4>] (0,0)--({cos (15*\j)},{sin (15*\j)})--({cos(15*(\j+1))},{sin(15*(\j+1))})--cycle;
% 			}
% 			\foreach \j in {1,2,3,...,48}{
% 				\filldraw[thin,dcolorb,cstfill,visible on=<5->] (0,0)--({cos (7.5*\j)},{sin (7.5*\j)})--({cos(7.5*(\j+1))},{sin(7.5*(\j+1))})--cycle;
% 			}
% 		\end{tikzpicture}
% 		\onslide<+->{
% 		\begin{itemize}
% 			\item 计算单位圆内接正 $6$ 边形的面积 $A_1=3\sin\dfrac\pi3=\dfrac{3\sqrt3}2$.
% 			\item 计算单位圆内接正 $12$ 边形的面积 $A_2=6\sin\dfrac\pi6=3$.
% 			\item 计算单位圆内接正 $24$ 边形的面积 $A_3=12\sin\dfrac\pi{12}=3(\sqrt6-\sqrt2)$.
% 			\item 计算单位圆内接正 $3\cdot 2^n$ 边形的面积 $A_n=3\cdot2^{n-1}\sin\dfrac\pi{3\cdot2^{n-1}}$.
% 		\end{itemize}
% 		}\onslide<+->{如此下去, 这个面积\alert{越来越}接近圆的面积 $\pi$. 其中正弦值可以通过半角公式计算得到.
% 		}\onslide<+->{为什么这样下去会越来越趋近于 $\pi$ 呢? 这也需要用到极限的概念.}
% 	\end{example}
% \end{frame}

% \subsection{极限的定义}
% \begin{frame}{数列极限的定义}
% 	\onslide<+->
% 	极限可以按如下方式理解: 给定一个函数 $y=f(x)$, 当 $x$ 越来越趋向而且无限接近于某个状态(称之为\emph{极限过程})时, $y$ 越来越趋向且无限接近某个值 $A$, 则 $A$ 就是 $y=f(x)$ 关于这个极限过程的\emph{极限}.

% 	\onslide<+->
% 	如何用严格的数学语言来描述呢? 
% 	\onslide<+->
% 	我们先考虑数列的情形.
% 	\onslide<+->
% 	所谓的\emph{(无穷)数列}是指按一定规则排列的无穷多个实数
% 	\[a_1,a_2,\dots,a_n,\dots\]
% 	\onslide<+->
% 	记为 $\{a_n\}_{n\ge1}$, $a_n$ 被称为\emph{第 $n$ 项}, 用于描述所有项的式子 $a_n=f(n)$ 被称为它的\emph{通项}.
% 	\onslide<+->
% 	注意和集合不同, 这里 $a_i$ 有顺序, 而且可以有相同的.

% 	\onslide<+->
% 	事实上一个数列和一个定义域是全体正整数的函数
% 		\[f:\{1,2,3,\dots\}\ra\BR\]
% 	是一回事.
% \end{frame}


% \begin{frame}{极限的例子}
% 	\onslide<+->
% 	\begin{example}
% 		\begin{itemize}
% 			\item $\set{\dfrac1{2^n}}: \dfrac12,\dfrac14,\dfrac18,\dfrac1{16},\dfrac1{32},\cdots$ 递减地越来越接近 $0$;
% 			\item $\set{n}: 1,2,3,4,5,\cdots$ 无限增大;
% 			\item $\set{3\cdot 2^{n-1}\sin\dfrac\pi{3\cdot2^{n-1}}}: \dfrac{3\sqrt3}2, 3,3(\sqrt6-\sqrt2),24\sin\dfrac\pi{24}, \cdots$ 递增地越来越接近 $\pi$;
% 			\item $\set{1+\dfrac{(-1)^n}n}:0,\dfrac32,\dfrac23,\dfrac54,\dfrac45,\dfrac76,\dfrac67,\cdots$ 交错地越来越接近 $1$;
% 			\item $\set{(-1)^n+\dfrac1n}:0,\dfrac32,-\dfrac23,\dfrac54,-\dfrac45,\dfrac76,-\dfrac67,\cdots$ 奇数项和偶数项分别交错地越来越接近 $1$ 和 $-1$.
% 		\end{itemize}
% 	\end{example}
% \end{frame}


% \subsection{极限的定义}
% \begin{frame}{数列极限的特点}
% 	\onslide<+->
% 	我们来尝试定义数列的极限: 数列 $\{a_n\}$ 的极限是 $a$ 是指 $a_n$ 越来越接近 $a$, 即 $|a_{n+1}-a|<|a_n-a|$.\fillbrace{\visible<+->{\falseex}}

% 	\onslide<+->
% 	例如数列
% 	\[\set{a_n=\frac{n+1}n}:2, \frac32, \frac43, \frac54,\cdots\]
% 	是单调递减的, 因此越来越接近 $0$.
% 	\onslide<+->
% 	但显然 $1$ 应该才是更符合极限的直观才对.
% 	\onslide<+->
% 	所以我们应当要求 $|a_n-a|$ 越来越趋向于 $0$ 且可以任意地小?

% 	\onslide<+->
% 	事实上, 我们可以看出, 修改数列中任意有限项是不影响它的无限的趋势的, 所以我们\alert{只需要求 $|a_n-a|$ 可以任意地小}即可.
% 	\onslide<+->
% 	例如, 对于 $\forall\varepsilon>0$, 当 $n>\dfrac1\varepsilon$ 时, $\abs{a_n-1}=\dfrac1n<\varepsilon$.
% 	\onslide<+->
% 	这便引出了数列极限的定义.
% \end{frame}


% \begin{frame}{数列极限的定义}
% 	\beqskip{8pt}
% 	\onslide<+->
% 	\begin{definition*}
% 		设有数列 $a_n$ 和常数 $a$. 如果
% 		\[\text{\color{red}$\forall\varepsilon>0, \exists N$ 使得当 $n>N$ 时, 有 $|a_n-a|<\varepsilon$},\]
% 		则称 $a$ 为 \emph{$a_n$ 当 $n \ra \infty$ 时的极限}, 记为
% 		\[\text{\emph{$\lim\limits_{n\ra\infty}a_n=a$} 或 $a_n\ra a (n\ra \infty)$}.\]
% 		\onslide<+->{此时称该数列\emph{收敛}.}

% 		\onslide<+->{如果不存在这样的常数 $a$, 则称该数列\emph{发散}(没有极限, 不收敛).}
% 	\end{definition*}
% 	\onslide<+->
% 	我们将红字部分称为 \alert{$\varepsilon$-$N$ 语言}.
% 	\onslide<+->
% 	注意该定义并不是 $\exists N,\forall\varepsilon$ 使得当 $n>N$ 时, 有 $\abs{a_n-a}<\varepsilon$.
% 	\onslide<+->
% 	对于不同的 $\varepsilon$, $N=N_\varepsilon$ 往往也不相同.
	
% 	\onslide<+->
% 	如果对于 $\varepsilon>0$ 存在这样的 $N$, 那么对于 $\varepsilon'>\varepsilon$, 这个 $N$ 仍然满足: 当 $n>N$ 时, 有 $|a_n-a|<\varepsilon$.
% 	\onslide<+->
% 	所以在实际问题中, 可不妨设 $\varepsilon$ 不太大, 例如 $\varepsilon<1$.
% 	\onslide<+->
% 	同理, 也可不妨设 $N$ 不太小, 例如 $N\ge 99$ 之类的.
% 	\endgroup
% \end{frame}


% \begin{frame}{例题: 极限的定义证明}
% 	\beqskip{4pt}
% 	\onslide<+->
% 	\begin{example}
% 		证明当 $|q|<1$ 时, $\lim\limits_{n\ra\infty}q^n=0$.
% 	\end{example}
% 	\onslide<+->
% 	\begin{analysis}
% 		这种问题的证明通常分为两步:
% 		\begin{itemize}
% 			\item 估计 $|a_n-a|$, 得到它和 $n$ 的不等式关系, 从而求得 $N=N_\varepsilon$. 这个过程中可以进行适当的放缩.
% 			\item 将上述 $N$ 代入极限的定义中.
% 		\end{itemize}
% 	\end{analysis}
% 	\onslide<+->
% 	\begin{proof}
% 		\[|q^n-0|=|q|^n<\varepsilon,\qquad n>\log_{|q|}\varepsilon.\]
% 		\onslide<+->{$\forall\varepsilon>0$, 令 \alert{$N=\log_{|q|}\varepsilon$}.
% 		}\onslide<+->{当 $n>N$ 时, 有 \alert{$|q^n-0|=|q|^ n<\varepsilon$}.
% 		}\onslide<+->{所以 \alert{$\lim\limits_{n\ra\infty}q^n=0$}.\qedhere}
% 	\end{proof}
% 	\onslide<+->
% 	对于其它证明极限的问题, 我们只需替换红字部分.
% 	\endgroup
% \end{frame}


% \begin{frame}{例题: 极限的定义证明}
% 	\onslide<+->
% 	\begin{example}
% 		证明 $\lim\limits_{n\ra\infty}\dfrac{\sin n}n=0$.
% 	\end{example}
% 	\onslide<+->
% 	\begin{proof}
% 		我们有 $\abs{\dfrac{\sin n}n-0}\le\dfrac1n$.
% 		\onslide<+->{$\forall\varepsilon>0$, 令 $N=\dfrac1\varepsilon$.
% 		 }\onslide<+->{当 $n>N$ 时, 有
% 		 \[\abs{\frac{\sin n}n-0}\le\frac1n<\varepsilon.\]
% 		 }\onslide<+->{所以 $\lim\limits_{n\ra\infty}\dfrac{\sin n}n=0$.\qedhere}
% 	\end{proof}
% \end{frame}


% \begin{frame}{例题: 极限的定义证明}\small
% 	\beqskip{5pt}
% 	\onslide<+->
% 	\begin{example}
% 		证明 $\lim\limits_{n\ra\infty}\dfrac{2n^2+2n-4}{n^2-8}=2$.
% 	\end{example}
% 	\onslide<+->
% 	\begin{proof}
% 		我们有 $\abs{\dfrac{2n^2+2n-4}{n^2-8}-2}=\dfrac{2n+12}{n^2-8}$.
% 		\onslide<+->{若 $n\ge12$, 则 $\displaystyle\abs{\frac{2n+12}{n^2-8}}\le\frac{3n}{n^2-n}=\frac3{n-1}$.
% 		}\onslide<+->{$\forall\varepsilon>0$, 令 $N=\max\set{1+\dfrac3\varepsilon,12}$.
% 		 }\onslide<+->{当 $n>N$ 时, 有
% 		\[\abs{\frac{2n^2+2n-4}{n^2-8}-2}\le\frac3{n-1}<\varepsilon.\]
% 		 }\onslide<+->{所以 $\lim\limits_{n\ra\infty}\dfrac{2n^2+2n-4}{n^2-8}=0$.\qedhere}
% 	\end{proof}
% 	\endgroup
% \end{frame}


% \begin{frame}{例题: 极限的定义}
% 	\onslide<+->
% 	\begin{example}
% 		单选题: ``极限 $\lim\limits_{n\ra\infty}a_n=a$ 存在''的充要条件是``$\forall\varepsilon>0$,\fillbrace{\visible<3->{C}}''.
% 		\xx{必有无穷多项 $a_n$ 满足 $|a_n-a|<\varepsilon$}
% 		{所有项 $a_n$ 满足 $|a_n-a|<\varepsilon$}
% 		{只有有限项 $a_n$ 满足 $|a_n-a|\ge \varepsilon$}
% 		{可能有无穷多项 $a_n$ 满足 $|a_n-a|\ge \varepsilon$}
% 	\end{example}
% 	\onslide<+->
% 	\begin{solution}
% 		$\forall\varepsilon>0$, 存在正整数 $N$ 使得当 $n>N$ 时, 有 $|a_n-a|<\varepsilon$.
% 		\onslide<+->{这等价于至多只有有限项 $a_1 ,\dots, a_N$ 满足 $|a_n-a|\ge \varepsilon$. 故选 C, 而 BD 均不正确.
% 		}\onslide<+->{对于 A , 反例 $a_n=(-1)^n, a=1$.}
% 	\end{solution}
% \end{frame}

% \subsection{收敛数列的性质}

% \begin{frame}{数列极限的唯一性}
% 	\onslide<+->
% 	\begin{theorem}[唯一性]
% 		收敛数列的极限是唯一的.
% 	\end{theorem}
% 	\onslide<+->
% 	\begin{proof}
% 		设 $a$ 和 $b$ 都是 $\set{a_n}$ 的极限.
% 		\onslide<+->{$\forall\varepsilon>0, \exists N, M>0$ 使得当 $n>N$ 时 $|a_n-a|<\varepsilon$; 当 $n>M$ 时 $|a_n-b|<\varepsilon$.
% 		 }\onslide<+->{从而由三角不等式
% 		 \[|a-b|\le|a-a_n|+|a_n-b|<2\varepsilon.\]
% 		 }\onslide<+->{若 $a\neq b$, 则可取 $\varepsilon=\dfrac{a-b}2>0$ 代入得到 $2\varepsilon<2\varepsilon$, 矛盾! 因此 $a=b$.\qedhere}
% 	\end{proof}
% \end{frame}


% \begin{frame}{数列极限的有界性}
% 	\onslide<+->
% 	\begin{theorem}[有界性]
% 		收敛数列是有界数列.
% 	\end{theorem}
% 	\onslide<+->
% 	\begin{proof}
% 		设数列 $\set{a_n}$ 收敛到 $a$, 则对于 $\varepsilon=1$, 存在正整数 $N$ 使得当 $n>N$ 时 $|a_n-a|<\varepsilon=1$, 从而 $|a_n|<|a|+1$.
% 		\onslide<+->{因此对于 $M=\max{|a_1|,\dots,|a_N|,|a|+1}$, 有 $|a_n|\le M$. 这说明 $\set{a_n}$ 是有界数列.\qedhere}
% 	\end{proof}
% 	\onslide<+->
% 	收敛数列一定有界, 但反之未必.
% 	\begin{example}
% 		对于数列 $\set{a_n}=(-1)^n$, 该数列是有界的但是不收敛.
% 	\end{example}
% \end{frame}


\begin{frame}{数列极限的保号性}
	\onslide<+->
	\begin{theorem}[保号性]
		\begin{enumerate}
			\item 如果 $\lim\limits_{n\ra\infty}a_n=a\alert{>}0$, 则 $\exists N$ 使得当 $n>N$ 时, 有 $a_n\alert{>}0$.
			\item 如果 $\lim\limits_{n\ra\infty}a_n=a\alert{<}0$, 则 $\exists N$ 使得当 $n>N$ 时, 有 $a_n\alert{<}0$.
		\end{enumerate}
	\end{theorem}
	\onslide<+->
	\begin{proof}
		\enumnum1 对于 $\varepsilon=\dfrac a2>0$, $\exists N$ 使得当 $n>N$ 时, 有 $|a_n-a|<\varepsilon=\dfrac a2$, 从而 $a_n>a-\dfrac a2=\dfrac a2>0$.
		\onslide<+->{\enumnum2 同理可得.\qedhere}
	\end{proof}
	\onslide<+->
	注意这里 $>0$ \alert{不能换成} $\ge 0$, 例如 $\lim\limits_{n\ra\infty}\dfrac{(-1)^n}n=0$.
	\onslide<+->
	$<0$ 也\alert{不能换成} $\le 0$.
\end{frame}


\begin{frame}{数列极限的保号性}
	\onslide<+->
	\begin{corollary}[逆否命题]
		\begin{enumerate}
			\item 如果收敛数列 $\set{a_n}$ 从某项起 $\ge 0$, 则它的极限 $\ge 0$.
			\item 如果收敛数列 $\set{a_n}$ 从某项起 $\le 0$, 则它的极限 $\le 0$.
		\end{enumerate}
	\end{corollary}
	\onslide<+->
	同理, 这里 $\ge$ 也不能换成 $>$, 例如 $\lim\limits_{n\ra\infty}\dfrac1n=0$.
	\onslide<+->
	\begin{corollary}
		如果收敛数列 $\set{a_n}, \set{b_n}$ 满足从某项起 $a_n\ge b_n$, 则 $\lim\limits_{n\ra\infty}a_n\ge\lim\limits_{n\ra\infty}b_n$.
	\end{corollary}
\end{frame}


\begin{frame}{数列与子数列的极限关系}
	\onslide<+->
	对于正整数集的一个无限子集合 $S\subseteq\BN_+$, 将其中元素从小到大排成一列
	\[S=\set{k_1,k_2,\dots,k_n,\dots},\]
	\onslide<+->
	则它对应了数列 $\set{a_n}$ 的一个\emph{子数列}
	\[\set{a_{k_n}}_{n\ge1}: a_{k_1}, a_{k_2}, \dots, a_{k_n}, \dots\]

	\onslide<+->
	特别地, 当 $S$ 为全体正奇数时, 称 $\set{a_{2n-1}}$ 为\emph{奇子数列}; 当 $S$ 为全体正偶数时, 称 $\set{a_{2n}}$ 为\emph{偶子数列}.
\end{frame}


\begin{frame}{数列与子数列的极限关系}
	\onslide<+->
	\begin{theorem}
		$\set{a_n}$ 收敛于 $a$ 当且仅当 $\set{a_{2n-1}}$ 和 $\set{a_{2n}}$ 均收敛于 $a$.
	\end{theorem}
	\onslide<+->
	\begin{proof*}
		必要性($\Rightarrow$): 如果 $\lim\limits_{n\ra\infty} a_n=a$, 则 $\forall\varepsilon>0, \exists N$ 使得当 $n>N$ 时, 有 $|a_n-a|<\varepsilon$.
		\onslide<+->{因此 $|a_{2n-1}-a|<\varepsilon, |a_{2n}-a|<\varepsilon$.
		}\onslide<+->{从而 $\set{a_{2n-1}}$ 和 $\set{a_{2n}}$ 均收敛于 $a$.}

		\onslide<+->{充分性($\Leftarrow$): 如果 $\lim\limits_{n\ra\infty} a_{2n-1}=\lim\limits_{n\ra\infty} a_{2n}=a$, 则 $\forall\varepsilon>0$, $\exists N,M$ 使得当 $n>N$ 时, 有 $|a_{2n-1}-a|<\varepsilon$; 当 $n>M$ 时, 有 $|a_{2n}-a<\varepsilon$.
		}\onslide<+->{所以当 $n>\max\set{2N-1,2M}$ 时, 有  $|a_n-a|<\varepsilon$. 故数列 $\set{a_n}$ 收敛到 $a$.\qedhere}
	\end{proof*}
\end{frame}


\begin{frame}{数列与子数列的极限关系}
	\onslide<+->
	\begin{theorem}
		$\set{a_n}$ 收敛于 $a$ 当且仅当它的所有子数列均收敛于 $a$.
	\end{theorem}
	\onslide<+->
	实际上, 设 $S_1,\dots,S_m\subseteq\BN_+$ 是有限多个无限集合, 且
		\[S_1\cup\cdots\cup S_m\]
	包含了所有大于 $M$ 的正整数.
	\onslide<+->
	那么 $\set{a_n}$ 收敛于 $a\iff$ 每个 $S_i$ 对应子数列均收敛于 $a$.
	\onslide<+->
	这是因为 $\forall\varepsilon>0$, 每个 $S_i$ 中至多只有有限多项不满足 $|a_n-a|<\varepsilon$, 从而一共也只有有限多项不满足这个条件.
\end{frame}


\begin{frame}{数列与子数列的极限关系}	
	\onslide<+->
	然而对于无穷多个 $S_i$, 这是不对的.
	\onslide<+->
	我们将所有 $\dfrac xy$ ($x,y\in\BN_+$)按如下规律排列成数列
	\[\set{a_n=\frac{x_n}{y_n}}:\frac11,\frac12,\frac21,\frac13,\frac22,\frac31,\frac14,\frac23,\frac32,\frac41,\dots\]

	\onslide<+->
	对于每一个正整数 $x$,
	\begin{itemize}
		\item 集合 $S_x=\set{n\mid x_n=x}$ 对应的子数列 $\dfrac11,\dfrac12,\dfrac13,\dots$ 收敛于 $0$;
		\item 无限集合 $S=\set{n\mid x_n=y_n}$ 对应的子数列 $\dfrac11,\dfrac22,\dfrac33,\dots$ 收敛于 $1$.
	\end{itemize} 
	\onslide<+->
	因此数列 $\set{a_n}$ 不收敛.

	\onslide<+->
	在数学中, 常常有这种在有限情形成立, 无限情形不成立的结论. 因此遇到涉及无限的情形要小心.
\end{frame}
