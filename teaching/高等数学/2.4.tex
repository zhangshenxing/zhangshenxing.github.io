\end{frame}


\begin{frame}2.4 无穷小和大•我们知道\lim\limits_fx=A等价于\lim\limits_fx-A=0. 这反映了研究极限可以转化到为0的函上来.定义如果\lim\limits_fx=0, 就称fx该过程时面是一些例子:
x-1x\ra1,xx\ra0+,1xx\ra\infty,sinnnn\ra\infty.•个或者列, 它不具体所谈论需要带
\end{frame}


\begin{frame}•定理在自变量的同一化过程中, 有限个无穷小代和或乘积仍然是. 这由义极四则运算法得到注意多未必还例如n\ra\infty时a𝑚,n=൝1n,0≤𝑚≤n;0,𝑚>n但bn=Σ𝑚=1\inftya𝑚,n=1不n\ra\inftya𝑚,n=൞1n,1≤𝑚≤n;n,n<𝑚≤2n;1,𝑚>2nbn=Π𝑚=1\inftya𝑚,n=1
\end{frame}


\begin{frame}•定理有界函和无穷小的乘积仍然是. 证明我们只x\rax0情形, 其它类似设fxgx则\lim\limits_x\rax0gx=0. fx<M•\forall\varepsilon>0,\exists 𝛿>0使得当0<x-x0<𝛿时gx<\varepsilon M于fxgx<M\cdot\varepsilon M=\varepsilon .所以\lim\limits_x\rax0fxgx=0, 即fxgx推论常
\end{frame}


\begin{frame}•定义在某个极限过程中, 如果1fx为
无穷小则称fx大. 以x\rax0例我们可将其表述\forall M>0,\exists 𝛿>0使得当0<x-x0<𝛿时有fx>M记作\lim\limits_x\rax0fx=\infty. 注意此\lim\limits_x\rax0fx实际上是不存的fx>M换fx>M(或fx<-M)fxx\rax0正负\lim\limits_x\rax0fx=+\infty-\infty. 这等价于1fx局部零对它五种类似地和一函者列具体所谈论需要带味着界但反之未必x\ra\inftyfx=xcosx并
\end{frame}


\begin{frame}•例求\lim\limits_x\ra11x-1-x+2x3-1.•解原式=\lim\limits_x\ra1x_2+x+1-(x+2)x3-1=\lim\limits_x\ra1x_2-1x3-1=\lim\limits_x\ra1x+1x_2+x+1=23.•可以看出\lim\limits_x\ra11x-1=\infty, \lim\limits_x\ra1x+2x3-1=\infty. 这种极限被称为\infty\pm\infty型不定, 通常我们需要将其化成00来处理.
\end{frame}


\begin{frame}•例\lim\limits_x\ra\infty2x_2-xx_2-1=\lim\limits_x\ra\infty2-1x_1-1x_2=21=2,•或者\lim\limits_x\ra\infty2x_2-xx_2-1=\lim\limits_t\ra02t2-1t1t2-1=\lim\limits_t\ra02-t1-t2=21=2.•这种极限被称为\infty\infty型不定式. 我们可以根据需要将x\ra\infty化t=1x\ra0.•\lim\limits_x\ra\inftyxx_2-1=\lim\limits_x\ra\infty1x_1-1x_2=01-0=0.
\end{frame}


\begin{frame}•一般地, 设a0b0\neq0, 则
\lim\limits_x\ra\inftya0x𝑚+a1x𝑚-1+⋯+a𝑚b0xn+b1xn-1+⋯+bn=a0b0,𝑚=n,0,𝑚<n,\infty,𝑚>n.例\lim\limits_x\rax0fx=\infty,\lim\limits_x\rax0gx=\infty, ( ).•(A) \lim\limits_x\rax0fx+gx=\infty(B)\lim\limits_x\rax0fx-gx=\infty•(C) \lim\limits_x\rax0fxgx=\infty(D)\lim\limits_x\rax0fxgx=1•解取gx=\pmfx可知A)(错误. gx=2f(x)\lim\limits_x\rax0fxgx=12, 因此选C).
\end{frame}


\begin{frame}•我们来讨论两个函相加时的极限与各自关系.
如果f存在, g不则由g=f+g-f可知f+g也𝐥𝐢𝐦𝒇
𝐥𝐢𝐦𝒈
𝐥𝐢𝐦𝒇+𝒈
A()
𝐵A+𝐵A局部有界或+\infty
+\infty
+\infty-\infty
-\infty
-\infty都能
+\infty
-\infty\infty\pm\infty型定式
\end{frame}


\begin{frame}•两个函相乘时的极限与各自关系如:
𝐥𝐢𝐦𝒇
𝐥𝐢𝐦𝒈
𝐥𝐢𝐦𝒇𝒈
A(存在)
𝐵A𝐵A\neq0(不或\infty\infty
0(\infty都有可能
0\cdot\infty型定式+\infty
+\infty
+\infty
+\infty
-\infty
-\infty
-\infty
-\infty
+\infty
\end{frame}


\begin{frame}•两个函相除时的极限与各自关系如:
𝐥𝐢𝐦𝒇
𝐥𝐢𝐦𝒈
𝐥𝐢𝐦𝒇𝒈
A(存在)
𝐵\neq0(A𝐵A\neq0(0(\infty
0(0(都有可能
00型不定式\infty
\infty\infty\infty
\end{frame}


\begin{frame}•无穷小的比较我们知道两个加减乘均是, 但商却未必. 例如\lim\limits_x\ra0xx_2=\infty,\lim\limits_x\ra0x_2x=0, \lim\limits_x\ra02xx=2.•这种00型不定式之所以些情形结果同因为分子母趋于零速度x_2x要快此二者相除仍然
\end{frame}


\begin{frame}•定义设在自变量的同一化过程中, 𝛼=𝛼x和𝛽=𝛽x𝛽\neq0是两个无穷小. •(1) 若\lim\limits_𝛼𝛽=0, 则称𝛼𝛽高阶记作𝛼=𝑜𝛽也𝛽𝛼低2) \lim\limits_𝛼𝛽=C\neq0, 𝛼与𝛽特别地如果\lim\limits_𝛼𝛽=1, 𝛼𝛽等价𝛼∼𝛽例x\ra0时x_2x即x_2=𝑜(x).•x\ra1\lim\limits_x\ra1x-1x3-1=13\neq0, 因此x-1x3-1且x3-1∼3x-1.
\end{frame}


\begin{frame}•例设在自变量的同一化过程, 𝛼是𝛽高阶无穷小( 𝛼,𝛽\neq0), 则列结论不正确).•(A) 𝛼𝛽𝛽B) 𝛼𝛽𝛽低C) 𝛼-𝛽𝛽D)𝛼+𝛽𝛽等价解𝛼𝛽𝛽=𝛼\ra0, 因此.𝛼-𝛽𝛽=𝛼𝛽-1\ra-1, 𝛼+𝛽𝛽=𝛼𝛽+1\ra1, 𝛽𝛼/𝛽=𝛽2𝛼未必趋于0, 错误如𝛼=x_2,𝛽=x\ra0.
\end{frame}


\begin{frame}•定理设在自变量的同一化过程中, 𝛼∼𝛼1,𝛽∼𝛽1, 且
\lim\limits_𝛼1𝛽1存则\lim\limits_𝛼𝛽\lim\limits_𝛼𝛽=\lim\limits_𝛼1𝛽1.•证明\lim\limits_𝛼𝛽=\lim\limits_𝛼𝛼1\cdot𝛼1𝛽1\cdot𝛽1𝛽=\lim\limits_𝛼𝛼1\cdot\lim\limits_𝛼1𝛽1\cdot\lim\limits_𝛽1𝛽=\lim\limits_𝛼1𝛽1.•该被称为等价无穷小代换. 由于大时倒因此也有这些结论可以用00型\infty\infty,0\cdot\infty不式注意我们只能对其做相加或减项
\end{frame}


\begin{frame}•如果𝛼\raA\neq0, 𝛽是无穷小, 则\lim\limits_𝛼𝛽A𝛽=\lim\limits_𝛼A=1,𝛼𝛽∼A𝛽.若𝛽=𝑜𝛼1+𝛽𝛼\ra1, 因此𝛼+𝛽=𝛼1+𝛽𝛼∼𝛼即两个阶不同的之和等价于其中较那定义我们称与𝛼𝑟为𝛼𝑘𝑟>0.•𝛼,𝛽均有𝛼𝛽高/低指就相应比注记并所都例1ln1xx\ra0+时但它任意x𝑟
\end{frame}


\begin{frame}•例x+x_2=x_1+x∼x为x\ra0的1阶无穷小. x+x+x∼x_18x34+x+1∼x_18x\ra0+18也可以这么看, x+x∼x,x+x+x∼x_14.•1+x-1-x=2x_1+x+1-x∼xx\ra01求\lim\limits_x\ra\inftyx-sinxx+cosx解该极限\infty\infty型不定式原=\lim\limits_x\ra\infty1-sinxx_1+cosxx由于1x是sinx,cosx有界因此sinxx,cosxx从而=11=1.


