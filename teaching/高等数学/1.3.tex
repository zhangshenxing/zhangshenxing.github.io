\section{初等函数}

\subsection{基本初等函数}

\begin{frame}{常数函数}
	\onslide<+->
	\begin{definition}
		\emph{常数函数}: $y=C, x\in (-\infty, +\infty)$.
	\end{definition}
	\begin{itemize}
		\item 值域为 $\{C\}$.
		\item 有界, 周期 (无最小正周期), 偶函数.
		\item 当 $C=0$ 时, 它也是奇函数.
		\item 图像是过点 $(0,C)$ 且平行于 $x$ 轴的直线.
	\end{itemize}
	\onslide<2->
	\begin{center}
		\begin{tikzpicture}
			\draw[cstaxis] (-3,0)--(3,0);
			\draw[cstaxis] (0,-1)--(0,2);
			\draw[cstcurve,dcolora] (-3,1)--(3,1);
			\draw
				(-0.2,-0.2) node {$O$}
				(2.8,-0.2) node {$x$}
				(-0.2,1.8) node {$y$}
				(-0.2,0.8) node {$C$}
				(1.8,1.3) node[dcolora] {$y=C$};
		\end{tikzpicture}
	\end{center}
\end{frame}


\begin{frame}{指数函数}
	\onslide<+->
	\begin{definition}
		\emph{指数函数}: $y=a^x (a>0, a\neq 1), x \in (-\infty, +\infty)$.
	\end{definition}
	\onslide<+->
	\begin{itemize}
		\item 值域为 $(0, +\infty)$.
		\item 图像过点 $(0,1)$.
		\item 当 $a>1$ 时, 它单调递增; 当 $0<a<1$ 时, 它单调递减.
		\item $y=a^x$ 的图像和 $y=\bigl(\frac1a\bigr)^x$ 的图像关于 $y$ 轴对称.
		\item 当 $a=1$ 时, 它退化为常数函数 $y=1$.
		\item 直线 $y=0$ 是它唯一的渐近线.
	\end{itemize}
	\onslide<2->
	\begin{center}
		\begin{tikzpicture}
			\draw[cstaxis] (-3,0)--(3,0);
			\draw[cstaxis] (0,-0.4)--(0,2.3);
			\draw[cstcurve,dcolora,smooth,domain=-3:1.2] plot (\x,{2^\x});
			\draw[cstcurve,dcolorb,smooth,domain=-3:1.2] plot (-\x,{2^\x});
			\draw
				(-0.2,-0.2) node {$O$}
				(2.8,-0.2) node {$x$}
				(-0.2,2.1) node {$y$}
				(-0.15,0.7) node[visible on=<4->] {$1$}
				(2.2,1.5) node[dcolora] {$y=a^x (a>1)$}
				(-2.6,1.5) node[dcolorb] {$y=a^x (0<a<1)$};
			\fill[cstdot,dcolore,visible on=<4->]	(0,1) circle;
		\end{tikzpicture}
	\end{center}
\end{frame}


\begin{frame}{对数函数}
	\onslide<+->
	\begin{definition}
		\emph{对数函数}: $y=\log_a x, a>0, a\neq 1, x \in (0, +\infty)$.
	\end{definition}
	\onslide<+->
	\begin{itemize}
		\item 对数函数和指数函数互为反函数.
		\item 值域为 $(-\infty, +\infty)$.
		\item 图像过点 $(1,0)$.
		\item 当 $a>1$ 时, 它单调递增;
		\item 当 $0 < a < 1$ 时, 它单调递减.
		\item 直线 $x=0$ 是它唯一的渐近线.
		\item $y=\log_a x$ 的图像和 $y=\log_{\frac1a}	x$ 的图像\\
		关于 $x$ 轴对称.
		\item 称 $\lg x=\log_{10} x$ 为\emph{常用对数}, $\ln x=\log_e x$ 为\emph{自然对数}, 其中无理数 $e=2.71828\cdots$ 被称为\emph{自然对数的底}.
	\end{itemize}
	\onslide<2->
	\begin{tikzpicture}[overlay,xshift=8.2cm,yshift=3.9cm]
		\draw[cstaxis] (-0.5,0)--(3,0);
		\draw[cstaxis] (0,-2.3)--(0,2.3);
		\draw[cstcurve,dcolora,smooth,domain=-2.3:1.5] plot ({2^\x},\x);
		\draw[cstcurve,dcolorb,smooth,domain=-2.3:1.5] plot ({2^\x},-\x);
		\draw
			(-0.2,-0.2) node {$O$}
			(2.8,-0.2) node {$x$}
			(-0.2,2.1) node {$y$}
			(1,-0.3) node[visible on=<5->] {$1$}
			(2,-1.9) node[dcolora] {$y=\log_ax (a>1)$}
			(2.3,1.9) node[dcolorb] {$y=\log_ax (0<a<1)$};
			\fill[cstdot,dcolore,visible on=<5->]	(1,0) circle;
	\end{tikzpicture}
\end{frame}


\begin{frame}{幂函数: 正有理幂次 $\frac pq$, $p,q$ 奇}
	\onslide<+->
	\begin{definition}
		\emph{幂函数}: $y=x^\mu (\mu\neq 0)$.
	\end{definition}
	\onslide<+->
	根据 $\mu$ 的不同, 它的定义域和性质也有所不同.

	\onslide<+->
	当 $\mu$ 是有理数时, 我们可将其表为 $\mu=\frac pq$, 其中 $p,q$ 为互质的整数.

	\onslide<+->
	\noindent\enumnum1 当 $\mu>0, p,q$ 为奇数时,
	\begin{itemize}
		\item 定义域和值域为 $(-\infty, +\infty)$.
		\item 奇函数, 图像过点 $(0,0), (1,1), (-1,-1)$.
		\item $y=x^\mu$ 与 $y=x^{\frac1\mu}$ 互为反函数.
	\end{itemize}
	\vspace{2.2cm}
	\onslide<4->
	\begin{tikzpicture}[overlay,xshift=9.7cm,yshift=2.7cm]
		\draw[cstaxis] (-2.3,0)--(2.3,0);
		\draw[cstaxis] (0,-2.3)--(0,2.3);
		\draw[cstcurve,dcolora,smooth,domain=-1.3:1.3] plot (\x,{\x^3});
		\draw[cstcurve,dcolorb,smooth,domain=-1.3:1.3] plot ({\x^3},\x);
		\draw[cstdash,dcolord,visible on=<7->] (-2,-2)--(2,2);
		\draw
			(-0.2,0.2) node {$O$}
			(2.1,-0.2) node {$x$}
			(-0.2,2.1) node {$y$}
			(0.65,2) node[dcolora] {$y=x^3$}
			(-2.8,-1.3) node[dcolorb] {$y=x^{\frac13}$};
		\begin{scope}[visible on=<6->]
			\draw
				(1,-0.2) node {$1$}
				(-1,0.2) node {$-1$}
				(-0.2,1) node {$1$}
				(0.3,-1) node {$-1$};
			\draw[cstdash,dcolore] (1,0)--(1,1)--(0,1);
			\draw[cstdash,dcolore] (-1,0)--(-1,-1)--(0,-1);
			\fill[cstdot,dcolore]	(1,1) circle;
			\fill[cstdot,dcolore]	(-1,-1) circle;
			\fill[cstdot,dcolore]	(0,0) circle;
		\end{scope}
	\end{tikzpicture}
\end{frame}


\begin{frame}{幂函数: 正有理幂次 $\frac pq$, $p,q$ 一奇一偶}
	\onslide<+->
	\noindent\enumnum2 当 $\mu > 0$, $p$ 偶 $q$ 奇时,
	\begin{itemize}
		\item 定义域为 $(-\infty, +\infty)$, 值域为 $[0, +\infty)$.
		\item 偶函数, 图像过点 $(0,0), (\pm1,1)$.
	\end{itemize}
	\onslide<+->
	\noindent\enumnum3 当 $\mu > 0$, $p$ 奇 $q$ 偶时,
	\begin{itemize}
		\item 定义域和值域为 $[0, +\infty)$.
		\item 图像过点 $(0,0), (1,1)$.
		\item $y = x^\mu$ 与 $y = x^{\frac1\mu}$ 在 $(0, +\infty)$ 的限制互为反函数.
	\end{itemize}
	\onslide<1->
	\begin{center}
		\begin{tikzpicture}
			\draw[cstaxis] (-2.3,0)--(2.3,0);
			\draw[cstaxis] (0,-0.5)--(0,2.3);
			\draw[cstcurve,dcolora,smooth,domain=-1.5:1.5] plot (\x,{\x*\x});
			\draw[cstcurve,dcolorb,smooth,domain=0:1.5,visible on=<4->] plot ({\x*\x},\x);
			\draw[cstdash,dcolord,visible on=<7->] (-0.4,-0.4)--(2,2);
			\draw
				(0.2,-0.2) node {$O$}
				(2.1,-0.2) node {$x$}
				(-0.2,2.1) node {$y$}
				(-1.9,1.5) node[dcolora] {$y=x^2$}
				(3,1.6) node[dcolorb,visible on=<4->] {$y=x^{\half}$};
			\begin{scope}[visible on=<3->]
				\draw
					(1,-0.2) node {$1$}
					(-1,-0.2) node {$-1$}
					(-0.2,1) node {$1$};
				\draw[cstdash,dcolore] (-1,0)--(-1,1)--(1,1)--(1,0);
				\fill[cstdot,dcolore]	(1,1) circle;
				\fill[cstdot,dcolore]	(-1,1) circle;
				\fill[cstdot,dcolore]	(0,0) circle;
			\end{scope}
		\end{tikzpicture}
	\end{center}
\end{frame}


\begin{frame}{幂函数: 负有理幂次 $\frac pq$, $p,q$ 奇}
	\onslide<+->
	\noindent\enumnum4 当 $\mu < 0, p, q$ 为奇数时,
	\begin{itemize}
		\item 定义域和值域为 $(-\infty, 0)\cup(0, +\infty)$.
		\item 奇函数, 图像过点 $(0,0), (1,1), (-1, -1)$.
		\item 有两条渐近线 $x = 0, y = 0$.
		\item $y = x^\mu$ 与 $y = x^{\frac1\mu}$ 互为反函数.
	\end{itemize}
	\vspace{3cm}
	\onslide<1->
	\begin{tikzpicture}[overlay,xshift=9.2cm,yshift=3cm]
		\draw[cstaxis] (-2.8,0)--(2.8,0);
		\draw[cstaxis] (0,-2.8)--(0,2.8);
		\draw[cstcurve,dcolora,smooth,domain=0.4:1.4] plot ({1/\x},{\x^3});
		\draw[cstcurve,dcolora,smooth,domain=-0.4:-1.4] plot ({1/\x},{\x^3});
		\draw[cstcurve,dcolorb,smooth,domain=0.4:1.4,visible on=<5->] plot ({\x^3},{1/\x});
		\draw[cstcurve,dcolorb,smooth,domain=-0.4:-1.4,visible on=<5->] plot ({\x^3},{1/\x});
		\draw[cstdash,dcolord,visible on=<5->] (-2,-2)--(2,2);
		\draw
			(0.2,-0.2) node {$O$}
			(2.6,-0.2) node {$x$}
			(-0.2,2.6) node {$y$}
			(-1.6,-2.2) node[dcolora] {$y=x^{-3}$}
			(2.2,1.2) node[dcolorb,visible on=<5->] {$y=x^{-\frac13}$};
		\begin{scope}[visible on=<3->]
			\draw
				(1,-0.2) node {$1$}
				(-1,0.2) node {$-1$}
				(-0.2,1) node {$1$}
				(0.3,-1) node {$-1$};
			\draw[cstdash,dcolore] (1,0)--(1,1)--(0,1);
			\draw[cstdash,dcolore] (-1,0)--(-1,-1)--(0,-1);
			\fill[cstdot,dcolore]	(1,1) circle;
			\fill[cstdot,dcolore]	(-1,-1) circle;
		\end{scope}
	\end{tikzpicture}
\end{frame}


\begin{frame}{幂函数: 负有理幂次 $\frac pq$, $p,q$ 一奇一偶}
	\onslide<+->
	\noindent\enumnum5 当 $\mu < 0, p$ 偶 $q$ 奇时,
	\begin{itemize}
		\item 定义域为 $(-\infty, 0)\cup(0, +\infty)$, 值域为 $(0, +\infty)$.
		\item 偶函数, 图像过点 $(\pm1,1)$, 有两条渐近线 $x = 0, y = 0$.
	\end{itemize}
	\onslide<+->
	\noindent\enumnum6 当 $\mu < 0, p$ 奇 $q$ 偶时,
	\begin{itemize}
		\item 定义域和值域为 $(0, +\infty)$.
		\item 图像过点 $(1,1)$, 有两条渐近线 $x = 0, y = 0$.
		\item 此时 $y = x^\mu$ 与 $y = x^{\frac1\mu}$ 在 $(0, +\infty)$ 的限制互为反函数.
	\end{itemize}
	\onslide<1->
	\begin{center}
		\begin{tikzpicture}
			\draw[cstaxis] (-2.3,0)--(2.3,0);
			\draw[cstaxis] (0,-0.5)--(0,2.6);
			\draw[cstcurve,dcolora,smooth,domain=0.5:1.6] plot ({1/\x},{\x*\x});
			\draw[cstcurve,dcolora,smooth,domain=0.5:1.6] plot ({-1/\x},{\x*\x});
			\draw[cstcurve,dcolorb,smooth,domain=0.5:1.6,visible on=<4->] plot ({\x*\x},{1/\x});
			\draw[cstdash,dcolord,visible on=<7->] (-0.4,-0.4)--(2,2);
			\draw
				(0.2,-0.2) node {$O$}
				(2.1,-0.2) node {$x$}
				(-0.2,2.4) node {$y$}
				(-1.7,1.1) node[dcolora] {$y=x^{-2}$}
				(3,1) node[dcolorb,visible on=<4->] {$y=x^{-\half}$};
			\begin{scope}[visible on=<3->]
				\draw
					(1,-0.2) node {$1$}
					(-1,-0.2) node {$-1$}
					(-0.2,1) node {$1$};
				\draw[cstdash,dcolore] (-1,0)--(-1,1)--(1,1)--(1,0);
				\fill[cstdot,dcolore]	(1,1) circle;
				\fill[cstdot,dcolore]	(-1,1) circle;
			\end{scope}
		\end{tikzpicture}
	\end{center}
\end{frame}


\begin{frame}{幂函数: 无理幂次}
	\onslide<+->
	\noindent\enumnum7 当 $\mu$ 为无理数时, $y =x^\mu$ 定义为 $y =e^{\mu\ln x}$,
	\begin{itemize}
		\item 定义域和值域为 $(0, +\infty)$.
		\item 图像过点 $(1,1)$.
		\item $\mu < 0$ 时有两条渐近线 $x = 0, y = 0$.
		\item 此时 $y = x^\mu$ 与 $y = x^{\frac1\mu}$ 互为反函数.
	\end{itemize}
	\onslide<1->
	\begin{center}
		\begin{tikzpicture}
			\draw[cstaxis] (-0.5,0)--(2.7,0);
			\draw[cstaxis] (0,-0.5)--(0,2.7);
			\draw[cstcurve,dcolora,smooth,domain=0:1.5] plot (\x,{\x^(sqrt 3)});
			\draw[cstcurve,dcolorb,smooth,domain=0.4:1.6,visible on=<4->] plot (1/\x,{\x^(sqrt 3)});
			\draw[cstcurve,dcolora,smooth,domain=0:1.5,visible on=<5->] plot ({\x^(sqrt 3)},\x);
			\draw[cstcurve,dcolorb,smooth,domain=0.4:1.5,visible on=<5->] plot ({\x^(sqrt 3)},1/\x);
			\draw[cstdash,dcolord,visible on=<5->] (-0.4,-0.4)--(2,2);
			\draw
				(0.2,-0.2) node {$O$}
				(2.5,-0.2) node {$x$}
				(-0.2,2.5) node {$y$}
				(1.6,2.3) node[dcolora] {$y=x^{\sqrt 3}$}
				(3.4,0.3) node[dcolorb,visible on=<4->] {$y=x^{-\sqrt 3}$}
				(2.8,1.6) node[dcolora,visible on=<5->] {$y=x^{\frac1{\sqrt 3}}$}
				(2.9,0.8) node[dcolorb,visible on=<5->] {$y=x^{-\frac1{\sqrt 3}}$};
			\begin{scope}[visible on=<3->]
				\draw
					(1,-0.2) node {$1$}
					(-0.2,1) node {$1$};
				\draw[cstdash,dcolore] (0,1)--(1,1)--(1,0);
				\fill[cstdot,dcolore]	(1,1) circle;
			\end{scope}
		\end{tikzpicture}
	\end{center}
\end{frame}


\begin{frame}{幂函数性质总结}
	\onslide<+->
	\begin{block}{总结}
		\begin{itemize}
			\item 幂函数 $y = x^\mu$ 总在 $(0, +\infty)$ 上有定义, 图像经过 $(1,1)$.
			\item $y = x^\mu|_{(0, +\infty)}$ 和 $y = x^{\frac1\mu}|_{(0, +\infty)}$ 互为反函数.
			\item $\mu>0$ 时, 在第一象限单调递增, 它的图像经过 $(0,0)$.
			\item $\mu<0$ 时, 在第一象限单调递减, 有两条渐近线 $x = 0, y = 0$.
		\end{itemize}
	\end{block}
\end{frame}


\begin{frame}{幂函数性质总结}
	\onslide<+->
	\begin{center}
		\renewcommand\arraystretch{1.4}
		\defaultrowcolors
		\resizebox{\textwidth}{!}{
		\begin{tabular}{|c|c|c|}\hline
			\tht $\mu=\dfrac pq$&\tht 定义域&\tht 值域\\\hline
			&&$p$ 为奇数时, $(-\infty, +\infty)$\\\cline{3-3}
				\multirow{-2}*{$\mu>0, q$ 奇}
				&\multirow{-2}*{$(-\infty, +\infty)$}
				& $p$ 为偶数时, $[0, +\infty)$\\\hline
			$\mu>0, q$ 偶
				& $[0, +\infty)$
				& $[0, +\infty)$\\\hline
			&&$p$ 为奇数时, $(-\infty,0)\cup(0,+\infty)$\\\cline{3-3}
				\multirow{-2}*{$\mu<0, q$ 奇}
				&\multirow{-2}*{$(-\infty,0)\cup(0,+\infty)$}
				& $p$ 为偶数时, $(0, +\infty)$\\\hline
			$\mu<0, q$ 偶
				& $(0, +\infty)$
				& $(0, +\infty)$\\\hline
			$\mu$ 为无理数
				& $(0, +\infty)$
				& $(0, +\infty)$\\\hline
		\end{tabular}}
	\end{center}
\end{frame}


\begin{frame}{三角函数}
	\onslide<+->
	\vspace{-\baselineskip}
	\begin{center}
		\renewcommand\arraystretch{1.3}
		\defaultrowcolors
		\resizebox{\textwidth}{!}{
		\begin{tabular}{|c|c|c|c|c|c|}\hline
			\tht 三角函数&\tht 定义域&\tht 值域&\tht 有界性&\tht 周期&\tht 奇偶性\\\hline
			正弦 $\sin x$&$(-\infty,+\infty)$&$[-1,1]$&有界&$2\pi$&奇函数\\\hline
			余弦 $\cos x$&$(-\infty,+\infty)$&$[-1,1]$&有界&$2\pi$&偶函数\\\hline
			正切 $\tan x$&$x\neq k+\half\pi, k\in\BZ$&$(-\infty,+\infty)$&无界&$\pi$&奇函数\\\hline
			余切 $\cot x$&$x\neq k\pi, k\in\BZ$&$(-\infty, +\infty)$&无界&$\pi$&奇函数\\\hline
			正割 $\sec x$&$x\neq k+\half\pi, k\in\BZ$&$(-\infty,-1]\cup[1,+\infty)$&无界&$2\pi$&偶函数\\\hline
			余割 $\csc x$&$x\neq k\pi, k\in\BZ$&$(-\infty,-1]\cup[1,+\infty)$&无界&$2\pi$&奇函数\\\hline
		\end{tabular}}
	\end{center}
	\onslide<+->
	\begin{center}
		\begin{tikzpicture}[scale=0.5]
			\draw[cstaxis] (-9,0)--(9,0);
			\draw[cstaxis] (0,-3.2)--(0,3.2);
			\begin{scope}[visible on=<2>]
				\draw[cstcurve,dcolora,smooth,domain=-360:360] plot ({\x/180*pi},{sin (\x)});
				\draw (2,1.4) node[dcolora] {$y=\sin x$};
				\draw[cstdash,dcolore] (1.57,0)--(1.57,1)--(0,1);
			\end{scope}
			\begin{scope}[visible on=<3>]
				\draw[cstcurve,dcolorb,smooth,domain=-360:360] plot ({\x/180*pi},{cos (\x)});
				\draw (1.9,1.4) node[dcolorb] {$y=\cos x$};
			\end{scope}
			\begin{scope}[visible on=<4>]
				\draw[cstcurve,dcolorc,smooth,domain=-70:70] plot ({\x/180*pi},{tan (\x)});
				\draw[cstcurve,dcolorc,smooth,domain=-70:70] plot ({(\x+180)/180*pi},{tan (\x)});
				\draw[cstcurve,dcolorc,smooth,domain=-70:70] plot ({(\x+360)/180*pi},{tan (\x)});
				\draw[cstcurve,dcolorc,smooth,domain=-70:70] plot ({(\x-180)/180*pi},{tan (\x)});
				\draw[cstcurve,dcolorc,smooth,domain=-70:70] plot ({(\x-360)/180*pi},{tan (\x)});
				\draw (7.4,-1.2) node[dcolorc] {$y=\tan x$};
				\draw[cstdash,dcolord] (1.57,-3)--(1.57,3);
				\draw[cstdash,dcolord] (-1.57,-3)--(-1.57,3);
				\draw[cstdash,dcolord] (4.71,-3)--(4.71,3);
				\draw[cstdash,dcolord] (-4.71,-3)--(-4.71,3);
			\end{scope}
			\begin{scope}[visible on=<5>]
				\draw[cstcurve,dcolora,smooth,domain=-70:70] plot ({(270-\x)/180*pi},{tan (\x)});
				\draw[cstcurve,dcolora,smooth,domain=-70:70] plot ({(90-\x)/180*pi},{tan (\x)});
				\draw[cstcurve,dcolora,smooth,domain=-70:70] plot ({(-90-\x)/180*pi},{tan (\x)});
				\draw[cstcurve,dcolora,smooth,domain=-70:70] plot ({(-270-\x)/180*pi},{tan (\x)});
				\draw (6,1.2) node[dcolora] {$y=\cot x$};
				\draw[cstdash,dcolord] (3.14,-3)--(3.14,3);
				\draw[cstdash,dcolord] (-3.14,-3)--(-3.14,3);
				\draw[cstdash,dcolord] (6.28,-3)--(6.28,3);
				\draw[cstdash,dcolord] (-6.28,-3)--(-6.28,3);
			\end{scope}
			\begin{scope}[visible on=<6>]
				\draw[cstcurve,dcolorb,smooth,domain=20:160] plot ({(\x-90)/180*pi},{1/(sin (\x))});
				\draw[cstcurve,dcolorb,smooth,domain=20:160] plot ({(\x+90)/180*pi},{-1/(sin (\x))});
				\draw[cstcurve,dcolorb,smooth,domain=20:160] plot ({(\x+270)/180*pi},{1/(sin (\x))});
				\draw[cstcurve,dcolorb,smooth,domain=20:160] plot ({(\x-270)/180*pi},{-1/(sin (\x))});
				\draw[cstcurve,dcolorb,smooth,domain=20:160] plot ({(\x-450)/180*pi},{1/(sin (\x))});
				\draw (2.6,1.2) node[dcolorb] {$y=\sec x$};
				\draw[cstdash,dcolord] (1.57,-3)--(1.57,3);
				\draw[cstdash,dcolord] (-1.57,-3)--(-1.57,3);
				\draw[cstdash,dcolord] (4.71,-3)--(4.71,3);
				\draw[cstdash,dcolord] (-4.71,-3)--(-4.71,3);
			\end{scope}
			\begin{scope}[visible on=<7>]
				\draw[cstcurve,dcolorc,smooth,domain=20:160] plot ({\x/180*pi},{1/(sin (\x))});
				\draw[cstcurve,dcolorc,smooth,domain=20:160] plot ({(\x+180)/180*pi},{-1/(sin (\x))});
				\draw[cstcurve,dcolorc,smooth,domain=20:160] plot ({(\x-180)/180*pi},{-1/(sin (\x))});
				\draw[cstcurve,dcolorc,smooth,domain=20:160] plot ({(\x-360)/180*pi},{1/(sin (\x))});
				\draw (4.2,1.2) node[dcolorc] {$y=\csc x$};
				\draw[cstdash,dcolord] (3.14,-3)--(3.14,3);
				\draw[cstdash,dcolord] (-3.14,-3)--(-3.14,3);
				\draw[cstdash,dcolord] (6.28,-3)--(6.28,3);
				\draw[cstdash,dcolord] (-6.28,-3)--(-6.28,3);
				\draw[cstdash,dcolore] (1.57,0)--(1.57,1)--(0,1);
			\end{scope}
			\draw[thick] (1.57,0)--(1.57,0.3);
			\draw[thick] (3.14,0)--(3.14,0.3);
			\draw[thick] (4.71,0)--(4.71,0.3);
			\draw[thick] (6.28,0)--(6.28,0.3);
			\draw[thick] (-1.57,0)--(-1.57,0.3);
			\draw[thick] (-3.14,0)--(-3.14,0.3);
			\draw[thick] (-4.71,0)--(-4.71,0.3);
			\draw[thick] (-6.28,0)--(-6.28,0.3);
			\draw[thick] (0,1)--(0.3,1);
			\draw[thick] (0,2)--(0.3,2);
			\draw[thick] (0,-1)--(0.3,-1);
			\draw[thick] (0,-2)--(0.3,-2);
			\draw
				(0.4,-0.4) node {$O$}
				(8.6,-0.4) node {$x$}
				(-0.4,2.8) node {$y$}
				(1.57,-0.6) node {$\frac\pi2$}
				(3.14,-0.4) node {$\pi$}
				(4.71,-0.6) node {$\frac{3\pi}2$}
				(6.28,-0.4) node {$2\pi$}
				(-1.57,-0.6) node {$-\frac\pi2$}
				(-3.14,-0.4) node {$-\pi$}
				(-4.71,-0.6) node {$-\frac{3\pi}2$}
				(-6.28,-0.4) node {$-2\pi$}
				(-0.3,1) node {$1$}
				(-0.3,2) node {$2$}
				(-0.6,-1) node {$-1$}
				(-0.6,-2) node {$-2$};
		\end{tikzpicture}
	\end{center}
\end{frame}


\begin{frame}{反三角函数}
	\onslide<+->
	\begin{definition}
		\begin{enumerate}
			\item \emph{反正弦函数 $y=\arcsin x$}.

			\onslide<+->{定义域为 $[-1,1]$, 值域为 $\left[-\frac\pi2,\frac\pi2\right]$.}

			\onslide<+->{有界、单调递增奇函数.}
			\item \emph{反余弦函数 $y=\arccos x$}.

			\onslide<+->{定义域为 $[-1,1]$, 值域为 $[0,\pi]$.}

			\onslide<+->{有界、单调递减函数.}
		\end{enumerate}
		\vspace{1.8cm}
	\onslide<2->{
		\begin{tikzpicture}[overlay,xshift=8.7cm,yshift=2cm]
			\draw[cstaxis] (-1.8,0)--(1.8,0);
			\draw[cstaxis] (0,-2)--(0,3.6);
			\draw[cstcurve,dcolora,smooth,domain=-90:90] plot ({sin (\x)},{\x/180*pi});
			\draw (2.2,1.4) node[dcolora] {$y=\arcsin x$};
			\draw[cstdash,dcolore] (0,1.57)--(1,1.57)--(1,0);
			\draw[cstdash,dcolore] (0,-1.57)--(-1,-1.57)--(-1,0);
			\begin{scope}[visible on=<5->]
				\draw[cstcurve,dcolorb,smooth,domain=0:180] plot ({cos (\x)},{\x/180*pi});
				\draw (0.5,2.5) node[dcolorb] {$y=\arccos x$};
				\draw[cstdash,dcolore] (0,3.14)--(-1,3.14)--(-1,0);
				\draw (0.2,3.14) node {$\pi$};
			\end{scope}
			\draw
				(0.2,-0.2) node {$O$}
				(1.6,-0.2) node {$x$}
				(0.2,3.4) node {$y$}
				(-0.3,1.57) node {$\frac\pi2$}
				(0.3,-1.57) node {$-\frac\pi2$}
				(1,-0.2) node {$1$}
				(-1,-0.2) node {$-1$};
		\end{tikzpicture}}
	\end{definition}
	\onslide<+->
	二者满足等式 \alert{$\arcsin x+\arccos x=\frac\pi2$}.
\end{frame}


\begin{frame}{反三角函数}
	\onslide<+->
	\begin{definition}
		\begin{enumerate}
			\item \emph{反正切函数 $y=\arctan x$}.
			\onslide<+->{定义域为 $(-\infty,+\infty)$, 值域为 $\left[-\frac\pi2,\frac\pi2\right]$.
			}\onslide<+->{有界、单调递增奇函数.
			}\onslide<+->{有两条渐近线 $y=\pm\frac\pi2$.}
			\item \emph{反余切函数 $y=\arccot x$}.
			\onslide<+->{定义域为 $(-\infty,+\infty)$, 值域为 $[0,\pi]$.
			}\onslide<+->{有界、单调递减函数.
			}\onslide<+->{有两条渐近线 $y=0,\pi$.}
		\end{enumerate}
	\onslide<2->{
		\begin{center}
		\begin{tikzpicture}[scale=0.5]
			\draw[cstaxis] (-5,0)--(5,0);
			\draw[cstaxis] (0,-2)--(0,4.2);
			\draw[cstcurve,dcolora,smooth,domain=-78:78] plot ({tan (\x)},{\x/180*pi});
			\draw (-3.4,-0.6) node[dcolora] {$y=\arctan x$};
			\draw[cstdash,dcolord] (-5,1.57)--(5,1.57);
			\draw[cstdash,dcolord] (-5,-1.57)--(5,-1.57);
			\begin{scope}[visible on=<6->]
				\draw[cstcurve,dcolorb,smooth,domain=-78:78] plot ({tan (\x)},{(90-\x)/180*pi});
				\draw (-3.4,2.1) node[dcolorb] {$y=\arccot x$};
				\draw[cstdash,dcolord] (-5,3.14)--(5,3.14);
				\draw (-0.3,2.8) node {$\pi$};
			\end{scope}
			\draw
				(-0.4,0.4) node {$O$}
				(4.6,-0.4) node {$x$}
				(0.4,3.8) node {$y$}
				(0.3,2.2) node {$\frac\pi2$}
				(0.6,-1) node {$-\frac\pi2$};
		\end{tikzpicture}
		\end{center}}
	\end{definition}
	\onslide<+->
	二者满足等式 \alert{$\arctan x+\arccot x=\frac\pi2$}.
\end{frame}


\subsection{初等函数}

\begin{frame}{初等函数}
	\onslide<+->
	\begin{definition}
		由基本初等函数进行有限次的四则运算和有限次的复合运算所得到的函数被称为\emph{初等函数}.
	\end{definition}
	\onslide<+->
	\begin{example}
		\begin{itemize}
			\item $y=\dfrac{x^2}{x-1}$;
			\item $y=\sin(2x+1)$;
			\item $y=|x|=\sqrt{x^2}$.
			\item $y=\sgn x, y=[x]$ 不是初等函数.
		\end{itemize}
	\end{example}
	\onslide<+->
	下面介绍两类在工程上常用的初等函数.
\end{frame}


\begin{frame}{双曲函数}
	\beqskip{0pt}
	\onslide<+->
	\begin{definition}
		定义\emph{双曲正弦/余弦/正切函数}为
		\begin{itemize}
			\item $\sh x=\dfrac{e^x-e^{-x}}2$,
			\item $\ch x=\dfrac{e^x+e^{-x}}2$,
			\item $\tanh x=\dfrac{e^x-e^{-x}}{e^x+e^{-x}}$.
		\end{itemize}
	\end{definition}
	\onslide<2->
	\begin{tikzpicture}[overlay,xshift=8.2cm,yshift=1.7cm]
		\draw[cstaxis] (-3,0)--(3,0);
		\draw[cstaxis] (0,-1.2)--(0,2.5);
		\draw[cstcurve,dcolora,smooth,domain=-1:1.6] plot (\x,{(e^\x-e^(-\x))/2});
		\draw (2.1,1.6) node[dcolora] {$y=\sh x$};
		\begin{scope}[visible on=<3->]
			\draw[cstcurve,dcolorb,smooth,domain=-1.5:1.5] plot (\x,{(e^\x+e^(-\x))/2});
			\draw (-1.8,1.3) node[dcolorb] {$y=\ch x$};
		\end{scope}
		\begin{scope}[visible on=<4->]
			\draw[cstcurve,dcolorc,smooth,domain=-3:3] plot (\x,{1-2/(e^(2*\x)+1)});
			\draw (-1.8,-0.6) node[dcolorc] {$y=\tanh x$};
		\end{scope}
		\begin{scope}[visible on=<7->]
			\draw[cstdash,dcolord] (-3,1)--(3,1);
			\draw[cstdash,dcolord] (-3,-1)--(3,-1);
		\end{scope}
		\draw
			(0.2,-0.2) node {$O$}
			(2.8,-0.2) node {$x$}
			(0.2,2.3) node {$y$};
	\end{tikzpicture}
	\begin{itemize}
		\item $\sh x$ 值域为 $\BR$, 是单调递增奇函数.
		\item $\ch x$ 值域为 $[1,+\infty)$, 是偶函数, 且在 $[0, +\infty)$ 上单调递增.
		\item $\tanh x$ 值域为 $(-1,1)$, 是单调递增奇函数, 有两条渐近线 $y=\pm1$.
	\end{itemize}
	\onslide<+->
	两端固定自然垂下的铁链的形状是双曲余弦函数的图像.
	\endgroup
\end{frame}


\begin{frame}{反双曲函数}
	\onslide<+->
	\begin{definition}
		定义\emph{反双曲正弦/余弦/正切函数}为
		\begin{itemize}
			\item $\arsh x=\ln(x+\sqrt{x^2+1})$,
			\item $\arch x=\ln(x+\sqrt{x^2-1})$,
			\item $\arth x=\dfrac12\ln\dfrac{1+x}{1-x}$.
		\end{itemize}\vspace{1cm}
	\end{definition}
	\onslide<2->
	\begin{tikzpicture}[overlay,xshift=8cm,yshift=2.3cm]
		\draw[cstaxis] (0,-1.4)--(0,1.9);
		\draw[cstaxis] (-2.5,0)--(3,0);
		\draw[cstcurve,dcolora,smooth,domain=-1.65:1.85] plot ({(e^\x-e^(-\x))/2},\x);
		\draw (-2,-0.7) node[dcolora] {$y=\arsh x$};
		\begin{scope}[visible on=<3->]
			\draw[cstcurve,dcolorb,smooth,domain=0:1.8] plot ({(e^\x+e^(-\x))/2},\x);
			\draw (2.1,0.4) node[dcolorb] {$y=\arch x$};
		\end{scope}
		\begin{scope}[visible on=<4->]
			\draw[cstcurve,dcolorc,smooth,domain=-1.9:1.9] plot ({1-2/(e^(2*\x)+1)},\x);
			\draw (0.05,-1.7) node[dcolorc] {$y=\arth x$};
		\end{scope}
		\begin{scope}[visible on=<7->]
			\draw[cstdash,dcolord] (1,-1.9)--(1,1.9);
			\draw[cstdash,dcolord] (-1,-1.9)--(-1,1.9);
		\end{scope}
		\draw
			(0.2,-0.2) node {$O$}
			(2.8,-0.2) node {$x$}
			(0.2,1.7) node {$y$};
	\end{tikzpicture}
	\begin{itemize}
		\item $\arsh x$ 定义域和值域为 $\BR$, 是单调递增奇函数.
		\item $\arch x$ 定义域为 $[1, +\infty)$, 值域为 $[0, +\infty)$, 是单调递增函数.
		\item $\arth x$ 定义域为 $(-1,1)$, 值域为 $\BR$, 是单调递增奇函数, 有两条渐近线 $x=\pm1$.
	\end{itemize}
\end{frame}


\begin{frame}{双曲函数与三角函数}
	\onslide<+->
	双曲函数有着类似三角函数的性质:
	\begin{itemize}
		\item $\ch^2x-\sh^2x=1$,
		\item $\sh(2x)=2\sh x\ch x, \ch(2x)=\ch^2x+\sh^2 x$,
		\item $\sh(x\pm y)=\sh x\ch y\pm\ch x\sh y$,
		\item $\ch(x\pm y)=\ch x\ch y\pm\sh x\sh y$,
		\item $\tanh(x\pm y)=\dfrac{\tanh x\pm\tanh y}{1\pm\tanh x\tanh y}$.
	\end{itemize}
	\onslide<+->
	其根本原因在于复数域上的欧拉公式 $e^{ix}=\cos x+i\sin x$.
	\onslide<+->
	可以看出 (形式上): $\ch(ix)=\cos x, \sh(ix)=i\sin x$.
\end{frame}
