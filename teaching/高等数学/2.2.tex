
\end{frame}


\begin{frame}
2.2 函数的极限
	我们参照数列极限的定义来定义函数的极限. 我们先考虑当 x \ra +\infty 时 $f(x)$ 的
极限. 回忆数列的极限
\forall 𝜀>0, ∃𝑁 使得当 n>𝑁 时, 有 an - a < 𝜀 .
	定义 设函数 $f(x)$ 当 x 充分大时有定义, 𝐴 为常数. 如果
\forall 𝜀>0, ∃𝑋 使得当 x>𝑋 时, 有 $f(x)$ - 𝐴 < 𝜀,
	则称 𝐴 为 $f(x)$ 当 x \ra +\infty 时的极限, 记为 lim
x\ra+\infty f (x )=𝐴 或 f(x ) \ra
𝐴 x \ra +\infty .
	从几何角度来看, 就是函数在 𝑋, +\infty 上的限制的图像被夹在直线 y=𝐴 \pm 𝜀 之
间.
	我们将红字部分称为 𝜀 -𝑋 语言.
\end{frame}


\begin{frame}
	类似地, 我们有:
	定义 设函数 $f(x)$ 当 x 充分小时有定义, 𝐴 为常数. 如果
\forall 𝜀>0, ∃𝑋 使得当 x < -𝑋 时, 有 $f(x)$ - 𝐴 < 𝜀,
	则称 𝐴 为 $f(x)$ 当 x \ra -\infty 时的极限, 记为 lim
x\ra-\infty f (x )=𝐴 或 f(x ) \ra
𝐴 x \ra -\infty .
	定义 设函数 $f(x)$ 当 x 充分大时有定义, 𝐴 为常数. 如果
\forall 𝜀>0, ∃𝑋 使得当 |x |>𝑋 时, 有 $f(x)$ - 𝐴 < 𝜀 ,
	则称 𝐴 为 $f(x)$ 当 x \ra \infty 时的极限, 记为 lim
x\ra\infty f (x )=𝐴 或 f (x ) \ra 𝐴 x \ra \infty .
	注意, 函数极限中需要分清 x \ra \infty, x \ra +\infty, x \ra -\infty, 而数列情形只有 n \ra \infty,
因为 n 是正整数.
\end{frame}


\begin{frame}
	类似于数列极限的性质, 我们有
	\begin{theorem}
lim
\end{theorem}
x\ra\infty $f(x)$=𝐴 ⇔ lim
x\ra+\infty $f(x)$=lim
x\ra-\infty $f(x)$=𝐴.
	当 y \ra \infty 时, x=x0 + 1
y \ra x0 . 因此如果存在 𝛿>0 使得函数 $f(x)$ 在 x0 的去
心 𝛿 邻域
𝑈
∘
x0 , 𝛿=x0 - 𝛿 , x0 \cup x0 , x0 + 𝛿
	上有定义, 则 lim
x\rax0
f (x ) 应当定义为 lim
y\ra\inftyf x0 + 1
y . 于是我们得到下述定义:
	定义 设函数 $f(x)$ 在 x0 的某个去心邻域内有定义, 𝐴 为常数. 如果
\forall 𝜀>0, ∃𝛿>0 使得当 x \in 𝑈
∘
x0 , 𝛿 时, 有 $f(x)$ - 𝐴 < 𝜀 ,
	则称 𝐴 为 $f(x)$ 当 x \ra x0 时的极限, 记为 lim
x\rax0
f (x )=𝐴 或 f (x ) \ra 𝐴 x \ra x0 .
\end{frame}


\begin{frame}
	从几何角度来看, 就是函数在 𝑈
∘
x0 , 𝛿 上的限制的图像被夹在直线 y=𝐴 \pm 𝜀 之
间.
	类似地可以定义单侧极限
	lim
x\rax0
+ $f(x)$=𝐴 ⇔ \forall 𝜀>0, ∃𝛿>0 使得当 x \in x0 , x0 + 𝛿 时, 有 $f(x)$ - 𝐴 < 𝜀 .
	lim
x\rax0
- $f(x)$=𝐴 ⇔ \forall 𝜀>0, ∃𝛿>0 使得当 x \in x0 - 𝛿 , 0 时, 有 $f(x)$ - 𝐴 < 𝜀 .
	我们将红字部分称为 𝜀 -𝛿 语言.
	\begin{theorem}
lim
\end{theorem}
x\rax0
f x=𝐴 ⇔ lim
x\rax0
+ $f(x)$=lim
x\rax0
- $f(x)$=𝐴.
	如果 lim
x\rax0
+ $f(x)$=𝐴, 我们记 f x0
+=𝐴. 如果 lim
x\rax0
- $f(x)$=𝐴, 我们记 f x0
-=𝐴.
\end{frame}


\begin{frame}
	\begin{example}
\begin{proof}
\end{example}
lim
\end{proof}
x\ra\infty
1
x=0.
	分析 和数列极限类似, 这种问题的证明通常也分为两部分
	先估计 $f(x)$ - 𝐴 , 得到它和 x - x0 < 𝛿 或 x>𝑋 的不等式关系,
从而求得 𝛿 或 𝑋. 这个过程中可以进行适当的放缩.
	将 𝛿 或 𝑋 代入极限的定义中.
	1
x - 0=1
x < 𝜀, x>1
𝜀 , 因此我们可以取 𝑋=1
𝜀 .
	\begin{proof}
\forall 𝜀>0, 令 𝑋=1
\end{proof}
𝜀 . 当 x>𝑋 时, 有 1
x - 0=1
x < 𝜀. 所以 lim
x\ra\infty
1
x =
0.
\end{frame}


\begin{frame}
	\begin{example}
\begin{proof}
\end{example}
a>1 时, lim
\end{proof}
x\ra-\inftya x=0.
	分析 由于 a>1 时, \log a x 是单调递增的. 因此 a x - 0=a x < 𝜀, x <
\log a 𝜀.
	\begin{proof}
\forall 𝜀>0, 令 𝑋=- \log a 𝜀. 当 x < -𝑋 时, 有 a x - 0=a x < 𝜀. 所以
\end{proof}
lim
x\ra-\inftya x=0.
	\begin{example}
\begin{proof}
\end{example}
lim
\end{proof}
x\rax0
ax + 𝑏=ax0 + 𝑏.
	分析 ax + 𝑏 - ax0 + 𝑏=a ⋅ x - x0 < 𝜀, 因此我们可以取 𝛿=𝜀
a .
注意我们需要单独考虑 a=0 的情形.
\end{frame}


\begin{frame}
	\begin{proof}
我们有 ax + 𝑏 - ax0 + 𝑏=a ⋅ x - x0 .
\end{proof}
	如果 a=0, \forall 𝜀>0, 令 𝛿=1. 当 0 < x - x0 < 𝛿 时, 有
ax + 𝑏 - ax0 + 𝑏=a ⋅ x - x0=0 < 𝜀.
	如果 a\neq 0, \forall 𝜀>0, 令 𝛿=𝜀
a. 当 0 < x - x0 < 𝛿 时, 有
ax + 𝑏 - ax0 + 𝑏=a ⋅ x - x0 < a 𝛿=𝜀.
	所以 lim
x\rax0
ax + 𝑏=ax0 + 𝑏.
	注记 当 $f(x)$=𝐴 时, 我们可以取任一正数作为 𝛿 , 只要 𝑈
∘
x, 𝛿 包含在该
函数的定义域范围内即可.
\end{frame}


\begin{frame}
	\begin{example}
\begin{proof}
\end{example}
lim
\end{proof}
x \rax0
sin x=sin x0 .
	与三角函数有关的放缩往往要用到和差化积公式
sin x - sin y=2 sin x - y
2 cos x + y
2 , cos x - cos y=-2 sin x + y
2 sin x - y
2 ,
	然后将不含 x - x0 的项放缩到 1; 以及三角函数基本不等式
sin x ≤ x , \forall x \in -\infty, +\infty , x ≤ tan x , \forall x \in - \pi
2 , \pi
2 .
	\begin{proof}
我们有 sin x - sin x0=2 sin x-x0
\end{proof}
2 cos x+x0
2 ≤ 2 sin x-x0
2 ≤ 2 x-x0
2=x - x0 .
	\forall 𝜀>0, 令 𝛿=𝜀 . 当 0 < x - x0 < 𝛿 时, 有
sin x - sin x0 ≤ x - x0 < 𝛿=𝜀 .
	所以 lim
x\rax0
sin x=sin x0 .
\end{frame}


\begin{frame}
	\begin{example}
\begin{proof}
\end{example}
lim
\end{proof}
x\ra\infty arctan x 不存在.
	分析 从图像上可以看出 lim
x\ra+\infty arctan x=\pi
2 . 我们想要使用 x 来控制
arctan x - \pi
2 . 不过这个形式不容易估计, 我们令 𝑡=arctan x, 则问题变
成了 𝑡 - \pi
2 和 tan 𝑡 的关系. 再令 𝑠=arctan x - \pi
2 , 则 𝑠 ≤ tan 𝑠 .
	不过这个不等式并不总是对的, 我们需要估计 𝑠=arctan x - \pi
2 的范围.
由于我们考虑的是 x \ra +\infty, 不妨设 x>0, 那么 arctan x \in 0, \pi
2 , 𝑠 =
arctan x - \pi
2 \in - \pi
2 , 0 .
\end{frame}


\begin{frame}
	\begin{proof}
我们来证明 lim
\end{proof}
x\ra+\infty arctan x=\pi
2 . 当 x>0 时,arctan x - \pi
2 \in - \pi
2 , 0 .
因此 arctan x - \pi
2 ≤ tan \pi
2 - arctan x=1
tan arctan x=1
x.
	\forall 𝜀>0, 令 𝑋=1
𝜀>0. 当 x>𝑋 时, 有
arctan x - \pi
2 ≤ 1
x < 1
𝑋=𝜀.
	所以 lim
x\ra+\infty arctan x=\pi
2 .
	同理, lim
x\ra-\infty arctan x=- \pi
2 . 因此 lim
x\ra\infty arctan x 不存在.
