\documentclass[12pt]{ctexart}
\usepackage{geometry}
\geometry{
  paperwidth=21cm,
  paperheight=29.7cm,
  textwidth=18.2cm,
  textheight=26.4cm
}
\usepackage{graphicx}
\usepackage{tikz}
\usepackage{fancyhdr}
\usepackage{lastpage}
\usepackage{xcolor}
\pagestyle{fancy}
\fancyhf{}
\renewcommand\headrulewidth{0pt} % 页眉线宽度
\renewcommand\footrulewidth{0pt} % 页脚线宽度
\fancyfoot[C]{\small \thepage}
% \fancyfoot[C]{\small \thepage/\pageref{LastPage}}
\newcounter{pdfpage}
\newcommand{\notes}[2][1]{{%
  \addtocounter{pdfpage}{#1}%
  \setlength{\fboxsep}{1mm}%
  \fbox{\includegraphics[clip,viewport=0 0 453 283,page=\thepdfpage,width=80mm]{1handout.pdf}}%
  \fbox{\begin{minipage}[b][50mm][t]{80mm}#2~\end{minipage}}%
  \par%
}}
\newcommand{\skippage}[1][1]{\addtocounter{pdfpage}{#1}}
\newcommand{\sect}[1]{{\stepcounter{section}\bfseries\color{blue}\thesection~#1}}
\usetikzlibrary{mindmap,shadows}

\begin{document}
\title{数学(下)课件备注}
\author{张神星}
\begin{tikzpicture}[overlay,xshift=-23mm,yshift=18mm]
  \fill[orange!20] (0,0) rectangle (21.1,-29.8);
  \fill[yellow,fill opacity=0.5] (0,0)--(0,-12)--(21.1,0)--cycle;
  \fill[magenta!60,fill opacity=0.5] (0,0)--(21.1,-12)--(21.1,0)--cycle;
  \fill[blue!50,fill opacity=0.5] (0,-17.8)--(0,-29.8)--(21.1,-29.8)--cycle;
  \fill[green!50,fill opacity=0.5] (21.1,-17.8)--(0,-29.8)--(21.1,-29.8)--cycle;
  \draw[fill opacity=0.1] (18,-24) node {\includegraphics[height=16cm]{image-hfut.png}};
\end{tikzpicture}
\makeatletter
\begin{center}%
  \vfill
  {\Huge \@title}\par\vspace{4cm}%
  {\LARGE \@author\par\vspace{1.5cm}%
  \@date}
  \vfill
\end{center}
\makeatother
\newpage

\notes{
  课件可在该网站向下翻到教学一栏处下载.
}
\notes{
  期中考试安排在第11周.

  作业/期中/期末/课堂测验的占比分别为15\%, 25\%, 50\%, 10\%.
  其中课堂测验分为3次, 每次10道选择题, 最后选得分最高的两次平均数作为最终得分.

  作业每章结束时交一次, 但最好平时上完一节就完成相应节的习题, 不要一次性堆到最后.
}
\notes{
  函数一章是基础知识, 主要是在回顾高中的内容基础上, 略作扩充和一般化.
  
  极限是核心内容之一.
  
  连续需要基于极限概念.
}
\notes{
  各章节的主要内容和关系.
}
\notes{
  每节的作业整体难度不大, 有余力的尽量都做一遍, 以巩固自己学习到的内容.
}
\notes{
  第一章内容主要是回顾函数的基础概念, 函数的表达和构造方式, 函数的特征, 常见函数, 以及常用的等式不等式技巧等.
}
\notes{
  第一节内容主要是回顾函数的基础概念, 函数的表达和构造方式.
}
\notes{
  我倾向于从映射的角度来介绍函数.
}
\notes{
  这些记号在高等数学课程中经常见到, $\forall,\exists$ 分别是各自首字母倒过来.
}
\notes{
  互动: 为什么不是映射?
}
\notes{
  提问: 出发的集合能不能是空集?

  当 $A=\emptyset,|A|=0$ 时, $|B|^{|A|}=1$.
  所以我们认为从空集到其它集合的映射是有一个的.
}
\notes{
  介绍一下空心体字母的写法.
}
\notes{
  
}
\notes{
  
}
\notes{
  
}
\notes{
  
}
\notes{
  
}
\notes{
  
}
\notes{
  
}
\notes{
  
}
\notes{
  
}
\notes{
  
}
\notes{
  
}
\notes{
  
}
\notes{
  
}
\notes{
  
}
\notes{
  
}
\notes{
  
}
\notes{
  
}
\notes{
  
}
\notes{
  
}
\notes{
  
}
\notes{
  
}
\notes{
  
}
\notes{
  
}
\notes{
  
}
\notes{
  
}
\notes{
  
}
\notes{
  
}
\notes{
  
}
\notes{
  
}
\notes{
  
}
\notes{
  
}
\notes{
  
}
\notes{
  
}
\notes{
  
}
\notes{
  
}
\notes{
  
}
\notes{
  
}
\notes{
  
}
\notes{
  
}
\notes{
  
}
\notes{
  
}
\notes{
  
}
\notes{
  
}
\notes{
  
}
\notes{
  
}
\notes{
  
}
\notes{
  
}
\notes{
  
}
\notes{
  
}
\notes{
  
}
\notes{
  
}
\notes{
  
}
\notes{
  
}
\notes{
  
}
\notes{
  
}
\notes{
  
}
\notes{
  
}
\notes{
  
}
\notes{
  
}
\notes{
  
}
\notes{
  
}
\notes{
  
}
\notes{
  
}
\notes{
  
}
\notes{
  
}
\notes{
  
}
\notes{
  
}
\notes{
  
}
\notes{
  
}
\notes{
  
}
\notes{
  
}
\notes{
  
}
\notes{
  
}
\notes{
  
}
\notes{
  
}
\notes{
  
}
\notes{
  
}
\notes{
  
}
\notes{
  
}
\notes{
  
}
\notes{
  
}
\notes{
  
}
\notes{
  
}
\notes{
  
}
\notes{
  
}
\notes{
  
}
\notes{
  
}
\notes{
  
}
\notes{
  
}

\end{document}
