\documentclass[cankaodaan]{hfutexam}
\usepackage{extarrows}
\newcommand{\diff}{\,\mathrm{d}}

\begin{document}
\BiaoTi{合肥工业大学试卷参考答案(A)}
\XueNian{2021}{2022}
\XueQi{二}
\KeChengDaiMa{034Y01}
\KeChengMingCheng{数学(下)}
\XueFen{5}
\KeChengXingZhi{必修}
\KaoShiXingShi{闭卷}
\ZhuanYeBanJi{少数民族预科班}
\KaoShiRiQi{2022年6月18日8:00-10:00}
\MingTiJiaoShi{集体}
%\XiZhuRenQianMing{dengbing.png}

\tigan{一、填空题(每小题3分,共18分)}

\textbf{请将你的答案对应填在横线上:}

\textbf{1.} \fillblank{$e$}, 
\textbf{2.} \fillblank{$2x\cos(x^2+1)\diff x$}, 
\textbf{3.} \fillblank{$\dfrac12$}, 

\textbf{4.} \fillblank{$y=x-1+2\ln 2$},
\textbf{5.} \fillblank{$1$},
\textbf{6.} \fillblank{$0$}.

\tigan{二、选择题(每小题3分,共18分)}

\textbf{请将你所选择的字母 A, B, C, D 之一对应填在下列表格里:}

\xuanzeti{\textbf{题号}}{\textbf{答案}}%
\xuanzeti{1}{A}%
\xuanzeti{2}{D}%
\xuanzeti{3}{B}%
\xuanzeti{4}{A}%
\xuanzeti{5}{C}%
\xuanzeti{6}{D}

\tigan{三、解答题(每小题8分,共64分)}
% 得分点命令 \score1, 得分点长度是自动的
% \Score{(2分, 缺少常数得1分)} 用于自定义得分说明

\textbf{1. (8分)【解】}
\begin{align*}
\lim_{x\to-1}\frac{x^2-1}{x^2+3x+2}&=\lim_{x\to-1}\frac{(x-1)(x+1)}{(x+2)(x+1)} \score3\\
&=\lim_{x\to-1}\frac{x-1}{x+2} \score3\\
&=\frac{-2}1=-2. \score2
\end{align*}

\textbf{2. (8分)【解】}
\begin{align*}
\lim_{x\to0}\frac{e^x-1-x}{\arcsin x^2}&=\lim_{x\to0}\frac{e^x-1-x}{x^2} \score3\\
&\xlongequal{\text{洛必达}}\lim_{x\to0}\frac{e^x-1}{2x} \score3\\
&=\lim_{x\to0}\frac{x}{2x}=\frac12. \score2
\end{align*}

\newpage

\textbf{3. (8分)【解】}
\begin{align*}
\frac{\diff y}{\diff x}&=\frac{\diff y/\diff t}{\diff x/\diff t} \score2\\
&=\frac{3t^2+1}{2t+1}, \score2\\
\frac{\diff^2 y}{\diff x^2}&=\frac{\diff y'/\diff t}{\diff x/\diff t} \score2\\
&=\frac{6t(2t+1)-(3t^2+1)2}{(2t+1)^3}=\frac{6t^2+6t-2}{(2t+1)^3}. \score2
\end{align*}

\textbf{4. (8分)【解】}

\indent 由于 $f(x)$ 在 $x=0$ 处连续, 因此
\begin{align*}
f(0)&=f(0^+) \score1\\
		&=b=\lim_{x\to0^-}x\arctan\frac1x=0\times\left(-\frac\pi2\right)=0. \score1
\end{align*}

\indent 由于 $f(x)$ 在 $x=0$ 处可导, 因此
\begin{align*}
f'_-(0)&=f'_+(0), \score1\\
f'_-(0)&=\lim_{x\to0^-}\frac{x\arctan\frac1x}x=\lim_{x\to0^-}\arctan\frac1x=-\frac\pi2 \score1\\
f'_+(0)&=(2x+a)|_{x=0}=a, \score1
\end{align*}
因此 $a=-\dfrac\pi2$. \score1

\indent 由于
\begin{align*}
\lim_{x\to+\infty}\frac yx&=\lim_{x\to+\infty}\left(x-\frac\pi2\right)=+\infty, \score1\\
\lim_{x\to-\infty}\frac yx&=\lim_{x\to-\infty}\arctan\frac1x=0,\\
\lim_{x\to-\infty}y&=\lim_{x\to-\infty}x\arctan\frac1x=\lim_{t\to0^-}\frac{\arctan t}t=1,
\end{align*}
因此曲线 $y=f(x)$ 的渐近线只有 $y=1$. \score1

\newpage

\textbf{5. (8分)【解】}

\indent 由
\[f'(x)=3x^2-2x-1=(3x+1)(x-1)=0 \score2\]
可得驻点 $x=-\dfrac13,1$. \score2

\indent 由于
\[f(-2)=-10,\quad f(2)=2,\quad f\left(-\frac13\right)=\frac5{27},\quad f(1)=-1, \score2\]
因此最大值为 $2$, 最小值为 $-10$. \score2

\textbf{6. (8分)【证明】}

\textbf{证法一}: 设 $f(x)=\tan x-x$, 则 \score2
\begin{align*}
f'(x)=\frac1{\cos^2x}-1=\tan^2x\ge0. \score2
\end{align*}
因此 $f(x)$ 在 $\left(-\dfrac\pi2,\dfrac\pi2\right)$ 上单调递增, 从而 \score2
\begin{align*}
f(x_2)\ge f(x_1),\quad\tan x_2-\tan x_1\ge x_2-x_1. \score2
\end{align*}

\textbf{证法二}: 设 $f(x)=\tan x$, 则 $f(x)$ 在 $[x_1,x_2]$ 上连续, $(x_1,x_2)$ 内可导. \score2

\indent 由拉格朗日中值定理, 存在 $\xi\in(x_1,x_2)$ 使得
\begin{align*}
\frac{f(x_2)-f(x_1)}{x_2-x_1}=f'(\xi), \score2
\end{align*}
即
\begin{align*}
\frac{\tan x_2-\tan x_1}{x_2-x_1}=\frac1{\cos^2\xi}\ge1. \score2
\end{align*}
所以 $\tan x_2-\tan x_1\ge x_2-x_1$. \score2
\newpage

\textbf{7. (8分)【证明】}

\indent 设 $F(x)=x^{2022}f(x)$, \score2\\
则 $F(x)$ 在 $[0,1]$ 上连续, $(0,1)$ 内可导, \score1\\
且 $F(0)=0,F(1)=f(1)=0$. \score1

\indent 由罗尔中值定理, 存在 $\xi\in(0,1)$ 使得 $F'(\xi)=0$. \score2\\
由于 $F'(x)=x^{2022}f'(x)+2022x^{2021}f(x)$ 且 $\xi\neq0$, \score1\\
所以 $\xi f'(\xi)+2022f(\xi)=1$. \score1


\textbf{8. (8分)【解】}

(1) 
\[f'(x)=\frac1x-\frac4{x^3}=\frac{x^2-4}{x^3}=\frac{(x+2)(x-2)}{x^3}. \score1\]
当 $0<x<2$ 时, $f'(x)<0$. 当 $x>2$ 时, $f'(x)>0$. \score1\\
因此 $(0,2]$ 是 $f(x)$ 的单减区间, $[2,+\infty)$ 是 $f(x)$ 的单增区间. \Score{(1分, 写成开区间不扣分)}\\
所以 $f(x)$ 只有唯一的极小值 $f(2)=\ln2+\dfrac12$. \score1

(2) 
\[f''(x)=-\frac1{x^2}+\frac{12}{x^4}=-\frac{x^2-12}{x^4}=-\frac{(x-2\sqrt3)(x+2\sqrt3)}{x^4}. \score1\]
当 $0<x<2\sqrt3$ 时, $f''(x)>0$. 当 $x>2\sqrt3$ 时, $f''(x)<0$. \score1\\
因此 $(0,2\sqrt3]$ 是曲线 $y=f(x)$ 的凹区间, \\
$[2\sqrt3,+\infty)$ 是曲线 $y=f(x)$ 的凸区间, \Score{(1分, 写成开区间不扣分)}\\
拐点为 $\left(2\sqrt3,\ln(2\sqrt3)+\dfrac16\right)$. \score1

\end{document}

