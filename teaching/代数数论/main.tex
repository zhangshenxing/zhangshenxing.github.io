% 支持的 theme 有 doremi, madoka, 
\documentclass[nocolor,theme=doremi,lang=cn,11pt,chinese,twoside,openright,usesamecnt]{elegantbook}
\PassOptionsToPackage{style=alphabetic}{biblatex}
% \PassOptionsToPackage{hyphens}{url}
\usepackage[all]{xy}
\usepackage{stmaryrd}
\usepackage{makeidx}
\usepackage{pfnote}
% \usepackage{breakurl}


% \newfontface\cmunrm{cmunrm.otf}\newcommand\cmu[1]{{\cmunrm{#1}}}

\let\question\relax
\elegantnewtheorem{question}{问题}{defstyle}{que}

\newcommand{\ra}{\rightarrow}
\newcommand{\lra}{\longrightarrow}
\newcommand{\la}{\leftarrow}
\newcommand{\lla}{\longleftarrow}
\newcommand{\inj}{\hookrightarrow}
\newcommand{\surj}{\twoheadrightarrow}
\newcommand{\sto}[1]{\stackrel{#1}{\longrightarrow}}
\newcommand{\simto}{\sto{\sim}}
\newcommand{\lto}{\longmapsto}
\newcommand{\ilim}{\varinjlim\limits}
\newcommand{\plim}{\varprojlim\limits}
\newcommand{\wh}{\widehat}
\newcommand{\wt}{\widetilde}
\newcommand{\ov}{\overline}
\newcommand{\ul}{\underline}
\newcommand\ldb{\llbracket}
\newcommand\rdb{\rrbracket}
\newcommand{\half}{\frac{1}{2}}
\newcommand{\mmid}{\parallel}
\newcommand{\fct}[4]{\begin{split}#1 &\lra #2 \\ #3 &\lto #4\end{split}}
\newcommand{\set}[1]{\left\{#1\right\}}
\newcommand{\pair}[1]{\langle{#1}\rangle}
\newcommand{\pmat}[4]{\begin{pmatrix}#1 & #2 \\ #3 & #4\end{pmatrix} }
\newcommand{\smat}[4]{\left(\begin{smallmatrix}#1 & #2 \\ #3 & #4\end{smallmatrix}\right) }
\newcommand{\hil}[3]{\left(\frac{{#1},{#2}}{#3}\right)}
\newcommand{\leg}[2]{\left(\frac{{#1}}{#2}\right)}
\newcommand{\cA}{{\mathsf{A}}}
\newcommand{\cB}{{\mathsf{B}}}
\newcommand{\cC}{{\mathsf{C}}}
\newcommand{\cAb}{{\mathsf{Ab}}}
\newcommand{\cComm}{{\mathsf{Comm}}}
\newcommand{\cFunc}{{\mathsf{Func}}}
\newcommand{\cGroups}{{\mathsf{Groups}}}
\newcommand{\cMod}{{\mathsf{Mod}}}
\newcommand{\cRings}{{\mathsf{Rings}}}
\newcommand{\cSets}{{\mathsf{Sets}}}
\newcommand{\cVect}{{\mathsf{Vect}}}
\newcommand{\bfN}{{\mathbf{N}}}
\newcommand{\bfX}{{\mathbf{X}}}
\newcommand{\bfY}{{\mathbf{Y}}}
\newcommand{\BA}{{\mathbb{A}}}
\newcommand{\BC}{{\mathbb{C}}}
\newcommand{\BF}{{\mathbb{F}}}
\newcommand{\BI}{{\mathbb{I}}}
\newcommand{\BN}{{\mathbb{N}}}
\newcommand{\BP}{{\mathbb{P}}}
\newcommand{\BQ}{{\mathbb{Q}}}
\newcommand{\BR}{{\mathbb{R}}}
\newcommand{\BZ}{{\mathbb{Z}}}
\newcommand{\CF}{{\mathcal{F}}}
\newcommand{\CG}{{\mathcal{G}}}
\newcommand{\CH}{{\mathcal{H}}}
\newcommand{\CI}{{\mathcal{I}}}
\newcommand{\CL}{{\mathcal{L}}}
\newcommand{\CN}{{\mathcal{N}}}
\newcommand{\CO}{{\mathcal{O}}}
\newcommand{\CP}{{\mathcal{P}}}
\newcommand{\CS}{{\mathcal{S}}}
\newcommand{\rmH}{{\mathrm{H}}}
\newcommand{\rmT}{{\mathrm{T}}}
\newcommand{\fa}{{\mathfrak{a}}}
\newcommand{\fb}{{\mathfrak{b}}}
\newcommand{\fd}{{\mathfrak{d}}}
\newcommand{\ff}{{\mathfrak{f}}}
\newcommand{\fm}{{\mathfrak{m}}}
\newcommand{\fp}{{\mathfrak{p}}}
\newcommand{\fq}{{\mathfrak{q}}}
\newcommand{\fA}{{\mathfrak{A}}}
\newcommand{\fD}{{\mathfrak{D}}}
\newcommand{\fF}{{\mathfrak{F}}}
\newcommand{\fH}{{\mathfrak{H}}}
\newcommand{\fP}{{\mathfrak{P}}}
\newcommand{\ab}{{\mathrm{ab}}}
\newcommand{\adele}{ad\'{e}le}
\newcommand{\Ann}{{\mathrm{Ann}}}
\newcommand{\Aut}{{\mathrm{Aut}}}
\newcommand{\bs}{\backslash}
\newcommand{\Char}{{\mathrm{char}}}
\newcommand{\Cl}{{\mathrm{Cl}}}
\newcommand{\coker}{{\mathrm{coker}\,}}
\newcommand{\Coker}{{\mathrm{Coker}\,}}
\newcommand{\coim}{{\mathrm{coim}\,}}
\newcommand{\CoIm}{{\mathrm{CoIm}\,}}
\newcommand{\covol}{{\mathrm{covol}}}
\newcommand{\cyc}{{\mathrm{cyc}}}
\newcommand{\diag}{{\mathrm{diag}}}
\newcommand{\diff}{\mathop{}\!\mathrm{d}}
\newcommand{\diffx}{\mathop{}\!\mathrm{d}^\times}
\newcommand{\disc}{{\mathrm{disc}}}
\renewcommand{\div}{{\mathrm{div}}}
\newcommand{\Div}{{\mathrm{Div}}}
\newcommand{\End}{{\mathrm{End}}}
\newcommand{\Fr}{{\mathrm{Frac}\,}}
\newcommand{\Frob}{{\mathrm{Frob}}}
\newcommand{\Ga}{\mathbb{G}_a}
\DeclareMathOperator{\Gal}{{\mathrm{Gal}}}
% \newcommand{\Gal}{G}
\newcommand{\Gm}{\mathbb{G}_m}
\newcommand{\GL}{{\mathrm{GL}}}
\newcommand{\Hom}{{\mathrm{Hom}}}
\newcommand{\id}{{\mathrm{id}}}
\newcommand{\idele}{id\'{e}le}
\renewcommand{\Im}{{\mathrm{Im}\,}}
\newcommand{\im}{{\mathrm{im}\,}}
\newcommand{\Ind}{{\mathrm{Ind}}}
\newcommand{\Ker}{{\mathrm{Ker}\,}}
\newcommand{\LT}{\mathcal{LT}}
\renewcommand{\mod}{\bmod}
\newcommand{\ns}{{\mathrm{ns}}}
\newcommand{\Obj}{{\mathrm{Obj}\,}}
\newcommand{\op}{{\mathrm{op}}}
\newcommand{\ord}{{\mathrm{ord}}}
\newcommand{\PGL}{{\mathrm{PGL}}}
\newcommand{\prim}{{\mathrm{prim}}}
\newcommand{\rank}{{\mathrm{rank}}}
\renewcommand{\Re}{{\mathrm{Re}}}
\newcommand{\Res}{{\mathrm{Res}}}
\newcommand{\sep}{{\mathrm{sep}}}
\newcommand{\sgn}{{\mathrm{sgn}}}
\newcommand{\SL}{{\mathrm{SL}}}
\newcommand{\suml}{\sum\limits}
\newcommand{\Tr}{{\mathrm{Tr}}}
\newcommand{\ur}{{\mathrm{ur}}}
\newcommand{\Ver}{{\mathrm{Ver}}}
\newcommand{\vK}{{\breve K}}
\newcommand{\vol}{{\mathrm{vol}}}
\newfontface\cmunrm{cmunrm.otf}
\newcommand\cmu[1]{{\cmunrm{#1}}}
\newcommand{\alert}[1]{\textcolor{main}{\bf #1}}

\newcommand{\cnen}[2]{{\kaishu$\overset{\text{\clap{#2}}}{\text{#1}}$}}
% \newcommand{\cnen}[2]{#1}
\newcommand{\nounen}[2]{{\color{third}\kaishu\cnen{#1}{#2}}\index{{#1}}}

\newcommand{\noun}[1]{{\color{third}\kaishu #1}\index{{#1}}}
\newcommand{\nouns}[2]{{\color{third}\kaishu #1}\index{{#2}}}
\newcommand{\nounsen}[3]{{\color{third}\kaishu\cnen{#1}{#2}}\index{{#3}}}

\setcounter{tocdepth}{2}
\newfontfamily\couriernew{Courier New}
\lstset{language=[LaTeX]TeX,
	basicstyle=\couriernew,
  morekeywords={AUTHOR, KEY, TITLE, YEAR, PAGES, HOWPUBLISHED, URL, LANGUAGE},
  keywordstyle=\color{winered}
}

\makeindex

\title{代数数论讲义}
\author{张神星}
\institute{合肥工业大学}
\date{\zhtoday}
\version{v2.5.0.0}
\logo{hfut.png}
\cover{theme/\themename/cover.jpg}
\addbibresource[location=local]{mybib.bib}

\begin{document}

\maketitle
\hypersetup{pageanchor=true}
\frontmatter

\chapter*{前言}
\addcontentsline{toc}{chapter}{前言}
本文为2020年春作者在中国科学技术大学教授代数数论(MA05109)的课程讲义.
本文主要沿着\cite{Neukirch1999}的脉络进行的, 对部分章节进行了增减.
本课程需要的前置内容包括线性代数、抽象代数和伽罗瓦理论.

2020年, 开年就是新型冠状病毒疫情, 这也导致学校的教学首次安排在线上教学. 现在, 在家中码字的我只希望, 人们能够早日战胜疫病, 加油!

\vskip 0.5cm
\begin{flushright}
张神星\\
2020年2月7日
\end{flushright}


\vskip 0.5cm

\noindent Bib 格式:
\begin{lstlisting}
@misc{ZhangNotes2021,
  AUTHOR = {张神星},
  KEY = {zhang2 shen4 xing2},
  TITLE = {代数数论讲义 (v2.3)},
  YEAR = {2021},
  PAGES = {vi+163},
  HOWPUBLISHED = {Course Notes},
  URL = {https://zhangshenxing.gitee.io/teaching/代数数论讲义.pdf},
  LANGUAGE = {Chinese}
}
\end{lstlisting}

\tableofcontents




\mainmatter
\input{chapter1.tex}
\input{chapter2.tex}
\input{chapter3.tex}
\input{chapter4.tex}
% 计划中: order, division algebra, Brauer group


\appendix
\input{appendix.tex}


\backmatter
\printbibliography[heading=bibintoc, title=\ebibname]

\chapter*{中外人名对照表}
%\addcontentsline{toc}{chapter}{中外人名对照表}
\markboth{中外人名对照表}{中外人名对照表}
\begin{center}
\begin{tabular}{lr}
阿贝尔&Niels Henrik Abel, 1802--1829\\
阿基米德&\cmu{Ἀρχιμήδης}, 公元前 287--前 212\\
埃尔布朗&Jacques Herbrand, 1908--1931\\
艾森斯坦&Ferdinand Gotthold Max Eisenstein, 1823--1852\\
奥斯特洛斯基&\cmu{Олександр Маркович Островський}, 1893--1986\\
贝克&Alan Baker, 1939--2018\\
伯奇&Bryan John Birch, 1931--~~~~\\
泊松&Siméon Denis Poisson, 1781--1840\\
戴德金&Julius Wilhelm Richard Dedekind, 1831--1916\\
狄利克雷&Johann Peter Gustav Lejeune Dirichlet, 1805--1859\\
法尔廷斯&Gerd Faltings, 1954--~~~~\\
方丹&Jean-Marc Fontaine, 1944--2019\\
费马&Pierre de Fermat, 1601--1665\\
傅里叶&Jean-Baptiste Joseph Fourier, 1768--1830\\
弗罗贝尼乌斯&Ferdinand Georg Frobenius, 1849--1917\\
伽罗瓦&Évariste Galois, 1811-1832\\
高斯&Johann Carl Friedrich Gauß, 1777--1855\\
谷山丰&谷山豊, 1927--1953\\
哈尔&Alfréd Haar, 1885--1933\\
哈塞&Helmut Hasse, 1898--1979\\
豪斯多夫&Felix Hausdorff, 1868--1942\\
赫克&Erich Hecke, 1887--1947\\
亨泽尔&Kurt Hensel, 1861--1941\\
怀尔斯&Sir Andrew John Wiles, 1953--~~~~\\
克拉斯纳&Marc Krasner, 1912--1985\\
克鲁尔&Wolfgang Krull, 1899--1971\\
克罗内克&Leopold Kronecker, 1823--1891\\
柯西&Augustin-Louis Cauchy, 1789--1857\\
库默尔&Ernst Eduard Kummer, 1810--1893\\
莱布尼茨&Gottfried Wilhelm Freiherr von Leibniz, 1646--1716\\
勒让德&Adrien-Marie Legendre, 1752--1833\\
黎曼&Georg Friedrich Bernhard Riemann, 1826--1866\\
卢宾&Jonathan Darby Lubin, 1936--~~~~\\
闵可夫斯基&Hermann Minkowski, 1864--1909\\
默比乌斯&August Ferdinand Möbius, 1790--1868
\end{tabular}
\end{center}
\begin{center}
\begin{tabular}{lr}
牛顿&Sir Isaac Newton, 1643--1727\\
诺特&Amalie Emmy Noether, 1882--1935\\
欧拉&Leonhardus Eulerus, 1707--1783\\
庞特里亚金&\cmu{Лев Семёнович Понтря́гин}, 1908--1988\\
佩尔&John Pell, 1611--1685\\
切博塔廖夫&\cmu{Мико́ла Григо́рович Чеботарьо́в}, 1894--1947\\
塞尔&Jean-Pierre Serre, 1926--~~~~\\
施瓦兹&Laurent-Moïse Schwartz, 1915--2002\\
斯温纳顿-戴尔&Sir Henry Peter Francis Swinnerton-Dyer, 1927--2018\\
沙法列维奇&\cmu{И́горь Ростисла́вович Шафаре́вич}, 1923--2017\\
泰勒&Brook Taylor, 1685--1731\\
泰特&John Torrence Tate, 1925--2019\\
泰希米勒&Paul Julius Oswald Teichm\"uller, 1913--1943\\
韦伯&Wilhelm Eduard Weber, 1804--1891\\
魏尔斯特拉斯&Karl Theodor Wilhelm Weierstraß, 1815--1897\\
维特&Ernst Witt, 1911--1991\\
韦伊&André Weil, 1906--1998\\
希尔伯特&David Hilbert, 1862--1943\\
伯努利&Jacques Bernoulli, 1654--1705\\
岩泽健吉&岩澤健吉, 1917--1998\\
志村五郎&志村五郎, 1930--2019
\end{tabular}
\end{center}

%\newpage
%\phantomsection
%\addcontentsline{toc}{chapter}{索引}

\printindex

\end{document}