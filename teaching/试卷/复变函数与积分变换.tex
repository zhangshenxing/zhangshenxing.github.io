\documentclass[simple]{hfutexam}
\usepackage{fixdif,derivative}
\usepackage{mathrsfs}
\DeclareMathOperator{\Res}{Res}
\DeclareMathOperator{\Ln}{Ln}
\DeclareMathOperator{\Arg}{Arg}
\newcommand\msl{\mathscr{L}}
\newcommand\BR{\mathbb{R}}
\newcommand\BZ{\mathbb{Z}}
\let\Im\relax\DeclareMathOperator{\Im}{Im}
\let\Re\relax\DeclareMathOperator{\Re}{Re}
\letdif{\delt}{Delta} % 微分运算符
\usepackage[nolimits]{cmupint} % 提供 \int, \oint
\newcommand{\ii}{\mathrm{i}}
\newcommand{\jj}{\mathrm{j}}
\newcommand{\ee}{\mathrm{e}}
\usepackage{textgreek}
\newcommand{\cpi}{{\text{\textpi}}}
\newcommand{\dirac}{{\text{\textdelta}}}
\newcommand{\dint}{\displaystyle\int}
\newcommand{\doint}{\displaystyle\oint}
\newcommand{\sumf}[1]{\displaystyle\sum_{n=#1}^{\infty}}
\newcommand{\sumff}{\displaystyle\sum_{n=-\infty}^{+\infty}}
\newcommand\ov{\overline}

\begin{document}
\BiaoTi{2022年合肥工业大学试卷(A)}
\XueQi{一}
\XueNian{2022}{2023}
\KeChengDaiMa{1400261B}
\KeChengMingCheng{复变函数与积分变换}
\XueFen{2.5}
\KeChengXingZhi{必修}
\KaoShiXingShi{闭卷}
\ZhuanYeBanJi{}
\KaoShiRiQi{2022年11月26日19:00-21:00}
\MingTiJiaoShi{集体}
\maketitle

\tigan{一、填空题(每小题3分,共15分)}
\begin{enumerate}
  \item $\ii^{-\ii}$ 的主值是\fillblank{}.
  \item 设 $z=-\ii$, 则 $1+z+z^2+z^3+z^4=$\fillblank{}.
  \item 设 $C$ 为正向圆周 $|z|=2$, 则积分 $\doint_C\Bigl(\frac{~\ov z~}{z}\Bigr)\d z=$\fillblank{}.
  \item 设 $a,b,c$ 为实数. 如果函数 $f(z)=x^2-2xy-y^2+\ii (ax^2+bxy+cy^2)$ 在复平面上处处解析, 则 $a+b+c=$\fillblank{}.
  \item 函数 $\sin t+\jj \cos t$ 的傅里叶变换为\fillblank{}.
\end{enumerate}

\tigan{二、选择题(每小题3分,共15分)}
\begin{enumerate}
  \item 方程 $\bigl||z+\ii|-|z-\ii|\bigr|=1$ 表示的曲线是(~~~~).
  \xx{直线}{不是圆的椭圆}{双曲线}{圆周}
  \item 不等式 $-1\le\arg z\le \cpi-1$ (包括 $0$) 确定是的(~~~~).
  \xx{有界多连通闭区域}{有界单连通区域}{无界多连通区域}{无界单连通闭区域}
  \item 幂级数 $\sumf1 (\ii z)^n$ 的收敛半径是(~~~~).
  \xx{$\ii$}{$-\ii$}{$1$}{$+\infty$}
  \item 下面哪个函数在 $z=0$ 处不可导?~(~~~~)
  \xx{$2x+3y\ii$}{$2x^2+3y^2\ii$}{$x^2-xy\ii$}{$\ee^x\cos y+\ii\ee^x\sin y$}
  \item 如果 $z_0$ 是 $f(z)$ 的一阶极点, $g(z)$ 的一阶零点, 则 $z_0$ 是 $f(z)^3g(z)^2$ 的(~~~~).
  \xx{一阶极点}{一阶零点}{可去奇点}{三阶极点}
\end{enumerate}

\tigan{三、解答题}
\begin{enumerate}
  \item \textbf{(6分)} 设 $z=\dfrac{3+\ii}{\ii}-\dfrac{10\ii}{3-\ii}$, 求 $z$ 的模和辐角.
  \item \textbf{(6分)} 解方程 $\sin z=2\cos z$.
  \item \textbf{(6分)} 设 $C$ 为从 $\ii$ 到 $\ii-\cpi$ 再到 $-\cpi$ 的折线, 求 $\dint_C\cos^2z\d z$.
  \item \textbf{(10分)} 设 $C$ 为正向圆周 $|z-3|=4$, 求 $\doint_C\frac{\ee^{\ii z}}{z^2-3\cpi z+2\cpi^2}\d z$.
  \item \textbf{(10分)} 假设 $v(x,y)=x^3+y^3-axy(x+y)$ 是调和函数,求参数 $a$ 以及解析函数 $f(z)$ 使得 $v(x,y)$ 是它的虚部.
  \item \textbf{(10分)} 确定函数 $f(z)=\dfrac{z+1}{(z-1)^2}$ 在圆环域\\
  \indent (1) $0<|z|<1$; \hspace{2em} (2) $1<|z|<+\infty$\\
  内的洛朗级数展开式.
  \item \textbf{(10分)} 求 $f(z)=\dfrac{\cos z}{z^2(z^2-\cpi^2)}$ 在有限复平面内的奇点和相应的留数.
  \item \textbf{(9分)} 用拉普拉斯变换求解微分方程初值问题
  \[\begin{cases}
    y''(t)+2y(t)=\sin t,\\
    y(0)=0,\quad y'(0)=2.
  \end{cases}\]
  \item \textbf{(3分)} 复变函数 $f(z)=\sin z$ 和实变量函数 $g(x)=\sin x$ 的性质有什么相似和不同之处? 试列举一二.
\end{enumerate}


\newpage
\BiaoTi{2022年合肥工业大学考试参考答案(A)}
\XueQi{一}
\XueNian{2022}{2023}
\KeChengDaiMa{1400261B}
\KeChengMingCheng{复变函数与积分变换}
\XueFen{2.5}
\KeChengXingZhi{必修}
\KaoShiXingShi{闭卷}
\ZhuanYeBanJi{}
\KaoShiRiQi{2022年11月26日19:00-21:00}
\MingTiJiaoShi{集体}
\maketitle

\tigan{一、填空题(每小题3分,共15分)}

\textbf{请将你的答案对应填在横线上:}

\textbf{1.} \fillblank[2.5cm]{$\ee^{\cpi/2}$}, 
\textbf{2.} \fillblank[2.5cm]{$1$}, 
\textbf{3.} \fillblank[2.5cm]{$0$}, 
\textbf{4.} \fillblank[2.5cm]{$2$}, 
\textbf{5.} \fillblank[2.5cm]{$2\cpi\jj\dirac(\omega+1)$}.

\tigan{二、选择题(每小题3分,共15分)}

\textbf{请将你所选择的字母 A, B, C, D 之一对应填在下列表格里:}

\xuanzeti{\textbf{题号}}{\textbf{答案}}%
\xuanzeti{1}{C}%
\xuanzeti{2}{D}%
\xuanzeti{3}{C}%
\xuanzeti{4}{A}%
\xuanzeti{5}{A}

\tigan{三、解答题}

\textbf{1. (6分) 【解】}

由于 $z=-3\ii+1-\ii(3+\ii)=2-6\ii$, \score2\\
因此 $|z|=2\sqrt{10}$, \score2\\
$\Arg z=2k\cpi-\arctan 3,\quad k\in\BZ$. \Score{(2分, 只有主值得1分)}

\textbf{2. (6分) 【解】}
\begin{align*}
  \frac{\ee^{\ii z}-\ee^{-\ii z}}{2\ii}&
  =2\cdot\frac{\ee^{\ii z}+\ee^{-\ii z}}2, \score2\\
  \ee^{\ii z}-\ee^{-\ii z}&
  =2\ii (\ee^{\ii z}+\ee^{-\ii z}), \\
  \ee^{2\ii z}&
  =\frac{1+2\ii}{1-2\ii}=\frac{(1+2\ii)^2}5, \score1\\
  2\ii z&
  =\Ln\frac{(1+2\ii)^2}5=(2\arctan 2+2k\cpi)\ii, \score1\\
  z&=\arctan 2+k\cpi,\quad k\in\BZ. \Score{(2分, 只有主值得1分)}
\end{align*}
\textbf{另解}: 设 $z_0=\arctan 2$, 则
\[
  \tan z=\tan z_0,\quad \sin z\cos z_0=\sin z_0\cos z, \score2
\]
\[
  \sin(z-z_0)=0, \score2
\]
\[
  z=z_0+k\cpi=\arctan 2+k\cpi,\quad k\in\BZ.\Score{(2分, 只有主值得1分)}
\]
\textbf{其它答案}: $z=\dfrac\cpi2-\dfrac12\arctan\dfrac43+k\cpi, k\in\BZ$.
\smallskip

\textbf{3. (6分) 【解】}

由于 $\cos^2z$ 解析, 且 \score1
\begin{align*}
  \int\cos^2z\d z&
  =\int\frac{1+\cos(2z)}2\d z \score1\\&
  =\frac{z}2+\frac{\sin(2z)}4+C, \score1
\end{align*}
因此
\begin{align*}
  \int_C\cos^2z\d z&
  =\biggl(\frac{z}2+\frac{\sin(2z)}4\biggr)\bigg|_\ii^{-\cpi} \score1\\&
  =-\frac\cpi2-\biggl(\frac\ii2+\frac{\sin(2\ii)}4\biggr) \score1\\&
  =-\frac\cpi2+\frac{(\ee^{-2}-4-\ee^2)\ii}8. \score1
\end{align*}

\textbf{4. (10分) 【解】}

由于 $f(z)=\dfrac{\ee^{\ii z}}{z^2-3\cpi z+2\cpi^2}$ 在 $|z-3|\le 4$ 内的奇点为 $\cpi,2\cpi$, \score3\\
因此
\begin{align*}
  \oint_Cf(z)\d z&
  =2\cpi\ii\bigl(\Res[f(z),\cpi]+\Res[f(z),2\cpi]\bigr) \score2\\&
  =2\cpi\ii\biggl(\frac{\ee^{\ii z}}{z-2\cpi}\bigg|_{z=\cpi}+\frac{\ee^{\ii z}}{z-\cpi}\bigg|_{z=2\cpi}\biggr) \score2\\&
  =2\cpi\ii\biggl(\frac1\cpi+\frac1\cpi\biggr)=4\ii. \score3
\end{align*}

\textbf{5. (10分) 【解】}

由 $\delt v=v_{xx}+v_{yy}=6x-2ay+6y-2ax=0$ 可知 $a=3$. \score3\\
由
\begin{align*}
  f'(z)&
  =v_y+\ii v_x \score2\\&
  =(3y^2-3x^2-6xy)+\ii (3x^2-6xy-3y^2) \score2\\&
  =3(\ii-1)(x+\ii y)^2=3(\ii-1)z^2 \score1
\end{align*}
可知 $f(z)=(\ii-1)z^3+C,\quad C\in\BR$. \Score{(2分, 没有常数项得1分)}

\textbf{其它解法}: 由 $u_x=v_y=3y^2-3x^2-6xy$ 得 $u=3xy^2-x^3-3x^2y+\psi(y)$. \score2\\
由 $u_y=-v_x=-(3x^2-6xy-3y^2)$ 得 $\psi'(y)=3y^2$, \\
$\psi(y)=y^3+C$, \Score{(3分, 没有常数项得2分)}
\begin{align*}
f(z)&=u+\ii v \\&
=3xy^2-x^3-3x^2y+y^3+C+\ii (x^3+y^3-3xy^2-3x^2y) \\&
=(\ii-1)z^3+C,\quad C\in\BR. \score2
\end{align*}

\textbf{6. (10分) 【解】}

由于 $f(z)$ 的奇点是 $1$, 因此 $f(z)$ 在这两个圆环域内都解析.

(1)
由于
\[\frac1{1-z}=\sumf0 z^n, \score1\]
因此
\begin{align*}
  f(z)&
  =\frac{z-1+2}{(z-1)^2}=\frac1{z-1}+\frac{2}{(z-1)^2}=-\frac1{1-z}+2\biggl(\frac1{1-z}\biggr)' \score2\\&
  =-\sumf0 z^n+2\biggl(\sumf0 z^n\biggr)'
  =-\sumf0 z^n+2\sumf1 nz^{n-1} \\&
  =-\sumf0 z^n+2\sumf0 (n+1)z^n
  =\sumf0 (2n+1)z^n. \score2
\end{align*}

(2) 
由于
\[
  \frac1{z-1}=\frac1z\cdot\frac{1}{1-\frac1z}
  =\sumf1 z^{-n}, \score1
\]
因此
\begin{align*}
  f(z)&=\frac1{z-1}-2\Bigl(\frac1{z-1}\Bigr)'
  =\sumf1 z^{-n}-2\Bigl(\sumf1 z^{-n}\Bigr)' \score2\\&
  =\sumf1 z^{-n}-2\sumf1 (-n)z^{-n-1} \\&
  =\sumf1 z^{-n}-2\sumf1 (-n+1)z^{-n}
  =\sumf1 (2n-1)z^{-n}. \score2
\end{align*}

\textbf{7. (10分) 【解】}

由于 $0$ 是分母的二阶零点, 因此它是 $f(z)$ 的二阶极点. \score1\\
由于 $\pm\cpi$ 是分母的一阶零点, 因此它们是 $f(z)$ 的一阶极点. \score1
\begin{align*}
  \Res[f(z),0]&
  =\Bigl(\frac{\cos z}{z^2-\cpi^2}\Bigr)'\bigg|_{z=0} \score2\\&
  =\frac{-\sin z\cdot(z^2-\cpi^2)-\cos z\cdot 2z}{(z^2-\cpi^2)^2}\bigg|_{z=0}=0, \score2\\
  \Res[f(z),\cpi]&
  =\frac{\cos z}{z^2(z+\cpi)}\bigg|_{z=\cpi}
  =-\frac1{2\cpi^3}, \score2\\
  \Res[f(z),-\cpi]&
  =\frac{\cos z}{z^2(z-\cpi)}\bigg|_{z=-\cpi}
  =\frac1{2\cpi^3}. \score2
\end{align*}

\textbf{8. (9分) 【解】}

设 $\msl[y]=Y$, 则
\[\msl[y'']=s^2Y-sy(0)-y'(0)=s^2Y-2, \score3\]
因此
\begin{align*}
  s^2Y-2+2Y&
  =\msl[\sin t]=\frac{1}{s^2+1}, \score2\\
  Y(s)&
  =\frac{2}{s^2+2}+\frac{1}{(s^2+1)(s^2+2)}
  =\frac{1}{s^2+1}+\frac{1}{s^2+2}, \score2\\
  y(t)&
  =\msl^{-1}\Bigl[\frac{1}{s^2+1}\Bigr]+\msl^{-1}\Bigl[\frac{1}{s^2+2}\Bigr]
  =\sin t+\frac{\sqrt 2}2\cdot\sin(\sqrt 2 t). \score2
\end{align*}

\textbf{9. (3分) 【解】}

例如(每项1分)
\begin{itemize}
  \item $f'(z)=\cos z,g'(x)=\cos x$. \score1
  \item $\sin z$ 处处可导, $\sin x$ 处处可导. \score1
  \item 麦克劳林展开的系数相同. \score1
  \item $\sin z$ 无界, $\sin x$ 有界. \score1
\end{itemize}


\newpage
\BiaoTi{2022年合肥工业大学试卷(B)}
\XueQi{一}
\XueNian{2022}{2023}
\KeChengDaiMa{1400261B}
\KeChengMingCheng{复变函数与积分变换}
\XueFen{2.5}
\KeChengXingZhi{必修}
\KaoShiXingShi{闭卷}
\ZhuanYeBanJi{}
\KaoShiRiQi{}
\MingTiJiaoShi{集体}
\maketitle

\tigan{一、填空题(每小题3分,共15分)}
\begin{enumerate}
  \item $-1+\sqrt 3\ii$ 的辐角主值是\fillblank{}.
  \item $\ii^{2022}-(-\ii)^{2022}=$\fillblank{}.
  \item 如果函数 $f(z)=\ee^{ax}(\cos y-\ii \sin y)$ 在复平面上处处解析, 则实数 $a=$\fillblank[25mm]{}.
  \item 设 $C$ 为正向圆周 $|z|=1$, 则积分 $\doint_C\Bigl(\frac{1+z+z^2}{z^3}\Bigr)\d z=$\fillblank{}.
  \item 函数 $\ee^{\jj t}$ 的傅里叶变换为\fillblank{}.
\end{enumerate}

\tigan{二、选择题(每小题3分,共15分)}
\begin{enumerate}
  \item 不等式 $1<|z|<2$ 确定是的(~~~~).
  \xx{有界多连通区域}{有界单连通区域}{无界多连通区域}{无界单连通区域}
  \item 方程 $|z+\ii|=|z-\ii|$ 表示的曲线是(~~~~).
  \xx{直线}{不是圆的椭圆}{双曲线}{圆周}
  \item 幂级数在其收敛圆周上(~~~~).
  \xx{一定处处绝对收敛}{一定处处条件收敛}{一定有发散的点}{可能处处收敛也可能有发散的点}
  \item 函数 $f(z)=u(x,y)+\ii v(x,y)$ 在 $z_0=x_0+\ii y_0$ 处可导的充要条件是(~~~~).
  \xx{$u,v$ 均在 $(x_0,y_0)$ 处连续}{$u,v$ 均在 $(x_0,y_0)$ 处有偏导数}{$u,v$ 均在 $(x_0,y_0)$ 处可微}{$u,v$ 均在 $(x_0,y_0)$ 处可微且满足C-R方程}
  \item $z=\cpi$ 是函数 $\dfrac{\sin z}{(z-\cpi)^2}$ 的(~~~~).
  \xx{一阶极点}{一阶零点}{可去奇点}{本性奇点}
\end{enumerate}

\tigan{三、解答题}
\begin{enumerate}
  \item \textbf{(6分)} 设 $z=\dfrac{2+\ii}{1-2\ii}$, 求 $z$ 的模和辐角.
  \item \textbf{(6分)} 求 $\sqrt[3]{-8}$.
  \item \textbf{(7分)} 设 $C$ 是从 $\ii$ 到 $2+\ii$ 的直线, 求 $\dint_C \ov z\d z$.
  \item \textbf{(7分)} 求 $\dint_{-\cpi\ii}^{\cpi\ii}(\ee^z+1)\d z$.
  \item \textbf{(7分)} 求 $\dint_0^\cpi (z+\cos 2z)\d z$.
  \item \textbf{(7分)} 设 $C$ 为正向圆周 $|z|=4$, 求 $\doint_C\frac{z-6}{z^2+9}\d z$.
  \item \textbf{(8分)} 已知 $f(z)=u+\ii v$ 是解析函数, 其中 $u(x,y)=x^2+axy-y^2, v=2x^2-2y^2+2xy$ 且 $a$ 是实数.
求参数 $a$ 以及解析函数 $f'(z)$, 其中 $f'(z)$ 需要写成 $z$ 的表达式.
  \item \textbf{(10分)} 确定函数 $f(z)=\dfrac{2}{z(z+2)}$ 在圆环域\\
\indent (1) $0<|z|<2$; \hspace{2em} (2) $2<|z|<+\infty$\\
内的洛朗级数展开式.
  \item \textbf{(9分)} 用拉普拉斯变换求解微分方程初值问题
\[\begin{cases}
  y''(t)+2y'(t)=8\ee^{2t},\\
  y(0)=0,\quad y'(0)=2.
\end{cases}\]
  \item \textbf{(3分)} 复积分的计算方法或公式有哪些? 请给出至少三条.
\end{enumerate}


\newpage
\BiaoTi{2022年合肥工业大学考试参考答案(B)}
\XueQi{一}
\XueNian{2022}{2023}
\KeChengDaiMa{1400261B}
\KeChengMingCheng{复变函数与积分变换}
\XueFen{2.5}
\KeChengXingZhi{必修}
\KaoShiXingShi{闭卷}
\ZhuanYeBanJi{}
\KaoShiRiQi{}
\MingTiJiaoShi{集体}
\maketitle

\tigan{一、填空题(每小题3分,共15分)}

\textbf{请将你的答案对应填在横线上:}

\textbf{1.} \fillblank[2.5cm]{$2\cpi/3$}, 
\textbf{2.} \fillblank[2.5cm]{$0$}, 
\textbf{3.} \fillblank[2.5cm]{$-1$}, 
\textbf{4.} \fillblank[2.5cm]{$2\cpi\ii$}, 
\textbf{5.} \fillblank[2.5cm]{$2\cpi\dirac(\omega-1)$}.

\tigan{二、选择题(每小题3分,共15分)}

\textbf{请将你所选择的字母 A, B, C, D 之一对应填在下列表格里:}

\xuanzeti{\textbf{题号}}{\textbf{答案}}%
\xuanzeti{1}{A}%
\xuanzeti{2}{A}%
\xuanzeti{3}{D}%
\xuanzeti{4}{D}%
\xuanzeti{5}{A}

\tigan{三、解答题}

\textbf{1. (6分) 【解】}
由于 $z=\dfrac{(2+\ii)(1+2\ii)}{5}=\ii$, \score2\\
因此 $|z|=1$, \score2\\
$\Arg z=\dfrac\cpi2+2k\cpi,\quad k\in\BZ$. \Score{(2分, 只有主值得1分)}
\smallskip

\textbf{2. (6分) 【解】}
由于 $-8=8\ee^{\cpi\ii}$,\score2\\
因此
\[
  \sqrt[3]{-8}=2\ee^{\frac13(\cpi\ii+2k\cpi)},\quad k=0,1,2\score2
\]
即
\[
  2\ee^{\frac{\cpi\ii}3}=1+\sqrt3\ii,\quad
  2\ee^{\cpi\ii}=-2,\quad
  2\ee^{\frac{5\cpi\ii}3}=1-\sqrt3\ii.\score2
\]

\textbf{3. (7分) 【解】}

直线 $C$ 的方程为 $z=t+\ii ,0\le t\le 2$.\score2\\
因此 $\d z=\d t$, $\ov z=t-\ii$.\score2
\begin{align*}
  \int\ov z\d z&
  =\int_0^2(t-\ii)\d t \score2\\&
  =\Bigl(\frac{t^2}2-\ii t\Bigr)\Big|_0^2
  =2-2\ii .\score1
\end{align*}

\textbf{4. (7分) 【解】}
由于 $\ee^z+1$ 处处解析, 因此\score2
\begin{align*}
  \int_{-\cpi\ii}^{\cpi\ii}(\ee^z+1)\d z&
  =(\ee^z+z)\Big|_{-\cpi\ii}^{\cpi\ii}\score2\\&
  =(\ee^{\cpi\ii}+\cpi\ii)-(\ee^{-\cpi\ii}-\cpi\ii)\score2\\&
  =2\cpi\ii.\score1
\end{align*}

\textbf{5. (7分) 【解】}
由于 $z+\cos 2z$ 处处解析, 因此\score2
\begin{align*}
  \int_0^\cpi (z+\cos 2z)\d z
  &=\Bigl(\frac{z^2}2+\frac{\sin 2z}2\Bigr)\Big|_0^\cpi\score2\\&
  =\frac{\cpi^2}2+\frac{\sin 2\cpi}2\score2\\&
  =\frac{\cpi^2}2.\score1
\end{align*}

\textbf{6. (7分) 【解】}

由于 $f(z)=\dfrac{z-6}{z^2+9}$ 在 $|z|\le 4$ 内的奇点为 $\pm3\ii$, \score2\\
因此
\begin{align*}
  \oint_Cf(z)\d z&
  =2\cpi\ii\bigl(\Res[f(z),3\ii]+\Res[f(z),-3\ii]\bigr) \score2\\&
  =2\cpi\ii\biggl(\frac{z-6}{z+3\ii}\bigg|_{z=3\ii}+\frac{z-6}{z-3\ii}\bigg|_{z=-3\ii}\biggr) \score2\\&
  =2\cpi\ii\Bigl(\frac{3\ii-6}{6\ii}+\frac{-3\ii-6}{-6\ii}\Bigr)
  =2\cpi\ii \score1
\end{align*}

\textbf{7. (8分) 【解】}

由 $u_x=2x+ay=v_y=-4y+2x$ 可知 $a=-4$. \score2\\
由
\begin{align*}
  f'(z)&
  =u_x+\ii v_x \score2\\&
  =(-4y+2x)+\ii (4x+2y) \score2\\&
  =(2+4\ii)z. \score2
\end{align*}

\textbf{8. (10分) 【解】}

由于 $f(z)$ 的奇点是 $0,2$, 因此 $f(z)$ 在这两个圆环域内都解析.

(1)
\begin{align*}
  f(z)&
  =\frac1z-\frac1{z+2} \score2\\&
  =\frac1z-\frac12\cdot\frac1{1+\dfrac z2}\score1 \\&
  =\frac1z-\frac12\sumf0\Bigl(-\frac z2\Bigr)^n \score1 \\&
  =\frac1z+\sumf0\frac{(-1)^{n+1}}{2^{n+1}}z^n
  =\sumf{-1}\frac{(-1)^{n+1}}{2^{n+1}}z^n.\score1
\end{align*}

(2) 
\begin{align*}
  f(z)&=\frac1z-\frac1{z+2} \score2\\&
  =\frac1z-\frac1z\cdot\frac1{1+\dfrac 2z}\score1 \\&
  =\frac1z-\frac1z\sumf0\Bigl(-\frac 2z\Bigr)^n \score1 \\&
  =-\sumf1\frac{(-2)^n}{z^{n+1}}
  =-\sum_{n=2}^\infty\frac{(-2)^{n-1}}{z^n}.\score1
\end{align*}

\textbf{9. (9分) 【解】}

设 $\msl[y]=Y$, 则
\[
  \msl[y'']=s^2Y-sy(0)-y'(0)=s^2Y-2, \score3
\]
因此
\[
  s^2Y-2+2sY=\msl[8\ee^{2t}]=\frac8{s-2}, \score2
\]
\begin{align*}
  Y(s)&
  =\frac{2s+4}{(s-2)(s^2+2s)}=\frac{2}{s(s-2)}=\frac1{s-2}-\frac1s, \score2\\
  y(t)&
  =\msl^{-1}\Bigl[\frac{1}{s-2}\Bigr]-\msl^{-1}\Bigl[\frac{1}{s}\Bigr]
  =\ee^{2t}-1. \score2
\end{align*}

\textbf{10. (3分) 【解】}
例如
\begin{itemize}
  \item 列出参数方程 $z=z(t)$ 并将积分表达为 $t$ 的积分形式;
  \item 单连通区域内解析函数的积分可以用原函数计算;
  \item 利用柯西-古萨定理;
  \item 利用复合闭路定理;
  \item 利用柯西积分公式;
  \item 利用高阶导数的柯西积分公式;
  \item 利用留数;
  \item 利用长大不等式.
\end{itemize}


\newpage
\BiaoTi{2023年合肥工业大学试卷(A)}
\XueQi{一}
\XueNian{2023}{2024}
\KeChengDaiMa{1400261B}
\KeChengMingCheng{复变函数与积分变换}
\XueFen{2.5}
\KeChengXingZhi{必修}
\KaoShiXingShi{闭卷}
\ZhuanYeBanJi{}
\KaoShiRiQi{2023年12月2日19:00-21:00}
\MingTiJiaoShi{集体}
\maketitle

\tigan{一、填空题(每小题3分,共15分)}
\begin{enumerate}
  \item $2^{-\ii}$ 的辐角主值是\fillblank{}.
  \item $2023-\ii$ 绕 $0$ 逆时针旋转 $\dfrac\cpi2$ 后得到的复数是\fillblank{}.
  \item 如果函数 $f(z)=\dfrac1{(z+5)\sin z}$ 可以在圆环域 $0<|z|<R$ 内作洛朗展开, 则 $R$ 的最大值为\fillblank{}.
  \item 设 $f(z)=\ee^z-|z|\cos z$, 则 $\doint_{|z|=1}f(z)\d z=$\fillblank{}.
  \item 函数 $f(t)=\cos(3t)$ 的傅里叶变换为 $F(\omega)=$\fillblank{}.
\end{enumerate}

\tigan{二、选择题(每小题3分,共15分)}
\begin{enumerate}
  \item 函数 $f(z)=\dfrac 1z$ 在下面哪个区域内有原函数?~(~~~~)
  \xx{$0<|z|<1$}{$\Re z>0$}{$|z-1|>2$}{$|z+1|+|z-1|>4$}
  \item 设 $f(z)=\doint_{|\zeta|=4}\dfrac{\sin\zeta-\cos\zeta}{\zeta-z}\d\zeta$, 则 $f'(\cpi)=$(~~~~)
  \xx{$0$}{$2\cpi\ii$}{$-2\cpi\ii$}{$\cpi\ii$}
  \item 幂级数 $\sumf1 \frac{n!}{n^n}z^n$ 的收敛半径是(~~~~).
  \xx{$0$}{$+\infty$}{$\ee$}{$1$}
  \item 下面哪个函数不能作为解析函数的虚部?~(~~~~)
  \xx{$2x+3y$}{$2x^2+3y^2$}{$x^2-xy-y^2$}{$\ee^x\cos y$}
  \item $z=0$ 是函数 $f(z)=\dfrac{(\ee^z-1)^2z^3}{\sin z^8}$ 的(~~~~).
  \xx{一阶极点}{本性奇点}{可去奇点}{三阶极点}
\end{enumerate}


\tigan{三、解答题}
\begin{enumerate}
  \item \textbf{(6分)} 计算 $(-1+\ii)^{10}-(-1-\ii)^{10}$.
  \item \textbf{(6分)} 解方程 $\cos z=\dfrac{3\sqrt2}4$.
  \item \textbf{(6分)} 设 $C$ 为有向曲线 $z(t)=\sin t+\ii t,0\le t\le \cpi$, 求 $\dint_C z\ee^z \d z$.
  \item \textbf{(10分)} 设 $C$ 为正向圆周 $|z-1|=4$, 求 $\doint_C\frac{\sin z}{z^2+1}\d z$.
  \item \textbf{(10分)} 假设 $u(x,y)=x^3+ax^2y+bxy^2-3y^3$ 是调和函数,求参数 $a,b$ 以及 $v(x,y)$ 使得 $v(0,0)=0$ 且 $f(z)=u+\ii v$ 是解析函数.
  \item \textbf{(10分)} 确定函数 $f(z)=\dfrac{z}{z^2-3z+2}$ 在圆环域\\[6pt]
  \indent (1) $|z|>2$; \hspace{2em} (2) $0<|z-2|<1$\\
  内的洛朗级数展开式.
  \item \textbf{(10分)} 设 $f(z)=\dfrac{\ee^z}{(z-\cpi\ii)(z-2\cpi\ii)^2}$. 求 $f(z)$ 在有限复平面内的奇点以及 $\doint_{|z|=8}f(z)\d z$.
  \item \textbf{(9分)} 用拉普拉斯变换求解微分方程初值问题
  \[\begin{cases}
    y''(t)+4y(t)=3\cos t,\\
    y(0)=1,\quad y'(0)=2.
  \end{cases}\]
  \item \textbf{(3分)} 复变函数 $f(z)=\ee^z$ 和实变量函数 $g(x)=\ee^x$ 的性质有什么相似和不同之处? 试举出三点.
\end{enumerate}


\newpage
\BiaoTi{2023年合肥工业大学考试参考答案(A)}
\XueQi{一}
\XueNian{2023}{2024}
\KeChengDaiMa{1400261B}
\KeChengMingCheng{复变函数与积分变换}
\XueFen{2.5}
\KeChengXingZhi{必修}
\KaoShiXingShi{闭卷}
\ZhuanYeBanJi{}
\KaoShiRiQi{2023年12月2日19:00-21:00}
\MingTiJiaoShi{集体}
\maketitle

\tigan{一、填空题(每小题3分,共15分)}

\textbf{请将你的答案对应填在横线上:}

\textbf{1.} \fillblank[2cm]{$-\ln 2$}, 
\textbf{2.} \fillblank[2.5cm]{$1+2023\ii$}, 
\textbf{3.} \fillblank[1.7cm]{$\cpi$}, 
\textbf{4.} \fillblank[1.7cm]{$0$}, 
\textbf{5.} \fillblank[4.5cm]{$\cpi[\dirac(\omega+3)+\dirac(\omega-3)]$}.

\tigan{二、选择题(每小题3分,共15分)}

\textbf{请将你所选择的字母 A, B, C, D 之一对应填在下列表格里:}

\xuanzeti{\textbf{题号}}{\textbf{答案}}%
\xuanzeti{1}{B}%
\xuanzeti{2}{C}%
\xuanzeti{3}{C}%
\xuanzeti{4}{B}%
\xuanzeti{5}{D}

\tigan{三、解答题}

\textbf{1. (6分) 【解】}
由于
{\large
\[-1+\ii =\sqrt 2 \ee^{\frac{3\cpi\ii}4}, \score2\]}
因此
{\large
\begin{align*}
  (-1+\ii)^{10}&=32\ee^{\frac{30\cpi\ii}4}=-32\ii , \score1\\
  (-1-\ii)^{10}&=32\ee^{-\frac{30\cpi\ii}4}=32\ii , \score1
\end{align*}}
故
{\large
\[(-1+\ii)^{10}-(-1-\ii)^{10}=-64\ii. \score2\]}
也可以直接计算 $(-1+\ii)^2=-2\ii $ 得到 $(-1+\ii)^{10}=-32\ii $.

\textbf{2. (6分) 【解】}
由
\[\cos z=\frac{\ee^{\ii z}+\ee^{-\ii z}}2=\frac{3\sqrt2}4,\score1\]
整理得到
\[\ee^{2\ii z}-\frac{3\sqrt2}2\ee^{\ii z}+1=0,\score2\]
\[\ee^{\ii z}=\frac12\Bigl[\frac{3\sqrt2}2\pm\sqrt{\Bigl(\frac{3\sqrt2}2\Bigr)^2-4}\Bigr]
  =\sqrt2\text{ 或 }\frac{\sqrt2}2,\score2\]
因此
\[iz=\Ln\sqrt2=\frac12\ln2+2k\cpi\ii\text{ 或 }\Ln\frac{\sqrt2}2=-\frac12\ln 2+2k\cpi\ii,\]
\[z=2k\cpi\pm\frac{\ln 2}2\ii , k\in\BZ. \score1\]

\vspace*{10pt}
\textbf{其它解法}: 由 $\cos z=\frac{3\sqrt2}4$ 得
\[\sin z=\sqrt{1-\cos^2z}=\pm\frac{\sqrt2 i}4,\score3\]
因此
\[\ee^{\ii z}=\cos z+\ii \sin z=\sqrt2\text{ 或 }\frac{\sqrt2}2.\score2\]
下同.

\textbf{3. (6分) 【解】}
由于 $z\ee^z$ 解析, 且 \score1
\[\int z\ee^z\d z=(z-1)\ee^z+C, \score2\]
而曲线 $C$ 的起点是 $0$, 终点是 $\cpi\ii$, 因此
\begin{align*}
\int_C z\ee^z\d z&=(z-1)\ee^z\bigg|_0^{\cpi\ii} \score1\\&
=(\cpi\ii-1)\ee^{\cpi\ii}-(-1)
=2-\cpi\ii. \score2
\end{align*}

\textbf{4. (10分) 【解】}
由于 $f(z)=\dfrac{\sin z}{(z+\ii)(z-\ii)}$ 在 $|z-1|\le 4$ 内的奇点为 $\pm i$, 因此\score2
\begin{align*}
\oint_Cf(z)\d z&=2\cpi\ii\bigl[\Res[f(z),-\ii ]+\Res[f(z),-2\ii ]\bigr] \score2\\&
=2\cpi\ii\biggl[\frac{\sin z}{z+\ii}\bigg|_{z=i}+\frac{\sin z}{z-\ii}\bigg|_{z=-\ii}\biggr] \score2\\&
=2\cpi\ii\biggl[\frac{\sin i}{2\ii}+\frac{\sin(-\ii)}{-2\ii}\biggr]\score 2\\&
=2\cpi\sin i=\cpi\ii\bigl(e-\frac1e\bigr). \score2
\end{align*}

\textbf{5. (10分) 【解】}
由
\[\Delta u=u_{xx}+u_{yy}=6x+2ay+2bx-18y=0\]
可知 $a=9,b=-3$. 由\score3
\begin{align*}
  f'(z)&=u_x-\ii u_y \score2\\&
  =(3x^2+18xy-3y^2)-\ii (9x^2-6xy-9y^2) \score2\\&
  =(3-9i)(x+\ii y)^2=(3-9i)z^2 \score1
\end{align*}
可知
\[f(z)=(1-3\ii)z^3+C,\quad C\in\BR. \score1 \]
由 $v(0,0)=0,f(0)=0$ 可知 $C=0$,
\[v=-3x^3+3x^2y+9xy^2-y^3.\score1\]

\vspace*{10pt}
\textbf{其它解法}: 由 $u_x=v_y=3x^2+18xy-3y^2$ 得
\[v=3x^2y+9xy^2-y^3+\psi(x). \score2\]
由 $v_x=-u_y=-(9x^2-6xy-9y^2)$ 得
\[\psi'(x)=-9x^2, \score2\]
\[\psi(x)=-3x^3+C,\quad v=-3x^3+3x^2y+9xy^2-y^3+C. \score2\]
由 $v(0,0)=0$ 可知 $C=0$,
\[v=-3x^3+3x^2y+9xy^2-y^3. \score1\]

\textbf{6. (10分) 【解】}
由于 $f(z)$ 的奇点是 $1,2$, 因此 $f(z)$ 在这两个圆环域内都解析.

(1)
由于
\begin{align*}
  \frac1{z-1}&=\frac1z\cdot\frac1{1-\dfrac1z}=\sumf1 \frac1{z^n}, \score2\\
  \frac1{z-2}&=\frac1z\cdot\frac1{1-\dfrac2z}=\sumf1 \frac{2^{n-1}}{z^n}, \score2
\end{align*}
因此
\[f(z)=\frac2{z-2}-\frac1{z-1}=\sumf1 \frac{2^n-1}{z^n}. \score2\]

(2) 
由于
\[\frac1{z-1}=\frac1{1+(z-2)}=\sumf0 (-1)^n(z-2)^n, \score2\]
因此
\begin{align*}
  f(z)&=\frac2{z-2}-\frac1{z-1}
  =\frac2{z-2}-\sumf0 (-1)^n(z-2)^n. \score2
\end{align*}

\textbf{7. (10分) 【解】}
由于 $\cpi\ii$ 是分母的一阶零点, 因此它是 $f(z)$ 的一阶极点. \score1\\
由于 $2\cpi\ii$ 是分母的二阶零点, 因此它是 $f(z)$ 的二阶极点. \score1
\[\Res[f(z),\cpi\ii]=\frac{\ee^z}{(z-2\cpi\ii)^2}\bigg|_{z=\cpi\ii}=\frac1{\cpi^2}, \score2\]
\begin{align*}
  \Res[f(z),2\cpi\ii]&=\Bigl(\frac{\ee^z}{z-\cpi\ii}\Bigr)'\bigg|_{z=2\cpi\ii} \score2\\&
  =\frac{\ee^z(z-\cpi\ii-1)}{(z-\cpi\ii)^2}\bigg|_{z=2\cpi\ii}=\frac{1-\cpi\ii}{\cpi^2}, \score2
\end{align*}
\[\oint_{|z|=8}f(z)\d z=2\cpi\ii\bigl[\Res[f(z),\cpi\ii]+\Res[f(z),2\cpi\ii]\bigr]=2+\frac{4}\cpi\ii. \score2\]

\textbf{8. (9分) 【解】}
设 $\msl[y]=Y$, 则
\[\msl[y'']=s^2Y-sy(0)-y'(0)=s^2Y-s-2, \score3\]
因此
\[s^2Y-s-2+4Y=3\msl[\cos t]=\frac{3s}{s^2+1}, \score2\]
\begin{align*}
Y(s)&=\frac{s+2}{s^2+4}+\frac{3s}{(s^2+1)(s^2+4)}=\frac{s^3+2s^2+4s+2}{(s^2+1)(s^2+4)}\\&
=\frac{2}{s^2+4}+\frac{s}{s^2+1}, \score2\\
y(t)&=\msl^{-1}\Bigl[\frac{2}{s^2+4}\Bigr]+\msl^{-1}\Bigl[\frac{s}{s^2+1}\Bigr]
=\sin 2t+\cos t. \score2
\end{align*}

\textbf{9. (3分) 【解】}
例如(每项1分)
\begin{itemize}
  \item $f'(z)=\ee^z,g'(x)=\ee^x$. \score1
  \item $\ee^z$ 处处可导, $\ee^x$ 处处可导. \score1
  \item 麦克劳林展开的系数相同. \score1
  \item $\ee^z$ 无界, $\ee^x$ 无界. \score1
  \item $\ee^z$ 是周期的, $\ee^x$ 不是. \score1
\end{itemize}





\newpage
\BiaoTi{2023年合肥工业大学试卷(B)}
\XueQi{一}
\XueNian{2023}{2024}
\KeChengDaiMa{1400261B}
\KeChengMingCheng{复变函数与积分变换}
\XueFen{2.5}
\KeChengXingZhi{必修}
\KaoShiXingShi{闭卷}
\ZhuanYeBanJi{}
\KaoShiRiQi{2024年3月5日 19:00-21:00}
\MingTiJiaoShi{集体}
\maketitle

\tigan{一、填空题(每小题3分,共15分)}
\begin{enumerate}
  \item $-1-\ii$ 的辐角主值是\fillblank{}.
  \item $\dfrac{(1+\ii)^3}{(1-\ii)^3}=$\fillblank{}.
  \item 函数 $f(z)=\dfrac1{(z+5)(z-3)}$ 在 $z_0=0$ 处展开的幂级数的收敛半径是\fillblank[25mm]{}.
  \item 设 $f(z)=\dfrac1{(z+\ii)^{2023}}$, 则 $\doint_{|z|=2}f(z)\d z=$\fillblank{}.
  \item 常值函数 $F(\omega)=2$ 的傅里叶逆变换为 $f(t)=$\fillblank{}.
\end{enumerate}

\tigan{二、选择题(每小题3分,共15分)}
\begin{enumerate}
  \item 区域 $0<\Re z<1$ 是(~~~~)
  \xx{有界单连通区域}{无界单连通区域}{有界多连通区域}{无界多连通区域}
  \item 设 $f(z)=\doint_{|\zeta|=2}\dfrac{\zeta^3+3\zeta}{\zeta-z}\d\zeta$, 则 $f'(\ii)=$(~~~~)
  \xx{$0$}{$3\ii$}{$-3\ii$}{$2\ii $}
  \item 幂级数 $\sumf1 \frac{z^n}{(1+\ii)^n}$ 的收敛半径是(~~~~).
  \xx{$0$}{$+\infty$}{$\sqrt2$}{$\dfrac{\sqrt2}2$}
  \item 下面哪个函数不是调和函数?~(~~~~)
  \xx{$3x-y$}{$x^2-y^2$}{$\ln(x^2+y^2)$}{$\sin x\cos y$}
  \item $z=\cpi$ 是函数 $f(z)=\dfrac{z-\cpi}{(\sin z)^3}$ 的(~~~~).
  \xx{一阶极点}{本性奇点}{可去奇点}{二阶极点}
\end{enumerate}


\tigan{三、解答题}
\begin{enumerate}
  \item \textbf{(6分)} 求 $z=\dfrac{5+\ii}{2+3\ii}$ 的模和辐角.
  \item \textbf{(6分)} 求 $\Ln(1+\sqrt3\ii)$.
  \item \textbf{(6分)} 设 $C$ 为从 $1+\ii$ 到 $1-\ii$ 的直线, 求 $\dint_C (3z^2+1) \d z$.
  \item \textbf{(10分)} 设 $C$ 为正向圆周 $|z+1|=4$, 求 $\doint_C\frac{\sin z+2z}{(z+\cpi)^2}\d z$.
  \item \textbf{(10分)} 假设 $v(x,y)=x^2+4xy+ay^2$ 是调和函数,求参数 $a$ 以及解析函数 $f(z)=u+\ii v$, 使得 $v$ 是 $f(z)$ 的虚部.
  \item \textbf{(10分)} 确定函数 $f(z)=\dfrac{z+1}{(z-1)(z-2)}$ 在圆环域\\
\indent (1) $|z-1|>1$; \hspace{2em} (2) $0<|z-2|<1$\\
内的洛朗级数展开式.
  \item \textbf{(10分)} 设 $f(z)=\dfrac{1}{(z+1)(z+2)^2}$. 求 $f(z)$ 在有限复平面内的奇点以及 $\doint_{|z|=3}f(z)\d z$.
  \item \textbf{(9分)} 用拉普拉斯变换求解微分方程初值问题
\[\begin{cases}
y''(t)-4y(t)=3\ee^t,\\
y(0)=0,\quad y'(0)=1.
\end{cases}\]
  \item \textbf{(3分)} 谈一谈复变函数在一点处连续、可导与解析之间的联系.
\end{enumerate}


\newpage
\BiaoTi{2023年合肥工业大学考试参考答案(B)}
\XueQi{一}
\XueNian{2023}{2024}
\KeChengDaiMa{1400261B}
\KeChengMingCheng{复变函数与积分变换}
\XueFen{2.5}
\KeChengXingZhi{必修}
\KaoShiXingShi{闭卷}
\ZhuanYeBanJi{}
\KaoShiRiQi{2024年3月5日 19:00-21:00}
\MingTiJiaoShi{集体}
\maketitle

\tigan{一、填空题(每小题3分,共15分)}

\textbf{请将你的答案对应填在横线上:}

\textbf{1.} \fillblank[2.5cm]{$-\dfrac34\cpi$}, 
\textbf{2.} \fillblank[2.5cm]{$-\ii$}, 
\textbf{3.} \fillblank[2.5cm]{$3$}, 
\textbf{4.} \fillblank[2.5cm]{$0$}, 
\textbf{5.} \fillblank[2.5cm]{$2\dirac(t)$}.

\tigan{二、选择题(每小题3分,共15分)}

\textbf{请将你所选择的字母 A, B, C, D 之一对应填在下列表格里:}

\xuanzeti{\textbf{题号}}{\textbf{答案}}%
\xuanzeti{1}{B}%
\xuanzeti{2}{A}%
\xuanzeti{3}{C}%
\xuanzeti{4}{D}%
\xuanzeti{5}{D}

\tigan{三、解答题}

\textbf{1. (6分) 【解】}
由于
\[z=\frac{5+\ii}{2+3\ii}=\frac{(5+\ii)(2-3\ii)}{13}=\frac{13-13\ii}{13}=1-\ii ,\score2\]
因此
\[|z|=\sqrt2,\quad \Arg z=2k\cpi-\frac14\cpi,\quad k\in\BZ.\Score{(各2分, 没有$k$减1分)}\]

\textbf{2. (6分) 【解】}
由于
{\large
\[1+\sqrt 3\ii=2\ee^{\frac{\cpi\ii}3},\score2\]
}
因此
\[\Ln(1+\sqrt3\ii)=\ln 2+\bigl(2k\cpi+\frac{\cpi}3\bigr)\ii, k\in\BZ.\Score{(实部虚部各2分, 没有$k$减1分)}\]

\textbf{3. (6分) 【解】}
由于 $3z^2+1$ 解析, 且 \score1
\[\int (3z^2+1)\d z=z^3+z+C, \score2\]
因此
\begin{align*}
  \int_C (3z^2+1) \d z&=(z^3+z)\big|_{1+\ii}^{1-\ii} \score1\\&
  =(1-\ii)^3-(1+\ii)^3+(1-\ii)-(1+\ii)\\&
  =-6i. \score2
\end{align*}

\textbf{4. (10分) 【解】}
由于 $f(z)=\dfrac{\sin z+2z}{(z+\cpi)^2}$ 在 $|z+1|\le 4$ 内的奇点为 $-\cpi$, \score2\\
因此
\begin{align*}
  \oint_Cf(z)\d z&=2\cpi\ii\Res[f(z),-\cpi]\score2\\&
  =2\cpi\ii(\sin z+2z)'\bigg|_{z=-\cpi} \score2\\&
  =2\cpi\ii[\cos(-\cpi)+2] \score2\\&
  =2\cpi\ii. \score2
\end{align*}

\textbf{5. (10分) 【解】}
由
\[\Delta v=v_{xx}+v_{yy}=2+2a=0\score2\]
可知 $a=-1$. \score1\\
由
\begin{align*}
  f'(z)&=v_y+\ii v_x \score2\\&
  =(4x-2y)+\ii (2x+4y) \score2\\&
  =(4+2\ii)(x+\ii y)=(4+2\ii)z \score1
\end{align*}
可知
\[f(z)=(2+\ii)z^2+C,\quad C\in\BR. \Score{(2分, 没有 $C$ 减一分)} \]

\vspace*{10pt}
\textbf{其它解法}: 由 $u_x=v_y=4x-2y$ 得
\[u=2x^2-2xy+\psi(y). \score2\]
由 $u_y=-v_x=-(2x+4y)$ 得
\[\psi'(y)=-4y, \score2\]
\[\psi(y)=-2y^2+C,\quad u=2x^2-2xy-2y^2+C. \score2\]
\[f(z)=2x^2-2xy-2y^2+C+\ii (x^2+4xy-y^2)=(2+\ii)z^2+C. \score1\]

\newpage
\textbf{6. (10分) 【解】}
由于 $f(z)$ 的奇点是 $1,2$, 因此 $f(z)$ 在这两个圆环域内都解析.

(1)
由于
\[\frac1{z-2}=\frac1{(z-1)-1}=\frac1{z-1}\cdot\frac1{1-\dfrac1{z-1}}=\sumf1 \frac1{(z-1)^n}, \score3\]
因此
\[f(z)=\frac3{z-2}-\frac2{z-1}=-\frac2{z-1}+\sumf1 \frac3{(z-1)^n}. \score2\]

(2) 
由于
\[\frac1{z-1}=\frac1{1+(z-2)}=\sumf0 (-1)^n(z-2)^n, \score3\]
因此
\[f(z)=\frac3{z-2}-\frac2{z-1}=\frac3{z-2}-\sumf0 2(-1)^n(z-2)^n. \score2\]

\textbf{7. (10分) 【解】}
由于 $-1$ 是分母的一阶零点, 因此它们是 $f(z)$ 的一阶极点. \score1\\
由于 $-2$ 是分母的二阶零点, 因此它是 $f(z)$ 的二阶极点. \score1
\[\Res[f(z),-1]=\frac{1}{(z+2)^2}\bigg|_{z=-1}=1, \score2\]
\begin{align*}
  \Res[f(z),-2]&=\Bigl(\frac{1}{z+1}\Bigr)'\bigg|_{z=-2} \score2\\&
  =-\frac{1}{(z+1)^2}\bigg|_{z=-2}=-1, \score2
\end{align*}
\[\oint_{|z|=3}f(z)\d z=2\cpi\ii\bigl[\Res[f(z),\cpi\ii]+\Res[f(z),2\cpi\ii]\bigr]=0. \score2\]

\textbf{8. (9分) 【解】}
设 $\msl[y]=Y$, 则
\[\msl[y'']=s^2Y-sy(0)-y'(0)=s^2Y-1, \score3\]
因此
\[s^2Y-1-4Y=3\msl[\ee^t]=\frac{3}{s-1}, \score2\]
\begin{align*}
Y(s)&=\frac{1}{s^2-4}+\frac{3}{(s-1)(s^2-4)}=\frac{s+2}{(s-1)(s^2-4)}\\&
=\frac{1}{(s-1)(s-2)}=\frac1{s-2}-\frac1{s-1}, \score2\\
y(t)&=\msl^{-1}\Bigl[\frac{1}{s-2}\Bigr]-\msl^{-1}\Bigl[\frac{1}{s-1}\Bigr]
=\ee^{2t}-\ee^t. \score2
\end{align*}

\textbf{9. (3分) 【解】}
言之成理即可。
\begin{itemize}
  \item 解析蕴含可导,但反过来不对。\score1
  \item $f(z)$ 需要在 $z_0$ 的一个邻域内都可导才解析。\score1
  \item 可导蕴含连续,但反过来不对。\score1
\end{itemize}


\newpage
\BiaoTi{2024年合肥工业大学试卷(A)}
\XueQi{一}
\XueNian{2024}{2025}
\KeChengDaiMa{1400261B}
\KeChengMingCheng{复变函数与积分变换}
\XueFen{2.5}
\KeChengXingZhi{必修}
\KaoShiXingShi{闭卷}
\ZhuanYeBanJi{}
\KaoShiRiQi{2024年12月1日 19:00-21:00}
\MingTiJiaoShi{集体}
\maketitle

\tigan{一、填空题(每小题3分,共15分)}
\begin{enumerate}
  \item 设 $\omega=\dfrac{-1-\sqrt3\ii}2$, 则 $\omega+\omega^2=$\fillblank{}.
  \item 对数函数主值 $\ln(-\ii)=$\fillblank{}.
  \item 设 $C$ 为正向圆周 $|z-1|=1$, 则积分 $\doint_{C}\ov z\d z=$\fillblank{}.
  \item 函数 $f(z)=\tan z$ 在 $z=\dfrac\cpi2$ 处的留数等于\fillblank{}.
  \item 常值函数 $f(t)=-2$ 的傅里叶变换为 $F(\omega)=$\fillblank{}.
\end{enumerate}

\tigan{二、选择题(每小题3分,共15分)}
\begin{enumerate}
  \item 集合 $|z-1|\ge|z-\ii |$ 是(~~~~).
  \xx{有界单连通区域}{无界单连通闭区域}{有界多连通区域}{无界多连通闭区域}
  \item 下面哪个数不是纯虚数?(~~~~)
  \xx{$\ln(-1)$}{$\cos i$}{$\sin i$}{$\sqrt{-\cpi}$ 主值}
  \item 设有向曲线 $C:z(t)=\sin{2t}+2\ii \cos{t},t\in[0,\cpi]$, 则积分 $\dint_C z\d z$ 等于(~~~~)
  \xx{$0$}{$-2\ii $}{$-4\ii$}{$4\ii$}
  \item 函数 $f(z)=\dfrac{z-1}{z^2-z-2}$ 不能在(~~~~)内作洛朗展开.
  \xx{$0<|z|<2$}{$2<|z|<4$}{$0<|z+1|<2$}{$1<|z+1|<3$}
  \item $z=0$ 是函数 $f(z)=\dfrac{z\sin z^3}{\ln(1-z^4)}$ 的(~~~~).
  \xx{一阶极点}{本性奇点}{可去奇点}{四阶极点}
\end{enumerate}


\tigan{三、解答题}
\begin{enumerate}
  \item \textbf{(6分)} 设 $z=\sqrt2(1-\ii)$. 计算 $z^5$.
  \item \textbf{(6分)} 解方程 $\sin z=\dfrac{2\sqrt3}3$.
  \item \textbf{(6分)} 设 $C$ 为从 $1$ 到 $1+\ii$ 再到 $\ii$ 的折线段, 求 $\dint_C (\Re z+\Im z) \d z$.
  \item \textbf{(10分)} 设 $C$ 为正向圆周 $|z|=2$, 求 $\doint_C\frac{\cos z}{z^2(z+\ii)}\d z$.
  \item \textbf{(10分)} 假设 $v(x,y)=x^2+xy+ay^2-2y$ 是调和函数,求参数 $a$ 以及 $u(x,y)$ 使得 $f(z)=u+\ii v$ 是解析函数且满足 $f(0)=0$.
  \item \textbf{(10分)} 确定函数 $f(z)=\dfrac{z-1}{z^2+3z+2}$ 在圆环域\par
(1) $0<|z|<1$; \hspace{2em} (2) $|z+1|>1$\par
内的洛朗级数展开式.
  \item \textbf{(10分)} 设 $f(z)=\dfrac{z^2-\cpi^2}{z\sin z}$. 求 $f(z)$ 在有限复平面内的奇点和类型, 求出极点的阶, 并计算 $\doint_{C}f(z)\d z$, 其中 $C$ 为正向圆周 $|z-6|=4$.
  \item \textbf{(9分)} 用拉普拉斯变换求解微分方程初值问题
\[\begin{cases}
y''(t)-4y(t)=\ee^{3t},\\
y(0)=1,\quad y'(0)=-1.
\end{cases}\]
  \item \textbf{(3分)} 复变函数 $f(z)=\sin z$ 和实变量函数 $g(x)=\sin x$ 的性质有什么相似和不同之处? 试举出三点.
\end{enumerate}


\newpage
\BiaoTi{2024年合肥工业大学考试参考答案(A)}
\XueQi{一}
\XueNian{2024}{2025}
\KeChengDaiMa{1400261B}
\KeChengMingCheng{复变函数与积分变换}
\XueFen{2.5}
\KeChengXingZhi{必修}
\KaoShiXingShi{闭卷}
\ZhuanYeBanJi{}
\KaoShiRiQi{2024年12月1日 19:00-21:00}
\MingTiJiaoShi{集体}
\maketitle

\tigan{一、填空题(每小题3分,共15分)}

\textbf{请将你的答案对应填在横线上:}

\textbf{1.} \fillblank[2cm]{$-1$}, 
\textbf{2.} \fillblank[2.5cm]{$-\frac{\cpi\ii}2$}, 
\textbf{3.} \fillblank[1.7cm]{$2\cpi\ii$}, 
\textbf{4.} \fillblank[1.7cm]{$-1$}, 
\textbf{5.} \fillblank[4.5cm]{$-4\cpi\dirac(\omega)$}.

\tigan{二、选择题(每小题3分,共15分)}

\textbf{请将你所选择的字母 A, B, C, D 之一对应填在下列表格里:}

\xuanzeti{\textbf{题号}}{\textbf{答案}}%
\xuanzeti{1}{B}%
\xuanzeti{2}{B}%
\xuanzeti{3}{A}%
\xuanzeti{4}{A}%
\xuanzeti{5}{C}

\tigan{三、解答题}

\textbf{1. (6分) 【解】}
由于
\[z=2\ee^{-\frac{\cpi\ii}4}.\score2\]
因此
\[z^5=2^5\ee^{-\frac{5\cpi\ii}4}=-16\sqrt2+16\sqrt 2 i.\score4\]
也可直接计算得到.

\textbf{2. (6分) 【解】}
由
\[\sin z=\frac{\ee^{\ii z}-\ee^{-\ii z}}{2\ii}=\frac{2\sqrt3}3,\score1\]
整理得到
\[\ee^{2\ii z}-\frac{4\sqrt3}3\ii\ee^{\ii z}-1=0,\score2\]
\[\ee^{\ii z}=\frac12\biggl(\frac{4\sqrt3}3\ii\pm\sqrt{\Bigl(\frac{4\sqrt3}3\ii\Bigr)^2+4}\biggr)
  =\sqrt3\ii\text{ 或 }\frac{\sqrt3}3\ii,\score2\]
因此
\[iz=\Ln(\sqrt3\ii)=\frac12\ln3+\frac{\cpi\ii}2+2k\cpi\ii\text{ 或 }\Ln\frac{\sqrt3}3=-\frac12\ln 3+\frac{\cpi\ii}2+2k\cpi\ii,\]
\[z=2k\cpi+\frac{\cpi }2\pm\frac{\ln 3}2\ii , k\in\BZ. \score1\]

另解: 由于
\[\cos z=\sqrt{1-\sin^2z}=\pm\frac{\sqrt3}3\ii,\score2\]
因此
\[\ee^{\ii z}=\cos z+\ii \sin z=\sqrt3\ii\text{\ 或\ }\frac{\sqrt3}3\ii.\score3\]
其余相同.

\textbf{3. (6分) 【解】}
分段计算.\score1\\
第一段 $C_1:z=1+\ii t,t\in[0,1]$, $\d z=i\d t$, $\Re z+\Im z=1+t$, 因此\score1
\[\int_{C_1}(\Re z+\Im z)\d z=\int_0^1 (1+t)\ii\d t=\frac{3}2\ii .\score1\]
第二段 $C_2:z=1+\ii-t,t\in[0,1]$, $\d z=-\d t$, $\Re z+\Im z=2-t$, 因此\score1
\[\int_{C_2}(\Re z+\Im z)\d z=\int_0^1 \bigl(-(2-t)\bigr)\d t=-\frac32.\score1\]
因此
\[\int_C (\Re z+\Im z)\d z=-\frac32+\frac32 i.\score1\]

\textbf{4. (10分) 【解】}
$f(z)=\dfrac{\cos z}{z^2(z+\ii)}$ 在 $|z|\le 2$ 内的奇点为 $0,-\ii$.\score2
\[\Res[f,0]=\Bigl(\frac{\cos z}{z+\ii}\Bigr)'\Big|_{z=0}
=\frac{-(z+\ii)\sin z-\cos z}{(z+\ii)^2}\Big|_{z=0}=1,\score2\]
\[\Res[f,-\ii ]=\frac{\cos z}{z^2}\Big|_{z=-\ii}=-\cos i.\score2\]
故
\begin{align*}
\oint_Cf(z)\d z&=2\cpi\ii\bigl[\Res[f(z),0]+\Res[f(z),-\ii ]\bigr] \score2\\&
=2\cpi\ii(1-\cos i)\score1\\&
=(2-e-\frac1e)\cpi\ii.\score 1
\end{align*}

\textbf{5. (10分) 【解】}
由
\[\Delta v=v_{xx}+v_{yy}=2+2a=0\]
可知 $a=-1$. 由\score3
\begin{align*}
  f'(z)&=v_y+\ii v_x \score2\\&
  =(x-2y-2)+\ii (2x+y) \score2\\&
  =(1+2\ii)(x+y\ii)-2=(1+2\ii)z-2 \score1
\end{align*}
可知
\[f(z)=\frac{1+2\ii}2z^2-2z+C,\quad C\in\BR. \score1 \]
由 $f(0)=0$ 可知 $C=0$,
\[u=\frac{x^2-y^2}2-2xy-2x.\score1\]

\vspace*{10pt}
\textbf{其它解法}: 由 $u_x=v_y=x-2y-2$ 得
\[u=\frac12x^2-2xy-2x+\psi(y). \score2\]
由 $u_y=-v_x=-(2x+y)$ 得
\[\psi'(x)=-y, \score2\]
\[\psi(x)=-\frac12y^2+C,\quad u=\frac12x^2-2xy-2x-\frac12y^2+C. \score2\]
由 $f(0)=0$ 可知 $C=0$,
\[u=\frac12x^2-2xy-2x-\frac12y^2. \score1\]

\textbf{6. (10分) 【解】}
由于 $f(z)$ 的奇点是 $-1,-2$, 因此 $f(z)$ 在这两个圆环域内都解析.

(1)
由于
\begin{align*}
  \frac1{z+1}&=\sumf0 (-1)^nz^n, \score2\\
  \frac1{z+2}&=\frac12\cdot\frac1{1+\dfrac z2}=\sumf0 \frac{(-1)^n}{2^{n+1}}z^n, \score2
\end{align*}
因此
\[f(z)=\frac3{z+2}-\frac2{z+1}=\sumf0 (-1)^n\Bigl(\frac3{2^{n+1}}-2\Bigr){z^n}. \score2\]

(2) 
由于
\[\frac1{z+2}=\frac1{z+1}\cdot\frac1{1+\dfrac 1{z+1}}=\sumf1 \frac{(-1)^{n+1}}{(z+1)^n}, \score2\]
因此
\[f(z)=\frac3{z+2}-\frac2{z+1}
=3\sumf1 \frac{(-1)^{n+1}}{(z+1)^n}-\frac2{z+1}=\frac1{z+1}+3\sum_{n=2}^\infty \frac{(-1)^{n+1}}{(z+1)^n}. \score2\]

\textbf{7. (10分) 【解】}
由于 $0$ 是分母的二阶零点, 因此它是 $f(z)$ 的二阶极点. \score1\\
由于 $\pm\cpi$ 是分子分母的一阶零点, 因此它是 $f(z)$ 的可去极点. \score1\\
对于整数 $k\neq 0,\pm1$, $k\cpi$ 是分母的一阶零点, 因此它是 $f(z)$ 的一阶极点. \score1
\[\Res[f(z),\cpi]=0,\score1\]
\[\Res[f(z),2\cpi]=\frac{z^2-\cpi^2}{\sin z+z\cos z}\bigg|_{z=2\cpi}=\frac32\cpi, \score2\]
\[\Res[f(z),3\cpi]=\frac{z^2-\cpi^2}{\sin z+z\cos z}\bigg|_{z=3\cpi}=-\frac83\cpi, \score2\]
\[\oint_{|z-6|=4}f(z)\d z=2\cpi\ii\Bigl[\Res[f(z),\cpi]+\Res[f(z),2\cpi]+\Res[f(z),3\cpi]\Bigr]=-\frac73\cpi^2 i. \score2\]

\textbf{8. (9分) 【解】}
设 $\msl[y]=Y$, 则
\[\msl[y'']=s^2Y-sy(0)-y'(0)=s^2Y-s+1, \score3\]
因此
\[s^2Y-s+1-4Y=\msl[\ee^{3t}]=\frac{1}{s-3}, \score2\]
\begin{align*}
Y(s)&=\frac{1}{s^2-4}\Bigl(s-1+\frac{1}{s-3}\Bigr)\\&
=\frac{s-2}{(s+2)(s-3)}=\frac15\Bigl(\frac1{s-3}+\frac4{s+2}\Bigr), \score2\\
y(t)&=\msl^{-1}\Bigl[\frac15\Bigl(\frac1{s-3}+\frac4{s+2}\Bigr)\Bigr]
=\frac15\ee^{3t}+\frac45\ee^{-2t}. \score2
\end{align*}

\textbf{9. (3分) 【解】}
每项1分, 例如
\begin{itemize}
  \item 导数形式相同;
  \item 一个有界一个无界;
  \item 都是奇函数;
  \item 麦克劳林展开的系数相同;
  \item 都是处处可导;
  \item 都是周期函数.
\end{itemize}


\newpage
\BiaoTi{2024年合肥工业大学试卷(B)}
\XueQi{一}
\XueNian{2024}{2025}
\KeChengDaiMa{1400261B}
\KeChengMingCheng{复变函数与积分变换}
\XueFen{2.5}
\KeChengXingZhi{必修}
\KaoShiXingShi{闭卷}
\ZhuanYeBanJi{}
\KaoShiRiQi{补考时间}
\MingTiJiaoShi{集体}
% \XiZhuRenQianMing{tiankelei.png}
\maketitle

\tigan{一、填空题(每小题3分,共15分)}
\begin{enumerate}
	\item 设 $z=1+\ii$, 则 $1+z+z^2=$\fillblank{}.
	\item 复数 $1-\ii$ 的辐角主值 $\arg(1-\ii)=$\fillblank{}.
	\item 函数 $f(z)=\dfrac{z}{\sin z}$ 在 $|z|<4$ 内有\fillblank{}个奇点.
	\item 积分 $\doint_{|z|=1}(u+\ii v)\d z=$\fillblank{}, 其中 $v(x,y)$ 是 $u(x,y)$ 在整个复平面内的共轭调和函数.
	\item 函数 $f(t)=\dirac(t-1)$ 的傅里叶变换为 $F(\omega)=$\fillblank{}.
\end{enumerate}

\tigan{二、选择题(每小题3分,共15分)}
\begin{enumerate}
\item 集合 $0<|z|<1$ 是(~~~~).
\xx{有界单连通区域}{无界单连通闭区域}{有界多连通区域}{无界多连通闭区域}
\item 若 $z$ 是实数, 下面哪个选项未必成立?(~~~~)
\xx{$z=\overline z$}{$\Im z=0$}{$\arg z=0$}{$z$ 在实轴上}
\item 若 $z_1,z_2$ 是非零复数, 下面哪个等式未必成立?(~~~~)
\xx{$\Arg(z_1z_2)=\Arg z_1+\Arg z_2$}{$\Ln(z_1z_2)=\Ln z_1+\Ln z_2$}{$|z_1z_2|=|z_1|\cdot|z_2|$}{$\arg(z_1z_2)=\arg z_1+\arg z_2$}
\item 下面哪个函数在 $z=0$ 处解析?(~~~~)
\xx{$x^2+y^2\ii$}{$x-y\ii$}{$\dfrac1z$}{$y-x\ii$}
\item 函数 $f(z)=\dfrac{z+1}{z^2+z-2}$ 不能在(~~~~)内作泰勒展开.
\xx{$|z|<1$}{$|z+1|<1$}{$|z-1|<1$}{$|z+\ii|<1$}
\end{enumerate}


\tigan{三、解答题}
\begin{enumerate}
\item \textbf{(6分)} 设 $z=1+\sqrt3\ii$. 计算 $z^5$.
\item \textbf{(6分)} 解方程 $\cos z=\dfrac54$.
\item \textbf{(6分)} 设 $C$ 为从 $1$ 到 $1+\ii$ 再到 $\ii$ 的折线段, 求 $\dint_C (\Re z+\ii\Im z) \d z$.
\item \textbf{(10分)} 设 $C$ 为正向圆周 $|z|=4$, 求 $\doint_C\frac{\ee^z}{z(z-\cpi\ii)^2}\d z$.
\item \textbf{(10分)} 假设 $v(x,y)=x^2+ay^2+x+y$ 是调和函数,求参数 $a$ 以及 $u(x,y)$ 使得 $f(z)=u+iv$ 是解析函数且满足 $f(0)=0$.
\item \textbf{(10分)} 确定函数 $f(z)=\dfrac{z^2}{z^2-3z+2}$ 在圆环域\par
(1) $0<|z|<1$; \hspace{2em} (2) $|z|>2$\par
内的洛朗级数展开式.
\item \textbf{(10分)} 设 $f(z)=\dfrac{z+\cos z}{z(z-\cpi)(z+\cpi)}$. 求 $f(z)$ 在有限复平面内的奇点和类型, 求出极点的阶, 并计算 $\doint_{C}f(z)\d z$, 其中 $C$ 为正向圆周 $|z|=4$.
\item \textbf{(9分)} 用拉普拉斯变换求解微分方程初值问题
\[\begin{cases}
y''(t)+4y(t)=3\cos t,\\
y(0)=2,\quad y'(0)=0.
\end{cases}\]
\item \textbf{(3分)} 留数有哪些应用? 试举出三点.
\end{enumerate}


\newpage
\BiaoTi{2024年合肥工业大学考试参考答案(B)}
\XueQi{一}
\XueNian{2024}{2025}
\KeChengDaiMa{1400261B}
\KeChengMingCheng{复变函数与积分变换}
\XueFen{2.5}
\KeChengXingZhi{必修}
\KaoShiXingShi{闭卷}
\ZhuanYeBanJi{}
\KaoShiRiQi{补考时间}
\MingTiJiaoShi{集体}
\maketitle

\tigan{一、填空题(每小题3分,共15分)}


\textbf{请将你的答案对应填在横线上:}

\textbf{1.} \fillblank[2cm]{$2+3\ii$}, 
\textbf{2.} \fillblank[2cm]{$-\dfrac{\cpi}2$}, 
\textbf{3.} \fillblank[1.7cm]{$3$}, 
\textbf{4.} \fillblank[1.7cm]{$0$}, 
\textbf{5.} \fillblank[1.7cm]{$\ee^{-\jj \omega}$}, 写成 $\ee^{-\ii\omega}$ 也可.

\tigan{二、选择题(每小题3分,共15分)}

\textbf{请将你所选择的字母 A, B, C, D 之一对应填在下列表格里:}

\xuanzeti{\textbf{题号}}{\textbf{答案}}%
\xuanzeti{1}{C}%
\xuanzeti{2}{C}%
\xuanzeti{3}{D}%
\xuanzeti{4}{D}%
\xuanzeti{5}{C}

\tigan{三、解答题}

\textbf{1. (6分) 【解】}
由于
\[z=2\ee^{\frac{\cpi \ii}3}.\score2\]
因此
\[z^5=2^5\ee^{\frac{5\cpi \ii}3}=16-16\sqrt 3 \ii.\score4\]
也可直接计算得到.

\textbf{2. (6分) 【解】}
由
\[\cos z=\frac{\ee^{\ii z}+\ee^{-\ii z}}{2}=\frac54,\score1\]
整理得到
\[\ee^{2\ii z}-\frac52\ee^{\ii z}+1=0,\score2\]
\[\ee^{\ii z}=2\text{ 或 }\frac12,\score2\]
因此
\[\ii z=\Ln2=2k\cpi\ii+\ln2\text{ 或 }\Ln\frac12=2k\cpi\ii-\ln2,\]
\[z=2k\cpi\pm \ii\ln 2, k\in\BZ. \score1\]

另解: 由于
\[\sin z=\sqrt{1-\cos^2z}=\pm\frac34\ii,\score2\]
因此
\[\ee^{\ii z}=\cos z+\ii\sin z=2\ \text{或}\ \frac12.\score3\]
其余相同.

\textbf{3. (6分) 【解】}
由于 $f(z)=\Re z+\ii\Im z=z$ 处处解析, 因此\score2
\begin{align*}
  \int_C (\Re z+\ii\Im z)\d z&
  =\int_C z\d z=\frac{z^2}2\Big|_1^\ii\score2\\&
  =-1.\score2
\end{align*}
也可以写出两段参数方程分别计算再相加.

\textbf{4. (10分) 【解】}
$f(z)=\dfrac{\ee^z}{z(z-\cpi\ii)^2}$ 在 $|z|\le 4$ 内的奇点为 $0,\cpi\ii$.\score2
\[\Res[f,0]=\frac{\ee^z}{(z-\cpi\ii)^2}\Big|_{z=0}=-\frac1{\cpi^2}.\score2\]
\begin{align*}
  \Res[f,\cpi\ii]&
  =\Bigl(\frac{\ee^z}{z}\Bigr)'\Big|_{z=\cpi\ii}\score1\\&
  =\frac{\ee^z(z-1)}{z^2}\Big|_{z=\cpi\ii}\score1\\&
  =\frac{\cpi\ii-1}{\cpi^2},\score1
\end{align*}
故
\begin{align*}
\oint_Cf(z)\d z&=2\cpi\ii\bigl[\Res[f(z),0]+\Res[f(z),-\cpi\ii]\bigr] \score2\\
&=-2-\frac4\cpi\ii.\score 1
\end{align*}

\textbf{5. (10分) 【解】}
由
\[\Delta v=v_{xx}+v_{yy}=2+2a=0\]
可知 $a=-1$. 由\score3
\begin{align*}
  f'(z)&=v_y+\ii v_x \score2\\
  &=(-2y+1)+\ii(2x+1) \score2\\
  &=2\ii(x+yi)+1+\ii=2\ii z+1+\ii \score1
\end{align*}
可知
\[f(z)=\ii z^2+(1+\ii)z+C, C\in\BR. \score1 \]
由 $f(0)=0$ 可知 $C=0$,
\[u=-2xy+x-y.\score1\]

\vspace{10pt}
\textbf{其它解法}: 由 $u_x=v_y=-2y+1$ 得
\[u=-2xy+x+\psi(y). \score2\]
由 $u_y=-v_x=-(2x+1)$ 得
\[\psi'(y)=-1, \score2\]
\[\psi(y)=-y+C,\quad u=-2xy+x-y+C. \score2\]
由 $f(0)=0$ 可知 $C=0$,
\[u=-2xy+x-y. \score1\]

\textbf{6. (10分) 【解】}
由于 $f(z)$ 的奇点是 $1,2$, 因此 $f(z)$ 在这两个圆环域内都解析.

(1)
由于
\begin{align*}
  \frac1{z-1}&=-\sum_{n=0}^\infty z^n, \score2\\
  \frac1{z-2}&=-\frac12\cdot\frac1{1-\dfrac z2}=-\sum_{n=0}^\infty \frac1{2^{n+1}}z^n, \score2
\end{align*}
因此
\[f(z)=1+\frac4{z-2}-\frac1{z-1}
=\sum_{n=2}^\infty \Bigl(1-\frac1{2^{n-1}}\Bigr){z^n}. \score1\]
$n$ 从 $1$ 开始也对.

(2) 
由于
\[\frac1{z-1}=\frac1z\cdot\frac1{1-\dfrac 1z}
=\sum_{n=1}^\infty \frac1{z^n}, \score2\]
\[\frac1{z-2}=\frac1z\cdot\frac1{1-\dfrac 2z}
=\sum_{n=1}^\infty \frac{2^{n-1}}{z^n}, \score2\]
因此
\[f(z)=1+\frac4{z-2}-\frac1{z-1}
=\sum_{n=0}^\infty \frac{2^{n+1}-1}{z^n}. \score1\]

\textbf{7. (10分) 【解】}
$0,\cpi,-\cpi$ 是 $f(z)$ 的一阶极点. \score2
\[\Res[f(z),0]=\frac{z+\cos z}{(z-\cpi)(z+\cpi)}\Big|_{z=0}=-\frac1{\cpi^2},\score2\]
\[\Res[f(z),\cpi]=\frac{z+\cos z}{z(z+\cpi)}\Big|_{z=\cpi}=\frac{\cpi-1}{2\cpi^2},\score2\]
\[\Res[f(z),-\cpi]=\frac{z+\cos z}{z(z-\cpi)}\Big|_{z=-\cpi}=\frac{-\cpi-1}{2\cpi^2},\score2\]
\[\oint_{|z|=4}f(z)\d z=2\cpi\ii\Bigl[\Res[f(z),0]+\Res[f(z),\cpi]+\Res[f(z),-\cpi]\Bigr]
=-\frac{4\ii}\cpi. \score2\]

\textbf{8. (9分) 【解】}
设 $\msl[y]=Y$, 则
\[\msl[y'']=s^2Y-sy(0)-y'(0)=s^2Y-2s, \score3\]
因此
\[s^2Y-2s+4Y=3\msl[\cos t]=\frac{3s}{s^2+1}, \score2\]
\begin{align*}
Y(s)&
=\frac{2s^3+5s}{(s^2+1)(s^2+4)}\\&
=\frac s{s^2+1}+\frac s{s^2+4}, \score2\\
y(t)&
=\msl^{-1}\left[\frac s{s^2+1}+\frac s{s^2+4}\right]
=\cos t+\cos 2t. \score2
\end{align*}

\textbf{9. (3分) 【解】}
每项1分, 例如
\begin{itemize}
  \item 用于计算复变函数积分;
  \item 用于计算广义函数积分;
  \item 用于计算三角函数的有理函数在 $0$ 到 $2\cpi$ 的积分;
  \item 用于计算级数;
  \item 用于计算拉普拉斯逆变换.
\end{itemize}
\end{document}
