\documentclass[simple]{hfutexam}
\usepackage{mathrsfs}
\usepackage{bookmark}
\DeclareMathOperator{\Res}{Res}
\DeclareMathOperator{\Ln}{Ln}
\DeclareMathOperator{\Arg}{Arg}
\newcommand\msl{\mathscr{L}}
\newcommand\BR{\mathbb{R}}
\newcommand\BZ{\mathbb{Z}}
\newcommand{\diff}{\,\mathrm{d}}
\let\Im\relax\DeclareMathOperator{\Im}{Im}
\let\Re\relax\DeclareMathOperator{\Re}{Re}

\begin{document}
\BiaoTi{2022年合肥工业大学试卷(A)}
\XueQi{一}
\XueNian{2022}{2023}
\KeChengDaiMa{1400261B}
\KeChengMingCheng{复变函数与积分变换}
\XueFen{2.5}
\KeChengXingZhi{必修}
\KaoShiXingShi{闭卷}
\ZhuanYeBanJi{}
\KaoShiRiQi{2022年11月26日19:00-21:00}
\MingTiJiaoShi{集体}
\maketitle

\tigan{一、填空题(每小题3分,共15分)}
\begin{enumerate}
\item $i^{-i}$ 的主值是\fillblank{}.
\item 设 $z=-i$, 则 $1+z+z^2+z^3+z^4=$\fillblank{}.
\item 设 $C$ 为正向圆周 $|z|=2$, 则积分 $\displaystyle\oint_C\left(\frac{\overline z}{z}\right)\diff z=$\fillblank{}.
\item 设 $a,b,c$ 为实数. 如果函数 $f(z)=x^2-2xy-y^2+i(ax^2+bxy+cy^2)$ 在复平面上处处解析, 则 $a+b+c=$\fillblank{}.
\item 函数 $\sin t+j\cos t$ 的傅里叶变换为\fillblank{}.
\end{enumerate}

\tigan{二、选择题(每小题3分,共15分)}
\begin{enumerate}
\item 方程 $\bigl||z+i|-|z-i|\bigr|=1$ 表示的曲线是(~~~~).
\xx{直线}{不是圆的椭圆}{双曲线}{圆周}
\item 不等式 $-1\le\arg z\le \pi-1$ 确定是的(~~~~).
\xx{有界多连通闭区域}{有界单连通区域}{无界多连通区域}{无界单连通闭区域}
\item 幂级数 $\displaystyle\sum_{n=1}^\infty (iz)^n$ 的收敛半径是(~~~~).
\xx{$i$}{$-i$}{$1$}{$+\infty$}
\item 下面哪个函数在 $z=0$ 处不可导?~(~~~~)
\xx{$2x+3yi$}{$2x^2+3y^2i$}{$x^2-xyi$}{$e^x\cos y+i e^x\sin y$}
\item 如果 $z_0$ 是 $f(z)$ 的一阶极点, $g(z)$ 的一阶零点, 则 $z_0$ 是 $f(z)^3g(z)^2$ 的(~~~~).
\xx{一阶极点}{一阶零点}{可去奇点}{三阶极点}
\end{enumerate}


\tigan{三、解答题}
\begin{enumerate}
\item \textbf{(6分)} 设 $z=\dfrac{3+i}{i}-\dfrac{10i}{3-i}$, 求 $z$ 的模和辐角.
\item \textbf{(6分)} 解方程 $\sin z=2\cos z$.
\item \textbf{(6分)} 设 $C$ 为从 $i$ 到 $i-\pi$ 再到 $-\pi$ 的折线, 求 $\displaystyle\int_C\cos^2z\diff z$.
\item \textbf{(10分)} 设 $C$ 为正向圆周 $|z-3|=4$, 求 $\displaystyle\oint_C\frac{e^{iz}}{z^2-3\pi z+2\pi^2}\diff z$.
\item \textbf{(10分)} 假设 $v(x,y)=x^3+y^3-axy(x+y)$ 是调和函数,求参数 $a$ 以及解析函数 $f(z)$ 使得 $v(x,y)$ 是它的虚部.
\item \textbf{(10分)} 确定函数 $f(z)=\dfrac{z+1}{(z-1)^2}$ 在圆环域\\
\indent (1) $0<|z|<1$; \hspace{2em} (2) $1<|z|<+\infty$\\
内的洛朗级数展开式.
\item \textbf{(10分)} 求 $f(z)=\dfrac{\cos z}{z^2(z^2-\pi^2)}$ 在有限复平面内的奇点和相应的留数.
\item \textbf{(9分)} 用拉普拉斯变换求解微分方程初值问题
\[\begin{cases}
y''(t)+2y(t)=\sin t,&\\
y(0)=0,\quad y'(0)=2.
\end{cases}\]
\item \textbf{(3分)} 复变函数 $f(z)=\sin z$ 和实变量函数 $g(x)=\sin x$ 的性质有什么相似和不同之处? 试列举一二.
\end{enumerate}
\newpage


\BiaoTi{2022年合肥工业大学考试参考答案(A)}
\XueQi{一}
\XueNian{2022}{2023}
\KeChengDaiMa{1400261B}
\KeChengMingCheng{复变函数与积分变换}
\XueFen{2.5}
\KeChengXingZhi{必修}
\KaoShiXingShi{闭卷}
\ZhuanYeBanJi{}
\KaoShiRiQi{2022年11月26日19:00-21:00}
\MingTiJiaoShi{集体}
\maketitle

\tigan{一、填空题(每小题3分,共15分)}

\textbf{请将你的答案对应填在横线上:}

\textbf{1.} \fillblank[2.5cm]{$e^{\pi/2}$}, 
\textbf{2.} \fillblank[2.5cm]{$1$}, 
\textbf{3.} \fillblank[2.5cm]{$0$}, 
\textbf{4.} \fillblank[2.5cm]{$2$}, 
\textbf{5.} \fillblank[2.5cm]{$2\pi j\delta(\omega+1)$}.

\tigan{二、选择题(每小题3分,共15分)}

\textbf{请将你所选择的字母 A, B, C, D 之一对应填在下列表格里:}

\xuanzeti{\textbf{题号}}{\textbf{答案}}%
\xuanzeti{1}{C}%
\xuanzeti{2}{D}%
\xuanzeti{3}{C}%
\xuanzeti{4}{A}%
\xuanzeti{5}{A}

\tigan{三、解答题}

\textbf{1. (6分) 【解】}

由于 $z=-3i+1-i(3+i)=2-6i$, \score2\\
因此 $|z|=2\sqrt{10}$, \score2\\
$\Arg z=2k\pi-\arctan 3,k\in\BZ$. \Score{(2分, 只有主值得1分)}

\textbf{2. (6分) 【解】}
\begin{align*}
\frac{e^{iz}-e^{-iz}}{2i}&=2\cdot\frac{e^{iz}+e^{-iz}}2, \score2\\
e^{iz}-e^{-iz}&=2i(e^{iz}+e^{-iz}), \\
e^{2iz}&=\frac{1+2i}{1-2i}=\frac{(1+2i)^2}5, \score1\\
2iz&=\Ln\frac{(1+2i)^2}5=(2\arctan 2+2k\pi)i, \score1\\
z&=\arctan 2+k\pi,\quad k\in\BZ. \Score{(2分, 只有主值得1分)}
\end{align*}
\textbf{其它答案}: $z=\dfrac\pi2-\dfrac12\arctan\dfrac43+k\pi, k\in\BZ$.

\textbf{3. (6分) 【解】}

由于 $\cos^2z$ 解析, 且 \score1
\begin{align*}
\int\cos^2z\diff z&=\int\frac{1+\cos(2z)}2\diff z \score1\\
&=\frac{z}2+\frac{\sin(2z)}4+C, \score1
\end{align*}
因此
\begin{align*}
\int_C\cos^2z\diff z&=\left[\frac{z}2+\frac{\sin(2z)}4\right]\bigg|_i^{-\pi} \score1\\
&=-\frac\pi2-\left[\frac i2+\frac{\sin(2i)}4\right] \score1\\
&=-\frac\pi2+\frac{(e^{-2}-4-e^2)i}8. \score1
\end{align*}

\textbf{4. (10分) 【解】}

由于 $f(z)=\dfrac{e^{iz}}{z^2-3\pi z+2\pi^2}$ 在 $|z-3|\le 4$ 内的奇点为 $\pi,2\pi$, \score3\\
因此
\begin{align*}
\oint_Cf(z)\diff z&=2\pi i\bigl[\Res[f(z),\pi]+\Res[f(z),2\pi]\bigr] \score2\\
&=2\pi i\biggl[\frac{e^{iz}}{z-2\pi}\bigg|_{z=\pi}+\frac{e^{iz}}{z-\pi}\bigg|_{z=2\pi}\biggr] \score2\\
&=2\pi i\biggl[\frac1\pi+\frac1\pi\biggr]=4i. \score3
\end{align*}

\textbf{5. (10分) 【解】}

由 $\Delta v=v_{xx}+v_{yy}=6x-2ay+6y-2ax=0$ 可知 $a=3$. \score3\\
由
\begin{align*}
f'(z)&=v_y+iv_x \score2\\
&=(3y^2-3x^2-6xy)+i(3x^2-6xy-3y^2) \score2\\
&=3(i-1)(x+iy)^2=3(i-1)z^2 \score1
\end{align*}
可知 $f(z)=(i-1)z^3+C, C\in\BR$. \Score{(2分, 没有常数项得1分)}

\textbf{其它解法}: 由 $u_x=v_y=3y^2-3x^2-6xy$ 得 $u=3xy^2-x^3-3x^2y+\psi(y)$. \score2\\
由 $u_y=-v_x=-(3x^2-6xy-3y^2)$ 得 $\psi'(y)=3y^2$, \\
$\psi(y)=y^3+C$, \Score{(3分, 没有常数项得2分)}
\begin{align*}
f(z)&=u+iv \\
&=3xy^2-x^3-3x^2y+y^3+C+i(x^3+y^3-3xy^2-3x^2y) \\
&=(i-1)z^3+C, C\in\BR. \score2
\end{align*}

\textbf{6. (10分) 【解】}

由于 $f(z)$ 的奇点是 $1$, 因此 $f(z)$ 在这两个圆环域内都解析.

(1)
由于
\[\frac1{1-z}=\sum_{n=0}^\infty z^n, \score1\]
因此
\begin{align*}
f(z)&=\frac{z-1+2}{(z-1)^2}=\frac1{z-1}+\frac{2}{(z-1)^2}=-\frac1{1-z}+2\left(\frac1{1-z}\right)' \score2\\
&=-\sum_{n=0}^\infty z^n+2\left(\sum_{n=0}^\infty z^n\right)'=-\sum_{n=0}^\infty z^n+2\sum_{n=1}^\infty nz^{n-1} \\
&=-\sum_{n=0}^\infty z^n+2\sum_{n=0}^\infty (n+1)z^n=\sum_{n=0}^\infty(2n+1)z^n. \score2
\end{align*}

(2) 
由于
\[\frac1{z-1}=\frac1z\cdot\frac{1}{1-\frac1z}=\sum_{n=1}^\infty z^{-n}, \score1\]
因此
\begin{align*}
f(z)&=\frac1{z-1}-2\left(\frac1{z-1}\right)'=\sum_{n=1}^\infty z^{-n}-2\left(\sum_{n=1}^\infty z^{-n}\right)' \score2\\
&=\sum_{n=1}^\infty z^{-n}-2\sum_{n=1}^\infty (-n)z^{-n-1} \\
&=\sum_{n=1}^\infty z^{-n}-2\sum_{n=1}^\infty (-n+1)z^{-n}=\sum_{n=1}^\infty (2n-1)z^{-n}. \score2
\end{align*}

\textbf{7. (10分) 【解】}

由于 $0$ 是分母的二阶零点, 因此它是 $f(z)$ 的二阶极点. \score1\\
由于 $\pm\pi$ 是分母的一阶零点, 因此它们是 $f(z)$ 的一阶极点. \score1
\begin{align*}
\Res[f(z),0]&=\left(\frac{\cos z}{z^2-\pi^2}\right)'\bigg|_{z=0} \score2\\
&=\frac{-\sin z\cdot(z^2-\pi^2)-\cos z\cdot 2z}{(z^2-\pi^2)^2}\bigg|_{z=0}=0, \score2\\
\Res[f(z),\pi]&=\frac{\cos z}{z^2(z+\pi)}\bigg|_{z=\pi}=-\frac1{2\pi^3}, \score2\\
\Res[f(z),-\pi]&=\frac{\cos z}{z^2(z-\pi)}\bigg|_{z=-\pi}=\frac1{2\pi^3}. \score2
\end{align*}

\textbf{8. (9分) 【解】}

设 $\msl[y]=Y$, 则
\[\msl[y'']=s^2Y-sy(0)-y'(0)=s^2Y-2, \score3\]
因此
\begin{align*}
s^2Y-2+2Y&=\msl[\sin t]=\frac{1}{s^2+1}, \score2\\
Y(s)&=\frac{2}{s^2+2}+\frac{1}{(s^2+1)(s^2+2)}=\frac{1}{s^2+1}+\frac{1}{s^2+2}, \score2\\
y(t)&=\msl^{-1}\left[\frac{1}{s^2+1}\right]+\msl^{-1}\left[\frac{1}{s^2+2}\right]
=\sin t+\frac{\sqrt 2}2\cdot\sin(\sqrt 2 t). \score2
\end{align*}

\textbf{9. (3分) 【解】}

例如(每项1分)
\begin{itemize}
\item $f'(z)=\cos z,g'(x)=\cos x$. \score1
\item $\sin z$ 处处可导, $\sin x$ 处处可导. \score1
\item 麦克劳林展开的系数相同. \score1
\item $\sin z$ 无界, $\sin x$ 有界. \score1
\end{itemize}


\newpage
\BiaoTi{2022年合肥工业大学试卷(B)}
\XueQi{一}
\XueNian{2022}{2023}
\KeChengDaiMa{1400261B}
\KeChengMingCheng{复变函数与积分变换}
\XueFen{2.5}
\KeChengXingZhi{必修}
\KaoShiXingShi{闭卷}
\ZhuanYeBanJi{}
\KaoShiRiQi{}
\MingTiJiaoShi{集体}
\maketitle

\tigan{一、填空题(每小题3分,共15分)}
\begin{enumerate}
  \item $-1+\sqrt 3i$ 的辐角主值是\fillblank{}.
  \item $i^{2022}-(-i)^{2022}=$\fillblank{}.
  \item 如果函数 $f(z)=e^{ax}(\cos y-i\sin y)$ 在复平面上处处解析, 则实数 $a=$\fillblank{}.
  \item 设 $C$ 为正向圆周 $|z|=1$, 则积分 $\displaystyle\oint_C\left(\frac{1+z+z^2}{z^3}\right)\diff z=$\fillblank{}.
  \item 函数 $e^{jt}$ 的傅里叶变换为\fillblank{}.
\end{enumerate}

\tigan{二、选择题(每小题3分,共15分)}
\begin{enumerate}
  \item 不等式 $1<|z|<2$ 确定是的(~~~~).
  \xx{有界多连通区域}{有界单连通区域}{无界多连通区域}{无界单连通区域}
  \item 方程 $|z+i|=|z-i|$ 表示的曲线是(~~~~).
  \xx{直线}{不是圆的椭圆}{双曲线}{圆周}
  \item 幂级数在其收敛圆周上(~~~~).
  \xx{一定处处绝对收敛}{一定处处条件收敛}{一定有发散的点}{可能处处收敛也可能有发散的点}
  \item 函数 $f(z)=u(x,y)+iv(x,y)$ 在 $z_0=x_0+iy_0$ 处可导的充要条件是(~~~~).
  \xx{$u,v$ 均在 $(x_0,y_0)$ 处连续}{$u,v$ 均在 $(x_0,y_0)$ 处有偏导数}{$u,v$ 均在 $(x_0,y_0)$ 处可微}{$u,v$ 均在 $(x_0,y_0)$ 处可微且满足C-R方程}
  \item $z=\pi$ 是函数 $\dfrac{\sin z}{(z-\pi)^2}$ 的(~~~~).
  \xx{一阶极点}{一阶零点}{可去奇点}{本性奇点}
\end{enumerate}

\tigan{三、解答题}
\begin{enumerate}
\item \textbf{(6分)} 设 $z=\dfrac{2+i}{1-2i}$, 求 $z$ 的模和辐角.
\item \textbf{(6分)} 求 $\sqrt[3]{-8}$.
\item \textbf{(7分)} 设 $C$ 是从 $i$ 到 $2+i$ 的直线, 求 $\displaystyle\int_C \overline z\diff z$.
\item \textbf{(7分)} 求 $\displaystyle\int_{-\pi i}^{\pi i}(e^z+1)\diff z$.
\item \textbf{(7分)} 求 $\displaystyle\int_0^\pi (z+\cos 2z)\diff z$.
\item \textbf{(7分)} 设 $C$ 为正向圆周 $|z|=4$, 求 $\displaystyle\oint_C\frac{z-6}{z^2+9}\diff z$.
\item \textbf{(8分)} 已知 $f(z)=u+iv$ 是解析函数, 其中 $u(x,y)=x^2+axy-y^2, v=2x^2-2y^2+2xy$ 且 $a$ 是实数.
求参数 $a$ 以及解析函数 $f'(z)$, 其中 $f'(z)$ 需要写成 $z$ 的表达式.
\item \textbf{(10分)} 确定函数 $f(z)=\dfrac{2}{z(z+2)}$ 在圆环域\\
\indent (1) $0<|z|<2$; \hspace{2em} (2) $2<|z|<+\infty$\\
内的洛朗级数展开式.
\item \textbf{(9分)} 用拉普拉斯变换求解微分方程初值问题
\[\begin{cases}
y''(t)+2y'(t)=8e^{2t},&\\
y(0)=0,\quad y'(0)=2.
\end{cases}\]
\item \textbf{(3分)} 复积分的计算方法或公式有哪些? 请给出至少三条.
\end{enumerate}

\newpage

\BiaoTi{2022年合肥工业大学考试参考答案(B)}
\XueQi{一}
\XueNian{2022}{2023}
\KeChengDaiMa{1400261B}
\KeChengMingCheng{复变函数与积分变换}
\XueFen{2.5}
\KeChengXingZhi{必修}
\KaoShiXingShi{闭卷}
\ZhuanYeBanJi{}
\KaoShiRiQi{}
\MingTiJiaoShi{集体}
\maketitle

\tigan{一、填空题(每小题3分,共15分)}

\textbf{请将你的答案对应填在横线上:}

\textbf{1.} \fillblank[2.5cm]{$\dfrac{2\pi}3$}, 
\textbf{2.} \fillblank[2.5cm]{$0$}, 
\textbf{3.} \fillblank[2.5cm]{$-1$}, 
\textbf{4.} \fillblank[2.5cm]{$2\pi i$}, 
\textbf{5.} \fillblank[2.5cm]{$2\pi\delta(\omega-1)$}.

\tigan{二、选择题(每小题3分,共15分)}

\textbf{请将你所选择的字母 A, B, C, D 之一对应填在下列表格里:}

\xuanzeti{\textbf{题号}}{\textbf{答案}}%
\xuanzeti{1}{A}%
\xuanzeti{2}{A}%
\xuanzeti{3}{D}%
\xuanzeti{4}{D}%
\xuanzeti{5}{A}

\tigan{三、解答题}

\textbf{1. (6分) 【解】}
由于 $\displaystyle z=\dfrac{(2+i)(1+2i)}{5}=i$, \score2\\
因此 $|z|=1$, \score2\\
$\Arg z=\dfrac\pi2+2k\pi,k\in\BZ$. \Score{(2分, 只有主值得1分)}

\textbf{2. (6分) 【解】}
由于 $-8=8e^{\pi i}$,\score2\\
因此
\[\sqrt[3]{-8}=2e^{\frac13(\pi i+2k\pi)},\quad k=0,1,2\score2\]
即
\[2e^{\frac{\pi i}3}=1+\sqrt3i,\quad
2e^{\pi i}=-2,\quad
2e^{\frac{5\pi i}3}=1-\sqrt3i.\score2\]

\textbf{3. (7分) 【解】}

直线 $C$ 的方程为 $z=t+i,0\le t\le 2$.\score2\\
因此 $\diff z=\diff t$, $\overline z=t-i$.\score2
\begin{align*}
\int\overline z\diff z&=\int_0^2(t-i)\diff t \score2\\
&=\left(\frac{t^2}2-it\right)\bigg|_0^2=2-2i.\score1
\end{align*}

\textbf{4. (7分) 【解】}
由于 $e^z+1$ 处处解析, 因此\score2
\begin{align*}
  \int_{-\pi i}^{\pi i}(e^z+1)\diff z
  &=(e^z+z)\Big|_{-\pi i}^{\pi i}\score2\\
  &=(e^{\pi i}+\pi i)-(e^{-\pi i}-\pi i)\score2\\
  &=2\pi i.\score1
\end{align*}

\textbf{5. (7分) 【解】}
由于 $z+\cos 2z$ 处处解析, 因此\score2
\begin{align*}
  \int_0^\pi (z+\cos 2z)\diff z
  &=\left(\frac{z^2}2+\frac{\sin 2z}2\right)\Big|_0^\pi\score2\\
  &=\frac{\pi^2}2+\frac{\sin 2\pi}2\score2\\
  &=\frac{\pi^2}2.\score1
\end{align*}

\textbf{6. (7分) 【解】}

由于 $f(z)=\dfrac{z-6}{z^2+9}$ 在 $|z|\le 4$ 内的奇点为 $\pm3i$, \score2\\
因此
\begin{align*}
\oint_Cf(z)\diff z&=2\pi i\bigl[\Res[f(z),3i]+\Res[f(z),-3i]\bigr] \score2\\
&=2\pi i\biggl[\frac{z-6}{z+3i}\bigg|_{z=3i}+\frac{z-6}{z-3i}\bigg|_{z=-3i}\biggr] \score2\\
&=2\pi i\left(\frac{3i-6}{6i}+\frac{-3i-6}{-6i}\right)=2\pi i \score1
\end{align*}

\textbf{7. (8分) 【解】}

由 $u_x=2x+ay=v_y=-4y+2x$ 可知 $a=-4$. \score2\\
由
\begin{align*}
f'(z)&=u_x+iv_x \score2\\
&=(-4y+2x)+i(4x+2y) \score2\\
&=(2+4i)z. \score2
\end{align*}

\textbf{8. (10分) 【解】}

由于 $f(z)$ 的奇点是 $0,2$, 因此 $f(z)$ 在这两个圆环域内都解析.

(1)
\begin{align*}
f(z)&=\frac1z-\frac1{z+2} \score2\\
&=\frac1z-\frac12\cdot\frac1{1+\frac z2}\score1 \\
&=\frac1z-\frac12\sum_{n=0}^\infty\left(-\frac z2\right)^n \score1 \\
&=\frac1z+\sum_{n=0}^\infty\frac{(-1)^{n+1}}{2^{n+1}}z^n=\sum_{n=-1}^\infty\frac{(-1)^{n+1}}{2^{n+1}}z^n.\score1
\end{align*}

(2) 
\begin{align*}
f(z)&=\frac1z-\frac1{z+2} \score2\\
&=\frac1z-\frac1z\cdot\frac1{1+\frac 2z}\score1 \\
&=\frac1z-\frac1z\sum_{n=0}^\infty\left(-\frac 2z\right)^n \score1 \\
&=-\sum_{n=1}^\infty\frac{(-2)^n}{z^{n+1}}=-\sum_{n=2}^\infty\frac{(-2)^{n-1}}{z^n}.\score1
\end{align*}

\textbf{9. (9分) 【解】}

设 $\msl[y]=Y$, 则
\[\msl[y'']=s^2Y-sy(0)-y'(0)=s^2Y-2, \score3\]
因此
\[s^2Y-2+2sY=\msl[8e^{2t}]=\frac8{s-2}, \score2\]
\begin{align*}
Y(s)&=\frac{2s+4}{(s-2)(s^2+2s)}=\frac{2}{s(s-2)}=\frac1{s-2}-\frac1s, \score2\\
y(t)&=\msl^{-1}\left[\frac{1}{s-2}\right]-\msl^{-1}\left[\frac{1}{s}\right]
=e^{2t}-1. \score2
\end{align*}

\textbf{10. (3分) 【解】}
例如
\begin{itemize}
  \item 列出参数方程 $z=z(t)$ 并将积分表达为 $t$ 的积分形式;
  \item 单连通区域内解析函数的积分可以用原函数计算;
  \item 利用柯西-古萨基本定理;
  \item 利用复合闭路定理;
  \item 利用柯西积分公式;
  \item 利用高阶导数的柯西积分公式;
  \item 利用留数;
  \item 利用长大不等式.
\end{itemize}


\newpage
\BiaoTi{2023年合肥工业大学试卷(A)}
\XueQi{一}
\XueNian{2023}{2024}
\KeChengDaiMa{1400261B}
\KeChengMingCheng{复变函数与积分变换}
\XueFen{2.5}
\KeChengXingZhi{必修}
\KaoShiXingShi{闭卷}
\ZhuanYeBanJi{}
\KaoShiRiQi{2023年12月2日19:00-21:00}
\MingTiJiaoShi{集体}
\maketitle

\tigan{一、填空题(每小题3分,共15分)}
\begin{enumerate}
\item $2^{-i}$ 的辐角主值是\fillblank{}.
\item $2023-i$ 绕 $0$ 逆时针旋转 $\dfrac\pi2$ 后得到的复数是\fillblank{}.
\item 如果函数 $f(z)=\dfrac1{(z+5)\sin z}$ 可以在圆环域 $0<|z|<R$ 内作洛朗展开, 则 $R$ 的最大值为\fillblank{}.
\item 设 $f(z)=e^z-|z|\cos z$, 则 $\displaystyle\oint_{|z|=1}f(z)\diff z=$\fillblank{}.
\item 函数 $f(t)=\cos(3t)$ 的傅里叶变换为 $F(\omega)=$\fillblank{}.
\end{enumerate}

\tigan{二、选择题(每小题3分,共15分)}
\begin{enumerate}
\item 函数 $f(z)=\dfrac 1z$ 在下面哪个区域内有原函数?~(~~~~)
\xx{$0<|z|<1$}{$\Re z>0$}{$|z-1|>2$}{$|z+1|+|z-1|>4$}
\item 设 $f(z)=\displaystyle\oint_{|\zeta|=4}\dfrac{\sin\zeta-\cos\zeta}{\zeta-z}\diff\zeta$, 则 $f'(\pi)=$(~~~~)
\xx{$0$}{$2\pi i$}{$-2\pi i$}{$\pi i$}
\item 幂级数 $\displaystyle\sum_{n=1}^\infty \frac{n!}{n^n}z^n$ 的收敛半径是(~~~~).
\xx{$0$}{$+\infty$}{$e$}{$1$}
\item 下面哪个函数不能作为解析函数的虚部?~(~~~~)
\xx{$2x+3y$}{$2x^2+3y^2$}{$x^2-xy-y^2$}{$e^x\cos y$}
\item $z=0$ 是函数 $f(z)=\dfrac{(e^z-1)^2z^3}{\sin z^8}$ 的(~~~~).
\xx{一阶极点}{本性奇点}{可去奇点}{三阶极点}
\end{enumerate}


\tigan{三、解答题}
\begin{enumerate}
\item \textbf{(6分)} 计算 $(-1+i)^{10}-(-1-i)^{10}$.
\item \textbf{(6分)} 解方程 $\cos z=\dfrac{3\sqrt2}4$.
\item \textbf{(6分)} 设 $C$ 为有向曲线 $z(t)=\sin t+it,0\le t\le \pi$, 求 $\displaystyle\int_C ze^z \diff z$.
\item \textbf{(10分)} 设 $C$ 为正向圆周 $|z-1|=4$, 求 $\displaystyle\oint_C\frac{\sin z}{z^2+1}\diff z$.
\item \textbf{(10分)} 假设 $u(x,y)=x^3+ax^2y+bxy^2-3y^3$ 是调和函数,求参数 $a,b$ 以及 $v(x,y)$ 使得 $v(0,0)=0$ 且 $f(z)=u+iv$ 是解析函数.
\item \textbf{(10分)} 确定函数 $f(z)=\dfrac{z}{z^2-3z+2}$ 在圆环域\\[6pt]
\indent (1) $|z|>2$; \hspace{2em} (2) $0<|z-2|<1$\\
内的洛朗级数展开式.
\item \textbf{(10分)} 设 $f(z)=\dfrac{e^z}{(z-\pi i)(z-2\pi i)^2}$. 求 $f(z)$ 在有限复平面内的奇点以及 $\displaystyle\oint_{|z|=8}f(z)\diff z$.
\item \textbf{(9分)} 用拉普拉斯变换求解微分方程初值问题
\[\begin{cases}
y''(t)+4y(t)=3\cos t,&\\
y(0)=1,\quad y'(0)=2.
\end{cases}\]
\item \textbf{(3分)} 复变函数 $f(z)=e^z$ 和实变量函数 $g(x)=e^x$ 的性质有什么相似和不同之处? 试举出三点.
\end{enumerate}


\newpage

\BiaoTi{2023年合肥工业大学考试参考答案(A)}
\XueQi{一}
\XueNian{2023}{2024}
\KeChengDaiMa{1400261B}
\KeChengMingCheng{复变函数与积分变换}
\XueFen{2.5}
\KeChengXingZhi{必修}
\KaoShiXingShi{闭卷}
\ZhuanYeBanJi{}
\KaoShiRiQi{2023年12月2日19:00-21:00}
\MingTiJiaoShi{集体}
\maketitle

\tigan{一、填空题(每小题3分,共15分)}

\textbf{请将你的答案对应填在横线上:}

\textbf{1.} \fillblank[2cm]{$-\ln 2$}, 
\textbf{2.} \fillblank[2.5cm]{$1+2023i$}, 
\textbf{3.} \fillblank[1.7cm]{$\pi$}, 
\textbf{4.} \fillblank[1.7cm]{$0$}, 
\textbf{5.} \fillblank[4.5cm]{$\pi[\delta(\omega+3)+\delta(\omega-3)]$}.

\tigan{二、选择题(每小题3分,共15分)}

\textbf{请将你所选择的字母 A, B, C, D 之一对应填在下列表格里:}

\xuanzeti{\textbf{题号}}{\textbf{答案}}%
\xuanzeti{1}{B}%
\xuanzeti{2}{C}%
\xuanzeti{3}{C}%
\xuanzeti{4}{B}%
\xuanzeti{5}{D}

\tigan{三、解答题}

\textbf{1. (6分) 【解】}
由于
{\large
\[-1+i=\sqrt 2 e^{\frac{3\pi i}4}, \score2\]}
因此
{\large
\begin{align*}
  (-1+i)^{10}&=32e^{\frac{30\pi i}4}=-32i, \score1\\
  (-1-i)^{10}&=32e^{-\frac{30\pi i}4}=32i, \score1
\end{align*}}
故
{\large
\[(-1+i)^{10}-(-1-i)^{10}=-64i. \score2\]}
也可以直接计算 $(-1+i)^2=-2i$ 得到 $(-1+i)^{10}=-32i$.

\textbf{2. (6分) 【解】}
由
\[\cos z=\frac{e^{iz}+e^{-iz}}2=\frac{3\sqrt2}4,\score1\]
整理得到
\[e^{2iz}-\frac{3\sqrt2}2e^{iz}+1=0,\score2\]
\[e^{iz}=\frac12\left[\frac{3\sqrt2}2\pm\sqrt{\Bigl(\frac{3\sqrt2}2\Bigr)^2-4}\right]
  =\sqrt2\text{ 或 }\frac{\sqrt2}2,\score2\]
因此
\[iz=\Ln\sqrt2=\frac12\ln2+2k\pi i\text{ 或 }\Ln\frac{\sqrt2}2=-\frac12\ln 2+2k\pi i,\]
\[z=2k\pi\pm\frac{\ln 2}2i, k\in\BZ. \score1\]

\textbf{3. (6分) 【解】}
由于 $ze^z$ 解析, 且 \score1
\[\int ze^z\diff z=(z-1)e^z+C, \score2\]
而曲线 $C$ 的起点是 $0$, 终点是 $\pi i$, 因此
\begin{align*}
\int_C ze^z\diff z&=(z-1)e^z\bigg|_0^{\pi i} \score1\\
&=(\pi i-1)e^{\pi i}-(-1)
=2-\pi i. \score2
\end{align*}

\textbf{4. (10分) 【解】}
由于 $f(z)=\dfrac{\sin z}{(z+i)(z-i)}$ 在 $|z-1|\le 4$ 内的奇点为 $\pm i$, 因此\score2
\begin{align*}
\oint_Cf(z)\diff z&=2\pi i\bigl[\Res[f(z),-i]+\Res[f(z),-2i]\bigr] \score2\\
&=2\pi i\biggl[\frac{\sin z}{z+i}\bigg|_{z=i}+\frac{\sin z}{z-i}\bigg|_{z=-i}\biggr] \score2\\
&=2\pi i\biggl[\frac{\sin i}{2i}+\frac{\sin(-i)}{-2i}\biggr]\score 2\\
&=2\pi\sin i=\pi i\bigl(e-\frac1e\bigr). \score2
\end{align*}

\textbf{5. (10分) 【解】}
由
\[\Delta u=u_{xx}+u_{yy}=6x+2ay+2bx-18y=0\]
可知 $a=9,b=-3$. 由\score3
\begin{align*}
  f'(z)&=u_x-iu_y \score2\\
  &=(3x^2+18xy-3y^2)-i(9x^2-6xy-9y^2) \score2\\
  &=(3-9i)(x+iy)^2=(3-9i)z^2 \score1
\end{align*}
可知
\[f(z)=(1-3i)z^3+C, C\in\BR. \score1 \]
由 $v(0,0)=0,f(0)=0$ 可知 $C=0$,
\[v=-3x^3+3x^2y+9xy^2-y^3.\score1\]

\vspace*{10pt}
\textbf{其它解法}: 由 $u_x=v_y=3x^2+18xy-3y^2$ 得
\[v=3x^2y+9xy^2-y^3+\psi(x). \score2\]
由 $v_x=-u_y=-(9x^2-6xy-9y^2)$ 得
\[\psi'(x)=-9x^2, \score2\]
\[\psi(x)=-3x^3+C,\quad v=-3x^3+3x^2y+9xy^2-y^3+C. \score2\]
由 $v(0,0)=0$ 可知 $C=0$,
\[v=-3x^3+3x^2y+9xy^2-y^3. \score1\]

\textbf{6. (10分) 【解】}
由于 $f(z)$ 的奇点是 $1,2$, 因此 $f(z)$ 在这两个圆环域内都解析.

(1)
由于
\begin{align*}
  \frac1{z-1}&=\frac1z\cdot\frac1{1-\dfrac1z}=\sum_{n=1}^\infty \frac1{z^n}, \score2\\
  \frac1{z-2}&=\frac1z\cdot\frac1{1-\dfrac2z}=\sum_{n=1}^\infty \frac{2^{n-1}}{z^n}, \score2
\end{align*}
因此
\[f(z)=\frac2{z-2}-\frac1{z-1}=\sum_{n=1}^\infty \frac{2^n-1}{z^n}. \score2\]

(2) 
由于
\[\frac1{z-1}=\frac1{1+(z-2)}=\sum_{n=0}^\infty (-1)^n(z-2)^n, \score2\]
因此
\begin{align*}
  f(z)&=\frac2{z-2}-\frac1{z-1}
  =\frac2{z-2}-\sum_{n=0}^\infty (-1)^n(z-2)^n. \score2
\end{align*}

\textbf{7. (10分) 【解】}
由于 $\pi i$ 是分母的一阶零点, 因此它是 $f(z)$ 的一阶极点. \score1\\
由于 $2\pi i$ 是分母的二阶零点, 因此它是 $f(z)$ 的二阶极点. \score1
\[\Res[f(z),\pi i]=\frac{e^z}{(z-2\pi i)^2}\bigg|_{z=\pi i}=\frac1{\pi^2}, \score2\]
\begin{align*}
  \Res[f(z),2\pi i]&=\left(\frac{e^z}{z-\pi i}\right)'\bigg|_{z=2\pi i} \score2\\
  &=\frac{e^z(z-\pi i-1)}{(z-\pi i)^2}\bigg|_{z=2\pi i}=\frac{1-\pi i}{\pi^2}, \score2
\end{align*}
\[\oint_{|z|=8}f(z)\diff z=2\pi i\bigl[\Res[f(z),\pi i]+\Res[f(z),2\pi i]\bigr]=2+\frac{4}\pi i. \score2\]

\textbf{8. (9分) 【解】}
设 $\msl[y]=Y$, 则
\[\msl[y'']=s^2Y-sy(0)-y'(0)=s^2Y-s-2, \score3\]
因此
\[s^2Y-s-2+4Y=3\msl[\cos t]=\frac{3s}{s^2+1}, \score2\]
\begin{align*}
Y(s)&=\frac{s+2}{s^2+4}+\frac{3s}{(s^2+1)(s^2+4)}=\frac{s^3+2s^2+4s+2}{(s^2+1)(s^2+4)}\\
&=\frac{2}{s^2+4}+\frac{s}{s^2+1}, \score2\\
y(t)&=\msl^{-1}\left[\frac{2}{s^2+4}\right]+\msl^{-1}\left[\frac{s}{s^2+1}\right]
=\sin 2t+\cos t. \score2
\end{align*}

\textbf{9. (3分) 【解】}
例如(每项1分)
\begin{itemize}
\item $f'(z)=e^z,g'(x)=e^x$. \score1
\item $e^z$ 处处可导, $e^x$ 处处可导. \score1
\item 麦克劳林展开的系数相同. \score1
\item $e^z$ 无界, $e^x$ 无界. \score1
\item $e^z$ 是周期的, $e^x$ 不是. \score1
\end{itemize}

\end{document}
