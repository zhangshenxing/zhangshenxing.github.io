\documentclass[aspectratio=169,handout]{ctexbeamer}
\usepackage{../../../latex/bamboo}
\providetranslation[to=ChineseUTF8]{Example}{例题}
\providetranslation[to=ChineseUTF8]{Solution}{解答}
\definecolor{main}{RGB}{0,0,224}%
\definecolor{second}{RGB}{224,0,0}%
\definecolor{third}{RGB}{112,0,112}%
\definecolor{fourth}{RGB}{0,128,0}%
\definecolor{fifth}{RGB}{255,128,0}%

% Black Theme
% \definecolor{main}{RGB}{0,0,0}%
% \definecolor{second}{RGB}{0,0,0}%
% \definecolor{third}{RGB}{0,0,0}%
% \definecolor{fourth}{RGB}{0,0,0}%
% \definecolor{fifth}{RGB}{0,0,0}%

% Doremi Theme
% \definecolor{main}{RGB}{73,119,213}%
% \definecolor{second}{RGB}{220,60,105}%
% \definecolor{third}{RGB}{122,89,166}%
% \definecolor{fourth}{RGB}{229,143,52}%
% \definecolor{fifth}{RGB}{244,220,10}%

\usepackage{bm}
\usepackage{extarrows}
\usepackage{mathrsfs}
\usepackage{stmaryrd}
\usepackage{multirow}
\usepackage{tikz}
\usepackage{color,calc}
\usepackage{caption}
\usepackage{subcaption}
\usepackage{booktabs}
% \usepackage{wrapfig}
% 文本色
\renewcommand\emph[1]{{\color{main}{\bf #1}}}
\ifcsundef{alert}{
  \newcommand{\alert}[1]{\textcolor{second}{\bf #1}}
}{}
% 非考试内容
\newcommand{\noexer}{\hfill\mdseries\itshape\color{black}\small 非考试内容}
% 枚举数字引用标志
\newcommand\enumnum[1]{{\mdseries\upshape\textcolor{main}{(#1)}}}
% 缩进
\renewcommand{\indent}{\hspace*{1em}}
\setlength{\parindent}{1em}
% 减少公式垂直间距, 配合\endgroup
\newcommand\beqskip[1]{\begingroup\abovedisplayskip=#1\belowdisplayskip=#1\belowdisplayshortskip=#1}

\RequirePackage{tasks}
\settasks{label-format=\textcolor{main},label={(\arabic*)},label-width=1.5em}
% 选择题选项
\NewTasksEnvironment[label={\upshape(\Alph*)},label-width=1.5em]{taskschoice}[()]
% 一行显示多题答案
\NewTasksEnvironment[label={\arabic*.},label-width=1.5em]{tasksans}[()]
% TIKZ 设置
\usetikzlibrary{
	quotes,
	shapes.arrows,
	arrows.meta,
	positioning,
	shapes.geometric,
	overlay-beamer-styles,
	patterns,
	calc,
	angles,
	decorations.pathreplacing,
	backgrounds % 背景边框
}
\tikzset{
	background rectangle/.style={semithick,draw=fourth,fill=white,rounded corners},
  % arrow
	cstra/.style      ={-Stealth},        % right arrow
	cstla/.style      ={Stealth-},        % left arrow
	cstlra/.style     ={Stealth-Stealth}, % left-right arrow
	cstwra/.style     ={-Straight Barb},  % wide ra
	cstwla/.style     ={Straight Barb-},
	cstwlra/.style    ={Straight Barb-Straight Barb},
	cstnra/.style      ={-Latex, line width=0.1cm},
	cstmra/.style      ={-Latex, line width=0.05cm},
	cstmlra/.style     ={Latex-Latex, line width=0.05cm},
	cstaxis/.style        ={-Stealth, thick}, %坐标轴
  % curve
	cstcurve/.style       ={very thick}, %一般曲线
	cstdash/.style        ={thick, dash pattern= on 0.2cm off 0.05cm}, %虚线
  % dot
	cstdot/.style         ={radius=.08}, %实心点
	cstdote/.style        ={radius=.07, fill=white}, %空心点
  % fill
	cstfill/.style       ={fill=black!10},
	cstfille/.style      ={pattern=north east lines, pattern color=black},
	cstfill1/.style       ={fill=main!20},
	cstfille1/.style      ={pattern=north east lines, pattern color=main},
	cstfill2/.style        ={fill=second!20},
	cstfille2/.style       ={pattern=north east lines, pattern color=second},
	cstfill3/.style        ={fill=third!20},
	cstfille3/.style       ={pattern=north east lines, pattern color=third},
	cstfill4/.style        ={fill=fourth!20},
	cstfille4/.style       ={pattern=north east lines, pattern color=fourth},
	cstfill5/.style        ={fill=fifth!20},
	cstfille5/.style       ={pattern=north east lines, pattern color=fifth},
  % node
	cstnode/.style        ={fill=white,draw=black,text=black,rounded corners=0.2cm,line width=1pt},
	cstnode1/.style       ={fill=main!15,draw=main!80,text=black,rounded corners=0.2cm,line width=1pt},
	cstnode2/.style       ={fill=second!15,draw=second!80,text=black,rounded corners=0.2cm,line width=1pt},
	cstnode3/.style       ={fill=third!15,draw=third!80,text=black,rounded corners=0.2cm,line width=1pt},
	cstnode4/.style       ={fill=fourth!15,draw=fourth!80,text=black,rounded corners=0.2cm,line width=1pt},
	cstnode5/.style       ={fill=fifth!15,draw=fifth!80,text=black,rounded corners=0.2cm,line width=1pt}
}
\renewcommand\logo[1]{
	\def\inserttitlegraphic{../../../image/logo/#1.png}
	\def\insertinstitutegraphic{../../../image/logo/#1name.png}
}
\email{zhangshenxing@hfut.edu.cn}
\website{https://zhangshenxing.github.io}
\date{}
\logo{hfut}
\author{张神星}
\institute{合肥工业大学}
\office{翡翠科教楼B1810东}

\RequirePackage[T1]{fontenc}
\setCJKsansfont[ItalicFont={KaiTi},BoldFont={LXGW ZhenKai}]{Source Han Sans HW SC}
\newfontface\cmunrm{cmunrm.otf}\newcommand\cmu[1]{{\cmunrm{#1}}}

\title{线性代数(复习课)}
\let\rank\relax\DeclareMathOperator\rank{\rank}
\begin{document}


\begin{frame}{矩阵及其运算}
	\onslide<+->
	\begin{itemize}
		\item 加减法和数乘、乘法.
		\item 一般 $\bfA\bfB\neq \bfB\bfA$, 但 $\bfA=\bfE,\bfO,f(\bfB)/g(\bfB)$ 时可以.
		\item 一般 $\bfA\bfC=\bfB\bfC$ 推不出 $\bfA=\bfB$, 但 $\bfC$ 可逆时可以.
		\item 矩阵的幂: 对角阵、$\lambda \bfE+\bfN$、$\bfu\bfv^\rmT$ 情形, 或相似与这些矩阵情形.
	\end{itemize}
\end{frame}
	
	
\begin{frame}{转置、伴随、逆}
	\onslide<+->
	\begin{itemize}
		\item $(\bfA\bfB)^\rmT=\bfB^\rmT\bfA^\rmT,\quad
			(\bfA\bfB)^*=\bfB^*\bfA^*,\quad
			(\bfA\bfB)^{-1}=\bfB^{-1}\bfA^{-1}.$
		\item $(\bfA^\rmT)^*=(\bfA^*)^\rmT,\quad
			(\bfA^\rmT)^{-1}=(\bfA^{-1})^\rmT,\quad
			(\bfA^*)^{-1}=(\bfA^{-1})^*.$
		\item $\bfA\bfA^*=\bfA^*\bfA=|\bfA|\bfE_n$, $(k\bfA)^*=k^{n-1}\bfA^*$,  $|\bfA^*|=|\bfA|^{n-1}$.
		\item $(\bfA^*)^*=\begin{cases}
			\bfA,&n=2;\\
			|\bfA|^{n-2}\bfA,&n\ge3.
		\end{cases}$
		\item $\rank(\bfA^*)=\begin{cases}
			n,&\rank(\bfA)=n;\\
			1,&\rank(\bfA)=n-1;\\
			0,&\rank(\bfA)\le n-2.
		\end{cases}.$
		\item $(\bfA,\bfE)\simr (\bfE,\bfA^{-1})$,
		$(\bfA,\bfB)\simr (\bfE,\bfA^{-1}\bfB)$.
		\item $\bfA$ 可逆$\iff |\bfA|\neq0 \iff \rank(\bfA)=n\iff\bfA$ 的行最简形为 $\bfE$ $\iff\bfA\sim\bfE\iff\bfA\simr\bfE\iff\bfA\simc\bfE\iff\bfA\bfx={\bf0}$ 只有零解 $\iff\bfA\bfx=\bfb$ 总有解$\iff\bfA\bfx=\bfb$ 总有唯一解 $\iff\bfA$ 特征值都非零.
	\end{itemize}
\end{frame}


\begin{frame}{行列式}
	\onslide<+->
	\begin{itemize}
		\item 行列式:$2$, $3$ 阶直接算. 对角阵和三角阵行列式的计算, 分块情形类似.
		\item 一般用初等变换计算.
		\item $\rank(\bfA)<n$ 时 $|\bfA|=0$.
		\item $|\bfA\bfB|=|\bfA|\cdot|\bfB|$, $|\bfA^\rmT|=|\bfA|$, $|k\bfA|=k^n|\bfA|$.
		\item 拉普拉斯展开, 以及 $\sum_{j=1}^n a_{ij}A_{jk}=0$ 或 $1$.
		\item 特殊形状行列式的计算, 范德蒙行列式.
		\item 互换两行(列)后, 方阵的行列式变为 $-1$ 倍.
		\item 方阵的某一行(列)乘 $k$ 后, 方阵的行列式变为 $k$ 倍.
		\item 将方阵某一行(列)对应向量写成两个向量之和, 则行列式也可对应拆成两个行列式之和.
		\item 具有相同的两行(列)的方阵的行列式为零: $|\cdots,\bfv,\cdots,\bfv,\cdots|=0$.
		\item 若方阵有一行(列)全为零, 则行列式为零: $|\cdots,{\bf0},\cdots|=0$.
		\item 若方阵有两行(列)成比例, 则行列式为零: $|\cdots,\bfv,\cdots,k\bfv,\cdots|=0$.
		\item 行列式中某一行(列)的公因子可以提到行列式外面.
	\end{itemize}
\end{frame}


\begin{frame}{向量组}
	\onslide<+->
	\begin{itemize}
		\item 线性组合: $\bmb=\lambda_1\bma_1+\cdots+\lambda_m\bma_m\iff \bmb=\bfA\bfx$ 有解 $\iff \bmb\in V$. 其中 $V$ 是这组向量生成的空间.
		\item 线性无关: $\lambda_1\bma_1+\cdots+\lambda_m\bma_m={\bf0}$ 只有零解$\iff$
		$\bfA\bfx={\bf0}$ 只有零解$\iff$
		$V$ 中向量可唯一表示$\iff$
		$\rank(\bfA)$ 列满秩$\iff$
		$\rank(S)=m$$\iff$
		$\dim V=m$.
		\item 线性相关: 存在不全为零的数使得 $\lambda_1\bma_1+\cdots+\lambda_m\bma_m={\bf0}$$\iff$
		$\rank(S)<m$.
		\item 设 $S,T$ 分别为 $\bfA,\bfB$ 的列向量组, 且分别生成空间 $V,W$. 
		$T$ 可以被 $S$ 线性表示$\iff$ $W\subseteq V$
		$\iff$ $\exists\bfX$ 使得 $\bfA\bfX=\bfB$.
		\item $S,T$ 向量组等价$\iff W=V$ $\iff$ $\exists\bfX,\bfY$ 使得 $\bfB=\bfA\bfX,\bfA=\bfB\bfY$. 
		\item 设 $\bma_1,\dots,\bma_m$ 线性无关, $(\bmb_1,\dots,\bmb_n)=(\bma_1,\dots,\bma_m)\bfC$. 
		$\bmb_1,\dots,\bmb_n$ 线性无关$\iff \bfC\bfx={\bf0}$ 只有零解.
		\item 向量组线性相关$\iff$其中至少有一个向量可以由其它向量线性表示.
		\item 若 $S$ 线性无关, $S\cup\{\bmb\}$ 线性相关, 则 $\bmb$ 可以由 $S$ 唯一线性表示.
		\item 部分相关$\implies$整体相关, 整体无关$\implies$部分无关.
		\item 高维相关$\implies$低维相关, 低维无关$\implies$高维无关.
		\item 多的由少的表示, 多的一定线性相关.
	\end{itemize}
\end{frame}


\begin{frame}{秩、基、极大无关组}
	\onslide<+->
	\begin{itemize}
		\item 设 $\rank(\bfA)=r$, 则存在非零的 $r$ 阶子式, 但所有的 $r+1$ 阶子式都是零.
		\item 设 $\bfA\in M_{m\times n}$, 则 $0\le \rank(\bfA)\le \min(m,n)$.
		\item $\rank(\bfA)=0\iff \bfA=\bfO$;
		\item $n$ 阶方阵 $\bfA$ 可逆 $\iff \rank(\bfA)=n$;
		\item $\rank(k\bfA)=\rank(\bfA)=\rank(\bfA^\rmT), k\neq 0$;
		\item $\bfA\sim\bfB\iff \rank(\bfA)=\rank(\bfB)$;
		\item $\rank(\bfA\bfB)\le \min\bigl(\rank(\bfA),\rank(\bfB)\bigr)$;
		\item 若 $\bfA_{m\times n}\bfB_{n\times\ell}=\bfO$, 则 $\rank(\bfA)+\rank(\bfB)\le n$;
		\item $\rank(a\bfA+b\bfB)\le \rank(\bfA,\bfB)\le \rank(\bfA)+\rank(\bfB)$.
		特别地, $\max\bigl(\rank(\bfA),\rank(\bfB)\bigr)\le \rank(\bfA,\bfB)$.
		\item 一组基: $S$ 大小是 $n$; $S$ 生成 $V$; $S$ 线性无关. 其中任意两条. 极大无关组类似.
	\end{itemize}
\end{frame}


\begin{frame}{线性方程组}
	\onslide<+->
	\begin{itemize}
		\item 设 $\bfA\in M_{m\times n}, \rank(\bfA)=r$.
		线性方程组 $\bfA\bfx={\bf0}$ 的基础解系包含 $n-r$ 个向量.
		\item 若 $\rank(\bfA)<\rank(\bfA,\bfb)$, 则 $\bfA \bfx=\bfb$ 无解;
		\item 若 $\rank(\bfA)=\rank(\bfA,\bfb)=n$, 则 $\bfA \bfx=\bfb$ 有唯一解;
		\item 若 $\rank(\bfA)=\rank(\bfA,\bfb)<n$, 则 $\bfA \bfx=\bfb$ 有无穷多解.
	\end{itemize}
\end{frame}


\begin{frame}{方阵}
	\onslide<+->
	\begin{itemize}
		\item 正交阵: 行(列)向量组是标准正交基.
		\item $\bfA$ 正交$\iff\bfA^\rmT,\bfA^{-1},\bfA^*$ 也正交.
		\item $|\bfA|=\pm1$.
		\item 对角元之和$=$迹 $\Tr(A)=$特征值之和. ($\Tr(A^2)=$特征值平方和)
		\item 行列式$=$特征值乘积.
		\item $g(\bfA)/h(\bfA)$ 的特征值, $\bfA^\rmT,\bfA^*,\bfP^{-1}\bfA\bfP$ 特征值.
		\item 对称阵: $\bfA=\bfA^\rmT$, 反对称阵: $\bfA=-\bfA^\rmT$.
		\item 特征值: 解特征多项式.
	\end{itemize}
\end{frame}



\begin{frame}{矩阵关系}
	\onslide<+->
	\begin{itemize}
		\item 行等价: $\bfA\simr\bfB\iff \bfB=\bfP\bfA,\bfP$ 可逆$\iff$ 行向量组等价$\iff$ 列向量组保持线性关系$\iff\bfA\bfx={\bf0}$ 和 $\bfB\bfx={\bf0}$ 等价.
		\item 等价: $\bfA\sim\bfB\iff \bfB=\bfP\bfA\bfQ,\bfP,\bfQ$ 可逆$\iff \rank(\bfA)=\rank(\bfB)$.
		\item 相似: 若 $\bfB=\bfP^{-1}\bfA\bfP$, 则特征值、迹、行列式、特征多项式相同.
		\item 可对角化$\iff$有 $n$ 个线性无关特征向量(作为 $\bfP$ 的列). 特征值两两不同, 或对所有特征值 $\rank(\bfA-\lambda\bfE)=n-k$ ($\lambda$ 是 $k$ 重).
		\item 不同特征值的特征向量线性无关.
		\item 相合: 即二次型等价 $\bfB=\bfP^\rmT\bfP$. 特征值特征向量(可)全实, 可对角化. 标准型 $\diag(\bfE_p,-\bfE_{r-p},\bfO)$.
		\item 正交相合: $\bfP$ 是正交阵. 将特征向量正交单位化.
		\item 正定: 特征值全正, $p=n$, 顺序主子式全正(对角元全正).
	\end{itemize}
\end{frame}





\begin{frame}{极大线性无关组和秩的计算方法}
	\onslide<+->
	\begin{enumerate}
		\item 将向量组以列向量形式组成矩阵 $\bfA=(\bma_1,\dots,\bma_m)$.
		\item 通过初等行变换将 $\bfA$ 变为行阶梯形矩阵.
			\begin{itemize}
				\item 行阶梯形矩阵非零行的行数就是秩 $\rank(\bfA)$;
				\item 行阶梯形矩阵每个非零行的首个非零元对应的 $\bfA$ 的列向量, 就是极大线性无关组.
			\end{itemize}
		\item 继续化简为行最简形矩阵, 则可将其余向量表示为极大线性无关组的线性组合.
	\end{enumerate}
\end{frame}



\begin{frame}{齐次线性方程组的解法}
	\onslide<+->
	\begin{enumerate}
		\item 将系数矩阵通过初等行变换化为行最简形.
		\item 将矩阵重新写成方程形式 $x_i+\cdots=0$.
		\item 移项, 使得等式左侧只有阶梯拐角列 $i$ 对应的 $x_i=\cdots$.
		\item 添加 $n-r$ 项 $x_j=x_j$, 使得等式左边凑成 $\bfx$.
		\item 等式右侧是非拐角列 $j$ 对应的 $x_j$ 的组合, 其系数形成的 $n-r$ 个向量就是基础解系.
	\end{enumerate}
\end{frame}



\begin{frame}{非齐次线性方程组的解法}
	\onslide<+->
	\begin{enumerate}
		\item 写: 写出方程组对应的增广矩阵 $(\bfA,\bfb)$;
		\item 变: 通过初等行变换将其化为行最简形;
		\item 判: 通过行最简形判定方程是否有解;
		\item 解: 若系数矩阵部分零行对应的常数项均为零, 则方程有解.
		\item 类似于齐次情形, 将矩阵重新写成方程形式、移项、添恒等式, 使得等式左边凑成 $\bfx$.
		\item 等式右侧的常数部分是特解,其余是非拐角列 $j$ 对应的 $x_j$ 的组合, 其系数形成的 $n-r$ 个向量就是基础解系.
	\end{enumerate}
\end{frame}


\begin{frame}{格拉姆-施密特正交化}
	\begin{align*}
		\bmb_1&=\bma_1\\
		\bmb_2&=\bma_2-\frac{[\bma_2,\bmb_1]}{[\bmb_1,\bmb_1]}\bmb_1\\
		\bmb_3&=\bma_3-\frac{[\bma_3,\bmb_1]}{[\bmb_1,\bmb_1]}\bmb_1-\frac{[\bma_3,\bmb_2]}{[\bmb_2,\bmb_2]}\bmb_2\\
		&\vdots\\
		\bmb_r&=\bma_r-\frac{[\bma_r,\bmb_1]}{[\bmb_1,\bmb_1]}\bmb_1-\cdots-\frac{[\bma_r,\bmb_{r-1}]}{[\bmb_{r-1},\bmb_{r-1}]}\bmb_{r-1}
	\end{align*}
	 $\bfe_1=\dfrac{\bmb_1}{\|\bmb_1\|},\dots,\bfe_r=\dfrac{\bmb_r}{\|\bmb_r\|}$ 就是 $V$ 的一组标准正交基.
\end{frame}




\begin{frame}{特征值和特征向量的计算}
	\onslide<+->
	\begin{enumerate}
		\item 求 $\bfA$ 的特征多项式 $f(\lambda)=|\bfA-\lambda \bfE|$;
		\item 解 $f(\lambda)=|\bfA-\lambda \bfE|=0$ 得到特征值;
		\item 对于每一个特征值 $\lambda_i$, 解 $(\bfA-\lambda_i\bfE)\bfx={\bf0}$, 其\alert{非零解}就是对应特征向量.
	\end{enumerate}
	\onslide<+->
	相似对角化的步骤如下:
	\begin{enumerate}
		\item 求出 $\bfA$ 的所有特征值 $\lambda_i$ 和特征向量 $\bfp_i$;
		\item 若 $k$ 重特征值均有 $k$ 个对应的线性无关的特征向量, 则可对角化.
		\item 若能, 将 $n$ 个对应的线性无关的特征向量 $\bfp_1,\dots,\bfp_n$ 组成方阵 $\bfP=(\bfp_1,\dots,\bfp_n)$, 
		\[\bfP^{-1}\bfA\bfP=\diag(\lambda_1,\dots,\lambda_n).\]
	\end{enumerate}
	\vspace{-\baselineskip}
	\onslide<+->
	实对称阵的正交合同对角化, 或求正交变换 $\bfx=\bfP\bfy$ 将实二次型 $f$ 化为标准形的步骤:
	\begin{enumerate}
		\item 写出 $f$ 对应的对称阵 $\bfA$.
		\item 求出 $\bfA$ 的特征值.
		\item \alert{若特征值是 $k$ 重的, 求出 $k$ 个特征向量后, 用格拉姆-施密特方法将其正交单位化.}
		\item 这些特征向量构成正交阵 $\bfP$, $\bfP^\rmT\bfA\bfP$ 为这些特征向量对应的特征值构成的对角阵.
		\item 写出正交变换 $\bfx=\bfP\bfy$ 以及对应的实二次型
		\[f=\lambda_1 y_1^2+\cdots+\lambda_n y_n^2.\]
	\end{enumerate}
\end{frame}

\end{document}