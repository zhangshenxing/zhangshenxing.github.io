\section{实对称阵的正交合同}

\subsection{实二次型}

\begin{frame}{引例题: 二次曲线的分类}
	\onslide<+->
	我们知道,
	\[\frac{x^2}{a^2}+\frac{y^2}{b^2}=1,\quad
	\frac{x^2}{a^2}-\frac{y^2}{b^2}=1\]
	分别表示椭圆和双曲线.
	\onslide<+->
	对于二次曲线
	\[Ax^2+Bxy+Cy^2=1,\]
	它又表示什么图形呢?
\end{frame}


\begin{frame}{实二次型的定义}
	\onslide<+->
	\begin{definition}
		若 $n$ 元多项式 $f(x_1,\dots,x_n)$ 满足
		\[f(\lambda x_1,\dots,\lambda x_n)=\lambda^2f(x_1,\dots,x_n),\quad\forall\lambda\in\BR,\]
		则称 $f$ 为\emph{二次齐次多项式}或\emph{实二次型}.
		它可以写成
		\[f(x)=a_{11}x_1^2+a_{22}x_2^2+\cdots+a_{nn}x_n^2+2a_{12}x_1x_2+\cdots+2a_{n-1,n}a_{n-1,n}.\]
	\end{definition}
	\onslide<+->
	\alert{本课程仅讨论实二次型}.
	\onslide<+->
	根据定义, $f$ 不能包含一次项和常数项.
	\onslide<+->
	若 $f$ 的交叉项 $x_ix_j (i<j)$ 系数均为零, 则称 $f$ 为\emph{实二次型的标准形}.
\end{frame}


\begin{frame}{实二次型的矩阵形式}
	\onslide<+->
	设实二次型 $f$ 的 $x_i^2$ 项的系数为 $a_{ii}$, $x_ix_j (i<j)$ 项的系数为 $2a_{ij}$.
	\onslide<+->
	设 $a_{ji}=a_{ij}$, 对称阵 $\bfA=(a_{ij})_{n}$, 则
	\begin{align*}
		(x_1,\dots,x_n)\bfA\begin{pmatrix}
			x_1\\\vdots\\x_n
		\end{pmatrix}
		&=(\sum_{i=1}^n a_{i1}x_i,\dots,\sum_{i=1}^n a_{in}x_i)\begin{pmatrix}
			x_1\\\vdots\\x_n
		\end{pmatrix}\\
		&=\sum_{j=1}^n\Bigl(\sum_{i=1}^n a_{ij}x_i\Bigr)x_j
		=\sum_{i,j=1}^n a_{ij}x_ix_j=f(x_1,\dots,x_n),
	\end{align*}
	\onslide<+->
	即 \alert{$f=\bfx^\rmT\bfA\bfx$}, 其中 $\bfx=(x_1,\dots,x_n)^\rmT$, \alert{$\bfA$ 为实对称阵}.

	\onslide<+->
	反过来, 任给一个实对称阵 $\bfA$, 多项式 $f(\bfx)=\bfx^\rmT\bfA\bfx$ 显然满足
	\[f(\lambda \bfx)=(\lambda\bfx)^\rmT\bfA(\lambda\bfx)=\lambda^2f(\bfx),\]
	故 $f$ 是实二次型.
	\onslide<+->
	因此\alert{实二次型 $f$ 与对称阵 $\bfA$ 之间存在一一对应的关系}.
\end{frame}


\begin{frame}{例题: 实二次型的矩阵形式}
	\onslide<+->
	\begin{example}
		写出实二次型 $f(x_1,x_2,x_3)=x_1^2+4x_1x_2+4x_2^2+2x_1x_3+x_3^2+4x_2x_3$ 对应的矩阵.
	\end{example}
	\onslide<+->
	\begin{solution}
		$\bfA=\begin{pmatrix}
			1&2&1\\
			2&4&2\\
			1&2&1
		\end{pmatrix}$.
	\end{solution}
	\onslide<+->
	若 $f$ 是标准形, 则 $f$ 对应矩阵 $\bfA$ 是对角阵.
\end{frame}


\begin{frame}{正交阵}
	\onslide<+->
	\begin{definition}
		若实方阵 $\bfA$ 满足 $\bfA^\rmT\bfA=\bfE$, 则称 $\bfA$ 为\emph{正交阵}.
	\end{definition}
	\onslide<+->
	正交阵满足如下性质:
	\begin{enumerate}
		\item $\bfA=(\bma_1,\dots,\bma_n)$ 是正交阵$\iff \bma_1,\dots,\bma_n$ 是标准正交向量组.
		\item $\bfA$ 是正交阵$\iff\bfA^\rmT=\bfA^{-1}$.
		\item $\bfA$ 是正交阵$\implies|\bfA|=\pm1$ 且 $\bfA^\rmT,\bfA^{-1},\bfA^*$ 都是正交阵.
	\end{enumerate}
	\onslide<+->
	\begin{definition}
		若 $\bfP$ 为正交阵, 称线性变换 $\bfy=\bfP\bfx$ 为\emph{正交变换}.
	\end{definition}
	\onslide<+->
	例如, $\BR^2$ 上的正交变换就是绕原点的旋转、反射, 以及它们的复合.
	\onslide<+->
	由于
	\[[\bfP\bfx,\bfP\bfy]=\bfx^\rmT\bfP^\rmT\bfP\bfy=\bfx^\rmT\bfy=[\bfx,\bfy],\]
	因此\alert{正交变换保持向量的长度和夹角}.
\end{frame}


\subsection{实对称阵和实二次型的合同}
\begin{frame}{合同}
	\onslide<+->
	\begin{definition}
		\begin{enumerate}
			\item 若存在可逆线性变换 $\bfx=\bfP\bfy$ 使得实二次型 $f$ 在变量 $\bfx,\bfy$ 下的矩阵分别为 $\bfA,\bfB$, 则称矩阵 $\bfA$ \emph{合同}或\emph{相合}于 $\bfB$.
			\item 若 $\bfP$ 是正交阵, 则称矩阵 $\bfA$ \emph{正交合同}或\emph{正交相合}于 $\bfB$.
		\end{enumerate}
	\end{definition}
	\onslide<+->
	若 $\bfA$ 是对称阵, $\bfP$ 可逆, 则 $\bfP^\rmT\bfA\bfP$ 也是对称阵.
	\onslide<+->
	由
	\[f=\bfx^\rmT\bfA\bfx=\bfy^\rmT\bfP^\rmT\bfA\bfP\bfy
	=\bfy^\rmT(\bfP^\rmT\bfA\bfP)\bfy\]
	可知 $\bfA$ (正交)合同于 $\bfB$ 当且仅当存在可逆(正交)方阵 $\bfP$ 使得 $\bfB=\bfP^\rmT\bfA\bfP$.
\end{frame}


\begin{frame}{合同、等价、相似的关系}
	\onslide<+->
	\begin{proposition}
		对称方阵的(正交)合同满足
		\begin{enumerate}
			\item 自反性: $\bfA$ 与自身正交合同;
			\item 对称性: 若 $\bfA$ (正交)合同于 $\bfB$, 则 $\bfB$ (正交)合同于 $\bfA$;
			\item 传递性: 若 $\bfA$ (正交)合同于 $\bfB$, $\bfB$ (正交)合同于 $\bfC$, 则 $\bfA$ (正交)合同于 $\bfC$.
		\end{enumerate}
	\end{proposition}

	\onslide<+->
	合同、等价、相似有如下关系:
	\begin{enumerate}
		\item 若 $\bfA,\bfB$ 合同, 则 $\bfA,\bfB$ 等价, $R(\bfA)=R(\bfB)$. 反之未必.
		\item 若 $\bfA,\bfB$ 正交合同, 则 $\bfA,\bfB$ 相似. 反之, \alert{若实对称阵 $\bfA,\bfB$ 相似, 则二者正交合同}.
	\end{enumerate}
\end{frame}


\begin{frame}{实二次型的对角化}
	\onslide<+->
	\begin{theorem}
		对于实对称阵 $\bfA$, 存在正交阵 $\bfP$ 使得 $\bfP^\rmT\bfA\bfP$ 是对角阵.
		从而 $\bfA$ 对应的实二次型在线性变换 $\bfy=\bfP\bfx$ 下变为标准形.
	\end{theorem}
	\onslide<+->
	\begin{proposition}
		实对称阵的特征值一定是实数, 从而其特征向量均可取实向量.
	\end{proposition}
	\onslide<+->
	\begin{proof}
		设 $\bfA$ 是实对称阵, 非零向量 $\bfx$ 满足 $\bfA\bfx=\lambda\bfx$.
		\onslide<+->{%
			两边取转置和共轭并右乘 $\bfx$ 得到
			\[\bar\lambda\ov{\bfx}^\rmT\bfx=\ov{\bfx}^\rmT\ov{\bfA}^\rmT\bfx=\ov{\bfx}^\rmT\bfA\bfx=\lambda\ov{\bfx}^\rmT\bfx.\]
		}\onslide<+->{%
			显然 $\ov{\bfx}^\rmT\bfx=|x_1|^2+\cdots+|x_n|^2>0$, 因此 $\lambda$ 是实数.
		}\onslide<+->{%
			由于特征向量是方程 $(\bfA-\lambda\bfE)\bfx={\bf0}$ 的解, 系数矩阵是实方阵, 因此特征向量可取实向量.\qedhere
		}
	\end{proof}
\end{frame}


\begin{frame}{实二次型的对角化的证明}
	\beqskip{3pt}
	\onslide<+->
	\begin{proof}[定理的证明]
		\onslide<+->{%
			归纳证明 $\bfA$ 存在 $n$ 个两两正交的单位特征向量.
		}\onslide<+->{%
			假设我们已找到 $k<n$ 个两两正交的单位特征向量 $\bfe_1,\dots,\bfe_k$, 分别对应特征值 $\lambda_1,\dots,\lambda_k$.
		}\onslide<+->{%
			设 $V$ 是与这些向量均正交的实向量全体, 即方程 $(\bfe_1,\dots,\bfe_k)^\rmT\bfx={\bf0}$ 的解空间.
		}\onslide<+->{%
			由于系数秩为 $k$, 存在基础解系 $\bfv_1,\dots,\bfv_{n-k}\in V$.
		}\onslide<+->{%
			对任意 $i,j$,
			\[[\bfe_i,\bfA\bfv_j]=\bfe_i^\rmT\bfA\bfv_j=(\bfA\bfe_i)^\rmT\bfv_j=\lambda_i \bfe_i^\rmT\bfv_j=0\implies
			\bfA\bfv_j\in V.\]
		}\onslide<+->{%
			设 $(n-k)$ 阶矩阵 $\bfB$ 的第 $j$ 列是 $\bfA\bfv_i$ 表示为 $\bfv_1,\dots,\bfv_{n-k}$ 线性组合的系数, 即
			\[\bfA(\bfv_1,\dots,\bfv_{n-k})=(\bfv_1,\dots,\bfv_{n-k})\bfB.\]
		}\onslide<+->{%
			设非零向量 $\bfx$ 满足 $\bfB\bfx=\lambda\bfx$, 则
			\[\bfA(\bfv_1,\dots,\bfv_{n-k})\bfx=\lambda(\bfv_1,\dots,\bfv_{n-k})\bfx,\]
			即非零向量 $(\bfv_1,\dots,\bfv_{n-k})\bfx$ 是 $\bfA$ 关于 $\lambda$ 的特征向量, 于是 $\lambda\in\BR$ 且可选择 $\bfx$ 使得它是实向量, 它和 $\bfe_1,\dots,\bfe_k$ 均正交.
		}\onslide<+->{%
			令 $\bfe_{k+1}$ 为该向量的标准化.
		}\onslide<+->{%
			归纳可知 $\bfA$ 存在 $n$ 个两两正交的特征向量, 它们构成的正交阵 $\bfP=(\bfe_1,\dots,\bfe_n)$ 满足题述要求.\qedhere
		}
	\end{proof}
	\endgroup
\end{frame}


\begin{frame}{对阵矩阵的性质}
	\onslide<+->
	由于特征值 $\lambda$ 对应的实特征向量就是 $\bfP$ 中 $\lambda$ 对应的那些列向量的线性组合,
	\onslide<+->
	因此:
	\begin{corollary}
		实对称阵的不同特征值对应的实特征向量正交.
	\end{corollary}
	\onslide<+->
	\begin{exercise}
		设 $\bma_1=(1,-3,1)^\rmT,\bma_2=(1,a,2)^\rmT$ 是实实对称阵 $\bfA$ 对应特征值 $\lambda_1=1$ 和 $\lambda_2=2$ 的特征向量, 则$a=$\fillblank{\visible<+->{$1$}}.
	\end{exercise}
	\onslide<+->
	\begin{exercise}
		若 $3$ 阶实实对称阵 $\bfA$ 满足 $\bfA^2=\bfA, R(\bfA)=1$, 则 $\bfA$ 的特征值为\fillblank[2cm]{\visible<+->{$0,0,1$}}.
	\end{exercise}
\end{frame}


\begin{frame}{例题: 对阵矩阵的性质}\small
\beqskip{5pt}
	\onslide<+->
	\begin{example}
		设 $3$ 阶实对称阵 $\bfA$ 的特征值为 $6,3,3$, 与特征值 $6$ 对应的特征向量为 $\bma_1=(1,1,1)^\rmT$, 求 $\bfA$. 
	\end{example}
	\onslide<+->
	\begin{solution}
		由于有两个与特征值 $3$ 对应的线性无关特征向量, 因此与 $\bma_1$ 正交的向量都是与特征值 $3$ 对应的特征向量.
		\onslide<+->{%
			由 $\bma_1^\rmT\bfx=0$ 得 $\bfa_2=(-1,1,0)^\rmT,\bfa_3=(-1,0,1)^\rmT$.
		}\onslide<+->{%
			故
			\[\bfA=\bfP\diag(6,3,3)\bfP^{-1}
			=\begin{pmatrix}
				1&-1&-1\\
				1&1&0\\
				1&0&1
			\end{pmatrix}\begin{pmatrix}
				6&&\\
				&3&\\
				&&3
			\end{pmatrix}\begin{pmatrix}
				1&-1&-1\\
				1&1&0\\
				1&0&1
			\end{pmatrix}^{-1}=\begin{pmatrix}
				4&1&1\\
				1&4&1\\
				1&1&4
			\end{pmatrix}.\]
		}\vspace{-.5\baselineskip}
	\end{solution}
	\onslide<+->
	\begin{solution}{另解}
		注意到 $\bfA-3\bfE$ 行和为 $3$, 迹为 $3$, 可设 $\bfA-3\bfE=\begin{pmatrix}
			c&a&b\\
			a&b&c\\
			b&c&a
		\end{pmatrix}$, $a+b+c=3$. 再由 $R(\bfA-3\bfE)=1$ 可知 $a=b=c=1$.
	\end{solution}
\endgroup
\end{frame}


\begin{frame}{实二次型对角化的步骤}
	\onslide<+->
	对称阵正交实二次型的对角化, 或求正交变换 $\bfx=\bfP\bfy$ 将实二次型 $f$ 化为标准形的步骤:
	\begin{enumerate}
		\item 写出 $f$ 对应的对称阵 $\bfA$.
		\item 求出 $\bfA$ 的特征值.
		\item \alert{若特征值是 $k\ge1$ 重的, 求出 $k$ 个特征向量后, 用格拉姆-施密特方法将其正交单位化.}
		\item 这些特征向量构成正交阵 $\bfP$, $\bfP^\rmT\bfA\bfP$ 为这些特征向量对应的特征值构成的对角阵.
		\item 写出正交变换 $\bfx=\bfP\bfy$ 以及对应的实二次型
		\[f=\lambda_1 y_1^2+\cdots+\lambda_n y_n^2.\]
	\end{enumerate}
\end{frame}


\begin{frame}{典型例题: 实二次型的对角化}
	\onslide<+->
	\begin{example}
		设 $\bfA=\begin{pmatrix}
			2&0&0\\
			0&0&1\\
			0&1&x
		\end{pmatrix},\bfB=\begin{pmatrix}
			2&0&0\\
			0&y&0\\
			0&0&-1
		\end{pmatrix}$ 相似, 求 $x,y$ 以及正交阵 $\bfP$ 使得 $\bfP^{-1}\bfA\bfP=\bfB$.
	\end{example}
	\onslide<+->
	\begin{solution}
		由 $\bfA,\bfB$ 相似得 $|\bfA|=-2=|\bfB|=-2y$, $\Tr(\bfA)=2+x=\Tr(\bfB)=1+y$,
		\onslide<+->{%
			故 $x=0,y=1$.
		\begin{itemize}
			\item 对于 $\lambda_1=2$, $\bfA-2\bfE=\begin{pmatrix}
				0&0&0\\
				0&-2&1\\
				0&1&-2
			\end{pmatrix}\simr\begin{pmatrix}
				0&1&0\\
				0&0&1\\
				0&0&0
			\end{pmatrix}\implies\bma_1=\begin{pmatrix}
				1\\0\\0
			\end{pmatrix}$.
		\end{itemize}}
	\end{solution}
\end{frame}


\begin{frame}{典型例题: 实二次型的对角化}
	\onslide<+->
	\begin{solution}[续解]
		\begin{itemize}
			\item 对于 $\lambda_2=1$, $\bfA-\bfE=\begin{pmatrix}
				1&0&0\\
				0&-1&1\\
				0&1&-1
			\end{pmatrix}\simr\begin{pmatrix}
				1&0&0\\
				0&1&-1\\
				0&0&0
			\end{pmatrix}\implies\bma_2=\begin{pmatrix}
				0\\1\\1
			\end{pmatrix}$.
			\item 对于 $\lambda_3=-1$, $\bfA-\bfE=\begin{pmatrix}
				3&0&0\\
				0&1&1\\
				0&1&1
			\end{pmatrix}\simr\begin{pmatrix}
				1&0&0\\
				0&1&1\\
				0&0&0
			\end{pmatrix}\implies\bma_3=\begin{pmatrix}
				0\\1\\-1
			\end{pmatrix}$.
			\item 将特征向量单位化得到 $\bfP=\begin{pmatrix}
				1&0&0\\
				0&\dfrac1{\sqrt2}&\dfrac1{\sqrt2}\\
				0&\dfrac1{\sqrt2}&-\dfrac1{\sqrt2}
			\end{pmatrix}$.
		\end{itemize}
	\end{solution}
\end{frame}


\begin{frame}{典型例题: 实二次型的对角化}
	\onslide<+->
	\begin{example}
		求正交变换 $\bfx=\bfP \bfy$ 化 $f=4x_1^2+4x_2^2+4x_3^2+4x_1x_2+4x_2x_3+4x_3x_1$ 为标准形.
	\end{example}
	\onslide<+->
	\begin{solution}
		\begin{itemize}
			\item $f$ 对应 $\bfA=\begin{pmatrix}
				4&2&2\\
				2&4&2\\
				2&2&4
			\end{pmatrix}$.
			\onslide<+->{%
				由 $|\bfA-\lambda\bfE|=-(\lambda-2)^2(\lambda-8)$ 得到特征值 $8,2,2$.
			}
			\item 对于 $\lambda_1=8$, $\bfA-8\bfE=\begin{pmatrix}
				-4&2&2\\
				2&-4&2\\
				2&2&-4
			\end{pmatrix}\simr\begin{pmatrix}
				1&0&-1\\
				0&1&-1\\
				0&0&0
			\end{pmatrix}\implies\bma_1=\begin{pmatrix}
				1\\1\\1
			\end{pmatrix}$.
			\onslide<+->{%
				将其单位化得到 $\bfe_1=\dfrac1{\sqrt3}\begin{pmatrix}
					1\\1\\1
				\end{pmatrix}$.
			}
		\end{itemize}
	\end{solution}
\end{frame}


\begin{frame}{典型例题: 实二次型的对角化}\small
\beqskip{7pt}
	\onslide<+->
	\begin{solution}[续解]
		\begin{itemize}
			\item 对于 $\lambda_2,\lambda_3=2$, $\bfA-2\bfE=\begin{pmatrix}
				2&2&2\\
				2&2&2\\
				2&2&2
			\end{pmatrix}\simr\begin{pmatrix}
				1&1&1\\
				0&0&0\\
				0&0&0
			\end{pmatrix}\!\!\!\!\implies\!\!\bma_2=\begin{pmatrix}
				-1\\1\\0
			\end{pmatrix},\bma_3=\begin{pmatrix}
				-1\\0\\1
			\end{pmatrix}$.

			\onslide<+->{%
				将其正交单位化得到 $\bmb_2=\begin{pmatrix}
					-1\\1\\0
				\end{pmatrix},\bfe_2=\dfrac1{\sqrt2}\begin{pmatrix}
					-1\\1\\0
				\end{pmatrix}$,
				\[\bmb_3=\bma_3-\dfrac{[\bma_3,\bmb_2]}{[\bmb_2,\bmb_2]}\bmb_2
				=\bma_3-\dfrac12\bmb_2=\begin{pmatrix}
					-1/2\\-1/2\\1
				\end{pmatrix},\quad\bfe_3=\dfrac1{\sqrt6}\begin{pmatrix}
					-1\\-1\\2
				\end{pmatrix}.\]
			}
			\item 因此经过正交变换 $\bfx=${\large$\begin{pmatrix}
				\frac1{\sqrt3}&-\frac1{\sqrt2}&-\frac1{\sqrt6}\\
				\frac1{\sqrt3}&\frac1{\sqrt2}&-\frac1{\sqrt6}\\
				\frac1{\sqrt3}&0&\frac2{\sqrt6}
			\end{pmatrix}$}$\bfy$, $f$ 化为标准形 $f=8y_1^2+2y_2^2+2y_3^2$.
		\end{itemize}
		\vspace{-.5\baselineskip}
	\end{solution}
\endgroup
\end{frame}


\begin{frame}{典型例题: 实二次型的对角化}
	\onslide<+->
	\begin{example}
		设实二次型 $f=ax_1^2+x_2^2+x_3^2+4x_1x_2+4x_2x_3+4x_3x_1$ 经过正交变换 $\bfx=\bfP\bfy$ 化为 $f=5y_1^2-y_2^2-y_3^2$.
		求常数 $a$ 和正交阵 $\bfP$.
	\end{example}
	\onslide<+->
	\begin{solution}
		$f$ 对应 $\bfA=\begin{pmatrix}
			a&2&2\\
			2&1&2\\
			2&2&1
		\end{pmatrix}$, $\Tr(\bfA)=a+2=5-1-1,a=1$.

		\onslide<+->{%
			同上例可得 $\displaystyle\bfP=\begin{pmatrix}
				\dfrac1{\sqrt3}&-\dfrac1{\sqrt2}&-\dfrac1{\sqrt6}\\
				\dfrac1{\sqrt3}&\dfrac1{\sqrt2}&-\dfrac1{\sqrt6}\\
				\dfrac1{\sqrt3}&0&\dfrac2{\sqrt6}
			\end{pmatrix}$.
		}
	\end{solution}
\end{frame}


\begin{frame}{典型例题: 实二次型的对角化}
	\onslide<+->
	\begin{exercise}
		设实二次型 $f=x_1^2+2x_2^2+ax_3^2-4x_1x_2-4x_2x_3$ 经过正交变换 $\bfx=\bfP\bfy$ 化为 $f=2y_1^2+5y_2^2+by_3^2$.
		求常数 $a,b$ 和正交阵 $\bfP$.
	\end{exercise}
	\onslide<+->
	\begin{answer}
		$a=3,b=-1,\bfP=\dfrac13\begin{pmatrix}
			-2&1&2\\
			1&-2&2\\
			2&2&1
		\end{pmatrix}$.
	\end{answer}
	\onslide<+->
	\begin{example}
		设实二次型 $f=a(x_1^2+x_2^2+x_3^2)+4x_1x_2+4x_2x_3+4x_3x_1$ 经正交变换化成标准形 $f(y_1,y_2,y_3)=6y_1^2$, 则 $a=$\fillblank{\visible<+->{2}}.
	\end{example}
\end{frame}
