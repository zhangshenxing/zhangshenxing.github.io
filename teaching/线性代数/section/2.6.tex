\section{线性方程组}

\subsection{齐次线性方程组解的存在性}

\begin{frame}{线性方程组}
	\onslide<+->
	\emph{线性方程组}是指
	\[\laeq[lclcclcl]{
		a_{11}&x_1+{}&a_{12}&x_2&{}+\cdots+{}&a_{1n}&x_n={}&b_1\\
		a_{21}&x_1+{}&a_{22}&x_2&{}+\cdots+{}&a_{2n}&x_n={}&b_2\\
		&&&&\vdots&&&\\
		a_{m1}&x_1+{}&a_{m2}&x_2&{}+\cdots+{}&a_{mn}&x_n={}&b_m
	}\]
	\onslide<+->
	它的系数形成了一个 $m\times n$ 矩阵 $\bfA$, 称为\emph{系数矩阵}.
	\onslide<+->
	系数和常数项一起形成了一个 $m\times(n+1)$ 矩阵 $(\bfA,\bfb)$, 称为\emph{增广矩阵}.

	\onslide<+->
	线性方程组等价于
	\[\bfA\bfx=\bfb,\]
	其中
	\[\bfx=(x_1,\dots,x_n)^\rmT.\]
\end{frame}


\begin{frame}{齐次线性方程组非零解的判定}
	\onslide<+->
	当 $\bfb={\bf0}$ 为零向量时, 称该线性方程组为\emph{齐次}的; 否则称为\emph{非齐次}的.
	\onslide<+->
	齐次线性方程组总有解 $\bfx={\bf0}$.
	\onslide<+->
	$\bfA\bfx={\bf0}$ 有非零解$\iff\bfA$ 的列向量线性相关
	\onslide<+->
	$\iff R(\bfA)<n$.
	\onslide<+->
	\begin{theorem}
		\begin{enumerate}
			\item $\bfA_{m\times n}\bfx={\bf0}$ 有(无穷多)非零解$\iff R(\bfA)<n$;
			\item $\bfA_{m\times n}\bfx={\bf0}$ 只有零解$\iff R(\bfA)=n$.
		\end{enumerate}
	\end{theorem}
	\onslide<+->
	\begin{corollary}
		设 $\bfA$ 是 $n$ 阶方阵.
		\begin{enumerate}
			\item $\bfA\bfx={\bf0}$ 有(无穷多)非零解$\iff |\bfA|=0$;
			\item $\bfA\bfx={\bf0}$ 只有零解$\iff |\bfA|\neq0$.
		\end{enumerate}
	\end{corollary}
	\onslide<+->
	\begin{corollary}
		若方程个数小于未知元个数, 则齐次线性方程组有非零解.
	\end{corollary}
\end{frame}


\begin{frame}{例题: 齐次线性方程组非零解的判定}
	\onslide<+->
	\begin{example}
		假设
		\[\laeq[cccccccc]{
			 &x_1&{}+{}&2&x_2&{}-{}&2&x_3=0\\
			4&x_1&{}+{}&a&x_2&{}+{}&3&x_3=0\\
			3&x_1&{}-{}& &x_2&{}+{}& &x_3=0
		}\]
		有非零解, 求 $a$.
	\end{example}
	\onslide<+->
	\begin{solution}
		此时系数矩阵行列式为零:
		\[0=\begin{vmatrix}
			1&2&-2\\
			4&a&3\\
			3&-1&1
		\end{vmatrix}=7a+21,\quad
		\visible<+->{a=-3.}\]
		\vspace{-\baselineskip}
	\end{solution}
\end{frame}


\begin{frame}{例题: 齐次线性方程组非零解的判定}
	\onslide<+->
	\begin{example}
		若下述方程有非零解, 求 $a$.
		\[\laeq[rcrcrc]{
			x_1&{}+{}&x_2&{}+{}&ax_3&{}=0\\
			-x_1&{}+{}&(a-1)x_2&{}+{}&(1-a)x_3&{}=0\\
			x_1&{}+{}&x_2&{}+{}&a^2x_3&{}=0\\
			x_1&{}+{}&x_2&{}+{}&(2a+1)x_3&{}=0
		}\]
	\end{example}
	\onslide<+->
	\begin{solution}
		\[\begin{pmatrix}
			1&1&a\\
			-1&a-1&1-a\\
			1&1&a^2\\
			1&1&2a+1
		\end{pmatrix}
		\visible<+->{\simr\begin{pmatrix}
			1&1&a\\
			0&a&1\\
			0&0&1\\
			0&0&0
		\end{pmatrix}}\]
		\onslide<+->{
			的秩小于 $3$, 因此 $a=0$.
		}
	\end{solution}
\end{frame}


\subsection{齐次线性方程组解的结构}

\begin{frame}{基础解系}
\beqskip{0pt}
	\onslide<+->
	\begin{definition}
		称空间 $\{\bfx\mid \bfA\bfx={\bf0}\}$ 的一组基为该齐次线性方程组的\emph{基础解系}.
	\end{definition}
	\onslide<+->
	\begin{theorem}
		设 $\bfA\in M_{m\times n}, R(\bfA)=r$.
		线性方程组 $\bfA\bfx={\bf0}$ 的基础解系包含 $n-r$ 个向量.
	\end{theorem}
	\onslide<+->
	\begin{solution}[证明]
		通过交换未知元的位置(相当于交换 $\bfA$ 列的位置), 不妨设 $\bfA$ 可化为行最简形
		\vspace{-.1\baselineskip}
		\[\begin{pNiceMatrix}
			1&\cdots&0&b_{11}&\cdots&b_{1,n-r}\\
			\vdots&\ddots&\vdots&\vdots&\ddots&\vdots\\
			0&\cdots&1&b_{r,1}&\cdots&b_{r,n-r}\\
			0&\cdots&0&0&\cdots&0\\
			\vdots&&\vdots&\vdots&&\vdots\\
			0&\cdots&0&0&\cdots&0
			\CodeAfter
			\tikz \draw[cstdash,main] (4-|1) -- (4-|7);
			\tikz \draw[cstdash,main] (1-|4) -- (7-|4);
		\end{pNiceMatrix}=\begin{pmatrix}
			\bfE_r&\bfB\\
			\bfO&\bfO
		\end{pmatrix}.\]
		\vspace{-.6\baselineskip}
	\end{solution}
\endgroup
\end{frame}


\begin{frame}{基础解系}
	\onslide<+->
	\begin{proof}[续证]
		方程化为 $(\bfE_r,\bfB)\bfx={\bf0}$,
		\onslide<+->{%
		即
		\[\begin{pmatrix}
			x_1\\\vdots\\x_r
		\end{pmatrix}=-\bfB\begin{pmatrix}
			x_{r+1}\\\vdots\\x_n
		\end{pmatrix},\quad
		\bfx=\begin{pmatrix}
			-\bfB\\\bfE_{n-r}
		\end{pmatrix}\begin{pmatrix}
			x_{r+1}\\\vdots\\x_n
		\end{pmatrix}.\]
		}\onslide<+->{%
			于是 $\bfC:=\begin{pmatrix}
				-\bfB\\\bfE_{n-r}
			\end{pmatrix}$ 的 $n-r$ 个列向量生成了整个解空间.
		}\onslide<+->{%
			由于 $R(\bfC)\ge R(\bfE_{n-r})=n-r$, $\bfC$ 列满秩, 因此它的列向量就是一组基础解系.\qedhere
		}
	\end{proof}
	\onslide<+->
	\begin{corollary}
		$\bfA\bfx={\bf0}$ 任意 $n-r$ 个线性无关的解都是一组基础解系.
	\end{corollary}
\end{frame}


\begin{frame}{求基础解系的步骤}
	\onslide<+->
	解齐次线性方程组的步骤:
	\begin{enumerate}
		\item 将系数矩阵通过初等行变换化为行最简形.
		\item 去掉零行, 并取负矩阵, 得到 $r\times n$ 矩阵.
		\item 添加 $n-r$ 行 $e_j^\rmT$, 使得对角元全都变成 $\pm1$,
		其中 $1$ 对应的是原来的非零行的第一个 $1$, 得到 $n\times n$ 矩阵.
		\item 去掉对角元是 $1$ 对应的列, 得到 $n\times (n-r)$ 矩阵.
		\item 这个矩阵的列向量就是一组基础解系.
	\end{enumerate}
\end{frame}


\begin{frame}{典型例题: 求基础解系}
	\onslide<+->
	\begin{example}
		解方程 $\laeq[rrrrc]{
			x_1  &{}+2x_2 &{}+4x_3 &{}+x_4&{}=0\\
			2x_1 &{}+4x_2 &{}-2x_3 &{}-x_4&{}=0\\
			3x_1 &{}+6x_2 &{}+2x_3 &      &{}=0
		}$.
	\end{example}
	\onslide<+->
	\begin{solution}
		\[\begin{pmatrix}
			1&2&4&1\\
			2&4&-2&-1\\
			3&6&2&0
		\end{pmatrix}\simr\begin{pmatrix}
			1&2&4&1\\
			0&0&-10&3\\
			0&0&-10&3
		\end{pmatrix}\simr\begin{pmatrix}
			1&2&0&-1/5\\
			0&0&1&3/10\\
			0&0&0&0
		\end{pmatrix}.\]
	\end{solution}
\end{frame}


\begin{frame}{典型例题: 求基础解系}
	\onslide<+->
	\begin{solution}[续解]
		\[\begin{pmatrix}
			-1&-2&0&1/5\\
			\alert{0}&\alert{1}&\alert{0}&\alert{0}\\
			0&0&-1&-3/10\\
			\alert{0}&\alert0&\alert0&\alert1
		\end{pmatrix}\implies\begin{pmatrix}
			-2&1/5\\
			\alert{1}&\alert{0}\\
			0&-3/10\\
			\alert0&\alert1
		\end{pmatrix}.\]
		\onslide<+->{%
			通解为
			\[\begin{pmatrix}
				x_1\\x_2\\x_3\\x_4
			\end{pmatrix}=k_1\begin{pmatrix}
				-2\\1\\0\\0
			\end{pmatrix}+k_2\begin{pmatrix}
				1/5\\0\\-3/10\\1
			\end{pmatrix},\quad k_1,k_2\text{为任意常数}.\]
		}
	\end{solution}
\end{frame}


\begin{frame}{典型例题: 求基础解系}
	\onslide<+->
	\begin{exercise}
		解方程 $\bfA\bfx={\bf0}$, 其中 $\bfA=\begin{pmatrix}
			1&2&1&1&1\\
			2&4&3&1&1\\
			-1&-2&1&3&-3\\
			0&0&2&5&-2
		\end{pmatrix}$.
	\end{exercise}
	\onslide<+->
	\begin{answer}
		\[\bfA\simr\begin{pmatrix}
			1&2&0&0&2\\
			0&0&1&0&-1\\
			0&0&0&1&0\\
			0&0&0&0&0
		\end{pmatrix}\implies\begin{pmatrix}
			-1&-2&0&0&-2\\
			\alert0&\alert1&\alert0&\alert0&\alert0\\
			0&0&-1&0&1\\
			0&0&0&-1&0\\
			\alert0&\alert0&\alert0&\alert0&\alert1
		\end{pmatrix}\implies\begin{pmatrix}
			-2&-2\\
			1&0\\
			0&1\\
			0&0\\
			0&1
		\end{pmatrix}.\]
	\end{answer}
\end{frame}


\begin{frame}{例题: 基础解系}
	\onslide<+->
	\begin{example}
		设 $\bfA\in M_{m\times n}, R(\bfA)=n-3$, $\bmx_1,\bmx_2,\bmx_3$ 为 $\bfA\bfx={\bf0}$ 的三个线性无关的解.
		则\fillbrace{\visible<+->{B}}是该方程的一组基础解系.
		\xx{$\bmx_1,-\bmx_2,\bmx_2-\bmx_3,\bmx_3-\bmx_1$}%
			{$\bmx_1,\bmx_1+\bmx_2,\bmx_1+\bmx_2+\bmx_3$}%
			{$\bmx_1,\bmx_2$}%
			{$\bmx_1,\bmx_1-\bmx_2-\bmx_3,\bmx_1+\bmx_2+\bmx_3$}
	\end{example}
	\onslide<+->
	\begin{example}
		设 $\bmx_1,\bmx_2,\bmx_3$ 是 $\bfA\bfx={\bf0}$ 的一组基础解系, 则\fillbrace{\visible<+->{D}}也是该方程的一组基础解系.
		\xx{与 $\bmx_1,\bmx_2,\bmx_3$ 等价的一组向量}%
			{与 $\bmx_1,\bmx_2,\bmx_3$ 同秩的一组向量}%
			{$\bmx_1-\bmx_2,\bmx_2-\bmx_3,\bmx_3-\bmx_1$}%
			{$\bmx_1+\bmx_2,\bmx_2+\bmx_3,\bmx_3+\bmx_1$}%
	\end{example}
\end{frame}


\begin{frame}{例题: 基础解系的应用}
	\onslide<+->
	\begin{example}
		设 $\bfA$ 是 $n$ 阶方阵, $R(\bfA)=n-1$ 且每行元素之和为 $0$.
		则齐次线性方程组 $\bfA\bfx={\bf0}$ 的解为\fillblank[6cm]{\visible<+->{$k(1,1,\dots,1)^\rmT,k$ 为任意常数}}.
	\end{example}
	\onslide<+->
	\begin{example}
		设 $\bfA_{m\times n}\bfB_{n\times s}=\bfO$, 证明 $R(\bfA)+R(\bfB)\le n$.
	\end{example}
	\onslide<+->
	\begin{proof}
		由于 $\bfB$ 的列向量都是 $\bfA\bfx={\bf0}$ 的解, 因此 $R(\bfB)$ 不超过该方程解空间的维数, 即 $n-R(\bfA)$.
	\end{proof}
\end{frame}


\begin{frame}{例题: 基础解系}
	\onslide<+->
	\begin{example}
		设 $\bfA$ 是实矩阵, 证明 $R(\bfA^\rmT\bfA)=R(\bfA)$.
	\end{example}
	\onslide<+->
	\begin{proof}
		若 $\bfA^\rmT\bfA\bfx={\bf0}$, 则
		\[0=\bfx^\rmT\bfA^\rmT\bfA\bfx=(\bfA\bfx)^\rmT\bfA\bfx.\]
		\onslide<+->{%
			设 $\bfA\bfx=(y_1,\dots,y_n)^\rmT$, 则右侧为 $y_1^2+\cdots+y_n^2=0$, 这迫使 $y_1=\cdots=y_n=0$, 于是 $\bfA\bfx=0$.
		}\onslide<+->{%
			所以 $\bfA^\rmT\bfA\bfx={\bf0}\iff\bfA\bfx={\bf0}$.
		}\onslide<+->{%
			二者列数相同, 因此二者秩相同.
		}
	\end{proof}
	\onslide<+->
	注意, 对于复矩阵这并不成立, 例如
	\[\bfA=\begin{pmatrix}
		1\\i
	\end{pmatrix}, \quad \bfA^\rmT\bfA=0.\]
	\onslide<+->
	此时有 $R\Bigl(\ov{\bfA}^\rmT\bfA\Bigr)=R(\bfA)$, 其中 $\ov{\bfA}$ 表示所有元素取共轭.
\end{frame}


\begin{frame}{例题: 基础解系}
	\onslide<+->
	\begin{example}
		设 $n$ 阶方阵 $\bfA$ 列向量的一个极大线性无关组为 $\bma_1,\dots,\bma_{n-1}$.
		则 $\bfA^*\bfx={\bf0}$ 的解为\fillblank[10cm]{\visible<+->{$k_1\bma_1+\cdots+k_{n-1}\bma_{n-1},k_1,\dots,k_{n-1}$ 为任意常数}}.
	\end{example}
	\onslide<+->
	\begin{example}
		设 $n$ 阶方阵 $\bfA$ 满足 $R(\bfA)=n-1$, 代数余子式 $A_{11}\neq 0$.
		则 $\bfA\bfx={\bf0}$ 的解为\fillblank[10cm]{\visible<+->{$k(A_{11},\dots,A_{1n})^\rmT,k_1,\dots,k_{n-1}$ 为任意常数}}.
	\end{example}
	\onslide<+->
	\begin{exercise}
		若 $\bfA=\begin{pmatrix}
			a&1&a^2\\
			1&a&1\\
			1&1&a
		\end{pmatrix}$ 且存在 $3$ 阶非零矩阵 $\bfB$ 使得 $\bfA\bfB=\bfO$, 则\fillbrace{\visible<+->{A}}.
		\xx{$a=1,|\bfB|=0$}%
			{$a=-2,|\bfB|=0$}%
			{$a=1,|\bfB|\neq0$}%
			{$a=-2,|\bfB|\neq0$}
	\end{exercise}
\end{frame}


\begin{frame}{例题: 基础解系}
	\onslide<+->
	\begin{exercise}
		若 $\bfA=(\bma_1,\bma_2,\bma_3,\bma_4,\bma_5)$ 且 $\bfA\bfx={\bf0}$ 的解为 $k_1(1,0,-1,0,1)^\rmT+k_2(1,0,0,1,-1)^\rmT$, 则 $\bfA$ 列向量组的一个极大无关组是\fillbrace{\visible<+->{D}}.
		\xx{$\bma_1,\bma_3,\bma_5$}%
			{$\bma_1,\bma_3,\bma_4$}%
			{$\bma_3,\bma_4,\bma_5$}%
			{$\bma_2,\bma_3,\bma_5$}
	\end{exercise}
\end{frame}


\subsection{非齐次线性方程组}

\begin{frame}{非齐次线性方程组解的存在性}
	\onslide<+->
	设 $\bfA\in M_{m\times n}$.
	\onslide<+->
	对于非齐次线性方程组 $\bfA\bfx=\bfb$,
	\onslide<+->
	若方程有解, 则 $\bfb$ 可以由 $\bfA$ 的列向量线性表示, 从而 $\bfA$ 的列向量组和 $(\bfA,\bfb)$ 的列向量组等价.
	\onslide<+->
	因此 $R(\bfA)=R(\bfA,\bfb)$.

	\onslide<+->
	注意到 $\bfA$ 列向量生成的空间 $V$ 是 $(\bfA,\bfb)$ 列向量生成的空间 $W$ 的子空间.
	\onslide<+->
	若 $R(\bfA)=R(\bfA,\bfb)$, 则 $V=W$, $\bfA$ 列向量组的一个极大无关组 $S$ 也是 $(\bfA,\bfb)$ 的极大无关组.
	\onslide<+->
	从而 $\bfb$ 是 $S$ 的线性组合, 也是 $\bfA$ 列向量的线性组合.

	\onslide<+->
	\begin{theorem}
		$\bfA \bfx=\bfb$ 有解 $\iff R(\bfA)=R(\bfA,\bfb)$.
	\end{theorem}
	\onslide<+->
	\begin{corollary}
		若 $R(\bfA_{m\times n})=m$ (即 $\bfA$ 行满秩), 则 $\bfA \bfx=\bfb$ 总有解.
	\end{corollary}
\end{frame}


\begin{frame}{非齐次线性方程组解的结构}
	\onslide<+->
	若非齐次线性方程组 $\bfA\bfx=\bfb$ 有解 $\bfx=\bfx_0$,
	则 $\bfA(\bfx-\bfx_0)={\bf0}$.
	\onslide<+->
	从而 $\bfx-\bfx_0$ 是 $\bfA\bfx={\bf0}$ 的解.
	\onslide<+->
	设 $\bmx_1,\dots,\bmx_{n-r}$ 为 $\bfA\bfx={\bf0}$ 的一组基础解系, 则 $\bfA\bfx={\bf0}$ 的通解为
	\[\bfx=\bfx_0+k_1\bmx_1+\cdots+k_{n-r}\bmx_{n-r},\]
	$k_1,\dots,k_{n-r}$ 为任意常数.

	\onslide<+->
	\begin{theorem}
		\begin{enumerate}
			\item 若 $R(\bfA)<R(\bfA,\bfb)$, 则 $\bfA \bfx=\bfb$ 无解;
			\item 若 $R(\bfA)=R(\bfA,\bfb)=n$, 则 $\bfA \bfx=\bfb$ 有唯一解;
			\item 若 $R(\bfA)=R(\bfA,\bfb)<n$, 则 $\bfA \bfx=\bfb$ 有无穷多解.
		\end{enumerate}
	\end{theorem}
	\onslide<+->
	\begin{corollary}
		若 $\bfA$ 是 $n$ 阶方阵, 则 $\bfA\bfx=\bfb$ 有唯一解$\iff|\bfA|\neq0$.
	\end{corollary}
	\onslide<+->
	若 $|\bfA|=0$, 则 $\bfA\bfx=\bfb$ 无解或有无穷多解.
\end{frame}


\begin{frame}{求解非齐次线性方程组的步骤}
	\onslide<+->
	解非齐次线性方程组的步骤:
	\begin{enumerate}
		\item 写: 写出方程组对应的增广矩阵 $(\bfA,\bfb)$;
		\item 变: 通过初等行变换将其化为行最简形;
		\item 判: 通过行最简形判定方程是否有解;
		\item 解: 若系数矩阵部分零行对应的常数项均为零, 则方程有解.
		其中特解为每个非零行对应未知元取对应常数项值, 其余取零.
		\item 通解$=$特解$+$对应的齐次方程的基础解系的线性组合.
	\end{enumerate}
\end{frame}


\begin{frame}{典型例题: 解非齐次线性方程组}
	\onslide<+->
	\begin{example}
		解方程 $\laeq[ccccccccccc]{
			 &x_1&{}+{}&2&x_2+{}&3&x_3&{}+{}&4&x_4={}&1\\
			2&x_1&{}-{}& &x_2+{}&2&x_3&{}-{}&2&x_4={}&3\\
			3&x_1&{}+{}& &x_2+{}&5&x_3&{}+{}&2&x_4={}&2
		}$
	\end{example}
	\onslide<+->
	\begin{solution}
		\[\begin{pNiceMatrix}
			1&2&3&4&1\\
			2&-1&2&-2&3\\
			3&1&5&2&2
			\augdash{4}{5}
		\end{pNiceMatrix}\simr\begin{pNiceMatrix}
			1&2&3&4&1\\
			0&-5&-4&-10&1\\
			0&-5&-4&-10&-1
			\augdash{4}{5}
		\end{pNiceMatrix}\simr\begin{pNiceMatrix}
			1&2&3&4&1\\
			0&-5&-4&-10&1\\
			0&0&0&0&1
			\augdash{4}{5}
		\end{pNiceMatrix}.\]
		\onslide<+->{%
			于是 $R(\bfA)=2<R(\bfA,\bfb)=3$, 无解.
		}
	\end{solution}
\end{frame}


\begin{frame}{典型例题: 解非齐次线性方程组}
	\onslide<+->
	\begin{example}
		解方程 $\laeq[cccccccc]{
			x_1-x_2&{}-{}& &x_3&{}+{}& &x_4={}&0\\
			x_1-x_2&{}+{}& &x_3&{}-{}&3&x_4={}&1\\
			x_1-x_2&{}-{}&2&x_3&{}+{}&3&x_4={}&-1/2
		}$
	\end{example}
	\onslide<+->
	\begin{solution}
		\[\begin{pNiceMatrix}
			1&-1&-1&1&0\\
			1&-1&1&-3&1\\
			1&-1&-2&3&-1/2
			\augdash{4}{5}
		\end{pNiceMatrix}\simr\begin{pNiceMatrix}
			1&-1&0&-1&1/2\\
			0&0&1&-2&1/2\\
			0&0&0&0&0
			\augdash{4}{5}
		\end{pNiceMatrix}.\]
		\onslide<+->{%
			于是 $R(\bfA)=2=R(\bfA,\bfb)=2$, 有解.
		}\onslide<+->{%
			特解为 $(1/2,0,1/2,0)^\rmT$.
		}
	\end{solution}
\end{frame}


\begin{frame}{典型例题: 解非齐次线性方程组}
	\onslide<+->
	\begin{solution}[续解]
		\[\begin{pmatrix}
			-1&1&0&1\\
			\alert0&\alert1&\alert0&\alert0\\
			0&0&-1&2\\
			\alert0&\alert0&\alert0&\alert1
		\end{pmatrix}\implies\begin{pmatrix}
			1&1\\
			1&0\\
			0&2\\
			0&1
		\end{pmatrix}\implies\text{基础解系}\ \bmx_1=\begin{pmatrix}
			1\\1\\0\\0
		\end{pmatrix},\quad
		\bmx_2=\begin{pmatrix}
			1\\0\\2\\1
		\end{pmatrix},\]
		\onslide<+->{%
			通解为
			\[\bfx=\begin{pmatrix}
				1/2\\0\\1/2\\0
			\end{pmatrix}+k_1\begin{pmatrix}
				1\\1\\0\\0
			\end{pmatrix}+k_2\begin{pmatrix}
				1\\0\\2\\1
			\end{pmatrix},\]
			$k_1,k_2$ 为任意常数.
		}
	\end{solution}
\end{frame}


\begin{frame}{典型例题: 解非齐次线性方程组}
	\onslide<+->
	\begin{example}
		已知
		\[\bma_1=(1,4,0,2)^\rmT,\bma_2=(2,7,1,3)^\rmT,
		\bma_3=(0,1,-1,a)^\rmT,\bmb=(3,10,b,4)^\rmT.\]
		问 $a,b$ 为何值时,
		\begin{enumerate}
			\item $\bmb$ 不能由 $\bma_1,\bma_2,\bma_3$ 线性表示;
			\item $\bmb$ 能由 $\bma_1,\bma_2,\bma_3$ 唯一线性表示;
			\item $\bmb$ 能由 $\bma_1,\bma_2,\bma_3$ 不唯一线性表示.
		\end{enumerate}
	\end{example}
\end{frame}


\begin{frame}{典型例题: 线性方程组的性质}
	\onslide<+->
	\begin{solution}
		即问 $\bfA\bfx=\bfb$ 的解的情况, 其中 $\bfA=(\bma_1,\bma_2,\bma_3)$.
		\onslide<+->{%
			\[(\bfA,\bfb)=\begin{pNiceMatrix}
				1&2&0&3\\
				4&7&1&10\\
				0&1&-1&b\\
				2&3&a&4
				\augdash54
			\end{pNiceMatrix}\simr\begin{pNiceMatrix}
				1&2&0&3\\
				0&-1&1&-2\\
				0&1&-1&b\\
				0&-1&a&-2
				\augdash54
			\end{pNiceMatrix}\simr\begin{pNiceMatrix}
				1&2&0&3\\
				0&1&-1&2\\
				0&0&a-1&0\\
				0&0&0&b-2
				\augdash54
			\end{pNiceMatrix}\]
		}\onslide<+->{
			于是可知 $R(\bfA)$ 和 $R(\bfA,\bfb)$, 故
			\begin{enumerate}
				\item $b\neq 2$ 时, $\bmb$ 不能由 $\bma_1,\bma_2,\bma_3$ 线性表示;
				\item $a\neq1, b=2$ 时, $\bmb$ 能由 $\bma_1,\bma_2,\bma_3$ 唯一线性表示;
				\item $a=1,b=2$ 时, $\bmb$ 能由 $\bma_1,\bma_2,\bma_3$ 不唯一线性表示.
			\end{enumerate}
		}
	\end{solution}
\end{frame}


\begin{frame}{例题: 线性方程组解的性质}
	\onslide<+->
	\begin{example}
		设 $\bfA\in M_{m\times n}$, 则\fillbrace{\visible<+->{D}}.
		\xx{若 $\bfA\bfx={\bf0}$ 仅有零解, 则 $\bfA\bfx=\bfb$ 有唯一解}%
		{若 $\bfA\bfx={\bf0}$ 有非零解, 则 $\bfA\bfx=\bfb$ 有无穷多解}%
		{若 $\bfA\bfx={\bfb}$ 有无穷多解, 则 $\bfA\bfx={\bf0}$ 只有零解}%
		{若 $\bfA\bfx={\bfb}$ 有无穷多解, 则 $\bfA\bfx={\bf0}$ 有非零解}
	\end{example}
	\onslide<+->
	\begin{example}
		设 $\bfA\in M_{m\times n}, R(\bfA)=m<n$, 则\fillbrace{\visible<+->{C}}.
		\xx{$\bfA$ 的任意 $m$ 个列向量线性无关}%
		{$\bfA$ 的任意一个 $m$ 阶子式不等于 $0$}%
		{$\bfA\bfx={\bfb}$ 一定有无穷多个解}%
		{$\bfA$ 经过初等行变换可化为 $(\bfE,\bfO)$ 的形式}
	\end{example}
\end{frame}


\begin{frame}{典型例题: 解非齐次线性方程组}
	\onslide<+->
	\begin{example}
		$a$ 为何值时, 以下方程\enumnum1有唯一解; \enumnum2无解; \enumnum3有无穷多解? 并在有无穷多解时求其通解.
		\[\laeq[rrrc]{
			(1+a)x_1+{}&x_2+{}&x_3={}&0\\
			x_1+{}&(1+a)x_2+{}&x_3={}&3\\
			x_1+{}&x_2+{}&(1+a)x_3={}&a
		}\]
	\end{example}
	\onslide<+->
	注意处理带未知数的矩阵时, 不宜实施 $\frac{1}{a+1}r_2,(a-2)r_3$ 等类似操作, 因为其分母或系数可能为零.
	\onslide<+->
	\begin{solution}
		\[(\bfA,\bfb)=\begin{pNiceMatrix}
			1+a&1&1&0\\
			1&1+a&1&3\\
			1&1&1+a&a
			\augdash44
		\end{pNiceMatrix}\simr\begin{pNiceMatrix}
			1&1+a&1&3\\
			0&-a&a&a-3\\
			0&a&a^2+2a&a^2+a
			\augdash44
		\end{pNiceMatrix}\]
	\end{solution}
\end{frame}


\begin{frame}{典型例题: 解非齐次线性方程组}
	\onslide<+->
	\begin{solution}[续解]
		\[\simr\begin{pNiceMatrix}
			1&1+a&1&3\\
			0&a&-a&3-a\\
			0&0&a^2+3a&a^2+2a-3
			\augdash44
		\end{pNiceMatrix}.\]
		\begin{enumerate}
			\item 若 $a\neq 0,-3$, 则 $R(\bfA)=R(\bfA,\bfb)=3$, 方程有唯一解.
			\item 若 $a=0$, 则 $(\bfA,\bfb)\simr\begin{pNiceMatrix}
				1&1&1&3\\
				0&0&0&3\\
				0&0&0&0
				\augdash44
			\end{pNiceMatrix},R(\bfA)=1<R(\bfA,\bfb)=2$, 方程无解.
		\end{enumerate}
	\end{solution}
\end{frame}


\begin{frame}{典型例题: 解非齐次线性方程组}
	\onslide<+->
	\begin{solution}[续解]
		\begin{enumerate}
			\setcounter{enumi}{2}
			\item 若 $a=-3$, 则 $(\bfA,\bfb)\simr\begin{pNiceMatrix}
				1&0&-1&-1\\
				0&1&-1&-2\\
				0&0&0&0
				\augdash44
			\end{pNiceMatrix},R(\bfA)=R(\bfA,\bfb)=2$, 方程有无穷多解.
			\onslide<+->{%
			特解为 $\begin{pmatrix}
				-1\\-2\\0
			\end{pmatrix}$, 基础解系为 $\begin{pmatrix}
				1\\1\\1
			\end{pmatrix}$, 通解为
			\[\bfx=\begin{pmatrix}
				-1\\-2\\0
			\end{pmatrix}+k\begin{pmatrix}
				1\\1\\1
			\end{pmatrix},\]
			$k$ 为任意常数.}
		\end{enumerate}
	\end{solution}
	\onslide<+->
	由于系数矩阵为 $3$ 阶方阵, 也可以先通过 $|\bfA|\neq 0$ 得到唯一解情形.
\end{frame}


\begin{frame}{典型例题: 解非齐次线性方程组}
	\onslide<+->
	\begin{exercise}
		$a,b$ 为何值时, 以下方程\enumnum1有唯一解; \enumnum2无解; \enumnum3有无穷多解? 并在有无穷多解时求其通解.
		\[\laeq[cccccccccc]{
			 &x_1+{}& &x_2&{}+{}&     &x_3+{}&     &x_4={}&1\\
			 &      & &x_2&{}-{}&     &x_3+{}&2    &x_4={}&1\\
			2&x_1+{}&3&x_2&{}+{}&(a+2)&x_3+{}&4    &x_4={}&b+3\\
			3&x_1+{}&5&x_2&{}+{}&     &x_3+{}&(a+8)&x_4={}&5
		}\]
	\end{exercise}
\end{frame}


\begin{frame}{典型例题: 解非齐次线性方程组}
	\onslide<+->
	\begin{answer}
		\[\begin{pNiceMatrix}
			1&1&1&1&1\\
			0&1&-1&2&1\\
			2&3&a+2&4&b+3\\
			3&5&1&a+8&5
			\augdash55
		\end{pNiceMatrix}\simr\begin{pNiceMatrix}
			1&0&2&-1&0\\
			0&1&-1&2&1\\
			0&0&a+1&0&b\\
			0&0&0&a+1&0
			\augdash55
		\end{pNiceMatrix}.\]
		\begin{enumerate}
			\item $a\neq-1$ 时有唯一解;
			\item $a=-1,b\neq0$ 时无解;
			\item $a=-1,b=0$ 时有无穷多解, 通解为
			\[\bfx=\begin{pmatrix}
				0\\1\\0\\0
			\end{pmatrix}+k_1\begin{pmatrix}
				-2\\1\\1\\0
			\end{pmatrix}+k_2\begin{pmatrix}
				1\\-2\\0\\1
			\end{pmatrix}.\]
			\vspace{-\baselineskip}
		\end{enumerate}
	\end{answer}
\end{frame}


\begin{frame}{例题: 线性方程组解的性质}
	\onslide<+->
	\begin{example}
		设四元非齐次线性方程组 $\bfA\bfx=\bfb$ 的系数矩阵 $\bfA$ 的
		秩为 $3$. 已知 $\bmet_1,\bmet_2,\bmet_3$ 是它的三个解向量, 且
		\[\bmet_1=(2,3,4,5)^\rmT,\quad \bmet_2+\bmet_3=(1,2,3,4)^\rmT.\]
		求 $\bfA\bfx=\bfb$ 的通解.
	\end{example}
	\onslide<+->
	\begin{solution}
		由于 $R(\bfA)=3$, 因此 $\bfA\bfx={\bf0}$ 的基础解系只包含一个向量.
		\onslide<+->{%
			根据解的性质,
			\[2\bmet_1-(\bmet_2+\bmet_3)=(3,4,5,6)^\rmT\]
			是 $\bfA\bfx={\bf0}$ 的一个解, 因此这是它的一个基础解系.
		}\onslide<+->{%
			故 $\bfA\bfx=\bfb$ 的通解为
			\[\bfx=\bmet_1+k(3,4,5,6)^\rmT
			=(2,3,4,5)^\rmT+k(3,4,5,6)^\rmT.\]
		}\vspace{-\baselineskip}
	\end{solution}
\end{frame}


\begin{frame}{例题: 线性方程组解的性质}
	\onslide<+->
	\begin{example}
		已知 $4$ 阶方阵 $\bfA=(\bma_1,\bma_2,\bma_3,\bma_4)$, 且 $\bma_2,\bma_3,\bma_4$ 线性无关, $\bma_1=2\bma_2-\bma_3$.
		若 $\bmb=\bma_1+\bma_2+\bma_3+\bma_4$, 求 $\bfA\bfx=\bmb$ 的通解.
	\end{example}
	\onslide<+->
	\begin{solution}
		由题设可知 $R(\bfA)=3$, 因此 $\bfA\bfx={\bf0}$ 的基础解系只包含一个向量.
		\onslide<+->{%
			由 $\bma_1=2\bma_2-\bma_3$ 可知 $(1,-2,1,0)^\rmT$ 是 $\bfA\bfx={\bf0}$ 的一个解, 因此这是它的一个基础解系.
		}\onslide<+->{%
			注意到 $(1,1,1,1)^\rmT$ 是 $\bfA\bfx=\bfb$ 的一个特解, 故通解为
			\[\bfx=(1,1,1,1)^\rmT+k(1,-2,1,0)^\rmT.\]
		}\vspace{-\baselineskip}
	\end{solution}
\end{frame}


\begin{frame}{例题: 线性方程组解的性质}
	\onslide<+->
	\begin{example}
		已知 $\bmb_1,\bmb_2$ 是 $\bfA\bfx=\bfb$ 的两个不同的解, $\bma_1,\bma_2$ 是 $\bfA\bfx={\bf0}$ 的基础解系, 则 $\bfA\bfx=\bfb$ 的通解为\fillbrace{\visible<+->{B}}, $k_1,k_2$ 为任意常数.
		\xx{$\dfrac{\bmb_1-\bmb_2}2+k_1\bma_1+k_2(\bma_1+\bma_2)$}%
			{$2\bmb_1-\bmb_2+k_1\bma_1+k_2(\bma_1-\bma_2)$}%
			{$\dfrac{\bmb_1+\bmb_2}2+k_1\bma_1+k_2(\bmb_1-\bmb_2)$}%
			{$\dfrac{\bmb_1-\bmb_2}2+k_1\bma_1+k_2(\bmb_1-\bmb_2)$}
	\end{example}
	\onslide<+->
	\begin{example}
		已知 $\bmet_1=(0,1,0)^\rmT,\bmet_2=(-3,2,2)^\rmT$ 是线性方程组 $\laeq[cccccccc]{
			 &x_1&{}-{}& &x_2+{}&2&x_3=&-1\\
			3&x_1&{}+{}& &x_2+{}&4&x_3=&1\\
			a&x_1&{}+{}&b&x_2+{}&c&x_3=&d
		}$ 的两个解向量, 则该方程组的通解为\fillblank[5cm]{\visible<+->{$(0,1,0)^\rmT+k(-3,1,2)^\rmT$}}.
	\end{example}
\end{frame}


\subsection{向量组的线性表示}

\begin{frame}{向量组的线性表示}
	\onslide<+->
	若 $\bfB$ 的列向量可由 $\bfA$ 的列向量组线性表示, 则 $(\bfA,\bfB)$ 的列向量组和 $\bfA$ 的列向量组等价,
	\onslide<+->
	因此 $R(\bfA)=R(\bfA,\bfB)$.

	\onslide<+->
	注意到 $\bfA$ 列向量生成的空间 $V$ 是 $(\bfA,\bfB)$ 列向量生成的空间 $W$ 的子空间.
	\onslide<+->
	若 $R(\bfA)=R(\bfA,\bfb)$, 则 $\bfA$ 列向量组的一个极大无关组 $S$ 也是 $(\bfA,\bfB)$ 的极大无关组.
	\onslide<+->
	从而 $\bfB$ 的列向量都是 $S$ 的线性组合, 也是 $\bfA$ 列向量的线性组合.

	\onslide<+->
	\begin{theorem}
		\begin{enumerate}
			\item $\bfB$ 的列向量组可由 $\bfA$ 的列向量组线性表示 $\iff \bfA\bfX=\bfB$ 有解 $\iff R(\bfA)=R(\bfA,\bfB)$.
			\item $\bfB$ 的列向量组和 $\bfA$ 的列向量组等价 $\iff R(\bfA)=R(\bfA,\bfB)=R(\bfB)$.
		\end{enumerate}
	\end{theorem}
\end{frame}


\begin{frame}{例题: 向量组等价}
	\onslide<+->
	\begin{example}
		证明向量组 $\bma_1,\bma_2$ 与 $\bmb_1,\bmb_2,\bmb_3$ 等价, 其中
		\[\bma_1=\begin{pmatrix}
			1\\-1\\1\\-1
		\end{pmatrix},\ \bma_2=\begin{pmatrix}
			3\\1\\1\\3
		\end{pmatrix},\ \bmb_1=\begin{pmatrix}
			2\\0\\1\\1
		\end{pmatrix},\ \bmb_2=\begin{pmatrix}
			1\\1\\0\\2
		\end{pmatrix},\ \bmb_3=\begin{pmatrix}
			3\\-1\\2\\0
		\end{pmatrix}.\]
	\end{example}
	\onslide<+->
	\begin{proof}
		\[(\bma_1,\bma_2,\bmb_1,\bmb_2,\bmb_3)=\begin{pNiceMatrix}
			1&3&2&1&3\\
			-1&1&-&1&-1\\
			1&1&1&0&2\\
			-1&3&1&2&0
			\augdash53
		\end{pNiceMatrix}\simr\begin{pNiceMatrix}
			1&3&2&1&3\\
			0&2&1&1&1\\
			0&0&0&0&0\\
			0&0&0&0&0
			\augdash53
		\end{pNiceMatrix}.\]
		\onslide<+->{%
			因此 $R(\bma_1,\bma_2,\bmb_1,\bmb_2,\bmb_3)=R(\bma_1,\bma_2)=R(\bmb_1,\bmb_2,\bmb_3)=2$.\qedhere}
	\end{proof}
\end{frame}


\begin{frame}{线性方程组的应用: 最小二乘法\noexer}
	\onslide<+->
	在物理实验中, 经常会出现实验数据与预期不符的情况.
	\onslide<+->
	例如变量 $y$ 应当为变量 $\bfx=(x_1,\dots,x_n)^\rmT$ 的线性组合, 即存在 $n$ 维向量 $\bmb$ 使得 $y=\bfx^\rmT\bmb$.
	\onslide<+->
	但从实验数据解方程却是无解.
	因此我们需要寻找参数 $\bmb$ 使得 $y=\bfx^\rmT\bmb$ 尽可能接近实验数据.

	\onslide<+->
	比较常见的是最小二乘法: 即寻找参数 $\bmb$ 使得
	\[\sum_{i=1}^k |y_i-\bfx_i^\rmT\bmb|^2
	=\|\bfy-\bfA\bmb\|^2\]
	尽可能小, 其中 $(\bfx_i,y_i)$ 是实验数据, $\bfy=(y_1,\dots,y_k)^\rmT$, $\bfA$ 是由行向量 $\bfx_i^\rmT$ 构成的 $k\times n$ 矩阵.
	\onslide<+->
	注意所有向量 $\bfA\bmb$ 形成一个向量空间 $V$, 也就是 $\bfA$ 的列向量生成的空间.
	\onslide<+->
	$\bfy$ 距离这个空间的距离	$\|\bfy-\bfA\bmb\|$ 达到最小时, $\bfy-\bfA\bmb$ 应当和这个空间正交.
	\onslide<+->
	于是 $\bfA^\rmT(\bfy-\bfA\bmb)={\bf0}$, 即 $\bmb$ 是方程
	\[\bfA^\rmT\bfA\bmb=\bfA^\rmT\bfy\]
	的解.
\end{frame}



