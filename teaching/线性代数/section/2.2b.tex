\section{矩阵的运算: 转置、行列式和伴随}

\subsection{矩阵的转置}

\begin{frame}{矩阵转置的定义}
	\onslide<+->
	上一章我们已经说过, 如果 $\bfA=(a_{ij})_{m\times n}$, 称
	\[\bfA^\rmT=\begin{pmatrix}
		a_{11}&a_{21}&\cdots&a_{m1}\\
		a_{12}&a_{22}&\cdots&a_{m2}\\
		\vdots&\vdots&\ddots&\vdots\\
		a_{1n}&a_{2n}&\cdots&a_{mn}
	\end{pmatrix}\]
	为矩阵 $\bfA$ 的\emph{转置}, 它是 $n\times m$ 矩阵.
	\onslide<+->
	例如行向量的转置是列向量, 方阵的转置还是方阵, 上三角阵的转置是下三角阵.
	\onslide<+->
	矩阵的转置满足如下性质:
	\begin{enumerate}
		\item $(\bfA^\rmT)^\rmT=\bfA$;
		\item $(\bfA+\bfB)^\rmT=\bfA^\rmT+\bfB^\rmT$;
		\item $(\lambda\bfA)^\rmT=\lambda\bfA^\rmT$;
		\item \alert{$(\bfA\bfB)^\rmT=\bfB^\rmT\bfA^\rmT$}.
	\end{enumerate}
\end{frame}


\begin{frame}{矩阵转置与乘法}
	\onslide<+->
	例如
	\[\begin{pmatrix}
		a_{11}&a_{12}&a_{13}\\
		a_{21}&a_{22}&a_{23}\\
		a_{31}&a_{32}&a_{33}
	\end{pmatrix}\begin{pmatrix}
		b_1\\b_2\\b_3
	\end{pmatrix}=\begin{pmatrix}
		c_1\\c_2\\c_3
	\end{pmatrix},\]
	\onslide<+->
	两边取转置得到
	\[(b_1,b_2,b_3)\begin{pmatrix}
		a_{11}&a_{21}&a_{31}\\
		a_{12}&a_{22}&a_{32}\\
		a_{13}&a_{23}&a_{33}
	\end{pmatrix}=(c_1,c_2,c_3).\]
\end{frame}


\begin{frame}{矩阵转置与乘法}
	\onslide<+->
	\begin{example}
		设 $\bfA=\begin{pmatrix}
			a&-b&-c&-d\\
			b&a&-d&c\\
			c&d&a&-b\\
			d&-c&b&a
		\end{pmatrix}$, 求 $|\bfA|$.
	\end{example}
	\onslide<+->
	这题当然可以直接硬算, 不过我们可以利用一点小技巧:
	\[\bfA\bfA^\rmT=\begin{pmatrix}
		a&-b&-c&-d\\
		b&a&-d&c\\
		c&d&a&-b\\
		d&-c&b&a
	\end{pmatrix}\begin{pmatrix}
		a&b&c&d\\
		-b&a&d&-c\\
		-c&-d&a&b\\
		-d&c&-b&a
	\end{pmatrix}=(a^2+b^2+c^2+d^2)\bfE.\]
\end{frame}


\begin{frame}{对称阵和反对称阵}
	\onslide<+->
	\begin{definition}
		\begin{itemize}
			\item 如果方阵 $\bfA$ 满足 $\bfA^\rmT=\bfA$, 称 $\bfA$ 为\emph{对称阵};
			\item 如果 $\bfA^\rmT=-\bfA$, 称 $\bfA$ 为\emph{反对称阵}.
		\end{itemize}
	\end{definition}
	\onslide<+->
	例如 $\begin{pmatrix}
		12&6&1\\
		6&8&0\\
		1&0&6
	\end{pmatrix}$
	是对称阵.
	\onslide<+->
	对角矩阵都是对称阵.
	
	\onslide<+->
	例如 $\begin{pmatrix}
		0&6&1\\
		-6&0&0\\
		-1&0&0
	\end{pmatrix}$
	是反对称阵.
	\onslide<+->
	反对称阵的对角线均为 $0$.
\end{frame}


\begin{frame}{对称阵和反对称阵}
	\onslide<+->
	\begin{example}
		证明: 如果 $\bfA,\bfB,\bfA\bfB$ 都是对称阵, 则 $\bfA\bfB=\bfB\bfA$.
	\end{example}
	\onslide<+->
	\begin{proof}
		由题设可知 $\bfA^\rmT=\bfA,\bfB^\rmT=\bfB$,
		\[\bfA\bfB=(\bfA\bfB)^\rmT=\bfB^\rmT\bfA^\rmT=\bfB\bfA.\qedhere\]
	\end{proof}
	\onslide<+->
	想一想: 如果 $\bfA,\bfB,\bfA\bfB$ 中有一个对称阵和两个反对称阵呢?
	\onslide<+->
	\begin{exercise}
		设 $\bfA$ 是 $n$ 阶方阵, \fillbrace{\visible<+->{\alert{A}}}一定是对称阵?
		\xx{$\bfA^\rmT\bfA$}{$\bfA-\bfA^\rmT$}{$\bfA^2$}{$\bfA^\rmT-\bfA$}
	\end{exercise}
	\onslide<+->
	一般地, 如果 $\bfA\in M_{m\times n}$, $\bfA\bfA^\rmT$ 是 $m$ 阶对称阵, $\bfA^\rmT\bfA$ 是 $n$ 阶对称阵.
\end{frame}


\begin{frame}{任一方阵可表为对称阵与反对称阵之和}
	\onslide<+->
	\begin{example}
		证明: 任一方阵均可写成一对称阵和一反对称阵之和.
	\end{example}
	\onslide<+->
	\begin{proof}
		\[\bfA=\frac{\bfA+\bfA^\rmT}2+\frac{\bfA-\bfA^\rmT}2.\qedhere\]
	\end{proof}
	\onslide<+->
	想一想: 如果函数 $f(x)$ 的定义域关于原点对称, 那么 $f(x)$ 一定可以表示成一个偶函数和一个奇函数之和.
\end{frame}


\subsection{方阵的行列式}

\begin{frame}{方阵的行列式}
	\onslide<+->
	方阵的行列式我们已在上一章详细研究过.
	\onslide<+->
	如果方阵 $\bfA$ 的行列式 $|\bfA|=0$, 称 $\bfA$ 为\emph{退化矩阵}, 否则称为\emph{非退化矩阵}.
	\onslide<+->
	行列式满足如下性质:
	\begin{enumerate}
		\item $|\bfA^\rmT|=|\bfA|$;
		\item $|\lambda\bfA|=\lambda^n|\bfA|$, 其中 $\bfA$ 是 $n$ 阶方阵;
		\item \alert{$|\bfA\bfB|=|\bfA|\cdot|\bfB|=|\bfB\bfA|$}.
	\end{enumerate}
\end{frame}


\begin{frame}{行列式与乘法交换}
	\onslide<+->
	我们来证明 $|\bfA\bfB|=|\bfA|\cdot|\bfB|$.
	\onslide<+->
	设 $\bfA=(a_{ij}),\bfB=(b_{ij})$,
	\[D=\detm{
		a_{11}&\cdots&a_{1n}&0&\cdots&0\\
		\vdots&\ddots&\vdots&\vdots&\ddots&0\\
		a_{1n}&\cdots&a_{nn}&0&\cdots&0\\
		-1&&&b_{11}&\cdots&b_{nn}\\
		&\ddots&&\vdots&\ddots&\vdots\\
		&&-1&b_{n1}&\cdots&b_{nn}
	}=\detm{\bfA&\bfO\\-\bfE&\bfB}=|\bfA|\cdot|\bfB|.\]

	\onslide<+->
	经过变换 $c_{n+j}+b_{1j}c_1+\cdots+b_{nj},j=1,\dots,n$ 得到
	$D=\detm{\bfA&\bfC\\-\bfE&\bfO}$,
	\onslide<+->
	其中 $\bfC=(c_{ij})$, $c_{ij}=\sum_{k=1}^nb_{kj}a_{ik}$.
	\onslide<+->
	换言之 $\bfC=\bfA\bfB$.
	\onslide<+->
	再进行变换 $r_i\swap r_{n+j},j=1,\dots,n$ 得到
	\[D=(-1)^n\detm{-\bfE&\bfO\\\bfA&\bfC}=(-1)^n|-\bfE|\cdot|\bfC|=|\bfC|=|\bfA\bfB|.\]
\end{frame}


\begin{frame}{例: 方阵的行列式}
	\onslide<+->
	\begin{exercise}
		设 $\bfA$ 为 $5$ 阶方阵, $|\bfA|=-1$, 则
		$|2\bfA|=$\fillblank{\visible<+->{\alert{$-32$}}},
		$\bigl||\bfA|\bfA\bigr|=$\fillblank{\visible<+->{\alert{$1$}}}.
	\end{exercise}
	\onslide<+->
	\begin{exercise}
		设 $\alpha=(1,0,-1),\bfA=\alpha^\rmT\alpha$, 则
		$|5\bfE-\bfA^3|=$\fillblank{\visible<+->{\alert{$-75$}}}.
	\end{exercise}
	\onslide<+->
	\begin{exercise}
		$\detm{
			2\sin a\cos a&\sin a\cos b+\cos a\sin b&\sin a\cos c+\cos a\sin c\\
			\sin b\cos a+\cos b\sin a&2\sin b\cos b&\sin b\cos c+\cos b\sin c\\
			\sin c\cos a+\cos c\sin a&\sin c\cos b+\cos c\sin b&2\sin c\cos c
		}=$\fillblank{\visible<+->{\alert{$0$}}}.
	\end{exercise}
\end{frame}

\begin{frame}{例: 方阵的行列式}
	\onslide<+->
	\begin{answer}
		注意到
		\begin{align*}
			&\begin{pmatrix}
			2\sin a\cos a&\sin a\cos b+\cos a\sin b&\sin a\cos c+\cos a\sin c\\
			\sin b\cos a+\cos b\sin a&2\sin b\cos b&\sin b\cos c+\cos b\sin c\\
			\sin c\cos a+\cos c\sin a&\sin c\cos b+\cos c\sin b&2\sin c\cos c
		\end{pmatrix}\\=&\begin{pmatrix}
			\sin a&\cos a&0\\
			\sin b&\cos b&0\\
			\sin c&\cos c&0
		\end{pmatrix}\begin{pmatrix}
			\cos a&\cos b&\cos c\\
			\sin a&\sin b&\sin c\\
			0&0&0
		\end{pmatrix}.
	\end{align*}
	\end{answer}
	\onslide<+->
	设 $\bfA\in M_{m\times n},\bfB\in M_{n\times m}$.
	\onslide<+->
	如果 $m>n$, 那么
	\[|\bfA\bfB|=\detm{(\bfA,\bfO_{m\times(m-n)})\begin{pmatrix}
		\bfB\\\bfO_{(m-n)\times m}
	\end{pmatrix}}=0=|\bfB\bfA|.\]
\end{frame}


\subsection{方阵的伴随矩阵}

\begin{frame}{伴随矩阵}
	\onslide<+->
	\begin{definition}
		设 $\bfA=(a_{ij})_{n\times n}$.
		由 $\bfA$ 的代数余子式形成的矩阵
		\[\bfA^*=(A_{ji})=\begin{pmatrix}
			A_{11}&A_{21}&\cdots&A_{n1}\\
			A_{12}&A_{21}&\cdots&A_{n2}\\
			\vdots&\vdots&\ddots&\vdots\\
			A_{1n}&A_{21}&\cdots&A_{nn}
		\end{pmatrix}\]
		称为矩阵 $\bfA$ 的\emph{伴随矩阵}.
	\end{definition}
	\onslide<+->
	注意, 伴随矩阵的 $(i,j)$ 元是代数余子式 \alert{$A_{ji}$ 而不是 $A_{ij}$}.
	\onslide<+->
	\begin{example}
		如果 $\bfA=\begin{pmatrix}
			a&b\\c&d
		\end{pmatrix}$, 那么 $\bfA^*=\begin{pmatrix}
			d&-b\\-c&a
		\end{pmatrix}$.
	\end{example}
\end{frame}


\begin{frame}{伴随矩阵的性质}
	\onslide<+->
	伴随矩阵满足如下重要性质:
	\begin{alertblock@}
		\begin{enumerate}
		\item $\bfA\bfA^*=\bfA^*\bfA=|\bfA|E_n$.
		\end{enumerate}
	\end{alertblock@}
	\onslide<+->
	这是因为
	\[\bfA\bfA^*=\begin{pmatrix}
		a_{11}&a_{12}&\cdots &a_{1n}\\
		a_{21}&a_{22}&\cdots &a_{2n}\\
		\vdots&\vdots&\ddots&\vdots\\
		a_{n1}&a_{n2}&\cdots &a_{nn}
	\end{pmatrix}\begin{pmatrix}
		A_{11}&A_{21}&\cdots&A_{n1}\\
		A_{12}&A_{21}&\cdots&A_{n2}\\
		\vdots&\vdots&\ddots&\vdots\\
		A_{1n}&A_{21}&\cdots&A_{nn}
	\end{pmatrix}\]
	的 $(i,j)$ 元是
	\[a_{i1}A_{j1}+\cdots+a_{in}A_{jn}=\begin{cases}
		|\bfA|,&i=j;\\
		0,&i\neq j.
	\end{cases}\]
\end{frame}



\begin{frame}{伴随矩阵的性质}
	\onslide<+->
	\begin{block@}
		\begin{enumerate}
			\setcounter{enumi}{1}
			\item $|\bfA^*|=|\bfA|^{n-1}$.
			\item $(k\bfA)^*=k^{n-1}\bfA^*$.
		\end{enumerate}
	\end{block@}
	\onslide<+->
	如果 $\bfA=\bfO$, 显然 $\bfA^*=\bfO$.
	\onslide<+->
	如果 $|\bfA|=0$ 但 $\bfA\neq \bfO$, 那么
	\[\bfA^*\begin{pmatrix}
		a_{1j}\\a_{2j}\\\vdots\\a_{nj}
	\end{pmatrix}=\bfO_{n\times 1}.\]
	\onslide<+->
	所以以 $\bfA^*$ 为系数的齐次线性方程组有非零解, 从而 $|\bfA^*|=0$.

	\onslide<+->
	如果 $|\bfA|\neq0$, 由 $|\bfA^*|\cdot|\bfA|=\bigl||\bfA|\bfE_n\bigr|=|\bfA|^n$ 可得.
\end{frame}

