\begin{frame}[<*>]{课程信息}
\begin{itemize}
	\item 课时:
		\begin{itemize}
			\item $10$周$40$课时;
			\item 2024-09-24 $\sim$ 2024-11-27
		\end{itemize}
	\item 课程QQ群(入群答案 1400071B)
		\begin{itemize}
			\item 003班(自动化) \emph{\textbf{973042523}}
			\item 004班(电气) \emph{\textbf{980820998}}
		\end{itemize}
	\item 教材: 唐烁\ 朱士信《线性代数》
	\item \alert{本课程讲授顺序与教材有所不同}
\end{itemize}
\begin{tikzpicture}[overlay,xshift=100mm,yshift=26mm]
	\draw (0,0) node {\includegraphics[height=50mm]{../image/book.jpg}};
\end{tikzpicture}
\end{frame}


\begin{frame}[<*>]{成绩构成}
	\vspace{-\baselineskip}
	\begin{center}
	\begin{tikzpicture}
		\begin{scope}[xshift=.2mm,yshift=1.7mm,scale=2]
			\filldraw[cstcurve,dcolorb,fill=dcolorb!20] (0,0)--(0,1) arc (90:144:1) -- cycle;
		\end{scope}
		\begin{scope}[scale=2.1]
			\filldraw[cstcurve,dcolorc,fill=dcolorc!20] (0,0)--({cos{144}},sin{144}) arc (144:324:1) -- cycle;
		\end{scope}
		\begin{scope}[xshift=1.8mm,yshift=.5mm,scale=1.95]
			\filldraw[cstcurve,dcolord,fill=dcolord!20] (0,0)--({cos{36}},-sin{36}) arc (-36:0:1) -- cycle;
		\end{scope}
		\begin{scope}[xshift=1.5mm,yshift=1.7mm,scale=2]
			\filldraw[cstcurve,dcolore,fill=dcolore!20] (0,0)--(1,0) arc (0:90:1) -- cycle;
		\end{scope}

		\begin{scope}
			\draw[dcolorb] (-.7,1.5)--(-1.6,2.5)--(-3,2.5);
			\filldraw[dcolorb,fill=dcolorb!20] (-.7,1.5) circle(.1);
			\node at (-5,1.7) [text width=53mm]  (1){
			\begin{block*}{作业 15分}
				作业为配套练习册, 每月交一次.
				\alert{作业不允许迟交}. 没带的请当天联系助教补交, 迟一天交 $-50\%$ 当次作业分, 迟两天或以上0分. 请假需提前交给我请假条. 
			\end{block*}};
		\end{scope}

		\begin{scope}
			\draw[dcolore] (.9,1.5)--(1.6,2.5)--(3,2.5);
			\filldraw[dcolore,fill=dcolore!20] (.9,1.5) circle(.1);
			\node at (4.7,1.8) [text width=43mm] (2){{\begin{block*}{课堂测验 25分}
				课堂测验共3次, 取最高的两次平均. 测验范围和时间会提前通知. \alert{测验时在教室内作答, 否则按未考处理}. 
			\end{block*}}};
		\end{scope}

		\begin{scope}
			\draw[dcolord] (1.5,-.4)--(2.2,-1.1)--(3,-1.1);
			\filldraw[dcolord,fill=dcolord!20] (1.5,-.4) circle(.1);
			\node at (4.7,-1.5) [text width=43mm] (3){\begin{block*}{期末报告 10分}
				期末之前会告知主题. 请交手写纸质版, 并自行留存电子版本以免意外丢失. 
			\end{block*}};
		\end{scope}

		\begin{scope}
			\draw[dcolorc] (-.7,-1)--(-1.6,-1.7)--(-3,-1.7);
			\filldraw[dcolorc,fill=dcolorc!20] (-.7,-1) circle(.1);
			\node at (-5,-1.7) [text width=53mm] (4){\begin{block*}{期末考试 50分}
				期末卷面需要达到45分才计算总评分数, 45分以下直接不及格. 
			\end{block*}};
		\end{scope}
	\end{tikzpicture}
\end{center}
\end{frame}


\begin{frame}{线性代数的意义}
	\onslide<+->
	线性代数是一门利用\alert{代数方法}研究\emph{线性方程、线性空间、线性变换}等线性结构的课程.
	\onslide<+->
	尽管真实的世界中, 例如函数关系, 往往是非线性的.
	但我们可以利用线性方法去模拟、近似、逼近它.
	\onslide<+->
	这便是线性代数它的意义.

	\onslide<+->
	线性代数的应用之广泛, 使得它成为了高等教育中大多数学科的必修数学课程.
	\onslide<+->
	我们不在此处逐一列举, 在之后的授课中我们会见到它的各种应用.
\end{frame}


\begin{frame}{课程内容关系}
	\vspace{-\baselineskip}
	\begin{center}
		\begin{tikzpicture}[node distance=22pt]
			\node[cstnodeg] (1) {标准线性空间};
			\node[cstnodeg] (2) [right=25pt of 1] {线性映射};
			\node[cstnodeg] (3) [right=35pt of 2] {矩阵};
			\node[cstnodeg] (4) [right=35pt of 3] {矩阵的运算};
			
			\node[cstnodeg] (11) [below=of 1] {向量};
			\node[cstnoder] (123) [right=80pt of 11] {行列式};
			\node (12) [below=of 2] {\phantom{行列式}};
			\node (13) [below=of 3] {\phantom{行列式}};
			\node[cstnoder] (14) [below=of 4] {伴随和逆};

			\node[cstnodeg] (21) [below=of 11] {向量组};
			\node[cstnodeg] (22) [below=of 12] {向量组的秩};
			\node[cstnodeg] (23) [below=of 13] {极大无关组};
			\node[cstnoder] (24) [below=of 14] {相抵: 秩};
			\node[cstnoder] (25) [right=of 24] {应用: 线性方程组};

			\node[cstnodeg] (31) [below=of 21] {子空间};
			\node[cstnodeg] (32) [below=of 22] {维数};
			\node[cstnodeg] (33) [below=of 23] {基};
			
			\node[cstnoder] (42) [below=of 32] {相似: 特征值与特征向量};
			\node[cstnoder] (44) [right=40pt of 42] {相合: 实二次型};
			\draw[cstmarrow,dcolorc] (1.east) to (2.west);
			\draw[cstmarrow,dcolorc] (2.east) to (3.west);
			\draw[cstmarrow,dcolorc] (3.east) to (4.west);
			\draw[cstmarrow,dcolorc] (1.south) to (11.north);
			\draw[cstmarrow,dcolorc] (2.south) to (123.north);
			\draw[cstmarrow,dcolorc] (3.south) to (123.north);
			\draw[cstmarrow,dcolorc] (123.east) to (14.west);
			\draw[cstmarrowd,dcolorc] (11.south) to (21.north);
			\draw[cstmarrowd,dcolorc] (21.south) to (31.north);
			\draw[cstmarrowd,dcolorc] (22.south) to (32.north);
			\draw[cstmarrowd,dcolorc] (23.south) to (33.north);
			\draw[cstmarrow,dcolorc] (21.east) to (22.west);
			\draw[cstmarrow,dcolorc] (22.east) to (23.west);
			\draw[cstmarrow,dcolorc] (23.east) to (24.west);
			\draw[cstmarrow,dcolorc] (24.east) to (25.west);
			\draw[cstmarrow,dcolorc] (31.east) to (32.west);
			\draw[cstmarrow,dcolorc] (32.east) to (33.west);
			\draw[cstmarrow,dcolorc] (33.south) to (42.north);
			\draw[cstmarrow,dcolorc] (33.south) to (44.north);
		\end{tikzpicture}
	\end{center}
\end{frame}


\begin{frame}{课程学习方法}
	\begin{center}
		\begin{tikzpicture}[node distance=25pt,text width=26mm]
			\node[cstnoder,align=center] (0) {\emph{核心}\\理解抽象概念\\掌握计算方法};
			\node[cstnodeg,align=center] (1) [above=of 0]{\emph{课前}\\预习课本};
			\node[cstnodeg,align=center] (2) [right=of 0] {\emph{课上}\\认真听课\\记好笔记};
			\node[cstnodeg,align=center] (3) [below=of 0] {\emph{课后}\\过一遍教材\\与课上知识点};
			\node[cstnodeg,align=center] (4) [left=of 0] {\emph{作业}\\检测学\\习效果};
			\draw[cstnarrow,dcolorc] (1.east) to[bend left] (2.north);
			\draw[cstnarrow,dcolorc] (2.south) to[bend left] (3.east);
			\draw[cstnarrow,dcolorc] (3.west) to[bend left] (4.south);
			\draw[cstnarrow,dcolorc] (4.north) to[bend left] (1.west);
		\end{tikzpicture}
	\end{center}
\end{frame}

