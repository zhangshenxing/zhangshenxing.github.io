\begin{frame}[<*>]{课程安排}
	\begin{columns}
		\column{0.48\textwidth}
			\begin{block*}{课时}
				本课程共$10$周$40$课时, 自2024年9月24日至2024年11月27日.
			\end{block*}
			\vspace{.4\baselineskip}
			\onslide<+->
			\begin{block}{课程QQ群(入群答案 1400071B)}
				\vspace{\baselineskip}
					003班(自动化) \emph{\textbf{973042523}}
					\vspace{\baselineskip}
					
					004班(电气) \emph{\textbf{980820998}}
			\end{block}
			\column{0.48\textwidth}
			\begin{block}{教材}
				\begin{figure}
					\includegraphics[height=32mm]{../image/book.jpg}
					\caption{唐烁\ 朱士信《线性代数》}
				\end{figure}
			\end{block}
		\end{columns}
\end{frame}


% \begin{frame}{成绩构成}
% 	\onslide<+->
% 	\begin{columns}
% 		\column{0.48\textwidth}
% 		\onslide<+->
% 		\begin{block*}{作业 15分}
% 			\vspace{\baselineskip}
% 			作业为配套练习册, 每章交一次.

% 			\vspace{\baselineskip}
% 			\alert{作业不允许迟交}. 没带的请当天联系助教补交, 迟一天交 $-50\%$ 当次作业分, 迟两天或以上0分. 请假需提前交给我请假条. 
% 			\vspace{1.1\baselineskip}
% 		\end{block*}

% 		\column{0.48\textwidth}
% 		\onslide<+->
% 		\begin{block*}{课堂测验 25分}
% 			课堂测验共3次, 取最高的两次平均. 测验范围和时间会提前通知. \alert{测验时在教室内作答,否则按未考处理}. 
% 		\end{block*}

% 		\onslide<+->
% 		\begin{block*}{期末报告 10分}
% 			期末之前会告知主题. 请交手写纸质版, 并自行留存电子版本以免意外丢失. 
% 		\end{block*}

% 		\onslide<+->
% 		\begin{block*}{期末考试 50分}
% 			期末卷面需要达到45分才计算总评分数, 45分以下直接不及格. 
% 	\end{block*}
% 	\end{columns}
% \end{frame}

\begin{frame}[<*>]{成绩构成}
	\vspace{-\baselineskip}
	\begin{center}
	\begin{tikzpicture}
		\begin{scope}[xshift=.2mm,yshift=1.7mm,scale=2]
			\filldraw[cstcurve,dcolorb,fill=white] (0,0)--(0,1) arc (90:144:1) -- cycle;
		\end{scope}
		\begin{scope}[scale=2.1]
			\filldraw[cstcurve,dcolorc,fill=white] (0,0)--({cos{144}},sin{144}) arc (144:324:1) -- cycle;
		\end{scope}
		\begin{scope}[xshift=1.8mm,yshift=.5mm,scale=1.95]
			\filldraw[cstcurve,dcolord,fill=white] (0,0)--({cos{36}},-sin{36}) arc (-36:0:1) -- cycle;
		\end{scope}
		\begin{scope}[xshift=1.5mm,yshift=1.7mm,scale=2]
			\filldraw[cstcurve,dcolore,fill=white] (0,0)--(1,0) arc (0:90:1) -- cycle;
		\end{scope}

		\begin{scope}
			\draw[dcolorb] (-.7,1.5)--(-1.6,2.5)--(-3,2.5);
			\filldraw[dcolorb,fill=white] (-.7,1.5) circle(.1);
			\node at (-5,1.7) [text width=53mm]  (1){
			\begin{block*}{作业 15分}
				作业为配套练习册, 每章交一次.
				\alert{作业不允许迟交}. 没带的请当天联系助教补交, 迟一天交 $-50\%$ 当次作业分, 迟两天或以上0分. 请假需提前交给我请假条. 
			\end{block*}};
		\end{scope}

		\begin{scope}
			\draw[dcolore] (.9,1.5)--(1.6,2.5)--(3,2.5);
			\filldraw[dcolore,fill=white] (.9,1.5) circle(.1);
			\node at (4.7,1.8) [text width=43mm] (2){{\begin{block*}{课堂测验 25分}
				课堂测验共3次, 取最高的两次平均. 测验范围和时间会提前通知. \alert{测验时在教室内作答,否则按未考处理}. 
			\end{block*}}};
		\end{scope}
		
		\begin{scope}
			\draw[dcolord] (1.5,-.4)--(2.2,-1.1)--(3,-1.1);
			\filldraw[dcolord,fill=white] (1.5,-.4) circle(.1);
			\node at (4.7,-1.5) [text width=43mm] (3){\begin{block*}{期末报告 10分}
				期末之前会告知主题. 请交手写纸质版, 并自行留存电子版本以免意外丢失. 
			\end{block*}};
		\end{scope}

		\begin{scope}
			\draw[dcolorc] (-.7,-1)--(-1.6,-1.7)--(-3,-1.7);
			\filldraw[dcolorc,fill=white] (-.7,-1) circle(.1);
			\node at (-5,-1.7) [text width=53mm] (4){\begin{block*}{期末考试 50分}
				期末卷面需要达到45分才计算总评分数, 45分以下直接不及格. 
			\end{block*}};
		\end{scope}
	\end{tikzpicture}
\end{center}
\end{frame}


\begin{frame}{线性代数的意义}
	\onslide<+->
	线性代数是一门研究\emph{线性方程、线性空间、线性变换}等线性结构的课程.
	\onslide<+->
	尽管真实的世界中, 例如函数关系, 往往是非线性的.
	但我们可以利用线性方法去模拟、近似、逼近它.
	\onslide<+->
	这便是线性代数它的意义.

	\onslide<+->
	线性代数的应用之广泛, 使得它成为了高等教育中大多数学科的必修数学课程.
	\onslide<+->
	我们不在此处逐一列举, 在之后的课程中我们会见到它的各种应用.
\end{frame}


\begin{frame}{课程内容}
	\vspace{-\baselineskip}
	\begin{center}
		\begin{tikzpicture}[
			small mindmap,
			every node/.style={concept, circular drop shadow,execute at begin node=\hskip0pt},
			root concept/.append style={concept color=black, fill=white, line width=1ex, text=black},
			font=\scriptsize, text=white,
			childa/.style={concept color=red,faded/.style={concept color=red!50}},
			childb/.style={concept color=blue,faded/.style={concept color=blue!50}},
			childc/.style={concept color=orange,faded/.style={concept color=orange!50}},
			childd/.style={concept color=green!50!black,faded/.style={concept color=green!50!black!50}},
			childe/.style={concept color=purple,faded/.style={concept color=green!50!black!50}},
			grow cyclic,
			level 1/.append style={level distance=2.8cm,font=\small},
			level 2/.append style={level distance=2.5cm,sibling angle=45,font=\scriptsize}]
			\node [root concept] {线性代数} % root
			child [grow=-5, level distance=3.5cm, childa] { node {起源}
				child[grow=0, level distance=2.5cm] { node {解线性方程} }
				child[grow=25, level distance=3.5cm] { node {行列式的引入与计算} }
				child[grow=50, level distance=2.5cm] { node {克拉默法则} }
			}
			child [grow=35, level distance=3.5cm, childb] { node {矩阵}
				child[grow=12,level distance=3.7cm] { node {线性方程的再抽象} }
				child[grow=35,level distance=4.2cm] { node {矩阵的定义与运算} }
				child[grow=58,level distance=3.5cm] { node {逆矩阵、分块矩阵} }
				child[grow=85,level distance=3cm] { node {初等变换和秩} }
			}
			child [grow=90, level distance=2.5cm, childc] { node {向量组}
				child[grow=50,level distance=2.5cm] { node {线性表示与线性相关} }
				child[grow=90,level distance=2.7cm] { node {极大无关组与秩} }
				child[grow=130,level distance=2.5cm] { node {向量空间与标准正交组} }
			}
			child [grow=130, level distance=3.3cm, childd] { node {线性方程}
				child[grow=115,level distance=3cm] { node {齐次线性方程} }
				child[grow=150,level distance=3.7cm] { node {非齐次线性方程} }
			}
			child [grow=180, level distance=3.3cm, childe] { node {相似和二次型}
				child[grow=115,level distance=2.8cm] { node {特征值与特征向量} }
				child[grow=135,level distance=4.2cm] { node {相似矩阵} }
				child[grow=155,level distance=3.3cm] { node {实对称矩阵和对角化} }
				child[grow=185,level distance=3cm] { node {二次型} }
			};
		\end{tikzpicture}
	\end{center}
\end{frame}

\begin{frame}{课程学习方法}
	\begin{center}
		\begin{tikzpicture}[node distance=25pt]
			\node[cstnodeg,align=center] (1) at (0,2)  {\emph{课前}\\预习课本};
			\node[cstnodeg,align=center] (2) at (3,0)  {\emph{课上}\\认真听课\\记好笔记};
			\node[cstnodeg,align=center] (3) at (0,-2) {\emph{课后}\\过一遍教材\\与课上知识点};
			\node[cstnodeg,align=center] (4) at (-3,0) {\emph{作业}\\检测学\\习效果};
			\draw[cstnarrow,dcolorc] (1.east) to[bend left] (2.north);
			\draw[cstnarrow,dcolorc] (2.south) to[bend left] (3.east);
			\draw[cstnarrow,dcolorc] (3.west) to[bend left] (4.south);
			\draw[cstnarrow,dcolorc] (4.north) to[bend left] (1.west);
		\end{tikzpicture}
	\end{center}
\end{frame}

