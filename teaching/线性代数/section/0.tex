\begin{frame}[<*>]{课程信息}
	\begin{figure}[bpt]
		\begin{minipage}{.65\textwidth}
			\begin{itemize}
				\item 课时:
					\begin{itemize}
						\item $10$周$40$课时;
						\item 2024-09-24 $\sim$ 2024-11-27
					\end{itemize}
				\item 课程QQ群(入群答案 1400071B)
					\begin{itemize}
						\item 003班(自动化) \alert{\textbf{973042523}}
						\item 004班(电气) \alert{\textbf{980820998}}
					\end{itemize}
				\item 教材: 唐烁\ 朱士信《线性代数》
				\item \alert{本课程讲授顺序与教材有所不同}
			\end{itemize}
		\end{minipage}
		\begin{minipage}{.3\textwidth}
			\begin{tikzpicture}
				\draw (0,0) node {\includegraphics[height=50mm]{../image/book.jpg}};
			\end{tikzpicture}
		\end{minipage}
	\end{figure}
\end{frame}


\begin{frame}{成绩构成}
	\vspace{-\baselineskip}
	\begin{center}
	\begin{tikzpicture}
		\begin{scope}
			\begin{scope}[xshift=.2mm,yshift=1.7mm,scale=2]
				\filldraw[cstcurve,main,cstfill1] (0,0)--(0,1) arc (90:144:1) -- cycle;
			\end{scope}
			\draw[main] (-.7,1.5)--(-1.6,2.5)--(-3,2.5);
			\filldraw[main,cstfill1] (-.7,1.5) circle(.1);
			\node at (-5,1.7) [text width=53mm]  (1){
				\begin{main*}{作业 15分}
					作业为配套练习册, 每两周交一次.
					\alert{作业不允许迟交}.
					没带的请当天联系助教补交, 迟一天交 $-50\%$ 当次作业分, 迟两天或以上0分.
					请假需提前交给我请假条. 
				\end{main*}
			};
		\end{scope}

		\begin{scope}
			\begin{scope}[xshift=1.5mm,yshift=1.7mm,scale=2]
				\filldraw[cstcurve,second,cstfill2] (0,0)--(1,0) arc (0:90:1) -- cycle;
			\end{scope}
			\draw[second] (.9,1.5)--(1.6,2.5)--(3,2.5);
			\filldraw[second,cstfill2] (.9,1.5) circle(.1);
			\node at (4.7,1.8) [text width=43mm] (2){{\begin{second*}{课堂测验 25分}
					课堂测验共3次, 取最高的两次平均. 测验范围和时间会提前通知. \alert{测验时在教室内作答, 否则按未考处理}.
			\end{second*}}};
		\end{scope}

		\begin{scope}
			\begin{scope}[xshift=1.8mm,yshift=.5mm,scale=1.95]
				\filldraw[cstcurve,third,cstfill3] (0,0)--({cos{36}},-sin{36}) arc (-36:0:1) -- cycle;
			\end{scope}
			\draw[third] (1.5,-.4)--(2.2,-1.1)--(3,-1.1);
			\filldraw[third,cstfill3] (1.5,-.4) circle(.1);
			\node at (4.7,-1.5) [text width=43mm] (3){\begin{third*}{期末报告 10分}
				期末之前会告知主题. 请交手写纸质版, 并自行留存电子版本以免意外丢失. 
			\end{third*}};
		\end{scope}

		\begin{scope}
			\begin{scope}[scale=2.1]
				\filldraw[cstcurve,fourth,cstfill4] (0,0)--({cos{144}},sin{144}) arc (144:324:1) -- cycle;
			\end{scope}
			\draw[fourth] (-.7,-1)--(-1.6,-1.7)--(-3,-1.7);
			\filldraw[fourth,cstfill4] (-.7,-1) circle(.1);
			\node at (-5,-1.7) [text width=53mm] (4){\begin{fourth*}{期末考试 50分}
				期末卷面需要达到45分才计算总评分数, 45分以下直接不及格.
			\end{fourth*}};
		\end{scope}
	\end{tikzpicture}
\end{center}
\end{frame}


\begin{frame}{线性代数的意义}
	\onslide<+->
	线性代数是一门利用\alert{代数方法}研究\alert{线性方程、线性空间、线性变换}等线性结构的课程.

	\onslide<+->
	线性代数通过从具体的、几何化的观念出发, 抽象出一套代数化的方法, 从而避免了高维情形缺乏几何直观的问题. 
	\onslide<+->
	如同微积分中“以直代曲”思想引出导数、切线、积分等一系列概念, 线性代数利用“以直代曲”思想将许多非线性问题的处理转化为线性问题, 非线性模型近似为线性模型等.
	
	\onslide<+->
	这些内容在统计学、密码学、运筹学、物理学、工程学、管理学、信息学、计算机科学等很多领域有着广泛的应用.
	\onslide<+->
	我们不在此处逐一列举, 在之后的授课中我们会见到它的各种应用.
\end{frame}


\begin{frame}{课程内容关系}
	\vspace{-\baselineskip}
	\begin{center}
		\begin{tikzpicture}[node distance=22pt]
			\node[cstnode4] (1) {标准线性空间};
			\node[cstnode4] (2) [right=25pt of 1] {线性映射};
			\node[cstnode4] (3) [right=35pt of 2] {矩阵};
			\node[cstnode4] (4) [right=35pt of 3] {矩阵的运算};
			
			\node[cstnode4] (11) [below=of 1] {向量};
			\node[cstnode2] (123) [right=80pt of 11] {行列式};
			\node (12) [below=of 2] {\phantom{行列式}};
			\node (13) [below=of 3] {\phantom{行列式}};
			\node[cstnode2] (14) [below=of 4] {伴随和逆};

			\node[cstnode4] (21) [below=of 11] {向量组};
			\node[cstnode4] (22) [below=of 12] {向量组的秩};
			\node[cstnode4] (23) [below=of 13] {极大无关组};
			\node[cstnode2] (24) [below=of 14] {相抵: 秩};
			\node[cstnode2] (25) [right=of 24] {应用: 线性方程组};

			\node[cstnode4] (31) [below=of 21] {子空间};
			\node[cstnode4] (32) [below=of 22] {维数};
			\node[cstnode4] (33) [below=of 23] {基};
			
			\node[cstnode2] (42) [below=of 32] {相似: 特征值与特征向量};
			\node[cstnode2] (44) [right=40pt of 42] {相合: 实二次型};
			\draw[cstmra,main] (1.east) to (2.west);
			\draw[cstmra,main] (2.east) to (3.west);
			\draw[cstmra,main] (3.east) to (4.west);
			\draw[cstmra,main] (1.south) to (11.north);
			\draw[cstmra,main] (2.south) to (123.north);
			\draw[cstmra,main] (3.south) to (123.north);
			\draw[cstmra,main] (123.east) to (14.west);
			\draw[cstmlra,main] (11.south) to (21.north);
			\draw[cstmlra,main] (21.south) to (31.north);
			\draw[cstmlra,main] (22.south) to (32.north);
			\draw[cstmlra,main] (23.south) to (33.north);
			\draw[cstmra,main] (21.east) to (22.west);
			\draw[cstmra,main] (22.east) to (23.west);
			\draw[cstmra,main] (23.east) to (24.west);
			\draw[cstmra,main] (24.east) to (25.west);
			\draw[cstmra,main] (31.east) to (32.west);
			\draw[cstmra,main] (32.east) to (33.west);
			\draw[cstmra,main] (33.south) to (42.north);
			\draw[cstmra,main] (33.south) to (44.north);
		\end{tikzpicture}
	\end{center}
\end{frame}


\begin{frame}{课程学习方法}
	\begin{center}
		\begin{tikzpicture}[node distance=25pt,text width=26mm]
			\node[cstnode2,align=center] (0) {\alert{核心}\\理解抽象概念\\掌握计算方法};
			\node[cstnode4,align=center] (1) [above=of 0]{\alert{课前}\\预习课本};
			\node[cstnode4,align=center] (2) [right=of 0] {\alert{课上}\\认真听课\\记好笔记};
			\node[cstnode4,align=center] (3) [below=of 0] {\alert{课后}\\过一遍教材\\与课上知识点};
			\node[cstnode4,align=center] (4) [left=of 0] {\alert{作业}\\检测学\\习效果};
			\draw[cstnra,main] (1.east) to[bend left] (2.north);
			\draw[cstnra,main] (2.south) to[bend left] (3.east);
			\draw[cstnra,main] (3.west) to[bend left] (4.south);
			\draw[cstnra,main] (4.north) to[bend left] (1.west);
		\end{tikzpicture}
	\end{center}
\end{frame}

