\section{矩阵的秩}

\subsection{矩阵秩的定义}


\begin{frame}{行秩、列秩与秩}
	\onslide<+->
	我们知道, 每个矩阵 $\bfA$ 都等价于某个标准型
	$\begin{pmatrix}
		\bfE_r&\bfO\\
		\bfO&\bfO
	\end{pmatrix}$.
	\onslide<+->
	称 $r$ 为 $\bfA$ 的\emph{秩}, 记作 \alert{$\rank(\bfA)$}.

	\onslide<+->
	称 $\bfA$ 的行向量组的秩为\emph{行秩}, 列向量组的秩为\emph{列秩}.
	\onslide<+->
	\begin{algorithm}{行秩等于列秩等于秩}
		$\bfA$ 的行秩和列秩均等于秩 $\rank(\bfA)$.
	\end{algorithm}
	\onslide<+->
	由此可知矩阵的秩是唯一确定的.

	\onslide<+->
	对于行阶梯形矩阵, 再实施初等变换使其变为行最简形矩阵或标准型矩阵, 并不会改变它的非零行的个数.
	\onslide<+->
	换言之, \alert{行阶梯形矩阵的秩就是非零行的个数}.

	\onslide<+->
	设 $\bfA$ 通过初等行变换变为行阶梯形矩阵 $\bfB$, 则二者秩相等, 二者的行向量组等价, 从而行秩也相等.
	\onslide<+->
	对于 $\bfB$, 它的行秩就是非零行的个数, 也就是 $\rank(\bfB)$.
	\onslide<+->
	因此 $\bfA$ 的行秩等于秩.
	\onslide<+->
	不难知道 $\rank(\bfA)=\rank(\bfA^\rmT)$,
	从而 $\bfA$ 的列秩$=\bfA^\rmT$ 的行秩$=\rank(\bfA^\rmT)=\rank(\bfA)$.
\end{frame}


\begin{frame}{行阶梯形矩阵的秩}
	\onslide<+->
	\begin{example}
		求矩阵 $\bfA=\begin{pmatrix}
			2&-1&0&3&-2\\
			0&3&1&-2&5\\
			0&0&0&4&-3\\
			0&0&0&0&0
		\end{pmatrix},\bfB=\begin{pmatrix}
			1&2&3\\
			2&3&-5\\
			4&7&1
		\end{pmatrix}$ 的秩.
	\end{example}
	\onslide<+->
	\begin{solution}
		$\bfA$ 是行阶梯形矩阵, 因此 $\rank(\bfA)=3$.
		\onslide<+->{\[\bfB\wsim{r_2-2r_1}{r_3-4r_1}\begin{pmatrix}
			1&2&3\\
			0&-1&-11\\
			0&-1&-11
		\end{pmatrix}\wsim{r_3-r_2}{-r_2}\begin{pmatrix}
			1&2&3\\
			0&1&11\\
			0&0&0
		\end{pmatrix}\onslide<+->{\implies \rank(\bfB)=2.}\]}
		\vspace{-\baselineskip}
	\end{solution}
\end{frame}


\begin{frame}{例: 计算矩阵的秩}
	\onslide<+->
	\begin{example}
		求矩阵 $\bfA=\begin{pmatrix}
			1&1&a\\
			-1&a-1&1-a\\
			1&1&a^2\\
			1&1&2a+1
		\end{pmatrix}$ 的秩.
	\end{example}
	\onslide<+->
	\begin{solution}
		\[\bfA\wsim{\substack{\nsmath{r_2+r_1}\\\nsmath{r_3-r_1}}}{r_4-r_1}\begin{pmatrix}
			1&1&a\\
			0&a&1\\
			0&0&a^2-a\\
			0&0&a+1
		\end{pmatrix}\onslide<+->{\wsim{r_3-ar_4}{-\frac12r_3}\begin{pmatrix}
			1&1&a\\
			0&a&1\\
			0&0&a\\
			0&0&a+1
		\end{pmatrix}}\onslide<+->{\sim\begin{pmatrix}
			1&1&a\\
			0&a&1\\
			0&0&1\\
			0&0&0
		\end{pmatrix}}\]
		\onslide<+->{%
		因此 $a\neq 0$ 时, $\rank(\bfA)=3$; $a=0$ 时, $\rank(\bfA)=2$.
		}
	\end{solution}
\end{frame}


\begin{frame}{例: 计算矩阵的秩}
	\onslide<+->
	注意处理带未知数的矩阵时, 不宜实施 $\dfrac{1}{a+1}r_2,(a-2)r_3$ 等类似操作, 因为其分母或系数可能为零.
	\onslide<+->
	\begin{exercise}
		求矩阵 $\bfA=\begin{pmatrix}
			1&1&-2&3\\
			2&1&-6&4\\
			3&2&m&7
		\end{pmatrix}$ 的秩.
	\end{exercise}
	\onslide<+->
	\begin{answer}
		$m\neq -8$ 时, $\rank(\bfA)=3$; $m=-8$ 时, $\rank(\bfA)=2$.
	\end{answer}
\end{frame}


\begin{frame}{例: 计算矩阵的秩}
	\onslide<+->
	\begin{example}
		求矩阵 $\bfA=\begin{pmatrix}
			a&1&1&1\\
			1&a&1&1\\
			1&1&a&1\\
			1&1&1&a
		\end{pmatrix}$ 的秩.
	\end{example}
	\onslide<+->
	\begin{solution}
		\[\bfA\!\wsim{\substack{\nsmath{r_1\swap r_4}\\\nsmath{r_2-r_1}}}{\substack{\nsmath{r_3-r_1}\\\nsmath{r_4-ar_1}}}\!\begin{pmatrix}
			1&1&1&a\\
			0&a-1&0&1-a\\
			0&0&a-1&1-a\\
			0&1-a&1-a&1-a^2
		\end{pmatrix}\!\!\onslide<+->{\wsim{r_4+r_2}{r_4+r_3}\!\!\begin{pmatrix}
			1&1&1&a\\
			0&a-1&0&1-a\\
			0&0&a-1&1-a\\
			0&0&0&-(a+3)(a-1)
		\end{pmatrix}}\]
		\onslide<+->{%
			因此 $a\neq 1,-3$ 时, $\rank(\bfA)=4$;
		}\onslide<+->{%
			$a=-3$ 时, $\rank(\bfA)=3$;
		}\onslide<+->{%
			$a=1$ 时, $\rank(\bfA)=1$.
		}
	\end{solution}
\end{frame}


\begin{frame}{矩阵秩与子式}
	\onslide<+->
	矩阵秩有另一种刻画方式.
	\onslide<+->
	矩阵 $\bfA$ 任取 $k$ 行 $k$ 列交叉得到的 $k^2$ 个元素(不改变位置次序)形成的 $k$ 阶方阵的行列式,
	\onslide<+->
	称为 $\bfA$ 的 \emph{$k$ 阶子式}.
	\onslide<+->
	例如 $n$ 阶方阵的余子式是 $n-1$ 阶子式.

	\onslide<+->
	\begin{theorem}
		设 $\rank(\bfA)=r$, 则存在非零的 $r$ 阶子式, 但所有的 $r+1$ 阶子式都是零.
	\end{theorem}
	\onslide<+->
	根据行列式的拉普拉斯展开, 若 $\bfA$ 的 $k$ 阶子式均为零, 则 $k+1$ 阶子式也都是零.
	\onslide<+->
	因此 $\bfA$ 的任意 $s>r$ 阶子式都是零.
	\onslide<+->
	\begin{corollary}
		\begin{enumerate}
			\item $\rank(\bfA)\ge r\iff\bfA$ 存在非零 $r$ 阶子式.
			\item $\rank(\bfA)\le r\iff\bfA$ 所有 $r+1$ 阶子式均为零.
			\item $\rank(\bfA)=r\implies \bfA$ 存在 $1,2,\dots,r$ 阶非零子式. 
		\end{enumerate}
	\end{corollary}
\end{frame}


\begin{frame}{矩阵秩的性质}
	\onslide<+->
	\begin{proof*}
		设 $\bfB=\bfP\bfA$, 其中 $\bfP$ 是初等矩阵.
		\begin{enumerate}
			\item 若 $\bfP=\bfE(i,j)$, 则 $\bfB$ 的 $k$ 阶子式总等于 $\bfA$ 的某个 $k$ 阶子式, 最多相差 $-1$.
			\item 若 $\bfP=\bfE(i(a))$, 则 $\bfB$ 的 $k$ 阶子式总等于 $\bfA$ 的某个 $k$ 阶子式或 $a$ 倍.
			\item 若 $\bfP=\bfE(i,j(a))$, 则 $\bfB$ 的 $k$ 阶子式总等于 $\bfA$ 的某个 $k$ 阶子式.
		\end{enumerate}
		\onslide<+->{%
			因此若 $\bfA$ 的 $k$ 阶子式都是零, 则 $\bfB$ 的 $k$ 阶子式也都是零.
		}

		\onslide<+->{
			由于 $\bfP^{-1}$ 也是初等矩阵, 因此反过来也成立.
		}\onslide<+->{%
			对于 $\bfB=\bfA\bfP$ 情形同理.
		}\onslide<+->{%
			因此, 若 $\bfA\sim\bfB$, 则 $\bfA$ 的 $k$ 阶子式都是零 $\iff$ $\bfB$ 的 $k$ 阶子式都是零.
		}

		\onslide<+->{%
			对于标准型矩阵, 该定理显然成立.
		}\onslide<+->{%
			因此该定理对任意矩阵都成立.\qedhere
		}
	\end{proof*}
\end{frame}


\subsection{矩阵秩的性质}

\begin{frame}{矩阵秩的性质}
	\onslide<+->
	\begin{proposition}
		设 $\bfA\in M_{m\times n}$, 则 $0\le \rank(\bfA)\le \min(m,n)$.
	\end{proposition}
	\onslide<+->
	\begin{definition}
		\begin{enumerate}
			\item 若 $\rank(\bfA)=m$, 称 $\bfA$ \emph{行满秩};
			\item 若 $\rank(\bfA)=n$, 称 $\bfA$ \emph{列满秩};
			\item 若 $\rank(\bfA)=m=n$, 称 $\bfA$ \emph{满秩}.
		\end{enumerate}
	\end{definition}
\end{frame}


\begin{frame}{矩阵秩的性质}
	\onslide<+->
	\begin{proposition}
		\begin{enumerate}
			\item $\rank(\bfA)=0\iff \bfA=\bfO$;
			\item $n$ 阶方阵 $\bfA$ 可逆 $\iff \rank(\bfA)=n$;
			\item $\rank(k\bfA)=\rank(\bfA)=\rank(\bfA^\rmT), k\neq 0$;
			\item $\bfA\sim\bfB\iff \rank(\bfA)=\rank(\bfB)$;
			\item $\rank(\bfA\bfB)\le \min\bigl(\rank(\bfA),\rank(\bfB)\bigr)$;
			\item 若 $\bfA_{m\times n}\bfB_{n\times\ell}=\bfO$, 则 $\rank(\bfA)+\rank(\bfB)\le n$;
			\item $\rank(a\bfA+b\bfB)\le \rank(\bfA,\bfB)\le \rank(\bfA)+\rank(\bfB)$.
			
			特别地, $\max\bigl(\rank(\bfA),\rank(\bfB)\bigr)\le \rank(\bfA,\bfB)$.
		\end{enumerate}
	\end{proposition}
\end{frame}


\begin{frame}{矩阵秩的性质}
	\onslide<+->
	\begin{proposition}
		\begin{enumerate}
			\setcounter{enumi}{4}
			\item $\rank(\bfA\bfB)\le \min\bigl(\rank(\bfA),\rank(\bfB)\bigr)$.
		\end{enumerate}
	\end{proposition}
	\onslide<+->
	\begin{proof}
		$\bfA\bfB$ 的列向量为 $\bfA$ 列向量组的线性组合, 从而 $\bfA\bfB$ 的列秩 $\le \bfA$ 的列秩, 即 $\rank(\bfA\bfB)\le \rank(\bfA)$.
		\onslide<+->{%
			于是
			\[\rank(\bfA\bfB)=\rank(\bfB^\rmT\bfA^\rmT)\le \rank(\bfB^\rmT)=\rank(\bfB).\qedhere\]
		}
		\vspace{-\baselineskip}
	\end{proof}
	\onslide<+->
	若 $\bfB$ 行满秩, 则 $\bfB$ 有 $\rank(\bfB)$ 阶子式非零, 它对应的方阵右乘 $\bfA$ 得到的列向量组和 $\bfA$ 列向量组等价, 从而 $\rank(\bfA\bfB)=\rank(\bfA)$;
	\onslide<+->
	若 $\bfB$ 列满秩, 则 $\rank(\bfB\bfA)=\rank(\bfA)$;
\end{frame}


\begin{frame}{矩阵秩的性质}
	\onslide<+->
	\begin{proposition}
		\begin{enumerate}
			\setcounter{enumi}{5}
			\item 若 $\bfA_{m\times n}\bfB_{n\times\ell}=\bfO$, 则 $\rank(\bfA)+\rank(\bfB)\le n$.
		\end{enumerate}
	\end{proposition}
	\onslide<+->
	\begin{proof}
		我们将会\S2.4证明空间
		\[V=\{\bfx\in\BR^n\mid \bfA\bfx={\bf0}\}\]
		的维数是 $n-\rank(\bfA)$.
		\onslide<+->{%
		设
		\[W=\{\bfB\bfy\mid\bfy\in \BR^\ell\}\]
		是 $\bfB$ 列向量生成的空间,
		}\onslide<+->{%
		则 $W$ 的维数是 $\rank(\bfB)$.
		由 $\bfA\bfB=\bfO$ 可知 $W\subseteq V$,
		因此 $\rank(\bfB)\le n-\rank(\bfA)$.\qedhere
		}
	\end{proof}
\end{frame}


\begin{frame}{矩阵秩的性质}
	\onslide<+->
	\begin{proposition}
		\begin{enumerate}
			\setcounter{enumi}{6}
			\item $\rank(a\bfA+b\bfB)\le \rank(\bfA,\bfB)\le \rank(\bfA)+\rank(\bfB)$.
			
			特别地, $\max\bigl(\rank(\bfA),\rank(\bfB)\bigr)\le \rank(\bfA,\bfB)$.
		\end{enumerate}
	\end{proposition}
	\onslide<+->
	\begin{proof}
		设 $S$, $T$ 分别为 $\bfA,\bfB$ 列向量组的极大线性无关组, 
		\onslide<+->{%
		则 $S$, $T$ 的大小分别是 $\rank(\bfA)$, $\rank(\bfB)$.
		}\onslide<+->{%
		由于 $(\bfA,\bfB)$ 的列向量组和 $S\cup T$ 向量组等价, 因此
		\[\rank(\bfA,\bfB)=\rank(S\cup T)\le \rank(\bfA)+\rank(\bfB).\]
		}\onslide<+->{%
		由于 $a\bfA+b\bfB$ 的列向量组可以由 $(\bfA,\bfB)$ 的列向量组线性表示, 因此
		\[\rank(a\bfA+b\bfB)\le \rank(\bfA,\bfB).\qedhere\]
		}
		\vspace{-\baselineskip}
	\end{proof}
\end{frame}


\begin{frame}{矩阵秩的性质}
	\onslide<+->
	\begin{proof}[另证]
		由于添加零行或零列不改变秩, 因此不妨设 $\bfA,\bfB$ 都是方阵.
		\onslide<+->{%
		由于
		\[a\bfA+b\bfB=(\bfA,\bfB)\begin{pmatrix}
			a\bfE&\\&b\bfE
		\end{pmatrix},\quad (\bfA,\bfB)=(\bfE,\bfE)\begin{pmatrix}
			\bfA&\\&\bfB
		\end{pmatrix}.\]
		}\onslide<+->{%
		因此 $\rank(a\bfA+b\bfB)\le \rank(\bfA,\bfB)\le \rank(\bfA)+\rank(\bfB)$.
		}
	\end{proof}
\end{frame}


\begin{frame}{矩阵秩性质的应用}
	\onslide<+->
	\begin{exercise}
		\begin{enumerate}
			\item 设 $\rank(\bfA)=2,\bfB=\begin{pmatrix}
				1&0&2\\
				0&2&0\\
				-1&0&3
			\end{pmatrix}$, 则 $\rank(\bfA\bfB)=$\fillblankframe{$2$}.
			\item 若 $\bfA$ 是 $n$ 阶方阵且 $\rank(\bfA\bfB)<\rank(\bfB)$, 则 $|\bfA|=$\fillblankframe{$0$}.
			\item 若 $\bfA=\begin{pmatrix}
				1&0&2\\
				0&1&1\\
				2&0&5
			\end{pmatrix},\bfB=\begin{pmatrix}
				3&4&1\\
				1&2&1\\
				4&6&t
			\end{pmatrix},\bfA\bfX=\bfB$ 且 $\rank(\bfX)=2$, 则 $t=$\fillblankframe{$2$}.
			\item 若 $\bfA=\begin{pmatrix}
				t&2&3\\
				2&1&-1\\
				0&0&5
			\end{pmatrix}$ 且存在非零矩阵 $\bfB$ 使得 $\bfA\bfB=\bfO$, 则 $t=$\fillblankframe{$4$}.
		\end{enumerate}
	\end{exercise}
\end{frame}


\begin{frame}{例: 矩阵秩性质的应用}
	\onslide<+->
	\begin{example}
		证明: 若 $n$ 阶方阵 $\bfA$ 满足 $\bfA^2=\bfA$, 则 $\rank(\bfA)+\rank(\bfA-\bfE)=n$.
	\end{example}
	\onslide<+->
	\begin{proof}
		由于 $\bfA(\bfA-\bfE)=\bfA^2-\bfA=\bfO$, 因此
		\[\rank(\bfA)+\rank(\bfA-\bfE)\le n.\]
		\onslide<+->{%
			由于 $\bfA+(\bfE-\bfA)=\bfE$, 因此
			\[n=\rank(\bfE)\le \rank(\bfA)+\rank(\bfE-\bfA).\]
		}\onslide<+->{%
			故 $\rank(\bfA)+\rank(\bfA-\bfE)=n$.\qedhere
		}
	\end{proof}
\end{frame}


\begin{frame}{例: 矩阵秩性质的应用}
	\onslide<+->
	\begin{algorithm}{伴随矩阵的秩}
		设 $\bfA$ 是 $n$ 阶方阵, 则
		\[\rank(\bfA^*)=\begin{cases}
			n,&\rank(\bfA)=n;\\
			1,&\rank(\bfA)=n-1;\\
			0,&\rank(\bfA)\le n-2.
		\end{cases}\]
	\end{algorithm}
	\onslide<+->
	\begin{proof}
		\begin{enumerate}
			\item 若 $\rank(\bfA)=n$, $\bfA$ 可逆, 从而 $\bfA^*$ 可逆, $\rank(\bfA^*)=n$.
			\item 若 $\rank(\bfA)=n-1$, 由 $\bfA\bfA^*=|\bfA|\bfE=\bfO$ 可知 $\rank(\bfA^*)\le 1$.
			\onslide<+->{由于 $\rank(\bfA)=n-1$, $\bfA$ 存在非零的 $n-1$ 子式, 从而 $\bfA^*\neq\bfO$.
			}\onslide<+->{故 $\rank(\bfA^*)=1$.}
			\item 若 $\rank(\bfA)\le n-2$, 则 $\bfA$ 的 $n-1$ 子式均为零, 从而 $\bfA^*=\bfO$.\qedhere
		\end{enumerate}
	\end{proof}
\end{frame}


\begin{frame}{例: 矩阵秩性质的应用}
	\onslide<+->
	\begin{exercise}
		\begin{enumerate}
			\item 设 $\bma=(1,0,-1,2)^\rmT,\bmb=(0,1,0,2)^\rmT$, 则 $\rank(\bma\bmb^\rmT)=$\fillblankframe{$1$}.
			\item 若 $\bfA=\begin{pmatrix}
				a&b&b\\
				b&a&b\\
				b&b&a
			\end{pmatrix}$ 且 $\rank(\bfA^*)=1$, 则\fillbraceframe{B}.
			\begin{exchoice}(2)
				() $a\neq b,a+2b\neq 0$
				() $a\neq b,a+2b=0$
				() $a=b,a\neq 0$
				() $a=b=0$
			\end{exchoice}
			\item 设 $\bfA,\bfB$ 均为 $n$ 阶非零矩阵, 且 $\bfA\bfB=\bfO$, 则 $\rank(\bfA)$ 与 $\rank(\bfB)$\fillbraceframe{B}.
			\begin{exchoice}(2)
				() 必有一个等于 $0$
				() 都小于 $n$
				() 都等于 $n$
				() 一个小于 $n$, 一个等于 $n$
			\end{exchoice}
		\end{enumerate}
	\end{exercise}
\end{frame}


\begin{frame}{例: 矩阵秩性质的应用}\small
	\onslide<+->
	\begin{exercise}
		\begin{enumerate}
			\setcounter{enumi}{3}
			\item 设 $\bfP$ 为 $3$ 阶非零矩阵, $\bfQ=\begin{pmatrix}
				1&2&3\\
				2&4&t\\
				3&6&9
			\end{pmatrix}$ 且 $\bfP\bfQ=\bfO$, 则\fillbraceframe{A}.
			\begin{exchoice}(2)
				() $t\neq 6$ 时, $\rank(\bfP)=1$
				() $t\neq 6$ 时, $\rank(\bfP)=2$
				() $t=6$ 时, $\rank(\bfP)=1$
				() $t=6$ 时, $\rank(\bfP)=2$
			\end{exchoice}
			\item 设 $\bfA,\bfB$ 为 $n$ 阶方阵, 则\fillbraceframe{A}.
			\begin{exchoice}(2)
				() $\rank(\bfA,\bfA\bfB)=\rank(\bfA)$
				() $\rank(\bfA,\bfB\bfA)=\rank(\bfA)$
				() $\rank(\bfA,\bfA\bfB)=\max\bigl(\rank(\bfA),\rank(\bfB)\bigr)$
				() $\rank(\bfA\bfB)=\rank(\bfA^\rmT\bfB^\rmT)$
			\end{exchoice}
		\end{enumerate}
	\end{exercise}
	\onslide<+->
	\begin{answer}
		存在 $\bfA\bfB=\bfO,\bfB\bfA\neq \bfO$, D 错误.
		令 $\bfA=\bfE$, C 错误.
		$(\bfE,\bfB)$ 行满秩, 选 A.
	\end{answer}
\end{frame}


\begin{frame}{例: 矩阵秩性质的应用}
	\begin{exercise}
		\begin{enumerate}
			\setcounter{enumi}{5}
			\item 设 $\bfA\in M_{m\times n},\bfB\in M_{n\times m}$, 则\fillbraceframe{A}.
			\begin{exchoice}(2)
				() 当 $m>n$ 时, 必有 $|\bfA\bfB|=0$
				() 当 $m>n$ 时, 必有 $|\bfA\bfB|\neq0$
				() 当 $m<n$ 时, 必有 $|\bfA\bfB|=0$
				() 当 $m<n$ 时, 必有 $|\bfA\bfB|\neq0$
			\end{exchoice}
			\item 设 $\bfA\in M_{n\times m},\bfB\in M_{m\times n},n<m$. 若 $\bfA\bfB=\bfE$, 则 $\rank(\bfB)=$\fillblankframe{$n$}.
			\item 若 $\begin{pmatrix}
				1&1&1\\0&1&-1\\2&3&a+2
			\end{pmatrix}$ 和 $\begin{pmatrix}
				1&2&2\\2&1&1\\a+3&a+6&a+4
			\end{pmatrix}$ 等价, 则\fillbraceframe{B}.
			\begin{exchoice}(4)
				() $a=-1$
				() $a\neq-1$
				() $a\neq 1$
				() $a=1$
			\end{exchoice}
			\item 设四阶方阵 $\bfA=(\bma_1,\bma_2,\bma_3,\bma_4)$ 满足 $\bma_1+\bma_2-2\bma_3={\bf0},\bma_2+5\bma_4={\bf0}$, 则 $\rank(\bfA^*)=$\fillblankframe{$0$}.
		\end{enumerate}
	\end{exercise}
\end{frame}


\subsection{极大线性无关组的计算方法}


\begin{frame}{线性相关的不变性}
	\beqskip{6pt}
	\onslide<+->
	\begin{theorem}
		若 $\bfA$ 经过初等行变换变为 $\bfB$, 则
		\begin{enumerate}
			\item $\bfA$ 的行向量组与 $\bfB$ 的行向量组等价;\label{enum:equi-row-vec}
			\item $\bfA$ 任意 $k$ 列和 $\bfB$ 对应的 $k$ 列具有相同的线性相关性.
		\end{enumerate}
	\end{theorem}
	\onslide<+->
	即\alert{初等行变换保持行向量组的等价性, 列向量组的线性组合关系}.
	\onslide<+->
	\begin{proof}
		\enumref{enum:equi-row-vec}我们已经证明过.
		\onslide<+->{%
			设 $\bfB=\bfP\bfA$, 其中 $\bfP$ 是可逆矩阵.
		}\onslide<+->{%
			若 $\bfB\bfx={\bf0}$, 则 $\bfP\bfA\bfx={\bf0}, \bfA\bfx={\bf0}$. 反之亦然, 即 $\bfA\bfx={\bf0}\iff\bfB\bfx={\bf0}$.
		}\onslide<+->{%
			设
			\[\bfA=(\bma_1,\dots,\bma_m),\qquad
			\bfB=(\bmb_1,\dots,\bmb_m).\]
		}\onslide<+->{%
		则对于 $\bfx=(\lambda_1,\dots,\lambda_m)^\rmT$,
		\[\bfA\bfx=\lambda_1\bma_1+\cdots+\lambda_m\bma_m={\bf0}
		\iff 
		\bfB\bfx=\lambda_1\bmb_1+\cdots+\lambda_m\bmb_m={\bf0}.\qedhere\]
		}\vspace{-\baselineskip}
	\end{proof}
	\endgroup
\end{frame}


\begin{frame}{极大线性无关组和秩的计算方法}
	\onslide<+->
	\begin{algorithm}{极大线性无关组和秩的计算方法}
		\begin{enumerate}
			\item 将向量组以列向量形式组成矩阵 $\bfA=(\bma_1,\dots,\bma_m)$.
			\item 通过初等行变换将 $\bfA$ 变为行阶梯形矩阵.
				\begin{itemize}
					\item 行阶梯形矩阵非零行的行数就是秩 $\rank(\bfA)$;
					\item 行阶梯形矩阵每个非零行的首个非零元对应的 $\bfA$ 的列向量, 就是极大线性无关组.
				\end{itemize}
			\item 继续化简为行最简形矩阵, 则可将其余向量表示为极大线性无关组的线性组合.
		\end{enumerate}
	\end{algorithm}
\end{frame}


\begin{frame}{典型例题: 求极大线性无关组}
	\onslide<+->
	\begin{example}
		求下述向量组的秩和一个极大无关组, 并把其余向量用这个极大无关组线性表示:
		\[\bma_1=\begin{pmatrix}
			-7\\-2\\1\\-11
		\end{pmatrix},\ 
		\bma_2=\begin{pmatrix}
			1\\-1\\5\\8
		\end{pmatrix},\ 
		\bma_3=\begin{pmatrix}
			3\\1\\-1\\4
		\end{pmatrix},\ 
		\bma_4=\begin{pmatrix}
			5\\3\\-7\\0
		\end{pmatrix},\ 
		\bma_5=\begin{pmatrix}
			-4\\-2\\1\\-11
		\end{pmatrix}.
		\]
	\end{example}
	\onslide<+->
	\begin{solution}
		\[\bfA=(\bma_1,\bma_2,\bma_3,\bma_4,\bma_5)
		=\begin{pmatrix}
			-7&1&3&5&-4\\
			-2&-1&1&3&-2\\
			1&5&-1&-7&1\\
			-11&8&4&0&-11
		\end{pmatrix}\]
	\end{solution}
\end{frame}


\begin{frame}{典型例题: 求极大线性无关组}
	\onslide<+->
	\begin{solution}[续解]
		\[\wsim{r_1\swap r_3}{}
		\begin{pmatrix}
			1&5&-1&-7&1\\
			-2&-1&1&3&-2\\
			-7&1&3&5&-4\\
			-11&8&4&0&-11
		\end{pmatrix}
		\sim\begin{pmatrix}
			1&5&-1&-7&1\\
			0&9&-1&-11&0\\
			0&36&-4&-44&3\\
			0&63&-7&-77&0
		\end{pmatrix}\]
		\onslide<+->{\[
		\sim\begin{pmatrix}
			1&5&-1&-7&1\\
			0&9&-1&-11&0\\
			0&0&0&0&3\\
			0&0&0&0&0
		\end{pmatrix}
		\sim\begin{pNiceMatrix}
			1&0&-4/9&-8/9&0\\
			0&1&-1/9&-11/9&0\\
			0&0&0&0&1\\
			0&0&0&0&0
			\CodeAfter
			\tikz \draw [cstcurve,second,rounded corners] (1-|1) rectangle (2-|2);
			\tikz \draw [cstcurve,second,rounded corners] (2-|2) rectangle (3-|3);
			\tikz \draw [cstcurve,second,rounded corners] (3-|5) rectangle (4-|6);
		\end{pNiceMatrix}\]}
		\onslide<+->{%
			因此 $\rank(\bfA)=3,\bma_1,\bma_2,\bma_5$ 是一个极大线性无关组, 且
			\[\bma_3=-\frac49 \bma_1-\frac19\bma_2,\quad
			\bma_4=-\frac89 \bma_1-\frac{11}9\bma_2.\]
		}
		\vspace{-\baselineskip}
	\end{solution}
\end{frame}


\begin{frame}{典型例题: 求极大线性无关组}
	\onslide<+->
	\begin{exercise}
		求下述矩阵列向量的一个极大无关组, 并把其余向量用这个极大无关组线性表示:
		\[\bfA=\begin{pmatrix}
			2&-1&-1&1&2\\
			1&1&-2&1&4\\
			4&-6&2&-2&4\\
			3&6&-9&7&9
		\end{pmatrix}\onslide<+->{\simr\begin{pmatrix}
			1&0&-1&0&4\\
			0&1&-1&0&3\\
			0&0&0&1&-3\\
			0&0&0&0&0
		\end{pmatrix}}\]
	\end{exercise}
	\onslide<+->
	\begin{answer}
		设 $\bma_j$ 是 $\bfA$ 的第 $j$ 列, 则 $\bma_1,\bma_2,\bma_4$ 是一个极大线性无关组, 且
		\[\bma_3=-\bma_1-\bma_2,\quad
		\bma_5=4\bma_1+3\bma_2-3\bma_4.\]
	\end{answer}
\end{frame}


\begin{frame}{典型例题: 求极大线性无关组}
	\onslide<+->
	\begin{example}
		假设下述向量组线性相关
		\[\bma_1=(1,1,1,1,2),\ 
		\bma_2=(2,1,3,2,3),\ 
		\bma_3=(2,3,3,2,3),\ 
		\bma_4=(1,3,-1,1,a).\]
		求 $a$, 并求它的秩和一个极大无关组, 并把其余向量用这个极大无关组线性表示.
	\end{example}
	\onslide<+->
	\begin{solution}
		\[\bfA=(\bma_1^\rmT,\bma_2^\rmT,\bma_3^\rmT,\bma_4^\rmT)
		\simr\begin{pmatrix}
			1&0&0&5\\
			0&1&0&-2\\
			0&0&1&0\\
			0&0&0&a-4\\
			0&0&0&0
		\end{pmatrix}.\]
		\onslide<+->{%
			因此 $a=4$, 秩为 $3$, $\bma_1,\bma_2,\bma_3$ 是一个极大线性无关组, 且 $\bma_4=5\bma_1-2\bma_2$.
		}
	\end{solution}
\end{frame}


\begin{frame}{例: 线性相关与线性无关}
	\onslide<+->
	\begin{exercise}
		\begin{enumerate}
			\item 设矩阵 $\bfA$ 经初等行变换化为 $\bfB$, 则二者的\fillbraceframe{A}.
			\begin{exchoice}(1)
				() 行向量组等价, 列向量组同相关性
				() 行向量组同相关性, 列向量组等价
				() 行向量组未必等价, 列向量组同相关性
				() 行向量组等价, 列向量组未必同相关性
			\end{exchoice}
			\item 设 $\bfA\in M_{m\times n},\bfB\in M_{n\times k},\bfA\bfB=\bfO,\bfB\neq \bfO$, 则\fillbraceframe{A}.
			\begin{exchoice}(2)
				() $\bfA$ 的列向量组线性相关
				() $\bfA$ 的行向量组线性相关
				() $\bfA$ 的列向量组线性无关
				() $\bfA$ 的行向量组线性无关
			\end{exchoice}
		\end{enumerate}
	\end{exercise}
\end{frame}


\begin{frame}{例: 线性相关与线性无关}
	\onslide<+->
	\begin{exercise}
		多选题: 设 $\bfA^*$ 是 $n>1$ 阶方阵, 以下说法正确的是\fillbraceframe{ABCD}.
		\begin{exchoice}(1)
			() 若 $\bfA$ 的列向量组线性相关, 则 $\bfA^*$ 的列向量组线性相关
			() 若 $\bfA$ 的列向量组线性无关, 则 $\bfA^*$ 的列向量组线性无关
			() 若 $\bfA^*$ 的某两列向量线性相关, 则 $\bfA$ 的列向量组线性相关
			() 若 $\bfA^*$ 的某两列向量线性无关, 则 $\bfA$ 的列向量组线性无关
		\end{exchoice}
	\end{exercise}
\end{frame}
