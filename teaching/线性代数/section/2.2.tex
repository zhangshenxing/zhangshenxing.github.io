\section{秩与极大线性无关组}


\subsection{秩}

\begin{frame}{基}
	\onslide<+->
	我们知道, $\BR^n$ 中任一向量可以\alert{唯一}表为 $\bfe_1,\dots,\bfe_n$ 的线性组合.
	\onslide<+->
	由此引出基的概念:
	\begin{definition}
		若 $\bma_1,\dots,\bma_m\in V$ 满足: 对任意 $\bfv\in V$, 存在唯一的一组数 $\lambda_1,\dots,\lambda_m$ 使得
		\[\bfv=\lambda_1\bma_1+\cdots+\lambda_m \bma_m,\]
		则称 $\bma_1,\dots,\bma_m$ 是 $V$ 的一组\emph{基}.
	\end{definition}
\end{frame}


\begin{frame}{基的等价刻画}
	\onslide<+->
	\begin{proposition}
		$\bma_1,\dots,\bma_m$ 是 $V$ 的基向量组$\iff$该向量组生成 $V$ 且线性无关.
	\end{proposition}
	\onslide<+->
	\begin{proof*}
		显然基向量组生成 $V$.
		\onslide<+->{%
		由于零向量只有唯一的线性表示方式, 因此基总是线性无关的一组向量.
		}

		\onslide<+->{%
		反过来, 假设向量组 $\bma_1,\dots,\bma_m$ 线性无关且生成 $V$.
		}\onslide<+->{%
			若 $\bfv\in V$ 有两种线性表达形式, 二式相减得到不全为零的数 $\lambda_1,\dots,\lambda_m$ 使得
			\[\lambda_1\bma_1+\cdots+\lambda_m\bma_m={\bf0}.\]
		}\onslide<+->{%
			这与 $\bma_1,\dots,\bma_s$ 线性无关矛盾.
		}\onslide<+->{%
			因此 $\forall\bfv\in V$ 只有唯一的一种线性表达形式.\qedhere
		}
	\end{proof*}
\end{frame}


\begin{frame}{维数}
	\onslide<+->
	\begin{definition}
		设 $V\subseteq \BR^n$ 是一个线性子空间.
		若存在(线性)双射 $f:\BR^r\to V$ 满足
		\begin{enumerate}[<*>]
			\item $f(\bfu+\bfv)=f(\bfu)+f(\bfv),\forall \bfu,\bfv\in\BR^r$;
			\item $f(\lambda\bfv)=\lambda f(\bfv),\forall \lambda\in\BR,\bfv\in\BR^r$,
		\end{enumerate}
		则称 $V$ 的\emph{维数}为 $r$, 记为 $r=\dim V$.
	\end{definition}
	\onslide<+->
	在逆矩阵一节我们已经知道 $m\neq n$ 时, $\BR^m$ 和 $\BR^n$ 之间没有线性双射.
	\onslide<+->
	因此子空间维数总是唯一的.

	\onslide<+->
	设 $\bma_1,\dots,\bma_m$ 是 $V$ 的一组基.
	\onslide<+->
	定义
	\begin{align*}
		f:\BR^m&\lra V\\
		(a_1,\dots,a_m)&\lra a_1\bma_1+\cdots+a_m\bma_m
	\end{align*}
	\onslide<+->
	则该映射是线性双射.
	\onslide<+->
	因此子空间的\alert{维数等于基的大小}.
\end{frame}


\begin{frame}{秩}
	\onslide<+->
	\begin{definition}
		设向量组 $\bma_1,\dots,\bma_m$ 生成空间 $V$.
		称 $V$ 的维数为该向量组的\emph{\cnen{秩}{Rank}}, 记作 $R(\bma_1,\dots,\bma_m)$.
	\end{definition}
	\onslide<+->
	由于该向量组线性无关当且仅当它们构成一组基,
	\onslide<+->
	因此:
	\begin{theorem}
		\begin{enumerate}
			\item 向量组 $\bma_1,\dots,\bma_m$ 线性无关$\iff m=R(\bma_1,\dots,\bma_m)$.
			\item 向量组 $\bma_1,\dots,\bma_m$ 线性相关$\iff m>R(\bma_1,\dots,\bma_m)$.
		\end{enumerate}
	\end{theorem}
\end{frame}


\begin{frame}{子空间维数关系}
	\onslide<+->
	设 $W\subseteq V$ 是两个子空间, $S=\{\bma_1,\dots,\bma_s\}$ 是 $W$ 的一组基, $T=\{\bmb_1,\dots,\bmb_t\}$ 是 $V$ 的一组基.
	\onslide<+->
	则 $S$ 可由 $T$ 线性表示.
	\onslide<+->
	由于 $S$ 线性无关, 因此 $s=\dim W\le t=\dim V$.
	\onslide<+->
	于是:
	\begin{theorem}
		设向量组 $S$ 可由向量组 $T$ 线性表示, 则 $R(S)\le R(T)$.
	\end{theorem}
\end{frame}


\begin{frame}{例: 向量组的秩}
	\begin{exercise}
		\begin{enumerate}[<*>]
			\item 若任一 $3$ 维向量都可由向量组
			\[\bma_1=(a,3,2)^\rmT,\quad
			\bma_2=(2,-1,3)^\rmT,\quad
			\bma_3=(3,2,1)^\rmT,\]
			线性表示, 则 $a\neq$\fillblankframe{\visible<2->{$5$}}.
			\item 判断题: 设 $S$ 和 $T$ 为两个 $n$ 维向量组, 且 $R(S)=R(T)$, 则 $S$ 和 $T$ 等价.\visible<3->{\alert{\text{\huge$\times$}}}
		\end{enumerate}
	\end{exercise}
\end{frame}


\begin{frame}{例: 向量组的秩}
	\onslide<+->
	\begin{example}
		若 $S_1=\{\bma_1,\bma_2,\bma_3\}$, $S_2=\{\bma_1,\bma_2,\bma_3,\bma_4\}$, $S_3=\{\bma_1,\bma_2,\bma_3,\bma_4,\bma_5\}$ 满足 $R(S_1)=R(S_2)=3, R(S_3)=4$.
		证明向量组 $S=\{\bma_1,\bma_2,\bma_3,\bma_5-\bma_4\}$ 线性无关.
	\end{example}
	\onslide<+->
	\begin{proof}
		由 $R(S_1)=3$ 可知 $S_1$ 线性无关.
		\onslide<+->{%
			由 $R(S_2)=3$ 可知 $S_2$ 线性相关.
		}\onslide<+->{%
			从而 $\bma_4$ 可由 $S_1$ 线性表示.
		}\onslide<+->{%
			于是 $S_3$ 可由 $S$ 线性表示.
		}\onslide<+->{%
			显然 $S$ 可由 $S_3$ 线性表示, 因此二者等价, $R(S)=R(S_3)=4$.\qedhere
		}
	\end{proof}
\end{frame}


\subsection{极大线性无关组}
\begin{frame}{极大线性无关组的定义}
	\onslide<+->
	\begin{definition}
		设 $S$ 为一个向量组.
		若 $S$ 的部分组 $S_0=\{\bma_1,\dots,\bma_m\}$ 满足
		\begin{enumerate}[<*>]
			\item $S_0$ 线性无关;
			\item $S_0$ 添加 $S$ 中的若干向量得到的向量组均线性相关.
		\end{enumerate}
		则称 $S_0$ 是 $S$ 的一个\emph{极大线性无关组}.
	\end{definition}
	\onslide<+->
	根据上一节相关结论可知, $S$ 中所有向量均可由 $S_0$ 线性表示.
	\onslide<+->
	换言之, $S_0$ 和 $S$ 等价, 它们生成相同的子空间 $V$, $m=R(S)$, $S_0$ 是 $V$ 的一组基.
\end{frame}


\begin{frame}{极大线性无关组的定义}
	\onslide<+->
	\begin{theorem}
		$S_0$ 是 $S$ 的极大线性无关组当且仅当
		\begin{enumerate}
			\item $S_0$ 线性无关;
			\item $S$ 中任意 $m+1$ 个向量线性相关.
		\end{enumerate}
	\end{theorem}
	\onslide<+->
	\begin{proof*}
		若 $S_0$ 是 $S$ 的极大线性无关组, 则 $S$ 中任意 $m+1$ 个向量可由 $S_0$ 线性表示.
		\onslide<+->{%
			从而线性相关.
		}

		\onslide<+->{%
			反之, 若 $S$ 中任意 $m+1$ 个向量线性相关, 则 $S$ 中任意 $s>m$ 个向量线性相关.
		}\onslide<+->{%
			于是 $S_0$ 添加 $S$ 中的若干向量得到的向量组均线性相关.\qedhere
		}
	\end{proof*}
\end{frame}


\begin{frame}{极大线性无关组的性质}
	\begin{enumerate}\bf
		\item 若 $R(S)=r$, 则 $S$ 中任意 $r$ 个线性无关的向量构成 $S$ 的一个极大线性无关组.
		\item 只含有零向量的向量组没有极大线性无关组(空集), 它的秩为 $0$ (空集生成 $0$ 维空间 $\{{\bf0}\}$).
		\item 极大线性无关组一般不是唯一的.
	\onslide<+->{%
		\mdseries 例如
		\[\bma_1=(1,0)^\rmT,\quad\bma_2=(0,1)^\rmT,\quad\bma_3=(1,1)^\rmT.\]}
		\onslide<+->{%
		$\bma_1,\bma_2$ 是一个极大线性无关组, $\bma_1,\bma_3$ 也是一个极大线性无关组.}
		\item 向量组和它的一个极大线性无关组是等价的, 于是同一向量组的任意两个极大线性无关组等价.
	\end{enumerate}
\end{frame}


\begin{frame}{例: 极大线性无关组}
	\onslide<+->
	\begin{example}
		矩阵 $\bfA=\begin{pmatrix}
			1&1&3&1\\
			0&1&-1&4\\
			0&0&0&5\\
			0&0&0&0
		\end{pmatrix}$
		的行向量组为
		\vspace{-.5\baselineskip}
		\[\bma_1^\rmT=(1,1,3,1),
			\bma_2^\rmT=(0,1,-1,4),
			\bma_3^\rmT=(0,0,0,5),
			\bma_4^\rmT=(0,0,0,0).
		\]
		\onslide<+->{%
			由于 $\bma_1^\rmT,\bma_2^\rmT,\bma_3^\rmT$ 的第 $1,2,4$ 个分量形成可逆矩阵 $\begin{pmatrix}
				1&1&1\\
				0&1&4\\
				0&0&5
			\end{pmatrix}$, 因此它们线性无关.
		}\onslide<+->{%
			它们构成一个极大线性无关组, $\bfA$ 的行向量组的秩是 $3$.
		}\onslide<+->{%
			类似可知, $\bfA$ 的列向量组的秩也是 $3$.
		}
	\end{example}
	\onslide<+->
	实际上, 任意矩阵的行向量组的秩等于列向量组的秩.
	\onslide<+->
	为了说明这一点, 我们需要先研究矩阵的变换.
\end{frame}
