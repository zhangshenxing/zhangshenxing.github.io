\section{标准正交基}

\subsection{向量的内积}

\begin{frame}{引例: 更好的基}
	\onslide<+->
	本节考虑的向量都是实向量.

	\onslide<+->
	设向量组 $S=\{\bma_1,\dots,\bma_s\}$ 的秩为 $r$, 则它们生成的线性空间 $V$ 的维数就是 $r$.
	\onslide<+->
	$S$ 的极大无关组 $S_0$ 的大小就是 $r$, 且 $S_0$ 是 $V$ 的一组基.

	\onslide<+->
	有时候我们想更进一步, 就像 $\BR^n$ 的标准正交基 $\bfe_1,\dots,\bfe_n$ 一样, 我们希望找到 $V$ 的一组基 $\bma_1,\dots,\bma_r$ 使得
	\begin{enumerate}
		\item $\bma_i$ \alert{长度}都是 $1$;
		\item $\bma_1,\dots,\bma_r$ 两两\alert{垂直}.
	\end{enumerate}
\end{frame}


\begin{frame}{内积}
	\onslide<+->
	\begin{definition}
		设 $\bma=(a_1,\dots,a_n)^\rmT,\bmb=(b_1,\dots,b_n)^\rmT\in\BR^n$, 定义\emph{内积}
		\[[\bma,\bmb]=\bma^\rmT\bmb=\bmb^\rmT\bma=a_1b_1+\cdots+a_nb_n\in\BR.\]
	\end{definition}
	\onslide<+->
	内积是数量积的推广, 它满足
	\begin{enumerate}
		\item $[\bma,\bmb]=[\bmb,\bma]$;
		\item $[\lambda\bma,\bmb]=[\bma,\lambda\bmb]=\lambda[\bma,\bmb]$;
		\item $[\bma+\bmb,\bmg]=[\bma,\bmg]+[\bmb,\bmg]$;
		\item $[\bma,\bma]\ge 0$. 当且仅当 $\bma={\bf0}$ 时, $[\bma,\bma]=0$.
	\end{enumerate}
	\onslide<+->
	这说明内积是一个对称正定双线性型.
\end{frame}


\begin{frame}{长度}
	\onslide<+->
	\begin{definition}
		设 $\bfx=(x_1,\dots,x_n)\in\BR^n$, 定义 $\bfx$ 的\emph{长度}或\emph{模}为
		\[\|\bfx\|=\sqrt{x_1^2+\cdots+x_n^2}.\]
		当 $\|\bfx\|=1$ 时, 称 $\bfx$ 为\emph{单位向量}.
		对于非零向量 $\bfx$, \alert{$\dfrac{\bfx}{\|\bfx\|}$} 为 $\bfx$ 的\emph{单位化向量}.
	\end{definition}
	\onslide<+->
	我们有 $\bfx={\bf0}\iff\|\bfx\|=0\iff[\bfx,\bfx]=0$.
	\onslide<+->
	\begin{definition}
		若 $[\bma,\bmb]=0$, 称 $\bma,\bmb$ \emph{正交}(垂直).
	\end{definition}
\end{frame}


\begin{frame}{柯西-施瓦兹不等式\noexer}
	\onslide<+->
	设 $\bma\neq{\bf0}$.
	\onslide<+->
	那么 $X$ 的二次多项式
	\[\|X\bma+\bmb\|^2=[X\bma+\bmb,X\bma+\bmb]
	=[\bma,\bma]X^2+2[\bma,\bmb]X+[\bmb,\bmb]\ge0\]
	恒成立.
	\onslide<+->
	因此其判别式
	\[\Delta=4([\bma,\bmb]^2-\|\bma\|^2\cdot\|\bmb\|^2)\le 0,\]
	\onslide<+->
	于是我们得到\emph{柯西-施瓦兹不等式}
	\[\pm[\bma,\bmb]\le \|\bma\|\cdot\|\bmb\|.\]
	\onslide<+->
	显然 $\bma={\bf0}$ 时它也成立.
\end{frame}


\begin{frame}{夹角}
	\onslide<+->
	\begin{definition}
		设非零向量 $\bma=(a_1,\dots,a_n)^\rmT,\bmb=(b_1,\dots,b_n)^\rmT\in\BR^n$, 定义 $\bma,\bmb$ 的\emph{夹角}为
		\[\theta=\arccos\dfrac{[\bma,\bmb]}{\|\bma\|\cdot\|\bmb\|}\in[0,\pi].\]
	\end{definition}
	\onslide<+->
	注意正交比夹角为 $\dfrac\pi2$ 略微广泛点, 因为零向量与任意向量正交.

	\onslide<+->
	若 $\bma$ 与 $\bmb$ 正交, 则
	\[\|\bma+\bmb\|^2=\|\bma\|^2+\|\bmb\|^2+2[\bma,\bmb]=\|\bma\|^2+\|\bmb\|^2,\]
	\onslide<+->
	此即\emph{勾股定理}.
\end{frame}

\subsection{正交向量组与格拉姆-施密特正交化}

\begin{frame}{例: 正交向量}
	\onslide<+->
	\begin{definition}
		\begin{enumerate}
			\item 若向量组 $S$ 中的向量两两正交且非零, 则称 $S$ 为正交向量组.
			\item 若向量组 $S$ 中的向量两两正交且均为单位向量, 则称 $S$ 为标准正交向量组.
		\end{enumerate}
	\end{definition}
	\onslide<+->
	\begin{example}
		设 $\bma_1=(1,1,1)^\rmT,\bma_2=(1,-2,1)^\rmT\in\BR^3$.
		求向量 $\bma_3$ 使得 $\bma_1,\bma_2,\bma_3$ 是正交向量组.
	\end{example}
	\onslide<+->
	\begin{solution}
		显然 $\bma_1,\bma_2$ 正交.
		\onslide<+->{%
			设 $\bma_3=(x_1,x_2,x_3)^\rmT$, 则
			\begin{align*}
				[\bma_1,\bma_3]&=x_1+x_2+x_3=0,\\
				[\bma_2,\bma_3]&=x_1-2x_2+x_3=0.
			\end{align*}
		}\onslide<+->{%
			解得 $(x_1,x_2,x_3)=(k,0,-k)$.
		}\onslide<+->{%
			故可取 $\bma_3=(1,0,-1)^\rmT$.
		}
	\end{solution}
\end{frame}


\begin{frame}{正交向量组必线性无关}
	\onslide<+->
	\begin{theorem}
		正交向量组必线性无关.
	\end{theorem}
	\onslide<+->
	\begin{proof}
		设 $\bma_1,\dots,\bma_r$ 是正交向量组,
		$\lambda_1\bma_1+\cdots+\lambda_r\bma_r={\bf0}$.
		\onslide<+->{%
			对任意 $1\le i\le r$,
			\[0=[{\bf0},\bma_i]=[\lambda_1\bma_1+\cdots+\lambda_r\bma_r,\bma_i]=\lambda_i[\bma_i,\bma_i].\]
		}\onslide<+->{%
			由于 $\bma_i$ 非零, $[\bma_i,\bma_i]\neq 0,\lambda_i=0$.
		}\onslide<+->{%
			故 $\bma_1,\dots,\bma_r$ 线性无关.\qedhere
		}
	\end{proof}
	
	\onslide<+->
	现在我们来看如何从空间 $V$ 的一组基 $\bma_1,\dots,\bma_r$ 得到一组标准正交基.
	\onslide<+->
	令 $\bmb_1=\bma_1$.
	\onslide<+->
	若 $\bmb_1,\dots,\bmb_k$ 已经是两两正交的单位向量, 设 $\bmb_{k+1}=\bma_{k+1}+\lambda_1\bmb_1+\dots+\lambda_k\bmb_k$ 与它们均正交, 
	\onslide<+->
	则对于 $i=1,\dots,k$,
	\[[\bmb_i,\bmb_{k+1}]=[\bmb_i,\bma_{k+1}]+\lambda_i[\bmb_i,\bmb_i]=0\implies\lambda_i=-\frac{[\bma_{k+1},\bmb_i]}{[\bmb_i,\bmb_i]}.\]
\end{frame}


\begin{frame}{格拉姆-施密特正交化}
	\onslide<+->
	由此得到\alert{格拉姆-施密特正交单位化方法}: 取
		\begin{align*}
			\bmb_1&=\bma_1\\
			\bmb_2&=\bma_2-\frac{[\bma_2,\bmb_1]}{[\bmb_1,\bmb_1]}\bmb_1\\
			\bmb_3&=\bma_3-\frac{[\bma_3,\bmb_1]}{[\bmb_1,\bmb_1]}\bmb_1-\frac{[\bma_3,\bmb_2]}{[\bmb_2,\bmb_2]}\bmb_2\\
			&\vdots\\
			\bmb_r&=\bma_r-\frac{[\bma_r,\bmb_1]}{[\bmb_1,\bmb_1]}\bmb_1-\cdots-\frac{[\bma_r,\bmb_{r-1}]}{[\bmb_{r-1},\bmb_{r-1}]}\bmb_{r-1}
		\end{align*}
	\onslide<+->
	则 $\bfe_1=\dfrac{\bmb_1}{\|\bmb_1\|},\dots,\bfe_r=\dfrac{\bmb_r}{\|\bmb_r\|}$ 就是 $V$ 的一组标准正交基.
\end{frame}


\begin{frame}{典型例题: 格拉姆-施密特正交化}\small
	\onslide<+->
	\begin{example}
		将 $\bma_1=(1,1,0)^\rmT,\bma_2=(1,0,1)^\rmT,\bma_3=(1,1,2)^\rmT$ 正交单位化.
	\end{example}
	\onslide<+->
	\begin{solution}
		\begin{align*}
			\bmb_1&=\bma_1=(1,1,0)^\rmT\\
			\visible<+->{\bmb_2}&\visible<.->{=\bma_2-\frac{[\bma_2,\bmb_1]}{[\bmb_1,\bmb_1]}\bmb_1=(1,0,1)^\rmT-\frac12(1,1,0)^\rmT=(\frac12,-\frac12,1)^\rmT}\\
			\visible<+->{\bmb_3}&\visible<.->{=\bma_3-\frac{[\bma_3,\bmb_1]}{[\bmb_1,\bmb_1]}\bmb_1-\frac{[\bma_3,\bmb_2]}{[\bmb_2,\bmb_2]}\bmb_2}
			\visible<+->{\!=\!(1,1,2)^\rmT-(1,1,0)^\rmT-\frac2{3/2}(\frac12,-\frac12,1)^\rmT
			\!=\!(\frac23,-\frac23,\frac23)^\rmT}
		\end{align*}
		\onslide<+->{%
			\[
			\bfe_1=\dfrac{\bfb_1}{\|\bfb_1\|}
				=\frac1{\sqrt2}(1,1,0)^\rmT,\quad
			\bfe_2=\dfrac{\bfb_2}{\|\bfb_2\|}
				=\frac1{\sqrt6}(1,-1,2)^\rmT,\quad
			\bfe_3=\dfrac{\bfb_3}{\|\bfb_3\|}
				=\frac1{\sqrt3}(-1,1,1)^\rmT.
		\]}
	\end{solution}
\end{frame}

