\section{逆矩阵}

\subsection{方阵的伴随矩阵}

\begin{frame}{伴随矩阵的定义}
	\onslide<+->
	\begin{definition}
		设 $\bfA=(a_{ij})$ 为 $n$ 阶方阵.
		由 $\bfA$ 的代数余子式形成的 $n$ 阶方阵
		\[\bfA^*=(A_{ji})=\begin{pmatrix}
			A_{11}&A_{21}&\cdots&A_{n1}\\
			A_{12}&A_{22}&\cdots&A_{n2}\\
			\vdots&\vdots&\ddots&\vdots\\
			A_{1n}&A_{2n}&\cdots&A_{nn}
		\end{pmatrix}\]
		称为 $\bfA$ 的\emph{伴随矩阵}.
	\end{definition}
	\onslide<+->
	注意, 伴随矩阵的 $(i,j)$ 元是代数余子式 \alert{$A_{ji}$ 而不是 $A_{ij}$}.

	\onslide<+->
	{\itshape $\bfA^*$ 对应的线性变换是 $\bfA$ 对应的线性变换 $\BR^n\ra\BR^n$ 诱导的 $n-1$ 次外代数上的线性变换 $\wedge^{n-1}\BR^n\to\wedge^{n-1}\BR^n$, 感兴趣的可自行阅读有关材料.}
\end{frame}


\begin{frame}{伴随矩阵的性质}
	\beqskip{3pt}
	\onslide<+->
	\begin{example}
		若 $\bfA=\begin{pmatrix}
			a&b\\c&d
		\end{pmatrix}$, 则 $\bfA^*=\begin{pmatrix}
			d&-b\\-c&a
		\end{pmatrix}$.
	\end{example}
	\onslide<+->
	伴随矩阵满足如下性质:
	\begin{enumerate}\bf
		\item \alert{$\bfA\bfA^*=\bfA^*\bfA=|\bfA|\bfE_n$.}
	\end{enumerate}
	\onslide<+->
	这是因为
	\[\bfA\bfA^*=\begin{pmatrix}
		a_{11}&a_{12}&\cdots &a_{1n}\\
		a_{21}&a_{22}&\cdots &a_{2n}\\
		\vdots&\vdots&\ddots&\vdots\\
		a_{n1}&a_{n2}&\cdots &a_{nn}
	\end{pmatrix}\begin{pmatrix}
		A_{11}&A_{21}&\cdots&A_{n1}\\
		A_{12}&A_{22}&\cdots&A_{n2}\\
		\vdots&\vdots&\ddots&\vdots\\
		A_{1n}&A_{2n}&\cdots&A_{nn}
	\end{pmatrix}\]
	\onslide<+->
	的 $(i,j)$ 元是
	$\displaystyle a_{i1}A_{j1}+\cdots+a_{in}A_{jn}=\begin{cases}
		|\bfA|,&i=j;\\
		0,&i\neq j.
	\end{cases}$
	\endgroup
\end{frame}


\begin{frame}{伴随矩阵的性质}
	\onslide<+->
	\begin{enumerate}\bf
		\setcounter{enumi}{1}
		\item $(k\bfA)^*=k^{n-1}\bfA^*$.
		\item $(\bfA^\rmT)^*=(\bfA^*)^\rmT$.
		\item \alert{$|\bfA^*|=|\bfA|^{n-1}$.}
	\end{enumerate}
	\onslide<+->
	若 $|\bfA|\neq0$, 由 $|\bfA^*|\cdot|\bfA|=|\bfA^*\bfA|=\bigl||\bfA|\bfE_n\bigr|=|\bfA|^n$ 可得.

	\onslide<+->
	若 $|\bfA|=0$ 但 $|\bfA^*|\neq 0$, 则
	\[\bfA\bfA^*(\bfA^*)^*=|\bfA|(\bfA^*)^*=\bfO
	\onslide<+->{=|\bfA^*|\bfA,}\]
	\onslide<+->
	于是 $\bfA=\bfO,\bfA^*=\bfO$, 矛盾! 因此 $|\bfA^*|=0$.
\end{frame}


\begin{frame}{伴随矩阵的伴随}
	\onslide<+->
	\begin{enumerate}\bf
		\setcounter{enumi}{4}
		\item $(\bfA^*)^*=\begin{cases}
			\bfA,&n=2;\\
			|\bfA|^{n-2}\bfA,&n\ge3.
		\end{cases}$
	\end{enumerate}
	\onslide<+->
	容易知道, $2$ 阶方阵满足 $(\bfA^*)^*=\bfA$.
	\onslide<+->
	若 $\bfA$ 是 $n\ge3$ 阶方阵且 $|\bfA|\neq 0$, 则由
	\[\bfA\bfA^*(\bfA^*)^*=|\bfA|(\bfA^*)^*=|\bfA^*|\bfA\]
	可知
	\[(\bfA^*)^*=\frac{|\bfA^*|}{|\bfA|}\bfA=|\bfA|^{n-2}\bfA.\]
	\onslide<+->
	若 $\bfA$ 是 $n\ge3$ 阶方阵且 $|\bfA|=0$, 我们会在\S2.4证明 $(\bfA^*)^*=\bfO$.
\end{frame}


\begin{frame}{例: 伴随矩阵的性质}
	\onslide<+->
	\begin{example}
		设非零 $n\ge 3$ 实方阵 $\bfA$ 满足对任意 $i,j$, $a_{ij}=A_{ij}$. 求 $|\bfA|$.
	\end{example}
	\onslide<+->
	\begin{solution}
		由题设可知 $\bfA^*=\bfA^\rmT$.
		\onslide<+->{%
			因此 $|\bfA|=|\bfA^*|=|\bfA|^{n-1}$, $|\bfA|=0,1$ 或 $-1$.
		}

		\onslide<+->{%
			注意 $\bfA\bfA^\rmT=\bfA\bfA^*=|\bfA|\bfE$ 的第 $i$ 个对角元
			\[\sum_{k=1}^n a_{ik}^2\ge 0.\]
		}\onslide<+->{%
			因此 $|\bfA|\ge0$.
		}\onslide<+->{%
			若 $|\bfA|=0$, 则所有的 $a_{ik}=0,\bfA=\bfO$, 矛盾! 因此 $|\bfA|=1$.
		}
	\end{solution}
	\onslide<+->
	事实上, $\bfA$ 是\emph{正交阵}, 即满足 $\bfA^\rmT\bfA=\bfE$ 的实方阵.
\end{frame}



\subsection{逆矩阵的定义和形式}
\begin{frame}{线性变换的逆}
	\onslide<+->
	给定一个线性变换 $f:\BR^n\ra\BR^m$, 若线性变换 $g:\BR^m\to\BR^n$ 满足
	\[(gf)(\bfx)=\bfx,\forall \bfx\in\BR^n,\quad
	(fg)(\bfy)=\bfy,\forall \bfy\in\BR^m,\]
	\onslide<+->
	则称 $g$ 是 $f$ 的逆.

	\onslide<+->
	设 $f,g$ 对应的矩阵分别是 $\bfA,\bfB$, 则
	\[\bfA\bfB=\bfE_m,\quad \bfB\bfA=\bfE_n.\]
	\onslide<+->
	由于 $m>n$ 时 $|\bfA\bfB|=0$; $m<n$ 时 $|\bfB\bfA|=0$.
	\onslide<+->
	因此线性变换的逆只可能在 $m=n$ 时存在.
\end{frame}


\begin{frame}{逆矩阵的定义和唯一性}
	\onslide<+->
	由此得到对应的矩阵的逆的定义:
	\onslide<+->
	\begin{definition}
		设 $\bfA$ 是 $n$ 阶方阵. 若存在 $n$ 阶方阵 $\bfB$ 使得
		\[\bfA\bfB=\bfB\bfA=\bfE_n,\]
		则称 $\bfA$ 是\emph{可逆矩阵}, $\bfB$ 是 $\bfA$ 的\emph{逆矩阵}.
	\end{definition}
	\onslide<+->
	可逆矩阵的逆矩阵唯一吗?
	\onslide<+->
	设 $\bfB,\bfB'$ 都是 $\bfA$ 的逆矩阵, 则
	\[\bfA\bfB=\bfE_n,\quad \bfB'\bfA=\bfE_n.\]
	\onslide<+->
	于是
	\[\bfB=(\bfB'\bfA)\bfB=\bfB'(\bfA\bfB)=\bfB'.\]
	\onslide<+->
	因此\alert{若逆矩阵存在必唯一, 记为 $\bfA^{-1}$}.
\end{frame}


\begin{frame}{逆矩阵存在性的刻画}
	\onslide<+->
	若 $\bfA$ 可逆, 从 $\bfA\bfA^{-1}=\bfE$ 可知
	\[|\bfA|\cdot|\bfA^{-1}|=1.\]
	\onslide<+->
	从而 $|\bfA|\neq 0$, $\bfA$ 非退化.

	\onslide<+->
	反之, 若 $\bfA$ 非退化, 则
	\[\bfA\bfA^*=\bfA^*\bfA=|\bfA|\bfE,\qquad
	\visible<+->{\bfA\cdot\frac1{|\bfA|}\bfA^*=\frac1{|\bfA|}\bfA^*\cdot\bfA=\bfE,}\qquad
	\visible<+->{\bfA^{-1}=\frac1{|\bfA|}\bfA^*.}\]
	\vspace{-.5\baselineskip}
	\onslide<+->
	\begin{second}{逆矩阵的存在性和形式}
		$n$ 阶方阵 $\bfA$ 可逆当且仅当 $|\bfA|\neq 0$.
		此时 $\bfA^{-1}=\dfrac1{|\bfA|}\bfA^*$.
	\end{second}
\end{frame}


\begin{frame}{逆矩阵存在性的刻画}
	\onslide<+->
	\begin{corollary}
		设 $\bfA,\bfB$ 为 $n$ 阶方阵.
		若 $\bfA\bfB=\bfE$ (或 $\bfB\bfA=\bfE$), 则 $\bfB=\bfA^{-1}$.
	\end{corollary}
	\onslide<+->
	\begin{proof}
		若 $\bfA\bfB=\bfE$, 则 $|\bfA|\cdot|\bfB|=1$, $|\bfA|\neq0$.
		\onslide<+->{%
			因此 $\bfA$ 可逆.
		}\onslide<+->{%
			\[\bfA^{-1}=\bfA^{-1}(\bfA\bfB)=\bfB.\qedhere\]
		}\vspace{-\baselineskip}
	\end{proof}

	\onslide<+->
	称行列式非零的方阵\emph{非退化}, 行列式为零的方阵\emph{退化}.
	\onslide<+->
	令 $\bfe_i$ 表示第 $i$ 个分量为 $1$ 其余为 $0$ 的 $n$ 维向量.
	\onslide<+->
	若方程
	\[\bfA\bfx_1=\bfe_1,\quad\dots,\quad\bfA\bfx_n=\bfe_n\]
	均有解, 则这些解形成 $\bfA^{-1}=(\bfx_1,\dots,\bfx_n)$.
	\onslide<+->
	若 $\bfA$ 退化, 至少有一个方程 $\bfA\bfx=\bfe_j$ 无解.
	\onslide<+->
	从而 $\bfA$ 对应的线性映射 $f:\BR^n\to\BR^n$ 不是满射, 这便是``退化''的含义.
\end{frame}


\begin{frame}{例: 计算逆矩阵}
	\onslide<+->
	\begin{example}
		证明 $\bfA=\begin{pmatrix}
			\cos\theta&-\sin\theta\\
			\sin\theta&\cos\theta
		\end{pmatrix}$ 可逆并求其逆矩阵.
	\end{example}
	\onslide<+->
	\begin{solution}
		由于 $|\bfA|=\cos^2\theta+\sin^2\theta=1$, 因此 $\bfA$ 可逆.
		\onslide<+->{%
			由于 $\bfA^*=\begin{pmatrix}
				\cos\theta&\sin\theta\\
				-\sin\theta&\cos\theta
			\end{pmatrix}$, 因此
			\[\bfA^{-1}=\frac1{|\bfA|}\bfA^*=\begin{pmatrix}
				\cos\theta&\sin\theta\\
				-\sin\theta&\cos\theta
			\end{pmatrix}.\]
		}\vspace{-\baselineskip}
	\end{solution}
	\onslide<+->
	注意到 $\bfA$ 对应的是平面上沿原点逆时针旋转 $\theta$, 因此 $\bfA^{-1}$ 对应的是平面上沿原点逆时针旋转 $-\theta$.
\end{frame}


\begin{frame}{例: 计算逆矩阵}\small
\beqskip{5pt}
	\onslide<+->
	\begin{example}
		$\bfA=\begin{pmatrix}
			1&2&1\\
			0&1&-1\\
			-1&3&4
		\end{pmatrix}, \bfB=\begin{pmatrix}
			2&7&-2\\
			-1&-3&5\\
			1&5&11
		\end{pmatrix}$ 是否可逆? 若可逆求其逆矩阵.
	\end{example}
	\onslide<+->
	\begin{solution}
		由于 $|\bfA|\xeq{\nsmath{r_3+r_1}}\begin{vmatrix}
			1&2&1\\
			0&1&-1\\
			0&5&5
		\end{vmatrix}=10$, 因此 $\bfA$ 可逆,
		\onslide<+->{%
			\[\bfA^{-1}=\frac1{10}\bfA^*
			=\frac1{10}\begin{pmatrix}
				A_{11}&A_{21}&A_{31}\\
				A_{12}&A_{22}&A_{32}\\
				A_{13}&A_{23}&A_{33}
			\end{pmatrix}
			\visible<+->{=\frac1{10}\begin{pmatrix}
				7&-5&-3\\
				1&5&1\\
				1&-5&1
			\end{pmatrix}.}\]
		}\onslide<+->{%
			由于 $|\bfB|\xeq[\nsmath{c_2-5c_1}]{\nsmath{c_3-11c_1}}\begin{vmatrix}
				2&-3&-24\\
				-1&2&16\\
				1&0&0
			\end{vmatrix}=\begin{vmatrix}
				-3&-24\\
				2&16
			\end{vmatrix}=0$, 因此 $\bfB$ 不可逆.
		}
	\end{solution}
\endgroup
\end{frame}


\begin{frame}{例: 计算逆矩阵}
	\onslide<+->
	\begin{example}
		设 $\lambda_1,\dots,\lambda_n\neq0$,
		\[\bmL=\begin{pmatrix}
			\lambda_1&&&\\
			&\lambda_2&&\\
			&&\ddots&\\
			&&&\lambda_n\\
		\end{pmatrix}=\diag(\lambda_1,\lambda_2,\dots,\lambda_n),\]
		\onslide<+->{则
		\[\bmL^{-1}=\begin{pmatrix}
			\lambda_1^{-1}&&&\\
			&\lambda_2^{-1}&&\\
			&&\ddots&\\
			&&&\lambda_n^{-1}\\
		\end{pmatrix}=\diag(\lambda_1^{-1},\lambda_2^{-1},\dots,\lambda_n^{-1}).\]}
	\end{example}
\end{frame}


\begin{frame}{逆矩阵的计算方法}
	\onslide<+->
	逆矩阵通常采用下述方法计算:
	\begin{enumerate}
		\item 利用公式 $\bfA^{-1}=\dfrac1{|\bfA|}\bfA^*$, 适用于 $2,3$ 阶方阵, 或用于抽象分析.
		\item 寻找方阵 $\bfB$ 使得 $\bfA\bfB=\bfE$, 适用于抽象矩阵求逆.
		\item 利用矩阵的初等变换求逆矩阵, 该方法我们会在\S2.3中学习.
	\end{enumerate}
	\onslide<+->
	\begin{example}
		设 $\bfA$ 为 $n$ 阶方阵且满足 $\bfA^2+3\bfA-2\bfE=\bfO$.
		求 $\bfA^{-1}$ 和 $(\bfA-\bfE)^{-1}$.
	\end{example}
	\onslide<+->
	\begin{solution}
		\begin{enumerate}
			\item 由于 $\bfA^2+3\bfA=2\bfE$, 
			\onslide<+->{%
			因此 $\bfA(\bfA+3\bfE)=2\bfE, \bfA^{-1}=\dfrac12(\bfA+3\bfE)$.}
			\item 由于 $(\bfA-\bfE)(\bfA+4\bfE)=-2\bfE$,
			\onslide<+->{%
			因此 $(\bfA-\bfE)^{-1}=-\dfrac12(\bfA+4\bfE)$.}
		\end{enumerate}
	\end{solution}
\end{frame}


\begin{frame}{方阵的多项式}
	\beqskip{3pt}
	\onslide<+->
	设
	\[f(x)=a_mx^m+\cdots+a_1x+a_0\]
	是一个多项式.
	\onslide<+->
	对于 $n$ 阶方阵 $\bfA$, 定义
	\[f(\bfA)=a_m\bfA^m+\cdots+a_1\bfA+a_0\bfE.\]
	\vspace{-\baselineskip}
	\begin{enumerate}
		\item 若 $\bfA=\diag(\lambda_1,\dots,\lambda_n)$, 则 $\bfA^k=\diag(\lambda_1^k,\dots,\lambda_n^k)$,
	\onslide<+->{%
		从而
		\[f(\bfA)=\sum_{k=0}^m a_k\bfA^k
		=\sum_{k=0}^m a_k\diag(\lambda_1^k,\dots,\lambda_n^k)
		=\diag\bigl(f(\lambda_1),\dots,f(\lambda_n)\bigr).\]
	}
		\item 注意到
			\[
				 (\bfP\bfA\bfP^{-1})^k
				=\bfP\bfA\bfP^{-1}\bfP\bfA\bfP^{-1}\cdots\bfP\bfA\bfP^{-1}
				=\bfP\bfA^k\bfP^{-1},
			\]
		\onslide<+->{%
			因此
			\[
					f(\bfP\bfA\bfP^{-1})=\sum_{k=0}^m a_k\bfP\bfA^k\bfP^{-1}
				=\bfP\Bigl(\sum_{k=0}^m a_k\bfA^k\Bigr)\bfP^{-1}
				=\bfP f(\bfA)\bfP^{-1}.
			\]
		}
	\end{enumerate}
	\endgroup
\end{frame}


\begin{frame}{方阵的多项式}
	\onslide<+->
	设 $f(\bfA)=\bfO$, 我们想求 $(\bfA-\alpha\bfE)^{-1}$.
	\onslide<+->
	注意到
	\[\frac{x^k-a^k}{x-a}=x^{k-1}+ax^{k-2}+\cdots+a^{k-1}\]
	是一个多项式, 因此存在多项式 $g(x)$ 使得
	\[f(x)-f(a)=(x-a)g(x).\]
	\onslide<+->
	由此可知
	\[f(\bfA)-f(\alpha)\bfE=(\bfA-\alpha\bfE)g(\bfA)=-f(\alpha)\bfE.\]
	\onslide<+->
	从而当 $f(\alpha)\neq 0$ 时, $(\bfA-\alpha\bfE)^{-1}=-\dfrac1{f(\alpha)}\bfB$.
	\onslide<+->
	当 $f(\alpha)=0$ 时, $\bfA-\alpha\bfE$ 未必可逆.
\end{frame}


\begin{frame}{例: 计算逆矩阵}
	\onslide<+->
	想一想: 若 $\bfA^3+\bfA^2-2\bfE=\bfO$, 如何求 $(\bfA^2+\bfE)^{-1}$?
	\begin{enumerate}
		\item 待定系数设 $(\bfA^2+\bfE)(a\bfA^2+b\bfA+c\bfE)=\bfE$, 然后使得两边相减是 $\bfA^3+\bfA^2-2\bfE$ 的倍数.
		\item 通过 $\bfA^6=(\bfA^3)^2=(\bfA^2-2\bfE)^2$ 得到 $\bfA^2$ 满足的方程.
		\item 通过分别计算 $(\bfA\pm i\bfE)^{-1}$ 并相乘得到.
	\end{enumerate}
	\onslide<+->
	\begin{example}
		多选题: 若 $\bfA,\bfB,\bfC$ 为同阶方阵, 且 $\bfA$ 可逆, 则\fillbrace{AC}.
		\begin{taskschoice}(2)
			() 若 $\bfA\bfB=\bfA\bfC$, 则 $\bfB=\bfC$
			() 若 $\bfA\bfB=\bfC\bfB$, 则 $\bfA=\bfC$
			() 若 $\bfA\bfB=\bfO$, 则 $\bfB=\bfO$
			() 若 $\bfB\bfC=\bfO$, 则 $\bfB=\bfO$
		\end{taskschoice}
	\end{example}
\end{frame}


\begin{frame}{例: 计算逆矩阵}
	\onslide<+->
	\begin{example}
		设 $n$ 阶方阵 $\bfA,\bfB,\bfC$ 满足 $\bfA\bfB\bfC=\bfE$, 则\fillbraceframe{D}.
		\begin{taskschoice}(4)
			() $\bfA\bfC\bfB=\bfE$
			() $\bfC\bfB\bfA=\bfE$
			() $\bfB\bfA\bfC=\bfE$
			() $\bfC\bfA\bfB=\bfE$
		\end{taskschoice}
	\end{example}
	\onslide<+->
	想一想 $\bfB^{-1}=?$
	\onslide<+->
	\begin{exercise}
		\begin{enumerate}
			\item 设 $\bfA=\begin{pmatrix}
				4&1&3\\4&a&7\\3&-1&4
			\end{pmatrix}$, 且存在两个不等的 $3\times2$ 矩阵 $\bfB,\bfC$ 使得 $\bfA\bfB=\bfA\bfC$, 则 $a=$\fillblankframe{$-3$}.
			\item 设 $3$ 阶方阵 $\bfA$ 满足 $\bfA^3-2\bfA+\bfE=\bfO$, 且 $|\bfA|=2$, 则 $|(\bfA^2-2\bfE)^{-1}|=$\fillblankframe{$-2$}.
		\end{enumerate}
	\end{exercise}
\end{frame}


\subsection{逆矩阵的性质}

\begin{frame}{逆矩阵的性质}
	\onslide<+->
	逆矩阵满足如下性质:
	\begin{enumerate}\bf
		\item 设 $\bfA$ 可逆.
			\begin{itemize}
				\item $\bfA^{-1}$ 也可逆, 且 $(\bfA^{-1})^{-1}=\bfA$;
				\item 若 $\lambda\neq 0$, 则 $\lambda\bfA$ 也可逆, 且 $(\lambda\bfA)^{-1}=\dfrac1\lambda\bfA^{-1}$;
				\item $\bfA^\rmT$ 也可逆, 且 $(\bfA^\rmT)^{-1}=(\bfA^{-1})^\rmT$;
				\item $|\bfA^{-1}|=|\bfA|^{-1}$.
			\end{itemize}
		\item 若 $\bfA$ 可逆, 则 $(\bfA^{-1})^*=(\bfA^*)^{-1}=\dfrac1{|\bfA|}\bfA$.
	\end{enumerate}
	\onslide<+->
	由于 $\bfA\bfA^*=|\bfA|\bfE$, 因此 $\bfA^*=|\bfA|\bfA^{-1}$.
	\onslide<+->
	于是
	\[(\bfA^{-1})^*=|\bfA^{-1}|(\bfA^{-1})^{-1}=\frac1{|\bfA|}\bfA,\quad\visible<+->{(\bfA^*)^{-1}=(|\bfA|\bfA^{-1})^{-1}=\frac1{|\bfA|}\bfA.}\]
\end{frame}



\begin{frame}{伴随矩阵和逆矩阵}
	\begin{enumerate}\bf
		\setcounter{enumi}{2}
		\item 若 $\bfA,\bfB$ 为同阶可逆矩阵, 则 $\bfA\bfB$ 也可逆, 且 \[\alert{(\bfA\bfB)^{-1}=\bfB^{-1}\bfA^{-1}}.\]
		\onslide<+->{%
			一般地 
			\[\alert{(\bfA_1\bfA_2\cdots\bfA_n)^{-1}=\bfA_n^{-1}\cdots\bfA_2^{-1}\bfA_1^{-1}}.\]
		}
	\end{enumerate}

	\onslide<+->
	注意矩阵不能相除 $\dfrac{\bfA}{\bfB}$, 因为一般 $\bfB^{-1}\bfA\neq\bfA\bfB^{-1}$.

	\onslide<+->
	一般地, $(\bfA+\bfB)^{-1}\neq\bfA^{-1}+\bfB^{-1}$.
	\onslide<+->
	例如 $\bfA=\begin{pmatrix}
		1&0\\0&-1
	\end{pmatrix},\bfB=\begin{pmatrix}
		1&0\\0&1
	\end{pmatrix}$ 均可逆, 但 $\bfA+\bfB=\begin{pmatrix}
		2&0\\0&0
	\end{pmatrix}$ 不可逆.
\end{frame}


\begin{frame}{例: 逆矩阵的性质}
	\onslide<+->
	\begin{exercise}
		设 $\bfA$ 是 $n$ 阶方阵, 若\fillbraceframe{D}, 则 $\bfA-\bfE$ 可逆.
		\begin{taskschoice}(2)
			() $\bfA$ 可逆
			() $|\bfA|=0$
			() $\bfA$ 的主对角线元素均为 $0$
			() 存在某个正整数 $m$ 使得 $\bfA^m=\bfO$
		\end{taskschoice}
	\end{exercise}
	\onslide<+->
	\begin{exercise}
		若 $\bfA$ 为 $n$ 阶方阵, 则下面命题正确的有\fillblankframe{$1$}个.
		\begin{enumerate}
			\item $\bfA^{-1}=\dfrac1{|\bfA|}\bfA^*$
			\item $\bfA^*=|\bfA|\bfA^{-1}$
			\item $|\bfA^*|=|\bfA|^{n-1}$
		\end{enumerate}
	\end{exercise}
\end{frame}


\begin{frame}{例: 逆矩阵的性质}
	\onslide<+->
	\begin{example}
		设 $\bfA$ 是 $3$ 阶方阵, $|\bfA|=\dfrac12$.
		求 $\bigl|(2\bfA)^{-1}-(2\bfA)^*\bigr|$.
	\end{example}
	\onslide<+->
	\begin{solution}
		\[(2\bfA)^{-1}-(2\bfA)^*
		=\frac12\bfA^{-1}-2^2\bfA^*=\bfA^*-4\bfA^*=-3\bfA^*,\]
		\onslide<+->{因此
		\[\bigl|(2\bfA)^{-1}-(2\bfA)^*\bigr|=-27|\bfA^*|=-27|\bfA|^2=-\frac{27}4.\]
		}
		\vspace{-\baselineskip}
	\end{solution}
\end{frame}


\begin{frame}{利用逆矩阵解矩阵方程}
	\onslide<+->
	若 $\bfA,\bfB$ 可逆, 下述矩阵方程可以由逆矩阵表出:
	\begin{enumerate}
		\item $\bfA\bfX=\bfC\implies \bfX=\bfA^{-1}\bfC$;
		\item $\bfX\bfA=\bfC\implies \bfX=\bfC\bfA^{-1}$;
		\item $\bfA\bfX\bfB=\bfC\implies \bfX=\bfA^{-1}\bfC\bfB^{-1}$.
	\end{enumerate}
\end{frame}


\begin{frame}{例: 解矩阵方程}\small
	\onslide<+->
	\begin{example}
		设 $\bfA=\begin{pmatrix}
			3&0&1\\1&1&0\\0&1&4
		\end{pmatrix}$. 若 $\bfA\bfX=\bfA+2\bfX$, 求 $\bfX$.
	\end{example}
	\onslide<+->
	\begin{solution}
		由题设得 $(\bfA-2\bfE)\bfX=\bfA$.
		\onslide<+->{%
			注意到
			\[|\bfA-2\bfE|=\begin{vmatrix}
				1&0&1\\1&-1&0\\0&1&2
			\end{vmatrix}=-1\neq0,\quad
				\visible<+->{(\bfA-2\bfE)^{-1}=\begin{pmatrix}
				2&-1&-1\\
				2&-2&-1\\
				-1&1&1
			\end{pmatrix}.}\]
		}\onslide<+->{%
			因此
			$\bfX=(\bfA-2\bfE)^{-1}\bfA=\begin{pmatrix}
				2&-1&-1\\
				2&-2&-1\\
				-1&1&1
			\end{pmatrix}\begin{pmatrix}
				3&0&1\\1&1&0\\0&1&4
			\end{pmatrix}=\begin{pmatrix}
				5&-2&-2\\
				4&-3&-2\\
				-2&2&3
			\end{pmatrix}$.
		}
	\end{solution}
	\onslide<+->
	也可由 $\bfX=\bfE+2(\bfA-2\bfE)^{-1}$ 计算得到.
\end{frame}


\begin{frame}{例: 解矩阵方程}
	\onslide<+->
	\begin{exercise}
		设 $3$ 阶矩阵 $\bfA,\bfB$ 满足 $\bfA^{-1}\bfB\bfA=6\bfA+\bfB\bfA$.
		若 $\bfA=\diag\biggl(\dfrac12,\dfrac14,\dfrac17\biggr)$, 求 $\bfB$.
	\end{exercise}
	\onslide<+->
	\begin{answer}
		右乘 $\bfA^{-1}$ 得到 $\bfB=6(\bfA^{-1}-\bfE)^{-1}=\diag(6,2,1)$.

		\onslide<+->{%
			也可以左乘 $\bfA$ 右乘 $\bfA^{-1}$ 得到
			\[(\bfE-\bfA)\bfB=6\bfA=6\bfE-6(\bfE-\bfA),\]
			\vspace{-\baselineskip}
		}\onslide<+->{%
			\[\bfB=6(\bfE-\bfA)^{-1}-6\bfE=\diag(6,2,1).\]
		}\vspace{-\baselineskip}
	\end{answer}
\end{frame}


\begin{frame}{例: 解矩阵方程}
	\onslide<+->
	\begin{example}
		解矩阵方程 $\bfA^*\bfX=\bfA^{-1}+2\bfX$, 其中 $\bfA=\begin{pmatrix}
			1&1&-1\\-1&1&1\\1&-1&1
		\end{pmatrix}$.
	\end{example}
	\onslide<+->
	\begin{solution}
		注意到 $|\bfA|=4$.
		\onslide<+->{两边同时左乘 $\bfA$ 得到 $4\bfX=\bfE+2\bfA\bfX$,
		}\onslide<+->{因此
		\[\bfX=(4\bfE-2\bfA)^{-1}=\begin{pmatrix}
			2&-2&2\\2&2&-2\\-2&2&2
		\end{pmatrix}^{-1}=\frac14\begin{pmatrix}
			1&1&0\\0&1&1\\1&0&1
		\end{pmatrix}.\]}
		\vspace{-\baselineskip}
	\end{solution}
	\onslide<+->
	这些计算中, 我们应尽量化简待计算的形式, 减少矩阵的计算量.
\end{frame}


\begin{frame}{例: 解矩阵方程}
	\onslide<+->
	\begin{exercise}
		解矩阵方程 $\bfA^*\bfX\bfA=2\bfX\bfA-8\bfE$, 其中 $\bfA=\begin{pmatrix}
			1&2&-2\\0&-2&4\\0&0&1
		\end{pmatrix}$.
	\end{exercise}
	\onslide<+->
	\begin{answer}
		两边同时左乘 $\bfA$ 右乘 $\bfA^{-1}$ 得到
		\[-2\bfX=2\bfA\bfX-8\bfE,\quad (\bfA+\bfE)\bfX=4\bfE,\]
		\[\bfX=4(\bfA+\bfE)^{-1}=4\begin{pmatrix}
			2&2&-2\\0&-1&4\\0&0&2
		\end{pmatrix}^{-1}=\begin{pmatrix}
			2&4&-6\\0&-4&8\\0&0&2
		\end{pmatrix}.\]
	\end{answer}
\end{frame}


\begin{frame}{例: 逆矩阵计算方阵的幂}
	\onslide<+->
	\begin{example}
		设 $\bfP=\begin{pmatrix}
			1&2\\1&4
		\end{pmatrix},\bmL=\begin{pmatrix}
			1&\\&2
		\end{pmatrix},\bfA\bfP=\bfP\bmL$, 求 $\bfA^n$.
	\end{example}
	\onslide<+->
	\begin{solution}
		$|\bfP|=2,\bfP^{-1}=\dfrac12\begin{pmatrix}
			4&-1\\-1&1
		\end{pmatrix},\bfA=\bfP\bmL\bfP^{-1}$,
	\onslide<+->{
		\[
			\bfA^n=\bfP\bmL^n\bfP^{-1}
			\visible<+->{=\dfrac12\begin{pmatrix}
				1&2\\1&4
			\end{pmatrix}\begin{pmatrix}
				1&\\&2^n
			\end{pmatrix}\begin{pmatrix}
				4&-1\\-1&1
			\end{pmatrix}=\begin{pmatrix}
				2-2^n&2^n-1\\2-2^{n+1}&2^{n+1}-1
			\end{pmatrix}.}
		\]
	}\vspace{-\baselineskip}
	\end{solution}
	\onslide<+->
	我们会在后面学习如何使用该技巧来计算一般方阵的幂次.
\end{frame}


\subsection{克拉默法则}

\begin{frame}{克拉默法则}
	\onslide<+->
	\begin{second}{克拉默法则}
		设 $\bfA$ 是 $n$ 阶方阵, $\bfb=(b_1,\dots,b_n)^\rmT$ 是 $n$ 维列向量,
		将 $\bfA$ 的第 $j$ 列换成 $\bfb$ 得到的方阵记为 $\bfA_j$.
		当 $|\bfA|\neq 0$, 线性方程组 $\bfA\bfx=\bfb$ 有唯一解
		\[\bfx=\biggl(\frac{|\bfA_1|}{|\bfA|},\frac{|\bfA_2|}{|\bfA|},\dots,\frac{|\bfA_n|}{|\bfA|}\biggr).\]
	\end{second}
	\onslide<+->
	\begin{proof}
		显然唯一解就是 $\bfx=\bfA^{-1}\bfb=\dfrac1{|\bfA|}\bfA^*\bfb$.
		\onslide<+->{%
			由于方阵 $\bfA_j$ 沿着第 $j$ 列展开得到
			\[|\bfA_j|=b_1A_{1j}+b_2A_{2j}+\cdots+b_nA_{nj}=(A_{1j},A_{2j},\dots,A_{nj})\bfb.\]
		}\onslide<+->{%
			因此 $\bfA^*\bfb=(|\bfA_1|,\dots,|\bfA_n|)^\rmT$, 从而 $\bfx$ 具有题述形式.\qedhere
		}
	\end{proof}
\end{frame}


\begin{frame}{例: 克拉默法则的应用}
	\onslide<+->
	我们将会在\S2.4知道, \alert{$|\bfA|\neq0$ 是方程有唯一解的充分必要条件}.
	\onslide<+->
	因此我们定义的行列式起到了线性方程组的``判别式''的作用.
	\onslide<+->
	\begin{example}
		已知 $\laeq[rrrc]{
			\lambda x_1+{}&x_2+{}&x_3={}&1\\
			x_1+{}&\lambda x_2+{}&x_3={}&1\\
			x_1+{}&x_2+{}&\lambda x_3={}&-2}$
		有无穷多解, 求 $\lambda$.
	\end{example}
	\onslide<+->
	\begin{solution}
		\[0=\begin{vmatrix}
			\lambda&1&1\\1&\lambda&1\\1&1&\lambda
		\end{vmatrix}
		=\lambda^3+2-3\lambda=(\lambda-1)^2(\lambda+2).\]
		\onslide<+->{%
			因此 $\lambda=1$ 或 $-2$.
		}\onslide<+->{%
			显然 $\lambda=1$ 时无解.
		}\onslide<+->{%
			$\lambda=-2$ 时, $x_1=t,x_2=-t,x_3=1$ 是方程的解.
		}\onslide<+->{%
			因此 $\lambda=-2$.
		}
	\end{solution}
\end{frame}


\begin{frame}{齐次线性方程组}
	\onslide<+->
	若线性方程组的常数都是零, 即 $\bfA\bfx={\bf0}$ 或
	\[\laeq[lclcclcl]{
		a_{11}&x_1+{}&a_{12}&x_2&{}+\cdots+{}&a_{1n}&x_n={}&0\\
		a_{21}&x_1+{}&a_{22}&x_2&{}+\cdots+{}&a_{2n}&x_n={}&0\\
		&&&&\vdots&&&\\
		a_{m1}&x_1+{}&a_{n2}&x_2&{}+\cdots+{}&a_{mn}&x_n={}&0
	}\]
	称之为\emph{齐次线性方程组}.
	\onslide<+->
	否则称之为\emph{非齐次线性方程组}.

	\onslide<+->
	显然 $\bfx={\bf0}$ 是齐次线性方程组的解, 称为\emph{零解}. 其它解被称为\emph{非零解}.
	\onslide<+->
	所以当 $m=n$ 时, \alert{$|\bfA|=0\iff$ 齐次线性方程组有无穷多(非零)解}.

	\onslide<+->
	对于非齐次线性方程组, 若 $\bfx_0$ 是一组解, 而 $\bfx$ 是对应的齐次线性方程组的解,
	\onslide<+->
	则 $\bfx_0+\bfx$ 也是非齐次线性方程组的解.
	\onslide<+->
	所以当 $m=n$ 时, \alert{$|\bfA|=0\iff$ 非齐次线性方程组无解或有无穷多解}.
\end{frame}


\begin{frame}{例: 克拉默法则的应用}
	\onslide<+->
	\begin{exercise}
		若 $\laeq[rrr]{
			\lambda x_1+{}&x_2+{}&x_3=0\\
			x_1+{}&\lambda x_2+{}&x_3=0\\
			x_1+{}&x_2+{}&\lambda x_3=0}$
		有非零解, 则 $\lambda=$\fillblankframe{$2,-1$}.
	\end{exercise}
	\onslide<+->
	\begin{example}
		证明: 若三条不同的直线
		\begin{align*}
			ax+by+c=0\\
			bx+cy+a=0\\
			cx+ay+b=0
		\end{align*}
		相交于一点, 则 $a+b+c=0$.
	\end{example}
\end{frame}


\begin{frame}{例: 克拉默法则的应用}
	\onslide<+->
	\begin{proof}
		线性方程组 $\laeq[cccccc]{
			a&x_1+{}&b&x_2+{}&c&x_3=0\\
			b&x_1+{}&c&x_2+{}&a&x_3=0\\
			c&x_1+{}&a&x_2+{}&b&x_3=0}$
			有非零解 $(x,y,1)$.
		\onslide<+->{%
			因此 $\begin{vmatrix}
				a&b&c\\b&c&a\\c&a&b
			\end{vmatrix}=0$.
		}\onslide<+->{%
		\begin{align*}
			\begin{vmatrix}
				a&b&c\\b&c&a\\c&a&b
			\end{vmatrix}&=(a+b+c)\begin{vmatrix}
				1&1&1\\b&c&a\\c&a&b
			\end{vmatrix}\\
			&\visible<+->{=(a+b+c)(bc+ac+ab-a^2-b^2-c^2)}\\
			&\visible<+->{=-\frac12(a+b+c)\bigl[(a-b)^2+(b-c)^2+(c-a)^2\bigr].}
		\end{align*}}
		\onslide<+->{%
			由于这是三条不同直线, 因此 $a,b,c$ 不可能全部相等, 从而 $a+b+c=0$.\qedhere
		}
	\end{proof}
	\onslide<+->
	想一想: 为什么 $a+b+c=0$ 时, 三条直线一定相交于一点?
\end{frame}


\begin{frame}{克拉默法则的注记}
	\onslide<+->
	使用克拉默法则解线性方程组需要:
	\begin{enumerate}
		\item 方程组的数量和未知数数量相同;
		\item 系数矩阵行列式非零.
	\end{enumerate}
	\onslide<+->
	但由于使用克拉默法则计算量较大, 一般不使用该方法解方程, 通常仅用于理论研究.

	\onslide<+->
	\begin{exercise}
		何时下述线性方程组有非零解? \visible<+->{\alert{$b=0$ 或 $-a_1-\cdots-a_n$.}}
		\[\laeq[rcrccrc]{
			(a_1+b)&x_1+{}&a_2&x_2&{}+\cdots+{}&a_n&x_n=0\\
			a_1&x_1+{}&(a_2+b)&x_2&{}+\cdots+{}&a_n&x_n=0\\
			&&&&\vdots&&\\
			a_1&x_1+{}&a_2&x_2&{}+\cdots+{}&(a_n+b)&x_n=0}\]
	\end{exercise}
\end{frame}


\begin{frame}{克拉默法则的应用}
	\onslide<+->
	\begin{example}
		设 $a_1,\dots,a_n$ 两两不同.
		解方程组 $\laeq[clcclc]{
			x_1+{}&a_1&x_2&{}+\cdots+{}&a_1^{n-1}&x_n=1\\
			x_1+{}&a_2&x_2&{}+\cdots+{}&a_2^{n-1}&x_n=1\\
			&&&\vvdots&&\\
			x_1+{}&a_n&x_2&{}+\cdots+{}&a_n^{n-1}&x_n=1}$
	\end{example}
	\onslide<+->
	\begin{solution}
		由于系数矩阵行列式为范德蒙行列式
		\[\prod_{1\le i<j\le n}(a_j-a_i)\neq0,\]
		因此方程有唯一解 $x_1=1,x_2=\cdots=x_n=0$.
	\end{solution}
\end{frame}




\subsection{逆矩阵的应用}

% \begin{frame}{逆矩阵的应用: 通信加密\noexer}
% 	\onslide<+->
% 	1929年, 希尔通过线性变换对信息进行加密和解密处理, 
% 	提出了密码史上具有重要地位的\emph{希尔密码系统}.
% 	\onslide<+->
% 	\begin{center}
% 		\begin{tikzpicture}
% 			\begin{scope}[align=center,node distance=18mm,minimum height=12mm]
% 				\node (1) [cstnode2] {明文字母};
% 				\node (2) [cstnode2,right=of 1] {明文数字\\矩阵 $\bfX$};
% 				\node (3) [cstnode2,right=30mm of 2]{密文数字\\矩阵 $\bfC$};
% 				\node (4) [cstnode2,right=of 3]{密文字母};
% 				\node (11) [cstnode1,below=of 1] {明文字母};
% 				\node (12) [cstnode1,below=of 2] {明文数字\\矩阵 $\bfX$};
% 				\node (13) [cstnode1,below=of 3] {密文数字\\矩阵 $\bfC$};
% 				\node (14) [cstnode1,below=of 4] {密文字母};
% 			\end{scope}
% 			\path (1.east) -- (12.west)
% 				node[midway,align=center,fifth] {$A\sim Z$\\对应\\$0\sim26$};
% 			\begin{scope}[cstra,cstcurve]
% 				\begin{scope}[fourth]
% 					\draw (1.east) -- (2.west) 
% 						node[midway,above,black] {查表};
% 					\draw (2.east) -- (3.west)
% 						node[midway,above,black] {$\bfC=\bfA\bfX$}
% 						node[midway,below,fourth] {加密秘钥: $A$};
% 					\draw (3.east) -- (4.west)
% 						node[midway,above,black] {查表};
% 				\end{scope}
% 				\draw[main] (4.south) -- (14.north)
% 					node[midway,left] {传输};
% 				\begin{scope}[second]
% 					\draw (14.west) -- (13.east)
% 						node[midway,above,black] {查表};
% 					\draw (13.west) -- (12.east)
% 						node[midway,above,black] {$\bfX=\bfA^{-1}\bfC$}
% 						node[midway,below,second] {解密秘钥: $A^{-1}$};
% 					\draw (12.west) -- (11.east) 
% 						node[midway,above,black] {查表};
% 				\end{scope}
% 			\end{scope}
% 		\end{tikzpicture}
% 	\end{center}
% \end{frame}


% \begin{frame}{逆矩阵的应用: 通信加密\noexer}
% 	\beqskip{2mm}
% 	\onslide<+->
% 	\begin{example}
% 		设接受收到的密文字母为``WBIZTNWJBRFSGNZ'', 加密密钥为 $\bfA=\begin{pmatrix}
% 			1&2&3\\1&1&2\\0&1&2
% 		\end{pmatrix}$. 
% 		请用希尔密码系统解密密文.
% 	\end{example}
% 	\onslide<+->
% 	\begin{solution}
% 		密文对应的数字为 
% 		\[22,1,8,25,19,13,22,9,1,17,5,18,6,13,25.\]
% 		\onslide<+->{由于密钥是 $3$ 阶方阵, 所以将上述数字按 $3$ 个一列写成
% 		\[\bfC=\begin{pmatrix}
% 			22&25&22&17&6\\
% 			1&19&9&5&13\\
% 			8&13&1&18&25
% 		\end{pmatrix}.\]}
% 		\vspace{-.8\baselineskip}
% 	\end{solution}
% 	\endgroup
% \end{frame}


% \begin{frame}{逆矩阵的应用: 通信加密\noexer}
% 	\begin{solution}[续解]
% 		\onslide<+->{解密密钥为
% 		$\bfA^{-1}=\begin{pmatrix}
% 			0&1&-1\\2&-2&-1\\-1&1&1
% 		\end{pmatrix}$,
% 		}\onslide<+->{%
% 		因此
% 		\[\bfA^{-1}\bfC=\begin{pmatrix}
% 			-7&6&8&-13&-12\\
% 			34&-1&25&6&-39\\
% 			-13&7&-12&6&32
% 		\end{pmatrix},\quad
% 		\bfX=\begin{pmatrix}
% 			19&6&8&13&14\\
% 			8&25&25&6&13\\
% 			13&7&14&6&6
% 		\end{pmatrix}.\]}\onslide<+->{%
% 		查表得到明文: \alert{TINGZHI ZONGGONG}.}
% 	\end{solution}
% \end{frame}


\begin{frame}{逆矩阵的应用: 图像处理\noexer}
	\begin{minipage}{.35\textwidth}
		\includegraphics[height=32mm]{../image/ps_in.jpg}
	\end{minipage}
	\begin{minipage}{.57\textwidth}
		\onslide<+->
		左图是一张夏天的风景图, 我们希望把它修改成秋天的景色. 
		\onslide<+->
		Photoshop提供了将颜色重新搭配的通道混合器.
		\onslide<+->
		用取色工具选取树叶、蓝天、地面的颜色, 分别得到 RGB 值为
		\[(59,181,19),\quad (90,185,249),\quad (210,205,186).\]
	\end{minipage}

	\onslide<+->
	我们希望将树叶变成金黄色 RGB$(234,228,70)$ 而保持蓝天和地面的颜色不变.
	\onslide<+->
	则我们需要的矩阵 $\bfA$ 满足
	\[\bfA\begin{pmatrix}
		59&90&210\\
		181&185&205\\
		19&249&186
	\end{pmatrix}=\begin{pmatrix}
		234&90&210\\
		228&185&205\\
		70&249&186
	\end{pmatrix}.\]
\end{frame}


\begin{frame}{逆矩阵的应用: 图像处理\noexer}
	\onslide<+->
	解得
	\[\bfA=\begin{pmatrix}
		234&90&210\\
		228&185&205\\
		70&249&186
	\end{pmatrix}\begin{pmatrix}
		59&90&210\\
		181&185&205\\
		19&249&186
	\end{pmatrix}^{-1}
	\approx\begin{pmatrix}
		0.43&1.23&-0.70\\
		-0.15&1.33&-0.19\\
		-0.17&0.36&0.79
	\end{pmatrix}.\]
	\onslide<+->
	分别在通道混合器的红绿蓝通道输入上面三行即可.
	\begin{center}
		\begin{tikzpicture}
			\draw
				(0,0) node {\includegraphics[height=3cm]{../image/ps_in.jpg}}
				(8,0) node {\includegraphics[height=3cm]{../image/ps_out.jpg}}
				(4,0) node[fourth] {{\Huge$\implies$}};
		\end{tikzpicture}
	\end{center}
\end{frame}



