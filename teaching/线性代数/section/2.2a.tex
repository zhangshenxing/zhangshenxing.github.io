\section{矩阵的运算: 线性运算和乘法}

\subsection{矩阵的概念}

\begin{frame}{矩阵记号}
	\onslide<+->
	我们已经在上一章知道了什么是矩阵和方阵, 并介绍了一些特殊形状的方阵.
	\onslide<+->
	分别用记号
	\begin{itemize}
		\item $M_{m\times n}(\BR)$ 表示 $m$ 行 $n$ 列实矩阵, 即元素都是实数的矩阵;
		\item $M_n(\BR)$ 表示 $n$ 阶实方阵;
		\item $M_{m\times n}(\BC)$ 表示 $m$ 行 $n$ 列复矩阵;
		\item $M_n(\BC)$ 表示 $n$ 阶复方阵.
	\end{itemize}
	\onslide<+->
	不强调在系数是实数还是复数时, 就简单记作 $M_{m\times n},M_n$.
\end{frame}


\begin{frame}{特殊矩阵}
	\onslide<+->
	方阵中
	\[\begin{pmatrix}
		a_{11}&a_{12}&\cdots&a_{1n}\\
		&a_{22}&\cdots&a_{2n}\\
		&&\ddots&\vdots\\
		&&&a_{nn}\\
	\end{pmatrix},\quad\begin{pmatrix}
		a_{11}&&&\\
		a_{21}&a_{22}&&\\
		\vdots&\vdots&\ddots&\\
		a_{n1}&a_{n2}&\cdots&a_{nn}\\
	\end{pmatrix},\quad\begin{pmatrix}
		\lambda_1&&&\\
		&\lambda_2&&\\
		&&\ddots&\\
		&&&\lambda_n\\
	\end{pmatrix}\in M_n\]
	分别为\emph{上三角矩阵}, \emph{下三角矩阵}和\emph{对角矩阵}.
	\onslide<+->
	对角矩阵也可记作
	\[\diag(\lambda_1,\lambda_2,\dots,\lambda_n).\]
	\onslide<+->
	\[\bfE_n=\diag(1,1,\dots,1)\in M_n\]
	为\emph{单位矩阵}.
\end{frame}


\begin{frame}{特殊矩阵}
	\onslide<+->
	元素全为零的矩阵为\emph{零矩阵} $\bfO\in M_{m\times n}$.
	\onslide<+->
	只有一行的矩阵
	\[(a_1,a_2,\dots,a_n)\in M_{1\times n}\]
	称为 $n$ 维\emph{行矩阵}或\emph{行向量}.
	\onslide<+->
	只有一行的矩阵
	\[\begin{pmatrix}
		b_1\\b_2\\\vdots\\b_n
	\end{pmatrix}\in M_{n\times 1}\]
	称为 $n$ 维\emph{列矩阵}或\emph{列向量}.
	\onslide<+->
	为了书写方便, 可以把列向量写成
	\[(b_1,b_2,\dots,b_n)^\rmT.\]
\end{frame}


\subsection{矩阵的线性运算}
\begin{frame}{矩阵的加法}
	\onslide<+->
	如果两个矩阵的行列数都相同, 称之为\emph{同型矩阵}.
	\onslide<+->
	\begin{definition}
		设 $\bfA=(a_{ij})_{m\times n},\bfB=(b_{ij})_{m\times n}$ 为同型矩阵.
		定义
		\[\bfA+\bfB=(a_{ij}+b_{ij})_{m\times n}.\]
	\end{definition}
	\onslide<+->
	矩阵的加法满足通常数的加法的几条规律:
	\begin{enumerate}
		\item $\bfA+\bfB=\bfB+\bfA$;
		\item $(\bfA+\bfB)+\bfC=\bfA+(\bfB+\bfC)$;
		\item $\bfA+\bfO=\bfA$.
	\end{enumerate}
\end{frame}


\begin{frame}{加法和行列式不交换}
	\onslide<+->
	注意两个方阵的和的行列式 $|\bfA+\bfB|$ 一般不等于各自行列式的和 $|\bfA|+|\bfB|$.
	\onslide<+->
	\begin{example}
		设 $\bfA=\begin{pmatrix}
			a_1&c_1&d_1\\
			a_2&c_2&d_2\\
			a_3&c_3&d_3
		\end{pmatrix},\bfB=\begin{pmatrix}
			b_1&c_1&d_1\\
			b_2&c_2&d_2\\
			b_3&c_3&d_3
		\end{pmatrix}$, $|\bfA|=2,|\bfB|=1$.
		计算 $|\bfA+\bfB|$.
	\end{example}
	\onslide<+->
	\begin{solution}
		\begin{align*}
			|\bfA+\bfB|&=\detm{
				a_1+b_1&2c_1&2d_1\\
				a_2+b_2&2c_2&2d_2\\
				a_3+b_3&2c_3&2d_3
			}
			\visible<+->{=\detm{
				a_1&2c_1&2d_1\\
				a_2&2c_2&2d_2\\
				a_3&2c_3&2d_3
			}+\detm{
				b_1&2c_1&2d_1\\
				b_2&2c_2&2d_2\\
				b_3&2c_3&2d_3
			}}\\
			&\visible<+->{=4|\bfA|+4|\bfB|=12.}
		\end{align*}
	\end{solution}
\end{frame}


\begin{frame}{矩阵的数乘}
	\onslide<+->
	\begin{definition}
		数 $\lambda$ 和矩阵 $\bfA=(a_{ij})_{m\times n}$ 的数乘定义为
		\[\lambda\bfA=(\lambda a_{ij})_{m\times n}.\]
	\end{definition}
	\onslide<+->
	数乘矩阵满足:
	\begin{enumerate}
		\item $(\lambda \mu)\bfA=\lambda(\mu\bfA)=\mu(\lambda\bfA)$;
		\item $(\lambda+\mu)\bfA=\lambda\bfA+\mu\bfA$;
		\item $\lambda(\bfA+\bfB)=\lambda\bfA+\lambda\bfB$;
		\item $1\cdot \bfA=\bfA, 0\cdot\bfA=\bfO,\lambda\bfO=\bfO$.
	\end{enumerate}
\end{frame}


\begin{frame}{负矩阵和减法}
	\onslide<+->
	$-1$ 与 $\bfA$ 的数乘称为 $\bfA$ 的\emph{负矩阵}
	\[-\bfA=(-a_{ij})_{m\times n}.\]
	\onslide<+->
	那么矩阵的减法就是
	\[\bfA-\bfB=\bfA+(-\bfB)=(a_{ij}-b_{ij})_{m\times n}.\]
	
	\onslide<+->
	想一想: $|\lambda \bfA|=\lambda |\bfA|$?
	\onslide<+->
	\alert{\Large$\times$}
	如果 $\bfA\in M_n$, 则
	$|\lambda \bfA|=\lambda^n |\bfA|$.
\end{frame}


\begin{frame}{矩阵线性运算的应用: 图像处理}
	\onslide<+->
	\begin{center}
		\includegraphics[height=3cm]{../image/matrix2.jpg}
	\end{center}

	一张图片由一些像素构成, 上图包含 $512\times341$ 个像素.
	\onslide<+->
	在 RGB 颜色模式下, 每个像素包含红绿蓝三个通道, 每个通道为一个 $0\sim255$ 之间的数, 数值越高对应颜色越饱满.
	\onslide<+->
	如果三个通道相同, 图片就是一张灰色的图(无色彩).
	此时图片对应一个 $512\times 341$ 的矩阵 $\bfA$.

	\onslide<+->
	想一想: 如何将这个图像变亮?
\end{frame}


\begin{frame}{矩阵线性运算的应用: 图像处理}
	\onslide<+->
	\begin{center}
		\begin{tikzpicture}
			\draw
			(0,0)node{\includegraphics[height=3cm]{../image/matrix2.jpg}}
			(8,0)node{\includegraphics[height=3cm]{../image/matrix4.jpg}}
			(4,0)node{{\Huge$\implies$}};
		\end{tikzpicture}
	\end{center}
	
	\onslide<+->
	我们只需要增加每个分量的值, 例如
	\[\bfA+\begin{pmatrix}
		50&50&\cdots&50\\
		50&50&\cdots&50\\
		\vdots&\vdots&\ddots&\vdots\\
		50&50&\cdots&50
	\end{pmatrix},\qquad
	1.5\bfA.\]
\end{frame}


\begin{frame}{矩阵线性运算的应用: 图像处理}
	\onslide<+->
	如何让图像反色?
	\begin{center}
		\begin{tikzpicture}
			\draw
			(0,0)node{\includegraphics[height=3cm]{../image/matrix2.jpg}}
			(8,0)node{\includegraphics[height=3cm]{../image/matrix3.jpg}}
			(4,0)node{{\Huge$\implies$}};
		\end{tikzpicture}
	\end{center}
	
	\onslide<+->
	\[\begin{pmatrix}
		255&255&\cdots&255\\
		255&255&\cdots&255\\
		\vdots&\vdots&\ddots&\vdots\\
		255&255&\cdots&255
	\end{pmatrix}-\bfA.\]
\end{frame}

\subsection{矩阵的乘法}

\begin{frame}{线性变换}
	\onslide<+->
	设 $n$ 个变量 $x_1,\dots,x_n$ 和 $m$ 变量 $y_1,\dots,y_m$ 满足关系:
	\[\laeq{
		y_1=a_{11}x_1+a_{12}x_2+\cdots+a_{1n}x_n\\
		y_2=a_{21}x_1+a_{22}x_2+\cdots+a_{2n}x_n\\
		\vvdots\\
		y_m=a_{m1}x_1+a_{m2}x_2+\cdots+a_{mn}x_n
	}.\]
	\onslide<+->
	它的系数形成了一个矩阵 $\bfA=(a_{ij})_{m\times n}$.

	\onslide<+->
	如果这些变量均取实数, 我们用 $\BR^n$ 表示 $n$ 个实数形成的数组.
	\onslide<+->
	那么上述关系定义了映射
	\[\msa: \BR^n\lra \BR^m.\]
	\onslide<+->
	这样的由线性关系给出的映射被称为\emph{线性变换}.
\end{frame}


\begin{frame}{线性变换的例子: 旋转}
	\onslide<+->
	如何用矩阵表示平面上的旋转?
	\onslide<+->
	设 $A(x_1,x_2)$ 是平面上的一个点, 沿着原点逆时针旋转角度 $\theta$ 变成 $B(y_1,y_2)$.
	\onslide<+->
	利用极坐标将 $A$ 表示为
	\[\laeq{x_1=\rho\cos\alpha,\\x_2=\rho\sin\alpha,}\]
	那么
	\[\laeq{y_1=\rho\cos(\alpha+\theta)=\rho(\cos\alpha\cos\theta-\sin\alpha\sin\theta)=(\cos\theta)x_1-(\sin\theta)x_2,\\
	y_2=\rho\sin(\alpha+\theta)=\rho(\cos\alpha\sin\theta+\sin\alpha\cos\theta)=(\sin\theta)x_1+(\cos\theta)x_2.}\]
	\onslide<+->
	因此上述旋转变换 $\msa$ 对应的矩阵为
	\[\bfA=\begin{pmatrix}
		\cos\theta&-\sin\theta\\
		\sin\theta&\cos\theta
	\end{pmatrix}.\]
\end{frame}


\begin{frame}{线性变换的复合}
	\onslide<+->
	给定两个矩阵 $\bfA=(a_{ij})_{m\times n}, \bfB=(b_{ij})_{n\times p}$, 其中 $\bfA$ 的列数和 $\bfB$ 的行数相等.
	\onslide<+->
	那么它们对应两个映射
	\[\msa:\BR^n\to\BR^m,\quad
	\msb:\BR^p\to \BR^n.\]
	\onslide<+->
	它们的复合
	\[\msa\circ\msb:\BR^p\to\BR^n\to\BR^m\]
	是否还是一个线性变换呢?
	如果是, 对应的矩阵是什么?
\end{frame}


\begin{frame}{线性变换的复合}
	\onslide<+->
	设 $x=(x_1,\dots,x_p)\in \BR^p$, 那么
	\[y=(y_1,\dots,y_n)=\msb(x)\in\BR^n\qquad
	\text{满足}\qquad
		y_k=\sum_{j=1}^p b_{kj} x_j.\]
	\onslide<+->
	\[z=(z_1,\dots,z_m)=\msa(y)\in\BR^n\]
	满足
	\[z_i=\sum_{k=1}^n a_{ik} y_k=\sum_{k=1}^n a_{ik} \sum_{j=1}^p b_{kj} x_j=\sum_{j=1}^p\Bigl(\sum_{k=1}^n a_{ik}b_{kj}\Bigr)x_j.\]
	\onslide<+->
	所以 $\msa\circ\msb$ 是线性变换, 且对应的矩阵为
	\[\bfC=(c_{ij})_{m\times p},\qquad c_{ij}=\sum_{k=1}^n a_{ik}b_{kj}.\]
	\onslide<+->
	我们把它定义为矩阵的乘法 $\bfC=\bfA\bfB$.
\end{frame}


\begin{frame}{矩阵乘法的定义}
	\onslide<+->
	\begin{definition}
		设 $\bfA=(a_{ij})_{m\times n}, \bfB=(b_{ij})_{n\times p}$.
		定义矩阵的乘法为  $\bfC=\bfA\bfB=(c_{ij})_{m\times p}$, 其中
		\[c_{ij}=\sum_{k=1}^n a_{ik}b_{kj}.\]
	\end{definition}
	\onslide<+->
	只有第一个矩阵的列数等于第二个矩阵的行数才能相乘.
	\onslide<+->
	\[\begin{pmatrix}
		1&2\\2&1
	\end{pmatrix}\begin{pmatrix}
		0&0&0\\0&0&0\\0&0&0
	\end{pmatrix}=?\visible<+->{\alert{\text{\huge$\times$}}}.\]
\end{frame}


\begin{frame}{行向量与列向量的乘法}
	\onslide<+->
	设 $\bfA=(a_1,\dots,a_n)$ 是 $n$ 为行向量, $\bfB=(b_1,\dots,b_n)^\rmT$ 是 $n$ 维列向量.
	\onslide<+->
	$\bfA\bfB,\bfB\bfA=?$
	\onslide<+->
	\[\bfA\bfB=\sum_{i=1}^n a_i b_i,\qquad
	\bfB\bfA=(b_ia_j)_{n\times n}\in M_n.\]

	\onslide<+->
	对于矩阵 $\bfA=(a_{ij})_{m\times n}, \bfB=(b_{ij})_{n\times p}$.
	\onslide<+->
	$\bfA\bfB$ 的 $(i,j)$ 元其实就是 $\bfA$ 第 $i$ 行对应的行向量和 $\bfB$ 第 $j$ 列对应的列向量相乘得到的数($1$ 阶方阵).
\end{frame}


\begin{frame}{例: 矩阵乘法的计算}
	\onslide<+->
	\begin{example}
		求矩阵 $\bfA=\begin{pmatrix}
			1&2&0&-1\\2&1&4&0
		\end{pmatrix}$ 与 $\bfB=\begin{pmatrix}
			2&0&1\\
			-2&3&1\\
			1&5&0\\
			1&-3&4
		\end{pmatrix}$ 的乘积 $\bfA\bfB$.
	\end{example}
	\onslide<+->
	\begin{solution}
		\[\begin{pmatrix}
			1&2&0&-1\\2&1&4&0
		\end{pmatrix}\begin{pmatrix}
			2&0&1\\
			-2&3&1\\
			1&5&0\\
			1&-3&4
		\end{pmatrix}=\begin{pmatrix}
			-3&9&-1\\
			6&23&3
		\end{pmatrix}.\]
	\end{solution}
\end{frame}


\begin{frame}{线性方程组与乘法}
	\onslide<+->
	设线性方程组
	\[\laeq{
		a_{11}x_1+a_{12}x_2+\cdots+a_{1n}x_n=b_1\\
		a_{21}x_1+a_{22}x_2+\cdots+a_{2n}x_n=b_2\\
		\vvdots\\
		a_{m1}x_1+a_{m2}x_2+\cdots+a_{mn}x_n=b_m,
	}\]
	的系数矩阵为 $\bfA$.
	\onslide<+->
	如果我们令
	\[\bfx=\begin{pmatrix}
		x_1\\x_2\\\vdots\\x_n
	\end{pmatrix}\in M_{n\times 1},\qquad
	\bfb=\begin{pmatrix}
		b_1\\b_2\\\vdots\\b_m
	\end{pmatrix}\in M_{m\times 1},\]
	\onslide<+->
	那么上述方程等价于矩阵方程 \alert{$\bfA\bfx=\bfb$}.
	\onslide<+->
	对应的齐次方程为 \alert{$\bfA\bfx={\bf0}$}.
\end{frame}


\begin{frame}{矩阵乘法的性质}
	\onslide<+->
	矩阵乘法满足如下性质:
	\begin{enumerate}
		\item $(\bfA\bfB)\bfC=\bfA(\bfB\bfC)$;
		\item $\lambda(\bfA\bfB)=(\lambda\bfA)\bfB=\bfA(\lambda\bfB)$;
		\item $\bfA(\bfB+\bfC)=\bfA\bfB+\bfA\bfC$;
		\item 如果 $\bfA\in M_{m\times n}$, 则 $\bfE_m \bfA=\bfA\bfE_n=\bfA$.
		\item 如果 $\bfA\in M_{m\times n}$, 则 $\bfO_{p\times m} \bfA=\bfO_{p\times n}, \bfA\bfO_{n\times p}=\bfO_{m\times p}$.
	\end{enumerate}
\end{frame}


\begin{frame}{矩阵乘法无交换律和消去律}
	\onslide<+->
	\alert{矩阵的乘法不能随意交换顺序}.
	\onslide<+->
	一般称 $\bfA\bfB$ 为 \emph{$\bfA$ 左乘 $\bfB$}, 或者 \emph{$\bfB$ 右乘 $\bfA$}.

	\onslide<+->
	如果 $\bfA\bfB=\bfB\bfA$, 则称 $\bfA,\bfB$ 是\emph{可交换}的.
	\onslide<+->
	此时 $\bfA,\bfB$ \alert{必为同阶方阵}.
	\onslide<+->
	例如
	\[\begin{pmatrix}
		1&2\\0&1
	\end{pmatrix}\begin{pmatrix}
		2&1\\0&2
	\end{pmatrix}=\begin{pmatrix}
		2&5\\0&2
	\end{pmatrix}=\begin{pmatrix}
		2&1\\0&2
	\end{pmatrix}\begin{pmatrix}
		1&2\\0&1
	\end{pmatrix}\]

	\onslide<+->
	矩阵乘法也没有消去律: $\bfA\bfB=\bfO$ 推不出 $\bfA=\bfO$ 或 $\bfB=\bfO$.
	\onslide<+->
	例如
	\[\begin{pmatrix}
		2&4\\1&2
	\end{pmatrix}\begin{pmatrix}
		2&-2\\-1&1
	\end{pmatrix}=\bfO_2.\]
	\onslide<+->
	由此可知: $\bfA\bfC=\bfB\bfC$ 推不出 $\bfA=\bfB$. 

	\onslide<+->
	\begin{exercise}
		设 $\bfA,\bfB$ 为 $n>1$ 阶方阵, 则 $\bfA+\bfA\bfB=$\fillbrace{\visible<+->{\alert{C}}}
		\xx{$\bfA(1+\bfB)$}{$(\bfE+\bfB)\bfA$}{$\bfA(\bfE+\bfB)$}{以上都不对}
	\end{exercise}
\end{frame}


\begin{frame}{例: 与给定矩阵可交换}
	\onslide<+->
	\begin{example}
		求与矩阵 $\bfA=\begin{pmatrix}
			0&1&0\\&0&1\\&&0
		\end{pmatrix}$ 可交换的所有矩阵.
	\end{example}
	\onslide<+->
	\begin{solution}
		设 $\bfB=(a_{ij})_{3\times 3}$ 与 $\bfA$ 可交换, 则
		\vspace{-.5\baselineskip}
		\[\bfA\bfB=\begin{pmatrix}
			a_{21}&a_{22}&a_{23}\\
			a_{31}&a_{32}&a_{33}\\
			0&0&0
		\end{pmatrix}=\bfB\bfA=\begin{pmatrix}
			0&a_{11}&a_{12}\\
			0&a_{21}&a_{22}\\
			0&a_{31}&a_{32}
		\end{pmatrix}.
		\vspace{-.5\baselineskip}\]
		\vspace{-.5\baselineskip}
		\onslide<+->{
			\[\implies a_{11}=a_{21}=a_{31}=a_{32}=0,\quad a_{11}=a_{22}=a_{33},\quad a_{23}=a_{12},
			\vspace{-.5\baselineskip}\]
		}\onslide<+->{%
		即 $\bfB=\begin{pmatrix}
			a_{11}&a_{12}&a_{13}\\
			&a_{11}&a_{12}\\
			&&a_{11}
		\end{pmatrix}$.}
		\vspace{-.3\baselineskip}
	\end{solution}
\end{frame}


\begin{frame}{矩阵乘法的应用: 图像校正}
	\onslide<+->
	某位同学拍身份证照片拍成了下图的样子, 如何能否修复好呢?
	\begin{center}
		\begin{tikzpicture}
			\draw (0,0) node {\includegraphics[height=3cm]{../image/idcard1.png}}
			(-2.2,-1.6) node[dcolora] {$O$}
			(2,-1) node[dcolora] {$A$}
			(-2.1,.9) node[dcolora] {$B$};
		\end{tikzpicture}
	\end{center}
	\onslide<+->
	以左下角为原点, 通过测量发现 $A$ 坐标为 $(463,88)$, $B$ 坐标为 $(17,311)$.

	\onslide<+->
	经过查询知道身份证长宽比为 $85.6:54$.
	\onslide<+->
	令 $A'=(427,0),B'=(270,0)$.
	我们希望找到一个线性变换, 将 $A,B$ 变为 $A',B'$.
\end{frame}


\begin{frame}{矩阵乘法的应用: 图像校正}
	\onslide<+->
	设该线性变换对应的矩阵为 $\bfA=\begin{pmatrix}
		a&b\\c&d
	\end{pmatrix}$, 那么
	\[\bfA\begin{pmatrix}
		463&17\\88&311
	\end{pmatrix}=\begin{pmatrix}
		427&0\\0&270
	\end{pmatrix},\qquad
	\visible<+->{
		\text{即}\qquad
	\laeq{
		463a+88b=427\\
		17a+311b=0\\
		463c+88d=0\\
		17c+311d=270
	}.}\]
	\onslide<+->
	解得 $\bfA=\begin{pmatrix}
		0.932&-0.051\\
		-0.167&0.877
	\end{pmatrix}.$
	\onslide<+->
	\begin{center}
		\includegraphics[height=2.5cm]{../image/idcard.png}
	\end{center}
\end{frame}

\subsection{矩阵的幂}

\begin{frame}{矩阵幂的定义}
	\onslide<+->
	\begin{definition}
		设 $\bfA$ 为 $n$ 阶方阵, 定义 $\bfA$ 的幂
		\[\bfA^0=\bfE_n,\quad \bfA^k=\underbrace{\bfA\cdot\bfA\cdot\cdots\cdot\bfA}_{k\ \text{个}}.\]
	\end{definition}
	\onslide<+->
	矩阵幂满足如下性质 ($k,\ell$ 为正整数):
	\begin{enumerate}
		\item $\bfA^{k+\ell}=\bfA^k\cdot \bfA^\ell$;
		\item $\bfA^{k\ell}=(\bfA^k)^\ell$.
	\end{enumerate}
	\onslide<+->
	注意 $(\bfA\bfB)^k$ 一般不等于 $\bfA^k\cdot\bfB^k$.
	\onslide<+->
	想一想下面的等式成立吗?
	\[(\bfA-\bfB)(\bfA+\bfB)=\bfA^2-\bfB^2?\]
	\[(\bfA+\bfB)^2=\bfA^2+2\bfA\bfB+\bfB^2?\]
\end{frame}


\begin{frame}{例: 矩阵的幂}
	\onslide<+->
	\begin{example}
		设
		$\bfA=\diag(\lambda_1,\cdots,\lambda_n).$
		求 $\bfA^k$.
	\end{example}
	\onslide<+->
	\begin{solution}
		\[\bfA^2=\bfA\cdot\bfA=\diag(\lambda_1^2,\cdots,\lambda_n^2),\]
		\[\bfA^3=\bfA\cdot\bfA^2=\diag(\lambda_1^3,\cdots,\lambda_n^3),\]
		递推下去可知
		\[\bfA^k=\diag(\lambda_1^k,\cdots,\lambda_n^k).\]
	\end{solution}
\end{frame}


\begin{frame}{例: 矩阵的幂}
	\onslide<+->
	\begin{example}
		设 $\bfA=\begin{pmatrix}
			\lambda&1&0\\
			&\lambda&1\\
			&&\lambda
		\end{pmatrix}.$
		求 $\bfA^k$.
	\end{example}
	\onslide<+->
	\begin{solution}
		\[\bfA^2=\begin{pmatrix}
			\lambda&1&0\\
			&\lambda&1\\
			&&\lambda
		\end{pmatrix}\begin{pmatrix}
			\lambda&1&0\\
			&\lambda&1\\
			&&\lambda
		\end{pmatrix}=\begin{pmatrix}
			\lambda^2&2\lambda&1\\
			&\lambda^2&2\lambda\\
			&&\lambda^2
		\end{pmatrix}.\]
	\end{solution}
	\end{frame}
	
	
	\begin{frame}{例: 矩阵的幂}
		\begin{solutionc}
			\[\bfA^2=\begin{pmatrix}
				\lambda&1&0\\
				&\lambda&1\\
				&&\lambda
			\end{pmatrix}\begin{pmatrix}
				\lambda^2&2\lambda&1\\
				&\lambda^2&2\lambda\\
				&&\lambda^2
			\end{pmatrix}=\begin{pmatrix}
				\lambda^3&3\lambda^2&3\lambda\\
				&\lambda^3&3\lambda^2\\
				&&\lambda^3
			\end{pmatrix}.\]
			\onslide<+->{归纳可知
		\[\bfA^k=\begin{pmatrix}
			\lambda^k&k\lambda^{k-1}&\dfrac{k(k-1)}2\lambda^{k-2}\\[3mm]
			&\lambda^k&k\lambda^{k-1}\\[3mm]
			&&\lambda^k
		\end{pmatrix}.\]}
	\end{solutionc}
\end{frame}


\begin{frame}{例: 矩阵的幂}
	\begin{proofblock}{另解}
		设 $\bfN=\begin{pmatrix}
			0&1&0\\
			&0&1\\
			&&0
		\end{pmatrix}$,
		则
		$\bfN^2=\begin{pmatrix}
			0&0&1\\
			&0&0\\
			&&0
		\end{pmatrix},\bfN^3=\bfO.$
		\onslide<+->{由于 $\bfA=\lambda\bfE+\bfN$ 且 $\bfE$ 和 $\bfN$ \alert{可交换}, 因此
		\begin{align*}
			\bfA^k&=\lambda^k\bfE+\rmC_k^1\lambda^{k-1} \bfN+\rmC_k^2\lambda^{k-2}\bfN^2\\
			&\visible<+->{=\begin{pmatrix}
				\lambda^k&k\lambda^{k-1}&\dfrac{k(k-1)}2\lambda^{k-2}\\[3mm]
				&\lambda^k&k\lambda^{k-1}\\[3mm]
				&&\lambda^k
			\end{pmatrix}.}
		\end{align*}}
	\end{proofblock}
\end{frame}


\begin{frame}{例: 矩阵的幂}
	\onslide<+->
	\begin{example}
		设
		$\bfA=\begin{pmatrix}
			\cos\theta&-\sin\theta\\
			\sin\theta&\cos\theta
		\end{pmatrix}.$
		求 $\bfA^k$.
	\end{example}
	\onslide<+->
	\begin{solution}
		注意到 $\bfA$ 对应平面 $\BR^2$ 上的线性变换是逆时针旋转 $\theta$, 所以 $\bfA^k$ 就是逆时针旋转 $n\theta$, 对应的矩阵为
		\[\bfA^k=\begin{pmatrix}
			\cos k\theta&-\sin k\theta\\
			\sin k\theta&\cos k\theta
		\end{pmatrix}.\]
	\end{solution}
\end{frame}


\begin{frame}{例: 矩阵的幂}
	\onslide<+->
	\begin{example}
		设 
		$\bfA=(1,2,3),\bfB=\begin{pmatrix}
			-1&2&0
		\end{pmatrix}.$
		求 $(\bfB\bfA)^k$.
	\end{example}
	\onslide<+->
	\begin{solution}
		注意到 $\bfA\bfB=3$, 因此
		\begin{align*}
			(\bfB\bfA)^k&=\bfB(\bfA\bfB)^{k-1}\bfA
			\visible<+->{=\bfB\cdots 3^{k-1}\cdot \bfA}
			\visible<+->{=3^{k-1}\bfB\bfA=\begin{pmatrix}
				-3^{k-1}&-2\cdot 3^{k-1}&-3^k\\
				2\cdot 3^{k-1}&4\cdot 3^{k-1}&2\cdot3^{k}\\
				0&0&0
			\end{pmatrix}.}
		\end{align*}
	\end{solution}
\end{frame}


\begin{frame}{例: 矩阵的幂}
	\onslide<+->
	\begin{exercise}
		设 
		$\bfA=\begin{pmatrix}
			1&2&3\\
			2&4&6\\
			3&6&9
		\end{pmatrix}.$
		求 $\bfA^k$.
	\end{exercise}
	\onslide<+->
	\begin{answer}
		注意到 $\bfA=\begin{pmatrix}
			1\\2\\3
		\end{pmatrix}(1,2,3)$, 因此 $\bfA^k=14^{k-1}\bfA$.
	\end{answer}
	\onslide<+->
	想一想: $\bfA^2=\bfE$ 能推出 $\bfA=\bfE$ 或 $-\bfE$ 吗?
\end{frame}


\begin{frame}{矩阵幂的应用: 换乘}
	\onslide<+->
	网上订票系统里记录了所有能直飞的航班线路.
	对于不能直达的城市, 该怎么确定是否有换乘方案呢?
	\onslide<+->
	例如 $4$ 个城市之间的航线如图所示:
	\begin{center}
	\begin{tikzpicture}
		\node[dcolorb] at (0,0)  (1){1};
		\node[dcolorb] at (0,-2)  (2){2};
		\node[dcolorb] at (2,-2)  (3){3};
		\node[dcolorb] at (2,0)  (4){4};
		\draw[dcolorb] (1) circle (.2);
		\draw[dcolorb] (2) circle (.2);
		\draw[dcolorb] (3) circle (.2);
		\draw[dcolorb] (4) circle (.2);
		\draw[Latex-Latex, line width=0.5mm,dcolorc] (1.east) to (4.west);
		\draw[Latex-Latex, line width=0.5mm,dcolorc] (1.south) to (2.north);
		\draw[-Latex, line width=0.5mm,dcolorc] (1.south east) to (3.north west);
		\draw[-Latex, line width=0.5mm,dcolorc] (4.south) to (3.north);
		\draw[-Latex, line width=0.5mm,dcolorc] (3.west) to (2.east);
		\draw[visible on=<3->] (3,-1) node {\Large$\iff$}
		(6,-1) node{邻接矩阵 $\bfA=\begin{pmatrix}
			0&1&1&1\\
			1&0&0&0\\
			0&1&0&0\\
			1&0&1&0
		\end{pmatrix}$};
	\end{tikzpicture}
\end{center}
	\onslide<+->
	邻接矩阵中 $a_{ij}=1$ 表示从 $i$ 到 $j$ 有直飞航线.
\end{frame}


\begin{frame}{矩阵幂的应用: 换乘}
	\onslide<+->
	那么 $\bfA^2$ 的 $(i,j)$ 元
	\[b_{ij}=\sum_{k=1}^4 a_{ik}a_{kj}\]
	就是从 $i$ 到 $j$ 换乘一次的方案数.
	\onslide<+->
	例如从\circleno{$1$} $\implies$\circleno{$3$} :
	\[\bfA^2=\begin{pmatrix}
		0&1&1&1\\
		1&0&0&0\\
		0&1&0&0\\
		1&0&1&0
	\end{pmatrix}\begin{pmatrix}
		0&1&1&1\\
		1&0&0&0\\
		0&1&0&0\\
		1&0&1&0
	\end{pmatrix}=\begin{pmatrix}
		2&1&1&0\\
		0&1&1&1\\
		1&0&0&0\\
		0&2&1&1
	\end{pmatrix}.\]
	\onslide<+->
	由于 $b_{23}=1$, 因此可通过\circleno{$2$} $\implies$\circleno{$1$} $\implies$\circleno{$3$} 换乘一次到达.

	\onslide<+->
	想一想: 如何从\circleno{$3$} 到达\circleno{$4$} ?
\end{frame}

