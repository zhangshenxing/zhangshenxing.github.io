\section{行列式的性质}

\subsection{行列式的变换性质}

\begin{frame}{转置的行列式}
	\onslide<+->
	如果 $\bfA=(a_{ij})_{m\times n}$, 称
	\[\bfA^\rmT=\begin{bmatrix}
		a_{11}&a_{21}&\cdots&a_{m1}\\
		a_{12}&a_{22}&\cdots&a_{m2}\\
		\vdots&\vdots&\ddots&\vdots\\
		a_{1n}&a_{2n}&\cdots&a_{mn}
	\end{bmatrix}\]
	为矩阵 $\bfA$ 的\emph{转置}, 它是 $n\times m$ 矩阵.

	\onslide<+->
	注意到对于 $\bfP$,  $c_{k_i}\swap c_{k_j}$ 和 $r_i\swap r_j$ 是一样的,
	\onslide<+->
	即对 $\bfP^\rmT$ 进行 $c_i\swap c_j$ 得到的方阵的转置.
	\onslide<+->
	因此 $\det(\bfP^\rmT)=\det(\bfP)$.
	
	\onslide<+->
	而 $\det(\bfA)$ 是所有置换对应的型如 $\bfP$ 方阵行列式之和, 因此
	\begin{alertblock@}
		转置不改变行列式: $\det(\bfA^\rmT)=\det(\bfA)$.
	\end{alertblock@}
\end{frame}


\begin{frame}{上三角阵的行列式}
	\onslide<+->
	\begin{example}
		计算 $\det(\bfA)$, 其中 $\bfA=\begin{bmatrix}
			a_{11}&a_{12}&\cdots&a_{1n}\\
			      &a_{22}&\cdots&a_{2n}\\
						&			 &\ddots&\vdots\\
						&			 &     	&a_{nn}
		\end{bmatrix}$ 是\emph{上三角阵}.
	\end{example}
	\onslide<+->
	\begin{solution}
		由于
		\[\bfA^\rmT=\begin{bmatrix}
			a_{11}&      &      &\\
			a_{12}&a_{22}&      &\\
			\vdots&\vdots&\ddots&\\
			a_{1n}&a_{2n}&\cdots&a_{nn}
		\end{bmatrix}\]
		是下三角阵,
		\onslide<+->{因此
		$\det(\bfA)=\det(\bfA^\rmT)=a_{11}a_{22}\cdots a_{nn}.$}
	\end{solution}
\end{frame}


\begin{frame}{互换方阵的行的行列式}
	\onslide<+->
	注意到交换 $\bfP$ 的两列后得到的方阵 $\bfP'$ 对应的排列 $\sgn$ 发生了改变, 从而 $\det(\bfP')=-\det(\bfP)$.
	\onslide<+->
	由此可知互换两行后, 方阵的行列式变为 $-1$ 倍.
	\onslide<+->
	再根据转置不改变行列式可知:
	\begin{alertblock@}
		互换两行(列)后, 方阵的行列式变为 $-1$ 倍.
	\end{alertblock@}

	\onslide<+->
	如果方阵有相同的两行, 那么交换这两行方阵不变但行列式变为 $-1$ 倍.
	\onslide<+->
	于是行列式只能为 $0$.
	\begin{block@}
		具有相同的两行(列)的方阵的行列式为 $0$.
	\end{block@}
\end{frame}


\begin{frame}{行乘常数的行列式}
	\onslide<+->
	如果方阵某一行元素均乘 $k$, 那么行列式展开式的每一项都乘 $k$, 从而行列式乘 $k$.
	\onslide<+->
	\begin{alertblock@}
		方阵的某一行(列)乘 $k$ 后, 方阵的行列式变为 $k$ 倍.
	\end{alertblock@}
	\onslide<+->
	\begin{exercise}
		判断题: $\detm{k a_{11}&k a_{12}&\cdots&k a_{1n}\\ka_{21}&ka_{22}&\cdots&k a_{2n}\\&\vdots&\vdots&\ddots\\k a_{n1}&ka_{n2}&\cdots&k a_{nn}}=k\detm{a_{11}&a_{12}&\cdots&a_{1n}\\a_{21}&a_{22}&\cdots&a_{2n}\\&\vdots&\vdots&\ddots\\a_{n1}&a_{n2}&\cdots&a_{nn}}$. \visible<+->{\Huge\color{red}{$\times$}}
	\end{exercise}
	\onslide<+->
	\begin{block@}
		\begin{itemize}
			\item 行列式中某一行(列)的公因子可以提到行列式外面.
			\item 如果方阵有一行(列)全为零, 则行列式为零.
			\item 如果方阵有两行(列)成比例, 则行列式为零.
		\end{itemize}
	\end{block@}
\end{frame}


\begin{frame}{沿任一行(列)展开}
	\onslide<+->
	\begin{alertblock@}
		方阵的行列式等于任一行(列)的元素与其对应的代数余子式乘积的和:
		\[\det(\bfA)=a_{i1}A_{i1}+a_{i2}A_{i2}+\cdots+a_{in}A_{in}.\]
	\end{alertblock@}
	\onslide<+->
	将方阵的第 $i$ 行移动到第一行的前面, 需要的对换数等于排列 $i,1,2,\cdots,i-1$ 变成 $1,2,\dots,i$ 的对换数, 即 $i-1$ 次.
	\onslide<+->
	移动后的矩阵的行列式展开式乘 $(-1)^{i-1}$ 就是上述等式右边.
	
	\onslide<+->
	由此也可以看出 $i\neq k$ 时,
	\[a_{i1}A_{k1}+a_{i2}A_{k2}+\cdots+a_{in}A_{kn}=0,\]
	\onslide<+->
	因为它是第 $i,k$ 行相同的方阵的行列式.
\end{frame}


\begin{frame}{行列式的线性性}
	\onslide<+->
	\begin{alertblock@}
		将方阵一行(列)每一个元素都写成两个数之和, 则行列式也可拆成两个行列式之和:
		\[\detm{a_{11}&a_{12}&\cdots&a_{1n}\\
		&\vdots&\vdots&\ddots\\
		a_{i1}+a'_{i1}&a_{i2}+a'_{i2}&\cdots&a_{in}+a'_{in}\\
		&\vdots&\vdots&\ddots\\
		a_{n1}&a_{n2}&\cdots&a_{nn}}=\detm{a_{11}&a_{12}&\cdots&a_{1n}\\
		&\vdots&\vdots&\ddots\\
		a_{i1}&a_{i2}&\cdots&a_{in}\\
		&\vdots&\vdots&\ddots\\
		a_{n1}&a_{n2}&\cdots&a_{nn}}+\detm{a_{11}&a_{12}&\cdots&a_{1n}\\
		&\vdots&\vdots&\ddots\\
		a'_{i1}&a'_{i2}&\cdots&a'_{in}\\
		&\vdots&\vdots&\ddots\\
		a_{n1}&a_{n2}&\cdots&a_{nn}}.\]
	\end{alertblock@}
	\onslide<+->
	\begin{alertblock@}
		将方阵一行(列)乘常数 $k$ 再加到另一行(列), 行列式不变.
	\end{alertblock@}
\end{frame}


\subsection{使用初等变换计算行列式}
\begin{frame}{三种初等变换}
	\onslide<+->
	计算行列式可以通过下列变换来进行化简:
	\onslide<+->
	\begin{block}{初等变换}
		\begin{enumerate}
		\item 互换两行(列): \alert{$r_i\swap r_j, c_i\swap c_j$}, 行列式变号;
		\item 一行(列)乘常数 $k$: \alert{$r_i\times k, c_i\times k$}, 行列式变为 $k$ 倍;
		\item $j$ 行(列)乘 $k$ 加到 $i$ 行(列): \alert{$r_i+kr_j, c_i+kc_j$}.
	\end{enumerate}
	\end{block}
\end{frame}


\begin{frame}{例: 计算行列式}
	\onslide<+->
	\begin{example}
		\[\detm{
			 2& 3& 1&-1\\
			-4&-5& 1& 3\\
			-3& 1&-5& 3\\
			 1&-2& 0&-1}
		\onslide<+->{\hspace{-2mm}\xeq{r_1\swap r_4}}
		\onslide<+->{-\detm{
			1&-2& 0&-1\\
		 -4&-5& 1& 3\\
		 -3& 1&-5& 3\\
		  2& 3& 1&-1}}
		\onslide<+->{\xeq[r_4-2r_1]{r_2+4r_1, r_3+3r_1}}
		\onslide<+->{-\detm{
			1&-2& 0&-1\\
		  0&-13& 1&-1\\
		  0&-5&-5&0\\
		  0& 7& 1&1}}
		\]
		\[\onslide<+->{=-\detm{
				-13& 1&-1\\
				-5&-5&0\\
				 7& 1&1}}
			\onslide<+->{\xeq{r_3+r_1}}
		\onslide<+->{-\detm{
			-13& 1&-1\\
			-5&-5&0\\
			-6& 2&0}}
		\onslide<+->{=(-1)^{1+3}\detm{-5&-5\\-6&2}=-40.}
		\]
	\end{example}
\end{frame}


\begin{frame}{例: 计算行列式}
	\onslide<+->
	\begin{exercise}
		$\detm{-2&0&1\\
		501&200&299\\
		500&200&300}=$\fillblank{\visible<+->{$-200$}}.
	\end{exercise}
	\onslide<+->
	\begin{example}
		证明:
		$\detm{
			a_1+b_1&b_1+c_1&c_1+a_1\\
			a_2+b_2&b_2+c_2&c_2+a_2\\
			a_3+b_3&b_3+c_3&c_3+a_3
		}=2\detm{
			a_1&b_1&c_1\\
			a_2&b_2&c_2\\
			a_3&b_3&c_3
		}$.
	\end{example}
\end{frame}


\begin{frame}{例: 计算行列式}
	\begin{proof}
		\[\detm{
			a_1+b_1&b_1+c_1&c_1+a_1\\
			a_2+b_2&b_2+c_2&c_2+a_2\\
			a_3+b_3&b_3+c_3&c_3+a_3
		}\xeq{c_1-c_2}\detm{
			a_1-c_1&b_1+c_1&c_1+a_1\\
			a_2-c_2&b_2+c_2&c_2+a_2\\
			a_3-c_3&b_3+c_3&c_3+a_3
		}\]\[
		\onslide<+->{\xeq{c_1+c_3}\detm{
			2a_1&b_1+c_1&c_1+a_1\\
			2a_2&b_2+c_2&c_2+a_2\\
			2a_3&b_3+c_3&c_3+a_3
		}}
		\onslide<+->{=2\detm{
			a_1&b_1+c_1&c_1+a_1\\
			a_2&b_2+c_2&c_2+a_2\\
			a_3&b_3+c_3&c_3+a_3
		}}\]\[
		\onslide<+->{\xeq{c_3-c_1}2\detm{
			a_1&b_1+c_1&c_1\\
			a_2&b_2+c_2&c_2\\
			a_3&b_3+c_3&c_3
		}}
		\onslide<+->{\xeq{c_2-c_3}2\detm{
			a_1&b_1&c_1\\
			a_2&b_2&c_2\\
			a_3&b_3&c_3
		}.}\]
	\end{proof}
\end{frame}


\begin{frame}{例: 特殊形状行列式}
	\onslide<+->
	\begin{exercise}
		$\detm{
			a_1+b_1&b_1+c_1&c_1+d_1&d_1+a_1\\
			a_2+b_2&b_2+c_2&c_2+d_2&d_2+a_2\\
			a_3+b_3&b_3+c_3&c_3+d_3&d_3+a_3\\
			a_4+b_4&b_4+c_4&c_4+d_4&d_4+a_4
		}=$\fillblank{\visible<+->{$0$}}.
	\end{exercise}
	\onslide<+->
	\begin{example}
		计算 $n$ 阶行列式 $\detm{
			a&1&\cdots&1\\
			1&a&\cdots&1\\
			\vdots&\vdots&\ddots&\vdots\\
			1&1&\cdots&a
		}$.
	\end{example}
\end{frame}


\begin{frame}{例: 特殊形状行列式}
	\onslide<+->
	\begin{solution}
		\[\detm{
			a&1&\cdots&1\\
			1&a&\cdots&1\\
			\vdots&\vdots&\ddots&\vdots\\
			1&1&\cdots&a
		}\xeq[i\ge2]{c_1+c_i}\detm{
			a+n-1&1&\cdots&1\\
			a+n-1&a&\cdots&1\\
			\vdots&\vdots&\ddots&\vdots\\
			a+n-1&1&\cdots&a
		}
		\onslide<+->{=(a+n-1)\detm{
			1&1&\cdots&1\\
			1&a&\cdots&1\\
			\vdots&\vdots&\ddots&\vdots\\
			1&1&\cdots&a
		}}\]\[
		\onslide<+->{\xeq[i\ge2]{r_i-r_1}\detm{
			1&1&\cdots&1\\
			0&a-1&\cdots&0\\
			\vdots&\vdots&\ddots&\vdots\\
			0&0&\cdots&a-1
		}}
		\onslide<+->{=(a+n-1)(a-1)^{n-1}.}
		\]
	\end{solution}
\end{frame}


\begin{frame}{例: 特殊形状行列式}
	\onslide<+->
	\begin{example}
		计算 $n$ 阶行列式 $\detm{
			1&1&\cdots&1\\
			1&2&\cdots&0\\
			\vdots&\vdots&\ddots&\vdots\\
			1&0&\cdots&n
		}$.
	\end{example}
	\onslide<+->
	\begin{solution}
		\[\detm{
			1&1&\cdots&1\\
			1&2&\cdots&0\\
			\vdots&\vdots&\ddots&\vdots\\
			1&0&\cdots&n
		}\xeq[i\ge 2]{r_1-\frac1i r_i}
		\detm{
			1-\frac12-\cdots-\frac1n&1&\cdots&1\\
			0&2&\cdots&0\\
			\vdots&\vdots&\ddots&\vdots\\
			0&0&\cdots&n
		}
		=\Bigl(1-\frac12-\cdots-\frac1n\Bigr)n!.\]
	\end{solution}
\end{frame}


\begin{frame}{例: 特殊形状行列式}
	\onslide<+->
	\begin{example}
		设
		\[\bfA=\begin{bmatrix}
			a_{11}&a_{12}&\cdots&a_{1m}\\
			a_{21}&a_{22}&\cdots&a_{2m}\\
			\vdots&\vdots&\ddots&\vdots
			a_{m1}&a_{m2}&\cdots&a_{mm}\\
		\end{bmatrix},\qquad
		\bfB=\begin{bmatrix}
			b_{11}&b_{12}&\cdots&b_{1n}\\
			b_{21}&b_{22}&\cdots&b_{2n}\\
			\vdots&\vdots&\ddots&\vdots
			b_{n1}&b_{n2}&\cdots&b_{nn}\\
		\end{bmatrix},\qquad
		\bfC=\begin{bmatrix}
			a_{11}&a_{12}&\cdots&a_{1m}\\
			a_{21}&a_{22}&\cdots&a_{2m}\\
			\vdots&\vdots&\ddots&\vdots
			a_{m1}&a_{m2}&\cdots&a_{mm}\\
		\end{bmatrix},\qquad
		\bfB=\begin{bmatrix}
			b_{11}&b_{12}&\cdots&b_{1n}\\
			b_{21}&b_{22}&\cdots&b_{2n}\\
			\vdots&\vdots&\ddots&\vdots
			b_{n1}&b_{n2}&\cdots&b_{nn}\\
		\end{bmatrix}.\]
	\end{example}
\end{frame}
