\section{矩阵的线性运算、乘法和转置}


\subsection{矩阵的线性运算}
\begin{frame}{线性映射的加法和数乘}
	\onslide<+->
	给定线性变换 $f,g:\BR^n\ra\BR^m$ 和数 $\lambda\in\BR$,
	\onslide<+->
	定义
	\[(f+g)(\bfx)=f(\bfx)+g(\bfx),\qquad (\lambda f)(\bfx)=\lambda(f(\bfx)).\]
	\onslide<+->
	对任意 $\bfx\in\BR^n$,
	\[(f+g)(\bfx)=f(\bfx)+g(\bfx)=g(\bfx)+f(\bfx)=(g+f)(\bfx).\]
	故 $f+g=g+f$.
	\onslide<+->
	同理
	\begin{enumerate}
		\item $(f+g)+h=f+(g+h)$;
		\item $f+0=f$, 这里 $0$ 表示零映射: $0(\bfx)={\bf0}$;
		\item 对任意线性映射 $f$, 存在线性映射 $g=(-1)f$ 使得 $f+g=0$;
		\item $(\lambda \mu)f=\lambda(\mu f)=\mu(\lambda f)$;
		\item $(\lambda+\mu)f=\lambda f+\mu f$;
		\item $\lambda(f+g)=\lambda f+\lambda g$;
		\item $1\cdot f=f, 0\cdot f=0,\lambda\cdot 0=0$.
	\end{enumerate}
\end{frame}


\begin{frame}{矩阵的加法和数乘}
	\onslide<+->
	由此得到对应的矩阵的加法和数乘:
	\onslide<+->
	\begin{definition}
		设 $\bfA=(a_{ij})_{m\times n},\bfB=(b_{ij})_{m\times n}$, $\lambda\in\BR$.
		定义
			\[\bfA+\bfB=(a_{ij}+b_{ij})_{m\times n},\qquad 
			(\lambda f)(\bfx)=\lambda(f(\bfx)).\]
	\end{definition}
	\onslide<+->
	只有\emph{同型矩阵}(行列数都相同的矩阵)才能相加.
	\onslide<+->
	列矩阵的加法和数乘就是其对应的列向量的加法和数乘.

	\onslide<+->
	自然地, 矩阵的减法为
	\[\bfA-\bfB=(a_{ij}-b_{ij})_{m\times n}.\]
\end{frame}


\begin{frame}{矩阵加法的性质}
	\onslide<+->
	由线性映射的相应性质可知:
	\begin{enumerate}
		\item $\bfA+\bfB=\bfB+\bfA$;
		\item $(\bfA+\bfB)+\bfC=\bfA+(\bfB+\bfC)$;
		\item $\bfA+\bfO=\bfA$;
		\item 对于任意矩阵 $\bfA$, 存在矩阵 $\bfB=(-1)\bfA$ 使得 $\bfA+\bfB=\bfO$. 称 $\bfB$ 为 $\bfA$ 的\emph{负矩阵}, 记作 $-\bfA$.
		\item $(\lambda \mu)\bfA=\lambda(\mu\bfA)=\mu(\lambda\bfA)$;
		\item $(\lambda+\mu)\bfA=\lambda\bfA+\mu\bfA$;
		\item $\lambda(\bfA+\bfB)=\lambda\bfA+\lambda\bfB$;
		\item $1\cdot \bfA=\bfA, 0\cdot\bfA=\bfO,\lambda\bfO=\bfO$.
	\end{enumerate}
\end{frame}


\begin{frame}{矩阵线性运算的应用: 图像处理\noexer}
	\onslide<+->
	\begin{center}
		\includegraphics[height=3cm]{../image/matrix2.jpg}
	\end{center}

	\onslide<+->
	一张图片由一些像素构成, 上图包含 $341\times512$ 个像素.
	\onslide<+->
	在 \cnen{\color{red}{R}}{红}\cnen{\color{green}{G}}{绿}\cnen{\color{blue}{B}}{蓝} 颜色模式下, 每个像素包含红绿蓝三个通道, 每个通道为一个 $0\sim255$ 之间的整数, 数值越高对应颜色越亮.
	\onslide<+->
	例如
	\begin{itemize}
		\item \textcolor{red}{$R=0$}, \textcolor{green}{$G=0$}, \textcolor{blue}{$B=0$} 表示纯黑色;
		\item \textcolor{red}{$R=255$}, \textcolor{green}{$G=255$}, \textcolor{blue}{$B=255$} 表示\colorbox[gray]{0.5}{\textcolor{white}{纯白色}};
		\item \textcolor{red}{$R=255$}, \textcolor{green}{$G=0$}, \textcolor{blue}{$B=255$} 表示\textcolor{-green}{纯紫色}.
	\end{itemize}
	\onslide<+->
	若三个通道相同, 图片就是一张灰色的图.
	此时图片对应一个 $341\times512$ 的矩阵 $\bfA$.
\end{frame}


\begin{frame}{矩阵线性运算的应用: 图像处理\noexer}
	\onslide<+->
	想一想: 如何将这个图像变亮?
	\onslide<+->
	\begin{center}
		\begin{tikzpicture}
			\draw
				(0,0) node {\includegraphics[height=3cm]{../image/matrix2.jpg}}
				(8,0) node {\includegraphics[height=3cm]{../image/matrix4.jpg}}
				(4,0) node[fourth] {{\Huge$\implies$}};
		\end{tikzpicture}
	\end{center}

	\onslide<+->
	我们只需要增加每个元素的值, 例如(超过 $255$ 的需要修正为 $255$)
	\[\bfA+\begin{pmatrix}
		50&50&\cdots&50\\
		50&50&\cdots&50\\
		\vdots&\vdots&\ddots&\vdots\\
		50&50&\cdots&50
	\end{pmatrix},\qquad
	1.5\bfA,\qquad\frac12(\bfA+\begin{pmatrix}
		255&255&\cdots&255\\
		255&255&\cdots&255\\
		\vdots&\vdots&\ddots&\vdots\\
		255&255&\cdots&255
	\end{pmatrix}).\]
\end{frame}


\begin{frame}{矩阵线性运算的应用: 图像处理\noexer}
	\onslide<+->
	如何让图像反色?
	\begin{center}
		\begin{tikzpicture}
			\draw
				(0,0) node {\includegraphics[height=3cm]{../image/matrix2.jpg}}
				(8,0) node {\includegraphics[height=3cm]{../image/matrix3.jpg}}
				(4,0) node[fourth] {{\Huge$\implies$}};
		\end{tikzpicture}
	\end{center}

	\onslide<+->
	\[\begin{pmatrix}
		255&255&\cdots&255\\
		255&255&\cdots&255\\
		\vdots&\vdots&\ddots&\vdots\\
		255&255&\cdots&255
	\end{pmatrix}-\bfA.\]
\end{frame}

\subsection{矩阵的乘法}


\begin{frame}{线性变换的复合}
	\onslide<+->
	设线性映射
	\[f:\BR^n\to\BR^m,\quad
	g:\BR^p\to \BR^n,\quad 
	h=f\circ g:\BR^p\to\BR^m\]
	对应的矩阵为 $\bfA=(a_{ij})_{m\times n}, \bfB=(b_{ij})_{n\times p},\bfC=(c_{ij})_{m\times p}$.
	\onslide<+->
	如何用 $\bfA,\bfB$ 来表示 $\bfC$ 呢?

	\onslide<+->
	设 $\bfx=(x_1,\dots,x_p)^\rmT\in \BR^p$, 则
	\[g(\bfx)=(y_1,\dots,y_n)^\rmT, \quad
		y_k=\sum_{j=1}^p b_{kj} x_j.\]
	\onslide<+->
	\[f\bigl(g(\bfx)\bigr)=(z_1,\dots,z_m)^\rmT, \quad
	z_i=\sum_{k=1}^n a_{ik} y_k=\sum_{k=1}^n a_{ik} \sum_{j=1}^p b_{kj} x_j=\sum_{j=1}^p\Bigl(\sum_{k=1}^n a_{ik}b_{kj}\Bigr)x_j.\]
	\onslide<+->
	\[\implies\bfC=(c_{ij})_{m\times p},\qquad c_{ij}=\sum_{k=1}^n a_{ik}b_{kj}.\]
	\onslide<+->
	我们把它定义为矩阵的乘法 $\bfC=\bfA\bfB$.
\end{frame}


\begin{frame}{矩阵乘法的定义}
	\onslide<+->
	\begin{definition}
		设 $\bfA=(a_{ij})_{m\times n}, \bfB=(b_{ij})_{n\times p}$.
		定义矩阵的乘法为  $\bfC=\bfA\bfB=(c_{ij})_{m\times p}$, 其中
		\[c_{ij}=\sum_{k=1}^n a_{ik}b_{kj}.\]
	\end{definition}
	\onslide<+->
	只有第一个矩阵的列数等于第二个矩阵的行数才能相乘.
	\onslide<+->
	\[\begin{pmatrix}
		1&2\\2&1
	\end{pmatrix}\begin{pmatrix}
		0&0&0\\0&0&0\\0&0&0
	\end{pmatrix}=?\visible<+->{\alert{\text{\huge$\times$}}}\]
	\onslide<+->
	简单来说, $\bfA\bfB$ 的 $(i,j)$ 元就是 $\bfA$ 的 $i$ 行和 $\bfB$ 的 $j$ 列对应分量相乘后求和得到的.
\end{frame}


\begin{frame}{行向量与列向量的乘法}
	\onslide<+->
	设 $\bfA=(a_1,\dots,a_n)$ 是 $n$ 维行向量, $\bfB=(b_1,\dots,b_n)^\rmT$ 是 $n$ 维列向量.
	\onslide<+->
	$\bfA\bfB,\bfB\bfA=?$
	\onslide<+->
	\[\bfA\bfB=\sum_{i=1}^n a_i b_i,\qquad
	\bfB\bfA=(b_ia_j)_{n\times n}\in M_n.\]

	\onslide<+->
	对于矩阵 $\bfA=(a_{ij})_{m\times n}, \bfB=(b_{ij})_{n\times p}$.
	\onslide<+->
	$\bfA\bfB$ 的 $(i,j)$ 元其实就是 $\bfA$ 第 $i$ 行对应的行向量和 $\bfB$ 第 $j$ 列对应的列向量相乘得到的数($1$ 阶方阵):
	\[\begin{pmatrix}
		\bma_1\\
		\bma_2\\
		\vdots\\
		\bma_m
	\end{pmatrix}(\bmb_1,\bmb_2,\cdots,\bmb_p)=\begin{pmatrix}
		\bma_1\bmb_1&\bma_1\bmb_2&\cdots&\bma_1\bmb_p\\
		\bma_2\bmb_1&\bma_2\bmb_2&\cdots&\bma_2\bmb_p\\
		\vdots&\vdots&\ddots&\vdots\\
		\bma_m\bmb_1&\bma_m\bmb_2&\cdots&\bma_m\bmb_p
	\end{pmatrix}.\]
\end{frame}


\begin{frame}{例: 矩阵乘法的计算}
	\onslide<+->
	\begin{example}
		求矩阵 $\bfA=\begin{pmatrix}
			1&2&0&-1\\2&1&4&0
		\end{pmatrix}$ 与 $\bfB=\begin{pmatrix}
			2&0&1\\
			-2&3&1\\
			1&5&0\\
			1&-3&4
		\end{pmatrix}$ 的乘积 $\bfA\bfB$.
	\end{example}
	\onslide<+->
	\begin{solution}
		\[\begin{pmatrix}
			1&2&0&-1\\2&1&4&0
		\end{pmatrix}\begin{pmatrix}
			2&0&1\\
			-2&3&1\\
			1&5&0\\
			1&-3&4
		\end{pmatrix}=\begin{pmatrix}
			-3&9&-1\\
			6&23&3
		\end{pmatrix}.\]
	\end{solution}
\end{frame}


\begin{frame}{矩阵乘法的性质}
	\onslide<+->
	线性映射满足如下性质:
	\begin{enumerate}
		\item $(f\circ g)\circ h=f\circ(g\circ h)$;
		\item $\lambda(f\circ g)=(\lambda f)\circ g=f\circ (\lambda g)$;
		\item $f\circ(g+h)=f\circ g+f\circ h$;
		\item $\id_{\BR^m}\circ f=f\circ\id_{\BR^n}=f$, 这里 $\id$ 表示恒等映射: $\id(\bfx)=\bfx$;
		\item $0\circ f=f\circ 0=0$, 这里 $0$ 表示零映射.
	\end{enumerate}
	\onslide<+->
	由此可知矩阵乘法满足如下性质:
	\begin{enumerate}
		\item $(\bfA\bfB)\bfC=\bfA(\bfB\bfC)$;
		\item $\lambda(\bfA\bfB)=(\lambda\bfA)\bfB=\bfA(\lambda\bfB)$;
		\item $\bfA(\bfB+\bfC)=\bfA\bfB+\bfA\bfC$;
		\item 若 $\bfA\in M_{m\times n}$, 则 $\bfE_m \bfA=\bfA\bfE_n=\bfA$.
		\item 若 $\bfA\in M_{m\times n}$, 则 $\bfO_{p\times m} \bfA=\bfO_{p\times n}, \bfA\bfO_{n\times p}=\bfO_{m\times p}$.
	\end{enumerate}
\end{frame}


\begin{frame}{矩阵乘法和线性方程组}
	\onslide<+->
	矩阵 $\bfA$ 对应的线性变换就是
	\[f:\BR^n\to\BR^m,\ \bfx\mapsto \bfA\bfx.\]
	\onslide<+->
	因此解线性方程组
	\[\laeq[lclcclcl]{
		a_{11}&x_1+{}&a_{12}&x_2&{}+\cdots+{}&a_{1n}&x_n={}&b_1\\
		a_{21}&x_1+{}&a_{22}&x_2&{}+\cdots+{}&a_{2n}&x_n={}&b_2\\
		&&&&\vdots&&&\\
		a_{m1}&x_1+{}&a_{m2}&x_2&{}+\cdots+{}&a_{mn}&x_n={}&b_m
	}\]
	\onslide<+->
	等价于解矩阵方阵 \alert{$\bfA\bfx=\bfb$}, 其中
	\[\bfx=\begin{pmatrix}
		x_1\\x_2\\\vdots\\x_n
	\end{pmatrix},\qquad
	\bfb=\begin{pmatrix}
		b_1\\b_2\\\vdots\\b_m
	\end{pmatrix}.\]
\end{frame}


\begin{frame}{矩阵乘法无交换律和消去律}
	\beqskip{3pt}
	\onslide<+->
	\alert{矩阵的乘法不能随意交换顺序}.
	\onslide<+->
	一般称 $\bfA\bfB$ 为 \emph{$\bfA$ 左乘 $\bfB$}, 或者 \emph{$\bfB$ 右乘 $\bfA$}.

	\onslide<+->
	若 $\bfA\bfB=\bfB\bfA$, 则称 $\bfA,\bfB$ 是\emph{可交换}的.
	\onslide<+->
	此时 $\bfA,\bfB$ \alert{必为同阶方阵}.
	\onslide<+->
	例如
	\[\begin{pmatrix}
		1&2\\0&1
	\end{pmatrix}\begin{pmatrix}
		2&1\\0&2
	\end{pmatrix}=\begin{pmatrix}
		2&5\\0&2
	\end{pmatrix}=\begin{pmatrix}
		2&1\\0&2
	\end{pmatrix}\begin{pmatrix}
		1&2\\0&1
	\end{pmatrix}\]
	\onslide<+->
	方阵总和同阶单位阵交换.

	\onslide<+->
	矩阵乘法也没有消去律: $\bfA\bfB=\bfO$ 推不出 $\bfA=\bfO$ 或 $\bfB=\bfO$.
	\onslide<+->
	例如
	\[\begin{pmatrix}
		2&4\\1&2
	\end{pmatrix}\begin{pmatrix}
		2&-2\\-1&1
	\end{pmatrix}=\bfO_2.\]
	\onslide<+->
	由此可知: $\bfA\bfC=\bfB\bfC$ 推不出 $\bfA=\bfB$. 

	\onslide<+->
	\begin{exercise}
		设 $\bfA,\bfB$ 为 $n>1$ 阶方阵, 则 $\bfA+\bfA\bfB=$\fillbrace{\visible<+->{\alert{C}}}.
		\begin{taskschoice}(4)
			() $\bfA(1+\bfB)$
			() $(\bfE+\bfB)\bfA$
			() $\bfA(\bfE+\bfB)$
			() 以上都不对
		\end{taskschoice}
	\end{exercise}
	\endgroup
\end{frame}


\begin{frame}{例: 与给定矩阵可交换}\small
	\beqskip{2mm}
	\onslide<+->
	\begin{example}
		求与矩阵 $\bfA=\begin{pmatrix}
			0&1&0\\&0&1\\&&0
		\end{pmatrix}$ 可交换的所有矩阵.
	\end{example}
	\onslide<+->
	\begin{solution}
		设 $\bfB=(a_{ij})_{3\times 3}$ 与 $\bfA$ 可交换, 则
		\[\bfA\bfB=\begin{pmatrix}
			a_{21}&a_{22}&a_{23}\\
			a_{31}&a_{32}&a_{33}\\
			0&0&0
		\end{pmatrix}=\bfB\bfA=\begin{pmatrix}
			0&a_{11}&a_{12}\\
			0&a_{21}&a_{22}\\
			0&a_{31}&a_{32}
		\end{pmatrix},\]
		\onslide<+->{
			\[a_{21}=a_{31}=a_{32}=0,\quad a_{11}=a_{22}=a_{33},\quad a_{23}=a_{12},\]
		}\onslide<+->{%
		即 $\bfB=\begin{pmatrix}
			a_{11}&a_{12}&a_{13}\\
			&a_{11}&a_{12}\\
			&&a_{11}
		\end{pmatrix}$.}
	\end{solution}
	\endgroup
\end{frame}


\begin{frame}{矩阵乘法的应用: 图像校正\noexer}
	\onslide<+->
	某位同学拍身份证照片拍成了下图的样子, 如何才能修复好呢?
	\begin{center}
		\begin{tikzpicture}
			\draw (2.75,2) node {\includegraphics[height=4.03cm]{../image/idcard1.png}};
			\begin{scope}[main,visible on=<2->]
				\draw
					(0,0) node[below left] {$O$}
					(5.21,0.88) node[below right] {$A$}
					(0.19,3.11) node[above left] {$B$};
			\end{scope}
		\end{tikzpicture}
	\end{center}
	\onslide<+->
	以左下角为原点, 通过测量发现 $A$ 坐标为 $(521,88)$, $B$ 坐标为 $(19,311)$.

	\onslide<+->
	经过查询知道身份证长宽比为 $42.7:27$.
	\onslide<+->
	令 $A'=(427,0),B'=(0,270)$.
	我们希望找到一个线性变换, 将 $A,B$ 变为 $A',B'$.
\end{frame}


\begin{frame}{矩阵乘法的应用: 图像校正\noexer}
	\onslide<+->
	设该线性变换对应的矩阵为 $\bfA=\begin{pmatrix}
		a&b\\c&d
	\end{pmatrix}$, 则
	\[\bfA\begin{pmatrix}
		521&19\\88&311
	\end{pmatrix}=\begin{pmatrix}
		427&0\\0&270
	\end{pmatrix},\qquad
	\visible<+->{
		\text{即}\quad
	\laeq[crcrc]{
		521&a+{}& 88&b={}&427\\
		 19&a+{}&311&b={}&0\\
		521&c+{}& 88&d={}&0\\
		 19&c+{}&311&d={}&270
	}}\]
	\onslide<+->
	解得 $\bfA=\begin{pmatrix}
		0.828&-0.051\\
		-0.148&0.877
	\end{pmatrix}$.
	\onslide<+->
	通过应用该变换, 图片被修复成如下效果:
	\begin{center}
		\includegraphics[height=2.5cm]{../image/idcard.png}
	\end{center}
\end{frame}

\subsection{矩阵的幂}

\begin{frame}{矩阵幂的定义}
	\onslide<+->
	\begin{definition}
		设 $\bfA$ 为 $n$ 阶方阵, 定义 $\bfA$ 的\emph{幂}
		\[\bfA^0=\bfE_n,\quad \bfA^k=\underbrace{\bfA\cdot\bfA\cdot\cdots\cdot\bfA}_{k\ \text{个}}.\]
	\end{definition}
	\onslide<+->
	矩阵幂满足如下性质 ($k,\ell$ 为非负整数):
	\begin{enumerate}
		\item $\bfA^{k+\ell}=\bfA^k\cdot \bfA^\ell$;
		\item $\bfA^{k\ell}=(\bfA^k)^\ell$.
	\end{enumerate}
	\onslide<+->
	注意 $(\bfA\bfB)^k$ 一般不等于 $\bfA^k\cdot\bfB^k$.
	\onslide<+->
	想一想下面的等式成立吗?
	\[(\bfA-\bfB)(\bfA+\bfB)=\bfA^2-\bfB^2?\]
	\[(\bfA+\bfB)^2=\bfA^2+2\bfA\bfB+\bfB^2?\]
\end{frame}


\begin{frame}{例: 矩阵幂的计算}
	\onslide<+->
	\begin{example}
		设
		$\bfA=\diag(\lambda_1,\cdots,\lambda_n).$
		求 $\bfA^k$.
	\end{example}
	\onslide<+->
	\begin{solution}
		\[\bfA^2=\bfA\cdot\bfA=\diag(\lambda_1^2,\cdots,\lambda_n^2),\]
		\vspace{-\baselineskip}
		\onslide<+->{%
			\[\bfA^3=\bfA\cdot\bfA^2=\diag(\lambda_1^3,\cdots,\lambda_n^3),\]
		}\onslide<+->{%
			递推下去可知
			\[\bfA^k=\diag(\lambda_1^k,\cdots,\lambda_n^k).\]
		}\vspace{-\baselineskip}
	\end{solution}
\end{frame}


\begin{frame}{例: 矩阵幂的计算}\small
\beqskip{0pt}
	\onslide<+->
	\begin{example}
		设 $\bfA=\begin{pmatrix}
			\lambda&1&0\\
			&\lambda&1\\
			&&\lambda
		\end{pmatrix}.$
		求 $\bfA^k$.
	\end{example}
	\onslide<+->
	\begin{solution}
		\[\bfA^2=\begin{pmatrix}
			\lambda&1&0\\
			&\lambda&1\\
			&&\lambda
		\end{pmatrix}\begin{pmatrix}
			\lambda&1&0\\
			&\lambda&1\\
			&&\lambda
		\end{pmatrix}=\begin{pmatrix}
			\lambda^2&2\lambda&1\\
			&\lambda^2&2\lambda\\
			&&\lambda^2
		\end{pmatrix},\]
		\onslide<+->{%
			\[\bfA^3=\begin{pmatrix}
				\lambda&1&0\\
				&\lambda&1\\
				&&\lambda
			\end{pmatrix}\begin{pmatrix}
				\lambda^2&2\lambda&1\\
				&\lambda^2&2\lambda\\
				&&\lambda^2
			\end{pmatrix}=\begin{pmatrix}
				\lambda^3&3\lambda^2&3\lambda\\
				&\lambda^3&3\lambda^2\\
				&&\lambda^3
			\end{pmatrix}.\]
		}\onslide<+->{%
			归纳可知
			$\bfA^k=\begin{pNiceMatrix}
				\lambda^k&k\lambda^{k-1}&\frac12k(k-1)\lambda^{k-2}\\
				&\lambda^k&k\lambda^{k-1}\\
				&&\lambda^k
			\end{pNiceMatrix}$.
		}\vspace{-.2\baselineskip}
	\end{solution}
\endgroup
\end{frame}


\begin{frame}{例: 矩阵幂的计算}
	\onslide<+->
	\begin{solution}[另解]
		设 $\bfN=\begin{pmatrix}
			0&1&0\\
			&0&1\\
			&&0
		\end{pmatrix}$, 则
		$\bfN^2=\begin{pmatrix}
			0&0&1\\
			&0&0\\
			&&0
		\end{pmatrix},\bfN^3=\bfO.$
		\onslide<+->{%
			由于 $\bfA=\lambda\bfE+\bfN$ 且 $\bfE$ 和 $\bfN$ \alert{可交换},
		}\onslide<+->{%
			因此
			\begin{align*}
				\bfA^k&=(\lambda\bfE+\bfN)^k=\lambda^k\bfE+\rmC_k^1\lambda^{k-1} \bfN+\rmC_k^2\lambda^{k-2}\bfN^2\\
				&\visible<+->{=\begin{pNiceMatrix}
					\lambda^k&k\lambda^{k-1}&\dfrac12k(k-1)\lambda^{k-2}\\
					&\lambda^k&k\lambda^{k-1}\\
					&&\lambda^k
				\end{pNiceMatrix}.}
			\end{align*}
		}\vspace{-\baselineskip}
	\end{solution}
\end{frame}


\begin{frame}{例: 矩阵幂的计算}
	\onslide<+->
	\begin{example}
		设
		$\bfA=\begin{pmatrix}
			\cos\theta&-\sin\theta\\
			\sin\theta&\cos\theta
		\end{pmatrix}.$
		求 $\bfA^k$.
	\end{example}
	\onslide<+->
	\begin{solution}
		注意到 $\bfA$ 对应平面 $\BR^2$ 上的线性变换是逆时针旋转 $\theta$, 所以 $\bfA^k$ 就是逆时针旋转 $n\theta$, 对应的矩阵为
		\[\bfA^k=\begin{pmatrix}
			\cos k\theta&-\sin k\theta\\
			\sin k\theta&\cos k\theta
		\end{pmatrix}.\]
	\end{solution}
\end{frame}


\begin{frame}{例: 矩阵幂的计算}
	\onslide<+->
	\begin{example}
		设 
		$\bfA=(1,2,3),\bfB=\begin{pmatrix}
			-1\\2\\0
		\end{pmatrix}.$
		求 $(\bfB\bfA)^k$.
	\end{example}
	\onslide<+->
	\begin{solution}
		注意到 $\bfA\bfB=3$,
		\onslide<+->{%
			因此
			\[(\bfB\bfA)^k=\bfB(\bfA\bfB)^{k-1}\bfA
				\visible<+->{=\bfB\cdot 3^{k-1}\cdot \bfA
					=3^{k-1}\bfB\bfA}
				\visible<+->{=\begin{pmatrix}
					-3^{k-1}&-2\cdot 3^{k-1}&-3^k\\
					2\cdot 3^{k-1}&4\cdot 3^{k-1}&2\cdot3^{k}\\
					0&0&0
				\end{pmatrix}.}\]
		}\vspace{-\baselineskip}
	\end{solution}
\end{frame}


\begin{frame}{例: 矩阵幂的计算}
	\onslide<+->
	\begin{exercise}
		设 
		$\bfA=\begin{pmatrix}
			1&2&3\\
			2&4&6\\
			3&6&9
		\end{pmatrix}.$
		求 $\bfA^k$.
	\end{exercise}
	\onslide<+->
	\begin{answer}
		注意到 $\bfA=\begin{pmatrix}
			1\\2\\3
		\end{pmatrix}(1,2,3)$, 因此 $\bfA^k=14^{k-1}\bfA=\begin{pmatrix}
			14^{k-1}&2\cdot14^{k-1}&3\cdot14^{k-1}\\
			2\cdot14^{k-1}&4\cdot14^{k-1}&6\cdot14^{k-1}\\
			3\cdot14^{k-1}&6\cdot14^{k-1}&9\cdot14^{k-1}
		\end{pmatrix}$.
	\end{answer}
	\onslide<+->
	想一想: $\bfA^2=\bfE$ 能推出 $\bfA=\bfE$ 或 $-\bfE$ 吗?
\end{frame}


\begin{frame}{矩阵幂的应用: 换乘\noexer}
	\onslide<+->
	网上订票系统里记录了所有能直飞的航班线路.
	对于不能直达的城市, 该怎么确定是否有换乘方案呢?
	\onslide<+->
	例如 $4$ 个城市之间的航线如图所示:
	\begin{center}
	\begin{tikzpicture}
		\begin{scope}[main, node distance=15mm]
			\node (1) {1};
			\node (2) [below=of 1] {2};
			\node (3) [right=of 2]{3};
			\node (4) [above=of 3]{4};
			\draw (1) circle (.2);
			\draw (2) circle (.2);
			\draw (3) circle (.2);
			\draw (4) circle (.2);
		\end{scope}
		\begin{scope}[line width=0.5mm,fourth]
			\draw[Latex-Latex] (1.east) -- (4.west);
			\draw[Latex-Latex] (1.south) -- (2.north);
			\draw[-Latex] (1.south east) -- (3.north west);
			\draw[-Latex] (3.west) -- (2.east);
			\draw[-Latex] (4.south) -- (3.north)
				node[midway,right,black,visible on=<3->] {{\Large$\iff$}\emph{邻接矩阵} $\bfA=\begin{pmatrix}
					0&1&1&1\\
					1&0&0&0\\
					0&1&0&0\\
					1&0&1&0
				\end{pmatrix}$};
		\end{scope}
	\end{tikzpicture}
\end{center}
	\onslide<+->
	邻接矩阵中 $a_{ij}=1$ 表示从 $i$ 到 $j$ 有直飞航线.
\end{frame}


\begin{frame}{矩阵幂的应用: 换乘\noexer}
	\onslide<+->
	于是 $\bfA^2$ 的 $(i,j)$ 元
	\[b_{ij}=\sum_{k=1}^4 a_{ik}a_{kj}\]
	就是从 $i$ 到 $j$ 换乘一次的方案数.
	\onslide<+->
	例如从\circleno{$2$} $\implies$\circleno{$3$} :
	\[\bfA^2=\begin{pNiceMatrix}
		0&1&1&1\\
		\alert1&0&0&0\\
		0&1&0&0\\
		1&0&1&0
		\CodeAfter
		\tikz \draw[cstdash,second] (2-|1) rectangle (3-|5);
	\end{pNiceMatrix}\begin{pNiceMatrix}
		0&1&\alert1&1\\
		1&0&0&0\\
		0&1&0&0\\
		1&0&1&0
		\CodeAfter
		\tikz \draw[cstdash,second] (1-|3) rectangle (5-|4);
	\end{pNiceMatrix}=\begin{pNiceMatrix}
		2&1&1&0\\
		0&1&\alert1&1\\
		1&0&0&0\\
		0&2&1&1
		\CodeAfter
		\tikz \draw[cstcurve,second,rounded corners] (2-|3) rectangle (3-|4);
	\end{pNiceMatrix}.\]
	\onslide<+->
	由于 $b_{23}=1$, 因此可通过\circleno{$2$} $\implies$\circleno{$1$} $\implies$\circleno{$3$} 换乘一次到达.

	\onslide<+->
	想一想: 如何从\circleno{$3$} 到达\circleno{$4$} ?
\end{frame}


\subsection{矩阵的转置}

\begin{frame}{矩阵转置的定义}
	\onslide<+->
	若 $\bfA=(a_{ij})_{m\times n}$, 称
	\[\bfA^\rmT=(a_{ji})_{n\times m}=\begin{pmatrix}
		a_{11}&a_{21}&\cdots&a_{m1}\\
		a_{12}&a_{22}&\cdots&a_{m2}\\
		\vdots&\vdots&\ddots&\vdots\\
		a_{1n}&a_{2n}&\cdots&a_{mn}
	\end{pmatrix}\in M_{\alert{n\times m}}\]
	为矩阵 $\bfA$ 的\emph{转置}.
	\onslide<+->
	如同我们之前使用过的写法, 行向量的转置就是列向量
	\[\begin{pmatrix}
			a_1\\\vdots\\a_n
		\end{pmatrix}=(a_1,\dots,a_n)^\rmT.\]
	\onslide<+->
	方阵的转置还是方阵, 上三角阵的转置是下三角阵.

	\onslide<+->
	$\bfA^\rmT$ 对应的线性变换是 $\bfA$ 对应的线性变换诱导的对偶空间上的线性变换, 感兴趣的可自行阅读有关材料.
\end{frame}


\begin{frame}{矩阵转置的性质}
	\onslide<+->
	矩阵的转置满足如下性质:
	\begin{enumerate}
		\item $(\bfA^\rmT)^\rmT=\bfA$;
		\item $(\bfA+\bfB)^\rmT=\bfA^\rmT+\bfB^\rmT$;
		\item $(\lambda\bfA)^\rmT=\lambda\bfA^\rmT$;
		\item \alert{$(\bfA\bfB)^\rmT=\bfB^\rmT\bfA^\rmT$}.
	\end{enumerate}
	\onslide<+->
	例如
	\[\begin{pmatrix}
		a_{11}&a_{12}&a_{13}\\
		a_{21}&a_{22}&a_{23}\\
		a_{31}&a_{32}&a_{33}
	\end{pmatrix}\begin{pmatrix}
		b_1\\b_2\\b_3
	\end{pmatrix}=\begin{pmatrix}
		c_1\\c_2\\c_3
	\end{pmatrix},\]
	\onslide<+->
	两边取转置得到
	\[(b_1,b_2,b_3)\begin{pmatrix}
		a_{11}&a_{21}&a_{31}\\
		a_{12}&a_{22}&a_{32}\\
		a_{13}&a_{23}&a_{33}
	\end{pmatrix}=(c_1,c_2,c_3).\]
\end{frame}


\begin{frame}{对称阵和反对称阵}
	\onslide<+->
	\begin{definition}
		\begin{itemize}
			\item 若方阵 $\bfA$ 满足 $\bfA^\rmT=\bfA$, 称 $\bfA$ 为\emph{对称阵};
			\item 若 $\bfA^\rmT=-\bfA$, 称 $\bfA$ 为\emph{反对称阵}.
		\end{itemize}
	\end{definition}
	\onslide<+->
	例如 $\begin{pmatrix}
		12&6&1\\
		6&8&0\\
		1&0&6
	\end{pmatrix}$
	是对称阵.
	\onslide<+->
	对角矩阵都是对称阵.

	\onslide<+->
	例如 $\begin{pmatrix}
		0&6&1\\
		-6&0&0\\
		-1&0&0
	\end{pmatrix}$
	是反对称阵.
	\onslide<+->
	反对称阵的对角线均为 $0$.
\end{frame}


\begin{frame}{对称阵和反对称阵}
	\onslide<+->
	\begin{example}
		证明: 若 $\bfA,\bfB,\bfA\bfB$ 都是对称阵, 则 $\bfA\bfB=\bfB\bfA$.
	\end{example}
	\onslide<+->
	\begin{proof}
		由题设可知 $\bfA^\rmT=\bfA,\bfB^\rmT=\bfB$,
		\[\bfA\bfB=(\bfA\bfB)^\rmT=\bfB^\rmT\bfA^\rmT=\bfB\bfA.\qedhere\]
	\end{proof}
	\onslide<+->
	想一想: 若 $\bfA,\bfB,\bfA\bfB$ 中有一个对称阵和两个反对称阵呢?
	\onslide<+->
	\begin{exercise}
		设 $\bfA$ 是 $n$ 阶方阵, \fillbrace{\visible<+->{\alert{A}}}一定是对称阵?
		\begin{taskschoice}(4)
			() $\bfA^\rmT\bfA$
			() $\bfA-\bfA^\rmT$
			() $\bfA^2$
			() $\bfA^\rmT-\bfA$
		\end{taskschoice}
	\end{exercise}
	\onslide<+->
	一般地, 若 $\bfA\in M_{m\times n}$, $\bfA\bfA^\rmT$ 是 $m$ 阶对称阵, $\bfA^\rmT\bfA$ 是 $n$ 阶对称阵.
\end{frame}


\begin{frame}{任一方阵可表为对称阵与反对称阵之和}
	\onslide<+->
	\begin{example}
		证明: 任一方阵均可写成一对称阵和一反对称阵之和.
	\end{example}
	\onslide<+->
	\begin{proof}
		\[\bfA=\frac{\bfA+\bfA^\rmT}2+\frac{\bfA-\bfA^\rmT}2.\qedhere\]
	\end{proof}
	\onslide<+->
	想一想:
	\begin{itemize}
		\item 若函数 $f(x)$ 的定义域关于原点对称, 则 $f(x)$ 可以表示成一个偶函数和一个奇函数之和.
		\item 复数 $z$ 可以写成 $z_1+z_2$, 其中 $\ov{z_1}=z_1,\ov{z_2}=-z_2$.
	\end{itemize}
\end{frame}

