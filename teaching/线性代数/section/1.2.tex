\section{行列式的性质}

\subsection{拉普拉斯展开}

\begin{frame}{互换方阵的列的行列式}
	\onslide<+->
	\begin{alertblock@}
		互换两列后, 方阵的行列式变为 $-1$ 倍.
	\end{alertblock@}
	\onslide<+->
	为了陈述方便, 记 \alert{$c_i\swap c_j (r_i\swap r_j)$} 为交换 $i,j$ 列(行).
	\onslide<+->
	注意 $c_i\swap c_j$ 可以通过
	\[c_j\swap c_{j-1},\quad c_{j-1}\swap c_{j-2},\quad \dots,\quad, c_{i+1}\swap c_i,\qquad
	c_{i+1}\swap c_{i+2},\quad \dots,\quad c_{j-1}\swap c_j\]
	实现, 一共 $2(j-i)+1$ 次.
	\onslide<+->
	因此只需证明互换相邻的两列后, 方阵的行列式变为 $-1$ 倍.

	\onslide<+->
	对方阵的阶 $n$ 归纳.
	\onslide<+->
	当 $n=2$ 时显然成立.

	\onslide<+->
	如果命题对于 $n-1$ 成立, 对于 $n$ 阶方阵 $\bfA=(a_{ij})$, 交换它的 $k,k+1$ 列得到方阵 $\bfB=(b_{ij})$.
	\onslide<+->
	设 $\bfA$ 在 $(i,j)$ 处的余子式为 $M_{ij}$, $\bfB$ 在 $(i,j)$ 处的余子式为 $N_{ij}$.
\end{frame}


\begin{frame}{互换方阵的列的行列式}
	\onslide<+->
	\begin{itemize}
		\item 当 $j\neq k,k+1$ 时, $\bfB$ 去掉 $1$ 行 $j$ 列得到的方阵是$\bfA$ 去掉 $1$ 行 $j$ 列得到的方阵互换两列得到的.
		\onslide<+->
		因此 $N_{1j}=-M_{1j}$.
		\item 当 $j=k$ 时, $\bfB$ 去掉 $1$ 行 $k$ 列得到的方阵是$\bfA$ 去掉 $1$ 行 $k+1$ 列得到的方阵.
		\onslide<+->
		因此 $N_{1k}=M_{1,k+1}$.
		\item 同理 $N_{1,k+1}=M_{1k}$.
	\end{itemize}
	\onslide<+->
	故
	\begin{align*}
		|\bfB|&=\sum_{j\neq k,k+1} (-1)^{1+j}a_{1j} N_{1j}
		+(-1)^{1+k}a_{1,k+1}N_{1k}+(-1)^{1+(k+1)}a_{1,k}N_{1,k+1}\\
		&=-\sum_{j\neq k,k+1} (-1)^{1+j}a_{1j} M_{1j}
		+(-1)^{1+k}a_{1,k+1}M_{1,k+1}+(-1)^{k}a_{1,k}M_{1k}\\
		&=-|\bfA|.
	\end{align*}
\end{frame}


\begin{frame}{转置的行列式}
	\onslide<+->
	如果 $\bfA=(a_{ij})_{m\times n}$, 称
	\[\bfA^\rmT=\begin{pmatrix}
		a_{11}&a_{21}&\cdots&a_{m1}\\
		a_{12}&a_{22}&\cdots&a_{m2}\\
		\vdots&\vdots&\ddots&\vdots\\
		a_{1n}&a_{2n}&\cdots&a_{mn}
	\end{pmatrix}\]
	为矩阵 $\bfA$ 的\emph{转置}, 它是 $n\times m$ 矩阵.
	\begin{alertblock@}
		\begin{enumerate}
			\item 转置不改变行列式: $|\bfA^\rmT|=|\bfA|$.
		\end{enumerate}
	\end{alertblock@}
\end{frame}


\begin{frame}{转置的行列式}
	\onslide<+->
	根据行列式的归纳定义可知, $|\bfA|$ 是由一些 $\pm a_{1 k_1}a_{2 k_2}\cdots a_{n k_n}$ 相加得到, 其中 $k_1 k_2 \cdots k_n$ 取遍 $1,2,\dots,n$ 的所有排列.
	\onslide<+->
	这个 $\pm$ 与具体的 $a_{ij}$ 取值无关, 因此它也就是 $|\bfP|$, 其中 $\bfP$ 的 $i$ 行 $k_i$ 列为 $1$, 其余项为零.

	\onslide<+->
	设 $\ell_1,\dots,\ell_n$ 是一个排列且满足 $k_{\ell_i}=i$.
	\onslide<+->
	也就是说, 如果把排列看成集合 $\{1,2,\dots,n\}$ 到自身的双射, $\ell$ 就是 $k$ 的逆映射.

	\onslide<+->
	现在
	\begin{align*}
		|\bfA|&=\sum|\bfP| a_{1k_1}\cdots a_{nk_n},\\
		|\bfA^\rmT|&=\sum|\bfP| a_{k_1 1}\cdots a_{k_n n}\\
		&=\sum|\bfP| a_{1\ell_1}\cdots a_{1 \ell_n}.
	\end{align*}
	\onslide<+->
	因此只需证明 $|\bfP|=|\bfQ|$, 其中 $\bfQ$ 的 $i$ 行 $\ell_i$ 列为 $1$, 其余项为零.
	\onslide<+->
	换言之, $\bfQ=\bfP^\rmT$.
\end{frame}


\begin{frame}{互换方阵的行(列)的行列式}
	\onslide<+->
	注意到交换 $\bfP$ 的 $k_i,k_j$ 列和交换 $i,j$ 行是一回事.
	\onslide<+->
	如果 $\bfP$ 可通过 $a$ 次互换列变成单位矩阵 $\bfE_n$, 那么 $|\bfP|=(-1)^a$ 且 $\bfP$ 可通过 $a$ 次互换行变成单位矩阵.
	\onslide<+->
	所以 $\bfQ$ 可通过 $a$ 次互换列变成单位矩阵, $|\bfQ|=(-1)^a=|\bfP|$.
	从而命题得证.
	
	\onslide<+->
	由此可知:
	\begin{alertblock@}
	\begin{enumerate}
		\setcounter{enumi}{1}
		\item 互换两行(列)后, 方阵的行列式变为 $-1$ 倍.
	\end{enumerate}
	\end{alertblock@}
	\onslide<+->
	如果方阵有相同的两行, 那么交换这两行方阵不变但行列式变为 $-1$ 倍.
	\onslide<+->
	于是行列式只能为 $0$.
	\begin{corollary}
		具有相同的两行(列)的方阵的行列式为 $0$.
	\end{corollary}
\end{frame}


\begin{frame}{上三角阵的行列式}
	\onslide<+->
	\begin{example}
		计算 $|\bfA|$, 其中 $\bfA=\begin{pmatrix}
			a_{11}&a_{12}&\cdots&a_{1n}\\
			      &a_{22}&\cdots&a_{2n}\\
						&			 &\ddots&\vdots\\
						&			 &     	&a_{nn}
		\end{pmatrix}$ 是\emph{上三角阵}.
	\end{example}
	\onslide<+->
	\begin{solution}
		由于
		\[\bfA^\rmT=\begin{pmatrix}
			a_{11}&      &      &\\
			a_{12}&a_{22}&      &\\
			\vdots&\vdots&\ddots&\\
			a_{1n}&a_{2n}&\cdots&a_{nn}
		\end{pmatrix}\]
		是下三角阵,
		\onslide<+->{因此
		$|\bfA|=|\bfA^\rmT|=a_{11}a_{22}\cdots a_{nn}.$}
	\end{solution}
\end{frame}


\begin{frame}{拉普拉斯展开}
	\onslide<+->
	\begin{theorem}[行列式沿任一行(列)展开, 拉普拉斯展开]
		方阵的行列式等于任一行(列)的元素与其对应的代数余子式乘积的和:
		\begin{align*}
			|\bfA|&=a_{i1}A_{i1}+a_{i2}A_{i2}+\cdots+a_{in}A_{in}\\
			&=a_{1j}A_{1j}+a_{2j}A_{2j}+\cdots+a_{nj}A_{nj}.
		\end{align*}
	\end{theorem}
	\onslide<+->
	\begin{proofs}
	设将方阵的第 $i$ 行移动到第一行的前面得到的方阵为 $\bfB$.
	\onslide<+->
	那么 $\bfB$ 就是 $\bfA$ 通过
	\[r_i\swap r_{i-1}, r_{i-1}\swap r_{i-2},\dots,r_2\swap r_1\]
	\onslide<+->
	一共 $i-1$ 次行互换得到的.
	\onslide<+->
	从而 $|\bfB|=(-1)^{i-1}|\bfA|$.
	\end{proofs}
\end{frame}


\begin{frame}{拉普拉斯展开}
	\onslide<+->
	\begin{proofe}
	$\bfB$ 在 $(1,j)$ 处的元素是 $a_{ij}$, 余子式是 $M_{ij}$, 因此
	\[|\bfB|=a_{i1}M_{i1}-a_{i2}M_{i2}+\cdots+(-1)^{n+1}a_{in}M_{in}.\]
	\onslide<+->
	两边同时乘以 $(-1)^{i+1}$ 得到
	\begin{align*}
		|\bfA|&=(-1)^{i+1}M_{i1}+(-1)^{i+2}M_{i2}+\cdots+(-1)^{i+n}M_{in}\\
		&=a_{i1}A_{i1}+a_{i2}A_{i2}+\cdots+a_{in}A_{in}.
	\end{align*}
	\onslide<+->
	再根据转置不改变行列式得到行列式沿一列展开的形式.
	\end{proofe}
	
	\onslide<+->
	由此也可以看出 $i\neq k$ 时,
	\[a_{i1}A_{k1}+a_{i2}A_{k2}+\cdots+a_{in}A_{kn}=0,\]
	\onslide<+->
	因为它是第 $i,k$ 行相同的方阵的行列式.
\end{frame}


\subsection{行列式的变换性质}
\begin{frame}{行乘常数的行列式}
	\onslide<+->
	如果方阵某一行元素均乘 $k$, 那么沿着这一行展开会发现行列式乘 $k$.
	\onslide<+->
	\begin{alertblock@}
		\begin{enumerate}
			\setcounter{enumi}{2}
			\item 方阵的某一行(列)乘 $k$ 后, 方阵的行列式变为 $k$ 倍.
		\end{enumerate}
	\end{alertblock@}
	\onslide<+->
	\begin{exercise}
		判断题: $\detm{k a_{11}&k a_{12}&\cdots&k a_{1n}\\ka_{21}&ka_{22}&\cdots&k a_{2n}\\&\vdots&\vdots&\ddots\\k a_{n1}&ka_{n2}&\cdots&k a_{nn}}=k\detm{a_{11}&a_{12}&\cdots&a_{1n}\\a_{21}&a_{22}&\cdots&a_{2n}\\&\vdots&\vdots&\ddots\\a_{n1}&a_{n2}&\cdots&a_{nn}}$. \visible<+->{\Huge\color{red}{$\times$}}
	\end{exercise}
	\onslide<+->
	\begin{corollary}
		\begin{itemize}
			\item 行列式中某一行(列)的公因子可以提到行列式外面.
			\item 如果方阵有一行(列)全为零, 则行列式为零.
			\item 如果方阵有两行(列)成比例, 则行列式为零.
		\end{itemize}
	\end{corollary}
\end{frame}


\begin{frame}{行列式的线性性}
	\onslide<+->
	\begin{alertblock@}
		\begin{enumerate}
			\setcounter{enumi}{4}
			\item 将方阵一行(列)每一个元素都写成两个数之和, 则行列式也可拆成两个行列式之和:
		\[\detm{a_{11}&a_{12}&\cdots&a_{1n}\\
		&\vdots&\vdots&\ddots\\
		a_{i1}+a'_{i1}&a_{i2}+a'_{i2}&\cdots&a_{in}+a'_{in}\\
		&\vdots&\vdots&\ddots\\
		a_{n1}&a_{n2}&\cdots&a_{nn}}=\detm{a_{11}&a_{12}&\cdots&a_{1n}\\
		&\vdots&\vdots&\ddots\\
		a_{i1}&a_{i2}&\cdots&a_{in}\\
		&\vdots&\vdots&\ddots\\
		a_{n1}&a_{n2}&\cdots&a_{nn}}+\detm{a_{11}&a_{12}&\cdots&a_{1n}\\
		&\vdots&\vdots&\ddots\\
		a'_{i1}&a'_{i2}&\cdots&a'_{in}\\
		&\vdots&\vdots&\ddots\\
		a_{n1}&a_{n2}&\cdots&a_{nn}}.\]
		\end{enumerate}
	\end{alertblock@}
	\onslide<+->
	\begin{alertblock@}
		\begin{enumerate}
			\setcounter{enumi}{3}
			\item 将方阵一行(列)乘常数 $k$ 再加到另一行(列), 行列式不变.
		\end{enumerate}
	\end{alertblock@}
\end{frame}


\subsection{使用初等变换和拉普拉斯展开计算行列式}
\begin{frame}{三种初等变换}
	\onslide<+->
	计算行列式可以通过下列变换来进行化简:
	\onslide<+->
	\begin{block}{初等变换}
		\begin{enumerate}
		\item 互换两行(列): \alert{$r_i\swap r_j, c_i\swap c_j$}, 行列式变号;
		\item 一行(列)乘常数 $k$: \alert{$r_i\times k, c_i\times k$}, 行列式变为 $k$ 倍;
		\item $j$ 行(列)乘 $k$ 加到 $i$ 行(列): \alert{$r_i+kr_j, c_i+kc_j$}.
	\end{enumerate}
	\end{block}
\end{frame}


\begin{frame}{例: 计算行列式}
	\onslide<+->
	\begin{example}
		\[\detm{
			 2& 3& 1&-1\\
			-4&-5& 1& 3\\
			-3& 1&-5& 3\\
			 1&-2& 0&-1}
		\onslide<+->{\hspace{-2mm}\xeq{r_1\swap r_4}}
		\onslide<+->{-\detm{
			1&-2& 0&-1\\
		 -4&-5& 1& 3\\
		 -3& 1&-5& 3\\
		  2& 3& 1&-1}}
		\onslide<+->{\xeq[r_4-2r_1]{r_2+4r_1, r_3+3r_1}}
		\onslide<+->{-\detm{
			1&-2& 0&-1\\
		  0&-13& 1&-1\\
		  0&-5&-5&0\\
		  0& 7& 1&1}}
		\]
		\[\onslide<+->{=-\detm{
				-13& 1&-1\\
				-5&-5&0\\
				 7& 1&1}}
			\onslide<+->{\xeq{r_3+r_1}}
		\onslide<+->{-\detm{
			-13& 1&-1\\
			-5&-5&0\\
			-6& 2&0}}
		\onslide<+->{=(-1)^{1+3}\detm{-5&-5\\-6&2}=-40.}
		\]
	\end{example}
\end{frame}


\begin{frame}{例: 计算行列式}
	\onslide<+->
	\begin{exercise}
		$\detm{-2&0&1\\
		501&200&299\\
		500&200&300}=$\fillblank{\visible<+->{$-200$}}.
	\end{exercise}
	\onslide<+->
	\begin{example}
		证明:
		$\detm{
			a_1+b_1&b_1+c_1&c_1+a_1\\
			a_2+b_2&b_2+c_2&c_2+a_2\\
			a_3+b_3&b_3+c_3&c_3+a_3
		}=2\detm{
			a_1&b_1&c_1\\
			a_2&b_2&c_2\\
			a_3&b_3&c_3
		}$.
	\end{example}
\end{frame}


\begin{frame}{例: 计算行列式}
	\onslide<+->
	利用初等列变换来化简.
	\onslide<+->
	\begin{proof}
		\[\detm{
			a_1+b_1&b_1+c_1&c_1+a_1\\
			a_2+b_2&b_2+c_2&c_2+a_2\\
			a_3+b_3&b_3+c_3&c_3+a_3
		}\xeq{c_1-c_2}\detm{
			a_1-c_1&b_1+c_1&c_1+a_1\\
			a_2-c_2&b_2+c_2&c_2+a_2\\
			a_3-c_3&b_3+c_3&c_3+a_3
		}\]\[
		\onslide<+->{\xeq{c_1+c_3}\detm{
			2a_1&b_1+c_1&c_1+a_1\\
			2a_2&b_2+c_2&c_2+a_2\\
			2a_3&b_3+c_3&c_3+a_3
		}}
		\onslide<+->{=2\detm{
			a_1&b_1+c_1&c_1+a_1\\
			a_2&b_2+c_2&c_2+a_2\\
			a_3&b_3+c_3&c_3+a_3
		}}\]\[
		\onslide<+->{\xeq{c_3-c_1}2\detm{
			a_1&b_1+c_1&c_1\\
			a_2&b_2+c_2&c_2\\
			a_3&b_3+c_3&c_3
		}}
		\onslide<+->{\xeq{c_2-c_3}2\detm{
			a_1&b_1&c_1\\
			a_2&b_2&c_2\\
			a_3&b_3&c_3
		}.}\qedhere\]
	\end{proof}
\end{frame}


\begin{frame}{例: 计算行列式}
	\onslide<+->
	\begin{exercise}
		$\detm{
			a_1+b_1&b_1+c_1&c_1+d_1&d_1+a_1\\
			a_2+b_2&b_2+c_2&c_2+d_2&d_2+a_2\\
			a_3+b_3&b_3+c_3&c_3+d_3&d_3+a_3\\
			a_4+b_4&b_4+c_4&c_4+d_4&d_4+a_4
		}=$\fillblank{\visible<+->{$0$}}.
	\end{exercise}
	\onslide<+->
	\begin{exercise}
		$\detm{
			1&1&1\\
			a&b&c\\
			b+c&c+a&a+b
		}=$\fillblank{\visible<+->{$0$}}.
	\end{exercise}
\end{frame}


\begin{frame}{例: 计算行列式}
	\onslide<+->
	\begin{exercise}
		解方程 $\detm{
			x+1&2&-1\\
			2&x+1&1\\
			-1&1&x+1}=0$.
	\end{exercise}
	\onslide<+->
	\begin{answer}
		$x=-3,\pm\sqrt3$.
	\end{answer}
	\onslide<+->
	\begin{example}
		如果 $abcd=1$, 证明 $\bfA=\begin{pmatrix}
			a^2+\frac1{a^2}&a&\frac1a&1\\[2pt]
			b^2+\frac1{b^2}&b&\frac1b&1\\[2pt]
			c^2+\frac1{c^2}&c&\frac1c&1\\[2pt]
			d^2+\frac1{d^2}&d&\frac1d&1
		\end{pmatrix}$ 行列式为零.
	\end{example}
\end{frame}


\begin{frame}{例: 计算行列式}
	\onslide<+->
	我们发现第一列可以拆分成两组数相加, 从而行列式也可以拆分成两个数.
	\onslide<+->
	\begin{proof}
		\begin{align*}
			|\bfA|&=\detm{
				a^2&a&\dfrac1a&1\\[7pt]
				b^2&b&\dfrac1b&1\\[7pt]
				c^2&c&\dfrac1c&1\\[7pt]
				d^2&d&\dfrac1d&1}+\detm{
				\dfrac1{a^2}&a&\dfrac1a&1\\[7pt]
				\dfrac1{b^2}&b&\dfrac1b&1\\[7pt]
				\dfrac1{c^2}&c&\dfrac1c&1\\[7pt]
				\dfrac1{d^2}&d&\dfrac1d&1}
			\onslide<+->{=abcd\detm{
				a&1&\dfrac1{a^2}&\dfrac1a\\[7pt]
				b&1&\dfrac1{b^2}&\dfrac1b\\[7pt]
				c&1&\dfrac1{c^2}&\dfrac1c\\[7pt]
				d&1&\dfrac1{d^2}&\dfrac1d}+\detm{
				a&\dfrac1{a^2}&1&\dfrac1a\\[7pt]
				b&\dfrac1{b^2}&1&\dfrac1b\\[7pt]
				c&\dfrac1{c^2}&1&\dfrac1c\\[7pt]
				d&\dfrac1{d^2}&1&\dfrac1d}}\\
				&\onslide<+->{=0.}\qedhere
		\end{align*}
	\end{proof}
\end{frame}


\begin{frame}{例: 特殊形状行列式}
	\onslide<+->
	计算 $n$ 阶矩阵的行列式可以使用初等变换将其变为三角型,也可以使用拉普拉斯展开来对其进行降阶。
	\onslide<+->
	\begin{example}
		\begin{align*}
			\detm{
			a&1&\cdots&1\\
			1&a&\cdots&1\\
			\vdots&\vdots&\ddots&\vdots\\
			1&1&\cdots&a
		}
		&\onslide<+->{\xeq[i\ge2]{c_1+c_i}\detm{
			a+n-1&1&\cdots&1\\
			a+n-1&a&\cdots&1\\
			\vdots&\vdots&\ddots&\vdots\\
			a+n-1&1&\cdots&a
		}}
		\onslide<+->{=(a+n-1)\detm{
			1&1&\cdots&1\\
			1&a&\cdots&1\\
			\vdots&\vdots&\ddots&\vdots\\
			1&1&\cdots&a
		}}\\
		&\onslide<+->{\xeq[i\ge2]{r_i-r_1}\detm{
			1&1&\cdots&1\\
			0&a-1&\cdots&0\\
			\vdots&\vdots&\ddots&\vdots\\
			0&0&\cdots&a-1
		}}
		\onslide<+->{=(a+n-1)(a-1)^{n-1}.}
		\end{align*}
	\end{example}
\end{frame}


\begin{frame}{例: 箭形行列式}
	\onslide<+->
	\begin{example}
		\[\detm{
			1&1&\cdots&1\\
			1&2&\cdots&0\\
			\vdots&\vdots&\ddots&\vdots\\
			1&0&\cdots&n
		}\onslide<+->{\xeq[i\ge 2]{r_1-\frac1i r_i}
		\detm{
			1-\frac12-\cdots-\frac1n&1&\cdots&1\\
			0&2&\cdots&0\\
			\vdots&\vdots&\ddots&\vdots\\
			0&0&\cdots&n
		}}
		\onslide<+->{=\Bigl(1-\frac12-\cdots-\frac1n\Bigr)n!.}\]
	\end{example}
	\onslide<+->
	\begin{exercise}
		计算 $n$ 阶行列式 $\detm{
			1+a_1&a_2&\cdots&a_n\\
			a_1&1+a_2&\cdots&a_n\\
			\vdots&\vdots&\ddots&\vdots\\
			a_1&a_2&\cdots&1+a_n
		}=$\fillblank[4cm]{\visible<+->{$1+a_1+\cdots+a_n$}}.
	\end{exercise}
\end{frame}


\begin{frame}{例: 特殊形状行列式}
	\onslide<+->
	\begin{example}
		\begin{align*}
		&\detm{
			a_1-b&a_2&a_3&\cdots&a_n\\
			a_1&a_2-b&a_3&\cdots&a_n\\
			a_1&a_2&a_3-b&\cdots&a_n\\
			\vdots&\vdots&\vdots&\ddots&\vdots\\
			a_1&a_2&a_3&\cdots&a_n-b
		}\onslide<+->{\xeq[i=1,2,\dots,n-1]{r_i-r_{i+1}}
		\detm{
			-b& b& 0&\cdots&0\\
			 0&-b& b&\cdots&0\\
			 0& 0&-b&\cdots&0\\
			\vdots&\vdots&\vdots&\ddots&\vdots\\
			a_1&a_2&a_3&\cdots&a_n-b
		}}\\
		&\onslide<+->{\xeq[i=1,2,\dots,n-1]{c_{i+1}+c_i}
		\detm{
			-b& 0& 0&\cdots&0\\
			 0&-b& 0&\cdots&0\\
			 0& 0&-b&\cdots&0\\
			\vdots&\vdots&\vdots&\ddots&\vdots\\
			a_1&a_1+a_2&a_1+a_2+a_3&\cdots&a_1+\cdots+a_n-b
		}}\\
		&\onslide<+->{=(a_1+\cdots+a_n-b)(-b)^{n-1}.}
	\end{align*}
	\end{example}
\end{frame}


\begin{frame}{例: 特殊形状行列式}
	\onslide<+->
	\begin{exercise}
		计算 $n$ 阶行列式 $\detm{
			1&2&3&\cdots&n-1&n\\
			-1&0&3&\cdots&n-1&n\\
			-1&-2&0&\cdots&n-1&n\\
			\vdots&\vdots&\vdots&\ddots&\vdots&\vdots\\
			-1&-2&-3&\cdots&0&n\\
			-1&-2&-3&\cdots&-(n-1)&0
		}=$\fillblank{\visible<+->{$n!$}}.
	\end{exercise}
\end{frame}


\begin{frame}{例: 特殊形状行列式}
	\onslide<+->
	\begin{exercise}
		计算矩阵 $\bfA_n=\begin{pmatrix}
			   x   &   -1    &    0    &\cdots&   0  &   0  \\
			   0   &    x    &   -1    &\cdots&   0  &   0  \\
			   0   &    0    &    x    &\cdots&   0  &   0  \\
			\vdots &  \vdots & \vdots  &\ddots&\vdots&\vdots\\
			   0   &    0    &    0    &\cdots&   x  &  -1  \\
				a_n  & a_{n-1} & a_{n-2} &\cdots&  a_2 & x+a_1
		\end{pmatrix}$ 的行列式.
	\end{exercise}
	\onslide<+->
	\begin{solution}
		沿着第一列展开得到
		\[|\bfA_n|=x|\bfA_{n-1}|+(-1)^{1+n}a_n(-1)^{n-1}=x|\bfA_{n-1}|+a_n,\]
		\onslide<+->{因此 $|\bfA_n|=x^n+a_1x^{n-1}+a_2x^{n-2}+\cdots+a_n$.}
	\end{solution}
\end{frame}

\subsection{三对角和范德蒙型行列式}

\begin{frame}{例: 三对角矩阵行列式}
	\onslide<+->
	\begin{exercise}
		计算矩阵 $\bfA_n=\begin{pmatrix}
				2  &   1  &   0  &\cdots&   0  &   0  \\
				1  &   2  &   1  &\cdots&   0  &   0  \\
				0  &   1  &   2  &\cdots&   0  &   0  \\
			\vdots&\vdots&\vdots&\ddots&\vdots&\vdots\\
				0  &   0  &   0  &\cdots&   2  &   1  \\
				0  &   0  &   0  &\cdots&   1  &   2  \\
		\end{pmatrix}$ 的行列式.
	\end{exercise}
\end{frame}


\begin{frame}{例: 三对角矩阵行列式}
	\onslide<+->
	\begin{solution}
		设 $D_n=|\bfA_n|$.
		\onslide<+->
		沿着第一列展开得到
		\[|\bfA_n|=2|\bfA_{n-1}|-\detm{
			1  &   1  &   0  &\cdots&   0  &   0  \\
			0  &   2  &   1  &\cdots&   0  &   0  \\
			0  &   1  &   2  &\cdots&   0  &   0  \\
		\vdots&\vdots&\vdots&\ddots&\vdots&\vdots\\
			0  &   0  &   0  &\cdots&   2  &   1  \\
			0  &   0  &   0  &\cdots&   1  &   2  \\}=2|\bfA_{n-1}|-|\bfA_{n-2}|,\]
		\onslide<+->
		因此
		\[|\bfA_n|-|\bfA_{n-1}|=|\bfA_{n-1}|-|\bfA_{n-2}|=\cdots=|\bfA_2|-|\bfA_1|=1,\]
		\onslide<+->
		从而 $|\bfA_n|=n-1+|\bfA_1|=n+1$.
	\end{solution}
\end{frame}


\begin{frame}{例: 三对角矩阵行列式}
	\onslide<+->
	如果主对角线元素均为 $2a$, 上下副对角线元素均为 $b$ 和 $c$, 则
		\[|\bfA_n|-2a|\bfA_{n-1}|+bc|\bfA_{n-2}|=0.\]
	\onslide<+->
	这种线性递推数列有通用解法.
	\onslide<+->
	设 $\lambda^2-2a\lambda+bc=0$ 的两个根为 $\lambda_1,\lambda_2$, 则
	\[|\bfA_n|=\begin{cases}
		\dfrac{\lambda_1^{n+1}-\lambda_2^{n+1}}{\lambda_1-\lambda_2},&\text{如果}\ \lambda_1\neq \lambda_2;\\[6pt]
		(n+1)a^n,&\text{如果}\ \lambda_1=\lambda_2=a.
	\end{cases}\]
\end{frame}

\begin{frame}{例: 范德蒙行列式}
	\onslide<+->
	\begin{exercise}
		如果 $\bfA=\begin{pmatrix}
			a_1&a_2&a_3&f\\
			b_1&b_2&b_3&f\\
			c_1&c_2&c_3&f\\
			d_1&d_2&d_3&f
		\end{pmatrix}$,
		那么 $A_{11}+A_{21}+A_{31}+A_{41}=$\fillblank{\visible<+->{$0$}}.
	\end{exercise}
	\onslide<+->
	\begin{example}[范德蒙行列式]
		设 $\bfA_n=\begin{pmatrix}
			1&1&1&\cdots&1\\
			x_1&x_2&x_3&\cdots&x_n\\
			x_1^2&x_2^2&x_3^2&\cdots&x_n^2\\
			\vdots&\vdots&\vdots&\ddots&\vdots\\
			x_1^{n-1}&x_2^{n-1}&x_3^{n-1}&\cdots&x_n^{n-1}
		\end{pmatrix}$.
		证明 \alert{$|\bfA_n|=\prod\limits_{1\le i<j\le n}(x_j-x_i)$}.
	\end{example}
\end{frame}


\begin{frame}{例: 范德蒙行列式}
	\onslide<+->
	\begin{proofs}
		归纳证明.
		\onslide<+->{当 $n=1,2$ 时显然成立.}

		\onslide<+->{设 $n\ge 3$, 则由 $r_i-x_1 r_{i-1}, i=n,n-1,\dots,2$ 得到
		\[|\bfA_n|=\detm{
		1&1&1&\cdots&1\\
		0&x_2-x_1&x_3-x_1&\cdots&x_n-x_1\\
		0&x_2(x_2-x_1)&x_3(x_3-x_1)&\cdots&x_n(x_n-x_1)\\
		\vdots&\vdots&\vdots&\ddots&\vdots\\
		0&x_2^{n-2}(x_2-x_1)&x_3^{n-2}(x_3-x_1)&\cdots&x_n^{n-2}(x_n-x_1)}.\]}
		\end{proofs}
	\end{frame}
	
	\begin{frame}{例: 范德蒙行列式}
		\onslide<+->
		\begin{proofe}
		\onslide<+->{
			沿着第一列展开, 然后提取每一列的公因式 $(x_j-x_1)$ 得到
		\[|\bfA_n|
		=\prod_{j=2}^n(x_j-x_1)\detm{
			1&1&\cdots&1\\
			x_2&x_3&\cdots&x_n\\
			\vdots&\vdots&\ddots&\vdots&\vdots\\
			x_2^{n-1}&x_3^{n-1}&\cdots&x_n^{n-1}}.\]}
			\onslide<+->{由归纳假设可知
			\[|\bfA_n|
		=\prod_{j=2}^n(x_j-x_1)\cdot \prod_{2\le i<j\le n}(x_j-x_i)=\prod_{1\le i<j\le n}(x_j-x_i).\qedhere\]}
	\end{proofe}
\end{frame}


\begin{frame}{例: 范德蒙行列式的应用}
	\onslide<+->
	\begin{exercise}
		\begin{enumerate}
			\item $\detm{
				x_1^{-3}&x_2^{-3}&x_3^{-3}&x_4^{-3}\\
				x_1^{-1}&x_2^{-1}&x_3^{-1}&x_4^{-1}\\
				x_1&x_2&x_3&x_4\\
				x_1^{3}&x_2^{3}&x_3^{3}&x_4^{3}
			}=$\fillblank[6cm][3mm]{\visible<4->{$x_1^{-3}x_2^{-3}x_3^{-3}x_4^{-3}\prod\limits_{1\le i<j\le 4}(x_j^2-x_i^2)$}}.
			\item 
		$\detm{
			1&1&1&1\\
			1&2&3&4\\
			1&4&9&16\\
			1&8&27&65
		}=$\fillblank{\visible<4->{$14$}}.
		\item
			设 $a,b,c$ 两两不等, 且 $\detm{
				a&b&c\\
				a^2&b^2&c^2\\
				b+c&c+a&a+b}0$, 则 $a+b+c=$\fillblank{\visible<4->{$0$}}.
		\end{enumerate}
	\end{exercise}
\end{frame}


\begin{frame}{行列式常见计算方法总结}
	\onslide<+->
	\begin{enumerate}
		\item $2,3$ 阶行列式可用对角线法直接展开.
		\item 上(下)三角阵行列式等于对角元的乘积.
		\item 行列式的计算一般需要用到\alert{三类初等变换}, 创造出足够多的零.
		\item 行列式沿一行(列)的展开往往是\alert{降阶法}的必要手段.
		\item 范德蒙型行列式可处理方阵为元素幂次递增的情形.
	\end{enumerate}
\end{frame}
