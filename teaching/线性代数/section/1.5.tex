\section{分块矩阵}

\subsection{分块矩阵的定义和运算}

\begin{frame}{分块矩阵的定义}
	\onslide<+->
	有时为了研究矩阵和其部分元素形成的矩阵的联系, 需要使用\emph{分块法}将其进行拆分:
	\onslide<+->
	\begin{definition}[分块矩阵]
		用若干条横线和竖线将矩阵 $\bfA$ 分成许多小矩阵, 每个小矩阵成为 $\bfA$ 的子块, 以子块为元素的矩阵称为\emph{分块矩阵}.
	\end{definition}
	\onslide<+->
	例如
	\[\bfA=\begin{pmatrix}
		\bfO_{m\times n}&\bfE_m\\
		\bfE_n&\bfO_{n\times m}
	\end{pmatrix}\]
	就是一个分块矩阵.
	\onslide<+->
	事实上我们已经不加声明地用过分块矩阵了.
\end{frame}


\begin{frame}{分块矩阵的运算: 加法和数乘}
	\onslide<+->
	若分块矩阵 $\bfA,\bfB$ 同型, 且每个对应分块也同型, 则 $\bfA+\bfB$ 就是对应分块相加形成的分块矩阵:
	\[\begin{pmatrix}
		\bfA_{11}&\cdots&\bfA_{1r}\\
		\vdots&\ddots&\vdots\\
		\bfA_{s1}&\cdots&\bfA_{sr}
	\end{pmatrix}+\begin{pmatrix}
		\bfB_{11}&\cdots&\bfB_{1r}\\
		\vdots&\ddots&\vdots\\
		\bfB_{s1}&\cdots&\bfB_{sr}
	\end{pmatrix}=\begin{pmatrix}
		\bfA_{11}+\bfB_{11}&\cdots&\bfA_{1r}+\bfB_{1r}\\
		\vdots&\ddots&\vdots\\
		\bfA_{s1}+\bfB_{s1}&\cdots&\bfA_{sr}+\bfB_{sr}
	\end{pmatrix}.\]

	\onslide<+->
	数 $\lambda$ 和分块矩阵的数乘, 就是 $\lambda$ 和对应分块数乘形成的分块矩阵:
	\[\lambda\begin{pmatrix}
		\bfA_{11}&\cdots&\bfA_{1r}\\
		\vdots&\ddots&\vdots\\
		\bfA_{s1}&\cdots&\bfA_{sr}
	\end{pmatrix}=\begin{pmatrix}
		\lambda\bfA_{11}&\cdots&\lambda\bfA_{1r}\\
		\vdots&\ddots&\vdots\\
		\lambda\bfA_{s1}&\cdots&\lambda\bfA_{sr}
	\end{pmatrix}.\]
\end{frame}


\begin{frame}{分块矩阵的运算: 乘法}
	\onslide<+->
	设
	\[\bfA=\begin{pmatrix}
		\bfA_{11}&\cdots&\bfA_{1r}\\
		\vdots&\ddots&\vdots\\
		\bfA_{s1}&\cdots&\bfA_{sr}
	\end{pmatrix},\quad
	\bfB=\begin{pmatrix}
		\bfB_{11}&\cdots&\bfB_{1t}\\
		\vdots&\ddots&\vdots\\
		\bfB_{r1}&\cdots&\bfB_{rt},
	\end{pmatrix}\]
	且 $\bfA_{ij}$ 的列数和 $\bfB_{jk}$ 的行数相同, 则
	\[\bfA\bfB=\begin{pmatrix}
		\bfC_{11}&\cdots&\bfC_{1r}\\
		\vdots&\ddots&\vdots\\
		\bfC_{s1}&\cdots&\bfC_{sr}
	\end{pmatrix},\quad\bfC_{ij}=\sum_{k=1}^r \bfA_{ik}\bfB_{kj}.\]
	\onslide<+->
	简单来说就是, 若对应的分块能做相应运算, 则分块矩阵的运算就如同把这些分块视作数一样运算.
\end{frame}


\begin{frame}{分块矩阵的运算: 转置}
	\onslide<+->
	设
	\[\bfA=\begin{pmatrix}
		\bfA_{11}&\cdots&\bfA_{1r}\\
		\vdots&\ddots&\vdots\\
		\bfA_{s1}&\cdots&\bfA_{sr}
	\end{pmatrix},\]
	则
	\[\bfA^\rmT=\begin{pmatrix}
		\bfA_{11}^\rmT&\cdots&\bfA_{s1}^\rmT\\
		\vdots&\ddots&\vdots\\
		\bfA_{1r}^\rmT&\cdots&\bfA_{sr}^\rmT
	\end{pmatrix}.\]
	\onslide<+->
	注意小块的位置需要转置, 每个小块也需要转置.
\end{frame}


\subsection{特殊分块矩阵}
\begin{frame}{分块对角阵}
	\onslide<+->
	若方阵
	\[\bfA=\begin{pmatrix}
		\bfA_1&&\\
		&\ddots&\\
		&&\bfA_m
	\end{pmatrix},\]
	其中 $\bfA_1,\dots,\bfA_m$ 都是方阵,
	\onslide<+->
	称 $\bfA$ 为\emph{分块对角阵}.
	\onslide<+->
	记作 $\bfA=\diag(\bfA_1,\dots,\bfA_m)$.

	\onslide<+->
	分块对角阵具有如下性质:
	\begin{enumerate}
		\item $|\bfA|=|\bfA_1|\cdots|\bfA_m|$;
		\item $\bfA$ 可逆当且仅当 $\bfA_1,\dots,\bfA_m$ 均可逆, 此时
		$\bfA^{-1}=\diag(\bfA_1^{-1},\dots,\bfA_m^{-1})$.
		\item $\bfA^k=\diag(\bfA_1^k,\dots,\bfA_m^k)$.
	\end{enumerate}
\end{frame}


\begin{frame}{例: 分块对角阵}
	\onslide<+->
	\begin{example}
		求 $\bfA=\begin{pmatrix}
			2&1&0&0\\
			1&1&0&0\\
			0&0&2&0\\
			0&0&-1&3
		\end{pmatrix}$ 的逆矩阵.
	\end{example}
	\onslide<+->
	\begin{solution}
		设 $\bfA_1=\begin{pmatrix}
			2&1\\1&1
		\end{pmatrix},\bfA_2=\begin{pmatrix}
			2&0\\-1&3
		\end{pmatrix}$,
		\onslide<+->{%
			则 $\bfA_1^{-1}=\begin{pmatrix}
				1&-1\\-1&2
			\end{pmatrix},\quad\bfA_2^{-1}=\dfrac16\begin{pmatrix}
				3&0\\1&2
			\end{pmatrix}$.
		}

		\onslide<+->{%
			故 $\bfA^{-1}=\diag(\bfA_1^{-1},\bfA_2^{-1})=\begin{pNiceMatrix}
				1&-1&&\\
				-1&2&&\\
				&&1/2&0\\
				&&1/6&1/3
			\end{pNiceMatrix}$.
		}
	\end{solution}
\end{frame}


\begin{frame}{例: 分块三角阵的逆}
\beqskip{2pt}
	\onslide<+->
	\begin{example}
		设 $\bfA,\bfB$ 均可逆, 求 $\begin{pmatrix}
			\bfA&\bfO\\\bfC&\bfB
		\end{pmatrix}$ 的逆矩阵.
	\end{example}
	\onslide<+->
	\begin{solution}
		由 $\begin{vmatrix}
			\bfA&\bfO\\\bfC&\bfB
		\end{vmatrix}=|\bfA|\cdot|\bfB|\neq0$ 可知该方阵可逆.
		\onslide<+->{%
			设 $\begin{pmatrix}
				\bfA&\bfO\\\bfC&\bfB
			\end{pmatrix}\begin{pmatrix}
				\bfA_1&\bfA_2\\\bfA_3&\bfA_4
			\end{pmatrix}=\bfE$.
		}
		
		\onslide<+->{%
			则 $\bfA\bfA_1=\bfE, \bfA\bfA_2=\bfO, \bfC\bfA_2+\bfB\bfA_4=\bfE$.
		}\onslide<+->{%
			于是 $\bfA_1=\bfA^{-1},\bfA_2=\bfO,\bfA_4=\bfB^{-1}$.
		}\onslide<+->{%
			再由 $\bfC\bfA_1+\bfB\bfA_3=\bfO$ 可得
			\[\bfA_3=-\bfB^{-1}\bfC\bfA_1=-\bfB^{-1}\bfC\bfA^{-1}.\]
		}\onslide<+->{%
			故 $\begin{pmatrix}
				\bfA&\bfO\\\bfC&\bfB
			\end{pmatrix}^{-1}=\begin{pmatrix}
				\bfA^{-1}&\bfO\\-\bfB^{-1}\bfC\bfA^{-1}&\bfB^{-1}
			\end{pmatrix}$.
		}
	\end{solution}
	\onslide<+->
	由此可知, (分块)上/下三角阵的逆还是(分块)上/下三角阵.
\endgroup
\end{frame}


\begin{frame}{例: 分块方阵的伴随}
	\onslide<+->
	\begin{exercise}
		设 $\bfA,\bfB$ 为同阶方阵, $\bfC=\begin{pmatrix}
			&\bfA\\\bfB&
		\end{pmatrix}$,
		则 $\bfC^*=$\fillbrace{\visible<+->{D}}.
		\begin{taskschoice}(2)
			() $\begin{pmatrix}&\bfA^*\\\bfB^*&\end{pmatrix}$
			() $\begin{pmatrix}&\bfB^*\\\bfA^*&\end{pmatrix}$
			() $\begin{pmatrix}&|\bfB|\bfA^*\\|\bfA|\bfB^*&\end{pmatrix}$
			() $\begin{pmatrix}&|\bfA|\bfB^*\\|\bfB|\bfA^*&\end{pmatrix}$
		\end{taskschoice}
	\end{exercise}
\end{frame}

