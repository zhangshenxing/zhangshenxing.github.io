\section{向量组}

\subsection{向量组的线性表示}

\begin{frame}{引例题: 齐次线性方程组的解}
	\onslide<+->
	我们知道齐次线性方程组是指 $\bfA\bfx={\bf0}$.
	\onslide<+->
	令
	\[V=\{\bfx\mid \bfA\bfx={\bf0}\}\]
	表示该方程的所有解形成的集合.
	\onslide<+->
	显然 ${\bf0}\in V$.
	\onslide<+->
	若 $\bfu,\bfv\in V$, 则 $\bfA\bfu=\bfA\bfv={\bf0}$.
	\onslide<+->
	于是
	\[\bfA(\bfu+\bfv)=\bfA(\lambda \bfv)={\bf0},\quad \forall\lambda\in\BC.\]
	\onslide<+->
	换言之, $V$ 上的加法和数乘的结果还落在 $V$ 中.
	\onslide<+->
	它是 $\BC^n$ 的一个\emph{子空间}.
	
	\onslide<+->
	如何用有限个向量来表示一个子空间中的所有向量呢?
	\onslide<+->
	我们需要线性组合的概念.
\end{frame}


\begin{frame}{向量组}
	\onslide<+->
	一些具有相同维数的向量放在一起形成\emph{向量组}(可以有重复的):
	\onslide<+->
	例如:
	\begin{itemize}
		\item $\bma_1=(1,1,-1)^\rmT, \bma_2=(2,1,2)^\rmT, \bma_3=(3,2,1)^\rmT$;
		\item $\bma_1^\rmT=(1,1,-1), \bma_2^\rmT=(2,1,2), \bma_3^\rmT=(3,2,1)$;
		\item $m\times n$ 矩阵 $\bfA$ 的 $m$ 行可以看成 $m$ 个行向量, 它们构成一个向量组, 叫做 $\bfA$ 的\emph{行向量组};
		\item 类似地, $\bfA$ 的列向量构成它的\emph{列向量组}.
		\item 对于 $n$ 维向量组
		\[\bfe_1=(1,0,\dots,0)^\rmT,\bfe_2=(0,1,\dots,0)^\rmT,\dots,\bfe_n=(0,0,\dots,1)^\rmT,\]
		\onslide<+->{%
			任意 $n$ 维向量 $\bfx=(x_1,\dots,x_n)^\rmT$ 可以表示为这个向量组中向量的数乘之和:
			\[\bfx=x_1\bfe_1+\cdots+x_n\bfe_n.\]
		}
	\end{itemize}
\end{frame}


\begin{frame}{向量组的线性表示}
	\onslide<+->
	\begin{definition}
		设 $\bma_1,\dots,\bma_m,\bmb$ 为 $n$ 维向量.
		若存在数 $\lambda_1,\dots,\lambda_m$ 使得
		\[\bmb=\lambda_1\bma_1+\cdots+\lambda_m\bma_m,\]
		则称 $\bmb$ 可以被向量组 $\{\bma_1,\dots,\bma_m\}$ \emph{线性表示}, 或称 $\bmb$ 是向量组 $\{\bma_1,\dots,\bma_m\}$ 的\emph{线性组合}.
	\end{definition}
	\onslide<+->
	\begin{example}
		\begin{enumerate}
			\item $n$ 维零向量是任一 $n$ 维向量组的线性组合.
			\item 任意 $n$ 维向量是 $\bfe_1,\dots,\bfe_n$ 的线性组合.
		\end{enumerate}
	\end{example}
\end{frame}


\begin{frame}{线性表示的等价刻画}
	\onslide<+->
	向量 $\bmb$ 能被向量组 $\bma_1,\dots,\bma_m$ 线性表示, 即存在 $\lambda_1,\dots,\lambda_m$ 使得
	\[\bmb=\lambda_1\bma_1+\cdots+\lambda_m\bma_m=(\bma_1,\dots,\bma_m)\begin{pmatrix}
		\lambda_1\\\vdots\\\lambda_m
	\end{pmatrix}.\]
	\vspace{-\baselineskip}
	\onslide<+->
	\begin{theorem}
		向量 $\bmb$ 能被向量组 $\bma_1,\dots,\bma_m$ 线性表示, 当且仅当 $\bfA\bfx=\bmb$ 有解, 其中
		\[\bfA=(\bma_1,\dots,\bma_m)\in M_{n\times m}.\]
	\end{theorem}
	\onslide<+->
	记 $V$ 为向量组 $S=\{\bma_1,\dots,\bma_m\}$ 能线性表示的向量全体.
	\onslide<+->
	容易知道
	\[V=\{\bmb\in\BC^n\mid \text{存在 $\bfx$ 使得}\ \bfA\bfx=\bmb\}\]
	是 $\BC^n$ 的子空间, 称为 $S$ \emph{生成的空间}.
	\onslide<+->
	它是包含 $S$ 中所有向量的最小的线性空间.
\end{frame}


\begin{frame}{向量组的等价}
	\begin{definition}
		\begin{enumerate}
			\item 设有两个向量组 $S=\{\bma_1,\dots,\bma_m\}, T=\{\bmb_1,\dots,\bmb_k\}$.
			若 $\bmb_1,\dots,\bmb_k$ 均可以被 $\{\bma_1,\dots,\bma_m\}$ 线性表示, 则称向量组 $T$ 可以被向量组 $S$ 线性表示.
			\item 若向量组 $S$ 和 $T$ 能相互线性表示, 则称 $S,T$ \emph{向量组等价}.
		\end{enumerate}
	\end{definition}
	\onslide<+->
	设 $S,T$ 分别生成空间 $V,W$.
	\onslide<+->
	$T$ 可以被 $S$ 线性表示$\iff$ $W\subseteq V$;
	\onslide<+->
	$S,T$ 向量组等价$\iff W=V$.

	\onslide<+->
	$T$ 能被 $S$ 线性表示, 当且仅当存在 $\bfx_1,\dots,\bfx_k$ 使得
	\[\bfA\bfx_1=\bmb_1,\quad\dots,\quad\bfA\bfx_k=\bmb_k,\]
	其中 $\bfA=(\bma_1,\dots,\bma_m)$.
	\onslide<+->
	即存在矩阵 $\bfX$ 使得 $\bfA\bfX=\bfB$, 其中 $\bfB=(\bmb_1,\dots,\bmb_k)$.

	\onslide<+->
	反过来, 若 $\bfA\bfX=\bfB$, 则 $\bfB$ 的列向量组可以由 $\bfA$ 的列向量组线性表示.
\end{frame}


\begin{frame}{向量组的等价}
	\onslide<+->
	\begin{theorem}
		\begin{enumerate}
			\item $\bfB$ 的列向量组可以由 $\bfA$ 的列向量组线性表示$\iff$存在矩阵 $\bfX$ 使得 $\bfB=\bfA\bfX$.
			\item $\bfA$ 的列向量组和 $\bfB$ 的列向量组作为向量组等价$\iff$存在矩阵 $\bfX,\bfY$ 使得 $\bfB=\bfA\bfX,\bfA=\bfB\bfY$.
		\end{enumerate}
	\end{theorem}
	\onslide<+->
	\begin{proposition}
		向量组的等价满足如下性质:
		\begin{enumerate}
			\item 自反性: $S\sim S$;
			\item 对称性: $S\sim T\implies T\sim S$;
			\item 传递性: $S\sim T,T\sim R\implies S\sim R$.
		\end{enumerate}
	\end{proposition}
\end{frame}


\subsection{线性相关与线性无关}

\begin{frame}{线性相关与线性无关}
	\onslide<+->
	\begin{definition}
		对于 $n$ 维向量组 $\{\bma_1,\dots,\bma_m\}$,
		若存在一组\alert{不全为零}的数 $\lambda_1,\dots,\lambda_m$ 使得
		\[\lambda_1\bma_1+\dots+\lambda_m\bma_m={\bf0},\]
		则称该向量组\emph{线性相关}.
		否则称该向量组\emph{线性无关}.
	\end{definition}
	\onslide<+->
	向量组 $\{\bma_1,\dots,\bma_m\}$ 线性无关当且仅当
	\[\lambda_1\bma_1+\dots+\lambda_m\bma_m={\bf0}\implies
	\lambda_1=\cdots=\lambda_m=0.\]
	\onslide<+->
	即 $(\bma_1,\cdots,\bma_m)\bfx={\bf0}$ 只有零解.
\end{frame}


\begin{frame}{例题: 线性相关与线性无关}
	\begin{example}
		\begin{enumerate}
			\item $\bma_1=(1,2,3)^\rmT,\bma_2=(2,3,4)^\rmT,\bma_3=(0,0,0)^\rmT$ 线性相关.
			实际上\alert{包含零向量的向量组总是线性相关的}.
			\item $\bma_1=(1,2,3)^\rmT,\bma_2=(2,4,6)^\rmT,\bma_3=(3,0,5)^\rmT$ 线性相关.
			\alert{线性相关的向量组添加更多向量还是线性相关的}.
			\item $\bma_1=(1,2,3)^\rmT,\bma_2=(2,3,4)^\rmT,\bma_3=(3,5,7)^\rmT$ 线性相关.
			它们构成的 $3$ 阶矩阵行列式为零.
			\alert{$n$ 维向量组 $\bma_1,\dots,\bma_n$ 线性无关 $\iff |\bma_1,\cdots,\bma_n|\neq 0$}.
			\item $\bfe_1=(1,0,0)^\rmT,\bfe_2=(0,1,0)^\rmT,\bfe_3=(0,0,1)^\rmT$ 线性无关.
			一般地, \alert{$n$ 维单位向量组 $\bfe_1,\dots,\bfe_n$ 线性无关}.
			\item $\bma$ 线性相关 $\iff \bma\neq{\bf0}$.
			\item $\bma_1,\bma_2$ 线性相关 $\iff \bma_1,\bma_2$ 对应分量成比例.
		\end{enumerate}
	\end{example}
\end{frame}


\begin{frame}{例题: 判断线性无关}
	\onslide<+->
	\begin{example}
		已知向量组 $\{\bma_1,\bma_2,\bma_3\}$ 线性无关, 证明向量组 $\{\bma_1+\bma_2,\bma_2+\bma_3,\bma_3+\bma_1\}$ 线性无关.
	\end{example}
	\onslide<+->
	\begin{proof}
		设 $\bfA=\begin{pmatrix}
			1&1&0\\
			0&1&1\\
			1&0&1
		\end{pmatrix}$,
		\onslide<+->{则 $|\bfA|=2$, $\bfA$ 可逆,
		}\onslide<+->{%
			且
			\[(\bma_1+\bma_2,\bma_2+\bma_3,\bma_3+\bma_1)=(\bma_1,\bma_2,\bma_3)\bfA.\]
		}\onslide<+->{%
			若 $(\bma_1+\bma_2,\bma_2+\bma_3,\bma_3+\bma_1)\bfx={\bf0}$, 则
			\[(\bma_1,\bma_2,\bma_3)\bfA\bfx={\bf0}\implies
			\bfA\bfx={\bf0}\implies\bfx={\bf0}.\qedhere\]
		}\vspace{-\baselineskip}
	\end{proof}
\end{frame}


\begin{frame}{例题: 判断线性无关}
	\onslide<+->
	\begin{exercise}
		已知向量组 $\{\bma_1,\bma_2\}$ 线性无关, 请问向量组 $\{\bma_1-\bma_2,\bma_1+\bma_2,\bma_1\}$ 是否线性无关?
	\end{exercise}
	\onslide<+->
	\begin{answer}
		线性相关, 因为 $(\bma_1-\bma_2)+(\bma_1+\bma_2)-2\bma_1={\bf0}$.
	\end{answer}
	\onslide<+->
	\begin{exercise}
		设 $\bfA=\begin{pmatrix}
			1&2&-2\\
			2&1&2\\
			3&0&4
		\end{pmatrix},\bma=\begin{pmatrix}
			k\\1\\1
		\end{pmatrix}$.
		若 $\bfA\bma$ 和 $\bma$ 线性相关, 则 $k=$\fillblank{\visible<+->{$-1$}}.
	\end{exercise}
\end{frame}



\subsection{线性相关和线性无关的性质}

\begin{frame}{线性相关和线性无关的等价刻画}
	\onslide<+->
	\begin{theorem}
		向量组 $\bma_1,\dots,\bma_m$ 线性相关$\iff$其中至少有一个向量可以由其它向量线性表示.
	\end{theorem}
	\onslide<+->
	\begin{proof*}
		若该向量组线性相关, 则存在不全为零的数 $\lambda_1,\dots,\lambda_m$ 使得
		\[\lambda_1\bma_1+\cdots+\lambda_m\bma_m={\bf0}.\]
		\onslide<+->{%
			设 $\lambda_i\neq 0$, 则
			$\bma_i=-\displaystyle\frac1{\lambda_i}\sum_{\substack{j=1\\j\neq i}}^m \lambda_j \bma_j$
			可由其它向量线性表示.
		}

		\onslide<+->{
			反之, 若 $\bma_i=\displaystyle\sum_{\substack{j=1\\j\neq i}}^m \lambda_j \bma_j$ 
			可由其它向量线性表示.
		}\onslide<+->{%
			则 $\displaystyle -\bma_i+\sum_{\substack{j=1\\j\neq i}}^m \lambda_j \bma_j={\bf0}$, 
			向量组 $\bma_1,\dots,\bma_m$ 线性相关.\qedhere
		}
	\end{proof*}
\end{frame}


\begin{frame}{线性相关和线性无关的等价刻画}
	\onslide<+->
	向量组 $\bma_1,\dots,\bma_m$ 线性无关$\iff$其中任一向量不可以由其它向量线性表示.

	\onslide<+->
	注意, 向量组 $\bma_1,\dots,\bma_m$ 线性相关\alert{$\ \ \not\!\!\!\!\implies$}其中任一向量可以由其它向量线性表示.
	\onslide<+->
	\begin{exercise}
		设 $\bma_1,\dots,\bma_m$ 是 $m$ 个 $n$ 维向量, 则下列结论是否正确的有\fillblank{\visible<8->{1}}个.
		\begin{enumerate}
			\item {若 $\bma_1,\dots,\bma_m$ 线性相关, 则其中任一向量均可由其余向量线性表示}%
			\item {若 $\bma_m$ 不能由 $\bma_1,\dots,\bma_{m-1}$ 线性表示, 则向量组 $\bma_1,\dots,\bma_m$ 线性无关}%
			\item {若 $\bma_1,\dots,\bma_m$ 线性相关, 且存在不全为零的 $\lambda_1,\dots,\lambda_{m-1}$ 使得\\ $\lambda_1\bma_1+\cdots+\lambda_{m-1}\bma_{m-1}={\bf0}$, 则 $\bma_m$ 不能由 $\bma_1,\dots,\bma_{m-1}$ 线性表示}%
			\item {若 $\bma_1,\dots,\bma_m$ 线性相关, 且 $\bma_m$ 不能由 $\bma_1,\dots,\bma_m$ 线性表示, 则 $\bma_1,\dots,\bma_{m-1}$ 线性相关}
		\end{enumerate}
	\end{exercise}
\end{frame}


\begin{frame}{线性相关和线性无关的性质}
\beqskip{3mm}
	\onslide<+->
	\begin{theorem}
		若向量组 $\bma_1,\dots,\bma_m$ 线性无关, 向量组 $\bma_1,\dots,\bma_m,\bmb$ 线性相关, 则 $\bmb$ 可以由 $\bma_1,\dots,\bma_m$ 线性表示, 且表达形式唯一.
	\end{theorem}
	\onslide<+->
	\begin{proof*}
		存在不全为零的数 $\lambda_1,\dots,\lambda_m,k$ 使得
		\[\lambda_1\bma_1+\cdots+\lambda_m\bma_m+k\bmb={\bf0}.\]
		\onslide<+->{%
			若 $k=0$, 则 $\lambda_1,\dots,\lambda_m$ 不全为零且
			$\lambda_1\bma_1+\cdots+\lambda_m\bma_m={\bf0}$.
		}\onslide<+->{%
			这与 $\bma_1,\dots,\bma_m$ 线性无关矛盾. 因此 $k\neq 0$.
		}
		\onslide<+->{%
			于是 $\bmb$ 可由 $\bma_1,\dots,\bma_m$ 线性表示.
		}

		\onslide<+->{
			若 $\bmb$ 有两种线性表达形式, 二式相减得到不全为零的数 $\lambda_1,\dots,\lambda_m,k$ 使得
			\[\lambda_1\bma_1+\cdots+\lambda_m\bma_m+k\bmb={\bf0}.\]
			矛盾.\qedhere
		}
	\end{proof*}
\endgroup
\end{frame}


\begin{frame}{线性相关和线性无关的性质}
	\onslide<+->
	\begin{theorem}
		设向量组 $S=\{\bma_1,\dots,\bma_m\},T=\{\bma_1,\dots,\bma_m,\bma_{m+1},\dots,\bma_s\}$.
		\begin{enumerate}
			\item 若向量组 $S$ 线性相关, 则 $T$ 也线性相关.
			\item 若向量组 $T$ 线性无关, 则 $S$ 也线性无关.
		\end{enumerate}
	\end{theorem}
	\onslide<+->
	即\alert{部分相关$\implies$整体相关, 整体无关$\implies$部分无关}.
	\onslide<+->
	\begin{example}
		$n$ 维向量组 $\bma_1,\dots,\bma_s (3\le s\le n)$ 线性无关$\iff$\fillbrace{\visible<+->{D}}
		\xx{$\bma_1,\dots,\bma_s$ 中存在一个向量不能由其余向量线性表示}%
		{$\bma_1,\dots,\bma_s$ 中任两个向量都线性无关}%
		{$\bma_1,\dots,\bma_s$ 中不含零向量}%
		{$\bma_1,\dots,\bma_s$ 中任一个向量都不能由其余向量线性表示}
	\end{example}
\end{frame}


\begin{frame}{例题: 线性相关和线性无关}
	\onslide<+->
	\begin{exercise}
		若向量组 $\bma,\bmb,\bmg$ 线性无关, $\bma,\bmb,\bmd$ 线性相关, 则\fillbrace{\visible<+->{C}}.
		\xx{$\bma$ 一定能由 $\bmb,\bmg,\bmd$ 线性表示}%
		{$\bmb$ 一定不能由 $\bma,\bmg,\bmd$ 线性表示}%
		{$\bmd$ 一定能由 $\bma,\bmb,\bmg$ 线性表示}%
		{$\bmd$ 一定不能由 $\bma,\bmb,\bmg$ 线性表示}
	\end{exercise}
	\onslide<+->
	\begin{example}
		设向量 $\bmb$ 可由 $\bma_1,\dots,\bma_m$ 线性表示, 但不能由向量组 $S=\{\bma_1,\dots,\bma_{m-1}\}$ 线性表示. 记 $T=\{\bma_1,\dots,\bma_{m-1},\bmb\}$, 则\fillbrace{\visible<+->{B}}.
		\xx{$\bma_m$ 不能由 $S$ 线性表示, 也不能由 $T$ 线性表示}%
		{$\bma_m$ 不能由 $S$ 线性表示, 但能由 $T$ 线性表示}%
		{$\bma_m$ 能由 $S$ 线性表示, 也能由 $T$ 线性表示}%
		{$\bma_m$ 能由 $S$ 线性表示, 但不能由 $T$ 线性表示}%
	\end{example}
\end{frame}


\begin{frame}{例题: 线性相关和线性无关}
	\onslide<+->
	\begin{example}
		设向量组 $\bma_1,\bma_2,\bma_3$ 线性相关, $\bma_2,\bma_3,\bma_4$ 线性无关, 证明
		\begin{enumerate}[<*>]
			\item $\bma_1$ 能由 $\bma_2,\bma_3$ 线性表示;
			\item $\bma_4$ 不能由 $\bma_1,\bma_2,\bma_3$ 线性表示;
		\end{enumerate}
	\end{example}
	\onslide<+->
	\begin{proof}
		\begin{enumerate}
			\item 由 $\bma_2,\bma_3,\bma_4$ 线性无关可知 $\bma_2,\bma_3$ 线性无关, 从而它们不能相互表示.
			\onslide<+->{%
				于是 $\bma_2$ 不能被 $\bma_3,\bma_1$ 线性表示, $\bma_3$ 不能被 $\bma_2,\bma_1$ 线性表示.
			}\onslide<+->{%
				但是 $\bma_1,\bma_2,\bma_3$ 线性相关, 所以 $\bma_1$ 能由 $\bma_2,\bma_3$ 线性表示.
			}
			\item 若 $\bma_4$ 能由 $\bma_1,\bma_2,\bma_3$ 线性表示, 由于 $\bma_1$ 能由 $\bma_2,\bma_3$ 线性表示, 于是 $\bma_4$ 也能由 $\bma_2,\bma_3$ 线性表示.
			\onslide<+->{这与 $\bma_2,\bma_3,\bma_4$ 线性无关矛盾.\qedhere}
		\end{enumerate}
	\end{proof}
\end{frame}


\begin{frame}{线性相关和线性无关的性质}
	\onslide<+->
	\begin{theorem}
		设 $\bma_j=(a_{1j},\dots,a_{nj})^\rmT, 
		\bmb_j=(a_{1j},\dots,a_{nj},a_{n+1,j})^\rmT$.
		\begin{enumerate}
			\item 若向量组 $\bma_1,\dots,\bma_m$ 线性无关, 则 $\bmb_1,\dots,\bmb_m$ 线性无关.
			\item 若向量组 $\bmb_1,\dots,\bmb_m$ 线性相关, 则 $\bma_1,\dots,\bma_m$ 线性相关.
		\end{enumerate}
	\end{theorem}
	\onslide<+->
	\begin{proof}
		设 $\bfA=(\bma_1,\dots,\bma_m),\bfB=(\bmb_1,\dots,\bmb_m)$, 则存在 $m$ 维向量 $\bmg$ 使得 $\bfB=\begin{pmatrix}
			\bfA\\\bmg^\rmT
		\end{pmatrix}$.
		\onslide<+->{%
			若向量组 $\bma_1,\dots,\bma_m$ 线性无关, 则 $\bfA\bfx={\bf0}$ 只有零解.
		}\onslide<+->{%
			而
			\[\bfB\bfx={\bf0}\iff\bfA\bfx={\bf0},\bmg^\rmT\bfx=0,\]
			因此 $\bfx={\bf0}$, $\bfB\bfx={\bf0}$ 只有零解, $\bmb_1,\dots,\bmb_m$ 线性无关.\qedhere
		}
	\end{proof}
	\onslide<+->
	即\alert{高维相关$\implies$低维相关, 低维无关$\implies$高维无关}.
\end{frame}


\begin{frame}{例题: 线性相关和线性无关}
	\onslide<+->
	\begin{example}
		判断下列向量组的线性相关性:
		\begin{enumerate}[<*>]
			\item $(1,2,3,4)^\rmT,(2,3,4,5)^\rmT,(0,0,0,0)^\rmT$; 
			\item $(a,b,1,0,0)^\rmT,(c,d,0,6,0)^\rmT,(a,c,0,5,6)^\rmT$;
			\item $(a,1,0,b,0)^\rmT,(c,0,6,d,0)^\rmT,(a,0,5,c,6)^\rmT$.
		\end{enumerate}
	\end{example}
	\onslide<+->
	\begin{solution}
		相关; 无关; 无关.
	\end{solution}
	\onslide<+->
	\begin{exercise}
		若 $(1,0,0,2)^\rmT,(0,1,5,0)^\rmT,(2,1,t+2,4)^\rmT$ 线性相关, 则 $t=$\fillblank{\visible<+->{3}}.
	\end{exercise}
\end{frame}


\begin{frame}{线性相关和线性无关的性质}
	\onslide<+->
	\begin{theorem}
		设向量组 $S=\{\bma_1,\dots,\bma_s\}$ 可由 $T=\{\bmb_1,\dots,\bmb_t\}$ 线性表示. 
		若 $s>t$, 则 $S$ 线性相关.
	\end{theorem}
	\onslide<+->
	即\alert{多的由少的表示, 多的一定线性相关}.
	\onslide<+->
	\begin{proof}
		设 $\bfA=(\bma_1,\dots,\bma_s),\bfB=(\bmb_1,\dots,\bmb_t)$.
		则存在矩阵 $\bfP$ 使得 $\bfA=\bfB\bfP_{t\times s}$.
		\onslide<+->{%
			将 $\bfP$ 补充 $s-t$ 个零行得到 $\bfQ=\begin{pmatrix}
				\bfP\\\bfO
			\end{pmatrix}$.
		}\onslide<+->{%
			则 $|\bfQ|=0$, 存在非零向量 $\bfx$ 使得 $\bfQ\bfx={\bf0}$.
		}\onslide<+->{%
			从而 $\bfP\bfx={\bf0},\bfA\bfx=\bfB\bfP\bfx={\bf0}$, $S$ 线性相关.\qedhere
		}
	\end{proof}
\end{frame}


\begin{frame}{推论和总结}
	\begin{corollary}
		\begin{enumerate}
			\item 设向量组 $S=\{\bma_1,\dots,\bma_s\}$ 可由 $\bmb_1,\dots,\bmb_t$ 线性表示. 若 $S$ 线性无关, 则 $s\le t$.
			\item $m>n$ 个 $n$ 维向量一定线性相关.
			\item 任意两个\alert{等价的线性无关}向量组所含向量的个数相同.
		\end{enumerate}
	\end{corollary}
	\begin{enumerate}
		\item 向量组线性相关$\iff$其中至少有一个向量可以由其它向量线性表示.
		\item 若 $S$ 线性无关, $S\cup\{\bmb\}$ 线性相关, 则 $\bmb$ 可以由 $S$ 唯一线性表示.
		\item 部分相关$\implies$整体相关, 整体无关$\implies$部分无关.
		\item 高维相关$\implies$低维相关, 低维无关$\implies$高维无关.
		\item 多的由少的表示, 多的一定线性相关.
	\end{enumerate}
\end{frame}



