\section{向量组}


\begin{frame}{向量组}
	\onslide<+->
	我们将一些具有相同维数的向量放在一起称之为\emph{向量组}(可以有重复的):
	\onslide<+->
	例如:
	\begin{itemize}
		\item $\bma_1=(1,1,-1)^\rmT, \bma_2=(2,1,2)^\rmT, \bma_3=(3,2,1)^\rmT$;
		\item $\bma_1^\rmT=(1,1,-1), \bma_2^\rmT=(2,1,2), \bma_3^\rmT=(3,2,1)$;
		\item $m\times n$ 矩阵 $\bfA$ 的 $m$ 行可以看成 $m$ 个行向量, 它们构成一个向量组, 叫做 $\bfA$ 的\emph{行向量组};
		\item 类似地, $\bfA$ 的列向量构成它的\emph{列向量组}.
		\item $\bfe_1=(1,0,\dots,0)^\rmT,\bfe_2=(0,1,\dots,0)^\rmT,\dots,\bfe_n=(0,0,\dots,1)^\rmT$.
	\end{itemize}
\end{frame}


\begin{frame}{基\noexer}
	\onslide<+->
	对于标准向量空间 $V=\BR^n$, 我们可以将任一 $\bfv\in V$ \emph{唯一}地表达为形式
	\[\bfv=\lambda_1\bfe_1+\lambda_2\bfe_2+\cdots+\lambda_n\bfe_n,\]
	\onslide<+->
	其中
	\[\bfe_1=(1,0,\dots,0)^\rmT,\bfe_2=(0,1,\dots,0)^\rmT,\dots,\bfe_n=(0,0,\dots,1)^\rmT.\]
	\onslide<+->
	\begin{definition}
		设 $V$ 是线性空间.
		若 $\bma_1,\dots,\bma_m\in V$ 满足: 对任意 $\bfv\in V$, 存在唯一的一组数 $\lambda_1,\dots,\lambda_m$ 使得
		\[\bfv=\lambda_1\bma_1+\cdots+\lambda_m \bma_m,\]
		则称 $\bma_1,\dots,\bma_m$ 是 $V$ 的一组\emph{基}.
	\end{definition}
	\onslide<+->
	如何判断一组向量是不是基呢?
	\onslide<+->
	这需要线性组合和线性无关的概念.
\end{frame}


\subsection{线性组合}

\begin{frame}{向量组的线性组合}
	\onslide<+->
	\begin{definition}
		设 $\bma_1,\dots,\bma_m,\bmb$ 为 $n$ 维向量.
		若存在一组数 $\lambda_1,\dots,\lambda_m$ 使得
		\[\bmb=\lambda_1\bma_1+\cdots+\lambda_m\bma_m,\]
		则称 $\bmb$ 可以被向量组 $\{\bma_1,\dots,\bma_m\}$ \emph{线性表示}, 或称 $\bmb$ 是向量组 $\{\bma_1,\dots,\bma_m\}$ 的\emph{线性组合}.
	\end{definition}
	\onslide<+->
	\begin{example}
		\begin{enumerate}
			\item $n$ 维零向量是任一 $n$ 维向量组的线性组合.
			\item 任意 $n$ 维向量是 $\bfe_1,\dots,\bfe_n$ 的线性组合.
			\item $v\in V$ 是它的一组基的线性组合.
			\item 空间中两条不共线的向量的线性组合全体就是过二者的平面.
		\end{enumerate}
	\end{example}
\end{frame}


\begin{frame}{线性表示的矩阵刻画}
	\onslide<+->
	向量 $\bmb$ 能被向量组 $\bma_1,\dots,\bma_m$ 线性表示, 
	\onslide<+->
	当且仅当存在 $\lambda_1,\dots,\lambda_m$ 使得
	\[\bmb=\lambda_1\bma_1+\cdots+\lambda_m\bma_m=(\bma_1,\dots,\bma_m)\begin{pmatrix}
		\lambda_1\\\vdots\\\lambda_m
	\end{pmatrix},\]
	\onslide<+->
	即 $\bfA\bfx=\bmb$ 有解, 其中 $\bfA=(\bma_1,\dots,\bma_m)$.
	\onslide<+->
	\begin{theorem}
		向量 $\bmb$ 能被 $\bfA$ 的列向量组线性表示, 当且仅当 $\bfA\bfx=\bmb$ 有解.
	\end{theorem}
\end{frame}


\begin{frame}{线性表示的空间刻画}
	\onslide<+->
	设向量组 $S$ 是 $\bfA$ 的列向量组.
	\onslide<+->
	记 $V$ 为向量组 $S$ 能线性表示的向量全体,
	\onslide<+->
	则
	\[V=\{\bfA\bfx\mid \bfx\in\BR^m\}\subseteq \BR^n\]
	是一个线性空间, 称为 $S$ \emph{生成的空间}.
	\onslide<+->
	它是包含 $S$ 中所有向量的最小的线性空间.

	\onslide<+->
	这样, $\bmb$ 能被 $S$ 线性表示$\iff \bmb\in V$.
\end{frame}


\begin{frame}{向量组等价及其矩阵、空间刻画}
	\onslide<+->
	\begin{definition}
		\begin{enumerate}
			\item 设有两个向量组 $S=\{\bma_1,\dots,\bma_m\}, T=\{\bmb_1,\dots,\bmb_k\}$.
			若 $\bmb_1,\dots,\bmb_k$ 均可以被 $S$ 线性表示, 则称向量组 $T$ 可以被向量组 $S$ 线性表示.
			\item 若向量组 $S$ 和 $T$ 能相互线性表示, 则称 $S,T$ \emph{向量组等价}.
		\end{enumerate}
	\end{definition}
	\onslide<+->
	设 $\bfA=(\bma_1,\dots,\bma_m),\bfB=(\bmb_1,\dots,\bmb_k)$.
	\onslide<+->
	$T$ 能被 $S$ 线性表示, 当且仅当存在 $\bfx_1,\dots,\bfx_k$ 使得
	\[\bfA\bfx_1=\bmb_1,\quad\dots,\quad\bfA\bfx_k=\bmb_k,\]
	\onslide<+->
	即存在矩阵 $\bfX$ 使得 $\bfA\bfX=\bfB$.
	\onslide<+->
	\begin{theorem}
		设 $S,T$ 分别为 $\bfA,\bfB$ 的列向量组, 且分别生成空间 $V,W$.
		\begin{enumerate}
			\item $T$ 可以被 $S$ 线性表示$\iff$ $W\subseteq V$
			$\iff$ $\exists\bfX$ 使得 $\bfA\bfX=\bfB$.
			\item $S,T$ 向量组等价$\iff W=V$ $\iff$ $\exists\bfX,\bfY$ 使得 $\bfB=\bfA\bfX,\bfA=\bfB\bfY$.
		\end{enumerate}
	\end{theorem}
\end{frame}


\begin{frame}{向量组的等价的性质}
	\onslide<+->
	\begin{proposition}
		向量组的等价满足如下性质:
		\begin{enumerate}
			\item 自反性: $S\sim S$;
			\item 对称性: $S\sim T\implies T\sim S$;
			\item 传递性: $S\sim T,T\sim R\implies S\sim R$.
		\end{enumerate}
	\end{proposition}
	
	\onslide<+->
	若矩阵 $\bfA\simc\bfB$ 列等价, 则存在可逆矩阵 $\bfQ$ 使得 $\bfB=\bfA\bfQ$,
	\onslide<+->
	于是二者的列向量组作为向量组等价.
	\onslide<+->
	但是反过来\alert{不成立}.
	\onslide<+->
	这是因为列等价的矩阵一定是同型矩阵, 但等价的向量组并不要求向量数量相同.
	\onslide<+->
	不过, 同型矩阵 \emph{$\bfA,\bfB$ 列向量组等价$\iff \bfA\simc\bfB$}.
	\onslide<+->
	我们稍后证明.
\end{frame}


\subsection{线性相关和线性无关}
\begin{frame}{线性相关与线性无关}
	\onslide<+->
	\begin{definition}
		对于 $n$ 维向量组 $\{\bma_1,\dots,\bma_m\}$,
		若存在一组\alert{不全为零}的数 $\lambda_1,\dots,\lambda_m$ 使得
		\[\lambda_1\bma_1+\dots+\lambda_m\bma_m={\bf0},\]
		则称该向量组\emph{线性相关}.
		否则称该向量组\emph{线性无关}.
	\end{definition}
	\onslide<+->
	\begin{example}
		\begin{enumerate}
			\item $\bma_1=(1,2,3)^\rmT$, $\bma_2=(2,3,4)^\rmT$, $\bma_3=(0,0,0)^\rmT$ 线性相关.
			\alert{包含零向量的向量组总是线性相关的}.
			\item $\bfe_1=(1,0,0)^\rmT,\bfe_2=(0,1,0)^\rmT,\bfe_3=(0,0,1)^\rmT$ 线性无关.
			一般地, \alert{$n$ 维基本向量组 $\bfe_1,\dots,\bfe_n$ 线性无关}.
			\item $\bma$ 线性相关 $\iff \bma={\bf0}$.
			\item $\bma_1,\bma_2$ 线性相关 $\iff \bma_1,\bma_2$ 对应分量成比例(共线).
		\end{enumerate}
	\end{example}
\end{frame}


\begin{frame}{线性无关的等价刻画}
	\onslide<+->
	向量组 $\bma_1,\dots,\bma_m$ 线性无关%
	\onslide<+->%
	当且仅当
	\[\lambda_1\bma_1+\dots+\lambda_m\bma_m={\bf0}\implies
	\lambda_1=\cdots=\lambda_m=0,\]
	\onslide<+->
	即 $\bfA\bfx={\bf0}$ 只有零解, 其中 $\bfA=(\bma_1,\cdots,\bma_m)$.

	\onslide<+->
	若 $\bmb=\bfA\bmx$, 则 $\bfA\bfx=\bmb\iff \bfA(\bfx-\bmx)={\bf0}$.
	\onslide<+->
	因此:
	\begin{theorem}
		\begin{enumerate}
			\item 设 $V$ 是 $\bfA$ 列向量生成的空间, 则以下结论等价:
			\begin{itemize}
				\item $\bfA$ 的列向量组线性无关;
				\item $\bfA\bfx={\bf0}$ 只有零解;
				\item $\exists\bfv\in V$, $\bfA\bfx=\bfv$ 只有唯一解;
				\item $\forall\bfv\in V$, $\bfA\bfx=\bfv$ 只有唯一解.
			\end{itemize}
			\item 向量组 $S$ 是线性空间 $V$ 的一组基 $\iff S$ 线性无关且生成 $V$.
		\end{enumerate}
	\end{theorem}
\end{frame}


\begin{frame}{例: 线性无关和线性相关}
	\onslide<+->
	\begin{exercise}
		设 $\bfA=\begin{pmatrix}
			1&2&-2\\
			2&1&2\\
			3&0&4
		\end{pmatrix},\bma=\begin{pmatrix}
			k\\1\\1
		\end{pmatrix}$.
		若 $\bfA\bma$ 和 $\bma$ 线性相关, 则 $k=$\fillblankframe{$-1$}.
	\end{exercise}
	\onslide<+->
	\begin{exercise}
		已知向量组 $\{\bma_1,\bma_2\}$ 线性无关, 请问向量组 $\{\bma_1-\bma_2,\bma_1+\bma_2,\bma_1\}$ 是否线性无关?
	\end{exercise}
	\onslide<+->
	\begin{answer}
		线性相关, 因为 $(\bma_1-\bma_2)+(\bma_1+\bma_2)-2\bma_1={\bf0}$.
	\end{answer}
\end{frame}


\begin{frame}{例: 判断线性无关}
	\beqskip{6pt}
	\onslide<+->
	\begin{example}
		已知向量组 $\{\bma_1,\bma_2,\bma_3\}$ 线性无关, 证明向量组 $\{\bma_1+\bma_2,\bma_2+\bma_3,\bma_3+\bma_1\}$ 线性无关.
	\end{example}
	\onslide<+->
	\begin{proof}
		我们可以用定义来直接证明.
		\onslide<+->{%
		设
		\[\lambda_1(\bma_1+\bma_2)+\lambda_2(\bma_2+\bma_3)+\lambda_3(\bma_3+\bma_1)={\bf0}.\]
		}\onslide<+->{%
		那么
		\[(\lambda_1+\lambda_3)\bma_1+(\lambda_1+\lambda_2)\bma_2+(\lambda_2+\lambda_3)\bma_3={\bf0}.\]
		}\onslide<+->{%
		由于 $\{\bma_1,\bma_2,\bma_3\}$ 线性无关, 因此
		\[\lambda_1+\lambda_3=\lambda_1+\lambda_2=\lambda_2+\lambda_3=0,\]
		}\onslide<+->{%
		解得 $\lambda_1=\lambda_2=\lambda_3=0$.
		证毕.\qedhere
		}
	\end{proof}
	\endgroup
\end{frame}


\begin{frame}{例: 判断线性无关}
	\onslide<+->
	我们来看另一种证法.
	\onslide<+->
	\begin{proof}[另证]
		我们有
		\[(\bma_1+\bma_2,\bma_2+\bma_3,\bma_3+\bma_1)=(\bma_1,\bma_2,\bma_3)\bfA,\]
		其中 $\bfA=\begin{pmatrix}
			1&0&1\\
			1&1&0\\
			0&1&1
		\end{pmatrix}$.
		\onslide<+->{%
		若 $(\bma_1+\bma_2,\bma_2+\bma_3,\bma_3+\bma_1)\bfx={\bf0}$, 则
			\[(\bma_1,\bma_2,\bma_3)\bfA\bfx={\bf0}\implies
			\bfA\bfx={\bf0}\]
		}\onslide<+->{%
		由于 $|\bfA|=2$, $\bfA$ 可逆, 因此 $\bfx={\bf0}$.\qedhere
		}
	\end{proof}
\end{frame}


\begin{frame}{例: 判断线性无关}
	\onslide<+->
	\begin{proposition}
		\begin{enumerate}
			\item 设 $\bma_1,\dots,\bma_m$ 线性无关, $(\bmb_1,\dots,\bmb_n)=(\bma_1,\dots,\bma_m)\bfC$.
			$\bmb_1,\dots,\bmb_n$ 线性无关$\iff \bfC\bfx={\bf0}$ 只有零解.
			\item 设 $\bma_1,\dots,\bma_m$ 线性无关, $(\bmb_1,\dots,\bmb_m)=(\bma_1,\dots,\bma_m)\bfC$.
			$\bmb_1,\dots,\bmb_m$ 线性无关$\iff |\bfC|\neq0$.
			\item $n$ 维向量组 $\bma_1,\dots,\bma_n$ 线性无关 $\iff |\bma_1,\cdots,\bma_n|\neq 0$.
		\end{enumerate}
	\end{proposition}
\end{frame}


\begin{frame}{例: 判断线性无关}
	\onslide<+->
	\begin{example}
		$\bma_1=(1,2,3)^\rmT$, $\bma_2=(2,3,4)^\rmT$, $\bma_3=(3,5,7)^\rmT$ 线性相关.
		\onslide<+->{%
		因为它们构成的 $3$ 阶矩阵行列式为零.}
	\end{example}
	\onslide<+->
	\begin{exercise}
		\begin{enumerate}
			\item 若向量组 $\bma_1=(1,1,1)^\rmT$, $\bma_2=(1,2,3)^\rmT$, $\bma_3=(1,3,t)^\rmT$ 线性相关, 则 $t=$\fillblankframe{$5$}.
			\item 若任一 $3$ 维向量都可由向量组 $\bma_1=(a,3,2)^\rmT$, $\bma_2=(2,-1,3)^\rmT$, $\bma_3=(3,2,1)^\rmT$ 线性表示, 则 $a\neq$\fillblankframe{$5$}.
		\end{enumerate}
	\end{exercise}
\end{frame}


\subsection{线性相关和线性无关的性质}

\begin{frame}{线性相关和线性无关的等价刻画}
	\onslide<+->
	\begin{theorem}
		向量组 $\bma_1,\dots,\bma_m$ 线性相关$\iff$其中至少有一个向量可以由其它向量线性表示.
	\end{theorem}
	\onslide<+->
	\begin{proof*}
		若该向量组线性相关, 则存在不全为零的数 $\lambda_1,\dots,\lambda_m$ 使得
		\[\lambda_1\bma_1+\cdots+\lambda_m\bma_m={\bf0}.\]
		\onslide<+->{%
			设 $\lambda_i\neq 0$, 则
			$\bma_i=-\displaystyle\frac1{\lambda_i}\sum_{\substack{j=1\\j\neq i}}^m \lambda_j \bma_j$
			可由其它向量线性表示.
		}

		\onslide<+->{
			反之, 若 $\bma_i=\displaystyle\sum_{\substack{j=1\\j\neq i}}^m \lambda_j \bma_j$ 
			可由其它向量线性表示.
		}\onslide<+->{%
			则 $\displaystyle -\bma_i+\sum_{\substack{j=1\\j\neq i}}^m \lambda_j \bma_j={\bf0}$, 
			向量组 $\bma_1,\dots,\bma_m$ 线性相关.\qedhere
		}
	\end{proof*}
\end{frame}


\begin{frame}{线性相关和线性无关的等价刻画}
	\onslide<+->
	向量组 $\bma_1,\dots,\bma_m$ 线性无关$\iff$其中任一向量不可以由其它向量线性表示.

	\onslide<+->
	注意, 向量组 $\bma_1,\dots,\bma_m$ 线性相关\alert{$\ \ \not\!\!\!\!\implies$}其中任一向量可以由其它向量线性表示.
	\onslide<+->
	\begin{exercise}
		设 $\bma_1,\dots,\bma_m$ 是 $m$ 个 $n$ 维向量, 则下列结论正确的有\fillblankframe{\visible<9->{$1$}}个.
		\begin{enumerate}
			\item {若 $\bma_1,\dots,\bma_m$ 线性相关, 则其中任一向量均可由其余向量线性表示}%
			\item {若 $\bma_m$ 不能由 $\bma_1,\dots,\bma_{m-1}$ 线性表示, 则向量组 $\bma_1,\dots,\bma_m$ 线性无关}%
			\item {若 $\bma_1,\dots,\bma_m$ 线性相关, 且存在不全为零的 $\lambda_1,\dots,\lambda_{m-1}$ 使得\\ $\lambda_1\bma_1+\cdots+\lambda_{m-1}\bma_{m-1}={\bf0}$, 则 $\bma_m$ 不能由 $\bma_1,\dots,\bma_{m-1}$ 线性表示}%
			\item {若 $\bma_1,\dots,\bma_m$ 线性相关, 且 $\bma_m$ 不能由 $\bma_1,\dots,\bma_m$ 线性表示, 则 $\bma_1,\dots,\bma_{m-1}$ 线性相关}
		\end{enumerate}
	\end{exercise}
\end{frame}


\begin{frame}{线性相关和线性无关的性质}
	\onslide<+->
	\begin{theorem}
		若向量组 $\bma_1,\dots,\bma_m$ 线性无关, 向量组 $\bma_1,\dots,\bma_m,\bmb$ 线性相关, 则 $\bmb$ 可以由 $\bma_1,\dots,\bma_m$ 线性表示, 且表达形式唯一.
	\end{theorem}
	\onslide<+->
	\begin{proof*}
		由于向量组 $\bma_1,\dots,\bma_m,\bmb$ 线性相关, 因此存在不全为零的数 $\lambda_1,\dots,\lambda_m,k$ 使得
		\[\lambda_1\bma_1+\cdots+\lambda_m\bma_m+k\bmb={\bf0}.\]
		\onslide<+->{%
		若 $k=0$, 则由 $\bma_1,\dots,\bma_m$ 线性无关可知 $\lambda_1=\cdots=\lambda_k=0$.
		}\onslide<+->{%
		这与 $\lambda_1,\dots,\lambda_m,k$ 不全为零矛盾.
		}\onslide<+->{%
		因此 $k\neq0$,
		\[\beta=-\frac1k(\lambda_1\bma_1+\cdots+\lambda_m\bma_m).\]
		}\onslide<+->{%
		由 $\bma_1,\dots,\bma_m$ 线性无关可知线性组合表达方式唯一.\qedhere
		}
	\end{proof*}
\end{frame}


\begin{frame}{线性相关和线性无关的性质}
	\onslide<+->
	\begin{theorem}
		设向量组 $S=\{\bma_1,\dots,\bma_m\},T=\{\bma_1,\dots,\bma_m,\bma_{m+1},\dots,\bma_s\}$.
		\begin{enumerate}
			\item 若向量组 $S$ 线性相关, 则 $T$ 也线性相关.
			\item 若向量组 $T$ 线性无关, 则 $S$ 也线性无关.
		\end{enumerate}
	\end{theorem}
	\onslide<+->
	即\alert{部分相关$\implies$整体相关, 整体无关$\implies$部分无关}.
	\onslide<+->
	\begin{example}
		$n$ 维向量组 $\bma_1,\dots,\bma_s (3\le s\le n)$ 线性无关$\iff$\fillbraceframe{D}.
		\begin{taskschoice}(1)
			() $\bma_1,\dots,\bma_s$ 中存在一个向量不能由其余向量线性表示 \alert{\onslide<.->{必要}}
			() $\bma_1,\dots,\bma_s$ 中任两个向量都线性无关 \alert{\onslide<.->{必要}}
			() $\bma_1,\dots,\bma_s$ 中不含零向量 \alert{\onslide<.->{必要}}
			() $\bma_1,\dots,\bma_s$ 中任一个向量都不能由其余向量线性表示
		\end{taskschoice}
	\end{example}
\end{frame}


\begin{frame}{例: 线性相关和线性无关}
	\onslide<+->
	\begin{exercise}
		若向量组 $\bma,\bmb,\bmg$ 线性无关, $\bma,\bmb,\bmd$ 线性相关, 则\fillbraceframe{C}.
		\begin{taskschoice}(2)
			() $\bma$ 一定能由 $\bmb,\bmg,\bmd$ 线性表示
			() $\bmb$ 一定不能由 $\bma,\bmg,\bmd$ 线性表示
			() $\bmd$ 一定能由 $\bma,\bmb,\bmg$ 线性表示
			() $\bmd$ 一定不能由 $\bma,\bmb,\bmg$ 线性表示
		\end{taskschoice}
	\end{exercise}
	\onslide<+->
	\begin{example}
		设向量 $\bmb$ 可由 $\bma_1,\dots,\bma_m$ 线性表示, 但不能由向量组 $S=\{\bma_1,\dots,\bma_{m-1}\}$ 线性表示. 记 $T=\{\bma_1,\dots,\bma_{m-1},\bmb\}$, 则\fillbraceframe{B}.
		\alert{\onslide<.->{$\beta=\lambda_1\bma_1+\cdots+\lambda_m\bma_m, \lambda_m\neq 0$}}
		\begin{taskschoice}(1)
			() $\bma_m$ 不能由 $S$ 线性表示, 也不能由 $T$ 线性表示
			() $\bma_m$ 不能由 $S$ 线性表示, 但能由 $T$ 线性表示
			() $\bma_m$ 能由 $S$ 线性表示, 也能由 $T$ 线性表示
			() $\bma_m$ 能由 $S$ 线性表示, 但不能由 $T$ 线性表示
		\end{taskschoice}
	\end{example}
\end{frame}


\begin{frame}{例: 线性相关和线性无关}
	\onslide<+->
	\begin{example}
		设向量组 $\bma_1,\bma_2,\bma_3$ 线性相关, $\bma_2,\bma_3,\bma_4$ 线性无关, 证明
		\begin{enumerate}
			\item $\bma_1$ 能由 $\bma_2,\bma_3$ 线性表示;
			\item $\bma_4$ 不能由 $\bma_1,\bma_2,\bma_3$ 线性表示.
		\end{enumerate}
	\end{example}
	\onslide<+->
	\begin{proof}
		\begin{enumerate}
			\item 由 $\bma_2,\bma_3,\bma_4$ 线性无关可知 $\bma_2,\bma_3$ 线性无关.%, 从而它们不能相互表示.
			\onslide<+->{%
			% 	于是 $\bma_2$ 不能被 $\bma_3,\bma_1$ 线性表示, $\bma_3$ 不能被 $\bma_2,\bma_1$ 线性表示.
			% }\onslide<+->{%
				但是 $\bma_1,\bma_2,\bma_3$ 线性相关, 所以 $\bma_1$ 能由 $\bma_2,\bma_3$ 线性表示.
			}
			\item 若 $\bma_4$ 能由 $\bma_1,\bma_2,\bma_3$ 线性表示, 由于 $\bma_1$ 能由 $\bma_2,\bma_3$ 线性表示, 于是 $\bma_4$ 也能由 $\bma_2,\bma_3$ 线性表示.
			\onslide<+->{这与 $\bma_2,\bma_3,\bma_4$ 线性无关矛盾.\qedhere}
		\end{enumerate}
	\end{proof}
\end{frame}


\begin{frame}{线性相关和线性无关的性质}
	\onslide<+->
	\begin{theorem}
		设 $\bma_j=(a_{1j},\dots,a_{nj})^\rmT, 
		\bmb_j=(a_{1j},\dots,a_{nj},a_{n+1,j})^\rmT$.
		\begin{enumerate}
			\item 若向量组 $\bma_1,\dots,\bma_m$ 线性无关, 则 $\bmb_1,\dots,\bmb_m$ 线性无关.
			\item 若向量组 $\bmb_1,\dots,\bmb_m$ 线性相关, 则 $\bma_1,\dots,\bma_m$ 线性相关.
		\end{enumerate}
	\end{theorem}
	\onslide<+->
	\begin{proof}
		设 $\bfA=(\bma_1,\dots,\bma_m),\bfB=(\bmb_1,\dots,\bmb_m)$, 则存在 $m$ 维向量 $\bmg$ 使得 $\bfB=\begin{pmatrix}
			\bfA\\\bmg^\rmT
		\end{pmatrix}$.
		\onslide<+->{%
			若向量组 $\bma_1,\dots,\bma_m$ 线性无关, 则 $\bfA\bfx={\bf0}$ 只有零解.
		}\onslide<+->{%
			而
			\[\bfB\bfx={\bf0}\iff\bfA\bfx={\bf0},\bmg^\rmT\bfx=0,\]
			因此 $\bfx={\bf0}$, $\bfB\bfx={\bf0}$ 只有零解, $\bmb_1,\dots,\bmb_m$ 线性无关.\qedhere
		}
	\end{proof}
	\onslide<+->
	即\alert{高维相关$\implies$低维相关, 低维无关$\implies$高维无关}.
\end{frame}


\begin{frame}{例: 线性相关和线性无关}
	\onslide<+->
	\begin{exercise}
		\begin{enumerate}
			\item 判断下列向量组的线性相关性:
			\begin{tasks}[label={(\roman*)},label-format=\upshape\textcolor{main}](1)
				\task $(1,2,3,4)^\rmT,(2,3,4,5)^\rmT,(0,0,0,0)^\rmT$. 
				\alert{\onslide<+->{相关}} 
				\task $(a,b,1,0,0)^\rmT,(c,d,0,6,0)^\rmT,(a,c,0,5,6)^\rmT$. \alert{\onslide<+->{无关}} 
				\task $(a,1,0,b,0)^\rmT,(c,0,6,d,0)^\rmT,(a,0,5,c,6)^\rmT$. \alert{\onslide<+->{无关}} 
			\end{tasks}
			\item 若 $(1,0,0,2)^\rmT,(0,1,5,0)^\rmT,(2,1,t+2,4)^\rmT$ 线性相关, 则 $t=$\fillblankframe{$3$}.
		\end{enumerate}
	\end{exercise}
\end{frame}


\begin{frame}{向量组大小与线性无关的关系}
	% \onslide<+->
	% 为了研究基向量组大小, 我们需要如下结论.
	\onslide<+->
	\begin{theorem}
		设向量组 $S=\{\bma_1,\dots,\bma_s\}$ 可由 $T=\{\bmb_1,\dots,\bmb_t\}$ 线性表示.
		\begin{tasks}(2)
			\task 若 $s>t$, 则 $S$ 线性相关.
			\task 若 $S$ 线性无关, 则 $s\le t$.
		\end{tasks}
	\end{theorem}
	\onslide<+->
	即\alert{多的由少的表示, 多的一定线性相关}.
	\onslide<+->
	\begin{proof}
		设 $\bfA=(\bma_1,\dots,\bma_s),\bfB=(\bmb_1,\dots,\bmb_t)$.
		则存在矩阵 $\bfP$ 使得 $\bfA=\bfB\bfP$.
		\onslide<+->{%
		由于 $\bfP$ 行数小于列数, 因此 $\bfP\bfx={\bf0}$ 有非零解 $\bfx$.
		}\onslide<+->{%
		从而 $\bfA\bfx=\bfB\bfP\bfx={\bf0}$, $S$ 线性相关.\qedhere
		}
	\end{proof}
	\onslide<+->
	\begin{corollary}
		\begin{enumerate}
			\item $m>n$ 个 $n$ 维向量一定线性相关.
			\item 任意两个\alert{等价的线性无关}向量组所含向量的个数相同.
		\end{enumerate}
	\end{corollary}
\end{frame}


\begin{frame}{推论和总结}
	\begin{enumerate}
		\item 向量组线性相关$\iff$其中至少有一个向量可以由其它向量线性表示.
		\item 若 $S$ 线性无关, $S\cup\{\bmb\}$ 线性相关, 则 $\bmb$ 可以由 $S$ 唯一线性表示.
		\item 部分相关$\implies$整体相关, 整体无关$\implies$部分无关.
		\item 高维相关$\implies$低维相关, 低维无关$\implies$高维无关.
		\item 多的由少的表示, 多的一定线性相关.
	\end{enumerate}
\end{frame}



\subsection{维数和秩}

\begin{frame}{维数和秩}
	\onslide<+->
	\begin{definition}
		\begin{enumerate}
			\item 若 $\bma_1,\dots,\bma_m$ 是线性空间 $V$ 的一组基, 则称 $m$ 为 $V$ 的\emph{维数}, 记作 $\dim V$.
			\item 设向量组 $S$ 生成空间 $V$.
			称 $V$ 的维数为该向量组的\emph{秩}(Rank), 记作 $R(S)$.
		\end{enumerate}
	\end{definition}
	\onslide<+->
	由前面的结论可知, $V$ 不同的基向量组的个数总是相同的, 即维数是唯一的.

	\onslide<+->
	设 $S=\{\bma_1,\dots,\bma_m\}$ 生成 $V$,
	\onslide<+->
	$T$ 是 $V$ 的一组基.
	\onslide<+->
	由于向量组等价$\iff$生成同一个空间, 因此 $S,T$ 是等价向量组.
	\onslide<+->
	由 $T$ 线性无关可知 $R(S)\le m$.
	\onslide<+->
	\begin{theorem}
		设 $\bfA=(\bma_1,\dots,\bma_m)$ 的列向量组为 $S$.
		\begin{enumerate}
			\item $S$ 线性无关$\iff R(S)=m\iff\bfA\bfx={\bf0}$ 只有零解.
			\item $S$ 线性相关$\iff R(S)<m\iff\bfA\bfx={\bf0}$ 有非零解.
		\end{enumerate}
	\end{theorem}
\end{frame}


\begin{frame}{例: 方阵的列向量组}
	\onslide<+->
	\begin{corollary}
		设方阵 $\bfA=(\bma_1,\dots,\bma_n)$ 的列向量组为 $S$.
		\begin{enumerate}
			\item $S$ 线性无关$\iff R(S)=m\iff\bfA\bfx={\bf0}$ 只有零解 $\iff |\bfA|\neq0$.
			\item $S$ 线性相关$\iff R(S)<m\iff\bfA\bfx={\bf0}$ 有非零解 $\iff |\bfA|=0$.
		\end{enumerate}
	\end{corollary}
	\onslide<+->
	\begin{exercise}
		设 $\bfA$ 是 $n$ 阶方阵, 且其行列式 $|\bfA|=0$. 下列说法正确的是\fillbraceframe{C}.
		\begin{taskschoice}(1)
			() $\bfA$ 必有一列元素全为零
			() $\bfA$ 必有两列元素对应成比例
			() $\bfA$ 必有一个列向量可由其余列向量线性表示
			() $\bfA$ 中任意列向量均可由其余列向量线性表示
		\end{taskschoice}
	\end{exercise}
\end{frame}


\begin{frame}{维数和秩}
	\onslide<+->
	\begin{theorem}
		\begin{enumerate}
			\item 设向量组 $S$ 可由向量组 $T$ 线性表示, 则 $R(S)\le R(T)$.
			\item 若线性空间 $V\subseteq W$, 则 $\dim V\le\dim W$.
			\item 设向量组 $S$ 可由向量组 $T$ 线性表示, 且 $R(S)=R(T)$, 则 $S,T$ 向量组等价.
			\item 若线性空间 $V\subseteq W$ 且 $\dim V=\dim W$, 则 $V=W$.
		\end{enumerate}
	\end{theorem}
	\onslide<+->
	我们来证明\enumnum4.
	\onslide<+->
	设 $S,T$ 是 $V,W$ 的一组基.
	\onslide<+->
	那么 $S,T$ 大小相同, 且 $S$ 可由 $T$ 线性表示.
	\onslide<+->
	设 $S,T$ 分别是 $\bfA,\bfB$ 的列向量组,
	\onslide<+->
	那么存在方阵 $\bfP$ 使得 $\bfA=\bfB\bfP$.
	\onslide<+->
	若 $\bfP$ 不可逆, 存在非零向量 $\bfx$ 使得 $\bfP\bfx={\bf0}$.
	\onslide<+->
	于是 $\bfA\bfx=\bfB\bfP\bfx={\bf0}$, $S$ 线性相关, 矛盾!
	\begin{theorem}
		设 $V$ 是 $n$ 维空间, $S$ 是由其中向量构成向量组.
		那么 $S$ 是一组基当且仅当如下任意两条满足(剩下一条自动成立):
		\begin{tasks}(3)
			\task $S$ 大小是 $n$;
			\task $S$ 生成 $V$;
			\task $S$ 线性无关.
		\end{tasks}
	\end{theorem}
\end{frame}


\begin{frame}{例: 向量组的秩}
	\onslide<+->
	\begin{example}
		若
		\[S_1=\{\bma_1,\bma_2,\bma_3\},\quad S_2=\{\bma_1,\bma_2,\bma_3,\bma_4\},\quad S_3=\{\bma_1,\bma_2,\bma_3,\bma_4,\bma_5\}\]
		满足 $R(S_1)=R(S_2)=3, R(S_3)=4$.
		证明向量组 $S=\{\bma_1,\bma_2,\bma_3,\bma_5-\bma_4\}$ 线性无关.
	\end{example}
	\onslide<+->
	\begin{proof}
		由 $R(S_1)=3$ 可知 $S_1$ 线性无关.
		\onslide<+->{%
			由 $R(S_2)=3$ 可知 $S_2$ 线性相关.
		}\onslide<+->{%
			从而 $\bma_4$ 可由 $S_1$ 线性表示.
		}\onslide<+->{%
			于是 $S_3$ 可由 $S$ 线性表示.
		}\onslide<+->{%
			显然 $S$ 可由 $S_3$ 线性表示, 因此二者等价, $R(S)=R(S_3)=4$.\qedhere
		}
	\end{proof}
	\onslide<+->
	\begin{exercise}
		判断题: 设 $S$ 和 $T$ 为两个 $n$ 维向量组, 且 $R(S)=R(T)$, 则 $S$ 和 $T$ 等价.\onslide<+->{\alert{\text{\huge$\times$}}}
	\end{exercise}
\end{frame}


\subsection{极大线性无关组}
\begin{frame}{极大线性无关组的定义}
	\onslide<+->
	如何从一组能生成空间 $V$ 的向量找到 $V$ 的一组基?
	\onslide<+->
	我们只需要取极大线性无关组.
	\onslide<+->
	\begin{definition}
		设 $S$ 为一个向量组.
		若 $S$ 的部分组 $S_0=\{\bma_1,\dots,\bma_m\}$ 满足
		\begin{enumerate}[<*>]
			\item $S_0$ 线性无关;
			\item $S_0$ 添加 $S$ 中的若干向量得到的向量组均线性相关.
		\end{enumerate}
		则称 $S_0$ 是 $S$ 的一个\emph{极大线性无关组}.
	\end{definition}
	\onslide<+->
	根据上一节相关结论可知, $S$ 中所有向量均可由 $S_0$ 线性表示.
	\onslide<+->
	换言之, $S_0$ 和 $S$ 等价, 它们生成相同的子空间 $V$, $m=R(S)$, $S_0$ 是 $V$ 的一组基.
\end{frame}


\begin{frame}{极大线性无关组的定义}
	\onslide<+->
	\begin{theorem}
		$S_0$ 是 $S$ 的极大线性无关组当且仅当
		\begin{enumerate}
			\item $S_0$ 线性无关;
			\item $S$ 中任意 $m+1$ 个向量线性相关.
		\end{enumerate}
	\end{theorem}
	\onslide<+->
	\begin{proof*}
		若 $S_0$ 是 $S$ 的极大线性无关组, 则 $S$ 中任意 $m+1$ 个向量可由 $S_0$ 线性表示.
		\onslide<+->{%
			从而线性相关.
		}

		\onslide<+->{%
			反之, 若 $S$ 中任意 $m+1$ 个向量线性相关, 则 $S$ 中任意 $s>m$ 个向量线性相关.
		}\onslide<+->{%
			于是 $S_0$ 添加 $S$ 中的若干向量得到的向量组均线性相关.\qedhere
		}
	\end{proof*}
\end{frame}


\begin{frame}{极大线性无关组的性质}
	\begin{enumerate}
		\item 若 $R(S)=r$, 则 $S$ 中任意 $r$ 个线性无关的向量构成 $S$ 的一个极大线性无关组.
		\item 只含有零向量的向量组没有极大线性无关组(空集), 它的秩为 $0$ (空集生成 $0$ 维空间 $\{{\bf0}\}$).
		\item 极大线性无关组一般不是唯一的.
	\onslide<+->{%
		\mdseries 例如
		\[\bma_1=(1,0)^\rmT,\quad\bma_2=(0,1)^\rmT,\quad\bma_3=(1,1)^\rmT.\]}
		\onslide<+->{%
		$\bma_1,\bma_2$ 是一个极大线性无关组, $\bma_1,\bma_3$ 也是一个极大线性无关组.}
		\item 向量组和它的一个极大线性无关组是等价的, 于是同一向量组的任意两个极大线性无关组等价.
	\end{enumerate}
\end{frame}


\begin{frame}{等价的向量组的极大线性无关组}
	\onslide<+->
	设 $\bfA=(\bma_1,\dots,\bma_r),\bfB=(\bmb_1,\dots,\bmb_r)$ 的列向量组是等价的线性无关组.
	\onslide<+->
	我们之前证明过若 $\bfB=\bfA\bfP$, 则 $\bfP$ 可逆.

	\onslide<+->
	一般地, 若 $\bfA=(\bma_1,\dots,\bma_m),\bfB=(\bmb_1,\dots,\bmb_m)$ 的列向量组是等价的向量组, 秩为 $r$.
	\onslide<+->
	通过适当的列变换, 可以让 $\bfA$ 的前 $r$ 列是极大无关组, 后面全是零向量.
	\onslide<+->
	设 $\bfA=(\bfA'_r,\bfO)$.
	\onslide<+->
	对 $\bfB$ 作类似操作, 则 $\bfA'\simc \bfB'$, 即存在可逆矩阵 $\bfP\in M_r$ 使得 $\bfB'=\bfA'\bfP$.
	\onslide<+->
	于是 
	\[\bfB=\bfA\begin{pmatrix}
		\bfP&\\&\bfE
	\end{pmatrix}\simc \bfA.\]
	\onslide<+->
	因此\alert{同型矩阵列(行)向量组等价$\iff$列(行)等价}.
\end{frame}


\begin{frame}{例: 极大线性无关组}
	\onslide<+->
	\begin{example}
		矩阵 $\bfA=\begin{pmatrix}
			1&1&3&1\\
			0&1&-1&4\\
			0&0&0&5\\
			0&0&0&0
		\end{pmatrix}$
		的行向量组为
		\vspace{-.5\baselineskip}
		\[\bma_1^\rmT=(1,1,3,1),\ 
			\bma_2^\rmT=(0,1,-1,4),\ 
			\bma_3^\rmT=(0,0,0,5),\ 
			\bma_4^\rmT=(0,0,0,0).
		\]
		\onslide<+->{%
			由于 $\bma_1^\rmT$, $\bma_2^\rmT$, $\bma_3^\rmT$ 的第 $1,2,4$ 个分量形成可逆矩阵 $\begin{pmatrix}
				1&1&1\\
				0&1&4\\
				0&0&5
			\end{pmatrix}$, 因此它们线性无关.
		}\onslide<+->{%
			它们构成一个极大线性无关组, $\bfA$ 的行向量组的秩是 $3$.
		}\onslide<+->{%
			类似可知, $\bfA$ 的列向量组的秩也是 $3$.
		}
	\end{example}
	\onslide<+->
	实际上, 任意矩阵的行向量组的秩等于列向量组的秩.
	\onslide<+->
	为了说明这一点, 我们考虑矩阵的秩.
\end{frame}
