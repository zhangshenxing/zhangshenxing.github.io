\section{克拉默法则}

\subsection{拉普拉斯展开的应用}

\begin{frame}{拉普拉斯展开}
	\onslide<+->
	回顾下拉普拉斯展开:
	\begin{theorem}[行列式沿任一行(列)展开, 拉普拉斯展开]
		方阵的行列式等于任一行(列)的元素与其对应的代数余子式乘积的和:
		\begin{align*}
			|\bfA|&=a_{i1}A_{i1}+a_{i2}A_{i2}+\cdots+a_{in}A_{in}\\
			&=a_{1j}A_{1j}+a_{2j}A_{2j}+\cdots+a_{nj}A_{nj}.
		\end{align*}
	\end{theorem}
	\onslide<+->
	其中代数余子式 $A_{ij}$ 是指去掉方阵 $\bfA$ 的 $i$ 行 $j$ 列得到的方阵的行列式的 $(-1)^{i+j}$ 倍.
\end{frame}



\begin{frame}{例: 拉普拉斯展开的应用}
	\onslide<+->
	\begin{example}
		设 $\bfA=\begin{pmatrix}
			3&0&4&0\\
			2&2&2&2\\
			0&-7&0&0\\
			5&3&-2&2
		\end{pmatrix}$.
		计算 $A_{41}+A_{42}+A_{43}+A_{44}$ 和 $M_{41}+M_{42}+M_{43}+M_{44}$.
	\end{example}
	\onslide<+->
	\begin{solution}
		由拉普拉斯展开可知
		\[A_{41}+A_{42}+A_{43}+A_{44}
		=\detm{
			3&0&4&0\\
			2&2&2&2\\
			0&-7&0&0\\
			1&1&1&1
		}=0.\]
	\end{solution}
\end{frame}


\begin{frame}{例: 拉普拉斯展开的应用}
	\onslide<+->
		\begin{solutionc}
			\begin{align*}
				M_{41}+M_{42}+M_{43}+M_{44}
				&=-A_{41}+A_{42}-A_{43}+A_{44}
				=\detm{
					3&0&4&0\\
					2&2&2&2\\
					0&-7&0&0\\
					-1&1&-1&1
				}\\
				&\onslide<+->{
					=7\detm{
						3&4&0\\
						2&2&2\\
						-1&-1&1
				}=-28.}
		\end{align*}
	\end{solutionc}
\end{frame}


\subsection{克拉默法则}

\begin{frame}{克拉默法则}
	\onslide<+->
	\begin{theorem}[克拉默法则]
		设线性方程组
		\[\laeq{
			a_{11}x_1+a_{12}x_2+\cdots+a_{1n}x_n=b_1\\
			a_{21}x_1+a_{22}x_2+\cdots+a_{2n}x_n=b_2\\
			\vvdots\\
			a_{n1}x_1+a_{n2}x_2+\cdots+a_{nn}x_n=b_n
		}\]
		的系数矩阵为 $\bfA$,
		将 $\bfA$ 的第 $j$ 列换成 $b_1,\dots,b_n$ 得到的方阵为 $\bfA_j$.
		那么当 $|\bfA|\neq 0$, 该线性方程组有唯一解
		\[x_1=\frac{|\bfA_1|}{|\bfA|},\quad
			x_2=\frac{|\bfA_2|}{|\bfA|},\quad\dots,\quad
			x_n=\frac{|\bfA_n|}{|\bfA|}.\]
	\end{theorem}
\end{frame}


\begin{frame}{克拉默法则的证明}
	\onslide<+->
	\begin{proofs}
		回顾
		\[a_{i1}A_{k1}+a_{i2}A_{k2}+\cdots+a_{in}A_{kn}=\begin{cases}
			0,&i\neq k;\\
			|\bfA|,&i=k.
		\end{cases}\]
		\onslide<+->{%
		方阵 $\bfA_j$ 沿着第 $j$ 列展开得到
		\[|\bfA_j|=b_1A_{1j}+b_2A_{2j}+\cdots+b_nA_{nj}.\]
		}\onslide<+->{%
		因此
		\[\sum_{j=1}^n a_{ij} |\bfA_j|
			=\sum_{j=1}^n a_{ij} \sum_{k=1}^n b_kA_{kj}
			=\sum_{k=1}^n b_k \sum_{j=1}^n a_{ij} A_{kj}
			=b_i |\bfA|.\]
		}\onslide<+->{%
		所以 $x_i=\dfrac{|\bfA_i|}{|\bfA|}$ 是题述方程的解.}
	\end{proofs}
\end{frame}


\begin{frame}{克拉默法则的证明}
	\onslide<+->
	\begin{proofe}
		再证唯一性.
		\onslide<+->{
		如果方程有解 $x_1,\dots,x_n$, 那么}
			\[\visible<.->{x_i|\bfA|=\detm{
				a_{11}&\cdots&x_ia_{1i}&\cdots&a_{n1}\\
				\vdots&\vdots&\vdots&\vdots&\vdots\\
				a_{n1}&\cdots&x_ia_{ni}&\cdots&a_{nn}}}
				\visible<+->{\xeq[j\neq i]{c_i+x_jc_j}\detm{
					a_{11}&\cdots&b_1&\cdots&a_{n1}\\
					\vdots&\vdots&\vdots&\vdots&\vdots\\
					a_{n1}&\cdots&b_n&\cdots&a_{nn}}}
			\visible<+->{=|\bfA_i|}.\]
		\onslide<+->{%
		因此 $x_i=\dfrac{|\bfA_i|}{|\bfA|}$.\qedhere
		}
	\end{proofe}
	\onslide<+->
	后面我们将会知道, \alert{$|\bfA|\neq0$ 是方程有唯一解的充分必要条件}.
	\onslide<+->
	因此我们定义的行列式确实起到了线性方程组的``判别式''的作用.
\end{frame}


\begin{frame}{克拉默法则的应用}
	\onslide<+->
	\begin{example}
		已知 $\laeq{
			\lambda x_1+x_2+x_3=1\\
			x_1+\lambda x_2+x_3=1\\
			x_1+x_2+\lambda x_3=-2}$
		有无穷多解, 求 $\lambda$.
	\end{example}
	\onslide<+->
	\begin{solution}
		\[0=\detm{\lambda&1&1\\1&\lambda&1\\1&1&\lambda}
		=\lambda^3+2-3\lambda=(\lambda-1)^2(\lambda+2).\]
		\onslide<+->{%
			因此 $\lambda=1$ 或 $-2$.
		}\onslide<+->{%
			显然 $\lambda=1$ 时无解.
		}\onslide<+->{%
			$\lambda=-2$ 时, $x_1=t,x_2=-t,x_3=1$ 是方程的解.
		}\onslide<+->{%
			因此 $\lambda=-2$.
		}
	\end{solution}
\end{frame}


\begin{frame}{齐次线性方程组}
	\onslide<+->
	如果线性方程组的常数都是零, 即
	\[\laeq{
		a_{11}x_1+a_{12}x_2+\cdots+a_{1n}x_n=0\\
		a_{21}x_1+a_{22}x_2+\cdots+a_{2n}x_n=0\\
		\vvdots\\
		a_{n1}x_1+a_{n2}x_2+\cdots+a_{nn}x_n=0
	}\]
	称之为\emph{齐次线性方程组}.
	\onslide<+->
	否则称之为\emph{非齐次线性方程组}.

	\onslide<+->
	显然 $x_1=\cdots=x_n=0$ 是齐次线性方程组的解, 称为\emph{零解}. 其它解被称为\emph{非零解}.
	\onslide<+->
	所以 \alert{$|\bfA|=0\iff$ 齐次线性方程组有无穷多(非零)解}.
	
	\onslide<+->
	对于非齐次线性方程组, 如果 $(a_1,\dots,a_n)$ 是一组解, 而 $(b_1,\dots,b_n)$ 是对应的齐次线性方程组的解, 那么
	$(a_1+b_1,\dots,a_n+b_n)$ 也是非齐次线性方程组的解.
	\onslide<+->
	所以 \alert{$|\bfA|=0\iff$ 非齐次线性方程组无解或有无穷多解}.
\end{frame}


\begin{frame}{克拉默法则的应用}
	\onslide<+->
	\begin{exercise}
		如果 $\laeq{
			\lambda x_1+x_2+x_3=0\\
			x_1+\lambda x_2+x_3=0\\
			x_1+x_2+\lambda x_3=0}$
		有非零解, 则 $\lambda=$\fillblank{\visible<+->{$2,-1$}}.
	\end{exercise}
	\onslide<+->
	\begin{example}
		证明: 如果三条不同的直线
		\begin{align*}
			ax+by+c=0\\
			bx+cy+a=0\\
			cx+ay+b=0
		\end{align*}
		相交于一点, 则 $a+b+c=0$.
	\end{example}
\end{frame}


\begin{frame}{克拉默法则的应用}
	\onslide<+->
	\begin{proof}
		线性方程组 $\laeq{
			ax_1+bx_2+cx_3=0\\
			bx_1+cx_2+ax_3=0\\
			cx_1+ax_2+bx_3=0}$
			有非零解 $(x,y,1)$.
		\onslide<+->{%
			因此 $\detm{a&b&c\\b&c&a\\c&a&b}=0$.
		}\onslide<+->{%
		\begin{align*}
			\detm{a&b&c\\b&c&a\\c&a&b}
			&=(a+b+c)\detm{1&1&1\\b&c&a\\c&a&b}\\
			&\visible<+->{=(a+b+c)(bc+ac+ab-a^2-b^2-c^2)}\\
			&\visible<+->{=-\frac12(a+b+c)[(a-b)^2+(b-c)^2+(c-a)^2].}
		\end{align*}}
		\onslide<+->{由于这是三条不同直线, 因此 $a,b,c$ 不可能全部相等, 从而 $a+b+c=0$.\qedhere}
	\end{proof}
	\onslide<+->
	想一想: 为什么 $a+b+c=0$ 时, 三条直线一定相交于一点?
\end{frame}


\begin{frame}{小结}
	\onslide<+->
	使用克拉默法则解线性方程组需要: \enumnum1 方程组的数量和未知数数量相同; \enumnum2 系数矩阵行列式非零.
	\onslide<+->
	但由于使用克拉默法则计算量较大, 一般不使用该方法解方程, 仅用于理论研究.
	
	\onslide<+->
	\begin{exercise}
		设 $a_1+\cdots+a_n\neq 0$.
		何时线性方程组 $\laeq{
			(a_1+b)x_1+a_2x_2+\cdots+a_n x_n=0\\
			a_1x_1+(a_2+b)x_2+\cdots+a_n x_n=0\\
			\vvdots\\
			a_1x_1+a_2x_2+\cdots+(a_n+b) x_n=0}$
		有非零解? \visible<+->{\alert{$b=0$.}}
	\end{exercise}
\end{frame}


\begin{frame}{克拉默法则的应用}
	\onslide<+->
	\begin{exercise}
		设 $a_i$ 两两不同.
		解方程组 $\laeq{
			x_1+a_1x_2+\cdots+a_1^{n-1} x_n=1\\
			x_1+a_2x_2+\cdots+a_2^{n-1} x_n=1\\
			\vvdots\\
			x_1+a_nx_2+\cdots+a_n^{n-1} x_n=1}$.
	\end{exercise}
	\onslide<+->
	\begin{answer}
		由于系数矩阵行列式为范德蒙行列式
		\[\prod_{1\le i<j\le n}(a_j-a_i)\neq0,\]
		因此方程有唯一解 $x_1=1,x_2=\cdots=x_n=0$.
	\end{answer}
\end{frame}
