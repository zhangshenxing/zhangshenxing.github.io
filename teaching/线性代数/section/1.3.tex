\section{方阵的行列式}

\subsection{行列式的定义}

\begin{frame}{引例: 平行四边形的面积\noexer}
	\onslide<+->
	设平面上有 $\parallelogram OACB$, 其中 $A,B$ 坐标分别为 $\bfu=(a,b)^\rmT, \bfv=(c,d)^\rmT$.
\onslide<+->{%
		如果 $\bfu=(1,0)^\rmT,\bfv=(0,1)^\rmT$, 那么面积为 $1$.
	}\onslide<+->{%
		如果将 $\bfu$ 换成 $k\bfu$, 那么面积变为 $|k|$ 倍.
	}\onslide<+->{%
		如果将 $\bfu$ 拆分为 $(a,0)^\rmT+(0,b)^\rmT$, 那么得到的三个平行四边形的面积有什么联系呢?
	}
	\begin{center}
	\begin{tikzpicture}[scale=.8]
		\begin{scope}
			\draw[cstaxis] (-1,0)--(4,0);
			\draw[cstaxis] (0,-1)--(0,3);
			\draw[cstcurve,main] (0,0)--(3,0.5)--(4,2)--(1,1.5)--cycle;
			\draw (0,0) node[main,below left] {$O$};
			\draw (3,0.5) node[main,right] {$A$};
			\draw (1,1.5) node[main,left]{$B$};
			\draw (4,2) node[main,right] {$C$};
			\fill[cstcurve,second,fill opacity=.5,visible on=<5->] (0,0)--(3,0)--(4,1.5)--(1,1.5)--cycle;
			\begin{scope}[visible on=<5->]
				\draw (3,0) node[second,below] {$A_1$};
				\draw (4,1.5) node[second,right] {$C_1$};
			\end{scope}
			\fill[cstcurve,third,fill opacity=.5,visible on=<6->] (0,0)--(0,0.5)--(1,2)--(1,1.5)--cycle;
			\begin{scope}[visible on=<6->]
				\draw (0,.5) node[third,left] {$A_2$};
				\draw (1,2) node[third,above] {$C_2$};
			\end{scope}
			\fill[cstcurve,fourth,fill opacity=.5,visible on=<8->] (1,1.5)--(4,1.5)--(4,2)--cycle;
			\fill[cstcurve,fourth,fill opacity=.5,visible on=<9->] (0,0)--(3,0)--(3,0.5)--cycle;
			\fill[cstcurve,third,fill opacity=.5,visible on=<10->] (3,0)--(3,0.5)--(4,2)--(4,1.5)--cycle;
		\end{scope}

		\begin{scope}[xshift=7cm,visible on=<11->]
			\draw[cstaxis] (-1,0)--(4,0);
			\draw[cstaxis] (0,-1)--(0,3);
			\draw[cstcurve,main] (0,0)--(3,0.5)--(2,2)--(-1,1.5)--cycle;
			\draw (0,0) node[main,below left] {$O$};
			\draw (3,0.5) node[main,right] {$A$};
			\draw (-1,1.5) node[main,left]{$B$};
			\draw (2,2) node[main,right] {$C$};
			\fill[cstcurve,second,fill opacity=.5,visible on=<12->] (0,0)--(3,0)--(2,1.5)--(-1,1.5)--cycle;
			\fill[cstcurve,third,fill opacity=.5,visible on=<12->] (0,0)--(0,0.5)--(-1,2)--(-1,1.5)--cycle;
			\begin{scope}[visible on=<12->]
				\draw (3,0) node[second,below] {$A_1$};
				\draw (2,1.5) node[second,right] {$C_1$};
				\draw (0,.5) node[third,right] {$A_2$};
				\draw (-1,2) node[third,above] {$C_2$};
			\end{scope}
			\fill[cstcurve,third,fill opacity=.5,visible on=<13->] (3,0)--(3,0.5)--(2,2)--(2,1.5)--cycle;
			\fill[cstcurve,fourth,fill opacity=.5,visible on=<13->] (0,0)--(3,0)--(3,0.5)--cycle;
			\fill[cstcurve,fourth,fill opacity=.5,visible on=<14->] (-1,1.5)--(2,1.5)--(2,2)--cycle;
		\end{scope}
	\end{tikzpicture}
	\end{center}
	\onslide<+->
	令 $A_1(a,0)$, 并作 $\parallelogram OA_1C_1B$;
	\onslide<+->
	令 $A_2(0,b)$, 并作 $\parallelogram OA_2C_2B$.
	\onslide<+->
	那么 $\parallelogram OACB$ 的面积是这两个相加还是相减?
	\onslide<+->\onslide<+->\onslide<+->
	使用割补法可知为\alert{二者相减}.
	
	\onslide<+->
	如果 $A$ 在第一象限, $B$ 在第二象限.
	\onslide<+->
	那么 $\parallelogram OACB$ 的面积是这两个相加还是相减?\onslide<+->\onslide<+->
	使用割补法可知为\alert{二者相加}.
\end{frame}


\begin{frame}{有向面积\noexer}
	\onslide<+->
	为何第一种情形是相减而第二种情形是相加呢?
	\onslide<+->
	观察发现: 这些平行四边形中只有第一种情形的 $\parallelogram OA_2C_2B$, 从 $OA_2$ 到 $OB$ 是顺时针方向.
	\onslide<+->
	如果定义\emph{有向面积}并记为
	\[|\bfu,\bfv|=\begin{vmatrix}
		a&c\\b&d
	\end{vmatrix}=\begin{cases}
		S_{\parallelogram OACB},&\text{从 $OA$ 到 $OB$ 是逆时针};\\
		-S_{\parallelogram OACB},&\text{从 $OA$ 到 $OB$ 是顺时针}.
	\end{cases}\]
	\onslide<+->
	那么
	\[\begin{vmatrix}
		a&c\\b&d
	\end{vmatrix}=\begin{vmatrix}
		a&c\\0&d
	\end{vmatrix}+\begin{vmatrix}
		0&c\\b&d
	\end{vmatrix}=a\begin{vmatrix}
		1&c\\0&d
	\end{vmatrix}+b\begin{vmatrix}
		0&c\\1&d
	\end{vmatrix}.\]
	\onslide<+->
	换言之, $\bfv$ 固定时, 则 $|\bfu,\bfv|$ 关于 $\bfu$ 是线性的.
	\onslide<+->
	同理, $\bfu$ 固定时, $|\bfu,\bfv|$ 关于 $\bfv$ 也是线性的.
\end{frame}


\begin{frame}{平行多面体情形\noexer}
	\onslide<+->
	将上述概念推广到 $n$ 维情形.
	\onslide<+->
	考虑 $n$ 维空间中由从原点出发的向量 $\bfv_1,\dots,\bfv_n$ 张成的平行多面体.
	\onslide<+->
	如果 $\bfv_1,\dots,\bfv_n$ 就是按顺序各个分量上的单位向量
	\[\bfe_1=(1,0,\dots,0)^\rmT,\quad
	\bfe_2=(0,1,\dots,0)^\rmT,\quad\cdots,\quad
	\bfe_n=(0,0,\dots,1)^\rmT,\]
	则有向面积为 $1$.
	\onslide<+->
	如果交换 $\bfv_i,\bfv_j$ 的位置, 有向面积相差 $-1$ 倍.
	\onslide<+->
	于是得到 $n$ 维情形的有向面积 $|\bfv_1,\dots,\bfv_n|$ 应当满足:
	\begin{enumerate}
		\item $|\bfe_1,\dots,\bfe_n|=1$;
		\item 反对称性: $|\cdots,\bfv_i,\cdots,\bfv_j,\cdots|=-|\cdots,\bfv_j,\cdots,\bfv_i,\cdots|$;
		\item $|\cdots,k\bma,\cdots|=k|\cdots,\bma,\cdots|$;
		\item $|\cdots,\bma+\bmb,\cdots|=|\cdots,\bma,\cdots|+|\cdots,\bmb,\cdots|$.
	\end{enumerate}
\end{frame}


\begin{frame}{行列式的定义\noexer}
	\onslide<+->
	设 $n$ 阶方阵 $\bfA=(a_{ij})$ 的各列形成的向量为 $\bfv_1,\dots,\bfv_n$,
	\onslide<+->
	称有向面积 $|\bfv_1,\cdots,\bfv_n|$ 就是方阵 $\bfA$ 的\emph{行列式}.
	\onslide<+->
	注意到
	\[\bfv_j=a_{1j}\bfe_1+a_{2j}\bfe_2+\cdots+a_{nj}\bfe_n,\]
	\onslide<+->
	利用线性性质将行列式展开将会得到 $n^n$ 项
	\[\sum_{k_1,k_2,\dots,k_n=1}^n|\bfe_{k_1},\cdots,\bfe_{k_n}| a_{k_11}a_{k_22}\cdots a_{k_nn}.\]

	\onslide<+->
	如果 $k_i=k_j$, 则交换 $\bfe_{k_i},\bfe_{k_j}$ 可知 $|\bfe_{k_1},\cdots,\bfe_{k_n}|=0$.
	\onslide<+->
	从而只剩下 $k_1,k_2,\dots,k_n$ 是 $1,2,\dots,n$ 的排列时的那些项.
	\onslide<+->
	例如: 
	\begin{align*}
		\begin{vmatrix}
			a_{11}&a_{12}\\
			a_{21}&a_{22}
		\end{vmatrix}&=|a_{11}\bfe_1,a_{12}\bfe_1+a_{22}\bfe_2|+|a_{21}\bfe_2,a_{12}\bfe_1+a_{22}\bfe_2|\\
		&=a_{11}a_{12}|\bfe_1,\bfe_1|+a_{11}a_{22}|\bfe_1,\bfe_2|+a_{21}a_{12}|\bfe_2,\bfe_1|+a_{21}a_{22}|\bfe_2,\bfe_2|\\
		&=a_{11}a_{22}-a_{21}a_{12}.
	\end{align*}
\end{frame}
	
	
\begin{frame}{行列式的展开形式\noexer}
	\onslide<+->
	设 $k_1,k_2,\dots,k_n$ 是 $1,2,\dots,n$ 的排列.
	\onslide<+->
	如果排列 $k_1,k_2,\dots,k_n$ 需要奇数次对换变成 $1,2,\dots,n$, 记 	$\sgn(k_1,\dots,k_n)=-1$; 否则 $\sgn(k_1,\dots,k_n)=+1$.
	\onslide<+->
	根据反对称性,
	\[|\bfe_{k_1},\cdots,\bfe_{k_n}|=\sgn(k_1,\dots,k_n)|\bfe_1,\cdots,\bfe_n|=\sgn(k_1,\dots,k_n).\]
	\onslide<+->
	\begin{definition}
		设 $\bfA=(a_{ij})$ 是 $n$ 阶方阵.
		定义 $\bfA$ 的\emph{行列式}为
		\[|\bfA|=\sum \sgn(k_1,\dots,k_n) a_{k_11}a_{k_22}\cdots a_{k_nn},\]
		其中 $k_1,k_2,\dots,k_n$ 取遍 $1,2,\dots,n$ 的全体排列.
	\end{definition}
\end{frame}


\begin{frame}{$2,3$ 阶行列式}
	\onslide<+->
	当 $n=2$ 时, $\sgn(12)=1,\sgn(21)=-1$,
	\onslide<+->
	于是
	\[\begin{vNiceMatrix}
		a_{11}&a_{12}\\
		a_{21}&a_{22}
		\CodeAfter
		\tikz \draw[thick,main,visible on=<3->] (1-|1) -- (3-|3);
		\tikz \draw[cstdash,second,visible on=<3->] (3-|1) -- (1-|3);
	\end{vNiceMatrix}
	:=a_{11}a_{22}-a_{12}a_{21}.\]

	\onslide<+->
	\onslide<+->
	当 $n=3$ 时,
	\[\sgn(123)=\sgn(231)=\sgn(312)=1,\]
	\[\sgn(132)=\sgn(213)=\sgn(321)=-1,\]
	\onslide<+->
	于是
	\[\begin{vNiceMatrix}
		a_{11}&a_{12}&a_{13}\\
		a_{21}&a_{22}&a_{23}\\
		a_{31}&a_{32}&a_{33}
		\CodeAfter
		\tikz \draw[thick,main,visible on=<6-8>] (1-|1) -- (4-|4);
		\tikz \draw[thick,second,visible on=<7-8>] (1-|2) -- (3-|4);
		\tikz \draw[thick,second,visible on=<7-8>] (3-|1) -- (4-|2);
		\tikz \draw[thick,third,visible on=<8>] (1-|3) -- (2-|4);
		\tikz \draw[thick,third,visible on=<8>] (2-|1) -- (4-|3);
		\tikz \draw[cstdash,main,visible on=<9->] (1-|4) -- (4-|1);
		\tikz \draw[cstdash,second,visible on=<10->] (1-|3) -- (3-|1);
		\tikz \draw[cstdash,second,visible on=<10->] (3-|4) -- (4-|3);
		\tikz \draw[cstdash,third,visible on=<11->] (1-|2) -- (2-|1);
		\tikz \draw[cstdash,third,visible on=<11->] (2-|4) -- (4-|2);
	\end{vNiceMatrix}
	:=
	a_{11}a_{22}a_{33}+a_{12}a_{23}a_{31}+a_{13}a_{21}a_{32}
	-a_{11}a_{23}a_{32}-a_{12}a_{21}a_{33}-a_{13}a_{22}a_{31}.
	\]
\end{frame}


\begin{frame}{例: $2,3$ 阶行列式的计算}
	\onslide<+->
	\begin{example}
		\begin{align*}
			\begin{vmatrix}
				1&3&2\\3&-5&1\\2&1&4
			\end{vmatrix}
			&\onslide<+->{=1\cdot (-5)\cdot 4+3\cdot 1\cdot2+2\cdot3\cdot1-1\cdot1\cdot1-3\cdot3\cdot4-2\cdot(-5)\cdot2}\\
			&\onslide<+->{=-20+6+6-1-36+20=-25.}
		\end{align*}
	\end{example}
	\onslide<+->
	\begin{exercise}
		若 $k>0$ 且 $\begin{vmatrix}
			k&2&1\\2&k&1\\k&1&2
		\end{vmatrix}=0$, 则 $k=$\fillblankframe{$2$}.
	\end{exercise}
\end{frame}


\begin{frame}{注记}
	\begin{enumerate}
		\item 行列式将一个方阵映射到一个数.
		\item $1$ 阶行列式就是方阵里面唯一的那个元素, 尽管也记作 $|\cdot|$, 但注意和绝对值区分.
		\item $2,3$ 阶行列式可以用对角线法直接得到展开式, 但是更高阶的没有这种表示方法.
		\item \emph{对角阵}的行列式
		\[\bigl|\diag(a_1,a_2,\dots,a_n)\bigr|=a_1a_2\cdots a_n,\]
		特别地 $|\bfE_n|=1,|\bfO_n|=0$.
		\item $|\bfA|$ 是由一些 $\pm a_{k_11}a_{k_22}\cdots a_{k_nn}$ 相加得到, 其中 $k_1,k_2,\dots,k_n$ 取遍 $1,2,\dots,n$ 的所有排列, 一共有 $n!$ 个这样的项, 其中一半取 $+$, 一半取 $-$ ($n\ge2$).
		\item {\itshape $|\bfA|$ 对应的线性变换(常数倍)是 $\bfA$ 对应的线性变换 $\BR^n\ra\BR^n$ 诱导的 $n$ 次外代数上的线性变换 $\bigwedge^n\BR^n\to\bigwedge^n\BR^n$, 感兴趣的可自行阅读有关材料.}
	\end{enumerate}
\end{frame}


% \begin{frame}{三阶行列式的几何意义}
% 	\onslide<+->
% 	类似地, 若 $A(a_1,a_2,a_3),B(b_1,b_2,b_3),C(c_1,c_2,c_3)$,
% 	\onslide<+->
% 	则三阶行列式 $\begin{vmatrix}
% 		a_1&a_2&a_3\\b_1&b_2&b_3\\c_1&c_2&c_3
% 	\end{vmatrix}$ 的绝对值就是下述平行六面体的体积.
% 	\begin{center}
% 	\begin{tikzpicture}[scale=.8]
% 		\draw[cstcurve,main] (0,0)--(3,0)--(4,1.5)--(1,1.5)--cycle;
% 		\draw[cstcurve,main] (4.5,1)--(5.5,2.5)--(2.5,2.5);
% 		\draw[cstdash,cstcurve,main] (2.5,2.5)--(1.5,1)--(4.5,1);
% 		\draw[cstdash,cstcurve,main] (0,0)--(1.5,1);
% 		\draw[cstcurve,main] (3,0)--(4.5,1);
% 		\draw[cstcurve,main] (4,1.5)--(5.5,2.5);
% 		\draw[cstcurve,main] (1,1.5)--(2.5,2.5);
% 		\draw (0,0) node[second,left] {$O$};
% 		\draw (3,0) node[second,right] {$A$};
% 		\draw (1,1.5) node[second,left] {$C$};
% 		\draw (1.5,1) node[second,below] {$B$};
% 	\end{tikzpicture}
% 	\end{center}
% 	\onslide<+->
% 	它的符号则表示使用右手四指从 $OA$ 旋转到 $OB$ 方向时, 大拇指所指方向与 $OC$ 是否在平面 $OAB$ 的同侧.
% \end{frame}


\subsection{行列式的性质}

\begin{frame}{行列式的乘性}
	\onslide<+->
	\begin{theorem@}
		\begin{enumerate}
			\item $|\bfA\bfB|=|\bfA|\cdot|\bfB|$.
		\end{enumerate}
	\end{theorem@}
	\onslide<+->
	\begin{proof}
		设 $f(\bfX):=|\bfA\bfX|$, 即 $f(\bfv_1,\dots,\bfv_n)=|\bfA\bfv_1,\dots,\bfA\bfv_n|$.
	\onslide<+->{%
			容易知道 $f$ 也满足反对称性, 且对任意 $\bfv_i$ 是线性的.
		}\onslide<+->{%
			类似于行列式展开可知, 对于 $\bfX=(a_{ij})$,
			\begin{align*}
				f(\bfX)&=\sum f(\bfe_{k_1},\dots,\bfe_{k_n})a_{k_11}a_{k_22}\cdots a_{k_nn}\\
				&=\sum f(\bfe_1,\dots,\bfe_n)\sgn(k_1,\dots,k_n)a_{k_11}a_{k_22}\cdots a_{k_nn}\\
				&=f(\bfE)|\bfX|=|\bfA|\cdot|\bfX|,
			\end{align*}
			其中 $k_1,k_2,\dots,k_n$ 取遍 $1,2,\dots,n$ 的全体排列.
		}\onslide<+->{%
			故 $|\bfA\bfB|=f(\bfB)=|\bfA|\cdot|\bfB|$.
		}
	\end{proof}
	\onslide<+->
	由此可知, 对于平行多面体 $V\subset \BR^n$, $\bfA$ 对应的线性映射将其面积变为 $|\bfA|$ 倍.
\end{frame}


\begin{frame}{例: 行列式的乘性}
	\onslide<+->
	\begin{example}
		证明:
		$\begin{vmatrix}
			a_1+b_1&b_1+c_1&c_1+a_1\\
			a_2+b_2&b_2+c_2&c_2+a_2\\
			a_3+b_3&b_3+c_3&c_3+a_3
		\end{vmatrix}=2\begin{vmatrix}
			a_1&b_1&c_1\\
			a_2&b_2&c_2\\
			a_3&b_3&c_3
		\end{vmatrix}$.
	\end{example}
	\onslide<+->
	\begin{proof}
		\[\begin{vmatrix}
			a_1+b_1&b_1+c_1&c_1+a_1\\
			a_2+b_2&b_2+c_2&c_2+a_2\\
			a_3+b_3&b_3+c_3&c_3+a_3
		\end{vmatrix}=\begin{vmatrix}
			a_1&b_1&c_1\\
			a_2&b_2&c_2\\
			a_3&b_3&c_3
		\end{vmatrix}\cdot\begin{vmatrix}
			1&0&1\\
			1&1&0\\
			0&1&1
		\end{vmatrix}\onslide<+->{=2\begin{vmatrix}
			a_1&b_1&c_1\\
			a_2&b_2&c_2\\
			a_3&b_3&c_3
		\end{vmatrix}.}\qedhere\]
	\end{proof}
\end{frame}


\begin{frame}{行列式的转置不变性}
	\onslide<+->
	\begin{theorem@}
		\begin{enumerate}
			\setcounter{enumi}{1}
			\item 转置不改变行列式: $|\bfA^\rmT|=|\bfA|$.
		\end{enumerate}
	\end{theorem@}
	\onslide<+->
	一个排列 $k_1,\dots,k_n$ 可以看成是集合 $\{1,2,\dots,n\}$ 到自身的双射 $i\mapsto k_i$.
	\onslide<+->
	设它的逆映射对应的排列是 $\ell_1,\dots,\ell_n$, 则 $\ell_{k_i}=i$.
	\onslide<+->
	由于
	\begin{align*}
		|\bfA|&=\sum\sgn(k_1,\dots,k_n) a_{k_11}\cdots a_{k_nn}=\sum\sgn(k_1,\dots,k_n)a_{1\ell_1}\cdots a_{n\ell_n},\\
		|\bfA^\rmT|&=\sum\sgn(\ell_1,\dots,\ell_n)a_{1\ell_1}\cdots a_{n\ell_n},
	\end{align*}
	\onslide<+->
	我们只需说明 $\sgn(k_1,\dots,k_n)=\sgn(\ell_1,\dots,\ell_n)$.
	\onslide<+->
	设
	\[\bfP=(\bfe_{k_1},\bfe_{k_2},\dots,\bfe_{k_n})\]
	的 $k_i$ 行 $i$ 列为 $1$, 其余项为零.
	\onslide<+->
	那么 $|\bfP|=\sgn(k_1,\dots,k_n)$, $|\bfP^\rmT|=\sgn(\ell_1,\dots,\ell_n)$.
	\onslide<+->
	由于 $\bfP\bfP^\rmT=\bfE$, 因此 $|\bfP|\cdot|\bfP^\rmT|=|\bfE|=1$.
	\onslide<+->
	而 $|\bfP|=\pm1$, 因此 $|\bfP|=|\bfP^\rmT|$.
\end{frame}


\begin{frame}{例: 方阵的行列式}
	\beqskip{6pt}
		\onslide<+->
		\begin{example}
			设 $\bfA=\begin{pmatrix}
				a&-b&-c&-d\\
				b&a&-d&c\\
				c&d&a&-b\\
				d&-c&b&a
			\end{pmatrix}$, 求 $|\bfA|$.
		\end{example}
		\onslide<+->
		\begin{solution}
			这题可以直接硬算, 不过我们可以利用一点小技巧:
		\onslide<+->{%
				\[\bfA\bfA^\rmT=\begin{pmatrix}
					a&-b&-c&-d\\
					b&a&-d&c\\
					c&d&a&-b\\
					d&-c&b&a
				\end{pmatrix}\begin{pmatrix}
					a&b&c&d\\
					-b&a&d&-c\\
					-c&-d&a&b\\
					-d&c&-b&a
				\end{pmatrix}=(a^2+b^2+c^2+d^2)\bfE.\]
			}\onslide<+->{%
				因此 $|\bfA|=\pm(a^2+b^2+c^2+d^2)^2$.
			}\onslide<+->{%
				因为 $|\bfA|$ 一定有 $a^4$ 项, 所以 $|\bfA|=(a^2+b^2+c^2+d^2)^2$.
			}
		\end{solution}
	\endgroup
\end{frame}


\begin{frame}{行列式线性性}
	\onslide<+->
	再根据行列式关于每个列向量的线性性和反对称性有:
	\onslide<+->
	\begin{theorem@}
		\begin{enumerate}
			\setcounter{enumi}{2}
			\item 互换两行(列)后, 方阵的行列式变为 $-1$ 倍.
			\item 方阵的某一行(列)乘 $k$ 后, 方阵的行列式变为 $k$ 倍.
			\item 将方阵某一行(列)对应向量写成两个向量之和, 则行列式也可对应拆成两个行列式之和.
		\end{enumerate}
	\end{theorem@}
	\onslide<+->
	\begin{corollary}
		\begin{enumerate}
			\item 具有相同的两行(列)的方阵的行列式为零: $|\cdots,\bfv,\cdots,\bfv,\cdots|=0$.
			\item 若方阵有一行(列)全为零, 则行列式为零: $|\cdots,{\bf0},\cdots|=0$.
			\item 若方阵有两行(列)成比例, 则行列式为零: $|\cdots,\bfv,\cdots,k\bfv,\cdots|=0$.
			\item 行列式中某一行(列)的公因子可以提到行列式外面.
		\end{enumerate}
	\end{corollary}
\end{frame}


\begin{frame}{初等变换}
	\onslide<+->
	计算行列式可以通过实施下列变换来化简:
	\onslide<+->
	\begin{third}{初等变换}
		\begin{enumerate}
			\item 互换两行(列): $\alertm{r_i\swap r_j, c_i\swap c_j}$, 行列式变号;
			\item 一行(列)乘\alert{非零常数} $k$: $\alertm{kr_i, kc_i}$, 行列式变为 $k$ 倍;
			\item $j$ 行(列)乘 $k$ 加到 $i$ 行(列): $\alertm{r_i+kr_j, c_i+kc_j}$, 行列式不变.
		\end{enumerate}
	\end{third}
	\onslide<+->
	实施第三类初等变换 $c_i+kc_j$ 时, 第 $j$ 列不变, 改变的是第 $i$ 列.
	\onslide<+->
	由于
	\begin{align*}
		|\cdots,\bfv_i,\cdots,\bfv_j,\cdots|
		&=|\cdots,\bfv_i,\cdots,\bfv_j,\cdots|
		+|\cdots,k\bfv_j,\cdots,\bfv_j,\cdots|\\
		&=|\cdots,\bfv_i+k\bfv_j,\cdots,\bfv_j,\cdots|,
	\end{align*}
	\onslide<+->
	因此第三类初等变换不改变行列式的值.
\end{frame}


\begin{frame}{例: 使用初等变换计算行列式}
	\onslide<+->
	\begin{exercise}
		\begin{enumerate}
			\item 判断题: $|\lambda \bfA|=\lambda|\bfA|$. \onslide<+->{\alert{$|\lambda \bfA|=\lambda^n|\bfA|$}}
			\item 判断题: $\begin{vmatrix}
				1&&&\\&2&&\\&&3&\\&&&4
			\end{vmatrix}=-\begin{vmatrix}
				&&&1\\&&2&\\&3&&\\4&&&
			\end{vmatrix}$. \onslide<+->{\alert{$|\bfe_4,\bfe_3,\bfe_2,\bfe_1|=1$}}
			\item 计算 $\begin{vmatrix}
				a_1+b_1&b_1+c_1&c_1+d_1&d_1+a_1\\
				a_2+b_2&b_2+c_2&c_2+d_2&d_2+a_2\\
				a_3+b_3&b_3+c_3&c_3+d_3&d_3+a_3\\
				a_4+b_4&b_4+c_4&c_4+d_4&d_4+a_4
			\end{vmatrix}=$\fillblankframe{$0$}.
			\item 设 $\bfA$ 为 $5$ 阶方阵, $|\bfA|=-1$, 则
				$|2\bfA|=$\fillblankframe{$-32$},
				$\bigl||\bfA|\bfA\bigr|=$\fillblankframe{$1$}.
		\end{enumerate}
	\end{exercise}
\end{frame}


\begin{frame}{例: 方阵的行列式}
	\onslide<+->
	\begin{example}
		计算
		$\begin{vmatrix}
			2\sin a\cos a&\sin a\cos b+\cos a\sin b&\sin a\cos c+\cos a\sin c\\
			\sin b\cos a+\cos b\sin a&2\sin b\cos b&\sin b\cos c+\cos b\sin c\\
			\sin c\cos a+\cos c\sin a&\sin c\cos b+\cos c\sin b&2\sin c\cos c
		\end{vmatrix}$.
	\end{example}
	\onslide<+->
	容易看出该方阵可写成两个方阵之和
	\[\begin{pmatrix}
		\sin a\cos a&\sin a\cos b&\sin a\cos c\\
		\sin b\cos a&\sin b\cos b&\sin b\cos c\\
		\sin c\cos a&\sin c\cos b&\sin c\cos c
	\end{pmatrix}+\begin{pmatrix}
		\cos a\sin a&\cos a\sin b&\cos a\sin c\\
		\cos b\sin a&\cos b\sin b&\cos b\sin c\\
		\cos c\sin a&\cos c\sin b&\cos c\sin c
	\end{pmatrix}.\]
	\onslide<+->
	这两个方阵各自满足各行成比例, 因此可分别写成
	\[\begin{pmatrix}
			\sin a\\
			\sin b\\
			\sin c
		\end{pmatrix}(\cos a,\cos b,\cos c),\qquad
		\begin{pmatrix}
			\cos a\\
			\cos b\\
			\cos c
		\end{pmatrix}(\sin a,\sin b,\sin c).\]
\end{frame}


\begin{frame}{例: 方阵的行列式}
	\onslide<+->
	因此原方阵为
	\[\begin{pmatrix}
			\sin a&\cos a\\
			\sin b&\cos b\\
			\sin c&\cos c
		\end{pmatrix}\cdot\begin{pmatrix}
			\cos a&\cos b&\cos c\\
			\sin a&\sin b&\sin c
		\end{pmatrix}.\]
	\onslide<+->
	\begin{solution}
		原式$=\begin{vmatrix}
			\sin a&\cos a&0\\
			\sin b&\cos b&0\\
			\sin c&\cos c&0
		\end{vmatrix}\cdot\begin{vmatrix}
			\cos a&\cos b&\cos c\\
			\sin a&\sin b&\sin c\\
			0&0&0
		\end{vmatrix}=0$.
	\end{solution}
	\onslide<+->
	设 $\bfA\in M_{m\times n},\bfB\in M_{n\times m}$.
	\onslide<+->
	若 $m>n$, 则
	\[\alertn{|\bfA\bfB|}=\left|
		(\bfA,\bfO_{m\times(m-n)})\begin{pmatrix}
		\bfB\\\bfO_{(m-n)\times m}
	\end{pmatrix}\right|\alertn{=0}.\]
\end{frame}


\begin{frame}{例: 方阵的行列式}
	\onslide<+->
	\begin{example}
		设 $n$ 阶方阵 $\bfA$ 是反对称阵.
		若 $n$ 是奇数, 则 $|\bfA|=0$.
	\end{example}
	\onslide<+->
	\begin{proof}
		由于 $\bfA^\rmT=-\bfA$, 于是
		\[|\bfA|=|\bfA^\rmT|=|{-\bfA}|=(-1)^n|\bfA|=-|\bfA|.\]
	\onslide<+->{%
			故 $|\bfA|=0$.\qedhere
		}
	\end{proof}
\end{frame}


\begin{frame}{例: 使用初等变换计算行列式}\small
	\onslide<+->
	\begin{example}
		若 $abcd=1$, 证明 $\bfA=\begin{pNiceMatrix}
			a^2+a^{-2}&a&a^{-1}&1\\
			b^2+b^{-2}&b&b^{-1}&1\\
			c^2+c^{-2}&c&c^{-1}&1\\
			d^2+d^{-2}&d&d^{-1}&1
		\end{pNiceMatrix}$ 行列式为零.
	\end{example}
	\onslide<+->
	\begin{proof}
		\[|\bfA|=
			\begin{vNiceMatrix}
				a^2&a&a^{-1}&1\\
				b^2&b&b^{-1}&1\\
				c^2&c&c^{-1}&1\\
				d^2&d&d^{-1}&1
			\end{vNiceMatrix}+\begin{vNiceMatrix}
				{a^{-2}}&a&a^{-1}&1\\
				{b^{-2}}&b&b^{-1}&1\\
				{c^{-2}}&c&c^{-1}&1\\
				{d^{-2}}&d&d^{-1}&1
			\end{vNiceMatrix}
			\onslide<+->{=abcd\begin{vNiceMatrix}
				a&1&{a^{-2}}&a^{-1}\\
				b&1&{b^{-2}}&b^{-1}\\
				c&1&{c^{-2}}&c^{-1}\\
				d&1&{d^{-2}}&d^{-1}
			\end{vNiceMatrix}+\begin{vNiceMatrix}
				a&{a^{-2}}&1&a^{-1}\\
				b&{b^{-2}}&1&b^{-1}\\
				c&{c^{-2}}&1&c^{-1}\\
				d&{d^{-2}}&1&d^{-1}
			\end{vNiceMatrix}.}\]
	\onslide<+->{%
			由于 $abcd=1$, 且等式右侧两个行列式相差 $-1$ 倍, 因此 $|\bfA|=0$.\qedhere
		}
	\end{proof}
\end{frame}


\subsection{拉普拉斯展开}

\begin{frame}{余子式和代数余子式}
	\onslide<+->
	我们来介绍行列式与方阵子式的联系.
	\onslide<+->
	\begin{definition}
		设 $\bfA=(a_{ij})$ 是 $n\ge2$ 阶方阵.
		\begin{enumerate}
			\item $\bfA$ 去掉第 $i$ 行和 $j$ 列得到的 $n-1$ 阶方阵的行列式称为 $\bfA$ 在 $(i,j)$ 处的\emph{余子式} (Minor), 记为 $M_{ij}$.
			\item 称 $A_{ij}=(-1)^{i+j}M_{ij}$ 为 $\bfA$ 在 $(i,j)$ 处的\emph{代数余子式} (Algebraic Minor).
		\end{enumerate}
	\end{definition}
	\onslide<+->
	注意余子式和代数余子式是数而不是矩阵.
\end{frame}


\begin{frame}{行列式与余子式的联系\noexer}
	\onslide<+->
	假设 $\bfA$ 的第 $n$ 列除了 $a_{nn}$ 都是零.
	\onslide<+->
	若 $k_n\neq n$, 则 $a_{k_11}\cdots a_{k_nn}=0$; 若 $k_n=n$, 则 $\sgn(k_1,\dots,k_{n-1},n)=\sgn(k_1,\dots,k_{n-1})$.
	\onslide<+->
	因此
	\[|\bfA|=\sum \sgn(k_1,\dots,k_{n-1},n)a_{k_1,1}\cdots a_{k_{n-1},n-1}a_{n,n}=a_{nn} M_{nn}=a_{nn} A_{nn}.\]

	\onslide<+->
	假设 $\bfA$ 的第 $j$ 列除了 $a_{ij}$ 都是零.
	\onslide<+->
	依次对 $\bfA$ 实施
	\[r_i\swap r_{i+1},\quad r_{i+1}\swap r_{i+2},\quad\dots,\quad r_{n-1}\swap r_n,\]
	得到的方阵 $\bfB$ 就是将 $\bfA$ 的第 $i$ 行移动到第 $n$ 行的后面得到的方阵.
	\onslide<+->
	由于一共 $n-i$ 次列互换, 因此 $|\bfB|=(-1)^{n-i}|\bfA|$.

	\onslide<+->
	同理, 将 $\bfB$ 的第 $j$ 列移动到第 $n$ 列的后面得到的方阵记为 $\bfC$, 则
	\[|\bfC|=(-1)^{n-j}|\bfB|=(-1)^{i+j}|\bfA|.\]
	\onslide<+->
	注意到 $\bfC$ 在 $(n,n)$ 处元素是 $a_{ij}$, 余子式是 $M_{ij}$,
	\onslide<+->
	因此
	\[|\bfC|=a_{ij}M_{ij},\qquad |\bfA|=(-1)^{i+j}a_{ij}M_{ij}=a_{ij}A_{ij}.\]
\end{frame}


\begin{frame}{行列式与余子式的联系}
	\onslide<+->
	将 $\bfA$ 的第 $j$ 列写成
	\[\begin{pmatrix}
		a_{1j}\\a_{2j}\\\vdots\\a_{nj}
	\end{pmatrix}
	=\begin{pmatrix}
		a_{1j}\\0\\\vdots\\0
	\end{pmatrix}
	+\begin{pmatrix}
		0\\a_{2j}\\\vdots\\0
	\end{pmatrix}+\cdots+\begin{pmatrix}
		0\\0\\\vdots\\a_{nj}
	\end{pmatrix},\]
	\onslide<+->
	根据行列式的线性性质, 我们得到
	\[|\bfA|=a_{1j}A_{1j}+a_{2j}A_{2j}+\cdots+a_{nj}A_{nj}.\]
	\onslide<+->
	由于转置不改变方阵的行列式, 于是得到
	\[|\bfA|=a_{i1}A_{i1}+a_{i2}A_{i2}+\cdots+a_{in}A_{in}.\]
\end{frame}


\begin{frame}{拉普拉斯展开}
	\onslide<+->
	\begin{second}{行列式沿任一行或列展开}
		方阵的行列式等于任一行(列)的元素与其对应的代数余子式乘积的和:
		\begin{align*}
			|\bfA|&=a_{i1}A_{i1}+a_{i2}A_{i2}+\cdots+a_{in}A_{in}\\
			&=a_{1j}A_{1j}+a_{2j}A_{2j}+\cdots+a_{nj}A_{nj}.
		\end{align*}
	\end{second}
	\onslide<+->
	由此也可以看出 \alert{$i\neq k$ 时,}
	\[\alertn{a_{i1}A_{k1}+a_{i2}A_{k2}+\cdots+a_{in}A_{kn}=0,}\]
	\onslide<+->
	因为它是第 $i,k$ 行相同的方阵的行列式.
\end{frame}


\begin{frame}{例: 三角阵的行列式}
	\onslide<+->
	\begin{example}
		\begin{align*}
			\begin{vmatrix}
				a_{11}&      &      &\\
				a_{21}&a_{22}&      &\\
				\vdots&\vdots&\ddots&\\
				a_{n1}&a_{n2}&\cdots&a_{nn}
			\end{vmatrix}
			&\onslide<+->{=a_{11}\begin{vmatrix}
				a_{22}&      &\\
				\vdots&\ddots&\\
				a_{n2}&\cdots&a_{nn}
			\end{vmatrix}}
			\onslide<+->{=a_{11}a_{22}\begin{vmatrix}
				a_{33}&      &\\
				\vdots&\ddots&\\
				a_{n3}&\cdots&a_{nn}
			\end{vmatrix}}\\
			&\onslide<+->{=\cdots=a_{11}a_{22}\cdots a_{nn}.}
		\end{align*}
		\onslide<+->{
			由于转置不改变行列式, 因此上\alert{三角阵行列式也等于对角元乘积}.}
	\end{example}
\end{frame}


\begin{frame}{例: 反对角阵的行列式}
	\onslide<+->
	\begin{example}
		计算 $|\bfA|$, 其中 $\bfA=\begin{pmatrix}
			&&&a_1\\&&a_2&\\&\udots&&\\a_n&&&
		\end{pmatrix}$.
	\end{example}
	\onslide<+->
	\begin{solution*}
		\begin{align*}
			|\bfA|&=(-1)^{n+1}a_1\begin{vmatrix}
				&&a_2\\&\udots&\\a_n&&
			\end{vmatrix}
			\onslide<+->{=(-1)^{n+1}a_1\cdot (-1)^{n}a_2\begin{vmatrix}
				&&a_3\\&\udots&\\a_n&&
			\end{vmatrix}}\\
			&\onslide<+->{=\cdots=\prod_{i=1}^n (-1)^{n-i}a_i}
			\onslide<+->{=(-1)^{\frac{n(n-1)}2}a_1a_2\cdots a_n.}
		\end{align*}
	\end{solution*}
\end{frame}


\begin{frame}{例: 利用初等变换计算行列式}
	\onslide<+->
	对于具体的方阵, 我们可以利用初等变换将其化为三角阵来计算行列式.
	\onslide<+->
	也可以在某一行或一列只有少数非零元时用拉普拉斯展开来降阶.
	\onslide<+->
	\begin{example}
		\begin{align*}
			\begin{vmatrix}
				2& 3& 1&-1\\
				-4&-5& 1& 3\\
				-3& 1&-5& 3\\
				1&-2& 0&-1
			\end{vmatrix}
			&\onslide<+->{\!\!\xeq[\nsmath{r_1-2r_4}]{\substack{\nsmath{r_2+4r_4}\\\nsmath{r_3+3r_4}}}\!\!
			\begin{vmatrix}
				\alertn0& 7& 1&1\\
				\alertn0&-13& 1&-1\\
				\alertn0&-5&-5&0\\
				\alertn1&-2& 0&-1
			\end{vmatrix}}
			\onslide<+->{=(-1)^{4+1}\begin{vmatrix}
				-13& 1&-1\\
				-5&-5&0\\
					7& 1&1
				\end{vmatrix}}\\
			&\onslide<+->{\xeq{\nsmath{r_3+r_1}}
			-\begin{vmatrix}
				-13& 1&\alertn{-1}\\
				-5&-5&\alertn0\\
				-6& 2&\alertn0
			\end{vmatrix}}
			\onslide<+->{=(-1)^{1+3}\begin{vmatrix}
				-5&-5\\-6&2
			\end{vmatrix}=-40.}
		\end{align*}
	\end{example}
\end{frame}


\begin{frame}{例: 利用初等变换计算行列式}
	\onslide<+->
	\begin{exercise}
		\begin{enumerate}
			\item $\begin{vmatrix}
				-2&0&1\\
				501&200&299\\
				500&200&300
			\end{vmatrix}=$\fillblankframe{$-200$}.
			\item $\begin{vmatrix}
				1&1&1\\
				a&b&c\\
				b+c&c+a&a+b
			\end{vmatrix}=$\fillblankframe{$0$}.
			\item 设 $\bma=(1,0,-1),\bfA=\bma^\rmT\bma$, 则
				$|5\bfE-\bfA^3|=$\fillblankframe{$-75$}.
		\end{enumerate}
	\end{exercise}
	\onslide<+->
	回忆: 若 $\bfA=\bma\bmb^\rmT$, 则 $\bfA^k=\lambda^{k-1}\bfA$, 其中 $k=\bmb^{\rmT}\bma$.
\end{frame}


\begin{frame}{例: 分块矩阵行列式}
	\onslide<+->
	\begin{example}
		设
		\[\bfA=\begin{pmatrix}
			a_{11}&\cdots&a_{1m}\\
			\vdots&\ddots&\vdots\\
			a_{m1}&\cdots&a_{mm}
		\end{pmatrix},\qquad
		\bfB=\begin{pmatrix}
			b_{11}&\cdots&b_{1n}\\
			\vdots&\ddots&\vdots\\
			b_{n1}&\cdots&b_{nn}
		\end{pmatrix},\]
		\[\bfC=\begin{pNiceMatrix}
			a_{11}&\cdots&a_{1m}&&&\\
			\vdots&\ddots&\vdots&&0&\\
			a_{m1}&\cdots&a_{mm}&&&\\
			*&\cdots&*&b_{11}&\cdots&b_{1n}\\
			\vdots&\ddots&\vdots&\vdots&\ddots&\vdots\\
			*&\cdots&*&b_{n1}&\cdots&b_{nn}
			\CodeAfter
			\tikz \draw[cstdash,second] (4-|1) -- (4-|7);
			\tikz \draw[cstdash,second] (1-|4) -- (7-|4);
		\end{pNiceMatrix}.\]
		证明 $|\bfC|=|\bfA|\cdot|\bfB|$.
	\end{example}
\end{frame}


\begin{frame}{例: 分块矩阵行列式}
	\onslide<+->
	\begin{proof*}
		对 $m$ 归纳.
		\onslide<+->{当 $m=1$ 时将 $|\bfC|$ 沿第一行展开可知成立.}

	\onslide<+->{%
			假设命题对于 $m-1$ 成立.
		}\onslide<+->{%
			设 $\bfA$ 在 $(1,j)$ 处的余子式为 $M_{1j}$, $\bfC$ 在 $(1,j)$ 处的余子式为 $N_{1j}$.
		}\onslide<+->{%
			则由归纳假设 $N_{1j}=M_{1j}|\bfB|$.
		}\onslide<+->{%
			因此
			\begin{align*}
				|\bfC|&=\sum_{j=1}^m (-1)^{1+j}a_{1j}N_{1j}\\
				&=\sum_{j=1}^m (-1)^{1+j}a_{1j}M_{1j}|\bfB|
				=|\bfA|\cdot|\bfB|.\qedhere
			\end{align*}
		}
	\end{proof*}
\end{frame}


\begin{frame}{例: 拉普拉斯展开的应用}
	\onslide<+->
	\begin{example}
		设 $\bfA=\begin{pmatrix}
			3&0&4&0\\
			2&2&2&2\\
			0&-7&0&0\\
			5&3&-2&2
		\end{pmatrix}$.
		计算 $A_{41}+A_{42}+A_{43}+A_{44}$ 和 $M_{41}+M_{42}+M_{43}+M_{44}$.
	\end{example}
	\onslide<+->
	\begin{solution}
		由拉普拉斯展开可知
		\[A_{41}+A_{42}+A_{43}+A_{44}
		=\begin{vmatrix}
			3&0&4&0\\
			2&2&2&2\\
			0&-7&0&0\\
			1&1&1&1
		\end{vmatrix}=0.\]
	\end{solution}
\end{frame}


\begin{frame}{例: 拉普拉斯展开的应用}
	\onslide<+->
		\begin{solution}[续解]
			\vspace{-\baselineskip}
			\begin{align*}
				M_{41}+M_{42}+M_{43}+M_{44}
				&=-A_{41}+A_{42}-A_{43}+A_{44}
				=\begin{vmatrix}
					3&0&4&0\\
					2&2&2&2\\
					0&-7&0&0\\
					-1&1&-1&1
				\end{vmatrix}\\
				&\onslide<+->{=7\begin{vmatrix}
					3&4&0\\
					2&2&2\\
					-1&-1&1
				\end{vmatrix}=-28.}
		\end{align*}
	\end{solution}
	\onslide<+->
	\begin{exercise}
		若 $\bfA=\begin{pmatrix}
			a_1&a_2&a_3&f\\
			b_1&b_2&b_3&f\\
			c_1&c_2&c_3&f\\
			d_1&d_2&d_3&f
		\end{pmatrix}$,
		则 $A_{11}+A_{21}+A_{31}+A_{41}=$\fillblankframe{$0$}.
		\vspace{-.2\baselineskip}
	\end{exercise}
\end{frame}


\subsection{行列式的计算举例}


\begin{frame}{例: 行和为常数的行列式}
	\onslide<+->
	计算 $n$ 阶矩阵的行列式可以使用初等变换将其变为三角型, 也可以使用拉普拉斯展开来对其实施降阶.
	\onslide<+->
	\begin{example}
		\begin{align*}
			\begin{vmatrix}
				a&1&\cdots&1\\
				1&a&\cdots&1\\
				\vdots&\vdots&\ddots&\vdots\\
				1&1&\cdots&a
			\end{vmatrix}
		&\onslide<+->{\xeq[\nsmath{i\ge2}]{\nsmath{c_1+c_i}}\begin{vmatrix}
				a+n-1&1&\cdots&1\\
				a+n-1&a&\cdots&1\\
				\vdots&\vdots&\ddots&\vdots\\
				a+n-1&1&\cdots&a
			\end{vmatrix}}
		\onslide<+->{=(a+n-1)\begin{vmatrix}
				1&1&\cdots&1\\
				1&a&\cdots&1\\
				\vdots&\vdots&\ddots&\vdots\\
				1&1&\cdots&a
			\end{vmatrix}}\\
		&\onslide<+->{\xeq[\nsmath{i\ge2}]{\nsmath{r_i-r_1}}(a+n-1)\begin{vmatrix}
				1&1&\cdots&1\\
				0&a-1&\cdots&0\\
				\vdots&\vdots&\ddots&\vdots\\
				0&0&\cdots&a-1
			\end{vmatrix}}
		\onslide<+->{=(a+n-1)(a-1)^{n-1}.}
		\end{align*}
	\end{example}
\end{frame}


\begin{frame}{例: 行和为常数的行列式}
	\onslide<+->
	若方阵的每行(列)之和为常数, 可用此法化简.
	\onslide<+->
	\begin{exercise}
		计算 $n$ 阶行列式 $\begin{vmatrix}
			1+a_1&a_2&\cdots&a_n\\
			a_1&1+a_2&\cdots&a_n\\
			\vdots&\vdots&\ddots&\vdots\\
			a_1&a_2&\cdots&1+a_n
		\end{vmatrix}=$\fillblankframe[4cm]{$1+a_1+\cdots+a_n$}.
	\end{exercise}
\end{frame}


\begin{frame}{例: 箭形行列式}
	\onslide<+->
	\begin{example}
		\[\begin{vmatrix}
			1&1&\cdots&1\\
			1&2&\cdots&0\\
			\vdots&\vdots&\ddots&\vdots\\
			1&0&\cdots&n
		\end{vmatrix}\onslide<+->{\xeq[\nsmath{i\ge 2}]{\nsmath{c_1-\dfrac1i c_i}}
		\begin{vmatrix}
			1-\dfrac12-\cdots-\dfrac1n&1&\cdots&1\\
			0&2&\cdots&0\\
			\vdots&\vdots&\ddots&\vdots\\
			0&0&\cdots&n
		\end{vmatrix}}
		\onslide<+->{=\Bigl(1-\frac12-\cdots-\frac1n\Bigr)n!.}\]
	\end{example}
	\onslide<+->
	一般的箭形行列式均可用此法处理.
\end{frame}


\begin{frame}{例: 特殊形状行列式}\small
	\beqskip{0pt}
	\onslide<+->
	\begin{exercise}
		计算 $n$ 阶行列式 $\begin{vmatrix}
			1&2&3&\cdots&n-1&n\\
			-1&0&3&\cdots&n-1&n\\
			-1&-2&0&\cdots&n-1&n\\
			\vdots&\vdots&\vdots&\ddots&\vdots&\vdots\\
			-1&-2&-3&\cdots&0&n\\
			-1&-2&-3&\cdots&-(n-1)&0
		\end{vmatrix}=$\fillblankframe{$n!$}.
	\end{exercise}
	\onslide<+->
	\begin{answer}
		对该方阵实施 $r_i+r_1,i\ge 2$ 即可化为上三角阵.
	\end{answer}
	\endgroup
\end{frame}


\begin{frame}{例: 降阶法}
	\beqskip{3pt}
	\onslide<+->
	\begin{example}
		计算矩阵 $\bfA_n=\begin{pmatrix}
			   x   &   -1    &    0    &\cdots&   0  &   0  \\
			   0   &    x    &   -1    &\cdots&   0  &   0  \\
			   0   &    0    &    x    &\cdots&   0  &   0  \\
			\vdots &  \vdots & \vdots  &\ddots&\vdots&\vdots\\
			   0   &    0    &    0    &\cdots&   x  &  -1  \\
				a_n  & a_{n-1} & a_{n-2} &\cdots&  a_2 & x+a_1
		\end{pmatrix}$ 的行列式.
	\end{example}
	\onslide<+->
	\begin{solution}
		沿着第一列展开得到
		\[|\bfA_n|=x|\bfA_{n-1}|+(-1)^{1+n}a_n(-1)^{n-1}=x|\bfA_{n-1}|+a_n,\]
	\onslide<+->{%
		递推或归纳可知
		\[|\bfA_n|=x(x|\bfA_{n-2}|+a_{n-1})+a_n=\cdots=x^n+a_1x^{n-1}+a_2x^{n-2}+\cdots+a_n.\]
	}\vspace{-\baselineskip}
	\end{solution}
	\endgroup
\end{frame}

\subsection{三对角和范德蒙型行列式}

\begin{frame}{例: 降阶法计算三对角矩阵行列式}
	\onslide<+->
	\begin{example}
		计算矩阵 $\bfA_n=\begin{pmatrix}
				2  &   1  &   0  &\cdots&   0  &   0  \\
				1  &   2  &   1  &\cdots&   0  &   0  \\
				0  &   1  &   2  &\cdots&   0  &   0  \\
			\vdots&\vdots&\vdots&\ddots&\vdots&\vdots\\
				0  &   0  &   0  &\cdots&   2  &   1  \\
				0  &   0  &   0  &\cdots&   1  &   2  \\
		\end{pmatrix}$ 的行列式.
	\end{example}
\end{frame}


\begin{frame}{例: 降阶法计算三对角矩阵行列式}
	\onslide<+->
	\begin{solution}
		设 $D_n=|\bfA_n|$.
	\onslide<+->{%
		沿着第一行展开得到
		\[|\bfA_n|=2|\bfA_{n-1}|-\begin{vmatrix}
			1  &   1  &   0  &\cdots&   0  &   0  \\
			0  &   2  &   1  &\cdots&   0  &   0  \\
			0  &   1  &   2  &\cdots&   0  &   0  \\
		\vdots&\vdots&\vdots&\ddots&\vdots&\vdots\\
			0  &   0  &   0  &\cdots&   2  &   1  \\
			0  &   0  &   0  &\cdots&   1  &   2
		\end{vmatrix}_{n-1}=2|\bfA_{n-1}|-|\bfA_{n-2}|,\]
	}\onslide<+->{%
		因此
		\[|\bfA_n|-|\bfA_{n-1}|=|\bfA_{n-1}|-|\bfA_{n-2}|=\cdots=|\bfA_2|-|\bfA_1|=1,\]
	}\onslide<+->{%
		从而 $|\bfA_n|=n-1+|\bfA_1|=n+1$.}
	\end{solution}
\end{frame}


\begin{frame}{例: 降阶法计算三对角矩阵行列式}
	\onslide<+->
	若主对角线元素均为 $a$, 上下副对角线元素均为 $b$ 和 $c$, 则
	\[
		|\bfA_n|-a|\bfA_{n-1}|+bc|\bfA_{n-2}|=0.
	\]
	\onslide<+->
	设 $\lambda^2-a\lambda+bc=0$ 的两个根为 $\lambda_1,\lambda_2$, 则归纳可知
	\begin{align*}
		|\bfA_n|&=\lambda_1^n+\lambda_1^{n-1}\lambda_2+\cdots+\lambda_1\lambda_2^{n-1}+\lambda_2^n\\
		&=\begin{cases}
			\dfrac{\lambda_1^{n+1}-\lambda_2^{n+1}}{\lambda_1-\lambda_2},&\text{若}\ \lambda_1\neq \lambda_2;\\[8pt]
			(n+1)\Bigl(\dfrac a2\Bigr)^n,&\text{若}\ \lambda_1=\lambda_2=\dfrac a2.
		\end{cases}
	\end{align*}
\end{frame}


\begin{frame}{例: 范德蒙行列式}
\beqskip{0pt}
	\onslide<+->
	\begin{main}{范德蒙行列式}
		设 $\bfA_n=\begin{pmatrix}
			1&1&1&\cdots&1\\
			x_1&x_2&x_3&\cdots&x_n\\
			x_1^2&x_2^2&x_3^2&\cdots&x_n^2\\
			\vdots&\vdots&\vdots&\ddots&\vdots\\
			x_1^{n-1}&x_2^{n-1}&x_3^{n-1}&\cdots&x_n^{n-1}
		\end{pmatrix}$.
		证明 \alert{$|\bfA_n|=\prod\limits_{1\le i<j\le n}(x_j-x_i)$}.
		\vspace{-.36\baselineskip}
	\end{main}
	\onslide<+->
	\begin{solution}[证明]
		归纳证明.
	\onslide<+->{%
			当 $n=1,2$ 时显然成立.
		}\onslide<+->{%
			设 $n\ge 3$, 由 $r_n-x_1 r_{n-1}, \dots,r_2-x_1r_1$ 得到
			\[|\bfA_n|=\begin{vmatrix}
				1&1&1&\cdots&1\\
				0&x_2-x_1&x_3-x_1&\cdots&x_n-x_1\\
				0&x_2(x_2-x_1)&x_3(x_3-x_1)&\cdots&x_n(x_n-x_1)\\
				\vdots&\vdots&\vdots&\ddots&\vdots\\
				0&x_2^{n-2}(x_2-x_1)&x_3^{n-2}(x_3-x_1)&\cdots&x_n^{n-2}(x_n-x_1)
			\end{vmatrix}.\]}
			\vspace{-.44\baselineskip}
	\end{solution}
\endgroup
\end{frame}


\begin{frame}{例: 范德蒙行列式}
		\onslide<+->
		\begin{proof}[续证]
		\onslide<+->{
			沿着第一列展开, 然后提取每一列的公因式 $(x_j-x_1)$ 得到
		\[|\bfA_n|
		=\prod_{j=2}^n(x_j-x_1)\begin{vmatrix}
				1&1&\cdots&1\\
				x_2&x_3&\cdots&x_n\\
				\vdots&\vdots&\ddots&\vdots\\
				x_2^{n-1}&x_3^{n-1}&\cdots&x_n^{n-1}
			\end{vmatrix}.\]}
			\onslide<+->{由归纳假设可知
			\[|\bfA_n|
		=\prod_{j=2}^n(x_j-x_1)\cdot \prod_{2\le i<j\le n}(x_j-x_i)=\prod_{1\le i<j\le n}(x_j-x_i).\qedhere\]}
	\end{proof}
\end{frame}


\begin{frame}{例: 范德蒙行列式的应用}
	\onslide<+->
	\begin{exercise}
		\begin{enumerate}
			\item $\begin{vmatrix}
				x_1^{-3}&x_2^{-3}&x_3^{-3}&x_4^{-3}\\
				x_1^{-1}&x_2^{-1}&x_3^{-1}&x_4^{-1}\\
				x_1&x_2&x_3&x_4\\
				x_1^{3}&x_2^{3}&x_3^{3}&x_4^{3}
			\end{vmatrix}=$\fillblankframe[6cm][3mm]{\onslide<+->{$x_1^{-3}x_2^{-3}x_3^{-3}x_4^{-3}\prod\limits_{1\le i<j\le 4}(x_j^2-x_i^2)$}}.
			\item $\begin{vmatrix}
				1&1&1&1\\
				1&2&3&4\\
				1&4&9&16\\
				1&8&27&65
			\end{vmatrix}=$\fillblankframe{$14$}.
			\onslide<.->{$\alert{=\begin{vmatrix}
				1&1&1&1\\
				1&2&3&4\\
				1&4&9&16\\
				1&8&27&64
			\end{vmatrix}+\begin{vmatrix}
				1&1&1&1\\
				1&2&3&4\\
				1&4&9&16\\
				0&0&0&1
			\end{vmatrix}}$}
			\item 设 $a,b,c$ 两两不等, 且 $\begin{vmatrix}
				a&b&c\\
				a^2&b^2&c^2\\
				b+c&c+a&a+b
			\end{vmatrix}=0$, 则 $a+b+c=$\fillblankframe{$0$}.
		\end{enumerate}
	\end{exercise}
\end{frame}


\begin{frame}{例: 范德蒙行列式\noexer}
	\onslide<+->
	范德蒙行列式还有另一种证明方式, 这种思路对于其它行列式的计算也有帮助.
	\onslide<+->
	\begin{proof}
		$f=|\bfA_n|$ 是 $x_1,\dots,x_n$ 的多项式, 且次数不超过 $1+2+\cdots+(n-1)$.
	\onslide<+->{%
			由于当 $x_i=x_j$ 时 $f=0$, 因此 $f$ 包含因式 $x_i-x_j$, 从而
			\[f=g\prod_{1\le i<j\le n}(x_j-x_i).\]
		}\onslide<+->{%
			比较两边次数可知 $g$ 是常数.
		}\onslide<+->{%
			注意到 $\prod\limits_{i=1}^n x_i^{i-1}$ 只出现在范德蒙行列式对角元的乘积中, 且它在 $\prod\limits_{1\le i<j\le n}(x_j-x_i)$ 中的系数是 $1$.
		}\onslide<+->{%
			因此 $g=1$.\qedhere
		}
	\end{proof}
\end{frame}


\begin{frame}{特殊形状行列式\noexer}
	% \beqskip{1pt}
	\onslide<+->
	\begin{example}
		计算 $\displaystyle\begin{vmatrix}
			1^{50}&2^{50}&\cdots&100^{50}\\
			2^{50}&3^{50}&\cdots&101^{50}\\
			\vdots&\vdots&\ddots&\vdots\\
			100^{50}&101^{50}&\cdots&199^{50}\\
		\end{vmatrix}$.
	\end{example}
	\onslide<+->
	\begin{proof}
		将第一行换成 $(x+1)^{50},\dots,(x+100)^{50}$, 并将行列式记为 $f(x)$.
	\onslide<+->{%
			那么 $f(1)=\cdots=f(99)=0$.
		}\onslide<+->{%
			注意到 $f$ 的次数不超过 $50$, 因此 $f\equiv 0$.\qedhere
		}
	\end{proof}
	\onslide<+->
	同理若 $k<n-1$, $\Bigl|\bigl((a_i+b_j)^k\bigr)_{1\le i,j\le n}\Bigr|=0$.
	% \onslide<+->
	% 对于 $k=n-1$, 可以通过
	% \[(x+b_j)^k=x^k+\rmC_k^1 x^{k-1} b_j+\cdots+\cdots+b_j^k\]
	% 将这个方阵拆分为
	% \[((a_i+b_j)^{n-1})_{ij}
	% =(a_i^{n-j})_{ij}\cdot \diag(\rmC_{n-1}^0,\rmC_{n-1}^1,\dots,\rmC_{n-1}^{n-1}) \cdot (b_j^{i-1})_{ij}.\]
	% \endgroup
\end{frame}


\begin{frame}{行列式常见计算方法总结}
	\begin{enumerate}
		\item $2,3$ 阶行列式可用对角线法直接展开.
		\item 三角阵行列式等于对角元的乘积, 分块三角阵行列式等于对角阵行列式乘积.
		\item 行列式的计算一般需要用到\alert{三类初等变换}, 创造出足够多的零.
		\item 行列式沿一行(列)的展开往往是\alert{降阶法}的必要手段.
		\item 范德蒙型行列式可处理方阵为元素幂次递增的情形.
	\end{enumerate}
\end{frame}

