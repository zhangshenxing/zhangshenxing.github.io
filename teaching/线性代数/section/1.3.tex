\section{方阵的行列式}

\subsection{行列式的定义}

\begin{frame}{行列式的抽象定义\noexer}
	\onslide<+->
	设 $n$ 维向量
	\[\bfe_1=(1,0,\dots,0)^\rmT,\quad
	\bfe_2=(0,1,\dots,0)^\rmT,\quad\cdots,\quad
	\bfe_n=(0,0,\dots,1)^\rmT.\]
	\onslide<+->
	那么 $V=\BC^n$ 中的任一元素均可唯一表达为如下形式:
	\[\bfx=x_1\bfe_1+\cdots+x_n\bfe_n.\]

	\onslide<+->
	定义 $V_2$ 为所有形如 $\bfv_1\wedge\bfv_2$ 的元素以及它们的有限和构成的集合, 其中 $\wedge$ 为满足如下性质的一种运算(\emph{外积}或\emph{楔积}):
	\begin{enumerate}
		\item 反对称性: $\bma\wedge \bmb=-\bmb\wedge\bma$;
		\item $\bma\wedge(\bmb+\bmg)=\bma\wedge\bmb+\bma\wedge\bmg$;
		\item $(\lambda\bma)\wedge\bmb=\bma\wedge(\lambda\bmb)=\lambda(\bma\wedge\bmb)$.
	\end{enumerate}
	\onslide<+->
	根据这些运算性质可以知道, $V_2$ 是一个线性空间, 且其中的元素可以唯一表达为如下形式(注意 $\bfv\wedge\bfv=0$):
	\[\sum_{1\le i<j\le n} x_{ij} \bfe_i\wedge\bfe_j.\]
\end{frame}


\begin{frame}{行列式的抽象定义\noexer}
	\onslide<+->
	类似地, 定义 $V_k$ 为所有形如 $\bfv_1\wedge\bfv_2\wedge\cdots\wedge\bfv_k$ 的元素以及它们的有限和构成的集合.
	\onslide<+->
	同理可知, $V_k$ 中的元素可以唯一表达为如下形式:
	\[\sum_{1\le i_1<\cdots<i_k\le n} x_{i_1,\dots,i_k} \bfe_{i_1}\wedge\cdots\wedge\bfe_{i_k}.\]

	\onslide<+->
	特别地, $k=n$ 时, $V_n$ 中的元素可以唯一表达为如下形式:
	\[x\bfe_1\wedge\cdots\wedge\bfe_n.\]
	\onslide<+->
	对于 $n$ 阶方阵 $\bfA$,
	\[\sum\bfv_1\wedge\dots\wedge \bfv_n\mapsto \sum\bfA\bfv_1\wedge \dots\wedge \bfA\bfv_n\]
	定义了 $V_n$ 上的一个线性映射.
	\onslide<+->
	根据 $V_n$ 的元素形式, 存在一个数 $|\bfA|$ 使得
	\[\bfA\bfv_1\wedge \dots\wedge \bfA\bfv_n=|\bfA| \bfv_1\wedge\dots\wedge \bfv_n,\]
	称之为 $\bfA$ 的\emph{行列式}, 记作 $|\bfA|$ 或 $\det(\bfA)$.
\end{frame}


\begin{frame}{行列式的具体定义\noexer}
	\onslide<+->
	设 $\bfA=(a_{ij})$ 的各列为 $\bfv_1,\dots,\bfv_n$, 则
	\[\bfA\bfe_1\wedge \dots\wedge \bfA\bfe_n=\bfv_1\wedge\cdots\wedge\bfv_n=\sum a_{1,k_1}\cdots a_{n,k_n} \bfe_{k_1}\wedge \dots\wedge \bfe_{k_n}\]
	其中 $k_1,k_2,\dots,k_n$ 取遍 $1,2,\dots,n$ 的全体排列.
	\onslide<+->
	记 $\sgn(k_1,k_2,\dots,k_n)$ 表示 $\bfe_{k_1}\wedge \dots\wedge \bfe_{k_n}$ 关于 $\bfe_1\wedge \dots\wedge \bfe_n$ 的倍数.
	\onslide<+->
	\begin{definition}
		设 $\bfA=(a_{ij})$ 是 $n$ 阶方阵.
		定义 $\bfA$ 的行列式为
		\[|\bfA|=\sum \sgn(k_1,k_2,\dots,k_n)a_{1,k_1}\cdots a_{n,k_n}.\]
		其中 $k_1,k_2,\dots,k_n$ 取遍 $1,2,\dots,n$ 的全体排列.
	\end{definition}
	\onslide<+->
	根据 $\wedge$ 的反对称性, 若排列 $k_1,k_2,\dots,k_n$ 需要奇数次对换变成 $1,2,\dots,n$, 则 $\sgn=-1$; 否则 $\sgn=1$.
\end{frame}


\begin{frame}{$2,3$ 阶行列式}
	\onslide<+->
	当 $n=2$ 时, $\sgn(1,2)=1,\sgn(2,1)=-1$,
	\onslide<+->
	于是(此处可省略矩阵的括号)
	\[\begin{vNiceMatrix}
		a_{11}&a_{12}\\
		a_{21}&a_{22}
		\CodeAfter
		\tikz \draw[thick,dcolora,visible on=<3->] (1-|1) -- (3-|3);
		\tikz \draw[cstdash,dcolorb,visible on=<3->] (3-|1) -- (1-|3);
	\end{vNiceMatrix}
	:=a_{11}a_{22}-a_{12}a_{21}.\]

	\onslide<+->
	\onslide<+->
	当 $n=3$ 时,
	\[\sgn(1,2,3)=\sgn(2,3,1)=\sgn(3,1,2)=1,\]
	\[\sgn(1,3,2)=\sgn(2,1,3)=\sgn(3,1,2)=-1,\]
	\onslide<+->
	于是
	\[\begin{vNiceMatrix}
		a_{11}&a_{12}&a_{13}\\
		a_{21}&a_{22}&a_{23}\\
		a_{31}&a_{32}&a_{33}
		\CodeAfter
		\tikz \draw[thick,dcolora,visible on=<6-8>] (1-|1) -- (4-|4);
		\tikz \draw[thick,dcolorb,visible on=<7-8>] (1-|2) -- (3-|4);
		\tikz \draw[thick,dcolorb,visible on=<7-8>] (3-|1) -- (4-|2);
		\tikz \draw[thick,dcolorc,visible on=<8>] (1-|3) -- (2-|4);
		\tikz \draw[thick,dcolorc,visible on=<8>] (2-|1) -- (4-|3);
		\tikz \draw[cstdash,dcolora,visible on=<9->] (1-|4) -- (4-|1);
		\tikz \draw[cstdash,dcolorb,visible on=<10->] (1-|3) -- (3-|1);
		\tikz \draw[cstdash,dcolorb,visible on=<10->] (3-|4) -- (4-|3);
		\tikz \draw[cstdash,dcolorc,visible on=<11->] (1-|2) -- (2-|1);
		\tikz \draw[cstdash,dcolorc,visible on=<11->] (2-|4) -- (4-|2);
	\end{vNiceMatrix}
	:=
	a_{11}a_{22}a_{33}+a_{12}a_{23}a_{31}+a_{13}a_{21}a_{32}
	-a_{11}a_{23}a_{32}-a_{12}a_{21}a_{33}-a_{13}a_{22}a_{31}.
	\]
\end{frame}


\begin{frame}{例: $2,3$ 阶行列式的计算}
	\onslide<+->
	\begin{example}
		$\begin{vmatrix}
			1&3&2\\3&-5&1\\2&1&4
	\end{vmatrix}$
		\onslide<+->{$=1\times (-5)\times 4+3\times 1\times2+2\times3\times1-1\times1\times1-3\times3\times4-2\times(-5)\times2$}

		\onslide<+->{\hspace{19.5mm}$=-20+6+6-1-36+20=-25.$}
	\end{example}
	\onslide<+->
	\begin{exercise}
		若 $k>0$ 且 $\begin{vmatrix}
			k&2&1\\2&k&1\\k&1&2
		\end{vmatrix}=0$, 则 $k=$\fillblank{\visible<+->{$2$}}.
	\end{exercise}
\end{frame}


\begin{frame}{注记}
	\onslide<+->
	\begin{enumerate}
		\item 行列式将一个方阵映射到一个数.
		\item $1$ 阶行列式就是方阵里面唯一的那个元素, 尽管也记作 $|\cdot|$, 但注意和绝对值区分.
		\item $2,3$ 阶行列式可以用对角线法直接得到展开式, 但是更高阶的没有这种表示方法.
		\item \emph{对角阵}的行列式
		\[\bigl|\diag(a_1,a_2,\dots,a_n)\bigr|=a_1a_2\cdots a_n,\]
		特别地 $|\bfE_n|=1,|\bfO_n|=0$.
		\item $|\bfA|$ 是由一些 $\pm a_{1 k_1}a_{2 k_2}\cdots a_{n k_n}$ 相加得到, 其中 $k_1,k_2,\dots,k_n$ 取遍 $1,2,\dots,n$ 的所有排列, 一共有 $n!$ 个这样的项, 其中一半取 $+$, 一半取 $-$ ($n\ge2$).
	\end{enumerate}
\end{frame}


\begin{frame}{二阶行列式的几何意义}
	\onslide<+->
	设平面上有平行四边形 $OACB$, 其中 $A(a,b), B(c,d)$.
	\begin{center}
	\begin{tikzpicture}[scale=.8]
		\draw[cstcurve,dcolora] (0,0)--(3,0)--(4,1.5)--(1,1.5)--cycle;
		\draw (0,0) node[dcolorb,left] {$O$};
		\draw (3,0) node[dcolorb,right] {$A$};
		\draw (1,1.5) node[dcolorb,left]{$B$};
		\draw (4,1.5) node[dcolorb,right] {$C$};
	\end{tikzpicture}
	\end{center}
	\onslide<+->
	二阶行列式 $\begin{vmatrix}
		a&b\\c&d
	\end{vmatrix}$ 的绝对值就是它的面积.
	\onslide<+->
	它的符号则表示从 $OA$ 沿最短角度旋转到 $OB$ 方向是逆时针还是顺时针.
\end{frame}


\begin{frame}{三阶行列式的几何意义}
	\onslide<+->
	类似地, 若 $A(a_1,a_2,a_3),B(b_1,b_2,b_3),C(c_1,c_2,c_3)$, 则三阶行列式 $\begin{vmatrix}
		a_1&a_2&a_3\\b_1&b_2&b_3\\c_1&c_2&c_3
	\end{vmatrix}$ 的绝对值就是下述平行六面体的体积.
	\begin{center}
	\begin{tikzpicture}[scale=.8]
		\draw[cstcurve,dcolora] (0,0)--(3,0)--(4,1.5)--(1,1.5)--cycle;
		\draw[cstcurve,dcolora] (4.5,1)--(5.5,2.5)--(2.5,2.5);
		\draw[cstdash,cstcurve,dcolora] (2.5,2.5)--(1.5,1)--(4.5,1);
		\draw[cstdash,cstcurve,dcolora] (0,0)--(1.5,1);
		\draw[cstcurve,dcolora] (3,0)--(4.5,1);
		\draw[cstcurve,dcolora] (4,1.5)--(5.5,2.5);
		\draw[cstcurve,dcolora] (1,1.5)--(2.5,2.5);
		\draw (0,0) node[dcolorb,left] {$O$};
		\draw (3,0) node[dcolorb,right] {$A$};
		\draw (1,1.5) node[dcolorb,left] {$C$};
		\draw (1.5,1) node[dcolorb,below] {$B$};
	\end{tikzpicture}
	\end{center}
	\onslide<+->
	它的符号则表示使用右手四指从 $OA$ 旋转到 $OB$ 方向时, 大拇指所指方向与 $OC$ 是否在平面 $OAB$ 的同侧.
\end{frame}


\subsection{行列式的性质}

\begin{frame}{行列式的线性性、反对称性和乘性}
	\onslide<+->
	我们将给出行列式的一系列性质, 这些性质可以帮助我们计算行列式.

	\onslide<+->
	根据 $\wedge$ 的线性性和反对称性有:
	\onslide<+->
	\begin{proposition}
		\begin{itemize}
			\item $|\bfv_1,\dots,k\bfv_i,\dots,\bfv_n|=k|\bfv_1,\dots,\bfv_i,\dots,\bfv_n|$.
			\item $|\bfv_1,\dots,\bma+\bmb,\dots,\bfv_n|=|\bfv_1,\dots,\bma,\dots,\bfv_n|+|\bfv_1,\dots,\bmb,\dots,\bfv_n|$.
			\item $|\dots,\bfv_i,\dots,\bfv_j,\dots|=-|\dots,\bfv_j,\dots,\bfv_i,\dots|$.
		\end{itemize}
	\end{proposition}
	\onslide<+->
	设 $\bfA,\bfB$ 都是 $n$ 阶方阵, 则
	\[\bfA\bfB\bfv_1\wedge \dots\wedge \bfA\bfB\bfv_n=|\bfA|\bfB\bfv_1\wedge \dots\wedge \bfB\bfv_n
	=|\bfA|\cdot|\bfB|\bfv_1\wedge\dots\wedge \bfv_n.\]
	\begin{alertblock@}
		\begin{enumerate}
			\item $|\bfA\bfB|=|\bfA|\cdot|\bfB|$.
		\end{enumerate}
	\end{alertblock@}
\end{frame}


\begin{frame}{转置的行列式}
	\onslide<+->
	\begin{alertblock@}
		\begin{enumerate}
			\setcounter{enumi}{1}
			\item 转置不改变行列式: $|\bfA^\rmT|=|\bfA|$.
		\end{enumerate}
	\end{alertblock@}
	\onslide<+->
	设 $k_1,\dots,k_n$ 是一个排列, $\bfP$ 的 $i$ 行 $k_i$ 列为 $1$, 其余项为零.
	\onslide<+->
	根据行列式定义可知, 该排列的 $\sgn=|\bfP|$.
	\onslide<+->
	设 $\ell_1,\dots,\ell_n$ 是一个排列且满足 $k_{\ell_i}=i$.
	\onslide<+->
	也就是说, 若把排列看成集合 $\{1,2,\dots,n\}$ 到自身的双射, $\ell$ 就是 $k$ 的逆映射.
	\onslide<+->
	于是
	\begin{align*}
		|\bfA|&=\sum|\bfP| a_{1k_1}\cdots a_{nk_n}=\sum|\bfP^\rmT| a_{1\ell_1}\cdots a_{n\ell_n},\\
		|\bfA^\rmT|&=\sum|\bfP| a_{k_1 1}\cdots a_{k_n n}=\sum|\bfP| a_{1\ell_1}\cdots a_{1 \ell_n}.
	\end{align*}
	\onslide<+->
	因此只需证明 $|\bfP|=|\bfP^\rmT|$.
	\onslide<+->
	注意到 $\bfP\bfP^\rmT=\bfE$, 因此 $|\bfP|\cdot|\bfP^\rmT|=|\bfE|=1$.
	\onslide<+->
	而 $|\bfP|=\sgn=\pm1$, 因此 $|\bfP|=|\bfP^\rmT|$.
\end{frame}


\begin{frame}{列变换的行列式}
	\onslide<+->
	由此可知:
	\onslide<+->
	\begin{alertblock@}
		\begin{enumerate}
			\setcounter{enumi}{2}
			\item 互换两行(列)后, 方阵的行列式变为 $-1$ 倍.
			\item 方阵的某一行(列)乘 $k$ 后, 方阵的行列式变为 $k$ 倍.
			\item 将方阵一行(列)每一个元素都写成两个数之和, 则行列式也可拆成两个行列式之和:
		\end{enumerate}
	\end{alertblock@}
	\onslide<+->
	\begin{corollary}
		\begin{enumerate}
			\item 具有相同的两行(列)的方阵的行列式为零: $|\cdots,\bfv,\cdots,\bfv,\cdots|=0$.
			\item 行列式中某一行(列)的公因子可以提到行列式外面.
			\item 若方阵有一行(列)全为零, 则行列式为零: $|\cdots,{\bf0},\cdots|=0$.
			\item 若方阵有两行(列)成比例, 则行列式为零: $|\cdots,\bfv,\cdots,k\bfv,\cdots|=0$.
		\end{enumerate}
	\end{corollary}
\end{frame}


\begin{frame}{三种初等变换}
	\onslide<+->
	计算行列式可以通过实施下列变换来化简:
	\onslide<+->
	\begin{block}{初等变换}
		\begin{enumerate}
			\item 互换两行(列): \alert{$r_i\swap r_j, c_i\swap c_j$}, 行列式变号;
			\item 一行(列)乘\alert{非零常数} $k$: \alert{$kr_i, kc_i$}, 行列式变为 $k$ 倍;
			\item $j$ 行(列)乘 $k$ 加到 $i$ 行(列): \alert{$r_i+kr_j, c_i+kc_j$}.
		\end{enumerate}
	\end{block}
	\onslide<+->
	\begin{exercise}
		\begin{enumerate}
			\item 判断题: $|\lambda \bfA|=\lambda|\bfA|$. \visible<+->{\alert{$|\lambda \bfA|=\lambda^n|\bfA|$}}
			\item 判断题: $\begin{vmatrix}
				1&&&\\&2&&\\&&3&\\&&&4
			\end{vmatrix}=-\begin{vmatrix}
				&&&1\\&&2&\\&3&&\\4&&&
			\end{vmatrix}$. \visible<+->{\Huge\alert{$\times$}}
		\end{enumerate}
	\end{exercise}
\end{frame}


\begin{frame}{例: 利用初等变换计算行列式}
	\onslide<+->
	\begin{example}
		\[\begin{vmatrix}
			 2& 3& 1&-1\\
			-4&-5& 1& 3\\
			-3& 1&-5& 3\\
			 1&-2& 0&-1
		\end{vmatrix}
		\onslide<+->{\!\!\xeq{r_1\swap r_4}\!\!}
		\onslide<+->{-\begin{vmatrix}
			1&-2& 0&-1\\
		 -4&-5& 1& 3\\
		 -3& 1&-5& 3\\
		  2& 3& 1&-1
		\end{vmatrix}}
		\onslide<+->{\!\!\xeq[r_4-2r_1]{\substack{\nsmath{r_2+4r_1}\\\nsmath{r_3+3r_1}}}\!\!}
		\onslide<+->{-\begin{vmatrix}
			1&-2& 0&-1\\
		  0&-13& 1&-1\\
		  0&-5&-5&0\\
		  0& 7& 1&1
		\end{vmatrix}}\]
		\[\onslide<+->{=-\begin{vmatrix}
			-13& 1&-1\\
			-5&-5&0\\
				7& 1&1
			\end{vmatrix}}
		\onslide<+->{\xeq{r_3+r_1}}
		\onslide<+->{-\begin{vmatrix}
			-13& 1&-1\\
			-5&-5&0\\
			-6& 2&0
		\end{vmatrix}}
		\onslide<+->{=(-1)^{1+3}\begin{vmatrix}
			-5&-5\\-6&2
		\end{vmatrix}=-40.}\]
	\end{example}
\end{frame}


\begin{frame}{例: 利用初等变换计算行列式}
	\onslide<+->
	\begin{exercise}
		$\begin{vmatrix}
			-2&0&1\\
			501&200&299\\
			500&200&300
		\end{vmatrix}=$\fillblank{\visible<+->{$-200$}}.
	\end{exercise}
	\onslide<+->
	\begin{example}
		证明:
		$\begin{vmatrix}
			a_1+b_1&b_1+c_1&c_1+a_1\\
			a_2+b_2&b_2+c_2&c_2+a_2\\
			a_3+b_3&b_3+c_3&c_3+a_3
		\end{vmatrix}=2\begin{vmatrix}
			a_1&b_1&c_1\\
			a_2&b_2&c_2\\
			a_3&b_3&c_3
		\end{vmatrix}$.
	\end{example}
\end{frame}


\begin{frame}{例: 利用初等变换计算行列式}
	\onslide<+->
	\begin{proof}
		\begin{align*}
			&\begin{vmatrix}
				a_1+b_1&b_1+c_1&c_1+a_1\\
				a_2+b_2&b_2+c_2&c_2+a_2\\
				a_3+b_3&b_3+c_3&c_3+a_3
			\end{vmatrix}\xeq{c_1-c_2}\begin{vmatrix}
				a_1-c_1&b_1+c_1&c_1+a_1\\
				a_2-c_2&b_2+c_2&c_2+a_2\\
				a_3-c_3&b_3+c_3&c_3+a_3
			\end{vmatrix}\\
			\visible<+->{\xeq{c_1+c_3}}&\visible<.->{\begin{vmatrix}
				2a_1&b_1+c_1&c_1+a_1\\
				2a_2&b_2+c_2&c_2+a_2\\
				2a_3&b_3+c_3&c_3+a_3
			\end{vmatrix}}
			\visible<+->{=2\begin{vmatrix}
				a_1&b_1+c_1&c_1+a_1\\
				a_2&b_2+c_2&c_2+a_2\\
				a_3&b_3+c_3&c_3+a_3
			\end{vmatrix}}\\
			\visible<+->{\xeq{c_3-c_1}}&\visible<.->{2\begin{vmatrix}
				a_1&b_1+c_1&c_1\\
				a_2&b_2+c_2&c_2\\
				a_3&b_3+c_3&c_3
			\end{vmatrix}}
			\visible<+->{\xeq{c_2-c_3}2\begin{vmatrix}
				a_1&b_1&c_1\\
				a_2&b_2&c_2\\
				a_3&b_3&c_3
			\end{vmatrix}.\qedhere}
		\end{align*}
	\end{proof}
\end{frame}


\begin{frame}{例: 利用初等变换计算行列式}
	\onslide<+->
	\begin{exercise}
		$\begin{vmatrix}
			a_1+b_1&b_1+c_1&c_1+d_1&d_1+a_1\\
			a_2+b_2&b_2+c_2&c_2+d_2&d_2+a_2\\
			a_3+b_3&b_3+c_3&c_3+d_3&d_3+a_3\\
			a_4+b_4&b_4+c_4&c_4+d_4&d_4+a_4
		\end{vmatrix}=$\fillblank{\visible<+->{$0$}}.
	\end{exercise}
	\onslide<+->
	\begin{exercise}
		$\begin{vmatrix}
			1&1&1\\
			a&b&c\\
			b+c&c+a&a+b
		\end{vmatrix}=$\fillblank{\visible<+->{$0$}}.
	\end{exercise}
\end{frame}


\begin{frame}{例: 利用初等变换计算行列式}
	\onslide<+->
	\begin{exercise}
		若 $\begin{vmatrix}
			x+1&2&-1\\
			2&x+1&1\\
			-1&1&x+1
		\end{vmatrix}=0$, 则 $x=$\fillblank[3cm]{\visible<+->{$-3,\pm\sqrt3$}}.
	\end{exercise}
	\onslide<+->
	\begin{example}
		若 $abcd=1$, 证明 $\bfA=\begin{pNiceMatrix}
			a^2+\dfrac1{a^2}&a&\dfrac1a&1\\
			b^2+\dfrac1{b^2}&b&\dfrac1b&1\\
			c^2+\dfrac1{c^2}&c&\dfrac1c&1\\
			d^2+\dfrac1{d^2}&d&\dfrac1d&1
		\end{pNiceMatrix}$ 行列式为零.
	\end{example}
\end{frame}


\begin{frame}{例: 计算行列式}
	\onslide<+->
	我们发现第一列可以拆分成两组数相加, 从而行列式也可以拆分成两个数.
	\onslide<+->
	\begin{proof}
		\begin{align*}
			|\bfA|&=
			\begin{vNiceMatrix}
				a^2&a&\dfrac1a&1\\
				b^2&b&\dfrac1b&1\\
				c^2&c&\dfrac1c&1\\
				d^2&d&\dfrac1d&1
			\end{vNiceMatrix}+\begin{vNiceMatrix}
				\dfrac1{a^2}&a&\dfrac1a&1\\
				\dfrac1{b^2}&b&\dfrac1b&1\\
				\dfrac1{c^2}&c&\dfrac1c&1\\
				\dfrac1{d^2}&d&\dfrac1d&1
			\end{vNiceMatrix}
			\onslide<+->{=abcd\begin{vNiceMatrix}
				a&1&\dfrac1{a^2}&\dfrac1a\\
				b&1&\dfrac1{b^2}&\dfrac1b\\
				c&1&\dfrac1{c^2}&\dfrac1c\\
				d&1&\dfrac1{d^2}&\dfrac1d
			\end{vNiceMatrix}+\begin{vNiceMatrix}
				a&\dfrac1{a^2}&1&\dfrac1a\\
				b&\dfrac1{b^2}&1&\dfrac1b\\
				c&\dfrac1{c^2}&1&\dfrac1c\\
				d&\dfrac1{d^2}&1&\dfrac1d
			\end{vNiceMatrix}}\\
			&\onslide<+->{=0.}\qedhere
		\end{align*}
	\end{proof}
\end{frame}


\subsection{拉普拉斯展开}

\begin{frame}{余子式和代数余子式}
	\onslide<+->
	我们来介绍行列式与方阵子式的联系.
	\onslide<+->
	\begin{definition}
		设 $\bfA=(a_{ij})$ 是 $n$ 阶方阵.
		\begin{enumerate}
			\item $\bfA$ 去掉第 $i$ 行和 $j$ 列得到的 $n-1$ 阶方阵的行列式称为 $\bfA$ 在 $(i,j)$ 处的\emph{\cnen{余子式}{Minor}}, 记为 $M_{ij}$.
			\item 称 $A_{ij}=(-1)^{i+j}M_{ij}$ 为 $\bfA$ 在 $(i,j)$ 处的\emph{\cnen{代数余子式}{Algebraic minor}}.
		\end{enumerate}
	\end{definition}
	\onslide<+->
	注意余子式和代数余子式是数而不是矩阵.

	\onslide<+->
	假设 $\bfA$ 的第一行除了 $a_{11}$ 都是零.
	\onslide<+->
	若 $k_1\neq 1$, 则 $a_{1,k_1}\cdots a_{n,k_n}=0$; 若 $k_1=1$, 则 $\sgn(1,k_2,\dots,k_n)=\sgn(k_2,\dots,k_n)$.
	\onslide<+->
	因此
	\[|\bfA|=\sum \sgn(1,k_2,\dots,k_n)a_{11}a_{2,k_2}\cdots a_{n,k_n}=a_{11} M_{11}=a_{11} A_{11}.\]
\end{frame}


\begin{frame}{拉普拉斯展开}
	\onslide<+->
	假设 $\bfA$ 的第 $i$ 行除了 $a_{ij}$ 都是零.
	\onslide<+->
	依次对 $\bfA$ 实施
	\[r_i\swap r_{i-1},\quad r_{i-1}\swap r_{i-2},\quad\dots,\quad r_2\swap r_1,\]
	得到的方阵 $\bfB$ 就是将 $\bfA$ 的第 $i$ 行移动到第一行的前面得到的方阵.
	\onslide<+->
	由于一共 $i-1$ 次行互换, 因此 $|\bfB|=(-1)^{i-1}|\bfA|$.

	\onslide<+->
	同理, 将 $\bfB$ 的第 $j$ 列移动到第一列的前面得到的方阵记为 $\bfC$, 则
	\[|\bfC|=(-1)^{j-1}|\bfB|=(-1)^{i+j}|\bfA|.\]
	\onslide<+->
	注意到 $\bfC$ 在 $(1,1)$ 处元素是 $a_{ij}$, 余子式是 $M_{ij}$, 因此 $|\bfC|=a_{ij}M_{ij}, |\bfA|=(-1)^{i+j}a_{ij}M_{ij}=a_{ij}A_{ij}$.
\end{frame}


\begin{frame}{拉普拉斯展开}

	\onslide<+->
	再根据行列式的线性性质, 我们得到\cnen{拉普拉斯}{Laplace}展开:
	\begin{theorem}[行列式沿任一行(列)展开]
		方阵的行列式等于任一行(列)的元素与其对应的代数余子式乘积的和:
		\begin{align*}
			|\bfA|&=a_{i1}A_{i1}+a_{i2}A_{i2}+\cdots+a_{in}A_{in}\\
			&=a_{1j}A_{1j}+a_{2j}A_{2j}+\cdots+a_{nj}A_{nj}.
		\end{align*}
	\end{theorem}

	\onslide<+->
	由此也可以看出 \alert{$i\neq k$ 时,}
	\[\alert{a_{i1}A_{k1}+a_{i2}A_{k2}+\cdots+a_{in}A_{kn}=0,}\]
	\onslide<+->
	因为它是第 $i,k$ 行相同的方阵的行列式.
\end{frame}


\begin{frame}{三角阵的行列式}
	\onslide<+->
	\begin{example}
		\begin{align*}
			\begin{vmatrix}
				a_{11}&      &      &\\
				a_{21}&a_{22}&      &\\
				\vdots&\vdots&\ddots&\\
				a_{n1}&a_{n2}&\cdots&a_{nn}
			\end{vmatrix}
			&\onslide<+->{=a_{11}\begin{vmatrix}
				a_{22}&      &\\
				\vdots&\ddots&\\
				a_{n2}&\cdots&a_{nn}
			\end{vmatrix}}
			\onslide<+->{=a_{11}a_{22}\begin{vmatrix}
				a_{33}&      &\\
				\vdots&\ddots&\\
				a_{n3}&\cdots&a_{nn}
			\end{vmatrix}}\\
			&\onslide<+->{=\cdots=a_{11}a_{22}\cdots a_{nn}.}
		\end{align*}
		\onslide<+->{
			由于转置不改变行列式, 因此上\alert{三角阵行列式也等于对角元乘积}.}
	\end{example}
\end{frame}


\begin{frame}{反对角阵的行列式}
	\onslide<+->
	\begin{example}
		计算 $|\bfA|$, 其中 $\bfA=\begin{pmatrix}
			&&&a_1\\&&a_2&\\&\udots&&\\a_n&&&
		\end{pmatrix}$.
	\end{example}
	\onslide<+->
	\begin{solution*}
		\begin{align*}
			|\bfA|&=(-1)^{n+1}a_1\begin{vmatrix}
				&&a_2\\&\udots&\\a_n&&
			\end{vmatrix}
			\visible<+->{=(-1)^{n+1}a_1\cdot (-1)^{n}a_2\begin{vmatrix}
				&&a_3\\&\udots&\\a_n&&
			\end{vmatrix}}\\
			&\visible<+->{=\cdots=\prod_{i=1}^n (-1)^{n-i}a_i}
			\visible<+->{=(-1)^{\frac{n(n-1)}2}a_1a_2\cdots a_n.}
		\end{align*}
	\end{solution*}
\end{frame}


\begin{frame}{例: 分块矩阵行列式}
	\onslide<+->
	\begin{example}
		设
		\[\bfA=\begin{pmatrix}
			a_{11}&\cdots&a_{1m}\\
			\vdots&\ddots&\vdots\\
			a_{m1}&\cdots&a_{mm}
		\end{pmatrix},\qquad
		\bfB=\begin{pmatrix}
			b_{11}&\cdots&b_{1n}\\
			\vdots&\ddots&\vdots\\
			b_{n1}&\cdots&b_{nn}
		\end{pmatrix},\]
		\[\bfC=\begin{pNiceMatrix}
			a_{11}&\cdots&a_{1m}&&&\\
			\vdots&\ddots&\vdots&&0&\\
			a_{m1}&\cdots&a_{mm}&&&\\
			*&\cdots&*&b_{11}&\cdots&b_{1n}\\
			\vdots&\ddots&\vdots&\vdots&\ddots&\vdots\\
			*&\cdots&*&b_{n1}&\cdots&b_{nn}
			\CodeAfter
			\tikz \draw[cstdash,dcolora] (4-|1) -- (4-|7);
			\tikz \draw[cstdash,dcolora] (1-|4) -- (7-|4);
		\end{pNiceMatrix}.\]
		证明 $|\bfC|=|\bfA|\cdot|\bfB|$.
	\end{example}
\end{frame}


\begin{frame}{例: 分块矩阵行列式}
	\onslide<+->
	\begin{proof*}
		对 $m$ 归纳.
		\onslide<+->{当 $m=1$ 时将 $|\bfC|$ 沿第一行展开可知成立.}

		\onslide<+->{%
			假设命题对于 $m-1$ 成立.
		}\onslide<+->{%
			设 $\bfA$ 在 $(1,j)$ 处的余子式为 $M_{1j}$, $\bfC$ 在 $(1,j)$ 处的余子式为 $N_{1j}$.
		}\onslide<+->{%
			则由归纳假设 $N_{1j}=M_{1j}|\bfB|$.
		}\onslide<+->{%
			因此
			\begin{align*}
				|\bfC|&=\sum_{j=1}^m (-1)^{1+j}a_{1j}N_{1j}\\
				&=\sum_{j=1}^m (-1)^{1+j}a_{1j}M_{1j}|\bfB|
				=|\bfA|\cdot|\bfB|.\qedhere
			\end{align*}
		}
	\end{proof*}
\end{frame}

\begin{frame}{例: 拉普拉斯展开的应用}
	\onslide<+->
	\begin{example}
		设 $\bfA=\begin{pmatrix}
			3&0&4&0\\
			2&2&2&2\\
			0&-7&0&0\\
			5&3&-2&2
		\end{pmatrix}$.
		计算 $A_{41}+A_{42}+A_{43}+A_{44}$ 和 $M_{41}+M_{42}+M_{43}+M_{44}$.
	\end{example}
	\onslide<+->
	\begin{solution}
		由拉普拉斯展开可知
		\[A_{41}+A_{42}+A_{43}+A_{44}
		=\begin{vmatrix}
			3&0&4&0\\
			2&2&2&2\\
			0&-7&0&0\\
			1&1&1&1
		\end{vmatrix}=0.\]
	\end{solution}
\end{frame}


\begin{frame}{例: 拉普拉斯展开的应用}
	\onslide<+->
		\begin{solutionc}
			\vspace{-\baselineskip}
			\begin{align*}
				M_{41}+M_{42}+M_{43}+M_{44}
				&=-A_{41}+A_{42}-A_{43}+A_{44}
				=\begin{vmatrix}
					3&0&4&0\\
					2&2&2&2\\
					0&-7&0&0\\
					-1&1&-1&1
				\end{vmatrix}\\
				&\onslide<+->{=7\begin{vmatrix}
					3&4&0\\
					2&2&2\\
					-1&-1&1
				\end{vmatrix}=-28.}
		\end{align*}
	\end{solutionc}
	\onslide<+->
	\begin{exercise}
		若 $\bfA=\begin{pmatrix}
			a_1&a_2&a_3&f\\
			b_1&b_2&b_3&f\\
			c_1&c_2&c_3&f\\
			d_1&d_2&d_3&f
		\end{pmatrix}$,
		则 $A_{11}+A_{21}+A_{31}+A_{41}=$\fillblank{\visible<+->{$0$}}.
		\vspace{-.2\baselineskip}
	\end{exercise}
\end{frame}


\subsection{行列式的计算举例}


\begin{frame}{例: 行和为常数的行列式}
	\onslide<+->
	计算 $n$ 阶矩阵的行列式可以使用初等变换将其变为三角型,也可以使用拉普拉斯展开来对其实施降阶。
	\onslide<+->
	\begin{example}
		\begin{align*}
			\begin{vmatrix}
				a&1&\cdots&1\\
				1&a&\cdots&1\\
				\vdots&\vdots&\ddots&\vdots\\
				1&1&\cdots&a
			\end{vmatrix}
		&\onslide<+->{\xeq[i\ge2]{c_1+c_i}\begin{vmatrix}
				a+n-1&1&\cdots&1\\
				a+n-1&a&\cdots&1\\
				\vdots&\vdots&\ddots&\vdots\\
				a+n-1&1&\cdots&a
			\end{vmatrix}}
		\onslide<+->{=(a+n-1)\begin{vmatrix}
				1&1&\cdots&1\\
				1&a&\cdots&1\\
				\vdots&\vdots&\ddots&\vdots\\
				1&1&\cdots&a
			\end{vmatrix}}\\
		&\onslide<+->{\xeq[i\ge2]{r_i-r_1}(a+n-1)\begin{vmatrix}
				1&1&\cdots&1\\
				0&a-1&\cdots&0\\
				\vdots&\vdots&\ddots&\vdots\\
				0&0&\cdots&a-1
			\end{vmatrix}}
		\onslide<+->{=(a+n-1)(a-1)^{n-1}.}
		\end{align*}
	\end{example}
\end{frame}


\begin{frame}{例: 行和为常数的行列式}
	\onslide<+->
	若方阵的每行(列)之和为常数, 可用此法化简.
	\onslide<+->
	\begin{exercise}
		计算 $n$ 阶行列式 $\begin{vmatrix}
			1+a_1&a_2&\cdots&a_n\\
			a_1&1+a_2&\cdots&a_n\\
			\vdots&\vdots&\ddots&\vdots\\
			a_1&a_2&\cdots&1+a_n
		\end{vmatrix}=$\fillblank[4cm]{\visible<+->{$1+a_1+\cdots+a_n$}}.
	\end{exercise}
\end{frame}


\begin{frame}{例: 箭形行列式}
	\onslide<+->
	\begin{example}
		\[\begin{vmatrix}
			1&1&\cdots&1\\
			1&2&\cdots&0\\
			\vdots&\vdots&\ddots&\vdots\\
			1&0&\cdots&n
		\end{vmatrix}\onslide<+->{\xeq[i\ge 2]{r_1-\dfrac1i r_i}
		\begin{vmatrix}
			1-\dfrac12-\cdots-\dfrac1n&1&\cdots&1\\
			0&2&\cdots&0\\
			\vdots&\vdots&\ddots&\vdots\\
			0&0&\cdots&n
		\end{vmatrix}}
		\onslide<+->{=\Bigl(1-\frac12-\cdots-\frac1n\Bigr)n!.}\]
	\end{example}
	\onslide<+->
	一般的箭形行列式均可用此法处理.
\end{frame}


\begin{frame}{例: 特殊形状行列式}
	\onslide<+->
	\begin{exercise}
		计算 $n$ 阶行列式 $\begin{vmatrix}
			1&2&3&\cdots&n-1&n\\
			-1&0&3&\cdots&n-1&n\\
			-1&-2&0&\cdots&n-1&n\\
			\vdots&\vdots&\vdots&\ddots&\vdots&\vdots\\
			-1&-2&-3&\cdots&0&n\\
			-1&-2&-3&\cdots&-(n-1)&0
		\end{vmatrix}=$\fillblank{\visible<+->{$n!$}}.
	\end{exercise}
\end{frame}


\begin{frame}{例: 降阶法}
	\onslide<+->
	\begin{example}
		计算矩阵 $\bfA_n=\begin{pmatrix}
			   x   &   -1    &    0    &\cdots&   0  &   0  \\
			   0   &    x    &   -1    &\cdots&   0  &   0  \\
			   0   &    0    &    x    &\cdots&   0  &   0  \\
			\vdots &  \vdots & \vdots  &\ddots&\vdots&\vdots\\
			   0   &    0    &    0    &\cdots&   x  &  -1  \\
				a_n  & a_{n-1} & a_{n-2} &\cdots&  a_2 & x+a_1
		\end{pmatrix}$ 的行列式.
	\end{example}
	\onslide<+->
	\begin{solution}
		沿着第一列展开得到
		\[|\bfA_n|=x|\bfA_{n-1}|+(-1)^{1+n}a_n(-1)^{n-1}=x|\bfA_{n-1}|+a_n,\]
		\onslide<+->{%
			递推可知 $|\bfA_n|=x^n+a_1x^{n-1}+a_2x^{n-2}+\cdots+a_n$.
		}
	\end{solution}
\end{frame}

\subsection{三对角和范德蒙型行列式}

\begin{frame}{例: 降阶法计算三对角矩阵行列式}
	\onslide<+->
	\begin{example}
		计算矩阵 $\bfA_n=\begin{pmatrix}
				2  &   1  &   0  &\cdots&   0  &   0  \\
				1  &   2  &   1  &\cdots&   0  &   0  \\
				0  &   1  &   2  &\cdots&   0  &   0  \\
			\vdots&\vdots&\vdots&\ddots&\vdots&\vdots\\
				0  &   0  &   0  &\cdots&   2  &   1  \\
				0  &   0  &   0  &\cdots&   1  &   2  \\
		\end{pmatrix}$ 的行列式.
	\end{example}
\end{frame}


\begin{frame}{例: 降阶法计算三对角矩阵行列式}
	\onslide<+->
	\begin{solution}
		设 $D_n=|\bfA_n|$.
		\onslide<+->{
		沿着第一行展开得到
		\[|\bfA_n|=2|\bfA_{n-1}|-\begin{vmatrix}
			1  &   1  &   0  &\cdots&   0  &   0  \\
			0  &   2  &   1  &\cdots&   0  &   0  \\
			0  &   1  &   2  &\cdots&   0  &   0  \\
		\vdots&\vdots&\vdots&\ddots&\vdots&\vdots\\
			0  &   0  &   0  &\cdots&   2  &   1  \\
			0  &   0  &   0  &\cdots&   1  &   2
		\end{vmatrix}_{n-1}=2|\bfA_{n-1}|-|\bfA_{n-2}|,\]
		}\onslide<+->{%
		因此
		\[|\bfA_n|-|\bfA_{n-1}|=|\bfA_{n-1}|-|\bfA_{n-2}|=\cdots=|\bfA_2|-|\bfA_1|=1,\]
		}\onslide<+->{%
		从而 $|\bfA_n|=n-1+|\bfA_1|=n+1$.}
	\end{solution}
\end{frame}


\begin{frame}{例: 降阶法计算三对角矩阵行列式}
	\onslide<+->
	若主对角线元素均为 $a$, 上下副对角线元素均为 $b$ 和 $c$, 则
		\[|\bfA_n|-a|\bfA_{n-1}|+bc|\bfA_{n-2}|=0.\]
	\onslide<+->
	这种线性递推数列有通用解法.
	\onslide<+->
	设 $\lambda^2-a\lambda+bc=0$ 的两个根为 $\lambda_1,\lambda_2$, 则归纳可知
	\begin{align*}
		|\bfA_n|&=\lambda_1^n+\lambda_1^{n-1}\lambda_2+\cdots+\lambda_1\lambda_2^{n-1}+\lambda_2^n\\
		&=\begin{cases}
			\dfrac{\lambda_1^{n+1}-\lambda_2^{n+1}}{\lambda_1-\lambda_2},&\text{若}\ \lambda_1\neq \lambda_2;\\[8pt]
			(n+1)\Bigl(\dfrac a2\Bigr)^n,&\text{若}\ \lambda_1=\lambda_2=\dfrac a2.
		\end{cases}
	\end{align*}
\end{frame}


\begin{frame}{例: 范德蒙行列式}\small
\beqskip{2pt}
	\onslide<+->
	\begin{example}[范德蒙行列式]
		设 $\bfA_n=\begin{pmatrix}
			1&1&1&\cdots&1\\
			x_1&x_2&x_3&\cdots&x_n\\
			x_1^2&x_2^2&x_3^2&\cdots&x_n^2\\
			\vdots&\vdots&\vdots&\ddots&\vdots\\
			x_1^{n-1}&x_2^{n-1}&x_3^{n-1}&\cdots&x_n^{n-1}
		\end{pmatrix}$.
		证明 \alert{$|\bfA_n|=\prod\limits_{1\le i<j\le n}(x_j-x_i)$}.
	\end{example}
	\onslide<+->
	\begin{proofs}
		归纳证明.
		\onslide<+->{%
			当 $n=1,2$ 时显然成立.
		}\onslide<+->{%
			设 $n\ge 3$, 由 $r_n-x_1 r_{n-1}, \dots,r_2-x_1r_1$ 得到
			\[|\bfA_n|=\begin{vmatrix}
				1&1&1&\cdots&1\\
				0&x_2-x_1&x_3-x_1&\cdots&x_n-x_1\\
				0&x_2(x_2-x_1)&x_3(x_3-x_1)&\cdots&x_n(x_n-x_1)\\
				\vdots&\vdots&\vdots&\ddots&\vdots\\
				0&x_2^{n-2}(x_2-x_1)&x_3^{n-2}(x_3-x_1)&\cdots&x_n^{n-2}(x_n-x_1)
			\end{vmatrix}.\]}
	\end{proofs}
\endgroup
\end{frame}

\begin{frame}{例: 范德蒙行列式}
		\onslide<+->
		\begin{proofe}
		\onslide<+->{
			沿着第一列展开, 然后提取每一列的公因式 $(x_j-x_1)$ 得到
		\[|\bfA_n|
		=\prod_{j=2}^n(x_j-x_1)\begin{vmatrix}
				1&1&\cdots&1\\
				x_2&x_3&\cdots&x_n\\
				\vdots&\vdots&\ddots&\vdots\\
				x_2^{n-1}&x_3^{n-1}&\cdots&x_n^{n-1}
			\end{vmatrix}.\]}
			\onslide<+->{由归纳假设可知
			\[|\bfA_n|
		=\prod_{j=2}^n(x_j-x_1)\cdot \prod_{2\le i<j\le n}(x_j-x_i)=\prod_{1\le i<j\le n}(x_j-x_i).\qedhere\]}
	\end{proofe}
\end{frame}


\begin{frame}{例: 范德蒙行列式的应用}
	\onslide<+->
	\begin{exercise}
		\begin{enumerate}
			\item $\begin{vmatrix}
				x_1^{-3}&x_2^{-3}&x_3^{-3}&x_4^{-3}\\
				x_1^{-1}&x_2^{-1}&x_3^{-1}&x_4^{-1}\\
				x_1&x_2&x_3&x_4\\
				x_1^{3}&x_2^{3}&x_3^{3}&x_4^{3}
			\end{vmatrix}=$\fillblank[6cm][3mm]{\visible<4->{$x_1^{-3}x_2^{-3}x_3^{-3}x_4^{-3}\prod\limits_{1\le i<j\le 4}(x_j^2-x_i^2)$}}.
			\item $\begin{vmatrix}
				1&1&1&1\\
				1&2&3&4\\
				1&4&9&16\\
				1&8&27&65
			\end{vmatrix}=$\fillblank{\visible<5->{$14$}}.
			\item 设 $a,b,c$ 两两不等, 且 $\begin{vmatrix}
				a&b&c\\
				a^2&b^2&c^2\\
				b+c&c+a&a+b
			\end{vmatrix}=0$, 则 $a+b+c=$\fillblank{\visible<6->{$0$}}.
		\end{enumerate}
	\end{exercise}
\end{frame}


\begin{frame}{行列式常见计算方法总结}
	\onslide<+->
	\begin{enumerate}
		\item $2,3$ 阶行列式可用对角线法直接展开.
		\item 上(下)三角阵行列式等于对角元的乘积.
		\item 行列式的计算一般需要用到\alert{三类初等变换}, 创造出足够多的零.
		\item 行列式沿一行(列)的展开往往是\alert{降阶法}的必要手段.
		\item \cnen{范德蒙}{Vandermonde}型行列式可处理方阵为元素幂次递增的情形.
	\end{enumerate}
\end{frame}


\begin{frame}{例: 方阵的行列式}
	\onslide<+->
	\begin{exercise}
		设 $\bfA$ 为 $5$ 阶方阵, $|\bfA|=-1$, 则
		$|2\bfA|=$\fillblank{\visible<+->{$-32$}},
		$\bigl||\bfA|\bfA\bigr|=$\fillblank{\visible<+->{$1$}}.
	\end{exercise}
	\onslide<+->
	\begin{exercise}
		设 $\bma=(1,0,-1),\bfA=\bma^\rmT\bma$, 则
		$|5\bfE-\bfA^3|=$\fillblank{\visible<+->{$-75$}}.
	\end{exercise}
	\onslide<+->
	\begin{exercise}
		$\begin{vmatrix}
			2\sin a\cos a&\sin a\cos b+\cos a\sin b&\sin a\cos c+\cos a\sin c\\
			\sin b\cos a+\cos b\sin a&2\sin b\cos b&\sin b\cos c+\cos b\sin c\\
			\sin c\cos a+\cos c\sin a&\sin c\cos b+\cos c\sin b&2\sin c\cos c
		\end{vmatrix}=$\fillblank{\visible<+->{$0$}}.
	\end{exercise}
\end{frame}

\begin{frame}{例: 方阵的行列式}
	\onslide<+->
	\begin{answer}
		注意到
		\begin{align*}
			&\begin{pmatrix}
			2\sin a\cos a&\sin a\cos b+\cos a\sin b&\sin a\cos c+\cos a\sin c\\
			\sin b\cos a+\cos b\sin a&2\sin b\cos b&\sin b\cos c+\cos b\sin c\\
			\sin c\cos a+\cos c\sin a&\sin c\cos b+\cos c\sin b&2\sin c\cos c
		\end{pmatrix}\\=&\begin{pmatrix}
			\sin a&\cos a&0\\
			\sin b&\cos b&0\\
			\sin c&\cos c&0
		\end{pmatrix}\begin{pmatrix}
			\cos a&\cos b&\cos c\\
			\sin a&\sin b&\sin c\\
			0&0&0
		\end{pmatrix}.
	\end{align*}
	\end{answer}
	\onslide<+->
	设 $\bfA\in M_{m\times n},\bfB\in M_{n\times m}$.
	\onslide<+->
	若 $m>n$, 则
	\[\alert{|\bfA\bfB|}=\left|
		(\bfA,\bfO_{m\times(m-n)})\begin{pmatrix}
		\bfB\\\bfO_{(m-n)\times m}
	\end{pmatrix}\right|\alert{=0=|\bfB\bfA|}.\]
\end{frame}


\begin{frame}{例: 方阵的行列式}
	\onslide<+->
	\begin{example}
		设 $\bfA=\begin{pmatrix}
			a&-b&-c&-d\\
			b&a&-d&c\\
			c&d&a&-b\\
			d&-c&b&a
		\end{pmatrix}$, 求 $|\bfA|$.
	\end{example}
	\onslide<+->
	这题当然可以直接硬算, 不过我们可以利用一点小技巧:
	\[\bfA\bfA^\rmT=\begin{pmatrix}
		a&-b&-c&-d\\
		b&a&-d&c\\
		c&d&a&-b\\
		d&-c&b&a
	\end{pmatrix}\begin{pmatrix}
		a&b&c&d\\
		-b&a&d&-c\\
		-c&-d&a&b\\
		-d&c&-b&a
	\end{pmatrix}=(a^2+b^2+c^2+d^2)\bfE.\]
	\onslide<+->
	因此 $|\bfA|=(a^2+b^2+c^2+d^2)^2$ (因为一定有 $a^4$ 项).
\end{frame}


