\section{逆矩阵}

\subsection{方阵的伴随矩阵}

\begin{frame}{伴随矩阵}
	\onslide<+->
	\begin{definition}
		设 $\bfA=(a_{ij})_{n\times n}$.
		由 $\bfA$ 的代数余子式形成的矩阵
		\[\bfA^*=(A_{ji})=\begin{pmatrix}
			A_{11}&A_{21}&\cdots&A_{n1}\\
			A_{12}&A_{21}&\cdots&A_{n2}\\
			\vdots&\vdots&\ddots&\vdots\\
			A_{1n}&A_{21}&\cdots&A_{nn}
		\end{pmatrix}\]
		称为矩阵 $\bfA$ 的\emph{伴随矩阵}.
	\end{definition}
	\onslide<+->
	注意, 伴随矩阵的 $(i,j)$ 元是代数余子式 \alert{$A_{ji}$ 而不是 $A_{ij}$}.
	\onslide<+->
	\begin{example}
		如果 $\bfA=\begin{pmatrix}
			a&b\\c&d
		\end{pmatrix}$, 那么 $\bfA^*=\begin{pmatrix}
			d&-b\\-c&a
		\end{pmatrix}$.
	\end{example}
\end{frame}


\begin{frame}{伴随矩阵的性质}
	\onslide<+->
	伴随矩阵满足如下重要性质:
	\begin{alertblock@}
		\begin{enumerate}
		\item $\bfA\bfA^*=\bfA^*\bfA=|\bfA|E_n$.
		\end{enumerate}
	\end{alertblock@}
	\onslide<+->
	这是因为
	\[\bfA\bfA^*=\begin{pmatrix}
		a_{11}&a_{12}&\cdots &a_{1n}\\
		a_{21}&a_{22}&\cdots &a_{2n}\\
		\vdots&\vdots&\ddots&\vdots\\
		a_{n1}&a_{n2}&\cdots &a_{nn}
	\end{pmatrix}\begin{pmatrix}
		A_{11}&A_{21}&\cdots&A_{n1}\\
		A_{12}&A_{21}&\cdots&A_{n2}\\
		\vdots&\vdots&\ddots&\vdots\\
		A_{1n}&A_{21}&\cdots&A_{nn}
	\end{pmatrix}\]
	的 $(i,j)$ 元是
	\[a_{i1}A_{j1}+\cdots+a_{in}A_{jn}=\begin{cases}
		|\bfA|,&i=j;\\
		0,&i\neq j.
	\end{cases}\]
\end{frame}


\begin{frame}{伴随矩阵的性质}
	\onslide<+->
	\begin{block@}
		\begin{enumerate}
			\setcounter{enumi}{1}
			\item $(k\bfA)^*=k^{n-1}\bfA^*$.
			\item $(\bfA^\rmT)^*=(\bfA^*)^\rmT$.
			\item $|\bfA^*|=|\bfA|^{n-1}$.
		\end{enumerate}
	\end{block@}
	\onslide<+->
	如果 $\bfA=\bfO$, 显然 $\bfA^*=\bfO$.
	\onslide<+->
	如果 $|\bfA|=0$ 但 $\bfA\neq \bfO$, 那么
	\[\bfA^*\begin{pmatrix}
		a_{1j}\\a_{2j}\\\vdots\\a_{nj}
	\end{pmatrix}=\bfO_{n\times 1}.\]
	\onslide<+->
	所以以 $\bfA^*$ 为系数的齐次线性方程组有非零解, 从而 $|\bfA^*|=0$.

	\onslide<+->
	如果 $|\bfA|\neq0$, 由 $|\bfA^*|\cdot|\bfA|=\bigl||\bfA|\bfE_n\bigr|=|\bfA|^n$ 可得.
\end{frame}


\begin{frame}{例: 伴随矩阵}
	\onslide<+->
	\begin{example}
		设非零 $n\ge 3$ 实方阵 $\bfA$ 满足对任意 $i,j$, $a_{ij}=A_{ij}$. 求 $|\bfA|$.
	\end{example}
	\onslide<+->
	\begin{solution}
		由题设可知 $(\bfA^*)^\rmT=\bfA$.
		\onslide<+->{%
		因此 $|\bfA^*|=|\bfA|=|\bfA|^{n-1}$,
		}\onslide<+->{%
		从而 $|\bfA|=0$ 或 $1$.
		}
		
		\onslide<+->{如果 $|\bfA|=0$, 则
			\[\bfA\bfA^*=\bfA\bfA^\rmT=|\bfA|\bfE=\bfO.\]
		}\onslide<+->{
			而 $\bfA\bfA^\rmT$ 的第 $i$ 个对角元为
			\[\sum_{k=1}^n a_{ik}^2\ge 0.\]
		}\onslide<+->{
			于是 $\bfA$ 所有元素均为零, 矛盾! 因此 $|\bfA|=1$.
		}
	\end{solution}
\end{frame}


\subsection{逆矩阵的定义和形式}
\begin{frame}{线性变换的逆}
	\onslide<+->
	给定一个线性变换 $f:\BC^n\ra\BC^m$, 如果线性变换 $g:\BC^m\to\BC^n$ 满足
	\[(gf)(\bfu)=\bfu,\forall \bfu\in\BC^n,\quad
	(fg)(\bfv)=\bfv,\forall \bfv\in\BC^m,\]
	\onslide<+->
	则称 $g$ 是 $f$ 的逆.

	\onslide<+->
	设 $f,g$ 对应的矩阵分别是 $\bfA,\bfB$, 则
	\[\bfA\bfB=\bfE_m,\quad \bfB\bfA=\bfE_n.\]
	\onslide<+->
	注意到 $m\neq n$ 时上述等式不可能成立, 因为 $|\bfA\bfB|=|\bfB\bfA|=0$.
	\onslide<+->
	因此线性变换的逆只可能在 $m=n$ 时存在.
\end{frame}


\begin{frame}{逆矩阵的定义和唯一性}
	\onslide<+->
	由此得到对应的矩阵的逆的定义:
	\onslide<+->
	\begin{definition}
		设 $\bfA$ 是 $n$ 阶方阵. 若存在 $n$ 阶方阵 $\bfB$ 使得
		\[\bfA\bfB=\bfB\bfA=\bfE_n,\]
		则称 $\bfA$ 是\emph{可逆矩阵}, $\bfB$ 是 $\bfA$ 的\emph{逆矩阵}.
	\end{definition}
	\onslide<+->
	\begin{enumerate}
		\item 如果 $\bfA$ 是可逆矩阵, 它的逆矩阵唯一吗?
	\end{enumerate}
	\onslide<+->
	设 $\bfB,\bfB'$ 都是 $\bfA$ 的逆矩阵, 则
	\[\bfA\bfB=\bfE_n,\quad \bfB'\bfA=\bfE_n.\]
	\onslide<+->
	于是
	\[\bfB=(\bfB'\bfA)\bfB=\bfB'(\bfA\bfB)=\bfB'.\]
	\onslide<+->
	因此若逆矩阵存在必唯一.
\end{frame}


\begin{frame}{逆矩阵的存在性}
	\onslide<+->
	\begin{enumerate}
		\setcounter{enumi}{1}
		\item 任何非零矩阵都有逆吗?
	\end{enumerate}
	\onslide<+->
	设 $\begin{pmatrix}
		1&1\\2&2
	\end{pmatrix}\begin{pmatrix}
		a&b\\c&d
	\end{pmatrix}=\begin{pmatrix}
		1&0\\0&1
	\end{pmatrix}$, 
	\onslide<+->
	则 
	\[a+c=1,\quad 2a+2c=0.\]
	\onslide<+->
	这不可能, 因此 $\begin{pmatrix}
		1&1\\2&2
	\end{pmatrix}$ 不可逆.
	\onslide<+->
	实际上从 $\bfA\bfB=\bfE$ 可知
	\[|\bfA|\cdot|\bfB|=1.\]
	\onslide<+->
	所以退化矩阵都是不可逆的.
\end{frame}


\begin{frame}{逆矩阵的存在性}
	\onslide<+->
	\begin{enumerate}
		\setcounter{enumi}{2}
		\item 那非退化矩阵都可逆吗?
	\end{enumerate}
	\onslide<+->
	注意到
	\[\bfA\bfA^*=\bfA^*\bfA=|\bfA|\bfE.\]
	\onslide<+->
	因此如果 $\bfA$ 非退化, 则
	\[\bfA\cdot\frac{\bfA^*}{|\bfA|}=\frac{\bfA^*}{|\bfA|}\cdot\bfA=\bfE,\]
	\onslide<+->
	\[\bfA^{-1}=\frac{\bfA^*}{|\bfA|}.\]
\end{frame}


\begin{frame}{逆矩阵存在性的刻画}
	\onslide<+->
	\begin{theorem}
		$n$ 阶方阵 $\bfA$ 可逆当且仅当 $|\bfA|\neq 0$.
		此时 $\bfA^{-1}=\dfrac{\bfA^*}{|\bfA|}$.
	\end{theorem}
	\onslide<+->
	\begin{corollary}
		设 $\bfA,\bfB$ 为 $n$ 阶矩阵.
		如果 $\bfA\bfB=\bfE$ (或 $\bfB\bfA=\bfE$), 则 $\bfB=\bfA^{-1}$.
	\end{corollary}
	\onslide<+->
	\begin{proof}
		如果 $\bfA\bfB=\bfE$, 则 $|\bfA|\cdot|\bfB|=1$, $|\bfA|\neq0$.
		\onslide<+->{%
		因此 $\bfA$ 可逆.
		}\onslide<+->{%
		\[\bfA^{-1}=\bfA^{-1}(\bfA\bfB)=\bfB.\qedhere\]}
		\vspace{-\baselineskip}
	\end{proof}
\end{frame}


\begin{frame}{例: 计算逆矩阵}
	\onslide<+->
	\begin{example}
		证明 $\bfA=\begin{pmatrix}
			\cos\theta&-\sin\theta\\
			\sin\theta&\cos\theta
		\end{pmatrix}$ 可逆并求其逆矩阵.
	\end{example}
	\onslide<+->
	\begin{solution}
		由于 $|\bfA|=\cos^2\theta+\sin^2\theta=1$, 因此 $\bfA$ 可逆.
		\onslide<+->{由于 $\bfA^*=\begin{pmatrix}
			\cos\theta&\sin\theta\\
			-\sin\theta&\cos\theta
		\end{pmatrix}$, 因此
		\[\bfA^{-1}=\frac{\bfA^*}{|\bfA|}=\begin{pmatrix}
			\cos\theta&\sin\theta\\
			-\sin\theta&\cos\theta
		\end{pmatrix}.\]}
	\end{solution}
	\onslide<+->
	注意到 $\bfA$ 对应的是平面上沿原点逆时针旋转 $\theta$, 因此 $\bfA^{-1}$ 对应的是平面上沿原点逆时针旋转 $-\theta$.
\end{frame}



\begin{frame}{例: 计算逆矩阵}
	\onslide<+->
	\begin{example}
		$\bfA=\begin{pmatrix}
			1&2&1\\
			0&1&-1\\
			-1&3&4
		\end{pmatrix}, \bfB=\begin{pmatrix}
			2&3&-1\\
			-1&-3&5\\
			1&5&11
		\end{pmatrix}$ 是否可逆? 若可逆求其逆矩阵.
	\end{example}
	\onslide<+->
	\begin{solution}
		由于 $|\bfA|=10$, 因此 $\bfA$ 可逆.
		\onslide<+->{%
			计算其代数余子式为
			\[A_{11}=7,\ A_{12}=1,\ A_{13}=1,\ A_{21}=-5,\ A_{22}=5\]
			\[A_{23}=-5,\ A_{31}=-3,\ A_{32}=1,\ A_{33}=1.\]
		}\onslide<+->{%
			因此 $\bfA^{-1}=\dfrac{\bfA^*}{10}=\dfrac1{10}\begin{pmatrix}
				7&-5&-3\\
				1&5&1\\
				1&-5&1
			\end{pmatrix}.$
		}\onslide<+->{由于 $|\bfB|=0$, 因此 $\bfB$ 不可逆.}
	\end{solution}
\end{frame}


\begin{frame}{例: 计算逆矩阵}
	\onslide<+->
	\begin{example}
		设 $\lambda_1,\dots,\lambda_n\neq0$.
		\[\Lambda=\begin{pmatrix}
			\lambda_1&&&\\
			&\lambda_2&&\\
			&&\ddots&\\
			&&&\lambda_n\\
		\end{pmatrix}=\diag(\lambda_1,\lambda_2,\dots,\lambda_n),\]
		\onslide<+->{则
		\[\Lambda^{-1}=\begin{pmatrix}
			\lambda_1^{-1}&&&\\
			&\lambda_2^{-1}&&\\
			&&\ddots&\\
			&&&\lambda_n^{-1}\\
		\end{pmatrix}=\diag(\lambda_1^{-1},\lambda_2^{-1},\dots,\lambda_n^{-1}).\]}
	\end{example}
\end{frame}


\begin{frame}{逆矩阵的计算方法}
	\onslide<+->
	逆矩阵通常采用下述方法计算:
	\begin{enumerate}
		\item 利用公式 $\bfA^{-1}=\dfrac1{|\bfA|}\bfA^*$, 适用于 $2,3$ 阶方阵, 或用于抽象分析.
		\item 寻找方阵 $\bfB$ 使得 $\bfA\bfB=\bfE$, 适用于抽象矩阵求逆.
		\item 利用矩阵的初等变换求逆矩阵, 该方法我们会在之后的学习中接触到.
	\end{enumerate}
	\onslide<+->
	\begin{example}
		设 $\bfA$ 为 $n$ 阶矩阵且满足 $\bfA^2+3\bfA-2\bfE=\bfO$.
		求 $\bfA^{-1}$ 和 $(\bfA-\bfE)^{-1}$.
	\end{example}
	\onslide<+->
	\begin{solution}
		\begin{enumerate}
			\item 由于 $\bfA^2+3\bfA=2\bfE$, 
			\onslide<+->{%
			因此 $\bfA(\bfA+3\bfE)=2\bfE, \bfA^{-1}=\dfrac{\bfA+3\bfE}2$.}
			\item 由于 $(\bfA-\bfE)(\bfA+4\bfE)=-2\bfE$,
			\onslide<+->{%
			因此 $(\bfA-\bfE)^{-1}=-\dfrac{\bfA+4\bfE}2$.}
		\end{enumerate}
	\end{solution}
\end{frame}


\begin{frame}{方阵的多项式}
	\onslide<+->
	设
	\[f(x)=a_mx^m+\cdots+a_1x+a_0\]
	是一个多项式.
	\onslide<+->
	定义
	\[f(\bfA)=a_m\bfA^m+\cdots+a_1\bfA+a_0\bfE.\]
	\onslide<+->
	例如, 
	\begin{enumerate}
		\item 若 $\bfA=\diag(\lambda_1,\dots,\lambda_n)$, 则 $f(\bfA)=\diag\bigl(f(\lambda_1),\dots,f(\lambda_n)\bigr).$
		\item $f(\bfP\bfA\bfP^{-1})=\bfP f(\bfA)\bfP^{-1}$.
	\end{enumerate}

	\onslide<+->
	若 $f(\bfA)=\bfO$, 我们想求 $(\bfA-\alpha\bfE)^{-1}$.
	\onslide<+->
	那么
	\[f(\bfA)-f(\alpha)\bfE=(\bfA-\alpha\bfE)\bfB=-f(\alpha)\bfE.\]
	\onslide<+->
	从而当 $f(\alpha)\neq 0$ 时, $(\bfA-\alpha\bfE)^{-1}=-\dfrac{\bfB}{f(\alpha)}$.
\end{frame}


\begin{frame}{例: 计算逆矩阵}
	\onslide<+->
	想一想: 如果 $\bfA^3+\bfA^2-2\bfE=\bfO$, 如何求 $(\bfA^2+\bfE)^{-1}$?
	\onslide<+->
	\begin{enumerate}
		\item 待定系数设 $(\bfA^2+\bfE)(a\bfA^2+b\bfA+c\bfE)=\bfE$, 然后使得两边相减是 $\bfA^3+\bfA^2-2\bfE$ 的倍数.
		\item 通过 $\bfA^6=(\bfA^3)^2$ 得到 $\bfA^2$ 满足的方程.
	\end{enumerate}
	\onslide<+->
	\begin{example}
		多选题: 若 $\bfA,\bfB,\bfC$ 为同阶方阵, 且 $\bfA$ 可逆, 则\fillbrace{\visible<+->{\alert{AC}}}.
		\xx{若 $\bfA\bfB=\bfA\bfC$, 则 $\bfB=\bfC$}%
		{若 $\bfA\bfB=\bfC\bfB$, 则 $\bfA=\bfC$}%
		{若 $\bfA\bfB=\bfO$, 则 $\bfB=\bfO$}%
		{若 $\bfB\bfC=\bfO$, 则 $\bfB=\bfO$}
	\end{example}
\end{frame}


\begin{frame}{例: 计算逆矩阵}
	\onslide<+->
	\begin{example}
		设 $n$ 阶方阵 $\bfA,\bfB,\bfC$ 满足 $\bfA\bfB\bfC=\bfE$, 则\fillbrace{\visible<+->{\alert{D}}}.
		\xx{$\bfA\bfC\bfB=\bfE$}%
		{$\bfC\bfB\bfA=\bfE$}%
		{$\bfB\bfA\bfC=\bfE$}%
		{$\bfC\bfA\bfB=\bfE$}
	\end{example}
	\onslide<+->
	想一想 $\bfB^{-1}=?$
	\onslide<+->
	\begin{exercise}
		设 $\bfA=\begin{pmatrix}
			4&1&3\\4&a&7\\3&-1&4
		\end{pmatrix}$, 且存在两个不等的 $3\times2$ 矩阵 $\bfB,\bfC$ 使得 $\bfA\bfB=\bfA\bfC$, 则 $a=$\fillblank{\visible<+->{\alert{$-3$}}}.
	\end{exercise}
	\onslide<+->
	\begin{exercise}
		设 $3$ 阶方阵 $\bfA$ 满足 $\bfA^3-2\bfA+\bfE=\bfO$, 且 $|\bfA|=2$, 则 $|(\bfA^2-2\bfE)^{-1}|=$\fillblank{\visible<+->{\alert{$-2$}}}.
	\end{exercise}
\end{frame}


\subsection{逆矩阵的性质}

\begin{frame}{逆矩阵的性质}
	\onslide<+->
	逆矩阵满足如下性质:
	\begin{enumerate}
		\item 设 $\bfA$ 可逆.
		\begin{itemize}
			\item $\bfA^{-1}$ 也可逆, 且 $(\bfA^{-1})^{-1}=\bfA$;
			\item 若 $\lambda\neq 0$, 则 $\lambda\bfA$ 也可逆, 且 $(\lambda\bfA)^{-1}=\dfrac1\lambda\bfA^{-1}$;
			\item $\bfA^\rmT$ 也可逆, 且 $(\bfA^\rmT)^{-1}=(\bfA^{-1})^\rmT$;
			\item $|\bfA^{-1}|=|\bfA|^{-1}$.
		\end{itemize}
		\item 若 $\bfA,\bfB$ 为同阶可逆矩阵, 则 $\bfA\bfB$ 也可逆, 且
		\[\alert{(\bfA\bfB)^{-1}=\bfB^{-1}\bfA^{-1}}.\]
		一般地
		\[\alert{(\bfA_1\bfA_2\cdots\bfA_n)^{-1}=\bfA_n^{-1}\cdots\bfA_2^{-1}\bfA_1^{-1}}.\]
	\end{enumerate}
	\onslide<+->
	注意矩阵不能相除 $\dfrac{\bfA}{\bfB}$, 因为一般 $\bfB^{-1}\bfA\neq\bfA\bfB^{-1}$.
\end{frame}


\begin{frame}{伴随矩阵和逆矩阵}
	\onslide<+->
	注意一般地, $(\bfA+\bfB)^{-1}\neq\bfA^{-1}+\bfB^{-1}$.
	\onslide<+->
	例如 $\bfA=\begin{pmatrix}
		1&0\\0&-1
	\end{pmatrix},\bfB=\begin{pmatrix}
		1&0\\0&1
	\end{pmatrix}$ 均可逆, 但 $\bfA+\bfB=\begin{pmatrix}
		2&0\\0&0
	\end{pmatrix}$ 不可逆.
	\onslide<+->
	\begin{block}{伴随矩阵的性质}
		\begin{enumerate}
			\setcounter{enumi}{4}
			\item 若 $\bfA$ 可逆, 则 $(\bfA^{-1})^*=(\bfA^*)^{-1}=\dfrac{\bfA}{|\bfA|}$.
		\end{enumerate}
	\end{block}
	\onslide<+->
	\begin{proof}
		由于 $\bfA\bfA^*=|\bfA|\bfE$, 因此 $\bfA^*=|\bfA|\bfA^{-1}$.
		\onslide<+->{于是
		\[(\bfA^{-1})^*=|\bfA^{-1}|(\bfA^{-1})^{-1}=\frac{\bfA}{|\bfA|},\quad\visible<+->{(\bfA^*)^{-1}=(|\bfA|\bfA^{-1})^{-1}=\frac{\bfA}{|\bfA|}.}\qedhere\]
		}
		\vspace{-\baselineskip}
	\end{proof}
\end{frame}


\begin{frame}{例: 求逆矩阵}
	\onslide<+->
	\begin{example}
		设 $\bfA$ 是 $3$ 阶方阵, $|\bfA|=\dfrac12$.
		求 $\bigl|(2\bfA)^{-1}-(2\bfA)^*\bigr|$.
	\end{example}
	\onslide<+->
	\begin{solution}
		\[(2\bfA)^{-1}-(2\bfA)^*
		=\frac12\bfA^{-1}-2^2\bfA^*=\bfA^*-4\bfA^*=-3\bfA^*,\]
		\onslide<+->{因此
		\[\bigl|(2\bfA)^{-1}-(2\bfA)^*\bigr|=-27|\bfA^*|=-27|\bfA|^2=-\frac{27}4.\]
		}
		\vspace{-\baselineskip}
	\end{solution}
\end{frame}


\begin{frame}{例: 解矩阵方程}
	\onslide<+->
	如果 $\bfA,\bfB$ 可逆, 下述矩阵方程可以由逆矩阵表出:
	\begin{enumerate}
		\item $\bfA\bfX=\bfC\implies \bfX=\bfA^{-1}\bfC$;
		\item $\bfX\bfA=\bfC\implies \bfX=\bfC\bfA^{-1}$;
		\item $\bfA\bfX\bfB=\bfC\implies \bfX=\bfA^{-1}\bfC\bfB^{-1}$.
	\end{enumerate}
	\onslide<+->
	\begin{example}
		设 $\bfA=\begin{pmatrix}
			3&0&1\\1&1&0\\0&1&4
		\end{pmatrix}$. 如果 $\bfA\bfX=\bfA+2\bfX$, 求 $\bfX$.
	\end{example}
\end{frame}


\begin{frame}{例: 解矩阵方程}
	\onslide<+->
	\begin{solution}
		由题设得 $(\bfA-2\bfE)\bfX=\bfA$.
		\onslide<+->{
			注意到
			\[|\bfA-2\bfE|=\detm{
				1&0&1\\1&-1&0\\0&1&2}=-1\neq0,\quad
				\visible<+->{(\bfA-2\bfE)^{-1}=\begin{pmatrix}
				2&-1&-1\\
				2&-2&-1\\
				-1&1&1
			\end{pmatrix}.}\]
		}\onslide<+->{因此
		\[\bfX=(\bfA-2\bfE)^{-1}\bfA=\begin{pmatrix}
			2&-1&-1\\
			2&-2&-1\\
			-1&1&1
		\end{pmatrix}\begin{pmatrix}
			3&0&1\\1&1&0\\0&1&4
		\end{pmatrix}=\begin{pmatrix}
			5&-2&-2\\
			4&-3&-2\\
			-2&2&3
		\end{pmatrix}.\]}
		\vspace{-\baselineskip}
	\end{solution}
	\onslide<+->
	也可由 $\bfX=\bfE+2(\bfA-2\bfE)^{-1}$ 计算得到.
\end{frame}


\begin{frame}{例: 解矩阵方程}
	\onslide<+->
	\begin{exercise}
		设 $3$ 阶矩阵 $\bfA,\bfB$ 满足 $\bfA^{-1}\bfB\bfA=6\bfA+\bfB\bfA$.
		如果 $\bfA=\diag(\dfrac12,\dfrac14,\dfrac17)$, 求 $\bfB$.
	\end{exercise}
	\onslide<+->
	\begin{answer}
		右乘 $\bfA^{-1}$ 得到 $\bfB=6(\bfA^{-1}-\bfE)^{-1}=\diag(6,2,1)$.

		\onslide<+->{也可以左乘 $\bfA$ 右乘 $\bfA^{-1}$ 得到
		\[(\bfE-\bfA)\bfB=6\bfA=6\bfE-6(\bfE-\bfA),\]
		\[\bfB=6(\bfE-\bfA)^{-1}-6\bfE=\diag(6,2,1).\]}
	\end{answer}
\end{frame}


\begin{frame}{例: 解矩阵方程}
	\onslide<+->
	\begin{example}
		解矩阵方程 $\bfA^*\bfX=\bfA^{-1}+2\bfX$, 其中 $\bfA=\begin{pmatrix}
			1&1&-1\\-1&1&1\\1&-1&1
		\end{pmatrix}$.
	\end{example}
	\onslide<+->
	\begin{solution}
		注意到 $|\bfA|=4$.
		\onslide<+->{两边同时左乘 $\bfA$ 得到 $4\bfX=\bfE+2\bfA\bfX$,
		}\onslide<+->{因此
		\[\bfX=(4\bfE-2\bfA)^{-1}=\begin{pmatrix}
			2&-2&2\\2&2&-2\\-2&2&2
		\end{pmatrix}^{-1}=\frac14\begin{pmatrix}
			1&1&0\\0&1&1\\1&0&1
		\end{pmatrix}.\]}
		\vspace{-\baselineskip}
	\end{solution}
\end{frame}


\begin{frame}{例: 解矩阵方程}
	\onslide<+->
	\begin{exercise}
		解矩阵方程 $\bfA^*\bfX\bfA=2\bfX\bfA-8\bfE$, 其中 $\bfA=\begin{pmatrix}
			1&2&-2\\0&-2&4\\0&0&1
		\end{pmatrix}$.
	\end{exercise}
	\onslide<+->
	\begin{answer}
		两边同时左乘 $\bfA$ 右乘 $\bfA^{-1}$ 得到
		\[-2\bfX=2\bfA\bfX-8\bfE,\quad (\bfA+\bfE)\bfX=4\bfE,\]
		\[\bfX=4(\bfA+\bfE)^{-1}=4\begin{pmatrix}
			2&2&-2\\0&-1&4\\0&0&2
		\end{pmatrix}^{-1}=4\begin{pmatrix}
			2&4&-6\\0&-4&8\\0&0&2
		\end{pmatrix}.\]
	\end{answer}
\end{frame}


\begin{frame}{例: 解矩阵方程}
	\onslide<+->
	\begin{example}
		设 $\bfP=\begin{pmatrix}
			1&2\\1&4
		\end{pmatrix},\Lambda=\begin{pmatrix}
			1&\\&2
		\end{pmatrix},\bfA\bfP=\bfP\Lambda$, 求 $\bfA^n$.
	\end{example}
	\onslide<+->
	\begin{solution}
		$|\bfP|=2,\bfP^{-1}=\dfrac12\begin{pmatrix}
			4&-1\\-1&1
		\end{pmatrix},\bfA=\bfP\Lambda\bfP^{-1}$,
		\onslide<+->{
			\begin{align*}
				\bfA^n&=\bfP\Lambda\bfP^{-1}\cdot \bfP\Lambda\bfP^{-1}\cdots \bfP\Lambda\bfP^{-1}=\bfP\Lambda^n\bfP^{-1}\\
				&\visible<+->{=\dfrac12\begin{pmatrix}
					1&2\\1&4
				\end{pmatrix}\begin{pmatrix}
					1&\\&2^n
				\end{pmatrix}\begin{pmatrix}
					4&-1\\-1&1
				\end{pmatrix}=\begin{pmatrix}
					2-2^n&2^n-1\\2-2^{n+1}&2^{n+1}-1
				\end{pmatrix}.}
			\end{align*}}
			\vspace{-\baselineskip}
	\end{solution}
	\onslide<+->
	我们会在后面学习到该技巧来计算一般方阵的幂次.
\end{frame}


\begin{frame}{逆矩阵和克拉默法则}
	\onslide<+->
	如果 $\bfA\bfx=\bfb$ 且 $\bfA$ 可逆, 
	\onslide<+->
	则 
		\[\bfx=\bfA^{-1}\bfb=\frac1{|\bfA|}\bfA^*\bfb\]
	\onslide<+->
	因此
	\[x_i=\frac1{|\bfA|}\sum_{k=1}^n A_{ki}b_k.\]
	\onslide<+->
	此即克拉默法则.
\end{frame}


\begin{frame}{伴随矩阵的伴随}
	\onslide<+->
	容易知道, $2$ 阶方阵满足 $(\bfA^*)^*=\bfA$.

	\onslide<+->
	如果 $\bfA$ 是 $n\ge3$ 阶非退化方阵, 则
	\[(\bfA^*)^*=|\bfA^*|(\bfA^*)^{-1}=|\bfA|^{n-1}\dfrac{\bfA}{|\bfA|}=|\bfA|^{n-2}\bfA.\]
	\onslide<+->
	如果 $\bfA$ 是 $n\ge3$ 阶退化方阵, 我们会在后面证明 $(\bfA^*)^*=\bfO$.
	\onslide<+->
	因此
	\begin{block}{伴随矩阵的性质}
		\begin{enumerate}
			\setcounter{enumi}{5}
			\item $(\bfA^*)^*=\begin{cases}
				\bfA,&n=2;\\
				|\bfA|^{n-2}\bfA,&n\ge3.
			\end{cases}$
		\end{enumerate}
	\end{block}
\end{frame}


\begin{frame}{例: 逆矩阵的性质}
	\onslide<+->
	\begin{exercise}
		单选题: 设 $\bfA$ 是 $n$ 阶方阵, 如果\fillbrace{\visible<+->{\alert{D}}}, 则 $\bfA-\bfE$ 可逆.
		\xx{$\bfA$ 可逆}{$|\bfA|=0$}{$\bfA$ 的主对角线元素均为 $0$}{存在某个正整数 $m$ 使得 $\bfA^m=\bfO$}
	\end{exercise}
	\onslide<+->
	\begin{exercise}
		若 $\bfA$ 为 $n$ 阶方阵, 则下面命题正确的有\fillblank{\visible<+->{\alert{1}}}个.
		\begin{enumerate}
			\item $\bfA^{-1}=\dfrac1{|\bfA|}\bfA^*$
			\item $\bfA^*=|\bfA|\bfA^{-1}$
			\item $|\bfA^*|=|\bfA|^{n-1}$
		\end{enumerate}
	\end{exercise}
\end{frame}


\subsection{逆矩阵的应用}

\begin{frame}{逆矩阵的应用: 通信加密}
	\onslide<+->
	1929年, 希尔通过线性变换对信息进行加密和解密处理, 
	提出了密码史上具有重要地位的\emph{希尔密码系统}.
	\onslide<+->
	\begin{center}
		\begin{tikzpicture}
			\node[cstnodeg,align=center,minimum height=12mm] (1) at (0,0) {明文字母};
			\node[cstnodeg,align=center,minimum height=12mm] (2) at (3.6,0) {明文数字\\矩阵 $\bfX$};
			\node[cstnodeg,align=center,minimum height=12mm] (3) at (8.4,0) {密文数字\\矩阵 $\bfC$};
			\node[cstnodeg,align=center,minimum height=12mm] (4) at (12,0) {密文字母};
			\node[cstnodeb,align=center,minimum height=12mm] (11) at (0,-2) {明文字母};
			\node[cstnodeb,align=center,minimum height=12mm] (12) at (3.6,-2) {明文数字\\矩阵 $\bfX$};
			\node[cstnodeb,align=center,minimum height=12mm] (13) at (8.4,-2) {密文数字\\矩阵 $\bfC$};
			\node[cstnodeb,align=center,minimum height=12mm] (14) at (12,-2) {密文字母};
			\draw[cstarrowto,dcolord,cstcurve] (1.east) to (2.west);
			\draw[cstarrowto,dcolord,cstcurve] (2.east) to (3.west);
			\draw[cstarrowto,dcolord,cstcurve] (3.east) to (4.west);
			\draw[cstarrowto,dcolora,cstcurve] (4.south) to (14.north);
			\draw[cstarrowto,dcolorb,cstcurve] (14.west) to (13.east);
			\draw[cstarrowto,dcolorb,cstcurve] (13.west) to (12.east);
			\draw[cstarrowto,dcolorb,cstcurve] (12.west) to (11.east);
			\draw (1.8,.3) node {查表};
			\draw (1.8,-2.3) node {查表};
			\draw (1.8,-1) node[align=center,dcolore] {$A\sim Z$\\对应\\$0\sim26$};
			\draw (6,.3) node {$\bfC=\bfA\bfX$};
			\draw (6,-.3) node[align=center,dcolord] {加密秘钥: $A$};
			\draw (6,-2.3) node {$\bfX=\bfA^{-1}\bfC$};
			\draw (6,-1.7) node[align=center,dcolorb] {解密秘钥: $A^{-1}$};
			\draw (10.2,.3) node {查表};
			\draw (10.2,-2.3) node {查表};
			\draw (11.2,-1) node[dcolora] {传输};
		\end{tikzpicture}
	\end{center}
\end{frame}


\begin{frame}{逆矩阵的应用: 通信加密}
	\beqskip{2mm}
	\onslide<+->
	\begin{example}
		设接受收到的密文字母为``WBIZTNWJBRFSGNZ'', 加密密钥为 $\bfA=\begin{pmatrix}
			1&2&3\\1&1&2\\0&1&2
		\end{pmatrix}$. 
		请用希尔密码系统解密密文.
	\end{example}
	\onslide<+->
	\begin{solution}
		密文对应的数字为 
		\[22,1,8,25,19,13,22,9,1,17,5,18,6,13,25.\]
		\onslide<+->{由于密钥是 $3$ 阶方阵, 所以将上述数字按 $3$ 个一列写成
		\[\bfC=\begin{pmatrix}
			22&25&22&17&6\\
			1&19&9&5&13\\
			8&13&1&18&25
		\end{pmatrix}.\]}
		\vspace{-\baselineskip}
	\end{solution}
	\endgroup
\end{frame}
	
	
\begin{frame}{逆矩阵的应用: 通信加密}
	\begin{solutionc}
		\onslide<+->{解密密钥为
		$\bfA^{-1}=\begin{pmatrix}
			0&1&-1\\2&-2&-1\\-1&1&1
		\end{pmatrix}$,
		}\onslide<+->{%
		因此
		\[\bfA^{-1}\bfC=\begin{pmatrix}
			-7&6&8&-13&-12\\
			34&-1&25&6&-39\\
			-13&7&-12&6&32
		\end{pmatrix},\quad
		\bfX=\begin{pmatrix}
			19&6&8&13&14\\
			8&25&25&6&13\\
			13&7&14&6&6
		\end{pmatrix}.\]}\onslide<+->{%
		查表得到明文: \alert{TINGZHI ZONGGONG}.}
	\end{solutionc}
\end{frame}

