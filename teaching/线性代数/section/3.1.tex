\section{方阵的相似}

\subsection{特征值与特征向量}

\begin{frame}{特征值与特征向量的定义}
	\onslide<+->
	设 $f:\BC^n\to \BC^n$ 是一个线性映射.
	\onslide<+->
	由于 $\BC^n$ 中的向量由一组基
	\[\bma_1,\dots,\bma_n\]
	唯一线性表示, 因此它完全由它在该组基下的像决定.
	\onslide<+->
	若 $f$ 将每个 $\bma_i$ 映射为它的倍数, 则 $f$ 将会变得很容易研究.

	\onslide<+->
	\begin{definition}
		若常数 $\lambda$ 和非零向量 $\bfx$ 满足 $\bfA\bfx=\lambda\bfx$, 称 $\lambda$ 为 $\bfA$ 的\emph{特征值}, $\bfx$ 为 $\bfA$ 关于 $\lambda$ 的\emph{特征向量}.
	\end{definition}
\end{frame}


\begin{frame}{特征多项式}
	\onslide<+->
	设 $\bfA\bfx=\lambda\bfx$, 则
	\[(\bfA-\lambda\bfE)\bfx={\bf0}.\]
	\onslide<+->
	该方程有非零解当且仅当
	\[|\bfA-\lambda\bfE|=\begin{vmatrix}
		a_{11}-\lambda&a_{12}&\cdots&a_{1n}\\
		a_{21}&a_{22}-\lambda&\cdots&a_{2n}\\
		\vdots&\vdots&\ddots&\vdots\\
		a_{n1}&a_{n2}&\cdots&a_{nn}-\lambda
	\end{vmatrix}=0.\]
	\onslide<+->
	注意到该行列式是 $\lambda$ 的 $n$ 次多项式, 最高项为 $(-1)^n\lambda^n$.
	\onslide<+->
	称之为 $\bfA$ 的\emph{特征多项式}.
\end{frame}


\begin{frame}{特征值和特征向量的性质}
	\onslide<+->
	在复数域中 $n$ 次多项式总有 $n$ 个根(计算重数),
	\onslide<+->
	也就是说 $\bfA$ 的特征多项式可以写成
	\[f(\lambda)=\pm(\lambda-\lambda_1)\cdots(\lambda-\lambda_n).\]
	\onslide<+->
	所以 $\bfA$ 的\alert{特征值有 $n$ 个(计算重数)}.

	\onslide<+->
	特征值和特征向量只对方阵存在, 且有如下性质:
	\begin{enumerate}
		\item 零向量不是特征向量;
		\item 若 $\bfx$ 是对应 $\lambda$ 的特征向量, 则它的非零倍也是;
		\item 若 $\bfx_1,\bfx_2\neq-\bfx_1$ 是对应 $\lambda$ 的特征向量, 则 $\bfx_1+\bfx_2$ 也是;
		\item $|\bfA|=0\iff 0$ 是特征值; $|\bfA|\neq0\iff 0$ 不是特征值;
		\item 若 $n$ 阶方阵 $\bfA$ 的各行元素之和为 $k$, 则 $k$ 是 $\bfA$ 	的一个特征值, 且特征向量为 $(1,\dots,1)^\rmT$.
	\end{enumerate}
\end{frame}


\begin{frame}{特征值和特征向量的计算步骤}
	\onslide<+->
	特征值和特征向量的计算步骤如下:
	\begin{enumerate}
		\item 求 $\bfA$ 的特征多项式 $f(\lambda)=|\bfA-\lambda \bfE|$;
		\item 解 $f(\lambda)=|\bfA-\lambda \bfE|=0$ 得到特征值;
		\item 对于每一个特征值 $\lambda_i$, 解 $(\bfA-\lambda_i)\bfx={\bf0}$, 其\alert{非零解}就是对应特征向量.
	\end{enumerate}
\end{frame}


\begin{frame}{典型例题: 求特征值和特征向量}\small
\beqskip{0pt}
	\onslide<+->
	\begin{example}
		求 $\bfA=\begin{pmatrix}
			1&3\\
			4&2
		\end{pmatrix}$ 的特征值和特征向量.
	\end{example}
	\onslide<+->
	\begin{solution}
		\begin{enumerate}
			\item 特征多项式 $|\bfA-\lambda\bfE|=\begin{vmatrix}
				1-\lambda&3\\
				4&2-\lambda
			\end{vmatrix}=\lambda^2-3\lambda-10$.
			\item 由 $\lambda^2-3\lambda-10=0$ 解得特征值 $\lambda_1=5,\lambda_2=-2$.
			\item 对于 $\lambda_1=5$, $\bfA-5\bfE=\begin{pmatrix}
				-4&3\\
				4&-3
			\end{pmatrix}\simr\begin{pmatrix}
				1&-3/4\\
				0&0
			\end{pmatrix}$, 得到基础解系 $\begin{pmatrix}
				3/4\\1
			\end{pmatrix}$.
			\onslide<+->{%
				故对应的所有特征向量为 $k(3,4)^\rmT,k\neq0$.
			}
			\item 对于 $\lambda_2=-2$, $\bfA+2\bfE=\begin{pmatrix}
				3&3\\
				4&4
			\end{pmatrix}\simr\begin{pmatrix}
				1&1\\
				0&0
			\end{pmatrix}$, 得到基础解系 $\begin{pmatrix}
				1\\-1
			\end{pmatrix}$.
			\onslide<+->{%
				故对应的所有特征向量为 $k(1,-1)^\rmT,k\neq0$.
			}
		\end{enumerate}
	\end{solution}
\endgroup
\end{frame}


\begin{frame}{例: 求特征值和特征向量}
	\onslide<+->
	\begin{example}
		求上三角阵 $\bfA=\begin{pmatrix}
			a_1&b&c\\
			0&a_2&d\\
			0&0&a_3
		\end{pmatrix}$ 的特征值.
	\end{example}
	\onslide<+->
	\begin{solution}
		特征多项式
		\[|\bfA-\lambda\bfE|=\begin{vmatrix}
			a_1-\lambda&b&c\\
			0&a_2-\lambda&d\\
			0&0&a_3-\lambda
			\end{vmatrix}=(a_1-\lambda)(a_2-\lambda)(a_3-\lambda).\]
		\onslide<+->{%
			因此特征值为 $a_1,a_2,a_3$.
		}
	\end{solution}
	\onslide<+->
	上三角阵、下三角阵、对角阵的特征值就是对角元.
\end{frame}


\begin{frame}{典型例题: 求特征值和特征向量}
	\onslide<+->
	\begin{example}
		求 $\bfA=\begin{pmatrix}
			-1&1&0\\
			-4&3&0\\
			1&0&2
		\end{pmatrix}$ 的特征值和特征向量.
	\end{example}
	\onslide<+->
	\begin{solution}
		由特征多项式
		\[|\bfA-\lambda\bfE|=\begin{vmatrix}
			-1-\lambda&1&0\\
			-4&3-\lambda&0\\
			1&0&2-\lambda
		\end{vmatrix}=(2-\lambda)\begin{vmatrix}
			-1-\lambda&1\\
			-4&3-\lambda
		\end{vmatrix}=(2-\lambda)(\lambda-1)^2=0\]
		得到特征值 $\lambda=1,1,2$.
	\end{solution}
\end{frame}


\begin{frame}{典型例题: 求特征值和特征向量}
	\begin{solution*}[续解]
		对于 $\lambda_1=1$,
		\[\bfA-\bfE=\begin{pmatrix}
			-2&1&0\\
			-4&2&0\\
			1&0&1
		\end{pmatrix}\simr\begin{pmatrix}
			1&0&1\\
			0&1&2\\
			0&0&0
		\end{pmatrix},\]
		得到基础解系 $(-1,-2,1)^\rmT$.
		\onslide<+->{%
			故对应的所有特征向量为 $k(-1,-2,1)^\rmT,k\neq0$.
		}

		\onslide<+->{
			对于 $\lambda_2=2$,
			\[\bfA-2\bfE=\begin{pmatrix}
				-3&1&0\\
				-4&1&0\\
				1&0&0
			\end{pmatrix}\simr\begin{pmatrix}
				1&0&0\\
				0&1&0\\
				0&0&0
			\end{pmatrix},\]
			得到基础解系 $(0,0,1)^\rmT$.
		}\onslide<+->{%
			故对应的所有特征向量为 $k(0,0,1)^\rmT,k\neq0$.
		}
	\end{solution*}
\end{frame}


\begin{frame}{典型例题: 求特征值和特征向量}
	\onslide<+->
	\begin{exercise}
		求 $\bfA=\begin{pmatrix}
			-2&1&1\\
			0&2&0\\
			-4&1&3
		\end{pmatrix}$ 的特征值和特征向量.
	\end{exercise}
	\onslide<+->
	\begin{answer}
		\begin{enumerate}
			\item 特征值 $\lambda=-1,2,2$.
			\item $-1$ 对应的所有特征向量为 $k(1,0,1)^\rmT,k\neq0$.
			\item $2$ 对应的所有特征向量为 $k_1(0,1,-1)^\rmT+k_2(1,0,4)^\rmT,k_1,k_2$ 不全为零.
		\end{enumerate}
	\end{answer}
	\onslide<+->
	若 $\lambda$ 是 $k$ 重特征值, 则它对应的\alert{线性无关的特征向量最多 $k$ 个}.
\end{frame}


\subsection{特征值和特征向量的性质}

\begin{frame}{特征值和特征向量的性质}
	\onslide<+->
	通过
	\[|\bfA-\lambda\bfE|=\begin{vmatrix}
		a_{11}-\lambda&a_{12}&\cdots&a_{1n}\\
		a_{21}&a_{22}-\lambda&\cdots&a_{2n}\\
		\vdots&\vdots&\ddots&\vdots\\
		a_{n1}&a_{n2}&\cdots&a_{nn}-\lambda
	\end{vmatrix}\]
	的展开可以看出,
	\onslide<+->
	若 $i\neq j$, 则 $i$ 行 $j$ 列的代数余子式中最多只会出现 $\lambda^{n-2}$ 项.
	\onslide<+->
	所以
	\begin{align*}
		|\bfA-\lambda\bfE|&=(a_{11}-\lambda)(a_{22}-\lambda)\cdots(a_{nn}-\lambda)+\text{至多 $n-2$ 次项}\\
		&=(-1)^n\bigl(\lambda^n-(a_{11}+a_{22}+\cdots+a_{nn})\lambda^{n-1}\bigr)+\text{至多 $n-2$ 次项}.
	\end{align*}
	\onslide<+->
	根据韦达定理, 特征值之和为 $a_{11}+a_{22}+\cdots+a_{nn}$.

	\onslide<+->
	$\lambda=0$ 时, 特征多项式
	\[f(\lambda)=(\lambda_1-\lambda)\cdots(\lambda_n-\lambda)\]
	的取值 $|\bfA|$ 就是特征值的乘积.
\end{frame}


\begin{frame}{特征值和特征向量的性质}
	\onslide<+->
	\begin{second}{特征值和特征向量的性质}
		定义 $\bfA$ 的\emph{迹}为对角元之和:
		\[\Tr(\bfA)=a_{11}+a_{22}+\cdots+a_{nn}.\]
		\vspace{-\baselineskip}
		\begin{enumerate}
			\item 特征值之和等于迹: $\lambda_1+\lambda_2+\cdots+\lambda_n=\Tr(\bfA)$;
			\item 特征值之积等于行列式: $\lambda_1\lambda_2\cdots\lambda_n=|\bfA|$.
		\end{enumerate}
	\end{second}
	\onslide<+->
	\begin{example}
		$\begin{pmatrix}
			1&1&0\\
			1&0&1\\
			0&1&1
		\end{pmatrix}$ 的特征值为\fillbrace{\visible<+->{C}}.
		\begin{taskschoice}(4)
			() $1,0,1$
			() $1,1,2$
			() $-1,1,2$
			() $-1,1,1$
		\end{taskschoice}
	\end{example}
\end{frame}


\begin{frame}{特征值和特征向量的性质}
	% \NiceMatrixOptions{cell-space-limits = 5pt}
	\onslide<+->
	\begin{theorem}
		若 $\lambda$ 是 $\bfA$ 的特征值, $\bfx$ 是对应特征向量, 则下述矩阵有如下对应的特征值与特征向量:
		\begin{center}
			\begin{tabular}{cccccccc} \toprule
				方阵 & $k\bfA$ & $\bfA^m$ & $\bfA^{-1}$ & $\bfA^*$ & $g(\bfA)h(\bfA)^{-1}$ & $\bfA^\rmT$ & $\bfP^{-1}\bfA\bfP$\\ \midrule
				特征值&$k\lambda$&$\lambda^m$&$\lambda^{-1}$&$|\bfA|/\lambda$&$g(\lambda)/h(\lambda)$&$\lambda$&$\lambda$\\ \midrule
				对应特征向量&$\bfx$&$\bfx$&$\bfx$&$\bfx$&$\bfx$&\alert{未必是 $\bfx$}&\alert{$\bfP^{-1}\bfx$}\\
				\bottomrule
			\end{tabular}
		\end{center}
		% \[\begin{NiceMatrix}[hvlines]
		% 	\CodeBefore
		% 	\rectanglecolor{red!15}{1-1}{1-8}
		% 	\rectanglecolor{green!15}{2-1}{2-8}
		% 	\rectanglecolor{blue!15}{3-1}{3-8}
		% 	\Body
		% 	\text{方阵}&k\bfA&\bfA^m&\bfA^{-1}&\bfA^*&g(\bfA)h(\bfA)^{-1}&\bfA^\rmT&\bfP^{-1}\bfA\bfP\\
		% 	\text{特征值}&k\lambda&\lambda^m&\lambda^{-1}&|\bfA|/\lambda&g(\lambda)/h(\lambda)&\lambda&\lambda\\
		% 	\text{对应特征向量}&\bfx&\bfx&\bfx&\bfx&\bfx&\alert{\text{未必是}\ \bfx}&\alert{\bfP^{-1}\bfx}
		% \end{NiceMatrix}\]
		这里 $g,h$ 是多项式, 且满足 $h(\bfA)$ 可逆.
	\end{theorem}
	\onslide<+->
	由此可知, 若 $g(\bfA)=\bfO$, 则 $\bfA$ 的所有特征值 $\lambda$ 均满足 $g(\lambda)=0$.
\end{frame}


\begin{frame}{例: 特征值和特征向量的性质}
	\onslide<+->
	\begin{example}
		设 $3$ 阶方阵 $\bfA$ 的特征值为 $2,2,3$, 则
		\begin{enumerate}
			\item $\bfA^2$ 的特征值为\fillblank[2cm]{\visible<+->{$4,4,9$}};
			\item $\bfA^2-2\bfA+\bfE$ 的特征值为\fillblank[2cm]{\visible<+->{$1,1,4$}};
			\item $|\bfA^2-2\bfA+\bfE|=$\fillblank{\visible<+->{$4$}};
			\item $\bfA^{-1}$ 的特征值为\fillblank[3cm]{\visible<+->{$1/2,1/2,1/3$}};
			\item $\bfA^*$ 的特征值为\fillblank[2cm]{\visible<+->{$4,6,6$}};
			\item $A_{11}+A_{22}+A_{33}=$\fillblank{\visible<+->{$16$}}, 其中 $A_{ij}$ 表示 $\bfA$ 的代数余子式.
		\end{enumerate}
	\end{example}
\end{frame}


\begin{frame}{例: 特征值和特征向量的性质}
	\onslide<+->
	\begin{example}
		设 $3$ 阶方阵 $\bfA$ 的特征值为 $1,-1,2$.
		求 $|\bfA^*+3\bfA-2\bfE|$.
	\end{example}
	\onslide<+->
	\begin{solution}
		由于 $|\bfA|=-2$, $\bfA^*=-2\bfA^{-1}$.
		\onslide<+->{%
			于是
			\[\bfA^*+3\bfA-2\bfE=-2\bfA^{-1}+3\bfA-2\bfE\]
			的特征值为 $-1,-3,3$,
		}\onslide<+->{%
			$|\bfA^*+3\bfA-2\bfE|=(-1)\times(-3)\times3=9$.
		}
	\end{solution}
	\onslide<+->
	\begin{exercise}
		若 $4$ 阶方阵 $\bfA$ 的特征值为 $1,2,-2,0$, 则下列矩阵可逆的是\fillbrace{\visible<+->{C}}.
		\begin{taskschoice}(4)
			() $\bfA$
			() $\bfA-2\bfE$
			() $\bfA+\bfE$
			() $\bfA-\bfE$
		\end{taskschoice}
	\end{exercise}
\end{frame}


\begin{frame}{例: 特征值和特征向量的性质}
	\onslide<+->
	\begin{example}
		若 $\bma=(1,k,1)^\rmT$ 为 $\bfA=\begin{pmatrix}
			2&1&1\\
			1&2&1\\
			1&1&2
		\end{pmatrix}$ 的逆 $\bfA^{-1}$ 的特征向量, 求 $k$.
	\end{example}
	\onslide<+->
	\begin{solution}
		$\bma$ 也是 $\bfA$ 的特征向量.
		\onslide<+->{%
			\[\bfA\bma=\begin{pmatrix}
				k+3\\2k+2\\k+3
			\end{pmatrix}=\lambda\begin{pmatrix}
				1\\k\\1
			\end{pmatrix},\]
		}\onslide<+->{%
			因此 $\lambda=k+3, 2k+2=(k+3)k,k^2+k-2=0,k=-2$ 或 $1$.
		}
	\end{solution}
\end{frame}


\begin{frame}{特征值和特征向量的性质}
\beqskip{0pt}
	\onslide<+->
	\begin{theorem}
		若 $\lambda_1,\dots,\lambda_m$ 是 $\bfA$ 的 $m$ 个两两不同的特征值, 则其对应的特征向量 $\bma_1,\dots,\bma_m$ 线性无关.
	\end{theorem}
	\onslide<+->
	即\alert{对应于不同特征值的特征向量线性无关}.
	\onslide<+->
	\begin{proof}
		设 $k_1\bma_1+\cdots+k_m\bma_m={\bf0}$.
		\onslide<+->{%
			左乘 $\bfA^k$ 得到
			$k_1\lambda_1^k\bma_1+\cdots+k_m\lambda_m^k\bma_m={\bf0}$.
		}\onslide<+->{%
			令 $k=1,2,\dots,m-1$, 我们得到
			\[(k_1\bma_1,\cdots,k_m\bma_m)\begin{pmatrix}
				1&\lambda_1&\cdots&\lambda_1^{m-1}\\
				1&\lambda_2&\cdots&\lambda_2^{m-1}\\
				\vdots&\vdots&\ddots&\vdots\\
				1&\lambda_m&\cdots&\lambda_m^{m-1}
			\end{pmatrix}=\bfO.\]
		}\onslide<+->{%
			注意到第二个方阵的行列式是范德蒙行列式, 当 $\lambda_1,\dots,\lambda_m$ 两两不同时它非零,
			}\onslide<+->{%
			从而
			\[(k_1\bma_1,\cdots,k_m\bma_m)=\bfO,\qquad
			k_1=\cdots=k_m=0.\qedhere\]
		}
		\vspace{-\baselineskip}
	\end{proof}
\endgroup
\end{frame}


\begin{frame}{特征值和特征向量的性质}
	\onslide<+->
	设 $\lambda_1$ 对应线性无关的特征向量 $\bma_1,\bma_2$,
	$\lambda_2$ 对应线性无关的特征向量 $\bmb_1,\bmb_2$.
	\onslide<+->
	若 $\lambda_1\neq \lambda_2$, 则 $\bma_1,\bma_2,\bmb_1,\bmb_2$ 也是线性无关的.
	\onslide<+->
	这是因为
	\[\bfA(k_1\bma_1+k_2\bma_2)=\lambda_1(k_1\bma_1+k_2\bma_2),\]
	\[\bfA(\ell_1\bmb_1+\ell_2\bmb_2)=\lambda_2(\ell_1\bmb_1+\ell_2\bmb_2).\]
	\onslide<+->
	若 $k_1\bma_1+k_2\bma_2+\ell_1\bmb_1+\ell_2\bmb_2={\bf0}$,
	同理可证明这些向量都是零向量.
	
	\onslide<+->
	由此也可以知道, 不同特征值的特征向量的线性组合不可能还是特征向量.
\end{frame}


\subsection{方阵的相似}

\begin{frame}{基的变换}
	\onslide<+->
	设 $f:\BC^n\to \BC^n$ 是一个线性映射.
	\onslide<+->
	对于任意 $\bfx\in \BC^n$, 它可以唯一表达为
	\[\bfx=x_1\bfe_1+\cdots+x_n\bfe_n.\]
	\onslide<+->
	通过将 $f(\bfe_1),\dots,f(\bfe_n)$ 表达为这组\emph{标准基}的线性组合, 我们建立了线性映射 $f$ 与一个 $n$ 阶方阵 $\bfA$ 的联系.

	\onslide<+->
	一般的线性空间并没有这样的标准正交基, 或者 $\BC^n$ 本身我们也可以选择其它基.
	\onslide<+->
	则这种对应会有什么变化呢?
	\onslide<+->
	设 $\bma_1,\dots,\bma_n$ 是 $\BC^n$ 中一组线性无关的向量, 
	\[\bfP=(\bma_1,\dots,\bma_n).\]
	\onslide<+->
	则
	\[f(\bma_i)=\bfA\bma_i=(\bma_1,\dots,\bma_n)\bfP^{-1}\bfA\bma_i,\]
	所以 $f(\bma_i)$ 表达为 $\bma_1,\dots,\bma_n$ 线性组合的系数形成的向量是 $\bfP^{-1}\bfA\bma_i$.
	\onslide<+->
	它们构成矩阵
	\[(\bfP^{-1}\bfA\bma_1,\dots,\bfP^{-1}\bfA\bma_n)=\bfP^{-1}\bfA\bfP.\]
\end{frame}


\begin{frame}{方阵的相似}
	\onslide<+->
	也就是说, 若我们将线性空间 $\BC^n$ 换一组基表达, 线性映射对应的矩阵就会变成 $\bfP^{-1}\bfA\bfP$.
	\onslide<+->
	我们称 $f$ 在不同基下矩阵为\emph{相似}的.
	\onslide<+->
	\begin{definition}
		若存在可逆矩阵 $\bfP$ 使得 $\bfB=\bfP^{-1}\bfA\bfP$, 则称 方阵 $\bfA$ 与 $\bfB$ \emph{相似}.
	\end{definition}
	\onslide<+->
	注意, 相似只对方阵有定义.
	\onslide<+->
	\begin{proposition}
		方阵的相似满足
		\begin{enumerate}
			\item 自反性: $\bfA$ 与自身相似;
			\item 对称性: 若 $\bfA$ 相似于 $\bfB$, 则 $\bfB$ 相似于 $\bfA$;
			\item 传递性: 若 $\bfA$ 相似于 $\bfB$, $\bfB$ 相似于 $\bfC$, 则 $\bfA$ 相似于 $\bfC$.
		\end{enumerate}
	\end{proposition}
\end{frame}


\begin{frame}{相似与等价}
	\onslide<+->
	若 $\bfA,\bfB$ 相似 ($\bfB=\bfP^{-1}\bfA\bfP$), 则 $\bfA,\bfB$ 等价 ($\bfB=\bfP\bfA\bfQ$).
		反之未必成立, 例如 $\bfA=\begin{pmatrix}
			1&0\\
			0&1
		\end{pmatrix}, \bfB=\begin{pmatrix}
			1&1\\
			0&1
		\end{pmatrix}$.
		这是因为和 $\bfA=\bfE$ 相似的只有它自己.
	\onslide<+->
	\begin{example}
		若 $3$ 阶方阵 $\bfA\wsim{r_1\swap r_3}{}\bfB\wsim{c_1\swap c_3}{}\bfC$, 则 $\bfA$ 与 $\bfC$ \fillbrace{\visible<+->{C}}.
		\begin{taskschoice}(2)
			() 等价但不相似
			() 相似但不等价
			() 等价而且相似
			() 既不等价也不相似
		\end{taskschoice}
	\end{example}
\end{frame}


\subsection{相似矩阵的性质}

\begin{frame}{相似矩阵的性质}
	\onslide<+->
	\begin{second}{相似矩阵的性质}
		若 $\bfA$ 与 $\bfB$ 相似, 则二者的特征多项式相同, 从而特征值也相同.
	\end{second}
	\onslide<+->
	这是因为若 $\bfP^{-1}\bfA\bfP=\bfB$, 则
	\[\bfP^{-1}(\bfA-\lambda\bfE)\bfP=\bfP^{-1}\bfA\bfP-\lambda\bfP^{-1}\bfE\bfP=\bfB-\lambda\bfE.\]
	\onslide<+->
	两边取行列式并利用行列式的可乘性得到 $|\bfA-\lambda\bfE|=|\bfB-\lambda\bfE|$.

	\onslide<+->
	注意反过来未必成立, 例如 $\bfA=\begin{pmatrix}
		1&0\\
		0&1
	\end{pmatrix}, \bfB=\begin{pmatrix}
		1&1\\
		0&1
	\end{pmatrix}$.
	的特征多项式相同, 但它们不相似.

	\onslide<+->
	相似矩阵具有相同的特征值, 但对应的特征向量未必相同.
\end{frame}


\begin{frame}{相似矩阵的性质}
	\onslide<+->
	\begin{second}{相似矩阵的性质}
		若 $\bfA$ 与 $\bfB$ 相似, 则
		\begin{enumerate}
			\item \alert{$\bfA,\bfB$ 特征值相同(包括重数)};
			\item \alert{$|\bfA|=|\bfB|$, $\Tr(\bfA)=\Tr(\bfB)$};
			\item $\bfA\sim\bfB$, 即 \alert{$R(\bfA)=R(\bfB)$};
			\item 对于多项式 $g,h$, 若 $h(\bfA)$ 可逆, 则 $h(\bfB)$ 也可逆, 且 $g(\bfA)/h(\bfA)$ 与 $g(\bfB)/h(\bfB)$ 相似.
			特别地, $\bfA-\lambda\bfE$ 与 $\bfB-\lambda\bfE$ 相似.
		\end{enumerate}
	\end{second}
	\onslide<+->
	\begin{corollary}
		若 $\bfA$ 与对角阵 $\bmL=\diag(\lambda_1,\dots,\lambda_n)$ 相似, 则 $\lambda_1,\dots,\lambda_n$ 是 $\bfA$ 的 $n$ 个特征值.
	\end{corollary}
\end{frame}


\begin{frame}{例: 相似矩阵的性质}
	\onslide<+->
	\begin{example}
		若 $4$ 阶方阵 $\bfA$ 与 $\bfB$ 相似, $\bfA$ 的特征值为 $1,2,3,4$, 则 $|\bfB+\bfE|=$\fillblank{\visible<+->{$120$}}.
	\end{example}
	\onslide<+->
	\begin{example}
		设$3$ 阶可逆阵 $\bfA,\bfB$ 相似, $\bfA^{-1}$ 的特征值为 $1/2,1/3,1/4$, 则 $|\bfE-\bfB|=$\fillblank{\visible<+->{$-6$}}.
	\end{example}
	\onslide<+->
	\begin{example}
		设矩阵 $\bfA=\begin{pmatrix}
			2&-1&4\\
			0&a&7\\
			0&0&3
		\end{pmatrix}$ 与 $\bmL=\begin{pmatrix}
			1&0&0\\
			0&2&0\\
			0&0&b
		\end{pmatrix}$ 相似, 求 $a$ 与 $b$ 的值.
	\end{example}
	\onslide<+->
	\begin{solution}
		二者特征值相同, 从而 $a=1,b=3$.
	\end{solution}
\end{frame}


\begin{frame}{例: 相似矩阵的性质}
	\onslide<+->
	\begin{exercise}
		\begin{enumerate}
			\item 若 $3$ 阶矩阵 $\bfA$ 与 $\bfB$ 相似, 且 $|\bfA-\bfE|=0, |\bfA+2\bfE|=0, |2\bfA-\bfE|=0$, 则 $|\bfB^{-1}-\bfE|=$\fillblank{\visible<4->{$3/4$}}.
			\item 若 $3$ 阶矩阵 $\bfA$ 与 $\bfB$ 相似, 且存在非零矩阵 $\bfC$ 使得 $\bfA\bfC=2\bfC, |\bfA+2\bfE|=0, |2\bfA-\bfE|=0$, 则 $|\bfB^{-1}-\bfE|=$\fillblank{\visible<5->{$3/4$}}.
		\end{enumerate}
	\end{exercise}
	\onslide<+->
	若 $\bfA\bfB=k\bfB$, 则 $\bfB$ 的每个非零列向量均为 $\bfA$ 的属于特征值 $k$ 的特征向量.
\end{frame}


\subsection{相似对角化}
\begin{frame}{相似对角化的定义}
	\onslide<+->
	\begin{definition}
		若 $n$ 阶方阵 $\bfA$ 相似于某个对角阵
		\[\bmL=\diag(\lambda_1,\dots,\lambda_n),\]
		则称 $\bfA$ 可\emph{(相似)对角化}.
	\end{definition}
	\onslide<+->
	设 $\bfP=(\bfp_1,\dots,\bfp_n)$, 则
	\[\bfP^{-1}\bfA\bfP=\bmL\iff
	\bfA\bfP=\bfP\bmL\iff
	(\bfA\bfp_1,\dots,\bfA\bfp_n)
	=(\lambda_1\bfp_1,\dots,\lambda_n\bfp_n),\]
	\onslide<+->
	即 $\bfA\bfp_i=\lambda_i\bfp_i$.
	\onslide<+->
	由于 $\bfP$ 可逆, $\bfA$ 拥有 $n$ 个线性无关的特征向量 $\bfp_1,\dots,\bfP_n$.
	\onslide<+->
	反之, 若 $\bfA$ 拥有 $n$ 个线性无关的特征向量, 则选择 $\bfP$ 以它们为列向量即可使 $\bfA$ 对角化.
	\onslide<+->
	\begin{second}{相似对角化的等价刻画}
		$n$ 阶矩阵 $\bfA$ 可对角化$\iff\bfA$ 有 $n$ 个线性无关的特征向量.
	\end{second}
\end{frame}


\begin{frame}{可对角化的刻画}
	\onslide<+->
	\begin{corollary}
		若 $\bfA$ 的特征值两两不同, 则 $\bfA$ 可对角化.
	\end{corollary}
	\onslide<+->
	\alert{反之未必成立}.
	\onslide<+->
	\begin{example}
		设 $3$ 阶方阵 $\bfB$ 的特征值为 $1,2,-1$, $\bfA=\bfB^3-2\bfB$, 求 $\bfA$ 的特征值及相似对角阵.
	\end{example}
	\onslide<+->
	\begin{solution}
		$\bfA$ 的特征值为 $-1,4,1$, 相似对角阵为 $\diag(-1,4,1)$.
	\end{solution}
	\onslide<+->
	回忆 $k$ 重特征值对应的线性无关的特征向量最多 $k$ 个.
	\onslide<+->
	\begin{second}{相似对角化的等价刻画}
		若 $\lambda$ 是 $\bfA$ 的 $k$ 重特征值, 则 $\bfA$ 可对角化 $\iff\forall\lambda, R(\bfA-\lambda\bfE)=n-k$, 即 $\lambda$ 对应的线性无关的特征向量恰有 $k$ 个.
	\end{second}
\end{frame}


\begin{frame}{例: 可对角化的刻画}
	\onslide<+->
	\begin{example}
		判断下列矩阵 $\bfA$ 能否相似对角化:
		\enumnum1 $\bfA$ 是二阶实矩阵且 $|\bfA|<0$;

		\enumnum2 $\bfA=\begin{pmatrix}
				1&2&1\\
				0&3&0\\
				0&0&0
			\end{pmatrix}$;
		\enumnum3 $\bfA=\begin{pmatrix}
				-2&1&1\\
				0&2&0\\
				-4&1&3
			\end{pmatrix}$;
		\enumnum4 $\bfA=\begin{pmatrix}
				-1&1&0\\
				-4&3&0\\
				1&0&2
			\end{pmatrix}$.
	\end{example}
	\onslide<+->
	\begin{solution}
		\enumnum1 特征值一正一负, 能对角化;
		\onslide<+->{%
			\enumnum2 特征值为 $1,3,0$, 能对角化;
		}
		
		\onslide<+->{%
			\enumnum3	$(2-\lambda)(\lambda^2-\lambda-2)\implies\lambda=-1,2,2$.
		}\onslide<+->{%
		 $\bfA-2\bfE=\begin{pmatrix}
				-4&1&1\\
				0&0&0\\
				-4&1&1
			\end{pmatrix}$ 秩 $1$, 能对角化;
		}
		
		\onslide<+->{%
			\enumnum4 $(2-\lambda)(\lambda^2-2\lambda+2)\implies\lambda=2,1,1$.
		}\onslide<+->{%
			$\bfA-\bfE=\begin{pmatrix}
				-2&1&0\\
				-4&2&0\\
				1&0&1
			\end{pmatrix}$ 秩 $2$, 不能对角化.
		}
	\end{solution}
\end{frame}


\begin{frame}{例: 可对角化矩阵的应用}
	\onslide<+->
	\begin{example}
		计算 $\bfA=\begin{pmatrix}
			a&b&\cdots&b\\
			b&a&\cdots&b\\
			\vdots&\vdots&\ddots&\vdots\\
			b&b&\cdots&a
		\end{pmatrix}$ 的行列式.
	\end{example}
	\onslide<+->
	\begin{solution}
		$b=0$ 时, $|\bfA|=a^n$.
		\onslide<+->{%
			设 $b\neq 0$, $\bfB=\bfA+(b-a)\bfE$,
		}\onslide<+->{%
			则 $\bfB$ 所有元素为 $b$, 秩为 $1$, $\bfB\bfx={\bf0}$ 基础解系有 $n-1$ 个向量.
		}\onslide<+->{%
			从而 $0$ 是 $\bfB$ 的至少 $n-1$ 重特征值.
		}\onslide<+->{%
			由于 $\bfB$ 行和为 $nb$, 因此 $nb$ 也是它的特征值.
		}\onslide<+->{%
			所以 $\bfB$ 的特征值就是 $nb$ 和 $0$ ($n-1$ 重), 特征多项式为
			\[|\bfB-\lambda\bfE|=(nb-\lambda)(-\lambda)^{n-1},\]
			\vspace{-\baselineskip}
		}\onslide<+->{%
			\[|\bfA|=\bigl(nb-(b-a)\bigr)(a-b)^{n-1}=(a-b)^{n-1}(nb-b+a).\]
			\vspace{-\baselineskip}
		}
	\end{solution}
\end{frame}


\begin{frame}{相似对角化的步骤}
	\onslide<+->
	相似对角化的步骤如下:
	\begin{enumerate}
		\item 求出 $\bfA$ 的所有特征值 $\lambda_i$ 和特征向量 $\bfp_i$;
		\item 根据上述判定方法判断 $\bfA$ 是否可以相似对角化;
		\item 若能, 将 $n$ 个对应的线性无关的特征向量 $\bfp_1,\dots,\bfp_n$ 组成方阵 $\bfP=(\bfp_1,\dots,\bfp_n)$, 
		\[\bfP^{-1}\bfA\bfP=\diag(\lambda_1,\dots,\lambda_n).\]
	\end{enumerate}
\end{frame}


\begin{frame}{例: 方阵的相似的应用}
	\onslide<+->
	\begin{example}
		设 $\bfA=\begin{pmatrix}
			1&1\\
			0&2
		\end{pmatrix}$, 计算 $\bfA^k$.
	\end{example}
	\onslide<+->
	\begin{solution}
		$\bfA$ 特征值为 $1,2$, 对应的特征向量可以取 $\bfp_1=\begin{pmatrix}
			1\\0
		\end{pmatrix},\bfp_2=\begin{pmatrix}
			1\\1
		\end{pmatrix}$.
		\onslide<+->{%
			设 $\bfP=\begin{pmatrix}
			1&1\\0&1
		\end{pmatrix}$, 则 $\bfP^{-1}\bfA\bfP=\begin{pmatrix}
			1&\\&2
		\end{pmatrix}$.
		}\onslide<+->{%
			因此
			\[\bfA^k=\bfP\begin{pmatrix}
				1&\\&2
			\end{pmatrix}^k\bfP^{-1}=\begin{pmatrix}
				1&1\\0&1
			\end{pmatrix}\begin{pmatrix}
				1&\\&2^k
			\end{pmatrix}\begin{pmatrix}
				1&-1\\0&1
			\end{pmatrix}=\begin{pmatrix}
				1&2^k-1\\
				0&2^k
			\end{pmatrix}.\]
		}
	\end{solution}
\end{frame}


\begin{frame}{例: 对角化的计算}\small
	\onslide<+->
	\begin{example}
		$\bfA=\begin{pmatrix}
			1&1&1\\
			0&0&0\\
			0&0&0
		\end{pmatrix}$ 能否对角化? 若能, 求 $\bfP$ 使得 $\bfP^{-1}\bfA\bfP$ 是对角阵.
	\end{example}
	\onslide<+->
	\begin{solution}
		\begin{enumerate}
			\item 上三角阵 $\bfA$ 特征值为 $1,0,0$.
			\item 对于 $\lambda_1=1, \bfA-\bfE=\begin{pmatrix}
				0&1&1\\
				0&-1&0\\
				0&0&-1
			\end{pmatrix}\simr\begin{pmatrix}
				0&1&0\\
				0&0&1\\
				0&0&0
			\end{pmatrix}$,
			\onslide<+->{%
				取特征向量 $\bfp_1=(1,0,0)^\rmT$.
			}
		\item 对于 $\lambda_2=\lambda_3=0, \bfA$ 对应的基础解系可以取 $\bfp_2=(-1,1,0)^\rmT,\bfp_3=(-1,0,1)^\rmT$.
		\item 因此 $\bfA$ 可对角化, 取 $\bfP=\begin{pmatrix}
					1&-1&-1\\
					0&1&0\\
					0&0&1
			\end{pmatrix}$, 则 $\bfP^{-1}\bfA\bfP=\diag(1,0,0)$.
		\end{enumerate}
	\end{solution}
\end{frame}


\begin{frame}{例: 对角化的性质}\small
\beqskip{4pt}
	\onslide<+->
	\begin{exercise}
		已知矩阵 $\bfA=\begin{pmatrix}
			0&0&1\\
			a&1&1\\
			1&0&0
		\end{pmatrix}$ 可对角化, 则\fillblank{\visible<+->{$-1$}}.
	\end{exercise}
	\onslide<+->
	\begin{exercise}
		设 $\bfP^{-1}\bfA\bfP=\diag(1,2,3)$.
		若 $\bfP=(\bma_1,\bma_2,\bma_3),\bfQ=(\bma_1,\bma_3,\bma_2)$	, 则 $\bfQ^{-1}\bfA\bfQ=$\fillblank[2cm]{\visible<+->{$\diag(1,3,2)$}}.
	\end{exercise}
	\onslide<+->
	\begin{exercise}
		若 $\bfA=\begin{pmatrix}
			2&0&0\\
			0&2&1\\
			0&0&1
		\end{pmatrix},\bfB=\begin{pmatrix}
			2&1&0\\
			0&2&0\\
			0&0&1
		\end{pmatrix},\bfC=\begin{pmatrix}
			1&0&0\\
			0&2&0\\
			0&0&2
		\end{pmatrix}$, 则\fillbrace{\visible<+->{C}}.
		\begin{taskschoice}(2)
			() $\bfA,\bfC$ 相似, $\bfB,\bfC$ 相似
			() $\bfA,\bfC$ 不相似, $\bfB,\bfC$ 相似
			() $\bfA,\bfC$ 相似, $\bfB,\bfC$ 不相似
			() $\bfA,\bfC$ 不相似, $\bfB,\bfC$ 不相似
		\end{taskschoice}
	\end{exercise}
\endgroup
\end{frame}


\begin{frame}{相似标准形\noexer}
	\onslide<+->
	任何方阵都相似于\emph{约当标准形}
	\[\diag(\bfJ_{k_1}(\lambda_1),\dots,\bfJ_{k_t}(\lambda_t)),\]
	其中 $\bfJ_k(\lambda)=\begin{pmatrix}
		\lambda&1&&\\
		&\lambda&\ddots&\\
		&&\ddots&1\\
		&&&\lambda
	\end{pmatrix}$ 是 $k$ 阶方阵.
	\onslide<+->
	当 $k_1=\cdots=k_t=1$ 时就是对角阵.

	\onslide<+->
	由此可知, \alert{$\bfA^*$ 的所有特征值的就是 $\bfA$ 的 $n$ 个特征值中 $n-1$ 个相乘得到的}.
	\begin{enumerate}
		\item 当 $|\bfA|\neq0$ 时, $\bfA^*$ 的所有特征值为 $|\bfA|/\lambda$.
		\item 当 $|\bfA|=0$ 且 $\lambda_1=0$ 是一重特征值, 则$\bfA^*$ 唯一的非零特征值为 $\bfA$ 非零特征值之乘积.
		\item 当 $|\bfA|=0$ 且 $\lambda_1=0$ 是 $\ge2$ 重特征值, 则$\bfA^*=\bfO$.
	\end{enumerate}
\end{frame}


\begin{frame}{相似标准形\noexer}
	\onslide<+->
	对于 $\bfA$ 的 $k$ 重特征值 $\lambda$,
	\onslide<+->
	若
	\[R\bigl((\bfA-\lambda\bfE)^i\bigr)=n-r_i,\quad
	0=r_0\le r_1\le r_2\le \cdots\le k,\]
	\onslide<+->
	则 $\lambda$ 对应的约当块有 $r_1$ 个, 其中 $J_k(\lambda)$ 有 $2r_k-r_{k+1}-r_{k-1}$ 个.
	
	\onslide<+->
	利用约当标准形可以计算任意矩阵的方幂.
	\onslide<+->
	设 $\bfN=\bfJ_k(0)$, 则 $\bfN^k=\bfO$,
	\[\bfJ_k(\lambda)^m
	=(\lambda\bfE+\bfN)^m=\sum_{i=0}^m \rmC_m^i \lambda^{m-i}\bfN^i.\]
\end{frame}


\begin{frame}{例: 特征值的性质}
	\onslide<+->
	\begin{exercise}
		若 $4$ 阶实矩阵 $\bfA^*$ 的特征值为 $-1,1,2,4$, 则下列矩阵可逆的是\fillbrace{\visible<+->{D}}.
		
		\begin{taskschoice}(4)
			() $\bfA+2\bfE$
			() $\bfA-2\bfE$
			() $\bfA+\frac12\bfE$
			() $\bfA-\bfE$
		\end{taskschoice}
	\end{exercise}
	\onslide<+->
	\begin{exercise}
		若 $n$ 阶实对称矩阵 $\bfA$ 满足 $\bfA^2-\bfA=\bfO$, 则下列矩阵不可逆的是\fillbrace{\visible<+->{C}}.
		
		\begin{taskschoice}(4)
			() $\bfA+2\bfE$
			() $\bfA+\bfE$
			() $\bfA-\bfE$
			() $\bfA-2\bfE$
		\end{taskschoice}
	\end{exercise}
	\onslide<+->
	\begin{exercise}
		若 $3$ 阶方阵 $\bfA$ 的特征值互不相同且 $|\bfA|=0$, 则 $R(\bfA)=$\fillblank{\visible<+->{$2$}}.
	\end{exercise}
\end{frame}


\begin{frame}{例: 对角化的计算}\small
	\onslide<+->
	\begin{example}
		设 $\bfA=\begin{pmatrix}
			1&-1&1\\
			x&4&y\\
			-3&-3&5
		\end{pmatrix}$ 有 $3$ 个线性无关特征向量, $\lambda=2$ 是 $\bfA$ 的二重特征值. 求可逆阵 $\bfP$, 使得 $\bfP^{-1}\bfA\bfP$ 为对角阵.
	\end{example}
	\onslide<+->
	\begin{solution}
		由 $\Tr(\bfA)=10$ 可知特征值为 $2,2,6$.
		\onslide<+->{%
			由 $R(\bfA-2\bfE)=1$ 可知 $x=2,y=-2$.
		}
		
		\onslide<+->{%
		$\bfA-2\bfE=\begin{pmatrix}
			-1&-1&1\\
			x&2&y\\
			-3&-3&3
		\end{pmatrix}\simr\begin{pmatrix}
			1&1&-1\\
			0&0&0\\
			0&0&0
		\end{pmatrix}\implies \bfp_1=\begin{pmatrix}
			-1\\1\\0
		\end{pmatrix},\bfp_2=\begin{pmatrix}
			1\\0\\1
		\end{pmatrix}$.
		}

		\onslide<+->{%
		$\bfA-6\bfE=\begin{pmatrix}
			-5&-1&1\\
			2&-2&-2\\
			-3&-3&-1
		\end{pmatrix}\simr\begin{pmatrix}
			1&0&-1/3\\
			0&1&2/3\\
			0&0&0
		\end{pmatrix} \implies\bfp_3=\begin{pmatrix}
			1\\-2\\3
		\end{pmatrix}$.
		$\bfP=\begin{pmatrix}
			-1&1&1\\
			1&0&-2\\
			0&1&3
		\end{pmatrix}$.}
	\end{solution}
\end{frame}


\begin{frame}{例: 对角化的计算}
	\onslide<+->
	\begin{example}
		已知 $n$ 阶方阵 $\bfA$ 的各行元素之和为 $2$, $\bfA\begin{pmatrix}
			1&2\\
			0&1\\
			-1&1
		\end{pmatrix}=\begin{pmatrix}
			-1&2\\
			0&1\\
			1&1
		\end{pmatrix}$, 则 $\bfA$ 相似于\fillbrace{\visible<+->{C}}.
		
		\begin{taskschoice}(4)
			() $\diag(1,1,2)$
			() $\diag(2,1,1)$
			() $\diag(2,1,-1)$
			() $\diag(2,-1,-1)$
		\end{taskschoice}
	\end{example}
	\onslide<+->
	\begin{exercise}
		设 $\bfA=\begin{pmatrix}
			2&a&2\\
			5&b&3\\
			-1&1&-1
		\end{pmatrix}$ 有特征值 $\pm1$, 问 $\bfA$ 能否相似对角化?
	\end{exercise}
	\onslide<+->
	\begin{answer}
		$|\bfA\pm\bfE|=0\implies a=-1,b=-3$.
		$\Tr(\bfA)=-2\implies\lambda_3=-2$, 可对角化.
	\end{answer}
\end{frame}


\begin{frame}{例: 对角化的计算}
	\onslide<+->
	\begin{example}
		设 $\bfA$ 为三阶方阵, $\bma_1,\bma_2,\bma_3$ 是线性无关的三维列向量且 $\bfA\bma_1=2\bma_1,\bfA\bma_2=3\bma_2+2\bma_3,\bfA\bma_3=2\bma_2+3\bma_3$.
		求 $\bfA$ 并证明 $\bfA$ 可对角化.
	\end{example}
	\onslide<+->
	\begin{solution}
		$\bfA(\bma_1,\bma_2,\bma_3)=(\bma_1,\bma_2,\bma_3)\bfB,\bfB=\begin{pmatrix}
			2&0&0\\
			0&3&2\\
			0&2&3
		\end{pmatrix}$.
		由于 $\bfB$ 是实对称矩阵, 因此 $\bfB$ 能对角化, 从而 $\bfA$ 也可以.
	\end{solution}
	\onslide<+->
	为什么实对称矩阵一定能对角化?
	在下一节中我们将回答这个问题.
\end{frame}


\begin{frame}{相似的应用: 求线性递推数列通项\noexer}
\beqskip{1pt}
	\onslide<+->
	线性递推数列是一种常见的数列,
	\onslide<+->
	例如著名的斐波那契数列: 假设一对刚出生的小兔一个月后就能长成大兔, 再过一个月就能生下一对小兔, 并且此后每个月都生一对小兔.
	若一年内没有发生死亡, 一对刚出生的兔子一年内能繁殖成多少对兔子?

	\onslide<+->
	设第 $n$ 个月的兔子数量为 $a_n$, 那么
	\[a_{n+2}=a_{n+1}+a_n,\quad a_2=a_1=1.\]
	\onslide<+->
	设 $\bfx_n=\begin{pmatrix}
		a_{n+1}\\a_n
	\end{pmatrix}$, 则 $\bfx_{n+1}=\bfA\bfx_n$, $\bfA=\begin{pmatrix}
		1&1\\
		1&0
	\end{pmatrix}$.
	\onslide<+->
	通过计算可知
	\[\bfA=\bfP\bmL\bfP^{-1},\quad
	\bmL=\diag\Bigl(\frac{1+\sqrt5}2,\frac{1-\sqrt5}2\Bigr), \quad \bfP=\begin{pmatrix}
		\frac{1+\sqrt5}2&\frac{1-\sqrt5}2\\
		1&1
	\end{pmatrix}.\]
	\onslide<+->
	因此
	$\bfx_n=\bfA^{n-1}\bfx_1=\bfP\bmL^{n-1}\bfP^{-1}\bfx_1$,
	$\displaystyle a_n=\frac1{\sqrt5}\biggl(\Bigl(\frac{1+\sqrt5}2\Bigr)^n-\Bigl(\frac{1-\sqrt5}2\Bigr)^n\biggr)$,
	\onslide<+->
	\[a_{12}\approx \frac{1.618^n}{\sqrt5}=143.9\approx144.\]
\endgroup
\end{frame}


\begin{frame}{相似的应用: 求线性递推数列通项\noexer}
	\onslide<+->
	对于一般的线性递推数列
	\[a_{n+k}-c_{k-1}a_{n+k-1}-\cdots-c_1a_{n+1}-c_0a_n=0,\]
	\onslide<+->
	$\bfx_n=(a_{n+k-1},\dots,a_{n+1},a_n)^\rmT$ 满足 $\bfx_{n+1}=\bfA\bfx_n$, 其中 $\bfA=\begin{pmatrix}
		c_{k-1}&\cdots&c_1&c_0\\
		1&&&\\
		&\ddots&&\\
		&&1&
	\end{pmatrix}$ 的特征多项式为
	\[(-1)^k(\lambda^k-c_{k-1}\lambda^{k-1}-\cdots-c_1\lambda-c_0)=0.\]
	\onslide<+->
	若特征值两两不同, 可知 $a_n$ 的通项为 $\lambda_1^n,\dots,\lambda_k^n$ 的线性组合;
	\onslide<+->
	若特征值有重数, 由约当块的幂次形式可知
	\[a_n=\sum_\lambda f_\lambda(n)\lambda^n,\]
	其中 $f_\lambda$ 是多项式, 且次数不超过 $\lambda$ 的重数 $-1$.
\end{frame}


\begin{frame}{相似的应用: 求常系数微分方程的解\noexer}
	\onslide<+->
	类似地, 对于微分方程
	\[y^{(k)}(t)-c_{k-1}y^{(k-1)}(t)-\cdots-c_1y'(t)-c_0y(t)=0,\]
	\onslide<+->
	$\bfx=(y^{(k-1)},\dots,y',y)^\rmT$ 满足 $\bfx'=\bfA\bfx$.
	\onslide<+->
	于是 ``$\bfx=\exp(\bfA t)$''.

	\onslide<+->
	其实这里的指数函数是通过幂级数展开定义的
	\[\exp(\bfA)=\sum_{i=0}^\infty\frac{\bfA^i}{i!}.\]
	\onslide<+->
	若 $\bfA=\bfP\diag(\lambda_1,\dots,\lambda_k)\bfP^{-1}$, 容易看出 $\exp(\bfA t)=\bfP\diag(e^{\lambda_1 t},\dots,e^{\lambda_k t})\bfP^{-1}$,
	\onslide<+->
	从而 $y$ 是 $e^{\lambda_1t},\dots,e^{\lambda_kt}$ 的线性组合.
	\onslide<+->
	对于一般情形, 通过考虑约当块的指数函数可知
	\[y(t)=\sum_\lambda f_\lambda(t)e^{\lambda t},\]
	其中 $f_\lambda$ 是多项式, 且次数不超过 $\lambda$ 的重数 $-1$.
\end{frame}

