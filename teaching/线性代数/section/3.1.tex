\section{向量组的线性表示}

\subsection{\texorpdfstring{$n$}{n} 维向量的定义及运算}

\begin{frame}{行向量和列向量}
	\onslide<+->
	当矩阵的行数或列数为 $1$ 时, 对应的矩阵被称为\emph{行向量}和\emph{列向量}:
	\[{\bma}^\rmT=(a_1,\dots,a_n),\quad
	{\bma}=\begin{pmatrix}
		a_1\\\vdots\\a_n
	\end{pmatrix}.\]
	\onslide<+->
	其中 $a_i$ 称为 $n$ 维行向量 ${\bma}^\rmT$ 或 $n$ 维列向量 ${\bma}$ 的\emph{第 $i$ 个分量}.
	\onslide<+->
	向量也是我们中学学习的平面和立体向量的高维推广.

	\onslide<+->
	注意行向量和列向量是不同的东西.
	当没有明确说明是行向量还是列向量时, 默认是\alert{列向量}.
\end{frame}


\begin{frame}{向量运算的性质}
	\onslide<+->
	分量全为零的向量称为\emph{零向量} ${\bf0}=(0,0,\dots,0)^\rmT$.
	\onslide<+->
	上一章中我们已经知道了向量的加法和数乘运算, 它们满足:
	\begin{enumerate}
		\item $\bma+\bmb=\bmb+\bma$;
		\item $(\bma+\bmb)+\bmg=\bma+(\bmb+\bmg)$;
		\item $\bma+{\bf0}=\bma$;
		\item $1\cdot\bma=\bma$;
		\item $k(\ell\bma)=(k\ell)\bma$;
		\item $(k+\ell)\bma=k\bma+\ell\bma$;
		\item $k(\bma+\bmb)=k\bma+k\bmb$.
	\end{enumerate}
\end{frame}


\subsection{向量组与矩阵}


\begin{frame}{向量运算的性质}
	\onslide<+->
	由一些具有相同维数的向量构成的集合称为\emph{向量组}.
	\onslide<+->
	例如
	\[\{\bma_1\}\]
\end{frame}
