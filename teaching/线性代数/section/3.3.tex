\section{实对称阵的合同}

\subsection{惯性指数}

\begin{frame}{引例: 二次曲线的分类}
	\onslide<+->
	设 $A,B,C$ 是不全为零的实数.
	二次曲线 $Ax^2+Bxy+Cy^2=1$ 左侧的实二次型对应方阵 $\bfA=\begin{pmatrix}
		A&B/2\\
		B/2&C
	\end{pmatrix}$.
	\onslide<+->
	由
	\[|\bfA-\lambda\bfE|=\begin{vmatrix}
		A-\lambda&B/2\\
		B/2&C-\lambda
	\end{vmatrix}=\lambda^2-(A+C)\lambda+(AC-B^2/4)\]
	可知,
	\begin{enumerate}
		\item 当 $B^2-4AC>0$ 时, $\bfA$ 特征值一正一负, 
		\onslide<+->
		从而通过正交变换 $\begin{pmatrix}
			x\\y
		\end{pmatrix}=\bfP\begin{pmatrix}
			s\\t
		\end{pmatrix}$ 可知该曲线为双曲线.
		\item 同理, $B^2-4AC<0$ 时该曲线为椭圆(或空集);
		\item $B^2-4AC=0$ 时该曲线为两条直线(若有一次项则为抛物线).
	\end{enumerate}
\end{frame}


\begin{frame}{惯性定理}
	\onslide<+->
	可以看出我们有时候只关心实二次型对应的矩阵的特征值的符号.
	\onslide<+->
	由于合同矩阵秩相同, 因此可定义:
	\begin{definition}
		称实二次型 $f$ 对应矩阵的秩为 $f$ 的\emph{秩}.
	\end{definition}
	\onslide<+->
	\begin{algorithm}{惯性定理}
		\begin{enumerate}
			\item 若 $\bfA$ 和 $\bfB$ 为合同的对角阵, 则 $\bfA,\bfB$ 对角元中正数的个数相同.\label{enum:inert-matrix}
			\item \label{enum:inert-quad-form}
			设实二次型 $f$ 的秩为 $r$.
			若可逆线性变换 $\bfx=\bfP\bfy=\bfQ\bfz$ 分别将 $f$ 变为
			\[
				f=k_1y_1^2+\cdots+k_ry_r^2,\qquad
				 =\ell_1z_1^2+\cdots+\ell_rz_r^2,
			\]
			则 $k_1,\dots,k_r$ 中正的个数和 $\ell_1,\dots,\ell_r$ 中正的个数相同.
		\end{enumerate}
	\end{algorithm}
\end{frame}


\begin{frame}{惯性指数}
	\onslide<+->
	\begin{proof}
		设 
		\begin{align*}
			\bfA&=\diag(k_1,\dots,k_r,0,\dots,0),\\
			\bfB&=\diag(\ell_1,\dots,\ell_r,0,\dots,0)=\bfP^\rmT\bfA\bfP,
		\end{align*}
		其中 $\bfP=(\bma_1,\dots,\bma_n)$ 为可逆矩阵.
		\onslide<+->{%
		不难得到 $\bfP^\rmT\bfA\bfP$ 的对角元:
		\[
			\bfB=\bfP^\rmT\bfA\bfP=\diag(k_1\bma_1^\rmT\bma_1,
			\dots,k_r\bma_r^\rmT\bma_r,0,\dots,0).
		\]
		}\onslide<+->{%
		从而 $\ell_i=k_i\|\bma_i\|^2$, 二者符号相同.
		}\onslide<+->{%
		\enumref{enum:inert-quad-form} 由\enumref{enum:inert-matrix} 得到.\qedhere
		}
	\end{proof}
	\onslide<+->
	\begin{definition}
		把实二次型 $f$ 标准形系数中为正数的个数称为 $f$ 的\emph{正惯性指数} $p$, 为负数的个数称为 $f$ 的\emph{负惯性指数} $q=r-q$.
	\end{definition}
\end{frame}


\begin{frame}{实对称阵的相合}
	\onslide<+->
	\begin{corollary}
		实二次型 $f=\bfx^\rmT\bfA\bfx$ 的正(负)惯性指数等于实对称阵 $\bfA$ 的正(负)特征值的个数.
	\end{corollary}
	\onslide<+->
	\begin{theorem}
		任意 $n$ 阶实对称矩阵 $\bfA$ 合同于对角矩阵
		\[\begin{pmatrix}
			\bfE_p&&\\
			&-\bfE_q&\\
			&&\bfO_{n-p-q}
		\end{pmatrix},\]
		其中 $p,q$ 分别为正负特征值个数(计算重数), $\rank(\bfA)=p+q$.
	\end{theorem}
	\onslide<+->
	从而正负惯性指数相同的实对称阵是合同的.
	\onslide<+->
	\begin{corollary}
		$n$ 阶实对称阵 $\bfA$ 与 $\bfB$ 合同$\iff \bfA,\bfB$ 的正负特征值个数均相同.
	\end{corollary}
\end{frame}


\begin{frame}{例: 惯性指数的应用}
	\onslide<+->
	\begin{example}
		设 $\bfA=\begin{pmatrix}
			1&2\\2&1
		\end{pmatrix}$, 则 $\bfA$ 合同于\fillbraceframe{D}.
		\begin{exchoice}(4)
			() $\begin{pmatrix}-2&1\\1&-2\end{pmatrix}$
			() $\begin{pmatrix}2&-1\\-1&2\end{pmatrix}$
			() $\begin{pmatrix}2&1\\1&2\end{pmatrix}$
			() $\begin{pmatrix}1&-2\\-2&1\end{pmatrix}$
		\end{exchoice}
	\end{example}
	\onslide<+->
	\begin{example}
		矩阵 $\begin{pmatrix}
			1&1\\1&1
		\end{pmatrix}$ 与 $\begin{pmatrix}
			3&0\\0&0
		\end{pmatrix}$\fillbraceframe{B}.
		\begin{exchoice}(2)
			() 合同且相似
			() 合同但不相似
			() 不合同但相似
			() 既不合同也不相似
		\end{exchoice}
	\end{example}
\end{frame}


\begin{frame}{实二次型规范形}
	\beqskip{3pt}
	\onslide<+->
	\begin{definition}
		若实二次型的标准形的系数只在 $-1, 0, 1$ 三个数中取值, 则称此实二次型为\emph{规范形}.
	\end{definition}
	\onslide<+->
	\begin{theorem}
		任意一个实二次型总可经过适当的可逆线性变换化为规范形, 且规范形是唯一的(可任意交换变量顺序):
		\[f=x_1^2+\cdots+x_p^2-x_{p+1}^2-\cdots-x_{p+q}^2,\]
		其中 $p,q$ 分别为正负惯性指数.
	\end{theorem}
	\onslide<+->
	\begin{example}
		若实对称矩阵 $\bfA$ 合同于 $\begin{pmatrix}
			1&0&0\\0&0&2\\0&2&0
		\end{pmatrix}$, 则通过可逆线性变换 $\bfx=\bfP\bfy$ 可将二次型 $\bfx^\rmT\bfA\bfx$ 化为规范形\fillblankframe[3cm][1pt]{$y_1^2+y_2^2-y_3^2$}.
	\end{example}
	\endgroup
\end{frame}


\begin{frame}{例: 二次曲面的分类}
	\onslide<+->
	对于三元实二次型, 正负惯性指数确定了二次曲面的类别.
	\begin{enumerate}
		\item $p=3,q=0$ 为椭球面 $\displaystyle\frac{x^2}{a^2}+\frac{y^2}{b^2}+\frac{z^2}{c^2}=1$.
		\item $p=2,q=1$ 为单叶双曲面 $\displaystyle\frac{x^2}{a^2}+\frac{y^2}{b^2}-\frac{z^2}{c^2}=1$.
		\item $p=1,q=2$ 为双叶双曲面 $\displaystyle\frac{x^2}{a^2}-\frac{y^2}{b^2}-\frac{z^2}{c^2}=1$.
		\item $p=2,q=0$ 为椭圆柱面 $\displaystyle\frac{x^2}{a^2}+\frac{y^2}{b^2}=1$.
		\item $p=1,q=1$ 为双曲柱面 $\displaystyle\frac{x^2}{a^2}+\frac{y^2}{b^2}=1$.
	\end{enumerate}
\end{frame}



\subsection{正定二次型}

\begin{frame}{正定和负定}
	\onslide<+->
	\begin{definition}
		设 $f=\bfx^\rmT\bfA\bfx$ 是二次型.
		\begin{enumerate}
			\item 若对任意 $\bfx\neq{\bf0}$, 都有 $f(\bfx)>0$, 则称 $f$ 为\emph{正定二次型}, 并称实对称阵 $\bfA$ 为\emph{正定矩阵}, 也记作 $\bfA>0$.
			\item 若对任意 $\bfx$, 都有 $f(\bfx)\ge0$, 则称 $f$ 为\emph{半正定二次型}, 并称实对称阵 $\bfA$ 为\emph{半正定矩阵}, 也记作 $\bfA\ge0$.
			\item 若 $-f$ (半)正定, 则称 $f$ 为\emph{(半)负定二次型}, 并称实对称阵 $\bfA$ 为\emph{(半)负定矩阵}, 也记作 $\bfA<0$ ($\bfA\le0$).
			\item 除此之外, 称 $f$ \emph{不定}.
		\end{enumerate}
	\end{definition}
	\onslide<+->
	可逆线性变换 $\bfx=\bfP\bfy$ 不会影响正定性.
	\onslide<+->
	$\bfA$ 正定$\iff \bfP^\rmT\bfA\bfP$ 正定.
\end{frame}


\begin{frame}{例: 正定和负定}
	\onslide<+->
	\begin{example}
		\begin{enumerate}
			\item $f(x_1,x_2,x_3)=x_1^2+x_2^2$ 半正定.
			\item $f(x_1,x_2,x_3)=x_1^2-2x_2^2+x_3^2$ 不定.
			\item $f(x_1,x_2,x_3)=(x_1+x_2)^2+(x_2+x_3)^2+(x_3+x_1)^2$ 正定.
			\item $f(x_1,x_2,x_3)=(x_1-x_2)^2+(x_2-x_3)^2+(x_3-x_1)^2$ 半正定.
			\item 椭球面 $f(x,y,z)=1$ 对应的二次型正定.
			\item 单叶/双叶双曲面 $f(x,y,z)=1$ 对应的二次型不定.
		\end{enumerate}
	\end{example}
\end{frame}


\begin{frame}{正定的判定}
	\onslide<+->
	\begin{theorem}
		设 $\bfA$ 是 $n$ 阶实对称阵, $f=\bfx^\rmT\bfA\bfx$. 如下命题等价:
		\begin{enumerate}
			\item $\bfA>0$ 正定, 即 $f$ 正定.
			\item $f$ 的正惯性指数为 $n$, 即 $\bfA$ 特征值全为正.
			\item 存在正交阵 $\bfP$ 使得 $\bfA=\bfP^\rmT\bfP$.
			\item (赫尔维茨定理) $\bfA$ 的各阶顺序主子式都为正, 即
			\[a_{11}>0,\quad
			\begin{vmatrix}
				a_{11}&a_{12}\\
				a_{21}&a_{22}
			\end{vmatrix}>0,\quad
			\begin{vmatrix}
				a_{11}&a_{12}&a_{13}\\
				a_{21}&a_{22}&a_{23}\\
				a_{31}&a_{32}&a_{33}
			\end{vmatrix}>0,\quad
			\dots,\quad
			|\bfA|>0.\]\label{enum:hurwitz}
		\end{enumerate}
		\vspace{-\baselineskip}
	\end{theorem}
	\onslide<+->
	将\enumref{enum:hurwitz}中 $>$ 换成 $\ge$ 即可判断半正定, 这也等价于 $f$ 的负惯性指数为 $0$, 即 $\bfA$ 特征值全非负.
\end{frame}


\begin{frame}{例: 正定的性质和判定}
	\beqskip{5pt}
	\onslide<+->
	\begin{corollary}
		若 $\bfA$ 正定, 则 $|\bfA|>0$ 且对角元全为正.
	\end{corollary}
	\onslide<+->
	\begin{example}
		若 $f(x_1,x_2,x_3)=2x_1^2+x_2^2+x_3^2+2x_1x_2+tx_2x_3$ 正定, 求 $t$ 的取值范围.
	\end{example}
	\onslide<+->
	\begin{solution}
		$f$ 对应 $\bfA=\begin{pmatrix}
			2&1&0\\
			1&1&t/2\\
			0&t/2&1
		\end{pmatrix}$, 
		\onslide<+->{%
		顺序主子式
		\[
			2>0,\quad
			\begin{vmatrix}
				2&1\\1&1
			\end{vmatrix}=1>0,\quad
			|\bfA|=1-\dfrac{t^2}2>0
		\]
		得到 $-\sqrt2<t<\sqrt2$.
		}
	\end{solution}
	\endgroup
\end{frame}


\begin{frame}{例: 求二次型的规范形}
	\onslide<+->
	\begin{example}
		求可逆线性变换 $\bfx=\bfP\bfy$二次型 $f=-5x_1^2+6x_2^2-2x_3^2+6x_1x_2+4x_1x_3$ 化为规范形.
	\end{example}
	\onslide<+->
	\begin{solution}
		\begin{align*}
			f&=-2(x_3-x_1)^2-3x_1^2+6x_2^2+6x_1x_2\\
			&=-2(x_3-x_1)^2-3(x_1-x_2)^2+9x_2^2
		\end{align*}
		\onslide<+->{%
		因此 $\bfx=\bfy\bfP,\bfP=\begin{pmatrix}
			-\sqrt2&\sqrt3&0\\
			0&-\sqrt3&3\\
			\sqrt2&0&0\\
		\end{pmatrix}^{-1}=\begin{pmatrix}
			0&0&\dfrac1{\sqrt2}\\
			\dfrac1{\sqrt3}&0&\dfrac1{\sqrt3}\\
			\dfrac13&\dfrac13&\dfrac13
		\end{pmatrix}$,
		将 $f$ 化为规范形 $f=-y_1^2-y_2^2+y_3^2$,
		}
	\end{solution}
\end{frame}


\begin{frame}{例: 正定的性质和判定}
	\begin{example}
		若实对称阵 $\bfA$ 正定, 证明 $|\bfA+\bfE|>1$.
	\end{example}
	\onslide<+->
	\begin{proof}
		由 $\bfA$ 正定可知其特征值均为正, 从而 $\bfA+\bfE$ 特征值都大于 $1$.
		\onslide<+->{%
		从而 $|\bfA+\bfE|>1$.\qedhere
		}
	\end{proof}
	\onslide<+->
	\begin{example}
		设 $3$ 阶实对称阵 $\bfA$ 满足 $\bfA^2+2\bfA=\bfO, \rank(\bfA)=2$. 当 $k$ 为何值时, 矩阵 $\bfA+k\bfE$ 为正定矩阵.
	\end{example}
	\onslide<+->
	\begin{solution}
		由 $\bfA^2+2\bfA=\bfO$ 可知 $\bfA$ 的特征值满足 $\lambda^2+2\lambda=0,\lambda=0,-2$.
		\onslide<+->{%
		由 $\rank(\bfA)=2$ 可知 $\bfA$ 特征值为 $0,-2,-2$,
		}\onslide<+->{%
		$\bfA+k\bfE$ 特征值为 $k,k-2,k-2$.
		}\onslide<+->{%
		因此 $k>2$.
		}
	\end{solution}
\end{frame}


\begin{frame}{例: 正定的性质和判定}
	\onslide<+->
	\begin{example}
		设 $\bfA$ 是 $m\times n$ 实矩阵且 $\rank(\bfA)=n$.
		证明 $\bfA^\rmT\bfA$ 正定.
	\end{example}
	\onslide<+->
	\begin{proof}
		显然 $\bfA^\rmT\bfA$ 是对称的.
		\onslide<+->{%
			注意到
			\[\bfx^\rmT(\bfA^\rmT\bfA)\bfx=(\bfA\bfx)^\rmT\bfA\bfx=\|\bfA\bfx\|^2.\]
		}\onslide<+->{%
			由于 $\rank(\bfA)=n$, $\bfA\bfx={\bf0}$ 只有零解.
		}\onslide<+->{%
			因此当 $\bfx\neq{\bf0}$ 时, $\bfA\bfx\neq{\bf0}$, 从而
			\[\bfx^\rmT(\bfA^\rmT\bfA)\bfx=\|\bfA\bfx\|^2>0.\qedhere\]
		}\vspace{-\baselineskip}
	\end{proof}
\end{frame}


\begin{frame}{例: 正定的性质和判定}
	\onslide<+->
	\begin{example}
		设 $\bfA$ 为 $n$ 阶实对称矩阵.
		证明 $\rank(\bfA)=n\iff$ 存在一个 $n$ 阶实方阵 $\bfB$ 使得 $\bfA\bfB+\bfB^\rmT\bfA$ 正定.
	\end{example}
	\onslide<+->
	\begin{proof}
		显然 $\bfA\bfB+\bfB^\rmT\bfA$ 是对称的, 且
		\[\bfx^\rmT(\bfA\bfB+\bfB^\rmT\bfA)\bfx=(\bfA\bfx)^\rmT\bfB\bfx+(\bfB\bfx)^\rmT(\bfA\bfx)=2[\bfA\bfx,\bfB\bfx].\]
		\onslide<+->{%
		若 $\rank(\bfA)=n$, 令 $\bfB=\bfA$, 则当 $\bfx\neq{\bf0}$ 时, $\bfA\bfx\neq{\bf0}$, 从而
		\[
			[\bfA\bfx,\bfB\bfx]=\|\bfA\bfx\|^2>0.
		\]
		}\onslide<+->{%
		若 $\rank(\bfA)<n$, 则存在 $\bfx\neq{\bf0}$ 使得 $\bfA\bfx={\bf0}$, 从而 $[\bfA\bfx,\bfB\bfx]=0$, $\bfA\bfB+\bfB^\rmT\bfA$ 不正定.\qedhere
		}
	\end{proof}
\end{frame}


\begin{frame}{例: 正定的性质和判定}
	\beqskip{2pt}
	\onslide<+->
	\begin{example}
		设 $x$ 是实数, 证明 $\begin{vmatrix}
			10&2&3&5\\
			2&10&1&-1\\
			3&1&10&x\\
			5&-1&x&10
		\end{vmatrix}\le 10000$.
	\end{example}
	\onslide<+->
	\begin{proof}
		设 $\bfA$ 为题述方阵.
	\onslide<+->{%
		它的前三个顺序主子式
		\[10>0,\quad\begin{vmatrix}
			10&2\\
			2&10
		\end{vmatrix}=96>0,\quad
		\begin{vmatrix}
			10&2&3\\
			2&10&1\\
			3&1&10
		\end{vmatrix}=872>0,\]
	}\onslide<+->{%
		若 $|\bfA|\le0$, 命题显然成立.
	}\onslide<+->{%
		若 $|\bfA|>0$, 则 $\bfA$ 正定, 从而特征值全正.
	}\onslide<+->{%
		因此
		\[|\bfA|=\lambda_1\lambda_2\lambda_3\lambda_4\le\Bigl(\frac{\lambda_1+\lambda_2+\lambda_3+\lambda_4}4\Bigr)^4=10000.\qedhere\]
	}\vspace{-.5\baselineskip}
	\end{proof}
	\endgroup
\end{frame}


\begin{frame}{正定的应用: 函数极值\noexer}
	\onslide<+->
	实对称阵可用于判断多元函数的极值.
	\onslide<+->
	设 $f(\bfx)=f(x_1,\dots,x_n)$ 是一个 $n$ 元实函数, $\bfa$ 是其定义域内一点, 且 $f$ 在 $\bfa$ 附近具有连续的二阶偏导数.
	\onslide<+->
	记 $f''_{ij}=\dfrac{\partial^2 f}{\partial x_i\partial x_j}$, 则 $f''_{ij}=f''_{ji}$.
	\onslide<+->
	于是 $\bfA=(f''_{ij}(\bfa))$ 是 $n$ 阶实对称阵.
	\onslide<+->
	\begin{theorem}
		设 $f$ 在 $\bfa$ 处各阶偏导均为零.
		\begin{enumerate}
			\item 若 $\bfA$ 正定, 则 $f$ 在 $\bfa$ 处取极小值;
			\item 若 $\bfA$ 负定, 则 $f$ 在 $\bfa$ 处取极大值.
		\end{enumerate}
	\end{theorem}
	\onslide<+->
	若 $\bfA$ 不定, 则无法判断 $\bfa$ 是否是极值点.
\end{frame}


\begin{frame}{合同的应用: 奇异值分解\noexer}
	\onslide<+->
	对于一般的 $m\times n$ 实矩阵 $\bfA$, 我们有\emph{奇异值分解}
	\[\bfA=\bfU_{m\times m}\bmS_{m\times n}\bfV^\rmT_{n\times n},\]
	其中 $\bfU,\bfV$ 分别是 $m,n$ 阶正交阵, $\bmS$ 是 $m\times n$ 型对角阵, 对角元非负且按降序排列.

	\onslide<+->
	首先对 $\bfA^\rmT\bfA$ 这一半正定对称阵做正交合同对角化
	\[\bfA^\rmT\bfA=\bfV^\rmT\bmL\bfV,\quad
	\bfV=(\bfv_1,\dots,\bfv_n),\quad
	\bmL=\diag(\lambda_1,\dots,\lambda_n),\lambda_1\ge\cdots\ge\lambda_n\ge0.\]
	\onslide<+->
	\emph{奇异值}是指 $\sigma_i=\sqrt{\lambda_i}$.
	\onslide<+->
	令 $\bmS\in M_{m\times n}$ 为对角阵, 对角元为 $\sigma_1,\dots,\sigma_n$.

	\onslide<+->
	对于 $1\le j\le r=\rank(\bfA)$, 令 $\bfu_j=\dfrac1{\sigma_j}\bfA \bfv_j$, 
	\onslide<+->
	设 $\bfu_{r+1},\dots,\bfu_m$ 是 $\bfA^\rmT\bfx={\bf0}$ 的一组标准正交基础解系,
	\onslide<+->
	则 $\bfU=(\bfu_1,\dots,\bfu_m)$ 是正交阵, 且 $\bfA=\bfU\bmS\bfV^\rmT$.
\end{frame}



\begin{frame}{合同的应用: 奇异值分解与压缩/降噪算法\noexer}
	\onslide<+->
	注意到, 如果 $\bfU'$ 是 $\bfU$ 的前 $r$ 列, $\bfV'$ 是 $\bfV$ 的前 $r$ 行, $\bmS'$ 是 $\bmS$ 的前 $r$ 行 $r$ 列,
	\onslide<+->
	则
	\[\bfA=\bfU'_{m\times r}\bmS'_{r\times r}\bfV'_{r\times n},\]
	其中 $\bmS'$ 是 $\bfA$ 奇异值降序的对角阵, $\bfU',\bfV'$ 为列/行为标准向量且两两正交的矩阵.

	\onslide<+->
	这意味着当 $r$ 相比 $m,n$ 较小时, 只需存储 $(m+n+1)r$ 个元素即可还原 $\bfA$.
	\onslide<+->
	这是一种\emph{无损压缩}算法.
	\onslide<+->
	如果我们只截取前 $k<r$ 个奇异值以及对应的 $\bfU,\bfV$ 部分, 则可以对 $\bfA$ 进行\emph{有损压缩}到 $\bfA'$.
	\onslide<+->
	例如 $\bfA$ 表示一张图像的像素信息, 保留它较大的奇异值往往对它的信息影响很小.
	\onslide<+->
	有时候, 我们甚至需要主动舍弃较小的奇异值, 只保留较大的奇异值来实现\emph{信号降噪}.

	\onslide<+->
	矩阵还有诸如 $LU$ 分解, $QR$ 分解, 科列斯基分解等.
	这些分解往往都在压缩或降噪中发挥着作用.

	\onslide<+->
	当 $m=n=3$ 时, 可以看出线性变换可以分解为旋转、放缩、旋转的复合.
\end{frame}
