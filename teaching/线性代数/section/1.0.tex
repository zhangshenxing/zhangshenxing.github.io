% \section{行列式的引入}

\begin{frame}{二元线性方程组\noexer}
	\onslide<+->
	线性代数起源于线性方程组的求解问题.
	\onslide<+->
	考虑二元线性方程组
	\begin{laeqn}
		3x_1-2x_2=12,&\\
		2x_1+x_2=1.
	\end{laeqn}
	\onslide<+->
	2(2)+(1)可得 $7x_1=14$.
	\onslide<+->
	从而 $x_1=2,x_2=-3$.
\end{frame}


\begin{frame}{二元线性方程组\noexer}
	\onslide<+->
	考虑一般情形:
	\begin{laeqn}
		a_{11}x_1+a_{12}x_2=b_1,&\\
		a_{21}x_1+a_{22}x_2=b_2.
	\end{laeqn}
	\onslide<+->
	分别作 $a_{22}\times (1), a_{12}\times (2)$ 得到
	\begin{laeqn}
		a_{22}a_{11}x_1+a_{22}a_{12}x_2=b_1 a_{22},&\\
		a_{12}a_{21}x_1+a_{12}a_{22}x_2=b_2 a_{12}.
	\end{laeqn}
	\onslide<+->
	然后 $(1)-(2)$ 得到
	\[(a_{11}a_{22}-a_{12}a_{21})x_1=b_1a_{22}-b_2a_{12}.\]
\end{frame}


\begin{frame}{二元线性方程组\noexer}
	\onslide<+->
	于是
	\[x_1=\frac{b_1 a_{22}-b_2 a_{12}}{a_{11}a_{22}-a_{12}a_{21}}.\]
	\onslide<+->
	类似地, 从 $-a_{21}(1)+a_{12}(2)$ 得到
	\[x_2=\frac{-a_{21}b_1+a_{11}b_2}{a_{11}a_{22}-a_{12}a_{21}}.\]

	\onslide<+->
	有没有问题?
	\onslide<+->
	当 $a_{11}a_{22}-a_{12}a_{21}=0$ 时, 不能使用这种方式求解.
	\onslide<+->
	实际上此时无解或有无穷多个解.
	
	\onslide<+->
	当 $a_{11}a_{22}-a_{12}a_{21}\neq0$ 时, 方程组有唯一解.
	\onslide<+->
	所以这个数值就充当了该方程组``判别式''的作用.
	
	\onslide<+->
	对于 $n$ 个未知数 $n$ 个方程的线性方程组, 能不能定义出类似的量来刻画它何时有唯一解呢? 
	\onslide<+->
	这便是\emph{行列式}的由来.
\end{frame}


% \begin{frame}{二阶行列式\noexer}
% 	\onslide<+->
% 	我们记
% 	\[\detm{a_{11}&a_{12}\\a_{21}&a_{22}}:=a_{11}a_{22}-a_{12}a_{21}.\]
% 	\onslide<+->
% 	那么我们发现当 $\detm{a_{11}&a_{12}\\a_{21}&a_{22}}\neq 0$ 时, 上述线性方程组有唯一解:
% 	\[x_1=\frac{\detm{b_1&a_{12}\\b_2&a_{22}}}{\detm{a_{11}&a_{12}\\a_{21}&a_{22}}},\quad
% 	x_2=\frac{\detm{a_{11}&b_1\\a_{21}&b_2}}{\detm{a_{11}&a_{12}\\a_{21}&a_{22}}}.\]
% \end{frame}

