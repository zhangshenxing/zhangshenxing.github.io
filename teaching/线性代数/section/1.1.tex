\section{行列式的定义}

\subsection{行列式的归纳定义}
\begin{frame}{矩阵和方阵}
	\onslide<+->
	对于 $n$ 个未知数 $n$ 的方程的线性方程组, 我们能不能定义出类似的量来刻画它何时有唯一解呢? 
	
	\onslide<+->
	首先引入矩阵的概念.
	\onslide<+->
	我们将 $mn$ 个数按照每行 $n$ 个元素, 一共 $m$ 行排列, 得到的数表称为 \emph{$m$ 行 $n$ 列矩阵}, 或简称为 \emph{$m\times n$ 矩阵}:
	\[\bfA=\begin{bmatrix}
		a_{11}&a_{12}&\cdots &a_{1n}\\
		a_{21}&a_{22}&\cdots &a_{2n}\\
		\vdots&\vdots&\ddots&\vdots\\
		a_{m1}&a_{m2}&\cdots &a_{mn}
	\end{bmatrix}.\]
	\onslide<+->
	其中 $a_{ij}$ 表示 $\bfA$ 的第 $i$ 行 $j$ 列元素, 并记 $\bfA=(a_{ij})_{m\times n}$.

	\onslide<+->
	当 $m=n$ 时, 称之为 \emph{$n$ 阶方阵}.
\end{frame}


\begin{frame}{线性方程组与矩阵}
	\onslide<+->
	对于线性方程组
	\begin{laeq*}
		a_{11}x_1+a_{12}x_2+\cdots+a_{1n}x_n&=&b_1\\
		a_{21}x_1+a_{22}x_2+\cdots+a_{2n}x_n&=&b_2\\
		\vdots&&\\
		a_{m1}x_1+a_{m2}x_2+\cdots+a_{mn}x_n&=&b_m,
	\end{laeq*}
	未知量 $x_1,\dots,x_n$ 前面的系数就构成了一个 $m\times n$ 矩阵, 称之为\emph{系数矩阵}
	\[\bfA=\begin{bmatrix}
		a_{11}&a_{12}&\cdots &a_{1n}\\
		a_{21}&a_{22}&\cdots &a_{2n}\\
		\vdots&\vdots&\ddots&\vdots\\
		a_{m1}&a_{m2}&\cdots &a_{mn}
	\end{bmatrix}.\]
	\onslide<+->
	线性代数的主要任务之一, 就是利用矩阵理论回答该方程组解的情况.
\end{frame}


\begin{frame}{$2$ 阶行列式}
	\onslide<+->
	为了定义出刻画 $n$ 个方程 $n$ 个变量的线性方程组是否有唯一解的``判别式'', 即矩阵 $\bfA$ 的\emph{行列式 $\det(\bfA)=|A|$}, 首先约定\emph{单位矩阵}
	\[E_n:=\begin{bmatrix}
		1&&&\\
		&1&&\\
		&&\ddots&\\
		&&&1
	\end{bmatrix}\]
	行列式为 $1$.
	\onslide<+->
	作此约定之后, $2$ 阶方阵的行列式就应当为
	\[\detm{a_{11}&a_{12}\\a_{21}&a_{22}}:=a_{11}a_{22}-a_{12}a_{21}
	\begin{tikzpicture}[overlay,xshift=-40mm,yshift=.5mm,visible on=<5->]
		\draw[cstcurve,dcolora] (-.2,.2)--(.2,-.2);
		\draw[cstcurve,dcolorb] (-.2,-.2)--(.2,.2);
	\end{tikzpicture}\]
	而不是它的一个非零倍数了.
\end{frame}


\begin{frame}{$3$ 阶行列式}
	\onslide<+->
	对于 $3$ 阶方阵, 可通过计算发现其行列式为(注意单位矩阵行列式应当为 $1$)
	\[\detm{a_{11}&a_{12}&a_{13}\\a_{21}&a_{22}&a_{23}\\
	a_{31}&a_{32}&a_{33}}:=
	a_{11}a_{22}a_{33}+a_{12}a_{23}a_{31}+a_{13}a_{21}a_{32}
	-a_{11}a_{23}a_{32}-a_{12}a_{21}a_{33}-a_{13}a_{22}a_{31}
	\begin{tikzpicture}[overlay,xshift=-133mm,yshift=.5mm,visible on=<3->]
		\draw[cstcurve,dcolora] (-.75,.5)--(.75,-.5);
		\draw[cstcurve,dcolorb] (0,.5)--(.9,-.1);
		\draw[cstcurve,dcolorb] (-1.05,-.3)--(-.6,-.6);
		\draw[cstcurve,dcolorc] (.6,.6)--(.9,.4);
		\draw[cstcurve,dcolorc] (-.9,.1)--(.15,-.6);
	\end{tikzpicture}
	\]
	\onslide<+->
	\[=a_{11}\detm{a_{22}&a_{23}\\a_{32}&a_{33}}
	-a_{12}\detm{a_{21}&a_{23}\\a_{31}&a_{33}}
	+a_{13}\detm{a_{21}&a_{22}\\a_{31}&a_{32}}.\]
\end{frame}


\begin{frame}{$n$ 阶方阵的行列式}
	\onslide<+->
	对于 $n$ 阶方阵 $\bfA=(a_{ij})$, 按照如下方式定义行列式
	\[\det(\bfA)=\detm{
		a_{11}&a_{12}&\cdots &a_{1n}\\
		a_{21}&a_{22}&\cdots &a_{2n}\\
		\vdots&\vdots&\ddots&\vdots\\
		a_{m1}&a_{m2}&\cdots &a_{mn}}.\]
		\onslide<+->
	\begin{definition}
		\begin{itemize}
			\item 当 $n=1$ 时, $\det(\bfA):=a_{11}$;
			\item 对于一般的 $n$, 归纳定义
			\[\det(\bfA)=a_{11}M_{11}-a_{12}M_{12}+a_{13}M_{13}-\cdots+(-1)^{n+1}a_{1n}M_{1n},\]
			其中 $M_{ij}$ 表示 $\bfA$ 去掉第 $i$ 行和 $j$ 列得到的 $n-1$ 阶方阵的行列式.
		\end{itemize}
	\end{definition}
\end{frame}


\begin{frame}{余子式和代数余子式}
	\onslide<+->
	称 $M_{ij}$ 为 $a_{ij}$ 的\emph{余子式}; 称 $A_{ij}=(-1)^{i+j}M_{ij}$ 为 $a_{ij}$ 的\emph{代数余子式}.
	\onslide<+->
	那么
	\[\alert{\det(\bfA)=a_{11}A_{11}+a_{12}A_{12}+a_{13}A_{13}+\cdots+a_{1n}A_{1n}.}\]
	\vspace{-\baselineskip}
	\onslide<+->
	\begin{example}
		$\detm{1&3&2\\3&-5&1\\2&1&4}$
		\onslide<+->{$=1\times (-5)\times 4+3\times 1\times2+2\times3\times1-1\times1\times1-3\times3\times4-2\times(-5)\times2$}\\
		\onslide<+->{$=-20+6+6-1-36+20=-25.$}
	\end{example}
	\onslide<+->
	\begin{example}
		$\detm{1&3&2\\3&-5&1\\2&1&4}$
		\onslide<+->{$=1\times\detm{-5&1\\1&4}-3\times\detm{3&1\\2&4}+2\detm{3&-5\\2&1}$}
		\onslide<+->{$=-21-3\times10+2\times13=-25.$}
	\end{example}
\end{frame}


\begin{frame}{注记}
	\onslide<+->
	\begin{exercise}
		\vspace{-.4\baselineskip}
		如果 $k>0$ 且 $\detm{k&2&1\\2&k&1\\k&1&2}=0$, 那么 $k=$\fillblank{\visible<+->{$2$}}.
	\end{exercise}
	\onslide<+->
	\begin{enumerate}
		\item 行列式是一个数, 或者说 $\det:M_n(\BR)\to\BR$ 是一个映射, 其中 $M_n(\BR)$ 表示所有实系数的 $n$ 阶方阵.
		\item $1$ 阶行列式就是方阵里面唯一的那个元素, 尽管也记作 $|\cdot|$, 但注意和绝对值区分开.
		\item $2,3$ 阶行列式可以用对角线法直接得到展开式, 但是更高阶的没有这种表示方法.
		\item 
		$\detm{
			a_1&&&\\
			&a_2&&\\
			&&\ddots&\\
			&&&a_n
		}=a_1a_2\cdots a_n.$
	\end{enumerate}
\end{frame}


\begin{frame}{$4$ 阶行列式无对角线法}
	\onslide<+->
	\begin{exercise}
		判断题: $\detm{1&&&\\&2&&\\&&3&\\&&&4}=-\detm{&&&1\\&&2&\\&3&&\\4&&&}$. \visible<+->{\Huge\color{red}{$\times$}}
	\end{exercise}
	\onslide<+->
	\begin{example}
		\begin{align*}
			\detm{a_{11}&      &      &\\
			a_{21}&a_{22}&      &\\
			\vdots&\vdots&\ddots&\\
			a_{n1}&a_{n2}&\cdots&a_{nn}}
		&\onslide<+->{=a_{11}\detm{
			a_{22}&      &\\
			\vdots&\ddots&\\
			a_{n2}&\cdots&a_{nn}
		}}
		\onslide<+->{=a_{11}a_{22}\detm{
			a_{33}&      &\\
			\vdots&\ddots&\\
			a_{n3}&\cdots&a_{nn}
		}}\\
		&\onslide<+->{=\cdots=a_{11}a_{22}\cdots a_{nn}.}
		\end{align*}
	\end{example}
\end{frame}


\subsection{行列式的展开}
\begin{frame}{行列式的展开}
	\onslide<+->
	我们知道,
	\begin{align*}	
	\det(\bfA)&=a_{11}A_{11}+a_{12}A_{12}+a_{13}A_{13}+\cdots+a_{1n}A_{1n}\\
	&=a_{11}M_{11}-a_{12}M_{12}+a_{13}M_{13}-\cdots+(-1)^{n+1}a_{1n}M_{1n}.
	\end{align*}
	\onslide<+->
	一直展开下去, $\det(\bfA)$ 是由一些 $\pm a_{1 k_1}a_{2 k_2}\cdots a_{n k_n}$ 相加得到, 其中 $k_1 k_2 \cdots k_n$ 是 $1,2\dots,n$ 的一个排列, 一共有 $n!$ 个这样的项.
	\onslide<+->
	问题是如何确定它的符号?
	
	\onslide<+->
	假设 $n$ 阶矩阵 $\bfP=(a_{ij})$ 第 $i$ 行除了 $k_i$ 列都是零.
	那么
	\[\det(\bfP)=\pm a_{1 k_1}a_{2 k_2}\cdots a_{n k_n}.\]
	\onslide<+->
	为了陈述方便, 记 \alert{$c_i\swap c_j (r_i\swap r_j)$} 为交换 $i,j$ 列(行).
	\onslide<+->
	如果 $c_{k_1-1}\swap c_{k_1}$, 那么 $\det(\bfP)$ 定义中的余子式不会变, 但是符号 $(-1)^{1+k_1}$ 和 $(-1)^{1+k_1-1}$ 会不同, 从而行列式相差 $-1$ 倍.
	\onslide<+->
	再依次
	\[c_{k_1-2}\swap c_{k_1-1},\quad c_{k_1-3}\swap c_{k_1-2},\quad \dots,\quad c_1\swap c_2\]
	则 $a_{1k_1}$ 被移动到了第 $1$ 列且余子式没有变, 行列式变为 $(-1)^{k_1-1}$ 倍.
\end{frame}


\begin{frame}{行列式展开项的符号}
	\onslide<+->
	后面的每行都类似操作, 最终 $\bfP$ 变成\emph{对角阵}
	\[\bfD=\begin{bmatrix}
		a_{1k_1}&&&\\
		&a_{2k_2}&&\\
		&&\ddots&\\
		&&&a_{nk_n}
	\end{bmatrix},\]
	\onslide<+->
	所以
	\[\det(\bfP)=(-1)^a\det(\bfD)=(-1)^a a_{1k_1}a_{2k_2}\cdots a_{nk_n},\]
	其中 $a$ 是需要交换 $\bfP$ 的相邻列使其变成对角阵的次数.
\end{frame}


\begin{frame}{行列式展开式}
	\onslide<+->
	由于 $c_i\swap c_j$ 可以通过
	\[c_j\swap c_{j-1},\quad c_{j-1}\swap c_{j-2},\quad \dots,\quad, c_{i+1}\swap c_i,\qquad
	c_{i+1}\swap c_{i+2},\quad \dots,\quad c_{j-1}\swap c_j\]
	实现, 一共 $2(j-i)+1$ 次.
	\onslide<+->
	因此 $a$ 可以换成交换 $\bfP$ 的列使其变成对角阵的次数.

	\onslide<+->
	\begin{theorem}
	如果排列 $k_1k_2\cdots k_n$ 经过 $a$ 次互换变为 $12\cdots n$, 记 $\sgn(k_1k_2\cdots k_n):=(-1)^a$.
	那么对于 $n$ 阶方阵 $\bfA$,
		\[\det(\bfA)=\sum \sgn(k_1k_2\cdots k_n) a_{1k_1}a_{2k_2}\cdots a_{nk_n},\]	
	其中求和取遍所有排列 $k_1k_2\cdots k_n$.
	\end{theorem}
\end{frame}


\begin{frame}{反对角阵的行列式}
	\onslide<+->
	\begin{example}
		计算 $\det(\bfA)$, 其中 $\bfA=\begin{bmatrix}
			&&&a_1\\&&a_2&\\&\udots&&\\a_n&&&
		\end{bmatrix}$.
	\end{example}
	\onslide<+->
	\begin{solution*}
		当 $n$ 是偶数时, 排列 $\set{n,n-1,\cdots,2,1}$ 经过对换 $(1n),(2,n-1),\dots,(\frac n2,\frac n2+1)$ 变为 $(12\cdots n)$,
		\onslide<+->{因此 $\det(\bfA)=(-1)^{\frac n2}a_1a_2\cdots a_n$.}

		\onslide<+->{当 $n$ 是奇数时, 排列 $(n,n-1,\cdots,2,1)$ 经过对换 $(1n),(2,n-1),\dots,(\frac{n-1}2,\frac{n+3}2)$ 变为 $(12\cdots n)$,}
		\onslide<+->{因此 $\det(\bfA)=(-1)^{\frac{n-1}2}a_1a_2\cdots a_n$.}

		\onslide<+->{也可以统一表为
		\[\det(\bfA)=(-1)^{\frac{n(n-1)}2}a_1a_2\cdots a_n.\]}
		\vspace{-\baselineskip}
	\end{solution*}
\end{frame}
