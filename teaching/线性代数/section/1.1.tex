\section{行列式的定义}

\subsection{行列式的归纳定义}
\begin{frame}{矩阵和方阵}
	\onslide<+->
	本章我们将回答如何判断 $n$ 个未知数 $n$ 个方程的线性方程组何时有唯一解.

	\onslide<+->
	首先引入矩阵的概念.
	\onslide<+->
	将 $mn$ 个数按照每行 $n$ 个元素, 一共 $m$ 行排列, 得到的数表称为 \emph{$m$ 行 $n$ 列矩阵}, 或简称为 \emph{$m\times n$ 矩阵}:
	\[\bfA=\begin{pmatrix}
		a_{11}&a_{12}&\cdots &a_{1n}\\
		a_{21}&a_{22}&\cdots &a_{2n}\\
		\vdots&\vdots&\ddots&\vdots\\
		a_{m1}&a_{m2}&\cdots &a_{mn}
	\end{pmatrix}.\]
	\onslide<+->
	其中 $a_{ij}$ 表示 $\bfA$ 的第 $i$ 行 $j$ 列元素, 并记 $\bfA=(a_{ij})_{m\times n}$.

	\onslide<+->
	当 $m=n$ 时, 称之为 \emph{$n$ 阶方阵}.
\end{frame}


\begin{frame}{线性方程组与矩阵}
	\onslide<+->
	对于线性方程组
	\[\laeq{
		a_{11}x_1+a_{12}x_2+\cdots+a_{1n}x_n=b_1\\
		a_{21}x_1+a_{22}x_2+\cdots+a_{2n}x_n=b_2\\
		\vvdots\\
		a_{m1}x_1+a_{m2}x_2+\cdots+a_{mn}x_n=b_m
	}\]
	\onslide<+->
	未知量 $x_1,\dots,x_n$ 前面的系数就构成了一个 $m\times n$ 矩阵, 称之为\emph{系数矩阵}
	\[\bfA=\begin{pmatrix}
		a_{11}&a_{12}&\cdots &a_{1n}\\
		a_{21}&a_{22}&\cdots &a_{2n}\\
		\vdots&\vdots&\ddots&\vdots\\
		a_{m1}&a_{m2}&\cdots &a_{mn}
	\end{pmatrix}.\]
\end{frame}


\begin{frame}{$2$ 阶行列式}
	\onslide<+->
	当 $m=n$ 时, 方阵 $\bfA$ 的\emph{行列式 $\det(\bfA)=|A|$} 就是用于刻画上述方程组是否有唯一解的``判别式''.
	\onslide<+->
	首先约定\emph{单位矩阵}
	\[\alert{\bfE_n=\bfI_n:=\begin{pmatrix}
		1&&&\\
		&1&&\\
		&&\ddots&\\
		&&&1
	\end{pmatrix}}\qquad 
	(\text{即方程组}
	\left\{\begin{array}{l}
		x_1=b_1\\
		x_2=b_2\\
		\vdots\\
		x_n=b_n
	\end{array}\right.\text{的系数矩阵})
	\]
	行列式为 $1$.
	\onslide<+->
	作此约定之后, $2$ 阶方阵的行列式就应当为
	\[\det(\bfA)=|A|=\detm{a_{11}&a_{12}\\a_{21}&a_{22}}:=a_{11}a_{22}-a_{12}a_{21}
	\begin{tikzpicture}[overlay,xshift=-40mm,yshift=.5mm,visible on=<4->]
		\draw[cstcurve,dcolora] (-.2,.2)--(.2,-.2);
		\draw[cstcurve,dcolorb] (-.2,-.2)--(.2,.2);
	\end{tikzpicture}\]
	而不是它的一个非零倍数了.
\end{frame}


\begin{frame}{$3$ 阶行列式}
	\onslide<+->
	对于 $3$ 阶方阵, 可通过计算发现其行列式为(注意 $|\bfE_3|=1$)
	\[\detm{a_{11}&a_{12}&a_{13}\\a_{21}&a_{22}&a_{23}\\
	a_{31}&a_{32}&a_{33}}:=
	a_{11}a_{22}a_{33}+a_{12}a_{23}a_{31}+a_{13}a_{21}a_{32}
	-a_{11}a_{23}a_{32}-a_{12}a_{21}a_{33}-a_{13}a_{22}a_{31}
	\begin{tikzpicture}[overlay,xshift=-133mm,yshift=.5mm,visible on=<3->]
		\draw[cstcurve,dcolora] (-.75,.5)--(.75,-.5);
		\draw[cstcurve,dcolorb] (0,.5)--(.9,-.1);
		\draw[cstcurve,dcolorb] (-1.05,-.3)--(-.6,-.6);
		\draw[cstcurve,dcolorc] (.6,.6)--(.9,.4);
		\draw[cstcurve,dcolorc] (-.9,.1)--(.15,-.6);
	\end{tikzpicture}
	\]
	\onslide<+->
	\[=a_{11}\detm{a_{22}&a_{23}\\a_{32}&a_{33}}
	-a_{12}\detm{a_{21}&a_{23}\\a_{31}&a_{33}}
	+a_{13}\detm{a_{21}&a_{22}\\a_{31}&a_{32}}.\]
\end{frame}


\begin{frame}{$n$ 阶方阵的行列式}
	\onslide<+->
	对于 $n$ 阶方阵 $\bfA=(a_{ij})$, 按照如下方式定义行列式 $|\bfA|$:
	\onslide<+->
	\begin{definition}
		\begin{itemize}
			\item 当 $n=1$ 时, $|\bfA|:=a_{11}$;
			\item 对于一般的 $n$, 归纳定义
			\[|\bfA|=a_{11}M_{11}-a_{12}M_{12}+a_{13}M_{13}-\cdots+(-1)^{n+1}a_{1n}M_{1n},\]
			其中 $M_{ij}$ 表示 $\bfA$ 去掉第 $i$ 行和 $j$ 列得到的 $n-1$ 阶方阵的行列式.
		\end{itemize}
	\end{definition}
	\onslide<+->
	\begin{definition}
		称 $M_{ij}$ 为 $a_{ij}$ 的\emph{余子式}; 称 $A_{ij}=(-1)^{i+j}M_{ij}$ 为 $a_{ij}$ 的\emph{代数余子式}.		
	\end{definition}
	\onslide<+->
	那么
	\[\alert{|\bfA|=a_{11}A_{11}+a_{12}A_{12}+a_{13}A_{13}+\cdots+a_{1n}A_{1n}.}\]
\end{frame}


\begin{frame}{余子式和代数余子式}
	\vspace{-.3\baselineskip}
	\onslide<+->
	\begin{example}
		$\detm{1&3&2\\3&-5&1\\2&1&4}$
		\onslide<+->{$=1\times (-5)\times 4+3\times 1\times2+2\times3\times1-1\times1\times1-3\times3\times4-2\times(-5)\times2$}

		\onslide<+->{\hspace{19.5mm}$=-20+6+6-1-36+20=-25.$}

		\onslide<+->{$\detm{1&3&2\\3&-5&1\\2&1&4}$
		\onslide<+->{$=1\times\detm{-5&1\\1&4}-3\times\detm{3&1\\2&4}+2\detm{3&-5\\2&1}$}}
		\onslide<+->{$=-21-3\times10+2\times13=-25.$}
	\end{example}
	\onslide<+->
	\begin{exercise}
		\vspace{-.4\baselineskip}
		如果 $k>0$ 且 $\detm{k&2&1\\2&k&1\\k&1&2}=0$, 那么 $k=$\fillblank{\visible<+->{$2$}}.
	\end{exercise}
\end{frame}


\begin{frame}{注记}
	\onslide<+->
	\begin{enumerate}
		\item 行列式将一个方阵映射到一个数.
		\item $1$ 阶行列式就是方阵里面唯一的那个元素, 尽管也记作 $|\cdot|$, 但注意和绝对值区分.
		\item $2,3$ 阶行列式可以用对角线法直接得到展开式, 但是更高阶的没有这种表示方法.
		\item \emph{对角阵}的行列式
		$\detm{
			a_1&&&\\
			&a_2&&\\
			&&\ddots&\\
			&&&a_n
		}=a_1a_2\cdots a_n.$
		\item 从行列式的归纳定义出发依次展开得到, $|\bfA|$ 是由一些 $\pm a_{1 k_1}a_{2 k_2}\cdots a_{n k_n}$ 相加得到, 其中 $k_1 k_2 \cdots k_n$ 取遍 $1,2,\dots,n$ 的所有\emph{排列}, 一共有 $n!$ 个这样的项, 其中一半取 $+$, 一半取 $-$ ($n\ge2$).
	\end{enumerate}
\end{frame}


\begin{frame}{$4$ 阶行列式无对角线法}
	\onslide<+->
	\begin{exercise}
		判断题: $\detm{1&&&\\&2&&\\&&3&\\&&&4}=-\detm{&&&1\\&&2&\\&3&&\\4&&&}$. \visible<+->{\Huge\color{red}{$\times$}}
	\end{exercise}
	\onslide<+->
	\begin{example}
		\begin{align*}
			\detm{a_{11}&      &      &\\
			a_{21}&a_{22}&      &\\
			\vdots&\vdots&\ddots&\\
			a_{n1}&a_{n2}&\cdots&a_{nn}}
		&\onslide<+->{=a_{11}\detm{
			a_{22}&      &\\
			\vdots&\ddots&\\
			a_{n2}&\cdots&a_{nn}
		}}
		\onslide<+->{=a_{11}a_{22}\detm{
			a_{33}&      &\\
			\vdots&\ddots&\\
			a_{n3}&\cdots&a_{nn}
		}}\\
		&\onslide<+->{=\cdots=a_{11}a_{22}\cdots a_{nn}.}
		\end{align*}
	\end{example}
\end{frame}


\begin{frame}{反对角阵的行列式}
	\onslide<+->
	\begin{example}
		计算 $|\bfA|$, 其中 $\bfA=\begin{pmatrix}
			&&&a_1\\&&a_2&\\&\udots&&\\a_n&&&
		\end{pmatrix}$.
	\end{example}
	\onslide<+->
	\begin{solution*}
		\begin{align*}
			|\bfA|&=(-1)^{n+1}a_1\detm{&&a_2\\&\udots&\\a_n&&}
			\visible<+->{=(-1)^{n+1}a_1\cdot (-1)^{n}a_2\detm{&&a_3\\&\udots&\\a_n&&}}\\
			&\visible<+->{=\cdots=\prod_{i=1}^n (-1)^{n-i}a_i}
			\visible<+->{=(-1)^{\frac{n(n-1)}2}a_1a_2\cdots a_n.}
		\end{align*}
	\end{solution*}
\end{frame}


\begin{frame}{例: 分块矩阵行列式}
	\onslide<+->
	\begin{example}
		设
		\[\bfA=\begin{pmatrix}
			a_{11}&\cdots&a_{1m}\\
			\vdots&\ddots&\vdots\\
			a_{m1}&\cdots&a_{mm}\\
		\end{pmatrix},\qquad
		\bfB=\begin{pmatrix}
			b_{11}&\cdots&b_{1n}\\
			\vdots&\ddots&\vdots\\
			b_{n1}&\cdots&b_{nn}\\
		\end{pmatrix},\]
		\[\bfC=\begin{pmatrix}
			a_{11}&\cdots&a_{1m}&&&\\
			\vdots&\ddots&\vdots&&0&\\
			a_{m1}&\cdots&a_{mm}&&&\\
			*&\cdots&*&b_{11}&\cdots&b_{1n}\\
			\vdots&\ddots&\vdots&\vdots&\ddots&\vdots\\
			*&\cdots&*&b_{n1}&\cdots&b_{nn}\\
		\end{pmatrix}.\]
		证明 $|\bfC|=|\bfA|\cdot|\bfB|$.
	\end{example}
\end{frame}


\begin{frame}{例: 分块矩阵行列式}
	\onslide<+->
	\begin{proof*}
		对 $m$ 归纳.
		\onslide<+->{当 $m=1$ 时由行列式定义可知成立.}

		\onslide<+->{%
			假设命题对于 $m-1$ 成立.
		}\onslide<+->{%
			设 $\bfA$ 在 $(1,j)$ 处的余子式为 $M_{1j}$, $\bfC$ 在 $(1,j)$ 处的余子式为 $N_{1j}$.
		}\onslide<+->{%
			则由归纳假设 $N_{1j}=M_{1j}|\bfB|$.
		}\onslide<+->{%
			因此
			\begin{align*}
				|\bfC|&=\sum_{j=1}^m (-1)^{1+j}a_{1j}N_{1j}\\
				&=\sum_{j=1}^m (-1)^{1+j}a_{1j}M_{1j}|\bfB|
				=|\bfA|\cdot|\bfB|.\qedhere
			\end{align*}
		}
	\end{proof*}
\end{frame}


\subsection{行列式的几何意义}
\begin{frame}{二阶行列式的几何意义}
	\onslide<+->
	设平面上有平行四边形 $OACB$, 其中 $A(a,b), B(c,d)$.
	\begin{center}
	\begin{tikzpicture}[scale=.8]
		\draw[cstcurve,dcolora] (0,0)--(3,0)--(4,1.5)--(1,1.5)--cycle;
		\draw (0,0) node[dcolorb,left] {$O$};
		\draw (3,0) node[dcolorb,right] {$A$};
		\draw (1,1.5) node[dcolorb,left]{$B$};
		\draw (4,1.5) node[dcolorb,right] {$C$};
	\end{tikzpicture}
	\end{center}
	\onslide<+->
	二阶行列式 $\detm{a&b\\c&d}$ 的绝对值就是它的面积.
	\onslide<+->
	它的符号则表示从 $OA$ 沿最短角度旋转到 $OB$ 方向是逆时针还是顺时针.
\end{frame}


\begin{frame}{三阶行列式的几何意义}
	\onslide<+->
	类似地, 如果 $A(a_1,a_2,a_3),B(b_1,b_2,b_3),C(c_1,c_2,c_3)$, 则三阶行列式 $\detm{a_1&a_2&a_3\\b_1&b_2&b_3\\c_1&c_2&c_3}$ 的绝对值就是下述平行六面体的体积.
	\begin{center}
	\begin{tikzpicture}[scale=.8]
		\draw[cstcurve,dcolora] (0,0)--(3,0)--(4,1.5)--(1,1.5)--cycle;
		\draw[cstcurve,dcolora] (4.5,1)--(5.5,2.5)--(2.5,2.5);
		\draw[cstdash,cstcurve,dcolora] (2.5,2.5)--(1.5,1)--(4.5,1);
		\draw[cstdash,cstcurve,dcolora] (0,0)--(1.5,1);
		\draw[cstcurve,dcolora] (3,0)--(4.5,1);
		\draw[cstcurve,dcolora] (4,1.5)--(5.5,2.5);
		\draw[cstcurve,dcolora] (1,1.5)--(2.5,2.5);
		\draw (0,0) node[dcolorb,left] {$O$};
		\draw (3,0) node[dcolorb,right] {$A$};
		\draw (1,1.5) node[dcolorb,left] {$C$};
		\draw (1.5,1) node[dcolorb,below] {$B$};
	\end{tikzpicture}
	\end{center}
	\onslide<+->
	它的符号则表示使用右手四指从 $OA$ 旋转到 $OB$ 方向时, 大拇指所指方向与 $OC$ 是否在平面 $OAB$ 的同侧.
\end{frame}
