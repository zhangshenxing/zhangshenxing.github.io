\section{向量和矩阵的定义}


\subsection{向量和向量空间}
\begin{frame}{平面向量和立体向量\noexer}
	\onslide<+->
	我们在高中学习过向量的概念.
	\onslide<+->
	在平面上建立一个直角坐标系. 对于平面上的点 $A$, 连接 $OA$ 的有向线段就是一个\emph{向量}.
	\onslide<+->
	它可以用 $\bfu=(x,y)$ 来表示.

	\onslide<+->
	由于向量和平面上的点是一一对应的, 因此我们可以用 $\BR^2$ 来表示平面上所有向量形成的集合.
	\onslide<+->
	在这个集合中有一个特殊的元素, 叫作\emph{零向量}:
	\[{\bf0}=(0,0),\]
	\onslide<+->
	而且我们可以定义加法和数乘:
	\[(x_1,y_1)+(x_2,y_2)=(x_1+x_2,y_1+y_2),\quad
	\lambda(x,y)=(\lambda x,\lambda y),\lambda\in\BR.\]

	\onslide<+->
	类似地, 立体空间中的所有向量形成集合 $\BR^3$.
	\onslide<+->
	在这个集合中也有零向量、加法和数乘.
\end{frame}


\begin{frame}{平面向量和立体向量的性质\noexer}
	\onslide<+->
	我们总使用粗小写字母 $\bma,\bmb,\bmg,\bfx,\bfy,\dots$ 表示向量.

	\onslide<+->
	$\BR^2$ 或 $\BR^3$ 上的零向量、加法和数乘满足:
	\begin{enumV}\bf
		\item $\bma+\bmb=\bmb+\bma$;\label{enum:V1}
		\item $\bma+(\bmb+\bmg)=(\bma+\bmb)+\bmg$;
		\item ${\bm0}+\bma=\bma$;
		\item 对任意 $\bma$, 存在 $\bmb$ 使得 $\bma+\bmb={\bf0}$. 称 $\bmb$ 为 $\bma$ 的\emph{负向量};
		\item $(\lambda\mu)\bma=\lambda(\mu\bma)$;
		\item $(\lambda+\mu)\bma=\lambda\bma+\mu\bma$;
		\item $\lambda(\bma+\bmb)=\lambda\bma+\lambda\bmb$;
		\item $1\cdot\bma=\bma$.\label{enum:V8}
	\end{enumV}
\end{frame}


\begin{frame}{向量的定义}
	\onslide<+->
	将上述概念稍作推广, 我们可以得到具有 $n$ 个\emph{分量}的向量
	\[\bfx=(x_1,x_2,\dots,x_n),\]
	称为 $n$ 维\emph{行向量}.
	\onslide<+->
	出于后续描述的统一性, 我们将向量的分量写成一列而不是一行, 从而得到 $n$ 维\emph{列向量}:
	\[\bfx=\begin{pmatrix}
		x_1\\
		x_2\\
		\vdots\\
		x_n
	\end{pmatrix}.\]
	\onslide<+->
	\alert{之后凡是提到向量, 均是指列向量}.
	\onslide<+->
	为了书写方便, 也可以把列向量写成
	\[\bfx=(x_1,x_2,\dots,x_n)^\rmT.\]
	其中 $\rmT$ 表示\emph{转置}(Transpose), 它是将行列关系对换的一种操作.
\end{frame}


\begin{frame}{标准向量空间\noexer}
	\onslide<+->
	将全体 $n$ 维列向量记为 $\BR^n$.
	\onslide<+->
	在这个集合中可类似定义零向量、加法和数乘:
	\begin{enumerate}
		\item ${\bf0}=(0,\dots,0)^\rmT$;
		\item $(a_1,\dots,a_n)^\rmT+(b_1,\dots,b_n)^\rmT=(a_1+b_1,\dots,a_n+b_n)^\rmT$;
		\item $\lambda(a_1,\dots,a_n)^\rmT=(\lambda a_1,\dots,\lambda a_n)^\rmT,\lambda\in\BR$.
	\end{enumerate}
	\onslide<+->
	而且它们也满足\ref{enum:V1}--\ref{enum:V8}.
	\onslide<+->
	\begin{definition}
		若集合 $V$ 带有零向量、加法和实数的数乘, 且满足\ref{enum:V1}--\ref{enum:V8}, 则称 $V$ 是一个\emph{实线性空间}.
	\end{definition}
	\onslide<+->
	这样, $\BR^n$ 成为了一个实线性空间.
\end{frame}


\begin{frame}{线性空间\noexer}
	\onslide<+->
	\begin{example}
		\begin{enumerate}
			\item 平面上经过原点的直线 $V=\{(x,y)\in\BR^2\mid ax+by=0\}$ 是实线性空间.
			\item $\BR^3$ 中经过原点的平面 $V=\{(x,y,z)\in\BR^3\mid ax+by+cz=0\}$ 是实线性空间.
			\item 如果 $V$ 是 $\BR^n$ 的子集, 且它和继承自 $\BR^n$ 的零向量、加法和数乘构成线性空间, 则称 $V$ 是\emph{向量空间}.
			\item 实系数多项式全体 $\BR[x]$ 是实线性空间, 其中的零向量是指零多项式.
			\item 次数不超过 $n$ 的多项式(含零多项式)全体是实线性空间.
		\end{enumerate}
	\end{example}
	\onslide<+->
	若数乘的系数是复数而不是实数, 称 $V$ 为\emph{复线性空间}.
	\onslide<+->
	我们可以类似地定义复向量, 且 $n$ 维复向量全体形成一个复线性空间 $\BC^n$.
\end{frame}


\subsection{线性映射}
\begin{frame}{标准向量空间之间的映射\noexer}
	\onslide<+->
	考虑如下映射
	\begin{align*}
		\msa:\BR^n&\lra\BR^m\\
		\begin{pmatrix}
			x_1\\
			x_2\\
			\vdots\\
			x_n
		\end{pmatrix}&\lto
		\begin{pmatrix}
			a_{11}x_1+a_{12}x_2+\cdots+a_{1n}x_n\\
			a_{21}x_1+a_{22}x_2+\cdots+a_{2n}x_n\\
			\vvdots\\
			a_{m1}x_1+a_{m2}x_2+\cdots+a_{mn}x_n
		\end{pmatrix}.
	\end{align*}
	\onslide<+->
	不难验证, 对任意 $\bfu,\bfv\in\BR^n,\lambda\in\BR$,
	\begin{enumV}[L]
		\item $\msa(\bfu+\bfv)=\msa(\bfu)+\msa(\bfv)$;\label{enum:L1}
		\item $\msa(\lambda\bfu)=\lambda \msa(\bfu)$.\label{enum:L2}
	\end{enumV}
\end{frame}


\begin{frame}{线性变换\noexer}
	\onslide<+->
	\begin{definition}
		若映射 $\msa:\BR^n\ra\BR^m$ 满足\ref{enum:L1},\ref{enum:L2}, 称 $\msa$ 是一个\emph{线性变换}或\emph{线性映射}.
	\end{definition}
	\onslide<+->
	反过来, 是不是所有的线性变换 $\BR^n\ra\BR^m$ 都可以表达为前述形式呢?

	\onslide<+->
	以线性映射 $\msa:\BR^3\to\BR^4$ 为例,
	\onslide<+->
	记
	\[\msa\begin{pmatrix}
		1\\0\\0
	\end{pmatrix}=\begin{pmatrix}
		a_{11}\\a_{21}\\a_{31}\\a_{41}
	\end{pmatrix},\qquad
	\msa\begin{pmatrix}
		0\\1\\0
	\end{pmatrix}=\begin{pmatrix}
		a_{12}\\a_{22}\\a_{32}\\a_{42}
	\end{pmatrix},\qquad
	\msa\begin{pmatrix}
		0\\0\\1
	\end{pmatrix}=\begin{pmatrix}
		a_{13}\\a_{23}\\a_{33}\\a_{43}
	\end{pmatrix}.\]
\end{frame}


\begin{frame}{线性变换的形式\noexer}
	\onslide<+->
	根据线性变换的性质,
	\begin{align*}
		\msa\begin{pmatrix}
			x_1\\x_2\\x_3
		\end{pmatrix}
		&=\msa\left(x_1\begin{pmatrix}
			1\\0\\0
		\end{pmatrix}+x_2\begin{pmatrix}
			0\\1\\0
		\end{pmatrix}+x_3\begin{pmatrix}
			0\\0\\1
		\end{pmatrix}\right)
		\visible<+->{=x_1\msa\begin{pmatrix}
			1\\0\\0
		\end{pmatrix}+x_2\msa\begin{pmatrix}
			0\\1\\0
		\end{pmatrix}+x_3\msa\begin{pmatrix}
			0\\0\\1
		\end{pmatrix}}\\
		\visible<+->{=}&\visible<.->{x_1\begin{pmatrix}
			a_{11}\\a_{21}\\a_{31}\\a_{41}
		\end{pmatrix}+x_2\begin{pmatrix}
			a_{12}\\a_{22}\\a_{32}\\a_{42}
		\end{pmatrix}+x_3\begin{pmatrix}
			a_{13}\\a_{23}\\a_{33}\\a_{43}
		\end{pmatrix}}
		\visible<+->{=\begin{pmatrix}
			a_{11}x_1+a_{12}x_2+a_{13}x_3\\
			a_{21}x_1+a_{22}x_2+a_{23}x_3\\
			a_{31}x_1+a_{32}x_2+a_{33}x_3\\
			a_{41}x_1+a_{42}x_2+a_{43}x_3
		\end{pmatrix}}.
	\end{align*}
	因此每个线性变换 $\BR^4\to\BR^3$ 都具有这种形式,
	\onslide<+->
	它完全由 $4\times 3$ 个系数
	\[(a_{ij})_{1\le i\le 4,1\le j\le 3}\]
	所确定.
	\onslide<+->
	由此引出矩阵的定义.
\end{frame}


\subsection{矩阵}
\begin{frame}{矩阵的定义}
	\onslide<+->
	\begin{definition}
		将 $mn$ 个数按照每行 $n$ 个元素, 每列 $m$ 个元素, 排成的数表称为 \emph{$m$ 行 $n$ 列矩阵}, 或简称为 \emph{$m\times n$ 矩阵}:
		\[\bfA=\begin{pmatrix}
			a_{11}&a_{12}&\cdots &a_{1n}\\
			a_{21}&a_{22}&\cdots &a_{2n}\\
			\vdots&\vdots&\ddots&\vdots\\
			a_{m1}&a_{m2}&\cdots &a_{mn}
		\end{pmatrix}_{m\times n},\]
		其中 $a_{ij}$ 表示 $\bfA$ 的第 $i$ 行 $j$ 列元素, 并记 $\bfA=(a_{ij})_{m\times n}$.
	\end{definition}
	\onslide<+->
	我们总使用粗大写字母 $\bfA,\bfB,\bmL,\dots$ 表示矩阵.
	\onslide<+->
	不强调矩阵的阶时, 也可省略右下角 $m\times n$.

	\onslide<+->
	矩阵的圆括号也可用方括号来代替.
\end{frame}


\begin{frame}{例: 矩阵}
	\onslide<+->
	\begin{example}
		\begin{enumerate}
			\item $\bfA=\begin{pmatrix}
				0&-1&-2&-3\\
				1&0&-1&-2\\
				2&1&0&-1\\
				3&2&1&0
			\end{pmatrix}=(i-j)_{4\times 4}$.
			\item $\bfA=\begin{pmatrix}
				1&1&\cdots&1\\
				1&2&\cdots&n\\
				\vdots&\vdots&\ddots&\vdots\\
				1&2^{n-1}&3^{n-1}&n^{n-1}
			\end{pmatrix}_n=(j^{i-1})_{n}$.
			\onslide<+->{$m=n$ 时可用 $n$ 来表示右下角的 $n\times n$.}
		\end{enumerate}
	\end{example}
\end{frame}


\begin{frame}{线性映射与矩阵一一对应\noexer}
	\onslide<+->
	\begin{itemize}
		\item 令 $M_{m\times n}$ 表示 $m$ 行 $n$ 列的矩阵(Matrix)全体.
		\item 若 $m=n$, 称相应矩阵为 \emph{$n$ 阶方阵}, 并用 $M_n$ 来表示 $n$ 阶方阵全体.
		\item 若要强调矩阵元素都是实数, 我们称相应的矩阵为\emph{实矩阵}, 并用 $M_{m\times n}(\BR), M_n(\BR)$ 来表示相应集合.
		\item 全体线性变换 $\msa:\BR^n\to\BR^m$ 和 $\bfA\in M_{m\times n}(\BR)$ 是一一对应的.
	\end{itemize}
	\onslide<+->
	类似可定义\emph{复矩阵}, $M_{m\times n}(\BC)$, $M_n(\BC)$, 且全体线性变换 $\BC^n\to\BC^m$ 和 $M_{m\times n}(\BC)$ 一一对应.
	\onslide<+->
	之后此种类比不再赘述.

	\begin{itemize}
		\item 称方阵的对角线上的元素 $a_{11},\dots,a_{nn}$ 为方阵的\emph{对角元}.
	\end{itemize}
\end{frame}


\begin{frame}{特殊矩阵}
	\onslide<+->
	元素全为零的矩阵为\emph{零矩阵} $\bfO=\bfO_{m\times n}\in M_{m\times n}$.
	\onslide<+->
	方阵中
	\[\begin{pmatrix}
		a_{11}&a_{12}&\cdots&a_{1n}\\
		&a_{22}&\cdots&a_{2n}\\
		&&\ddots&\vdots\\
		&&&a_{nn}\\
	\end{pmatrix},\quad\begin{pmatrix}
		a_{11}&&&\\
		a_{21}&a_{22}&&\\
		\vdots&\vdots&\ddots&\\
		a_{n1}&a_{n2}&\cdots&a_{nn}\\
	\end{pmatrix},\quad\begin{pmatrix}
		\lambda_1&&&\\
		&\lambda_2&&\\
		&&\ddots&\\
		&&&\lambda_n\\
	\end{pmatrix}\in M_n\]
	分别为\emph{上三角阵}, \emph{下三角阵}和\emph{对角阵} (空白部分表示元素都是零).
	\onslide<+->
	为书写方便, 对角阵也可记作 (对角: diagonal)
	\[\alertn{\diag(\lambda_1,\lambda_2,\dots,\lambda_n)}.\]
	\onslide<+->
	称方阵
	\[\bfE=\bfE_n=\diag(1,1,\dots,1)\in M_n\]
	为\emph{单位阵}.
\end{frame}


\begin{frame}{矩阵与向量}
	\onslide<+->
	将只有一行的矩阵
	\[(a_1,a_2,\dots,a_n)\in M_{1\times n}\]
	称为 $n$ 维\emph{行矩阵}.
	\onslide<+->
	只有一列的矩阵
	\[\begin{pmatrix}
		b_1\\b_2\\\vdots\\b_n
	\end{pmatrix}\in M_{n\times 1}\]
	称为 $n$ 维\emph{列矩阵}.

	\onslide<+->
	我们将在后面定义矩阵的加法和数乘, 它与将行(列)矩阵看作行(列)向量的加法和数乘是相同的.
	\onslide<+->
	于是可将行(列)矩阵和对应的行(列)向量视为同一对象.
\end{frame}


\begin{frame}{例: 旋转变换}
	\onslide<+->
	如何用矩阵表示平面 $\BR^2$ 上的旋转?
	\onslide<+->
	设 $A(x_1,x_2)$ 是平面上的一个点, 沿着原点逆时针旋转角度 $\theta$ 变成 $B(y_1,y_2)$.
	\onslide<+->
	利用极坐标将 $A$ 表示为
	\[\laeq{
		x_1=\rho\cos\alpha,\\
		x_2=\rho\sin\alpha,}\]
	\onslide<+->
	则
	\[\laeq[lcccccrcr]{
		y_1&{}={}&\rho\cos(\alpha+\theta)&{}={}&\rho(\cos\alpha\cos\theta-\sin\alpha\sin\theta)&{}={}&(\cos\theta)x_1&{}-{}&(\sin\theta)x_2,\\
		y_2&{}={}&\rho\sin(\alpha+\theta)&{}={}&\rho(\cos\alpha\sin\theta+\sin\alpha\cos\theta)&{}={}&(\sin\theta)x_1&{}+{}&(\cos\theta)x_2.
	}\]
	\onslide<+->
	因此上述\emph{旋转变换} $\msr_\theta$ 对应的矩阵为
	\[\bfR_\theta=\begin{pmatrix}
		\cos\theta&-\sin\theta\\
		\sin\theta&\cos\theta
	\end{pmatrix}\in M_2(\BR).\]
\end{frame}


\begin{frame}{例: 线性变换}
	\begin{itemize}
		\item $\bfA=\begin{pmatrix}
			\lambda_1&&\\&\ddots&\\&&\lambda_n
		\end{pmatrix}$ 表示各个分量分别放大为 $\lambda_i$ 倍的线性变换.
		\item $\bfA=\begin{pmatrix}
			0&1\\1&0
		\end{pmatrix}$ 表示平面中沿着直线 $x_1=x_2$ 翻转.
		\item $\bfA=\begin{pmatrix}
			0&1&0\\0&0&1\\1&0&0
		\end{pmatrix}$ 表示三维空间中沿着直线 $x_1=x_2=x_3$ 旋转 $\dfrac{2\pi}3$.
		\item 想一想: $\bfA=\begin{pmatrix}
			1&0&0\\0&1&0\\0&0&-1
		\end{pmatrix}$ 表示什么线性变换?
	\end{itemize}
\end{frame}


