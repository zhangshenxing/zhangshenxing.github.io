\section{向量和矩阵的定义}


\subsection{向量和向量空间}
\begin{frame}{平面向量和立体向量}
	\onslide<+->
	我们在高中学习过向量的概念.
	\onslide<+->
	在平面上建立一个直角坐标系. 对于平面上的点 $A$, 连接 $OA$ 的有向线段就是一个\emph{向量}.
	\onslide<+->
	它可以用 $\bfu=(x,y)$ 来表示.

	\onslide<+->
	记平面上的所有向量为集合 $\BR^2$.
	\onslide<+->
	在这个集合中有一个特殊的元素, 叫作\emph{零向量}:
	\[{\bf0}=(0,0),\]
	\onslide<+->
	而且我们可以定义加法和数乘:
	\[(x_1,y_1)+(x_2,y_2)=(x_1+x_2,y_1+y_2),\quad
	\lambda(x,y)=(\lambda x,\lambda y),\lambda\in\BR.\]

	\onslide<+->
	类似地, 立体空间中的所有向量形成集合 $\BR^3$.
	\onslide<+->
	在这个集合中也有零向量、加法和数乘.
\end{frame}


\begin{frame}{平面向量和立体向量的性质}
	\onslide<+->
	$\BR^2$ 或 $\BR^3$ 上的零向量、加法和数乘满足:
	\begin{enumerate}
		\item $\bma+\bmb=\bmb+\bma$;
		\item $\bma+(\bmb+\bmg)=(\bma+\bmb)+\bmg$;
		\item ${\bm0}+\bma=\bma$;
		\item 对任意 $\bma$, 存在 $\bmb$ 使得 $\bma+\bmb={\bf0}$. 称 $\bmb$ 为 $\bma$ 的\emph{负向量};
		\item $1\cdot\bma=\bma$;
		\item $(k\ell)\bma=k(\ell\bma)$;
		\item $(k+\ell)\bma=k\bma+\ell\bma$;
		\item $k(\bma+\bmb)=k\bma+k\bmb$.
	\end{enumerate}
\end{frame}


\begin{frame}{向量的定义}
	\onslide<+->
	将上述概念稍作推广, 我们可以得到具有 $n$ 个\emph{分量}的向量
	\[\bfx=(x_1,x_2,\dots,x_n),\]
	称为 $n$ 维\emph{行向量}.
	\onslide<+->
	出于后续描述的统一性, 我们将向量的分量写成一列而不是一行, 从而得到 $n$ 维\emph{列向量}:
	\[\bfx=\begin{pmatrix}
		x_1\\
		x_2\\
		\vdots\\
		x_n
	\end{pmatrix}.\]
	\onslide<+->
	\alert{之后凡是提到向量, 均是指列向量}.
	\onslide<+->
	为了书写方便, 也可以把列向量写成
	\[\bfx=(x_1,x_2,\dots,x_n)^\rmT.\]
	其中 $\rmT$ 表示\cnen{\emph{转置}}{Transpose}, 它是将行列关系对换的一种操作.
\end{frame}


\begin{frame}{标准向量空间}
	\onslide<+->
	将全体 $n$ 维列向量记为 $\BR^n$.
	\onslide<+->
	在这个集合中可类似定义零向量、加法和数乘:
	\begin{enumerate}
		\item ${\bf0}=(0,\dots,0)^\rmT$;
		\item $(a_1,\dots,a_n)^\rmT+(b_1,\dots,b_n)^\rmT=(a_1+b_1,\dots,a_n+b_n)^\rmT$;
		\item $\lambda(a_1,\dots,a_n)^\rmT=(\lambda a_1,\dots,\lambda a_n)^\rmT,\lambda\in\BR$.
	\end{enumerate}
	\onslide<+->
	而且它们也满足前述的八条性质.
	\onslide<+->
	\begin{definition}
		若集合 $V$ 带有零向量、加法和数乘, 且满足前述八条性质, 则称 $V$ 是一个\emph{线性空间}.
	\end{definition}
	\onslide<+->
	这样, $\BR^n$ 成为了一个线性空间.

	\onslide<+->
	若数乘的系数是复数而不是实数, 称线性空间 $V$ 为\emph{复线性空间}, 而前述线性空间为\emph{实线性空间}.
	\onslide<+->
	我们可以类似地定义复向量, 且 $n$ 维复向量全体形成一个复线性空间 $\BC^n$.
\end{frame}


\begin{frame}{子空间、向量空间\noexer}
	\onslide<+->
	若 $W$ 是 $V$ 的一个子集, 且
	\begin{enumerate}
		\item ${\bf0}\in V$;
		\item $\bma,\bmb\in V\implies \bma+\bmb\in V$;
		\item $\bma\in V\implies \lambda\bma\in V, \forall\lambda\in\BR$,
	\end{enumerate}
	\onslide<+->
	则 $W$ 以及它继承自 $V$ 的这些要素构成线性空间, 称之为 $V$ 的\emph{子空间}.

	\onslide<+->
	将线性空间 $\BR^n$ 或 $\BC^n$ 的子空间称为实/复\emph{向量空间}.
	\onslide<+->
	\begin{example}
		\begin{enumerate}
			\item $\BR^2$ 中经过原点的直线是 $\BR^2$ 的子空间.
			\item $\BR^3$ 中经过原点的直线和平面都是 $\BR^3$ 的子空间. 
			\item 仅包含零向量的集合是 $\BR^n$ 或 $\BC^n$ 的一个子空间.
		\end{enumerate}
	\end{example}
\end{frame}


\subsection{线性映射}
\begin{frame}{线性方程组对应的映射\noexer}
	\onslide<+->
	考虑向量空间之间的映射.
	\onslide<+->
	对于系数都是实数的线性方程组
	\[\laeq[lclcclcl]{
		a_{11}&x_1+{}&a_{12}&x_2&{}+\cdots+{}&a_{1n}&x_n={}&y_1\\
		a_{21}&x_1+{}&a_{22}&x_2&{}+\cdots+{}&a_{2n}&x_n={}&y_2\\
		&&&&\vdots&&&\\
		a_{m1}&x_1+{}&a_{m2}&x_2&{}+\cdots+{}&a_{mn}&x_n={}&y_m
	}\]
	\onslide<+->
	若把 $x_1,\dots,x_n$ 看作自变量, $y_1,\dots,y_m$ 看作因变量, 则上述关系给出了一个映射:
	\[f:\BR^n\lra\BR^m.\]
	\onslide<+->
	不难看出, 这个映射满足
	\begin{enumerate}
		\item $f(\bma+\bmb)=f(\bma)+f(\bmb),\forall \bma,\bmb\in \BR^n$;
		\item $f(\lambda \bma)=\lambda f(\bma),\forall \lambda\in\BR,\forall\bma\in \BR^n$.
	\end{enumerate}
\end{frame}


\begin{frame}{线性变换\noexer}
	\onslide<+->
	\begin{definition}
		设 $V,W$ 是实向量空间.
		若映射 $f:V\to W$ 满足
		\begin{enumerate}[<*>]
			\item $f(\bma+\bmb)=f(\bma)+f(\bmb),\forall \bma,\bmb\in V$;
			\item $f(\lambda \bma)=\lambda f(\bma),\forall \lambda\in\BR,\forall\bma\in V$,
		\end{enumerate}
		称 $f$ 是一个\emph{线性变换}或\emph{线性映射}.
	\end{definition}
	\onslide<+->
	于是给定 $mn$ 个实数 $a_{ij}$, 我们便可构造出一个线性变换:
	\onslide<+->
	\begin{align*}
		f:\BR^n&\lra\BR^m\\
		\begin{pmatrix}
			x_1\\
			x_2\\
			\vdots\\
			x_n
		\end{pmatrix}&\lto
		\begin{pmatrix}
			a_{11}x_1+a_{12}x_2+\cdots+a_{1n}x_n\\
			a_{21}x_1+a_{22}x_2+\cdots+a_{2n}x_n\\
			\vvdots\\
			a_{m1}x_1+a_{m2}x_2+\cdots+a_{mn}x_n
		\end{pmatrix}.
	\end{align*}
	\onslide<+->
	反过来, 是不是所有的线性变换 $\BR^n\ra\BR^m$ 都可以表达为这种形式呢?
\end{frame}


\begin{frame}{标准向量空间\noexer}
	\onslide<+->
	设 $f:\BR^n\to\BR^m$ 是一个线性变换,
	\onslide<+->
	记
	\[f\begin{pmatrix}
		1\\0\\\vdots\\0
	\end{pmatrix}=\begin{pmatrix}
		a_{11}\\a_{21}\\\vdots\\a_{m1}
	\end{pmatrix},\quad
	f\begin{pmatrix}
		0\\1\\\vdots\\0
	\end{pmatrix}=\begin{pmatrix}
		a_{12}\\a_{22}\\\vdots\\a_{m2}
	\end{pmatrix},\cdots,
	f\begin{pmatrix}
		0\\0\\\vdots\\1
	\end{pmatrix}=\begin{pmatrix}
		a_{1n}\\a_{2n}\\\vdots\\a_{mn}
	\end{pmatrix}.\]
\end{frame}


\begin{frame}{线性变换的形式\noexer}
	\onslide<+->
	根据线性变换的性质,
	\begin{align*}
		&f\begin{pmatrix}
			x_1\\x_2\\\vdots\\x_n
		\end{pmatrix}
		=f\left(x_1\begin{pmatrix}
			1\\0\\\vdots\\0
		\end{pmatrix}+\dots+x_n\begin{pmatrix}
			0\\0\\\vdots\\1
		\end{pmatrix}\right)
		\visible<+->{=x_1f\begin{pmatrix}
			1\\0\\\vdots\\0
		\end{pmatrix}+\dots+x_nf\begin{pmatrix}
			0\\0\\\vdots\\1
		\end{pmatrix}}\\
		\visible<+->{=}&\visible<.->{x_1\begin{pmatrix}
			a_{11}\\a_{21}\\\vdots\\a_{m1}
		\end{pmatrix}+x_2\begin{pmatrix}
			a_{12}\\a_{22}\\\vdots\\a_{m2}
		\end{pmatrix}+\dots+x_n\begin{pmatrix}
			a_{1n}\\a_{2n}\\\vdots\\a_{mn}
		\end{pmatrix}}
		\visible<+->{=\begin{pmatrix}
			a_{11}x_1+a_{12}x_2+\cdots+a_{1n}x_n\\
			a_{21}x_1+a_{22}x_2+\cdots+a_{2n}x_n\\
			\vvdots\\
			a_{m1}x_1+a_{m2}x_2+\cdots+a_{mn}x_n
		\end{pmatrix}}.
	\end{align*}
	\onslide<+->
	所以每个 $\BR^n\to\BR^m$ 的线性变换都具有这种形式.

	\onslide<+->
	将 $\BR$ 换成 $\BC$ 我们亦有类似结论.
\end{frame}


\subsection{矩阵}
\begin{frame}{矩阵的定义}
	\onslide<+->
	现在给出矩阵的定义:
	\onslide<+->
	\begin{definition}
		将 $mn$ 个数按照每行 $n$ 个元素, 每列 $m$ 个元素, 排成的数表称为 \emph{$m$ 行 $n$ 列\cnen{矩阵}{Matrix}}, 或简称为 \emph{$m\times n$ 矩阵}:
		\[\bfA=\begin{pmatrix}
			a_{11}&a_{12}&\cdots &a_{1n}\\
			a_{21}&a_{22}&\cdots &a_{2n}\\
			\vdots&\vdots&\ddots&\vdots\\
			a_{m1}&a_{m2}&\cdots &a_{mn}
		\end{pmatrix},\]
		其中 $a_{ij}$ 表示 $\bfA$ 的第 $i$ 行 $j$ 列元素, 并记 $\bfA=(a_{ij})_{m\times n}$.
	\end{definition}
	\onslide<+->
	记 $M_{m\times n}$ 表示 $m$ 行 $n$ 列的矩阵全体.
\end{frame}


\begin{frame}{线性映射与矩阵一一对应\noexer}
	\onslide<+->
	若矩阵元素都是实数或复数, 称相应的矩阵为\emph{实矩阵}或\emph{复矩阵}, 并分别用 $M_{m\times n}(\BR),M_{m\times n}(\BC)$ 表示它们全体.

	\onslide<+->
	于是 $\BR^n\to\BR^m$ 的线性变换全体就和 $M_{m\times n}(\BR)$ 建立了一一对应.
	\onslide<+->
	类似地, $\BC^n\to\BC^m$ 的线性变换全体和 $M_{m\times n}(\BC)$ 也是一一对应的, 之后此种类比不再赘述.

	\onslide<+->
	若 $m=n$, 称相应矩阵为 \emph{$n$ 阶方阵}, 并记 $n$ 阶 (实、复)方阵全体为 $M_n,M_n(\BR),M_n(\BC)$.
	\onslide<+->
	方阵的对角线上的元素 $a_{11},\dots,a_{nn}$ 称为方阵的\emph{对角元}.
\end{frame}


\begin{frame}{特殊矩阵}
	\onslide<+->
	元素全为零的矩阵为\emph{零矩阵} $\bfO=\bfO_{m\times n}\in M_{m\times n}$.
	\onslide<+->
	方阵中
	\[\begin{pmatrix}
		a_{11}&a_{12}&\cdots&a_{1n}\\
		&a_{22}&\cdots&a_{2n}\\
		&&\ddots&\vdots\\
		&&&a_{nn}\\
	\end{pmatrix},\quad\begin{pmatrix}
		a_{11}&&&\\
		a_{21}&a_{22}&&\\
		\vdots&\vdots&\ddots&\\
		a_{n1}&a_{n2}&\cdots&a_{nn}\\
	\end{pmatrix},\quad\begin{pmatrix}
		\lambda_1&&&\\
		&\lambda_2&&\\
		&&\ddots&\\
		&&&\lambda_n\\
	\end{pmatrix}\in M_n\]
	分别为\emph{上三角阵}, \emph{下三角阵}和\emph{对角阵}.
	\onslide<+->
	为书写方便, 对角阵也可记作
	\[\alert{\diag(\lambda_1,\lambda_2,\dots,\lambda_n)}.\]
	\onslide<+->
	称方阵
	\[\bfE=\bfE_n=\diag(1,1,\dots,1)\in M_n\]
	为\emph{单位阵}.
\end{frame}


\begin{frame}{矩阵与向量}
	\onslide<+->
	将只有一行的矩阵
	\[(a_1,a_2,\dots,a_n)\in M_{1\times n}\]
	称为 $n$ 维\emph{行矩阵}.
	\onslide<+->
	只有一列的矩阵
	\[\begin{pmatrix}
		b_1\\b_2\\\vdots\\b_n
	\end{pmatrix}\in M_{n\times 1}\]
	称为 $n$ 维\emph{列矩阵}.

	\onslide<+->
	我们将在后面定义矩阵的加法和数乘, 它与将行(列)矩阵看作行(列)向量的加法和数乘是相同的.
	\onslide<+->
	于是可将行(列)矩阵和对应的行(列)向量视为同一对象.
\end{frame}


\begin{frame}{例题: 旋转变换}
	\onslide<+->
	如何用矩阵表示平面 $\BR^2$ 上的旋转?
	\onslide<+->
	设 $A(x_1,x_2)$ 是平面上的一个点, 沿着原点逆时针旋转角度 $\theta$ 变成 $B(y_1,y_2)$.
	\onslide<+->
	利用极坐标将 $A$ 表示为
	\[\laeq{
		x_1=\rho\cos\alpha,\\
		x_2=\rho\sin\alpha,}\]
	\onslide<+->
	则
	\[\laeq[lcccccrcr]{
		y_1&{}={}&\rho\cos(\alpha+\theta)&{}={}&\rho(\cos\alpha\cos\theta-\sin\alpha\sin\theta)&{}={}&(\cos\theta)x_1&{}-{}&(\sin\theta)x_2,\\
		y_2&{}={}&\rho\sin(\alpha+\theta)&{}={}&\rho(\cos\alpha\sin\theta+\sin\alpha\cos\theta)&{}={}&(\sin\theta)x_1&{}+{}&(\cos\theta)x_2.
	}\]
	\onslide<+->
	因此上述\emph{旋转变换}对应的矩阵为
	\[\bfR_\theta=\begin{pmatrix}
		\cos\theta&-\sin\theta\\
		\sin\theta&\cos\theta
	\end{pmatrix}\in M_2(\BR).\]
\end{frame}


\begin{frame}{例题: 线性变换}
	\begin{itemize}
		\item $\bfA=\begin{pmatrix}
			\lambda_1&&\\&\ddots&\\&&\lambda_n
		\end{pmatrix}$ 表示各个分量分别放大 $\lambda_i$ 倍的线性变换.
		\item $\bfA=\begin{pmatrix}
			0&1\\1&0
		\end{pmatrix}$ 表示平面中沿着直线 $x_1=x_2$ 翻转.
		\item $\bfA=\begin{pmatrix}
			0&1&0\\0&0&1\\1&0&0
		\end{pmatrix}$ 表示三维空间中沿着直线 $x_1=x_2=x_3$ 旋转 $\dfrac{2\pi}3$.
		\item 想一想: $\bfA=\begin{pmatrix}
			1&0&0\\0&1&0\\0&0&-1
		\end{pmatrix}$ 表示什么线性变换?
	\end{itemize}
\end{frame}


