\section{矩阵的初等变换}

\subsection{初等矩阵}

\begin{frame}{初等行变换解线性方程组}
	\onslide<+->
	我们曾利用如下三种初等变换来帮助计算行列式:
	\onslide<+->
	\begin{third}{初等变换}
		\begin{enumerate}
			\item 互换两行(列): $\alertm{r_i\swap r_j, c_i\swap c_j}$, 行列式变号;
			\item 一行(列)乘\alert{非零常数} $k$: $\alertm{kr_i, kc_i}$, 行列式变为 $k$ 倍;
			\item $j$ 行(列)乘 $k$ 加到 $i$ 行(列): $\alertm{r_i+kr_j, c_i+kc_j}$, 行列式不变.
		\end{enumerate}
	\end{third}
	\onslide<+->
	实际上我们也可以对矩阵实施初等变换,
	\onslide<+->
	而且这三类变换过程都是可逆的, 且其逆变换是同一类变换.
	\onslide<+->
	以行变换为例:
	\begin{enumerate}
		\item $r_i\swap r_j$ 的逆是 $r_i\swap r_j$;
		\item $kr_i$ 的逆是 $\dfrac1k r_i$;
		\item $r_i+kr_j$ 的逆是 $r_i-kr_j$.
	\end{enumerate}
	\onslide<+->
	我们使用矩阵来表示上述变换.
\end{frame}


\begin{frame}{第一类初等矩阵}
	\onslide<+->
	单位阵 $\bfE$ 经过一次初等变换得到的方阵称为\emph{初等矩阵}.
	\begin{enumerate}
		\item $r_i\swap r_j$ 和 $c_i\swap c_j$ 都对应初等矩阵\small
		\[\bfE(i,j)=\begin{pNiceMatrix}[last-col]
			\Emat&       &      &       &   &\\
					&\alertn0&\cdots&\alertn1&   &\text{$\la$ 第 $i$ 行}\\
					&\vdots &\Emat &\vdots &   &\\
					&\alertn1&\cdots&\alertn0&   &\text{$\la$ 第 $j$ 行}\\
					&       &      &       &\Emat&
			\CodeAfter
			\tikz \draw[cstdash,main] (1-|2) -- (6-|2);
			\tikz \draw[cstdash,main] (1-|3) -- (6-|3);
			\tikz \draw[cstdash,main] (1-|4) -- (6-|4);
			\tikz \draw[cstdash,main] (1-|5) -- (6-|5);
			\tikz \draw[cstdash,fourth] (2-|1) -- (2-|6);
			\tikz \draw[cstdash,fourth] (3-|1) -- (3-|6);
			\tikz \draw[cstdash,fourth] (4-|1) -- (4-|6);
			\tikz \draw[cstdash,fourth] (5-|1) -- (5-|6);
		\end{pNiceMatrix}\]
	\end{enumerate}
\end{frame}


\begin{frame}{第二类初等矩阵}
	\begin{enumerate}
		\setcounter{enumi}{1}
		\item $k r_i, k c_i$ 都对应初等矩阵
		\[\bfE(i(k))=\begin{pNiceMatrix}[last-col,last-row]
			\Emat&&\\
			&\alertn k&&\text{$\la$ 第 $i$ 行}\\
			&&\Emat&\\
			&\mathclap{\substack{\nsmath{\uparrow}\\\text{\normalsize 第 $i$ 列}}}&&
			\CodeAfter
			\tikz \draw[cstdash,main] (1-|2) -- (4-|2);
			\tikz \draw[cstdash,main] (1-|3) -- (4-|3);
			\tikz \draw[cstdash,fourth] (2-|1) -- (2-|4);
			\tikz \draw[cstdash,fourth] (3-|1) -- (3-|4);
		\end{pNiceMatrix}\]
	\end{enumerate}
\end{frame}


\begin{frame}{第三类初等矩阵}
	\begin{enumerate}
		\setcounter{enumi}{2}
		\item $r_i+kr_j, c_j+kc_i$ 都对应初等矩阵
		\[\bfE(i,j(k))=\begin{pNiceMatrix}[last-col,last-row]
			1&      & &      &        &      & &\\
			&\ddots& &      &        &      & &\\
			&      &1&      &\alertn k&      & &\text{$\la$ 第 $i$ 行}\\
			&      & &\ddots&        &      & &\\
			&      & &      &  1     &      & &\text{$\la$ 第 $j$ 行}\\
			&      & &      &        &\ddots& &\\
			&      & &      &        &      &1\\
			&&\mathclap{\substack{\nsmath{\uparrow}\\\text{\normalsize 第 $i$ 列}}}&&\mathclap{\substack{\nsmath{\uparrow}\\\text{\normalsize 第 $j$ 列}}}&&&
			\CodeAfter
			\tikz \draw[cstdash,main] (1-|3) -- (8-|3);
			\tikz \draw[cstdash,main] (1-|4) -- (8-|4);
			\tikz \draw[cstdash,main] (1-|5) -- (8-|5);
			\tikz \draw[cstdash,main] (1-|6) -- (8-|6);
			\tikz \draw[cstdash,fourth] (3-|1) -- (3-|8);
			\tikz \draw[cstdash,fourth] (4-|1) -- (4-|8);
			\tikz \draw[cstdash,fourth] (5-|1) -- (5-|8);
			\tikz \draw[cstdash,fourth] (6-|1) -- (6-|8);
		\end{pNiceMatrix}\]
	\end{enumerate}
	\onslide<+->
	需要注意的是 \alert{$c_i+kc_j$ 对应的初等矩阵不是 $E(i,j(k))$ 而是 $E(j,i(k))$}.
\end{frame}


\begin{frame}{初等矩阵左乘矩阵}
	\onslide<+->
	我们来看
	\[\bfE(1,3)\bfA=\begin{pmatrix}
		0&0&1\\
		0&1&0\\
		1&0&0
	\end{pmatrix}\begin{pmatrix}
		1&2&3\\
		4&5&6\\
		7&8&9
	\end{pmatrix}=\begin{pmatrix}
		7&8&9\\
		4&5&6\\
		1&2&3
	\end{pmatrix}.\]
	\onslide<+->
	$\bfE(i,j)$ 左乘在矩阵 $\bfA$ 上, 即对 $\bfA$ 实施 $r_i\swap r_j$.

	\onslide<+->
	从
	\[\bfE(i,j)\bfA=\bfE(i,j)\begin{pmatrix}
		\vdots\\
		\bma_i\\
		\vdots\\
		\bma_j\\
		\vdots
	\end{pmatrix}=\begin{pmatrix}
		\vdots\\
		\bma_j\\
		\vdots\\
		\bma_i\\
		\vdots
	\end{pmatrix}\]
	可以看出确实如此.
\end{frame}


\begin{frame}{初等矩阵与初等变换}
	\onslide<+->
	\[\bfE(2(k))\bfA=\begin{pmatrix}
		1&0&0\\
		0&k&0\\
		0&0&1
	\end{pmatrix}\begin{pmatrix}
		1&2&3\\
		4&5&6\\
		7&8&9
	\end{pmatrix}=\begin{pmatrix}
		1&2&3\\
		4k&5k&6k\\
		7&8&9
	\end{pmatrix}.\]
	\onslide<+->
	$\bfE(i(k))$ 左乘在矩阵 $\bfA$ 上, 即对 $\bfA$ 实施 $kr_i$.
	\onslide<+->
	\[\bfE(3,1(k))\bfA=\begin{pmatrix}
		1&0&0\\
		0&1&0\\
		k&0&1
	\end{pmatrix}\begin{pmatrix}
		1&2&3\\
		4&5&6\\
		7&8&9
	\end{pmatrix}=\begin{pmatrix}
		1&2&3\\
		4&5&6\\
		7+k&8+2k&9+3k
	\end{pmatrix}.\]
	\onslide<+->
	$\bfE(i,j(k))$ 左乘在矩阵 $\bfA$ 上, 即对 $\bfA$ 实施 $r_i+kr_j$.
	\onslide<+->
	即, 初等矩阵左乘矩阵 $\bfA$ 等同于对 $\bfA$ 实施对应的初等行变换.

	\onslide<+->
	同理, 初等矩阵右乘矩阵 $\bfA$ 等同于对 $\bfA$ 实施对应的初等列变换.
\end{frame}


\begin{frame}{初等矩阵与初等变换}
	\onslide<+->
	\begin{theorem}
		设 $\bfA\in M_{m\times n}$.
		\begin{enumerate}
			\item 对 $\bfA$ 实施一次\emph{初等行变换}, 相当于在\emph{$\bfA$ 的左边}乘对应的 $m$ 阶初等矩阵.
			\item 对 $\bfA$ 实施一次\alert{初等列变换}, 相当于在\alert{$\bfA$ 的右边}乘对应的 $n$ 阶初等矩阵.
		\end{enumerate}
		\onslide<+->{%
			对应的初等矩阵就是对单位阵 $\bfE$ 实施相应的初等变换得到的矩阵.
		}
	\end{theorem}
	\onslide<+->
	即\alert{左行右列}.
\end{frame}


\begin{frame}{例: 初等矩阵与初等变换}
	\onslide<+->
	\begin{example}
		设 $\bfA$ 为 $3$ 阶可逆方阵, 将 $\bfA$ 的第 $1$ 列与第 $2$ 列交换得 $\bfB$, 再把 $\bfB$ 的第 $2$ 列加到第 $3$ 列得到 $\bfC$.求 $\bfQ=\bfA^{-1}\bfC$.
	\end{example}
	\onslide<+->
	\begin{solution}
		$\bfB=\bfA\begin{pmatrix}
			0&1&0\\1&0&0\\0&0&1
		\end{pmatrix},\bfC=\bfB\begin{pmatrix}
			1&0&0\\0&1&1\\0&0&1
		\end{pmatrix}=\bfA\bfQ$.
		\onslide<+->{因此
		\[\bfQ=\begin{pmatrix}
			0&1&0\\1&0&0\\0&0&1
		\end{pmatrix}\begin{pmatrix}
			1&0&0\\0&1&1\\0&0&1
		\end{pmatrix}=\begin{pmatrix}
			0&1&1\\1&0&0\\0&0&1
		\end{pmatrix}.\]}
	\end{solution}
\end{frame}


\begin{frame}{例: 初等矩阵与初等变换}
	\onslide<+->
	\begin{exercise}
		设 $\bfA=\begin{pmatrix}
			a_{11}&a_{12}&a_{13}\\
			a_{21}&a_{22}&a_{23}\\
			a_{31}&a_{32}&a_{33}
		\end{pmatrix},\bfB=\begin{pmatrix}
			a_{11}&a_{13}&a_{12}\\
			a_{21}&a_{23}&a_{22}\\
			a_{31}+2a_{11}&a_{33}+2a_{13}&a_{32}+2a_{12}
		\end{pmatrix}$,\\
		$\bfP_1=\begin{pmatrix}
			1&0&0\\0&0&1\\0&1&0
		\end{pmatrix},\bfP_2=\begin{pmatrix}
			0&0&1\\0&1&0\\1&0&0
		\end{pmatrix},\bfP_3=\begin{pmatrix}
			1&0&0\\0&1&0\\2&0&1
		\end{pmatrix}$.
		则 $\bfB=$\fillbraceframe{C}.
		\begin{exchoice}(4)
			() $\bfP_3\bfA\bfP_2$
			() $\bfP_2\bfA\bfP_3$
			() $\bfP_3\bfA\bfP_1$
			() $\bfP_1\bfP_2\bfA$
		\end{exchoice}
	\end{exercise}
\end{frame}


\begin{frame}{例: 初等矩阵的逆}
	\onslide<+->
	由于初等变换都是可逆, 因此初等矩阵也都是可逆的:
	\begin{enumerate}
		\item $\bfE(i,j)\bfE(i,j)=\bfE\implies \alertn{\bfE(i,j)^{-1}=\bfE(i,j)}$;
		\item $\bfE\bigl(i(k)\bigr)\bfE\bigl(i(\frac1k)\bigr)=\bfE\implies\alertn{\bfE\bigl(i(k)\bigr)^{-1}=\bfE\bigl(i(\frac1k)\bigr)}$;
		\item $\bfE\bigl(i,j(k)\bigr)\bfE\bigl(i,j(-k)\bigr)=\bfE\implies\alertn{\bfE\bigl(i,j(k)\bigr)^{-1}=\bfE\bigl(i,j(-k)\bigr)}$.
	\end{enumerate}
	\onslide<+->
	\begin{example}
		将可逆方阵 $\bfA$ 的第 $1$ 行的 $2$ 倍加到第 $2$ 行得到 $\bfB$, 则对 $\bfA^{-1}$ 实施初等变换\fillbraceframe{D}可得到 $\bfB^{-1}$.
		\begin{exchoice}(4)
			() $r_2+2r_1$
			() $r_2-2r_1$
			() $c_1+2c_2$
			() $c_1-2c_2$
		\end{exchoice}
	\end{example}
	\onslide<+->
	\begin{exercise}
		设 $\bfA$ 是 $n$ 阶可逆矩阵, 将 $\bfA$ 的第 $i$ 行与第 $j$ 行对换后得到的矩阵记为 $\bfB$, 则 $\bfA\bfB^{-1}=$\fillblankframe[3cm]{$\bfE(i,j)$}.
	\end{exercise}
\end{frame}


\begin{frame}{例: 初等矩阵}\small
\beqskip{0pt}
	\onslide<+->
	\begin{example}
		设 $\bfP_1=\begin{pmatrix}
			0&0&1&0\\
			0&1&0&0\\
			1&0&0&0\\
			0&0&0&1
		\end{pmatrix},\bfP_2=\begin{pmatrix}
			1&&&\\
			0&1&&\\
			0&0&1&\\
			a&0&0&1
		\end{pmatrix},\bfP_3=\begin{pmatrix}
			1&&&\\
			&k&&\\
			&&1&\\
			&&&1
		\end{pmatrix}$, 求 $\bfP_1\bfP_2\bfP_3$ 及逆.
	\end{example}
	\onslide<+->
	\begin{solution}
		\[\bfP_1\wsim{c_1+ac_4}{}\bfP_1\bfP_2=\begin{pmatrix}
			0&0&1&0\\
			0&1&0&0\\
			1&0&0&0\\
			a&0&0&1
		\end{pmatrix}\visible<+->{\wsim{kc_2}{}
		\bfP_1\bfP_2\bfP_3=\begin{pmatrix}
			0&0&1&0\\
			0&k&0&0\\
			1&0&0&0\\
			a&0&0&1
		\end{pmatrix},}\]
		\onslide<+->{%
		\[\bfP_1^{-1}=\bfP_1\!\!\wsim{r_4-ar_1}{}\!\!
		(\bfP_1\bfP_2)^{-1}=\begin{pmatrix}
			0&0&1&0\\
			0&1&0&0\\
			1&0&0&0\\
			0&0&-a&1
		\end{pmatrix}\visible<+->{\!\!\wsim{\dfrac1k r_2}{}\!\!
		(\bfP_1\bfP_2\bfP_3)^{-1}=\begin{pmatrix}
			0&0&1&0\\
			0&1/k&0&0\\
			1&0&0&0\\
			0&0&-a&1
		\end{pmatrix}.}\]}
		\vspace{-.3\baselineskip}
	\end{solution}
\endgroup
\end{frame}


\begin{frame}{例: 初等变换}
	\onslide<+->
	\begin{exercise}
		设\[\bfA=\begin{pmatrix}
			a_{11}&a_{12}&a_{13}&a_{14}\\
			a_{21}&a_{22}&a_{23}&a_{24}\\
			a_{31}&a_{32}&a_{33}&a_{34}\\
			a_{41}&a_{42}&a_{43}&a_{44}
		\end{pmatrix},\qquad\bfB=\begin{pmatrix}
			a_{14}&a_{13}&a_{12}&a_{11}\\
			a_{24}&a_{23}&a_{22}&a_{21}\\
			a_{34}&a_{33}&a_{32}&a_{31}\\
			a_{44}&a_{43}&a_{42}&a_{41}
		\end{pmatrix},\]\[\bfP_1=\begin{pmatrix}
			0&0&0&1\\
			0&1&0&0\\
			0&0&1&0\\
			1&0&0&0
		\end{pmatrix},\qquad\bfP_2=\begin{pmatrix}
			1&0&0&0\\
			0&0&1&0\\
			0&1&0&0\\
			0&0&0&1
		\end{pmatrix}.\]
		若 $\bfA$ 可逆, 则 $\bfB^{-1}=$\fillbraceframe{C}.
		\begin{exchoice}(4)
			() $\bfA^{-1}\bfP_1\bfP_2$
			() $\bfP_1\bfA^{-1}\bfP_2$
			() $\bfP_1\bfP_2\bfA^{-1}$
			() $\bfP_2\bfA^{-1}\bfP_1$
		\end{exchoice}
	\end{exercise}
\end{frame}


\subsection{矩阵等价}

\begin{frame}{矩阵化为行阶梯形}
	\onslide<+->
	我们来看初等行变换能够将矩阵化简成何种形式.
	\onslide<+->
	\[\begin{pmatrix}
		1&3&-2&4\\
		3&6&-2&11\\
		2&1&1&3
	\end{pmatrix}
	\visible<+->{\wsim{r_2-3r_1}{r_3-2r_1}\begin{pmatrix}
		1&3&-2&4\\
		0&-3&4&-1\\
		0&-5&5&-5
	\end{pmatrix}}
	\visible<+->{\wsim{-\dfrac15r_3}{}\begin{pmatrix}
		1&3&-2&4\\
		0&-3&4&-1\\
		0&1&-1&1
	\end{pmatrix}}\]
	\onslide<+->
	\[\wsim{r_2\swap r_3}{}\begin{pmatrix}
		1&3&-2&4\\
		0&1&-1&1\\
		0&-3&4&-1
	\end{pmatrix}
	\visible<+->{\wsim{r_3+3r_2}{}\begin{pNiceMatrix}
		1&3&-2&4\\
		0&1&-1&1\\
		0&0&1&2
		\CodeAfter
		\tikz \draw [cstcurve,second,visible on=<+->] (1-|1) |- (2-|2) |- (3-|3) |- (4-|5);
	\end{pNiceMatrix}}\]
	\onslide<+->
	经过若干次初等\alert{行变换}, 矩阵变为\emph{行阶梯形矩阵}.
\end{frame}


\begin{frame}{行阶梯形矩阵}
	\onslide<+->
	\begin{definition}
		满足下述条件的矩阵称为\emph{行阶梯形矩阵}:
		\begin{enumerate}
			\item 每个非零行的第一个非零元只出现在上一行第一个非零元的右边;
			\item 零行只可能出现在最下方.
		\end{enumerate}
	\end{definition}
	\onslide<+->
	换言之, 若 $\bfA\in M_{m\times n}$, 存在正整数
	\[1\le k_1<k_2<\cdots<k_\ell,\qquad \ell\le m\]
	使得 $a_{1,k_1},\dots,a_{\ell,k_\ell}$ 均非零; $j<k_i$ 或 $i>\ell$ 时 $a_{ij}=0$.
	\onslide<+->
	\[\begin{pNiceMatrix}
		1&3&-2&4\\
		0&1&-1&1\\
		0&0&1&2
		\CodeAfter
		\tikz \draw [cstcurve,second] (1-|1) |- (2-|2) |- (3-|3) |- (4-|5);
	\end{pNiceMatrix}
	\qquad
	\visible<+->{\begin{pNiceMatrix}
		0&3&0&0&-1\\
		0&0&1&0&2\\
		0&0&0&0&1\\
		0&0&0&0&0
		\CodeAfter
		\tikz \draw [cstcurve,second] (1-|2) |- (2-|3) |- (3-|5) |- (4-|6);
	\end{pNiceMatrix}}
	\qquad
	\visible<+->{\begin{pmatrix}
		2&3&4&-1\\
		1&0&1&2\\
		0&0&2&1\\
		0&0&3&-1
	\end{pmatrix}
	\alert{\text{\Large$\times$}}}\]
	\onslide<+->
	任何矩阵都可通过初等行变换化为行阶梯形.
\end{frame}


\begin{frame}{行最简形矩阵}
	\onslide<+->
	\vspace{-\baselineskip}
	\[\begin{pmatrix}
		1&3&-2&4\\
		0&1&-1&1\\
		0&0&1&2
	\end{pmatrix}
	\visible<+->{\wsim{r_2+r_3}{r_1+2r_3}\begin{pmatrix}
		1&3&0&8\\
		0&1&0&3\\
		0&0&1&2
	\end{pmatrix}}
	\visible<+->{\wsim{r_1-3r_2}{}\begin{pNiceMatrix}
		1&\alertn0&\alertn0&-1\\
		\alertn0&1&\alertn0&3\\
		\alertn0&\alertn0&1&2
		\CodeAfter
		\tikz \draw [cstcurve,second,rounded corners] (1-|1) rectangle (2-|2);
		\tikz \draw [cstcurve,second,rounded corners] (2-|2) rectangle (3-|3);
		\tikz \draw [cstcurve,second,rounded corners] (3-|3) rectangle (4-|4);
	\end{pNiceMatrix}}\]
	\onslide<+->
	再经过若干次初等变换, 矩阵变为\emph{行最简形矩阵}.
	\onslide<+->
	\begin{definition}
		满足下述条件的行阶梯形矩阵称为\emph{行最简形矩阵}:
		\begin{enumerate}
			\item 每个非零行的第一个非零元是 $1$;
			\item 每个非零行的第一个非零元所在列其它元素均为 $0$.
		\end{enumerate}
	\end{definition}
	\onslide<+->
	例如 $\displaystyle\begin{pNiceMatrix}
		0&1&\alertn0&2&\alertn0\\
		0&\alertn0&1&1&\alertn0\\
		0&\alertn0&\alertn0&0&1\\
		0&\alertn0&\alertn0&0&\alertn0
		\CodeAfter
		\tikz \draw [cstcurve,second,rounded corners] (1-|2) rectangle (2-|3);
		\tikz \draw [cstcurve,second,rounded corners] (2-|3) rectangle (3-|4);
		\tikz \draw [cstcurve,second,rounded corners] (3-|5) rectangle (4-|6);
	\end{pNiceMatrix}$.
	\onslide<+->
	任何矩阵都可通过初等行变换化为行最简形.
\end{frame}


\begin{frame}{例: 行最简形矩阵}
	\onslide<+->
	\begin{example}
		用初等行变换将 $\bfA=\begin{pmatrix}
			1&3&-9&3\\
			0&1&-3&4\\
			-2&-3&9&6
		\end{pmatrix}$ 化为行最简形矩阵.
	\end{example}
	\onslide<+->
	\begin{solution}
		\[\begin{pmatrix}
			1&3&-9&3\\
			0&1&-3&4\\
			-2&-3&9&6
		\end{pmatrix}
		\visible<+->{\wsim{r_3+2r_1}{}\begin{pmatrix}
			1&3&-9&3\\
			0&1&-3&4\\
			0&3&-9&12
		\end{pmatrix}}
		\visible<+->{\wsim{
			r_3-3r_2}{}\begin{pmatrix}
				1&3&-9&3\\
				0&1&-3&4\\
				0&0&0&0
		\end{pmatrix}}\]\[
		\visible<+->{\wsim{
			r_1-3r_2}{}\begin{pNiceMatrix}
				1&0&0&-9\\
				0&1&-3&4\\
				0&0&0&0
				\CodeAfter
				\tikz \draw [cstcurve,second,rounded corners] (1-|1) rectangle (2-|2);
				\tikz \draw [cstcurve,second,rounded corners] (2-|2) rectangle (3-|3);
		\end{pNiceMatrix}.}
		\]
	\end{solution}
\end{frame}


\begin{frame}{方阵的行最简形}
	\onslide<+->
	\begin{proposition}
		$n$ 阶方阵 $\bfA$ 可逆$\iff$它的行最简形为 $\bfE_n$.
	\end{proposition}
	\onslide<+->
	注意到初等行变换不改变行列式的非零性.
	\onslide<+->
	若 $\bfA$ 可逆, 则它的行最简形 $\bfB$ 不能有零行, 从而 $\bfB$ 有 $n$ 个非零行.
	\onslide<+->
	而每个非零行的首个非零元所在列数递增, 因此第 $i$ 行的首个非零元一定在第 $i$ 列, 从而 $\bfB=\bfE$.

	\onslide<+->
	由此可知, 存在一系列初等矩阵 $\bfP_1,\dots,\bfP_k$ 使得
	$\bfP_1\cdots\bfP_k\bfA=\bfE$.
	\onslide<+->
	从而 $\bfA=\bfP_k^{-1}\cdots\bfP_1^{-1}$, \alert{可逆方阵可以写成有限个初等矩阵的乘积}.
\end{frame}


\begin{frame}{行最简形的应用}
	\onslide<+->
	\begin{proposition}
		设 $\bfA$ 是 $n$ 阶方阵.
		若 $|\bfA|=0$, 则 $\bfA\bfx={\bf0}$ 有非零解.
	\end{proposition}
	\onslide<+->
	\begin{proof}
		设 $\bfA^\rmT$ 的行最简形为 $\bfB=\bfP\bfA^\rmT$,
		\onslide<+->{%
		其中 $\bfP$ 是一些初等矩阵的乘积, 从而可逆.
		}\onslide<+->{%
		由于 $|\bfA^\rmT|=|\bfA|=0$, 因此 $\bfB\neq\bfE$ 有零行.
		}\onslide<+->{%
		设 $\bfv=(0,\dots,0,1)^\rmT$, 那么 $\bfB^\rmT\bfv={\bf0}$.
		}\onslide<+->{%
		从而 $\bfA\bfP^\rmT\bfv={\bf0}$.
		}\onslide<+->{%
		由于 $\bfP^\rmT$ 可逆, 它的最后一列 $\bfP^\rmT\bfv$ 非零.\qedhere.
		}
	\end{proof}
	\onslide<+->
	\begin{corollary}
		设 $\bfA$ 为 $m\times n$ 矩阵.
		若 $m<n$, 则 $\bfA\bfx={\bf0}$ 有非零解.
	\end{corollary}
	\onslide<+->
	\begin{proof}
		将 $\bfA$ 补充 $n-m$ 个零行得到 $\bfB=\begin{pmatrix}
			\bfA\\\bfO
		\end{pmatrix}$,
		\onslide<+->{%
		则 $|\bfB|=0$, 存在非零向量 $\bfx$ 使得 $\bfB\bfx={\bf0}$.
		}\onslide<+->{%
		从而 $\bfA\bfx={\bf0}$.\qedhere
		}
	\end{proof}
\end{frame}


\begin{frame}{矩阵等价的定义}
	\onslide<+->
	\begin{definition}
		\begin{enumerate}
			\item 若 $\bfA$ 经过有限次初等行变换变为 $\bfB$, 则称 $\bfA$ 和 $\bfB$ \emph{行等价}, 记作\emph{$\bfA\simr\bfB$}.
			\item 若 $\bfA$ 经过有限次初等列变换变为 $\bfB$, 则称 $\bfA$ 和 $\bfB$ \emph{列等价}, 记作\emph{$\bfA\simc\bfB$}.
			\item 若 $\bfA$ 经过有限次初等行变换和初等列变换变为 $\bfB$, 则称 $\bfA$ 和 $\bfB$ \emph{等价}, 记作\emph{$\bfA\sim\bfB$}.
		\end{enumerate}
	\end{definition}
	\onslide<+->
	由于可逆方阵可以写成有限个初等矩阵的乘积, 因此:
	\onslide<+->
	\begin{theorem}
		\begin{enumerate}
			\item $\bfA\simr\bfB$ 当且仅当存在可逆矩阵 $\bfP$ 使得 $\bfB=\bfP\bfA$.
			\item $\bfA\simc\bfB$ 当且仅当存在可逆矩阵 $\bfQ$ 使得 $\bfB=\bfA\bfQ$.
			\item $\bfA\sim\bfB$ 当且仅当存在可逆矩阵 $\bfP,\bfQ$ 使得 $\bfB=\bfP\bfA\bfQ$.
		\end{enumerate}
	\end{theorem}
\end{frame}


\begin{frame}{矩阵等价的性质}
	\onslide<+->
	由此可知
	\begin{proposition}
		矩阵的行等价、列等价、等价均满足
		\begin{enumerate}
			\item 自反性: $\bfA\sim\bfA$;
			\item 对称性: $\bfA\sim\bfB\implies\bfB\sim\bfA$;
			\item 传递性: $\bfA\sim\bfB,\bfB\sim\bfC\implies\bfA\sim\bfC$.
		\end{enumerate}
	\end{proposition}
\end{frame}


\begin{frame}{矩阵的标准型}
	\onslide<+->
	任一矩阵通过有限次初等行变换变为行最简形后, 可通过初等\alert{列变换}将其变为\emph{标准型} $\begin{pmatrix}
		\bfE_r&\bfO\\\bfO&\bfO
	\end{pmatrix}$.
	\onslide<+->
	例如:
	\[\begin{pmatrix}
		1&0&0&-9\\
		0&1&-3&4\\
		0&0&0&0
	\end{pmatrix}\visible<+->{\wsim{c_4+9c_1}{}\begin{pmatrix}
		1&0&0&0\\
		0&1&-3&4\\
		0&0&0&0
	\end{pmatrix}}
	\visible<+->{\wsim{c_3+3c_2}{c_4-4c_2}\begin{pmatrix}
		1&0&0&0\\
		0&1&0&0\\
		0&0&0&0
	\end{pmatrix}.}\]
	\vspace{-\baselineskip}

	\onslide<+->
	矩阵的等价也叫做\emph{相抵}, 上述标准型也叫作\emph{相抵标准型}.
	\onslide<+->
	我们会看到不同的 $r$ 对应的相抵标准型不等价.
	\onslide<+->
	所以相抵标准型相当于在每一个等价类中找到了一个具有代表性的矩阵.

	\onslide<+->
	\begin{proposition}
		$n$ 阶方阵 $\bfA$ 可逆当且仅当它的标准型为 $\bfE_n$.
	\end{proposition}
\end{frame}


\begin{frame}{矩阵的变换关系}
	\onslide<+->
	\begin{center}
		\begin{tikzpicture}[node distance=35mm]
			\node (1) {任意矩阵};
			\node (2) [right=of 1]{行阶梯形矩阵};
			\node (3) [right=of 2]{行最简形矩阵};
			\node (4) [below=of 3]{标准型矩阵};
			\draw[Implies-Implies,double,double distance=1mm,line width=1pt] (1)--(2)
				node[midway,above] {\kaishu 有限次};
			\path (1)--(2)
				node[midway,below] {\kaishu 初等行变换};
			\draw[Implies-Implies,double,double distance=1mm,line width=1pt] (2)--(3)
				node[midway,above] {\kaishu 有限次};
			\path (2)--(3) node[midway,below] {\kaishu 初等行变换};
			\draw[Implies-Implies,double,double distance=1mm,line width=1pt,main] (3)--(4)
				node[midway,above,sloped] {\kaishu 有限次初等列变换};
			\draw[Implies-Implies,double,double distance=1mm,line width=1pt,second] (1)--(4)
				node[midway,above,sloped] {\kaishu 有限次初等变换};
			\begin{scope}[visible on=<2->]
				\node (1n) [above=5pt of 1,fourth]{可逆方阵};
				\node (2n) [above=5pt of 2,fourth]{可逆上三角阵};
				\node (3n) [above=5pt of 3,fourth]{$\bfE_n$};
				\node (4n) [below=5pt of 4,fourth]{$\bfE_n$};
			\end{scope}
		\end{tikzpicture}
	\end{center}
\end{frame}


\begin{frame}{例: 初等变换}\small
\beqskip{2pt}
	\onslide<+->
	\begin{example}
		将矩阵 $\bfA=\begin{pmatrix}
			1&0&0\\
			2&0&-1\\
			0&-1&0
		\end{pmatrix}$ 表示成有限个初等阵的乘积.
	\end{example}
	\onslide<+->
	\begin{solution}
		\[\bfA\wsim{r_2\swap r_3}{}\begin{pmatrix}
			1&0&0\\
			0&-1&0\\
			2&0&-1
		\end{pmatrix}
		\wsim{r_3-2r_1}{}\begin{pmatrix}
			1&0&0\\
			0&-1&0\\
			0&0&-1
		\end{pmatrix}
		\wsim{-r_2}{-r_3}\begin{pmatrix}
			1&0&0\\
			0&1&0\\
			0&0&1
		\end{pmatrix}.\]
		\onslide<+->{因此 $\begin{pmatrix}
			1&0&0\\0&1&0\\0&0&-1
		\end{pmatrix}\begin{pmatrix}
			1&0&0\\0&-1&0\\0&0&1
		\end{pmatrix}\begin{pmatrix}
			1&0&0\\0&1&0\\-2&0&1
		\end{pmatrix}\begin{pmatrix}
			1&0&0\\0&0&1\\0&1&0
		\end{pmatrix}\bfA=\bfE$, 
		}\onslide<+->{%
		\[\bfA=\begin{pmatrix}
			1&0&0\\0&0&1\\0&1&0
		\end{pmatrix}\begin{pmatrix}
			1&0&0\\0&1&0\\2&0&1
		\end{pmatrix}\begin{pmatrix}
			1&0&0\\0&-1&0\\0&0&1
		\end{pmatrix}\begin{pmatrix}
			1&0&0\\0&1&0\\0&0&-1
		\end{pmatrix}.\]}
		\vspace{-.3\baselineskip}
	\end{solution}
\endgroup
\end{frame}


\subsection{初等变换解矩阵方程}

\begin{frame}{初等变换解矩阵方程}
	\onslide<+->
	若 $\bfA$ 可逆, 则 $\bfA\bfX=\bfB\iff \bfX=\bfA^{-1}\bfB$.
	\onslide<+->
	若 $(\bfA,\bfB)\simr (\bfE,\bfX)$, 则存在可逆矩阵 $\bfP$ 使得 $\bfP(\bfA,\bfB)=(\bfE,\bfX)$.
	\onslide<+->
	即 $\bfP=\bfA^{-1},\bfX=\bfA^{-1}\bfB$.
	\onslide<+->
	所以
	\[\alertn{\bfA\bfX=\bfB\iff \bfX=\bfA^{-1}\bfB\iff (\bfA,\bfB)\simr (\bfE,\bfX)}.\]	
	\onslide<+->
	同理
	\[\alertn{\bfX\bfA=\bfC\iff \bfX=\bfC\bfA^{-1}\iff\begin{pmatrix}
		\bfA\\
		\bfB
	\end{pmatrix}\simc\begin{pmatrix}
		\bfE\\
		\bfX
	\end{pmatrix}}.\]
	\onslide<+->
	所以我们可使用初等变换解该类型矩阵方程.
	
	\onslide<+->
	特别地, \alert{$(\bfA,\bfE)\simr(\bfE,\bfA^{-1})$} 可用来帮助计算矩阵的逆.
	\onslide<+->
	\begin{example}
		求 $\bfA=\begin{pmatrix}
			1&1&2\\1&2&3\\2&1&4
		\end{pmatrix}$ 的逆.
	\end{example}
\end{frame}


\begin{frame}{例: 初等变换解矩阵方程}
\beqskip{2pt}
	\onslide<+->
	\begin{solution}
		\begin{align*}
			(\bfA,\bfE)
			&=\begin{pNiceMatrix}
				1&1&2&1&0&0\\
				1&2&3&0&1&0\\
				2&1&4&0&0&1
				\augdash{4}{4}
			\end{pNiceMatrix}
			\onslide<+->{%
			\wsim[3]{r_2-r_1}{r_3-2r_1}\begin{pNiceMatrix}
				1&1&2&1&0&0\\
				0&1&1&-1&1&0\\
				0&-1&0&-2&0&1
				\augdash{4}{4}
			\end{pNiceMatrix}
			}\\
			&\onslide<+->{%
			\wsim[3]{r_3+r_2}{}\begin{pNiceMatrix}
				1&1&2&1&0&0\\
				0&1&1&-1&1&0\\
				0&0&1&-3&1&1
				\augdash{4}{4}
			\end{pNiceMatrix}
			}\onslide<+->{%
			\wsim[3]{r_1-2r_3}{r_2-r_3}\begin{pNiceMatrix}
				1&1&0&7&-2&-2\\
				0&1&0&2&0&-1\\
				0&0&1&-3&1&1
				\augdash{4}{4}
			\end{pNiceMatrix}
			}\\
			&\onslide<+->{%
			\wsim[3]{r_1-r_2}{}\begin{pNiceMatrix}
				1&0&0&5&-2&-1\\
				0&1&0&2&0&-1\\
				0&0&1&-3&1&1
				\augdash{4}{4}
			\end{pNiceMatrix}.
			}
		\end{align*}
		\onslide<+->{%
		故 $\bfA^{-1}=\begin{pmatrix}
			5&-2&-1\\
			2&0&-1\\
			-3&1&1
		\end{pmatrix}$.}
	\end{solution}
\endgroup
\end{frame}


\begin{frame}{例: 初等变换解矩阵方程}
	\onslide<+->
	\begin{example}
		若 $\bfA=\begin{pmatrix}
			2&2&0\\2&1&3\\0&1&0
		\end{pmatrix},\bfA\bfX=\bfA+\bfX$, 求 $\bfX$.
	\end{example}
	\onslide<+->
	\begin{solution}
		由题设知 $(\bfA-\bfE)\bfX=\bfA$, $\bfX=(\bfA-\bfE)^{-1}\bfA$.
		\onslide<+->{\[(\bfA-\bfE,\bfA)=\begin{pNiceMatrix}
			1&2&0&2&2&0\\
			2&0&3&2&1&3\\
			0&1&-1&0&1&0
			\augdash{4}{4}
		\end{pNiceMatrix}
		\visible<+->{\wsim[3]{r_2-2r_1}{r_2\swap r_3}\begin{pNiceMatrix}
			1&2&0&2&2&0\\
			0&1&-1&0&1&0\\
			0&-4&3&-2&-3&3
			\augdash{4}{4}
		\end{pNiceMatrix}}\]}
		\vspace{-\baselineskip}
	\end{solution}
\end{frame}


\begin{frame}{例: 初等变换解矩阵方程}
	\onslide<+->
	\begin{solution}[续解]
		\[(\bfA-\bfE,\bfA)\simr\begin{pNiceMatrix}
			1&2&0&2&2&0\\
			0&1&-1&0&1&0\\
			0&-4&3&-2&-3&3
			\augdash{4}{4}
		\end{pNiceMatrix}\wsim[3]{r_3+4r_2}{-r_3}\begin{pNiceMatrix}
			1&2&0&2&2&0\\
			0&1&-1&0&1&0\\
			0&0&1&2&-1&-3
			\augdash{4}{4}
		\end{pNiceMatrix}\]

		\onslide<+->{\[\wsim[3]{r_2+r_3}{}\begin{pNiceMatrix}
			1&2&0&2&2&0\\
			0&1&0&2&0&-3\\
			0&0&1&2&-1&-3
			\augdash{4}{4}
		\end{pNiceMatrix}\visible<+->{\wsim[3]{r_1-2r_2}{}\begin{pNiceMatrix}
			1&0&0&-2&2&6\\
			0&1&0&2&0&-3\\
			0&0&1&2&-1&-3
			\augdash{4}{4}
			\end{pNiceMatrix}.}\]}
		\onslide<+->{故 $\bfX=\begin{pmatrix}
			-2&2&6\\
			2&0&-3\\
			2&-1&-3
		\end{pmatrix}$.}
	\end{solution}
\end{frame}


\begin{frame}{例: 解矩阵方程}
	\beqskip{5pt}
	\onslide<+->
	\begin{example}
		解矩阵方程 $\bfA^*\bfX=2\bfE+2\bfX$, 其中 $\bfA=\begin{pmatrix}
			1&1&-1\\-1&1&1\\1&-1&1
		\end{pmatrix}$.
	\end{example}
	\onslide<+->
	\begin{solution}
		注意到 $|\bfA|=4$.
		\onslide<+->{%
		两边同时左乘 $\bfA$ 得到 $4\bfX=2\bfA+2\bfA\bfX$, $(2\bfE-\bfA)\bfX=\bfA$.
		}\onslide<+->{%
		\[(2\bfE-\bfA,\bfA)=\begin{pNiceMatrix}
			 1&-1& 1& 1& 1&-1\\
			 1& 1&-1&-1& 1& 1\\
			-1& 1& 1& 1&-1& 1
			\augdash{4}{4}
		\end{pNiceMatrix}
		\visible<+->{\simr\begin{pNiceMatrix}
			1&0&0&0&1&0\\
			0&1&0&0&0&1\\
			0&0&1&1&0&0
			\augdash{4}{4}
		\end{pNiceMatrix}.}\]
		}\onslide<+->{%
		故 $\bfX=\begin{pmatrix}
			0&1&0\\
			0&0&1\\
			1&0&0
		\end{pmatrix}$.
		}
	\end{solution}
	\endgroup
\end{frame}


\begin{frame}{例: 解矩阵方程}
	\onslide<+->
	\begin{exercise}
		解矩阵方程 $\bfA^*\bfX\bfA=2\bfX\bfA-8\bfE$, 其中 $\bfA=\begin{pmatrix}
			1&2&-2\\0&-2&4\\0&0&1
		\end{pmatrix}$.
	\end{exercise}
	\onslide<+->
	\begin{answer}
		两边同时左乘 $\bfA$ 右乘 $\bfA^{-1}$ 得到 $-2\bfX=2\bfA\bfX-8\bfE$, $(\bfA+\bfE)\bfX=4\bfE$,
		\[(\bfA+\bfE,4\bfE)=\begin{pNiceMatrix}
			 2& 2&-2& 4& 0& 0\\
			 0&-1& 4& 0& 4& 0\\
			 0& 0& 2& 0& 0& 4
			\augdash{4}{4}
		\end{pNiceMatrix}
		\simr\begin{pNiceMatrix}
			1&0&0&2& 4&-6\\
			0&1&0&0&-4& 8\\
			0&0&1&0& 0& 2
			\augdash{4}{4}
		\end{pNiceMatrix}.\]
	\end{answer}
\end{frame}


\begin{frame}{例: 初等变换}
	\onslide<+->
	\begin{exercise}
		\begin{enumerate}
			\item 设 $\bfA$ 是 $3$ 阶方阵, 存在可逆阵 $\bfP$ 使得 $\bfP^{-1}\bfA\bfP=\begin{pmatrix}
				1&&\\&2&\\&&3
			\end{pmatrix}$, 则 $\bfP^{-1}\bfA^*\bfP=$\fillblankframe[3cm]{$\diag(6,3,2)$}.
			\item 设 $\bfA$ 是 $3$ 阶方阵, 存在可逆阵 $\bfP=(\bma_1,\bma_2,\bma_3)$ 使得 $\bfP^{-1}\bfA\bfP=\begin{pmatrix}
				1&&\\&2&\\&&3
			\end{pmatrix}$.
			若 $\bfQ=(\bma_1,\bma_3,\bma_2)$, 则 $\bfQ^{-1}\bfA\bfQ=$\fillblankframe[3cm]{$\diag(1,3,2)$}.
			\item 设 $n$ 阶方阵 $\bfA,\bfB$ 满足 $\bfA\bfB=\bfE$, 则以下说法正确的有\fillblankframe{$4$}个.
			\begin{tasks}[label={(\roman*)},label-format=\upshape\textcolor{main}](5)
				\task*(3) $\bfA$ 等价于 $\bfE$;
				\task*(2) $\bfA$ 等价于 $\bfB$;
				\task*(3) $\bfA$ 可经过有限次初等行变换化为 $\bfB$;
				\task*(2) $\bfA\bfB=\bfB\bfA$.
			\end{tasks}
		\end{enumerate}
	\end{exercise}
\end{frame}
