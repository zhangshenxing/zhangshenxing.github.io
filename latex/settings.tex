\definecolor{main}{RGB}{0,0,224}%
\definecolor{second}{RGB}{224,0,0}%
\definecolor{third}{RGB}{112,0,112}%
% \definecolor{fourth}{RGB}{196,128,0}%
\definecolor{fourth}{RGB}{0,128,0}%
\definecolor{fifth}{RGB}{255,128,0}%

% \definecolor{main}{RGB}{73,119,213}%
% \definecolor{second}{RGB}{220,60,105}%
% \definecolor{third}{RGB}{122,89,166}%
% \definecolor{fourth}{RGB}{229,143,52}%
% \definecolor{fifth}{RGB}{244,220,10}%

% \setbeamertemplate{footline}[simple]
\usepackage{bm}
\usepackage{extarrows}
\usepackage{mathrsfs}
\usepackage{stmaryrd}
\usepackage{multirow}
\usepackage{tikz}
\usepackage{color,calc}
\usepackage{caption}
\usepackage{subcaption}
\usepackage{booktabs}
% 表格色
\newcommand\defaultrowcolors{\rowcolors{1}{main!15}{main!15}}
\newcommand\tht{\cellcolor{main}\color{white}}
% 文本色
\renewcommand\emph[1]{{\color{main}{#1}}}
\ifcsundef{alert}{  
  \newcommand{\alert}[1]{\textcolor{second}{\bf #1}}
}{}
\newcommand\abox[1]{\colorbox{yellow}{\alert{#1}}}
\newcommand\aboxeq[1]{\abox{$\displaystyle #1$}}
% 非考试内容
\newcommand{\noexer}{\hfill\mdseries\itshape\color{black}\small 非考试内容}
% 枚举数字引用标志
\newcommand\enumnum[1]{{\mdseries\upshape\textcolor{main}{(#1)}}}
% 缩进
\renewcommand{\indent}{\hspace*{1em}}
\setlength{\parindent}{1em}
\newcommand\peq{\mathrel{\phantom{=}}} % 用于对齐的等号幻影
\NewDocumentCommand\meq{O{0pt} O{0pt} m}{\par\vspace{#1}\begin{center}$\displaystyle #3$\end{center}\vspace{#2}} % 减少列表公式后的多余空白
\newcommand\beqskip[1]{\begingroup\abovedisplayskip=#1\belowdisplayskip=#1\belowdisplayshortskip=#1} % 减少公式垂直间距, 配合\endgroup


\RequirePackage{enumerate}
\RequirePackage[shortlabels,inline]{enumitem}
\RequirePackage{tasks}
\newcommand*{\eitemi}{\tikz \draw [baseline, ball color=main,draw=none] circle (2pt);}
\newcommand*{\eitemii}{\tikz \draw [baseline, fill=main,draw=none,circular drop shadow] circle (2pt);}
\newcommand*{\eitemiii}{\tikz \draw [baseline, fill=main,draw=none] circle (2pt);}
\setlist[itemize,1]{label={\eitemi}}
\setlist[itemize,2]{label={\eitemii}}
\setlist[itemize,3]{label={\eitemiii}}
\setlist[enumerate]{format=\upshape\textcolor{main}}
\setlist[enumerate,1]{label=\color{main}(\arabic*)}
\setlist[enumerate,2]{label=\color{main}(\alph*).}
\setlist[enumerate,3]{label=\color{main}\Roman*.}
\setlist[enumerate,4]{label=\color{main}\Alph*.}
\newlist{homeworklist}{enumerate}{1}
\setlist[homeworklist,1]{label=\color{main}{\zhnum*、\hspace*{-\labelsep}}}
\newlist{exlist}{enumerate}{3}
\setlist[exlist,1]{label=\color{main}{\arabic*.},itemindent=0em}
\setlist[exlist,2]{label=\color{main}{(\arabic*)}}
\setlist[exlist,3]{label=\color{main}{(\roman*)}}
\settasks{label-format=\textcolor{main},label={(\arabic*)},label-width=1.5em}
\setlist{nolistsep}
\NewTasksEnvironment[label={(\Alph*)},label-width=1.5em]{taskschoice}[()]
\NewTasksEnvironment[label={\arabic*.},label-width=1.5em]{tasksans}[()]

% 选择题, 根据选项内容长度自动排版
\newlength{\ltemp}\newlength{\lxxmax}\newlength{\lquar}\newlength{\lhalf}\newlength{\lfull}
\newcounter{lxxtype}
\NewDocumentCommand\xx{O{0} m m m m}{%
	\setlength{\lfull}{\columnwidth}%
	\addtolength{\lfull}{-\leftmargin}%
	\setlength{\lhalf}{0.5\lfull}%
	\setlength{\lquar}{0.25\lfull}%
	\setcounter{lxxtype}{0}%
	\ifnum#1=1\setcounter{lxxtype}{1}\fi%
	\ifnum#1=2\setcounter{lxxtype}{2}\fi%
	\ifnum#1=4\setcounter{lxxtype}{4}\fi%
	\settowidth{\lxxmax}{(A)~#2~}% 获取最长选项长度
	\settowidth{\ltemp}{(B)~#3~}%
	\ifdimcomp\ltemp>\lxxmax{\setlength{\lxxmax}{\ltemp}}{}%
	\settowidth{\ltemp}{(C)~#4~}%
	\ifdimcomp\ltemp>\lxxmax{\setlength{\lxxmax}{\ltemp}}{}%
	\settowidth{\ltemp}{(D)~#5~}%
	\ifdimcomp\ltemp>\lxxmax{\setlength{\lxxmax}{\ltemp}}{}%
	\ifnum\value{lxxtype}=0%
		\setcounter{lxxtype}{4}%
		\ifdimcomp\lxxmax>\lquar{\setcounter{lxxtype}{2}}{}%
		\ifnum\value{lxxtype}=2%
			\ifdimcomp\lxxmax>\lhalf{\setcounter{lxxtype}{1}}{}%
		\fi%
	\fi%
	\vspace{5pt}%
	\ifnum\value{lxxtype}=1%
		\\\makebox[\lfull][l]{(A)~#2}%
		\\\makebox[\lfull][l]{(B)~#3}%
		\\\makebox[\lfull][l]{(C)~#4}%
		\\\makebox[\lfull][l]{(D)~#5}%
	\fi%
	\ifnum\value{lxxtype}=2%
		\\\makebox[\lhalf][l]{(A)~#2}%
			\makebox[\lhalf][l]{(B)~#3}%
		\\\makebox[\lhalf][l]{(C)~#4}%
			\makebox[\lhalf][l]{(D)~#5}%
	\fi%
	\ifnum\value{lxxtype}=4%
		\\\makebox[\lquar][l]{(A)~#2}% 
			\makebox[\lquar][l]{(B)~#3}%
			\makebox[\lquar][l]{(C)~#4}%
			\makebox[\lquar][l]{(D)~#5}%
	\fi%
}
% TIKZ 设置
\usetikzlibrary{
	quotes,
	shapes.arrows,
	arrows.meta,
	positioning,
	shapes.geometric,
	overlay-beamer-styles,
	patterns,
	calc,
	angles,
	decorations.pathreplacing,
	backgrounds % 背景边框
}
\tikzset{
	background rectangle/.style={semithick,draw=fourth,fill=white,rounded corners},
  % arrow
	cstra/.style      ={-Stealth},        % right arrow
	cstla/.style      ={Stealth-},        % left arrow
	cstlra/.style     ={Stealth-Stealth}, % left-right arrow
	cstwra/.style     ={-Straight Barb},  % wide ra
	cstwla/.style     ={Straight Barb-},
	cstwlra/.style    ={Straight Barb-Straight Barb},
	cstnarrow/.style      ={-Latex, line width=0.1cm}, %文本框间箭头
	cstaxis/.style        ={-Stealth, thick}, %坐标轴
  % curve
	cstcurve/.style       ={very thick}, %一般曲线
	cstdash/.style        ={thick, dash pattern= on 0.2cm off 0.05cm}, %虚线
  % dot
	cstdot/.style         ={radius=.08}, %实心点
	cstdote/.style        ={radius=.07, fill=white}, %空心点
  % fill
	cstfill/.style       ={fill=black!10},
	cstfille/.style      ={pattern=north east lines, pattern color=black},
	cstfill1/.style       ={fill=main!20},
	cstfille1/.style      ={pattern=north east lines, pattern color=main},
	cstfill2/.style        ={fill=second!20},
	cstfille2/.style       ={pattern=north east lines, pattern color=second},
	cstfill3/.style        ={fill=third!20},
	cstfille3/.style       ={pattern=north east lines, pattern color=third},
	cstfill4/.style        ={fill=fourth!20},
	cstfille4/.style       ={pattern=north east lines, pattern color=fourth},
	cstfill5/.style        ={fill=fifth!20},
	cstfille5/.style       ={pattern=north east lines, pattern color=fifth},
  % node
	cstnode/.style        ={fill=white,draw=black,text=black,rounded corners=0.2cm,line width=1pt},
	cstnode1/.style       ={fill=main!15,draw=main!80,text=black,rounded corners=0.2cm,line width=1pt},
	cstnode2/.style       ={fill=second!15,draw=second!80,text=black,rounded corners=0.2cm,line width=1pt},
	cstnode3/.style       ={fill=third!15,draw=third!80,text=black,rounded corners=0.2cm,line width=1pt},
	cstnode4/.style       ={fill=fourth!15,draw=fourth!80,text=black,rounded corners=0.2cm,line width=1pt},
	cstnode5/.style       ={fill=fifth!15,draw=fifth!80,text=black,rounded corners=0.2cm,line width=1pt}
}