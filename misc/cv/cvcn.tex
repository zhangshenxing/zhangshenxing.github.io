\documentclass[11pt]{article}
\usepackage{cv_style}
\usepackage{multicol}\usepackage{ctex}

\begin{document}

\author{张神星}
\address{安徽合肥丹霞路485号合肥工业大学翡翠科教楼B1810}
\email{zhangshenxing@hfut.edu.cn}
\website{https://zhangshenxing.github.io/}
\maketitle

\section{教育经历}
\begin{cvstage}
	\editem{数学博士}
		{2015年11月}
		{中国科学技术大学数学科学学院}
		{导师: 欧阳毅}
	\editem{数学学士}
		{2010年6月}
		{中国科学技术大学少年班学院}
		{导师: 欧阳毅}
	\editem{访问学生}
		{2015年5月 -- 2015年6月}
		{加州大学伯克利分校}
		{导师: 袁新意}
\end{cvstage}


\section{学术经历}
\begin{cvstage}
	\editem{副研究员}
		{2021年12月 -- 至今}
		{合肥工业大学数学学院}
		{}
	\editem{博士后}
		{2018年4月 -- 2021年11月}
		{中国科学技术大学数学科学学院}
		{合作导师: 欧阳毅}
	\editem{博士后}
		{2016年3月 -- 2018年2月}
		{中国科学院数学与系统科学研究院}
		{合作导师: 田野}
\end{cvstage}


\section{出版物与预印本}
\subsection{论文}
\begin{cvlist}
	\item \textbf{S. Zhang}.
	The generating fields of twisted Kloosterman sums.
	{\em Int. J. Number Theory}, (2025), to appear.
	
	\item \textbf{S. Zhang}.
	On the linearity of the periods of subtraction games.
	{\em Theor. Comput. Sci.}, 985 (2024).
	
	\item \textbf{S. Zhang}.
	On a comparison of Cassels pairings of different elliptic curves.
	{\em Acta Arith.} 211 (2023), no. 1, 1--23.

	\item \textbf{S. Zhang}.
	On the Newton polygons of twisted $L$-functions of binomials.
	{\em Finite Fields Appl.} 80 (2022), Paper No. 102026, 20 pp.

	\item J. Li, \textbf{S. Zhang}.
	The $3$-class groups of $\mathbb{Q}(\sqrt[3]{p})$ and its normal closure.
	{\em Math. Z.} 300 (2022), no. 1, 209--215.

	\item J. Li, Y. Ouyang, Y. Xu, \textbf{S. Zhang}.
	$\ell$-Class groups of fields in Kummer towers.
	{\em Publ. Mat.} 66(2022), no. 1, 235--267.

	\item \textbf{S. Zhang}.
	The generating fields of two twisted Kloosterman sums.
	{\em J. Univ. Sci. Technol. China} 51 (2021), no. 12, 879-888.

	\item Y. Ouyang, \textbf{S. Zhang}.
	Birch's lemma over global function fields.
	{\em Proc. Amer. Math. Soc.} 145 (2017), no. 2, 577--584.

	\item Y. Ouyang, \textbf{S. Zhang}.
	Newton polygons of $L$-functions of polynomials $x^d+ax^{d-1}$ with $p\equiv -1 \bmod d$.
	{\em Finite Fields Appl.} 37 (2016), 285--294.

	\item Y. Ouyang, \textbf{S. Zhang}.
	On second 2-descent and non-congruent numbers.
	{\em Acta Arith.} 170 (2015), no. 4, 343--360.

	\item Y. Ouyang, \textbf{S. Zhang}.
	On non-congruent numbers with 1 modulo 4 prime factors.
	{\em Sci. China Math.} 57 (2014), no. 3, 649--658.
\end{cvlist}


\subsection{预印本}
\begin{cvlist}
	\item \textbf{S. Zhang}.
	On the Diophantine equation $my^2=x^3-k^3$.
	(2025), submitted.

	\item Z. Wang, \textbf{S. Zhang}.
	On the quadratic twist of elliptic curves with full $2$-torsion.
	(2022), submitted.

	\item \textbf{S. Zhang}.
	The virtual period of the degree sequences of the exponential sums.
	{\em Arxiv:} 2010.08342, preprint.
\end{cvlist}

\section{基金和奖励}
\subsection{主持基金}
\begin{cvlist}
	\item 合肥工业大学学术新人提升B计划(JZ2023HGTB0217)
	\item 国家自然科学基金委青年科学基金(12001510)
	\item 博士后科学基金(2017M611027)
\end{cvlist}


\subsection{参与基金}
\begin{cvlist}
	\item 基础加强计划重点基础研究项目
	\item 中国科学技术大学青年创新基金(WK0010000061)
	\item 国家自然科学基金委面上项目(12271335, 11571328, 11171317)
	\item 国家自然科学基金委青年科学基金(11601255, 11201445)
\end{cvlist}


\subsection{奖项}
\begin{cvlist}
	\item 2023年合肥工业大学青年教师教学基本功比赛二等奖
	\item 2013年研究生国家奖学金
	\item 2009年本科生国家奖学金
	\item 2006年安徽省优秀青年学生
\end{cvlist}


\section{教学经历}
\subsection{教学}
\begin{cvstage}
	\item 合肥工业大学\begin{cvstage}
		\csitem{1400261B 复变函数与积分变换}{2022-2024年秋}
		\csitem{1400071B 线性代数}{2024年秋}
		\csitem{034Y01 数学(下)}{2022-2023年春}
	\end{cvstage}
	\item 中国科学技术大学\begin{cvstage}
		\csitem{001548 复变函数B}{2020年秋}
		\csitem{MA05109 代数数论}{2020年春}
	\end{cvstage}
\end{cvstage}


\subsection{助教}
\begin{cvstage}
	\item 清华大学\begin{cvstage}
		\csitem{丘成桐清华数学夏令营, 线性代数}{2021-2023年夏}
	\end{cvstage}
	\item 合肥工业大学\begin{cvstage}
		\csitem{概率论与数理统计}{2021年秋}
	\end{cvstage}
	\item 中国科学院大学\begin{cvstage}
		\csitem{代数与数论暑期学校}{2019年夏}
		\csitem{初等数论}{2016年秋}
	\end{cvstage}
	\item 中国科学技术大学\begin{cvstage}
		\csitem{001356 代数学基础}{2014年秋}
		\csitem{001704 近世代数(H)}{2013年春}
		\csitem{001356 代数学基础}{2012年秋}
		\csitem{001010 近世代数}{2011年秋}
		\csitem{001012 复变函数}{2011年春}
	\end{cvstage}
\end{cvstage}

\subsection{教学论文}
\begin{cvlist}
	\item 袁志杰, \textbf{张神星}.
	复变函数在不同圆环域内洛朗展开的探究[J].
	{\em 南阳师范学院学报}, 2024, 23(3): 38--41.
	\item 欧阳毅, \textbf{张神星}.
	善用初等变换展开线性代数理论教学[J].
	{\em 大学数学}, 2023, 39(5): 68--75.
\end{cvlist}

\subsection{指导学生}
\begin{cvstage}
	\item 本科毕业论文\begin{cvstage}
		\csitem{邱修煜, 椭圆曲线的同源与密码学应用的研究}{2024}
		\csitem{刘旭鸿, 二次域类群的结构以及算术应用}{2022}
	\end{cvstage}
	\item 大学生创新创业训练计划\begin{cvstage}
		\csitem{朱思雯, 王悦妍, 邓颖, 姚佳婧, 罗天曦, 新型冠状病毒肺炎传染模型}{2022-2023}
	\end{cvstage}
\end{cvstage}


\section{期刊审稿人}
\begin{multicols}{2}
	\begin{cvlist}
		\item {\em International Journal of Number theory}
		\item {\em Algebra Colloquium}
		\item {\em The Ramanujan Journal}
		\item {\em Finite Fields and Their Applications}
		\item {\em Pure and Applied Mathematics Quarterly}
		\item {\em Bulletin of the Belgian Mathematical Society – Simon Stevin}
		\item {\em 中国科学 数学}
		\item {\em 大学数学}
	\end{cvlist}
\end{multicols}


\section{研究方向}
\begin{multicols}{2}
	\begin{cvlist}
		\item 椭圆曲线与非同余数问题
		\item 椭圆曲线与 Heegner 点
		\item 指数和
		\item 类群与岩泽理论
		% \item $p$ 进伽罗瓦表示与 $p$ 进霍奇理论
	\end{cvlist}
\end{multicols}


\medskip\emph{最后更新: 2025年3月29日}

\end{document}
