\documentclass[11pt,a4paper]{article}
\usepackage{cv_style}
\usepackage{ctex}

\begin{document}
\cvtitle{张神星}
  {安徽合肥丹霞路485号合肥工业大学翡翠科教楼B1810}
  {zhangshenxing@hfut.edu.cn}
  {https://zhangshenxing.gitee.io/}


\cvsection{教育经历}
\begin{enumerate}
\editem{数学博士}
  {2015年11月}
  {中国科学技术大学数学科学学院}
  {导师: 欧阳毅}
\editem{数学学士}
  {2010年6月}
  {中国科学技术大学少年班学院}
  {导师: 欧阳毅}
\editem{访问学生}
  {2015年5月 -- 2015年6月}
  {加州大学伯克利分校}
  {导师: 袁新意}
\end{enumerate}


\cvsection{学术经历}
\begin{enumerate}
\editem{副研究员}
  {2021年12月 -- 至今}
  {合肥工业大学数学学院}
  {}
\editem{博士后}
  {2018年4月 -- 2021年11月}
  {中国科学技术大学数学科学学院}
  {合作导师: 欧阳毅}
\editem{博士后}
  {2016年3月 -- 2018年2月}
  {中国科学院数学与系统科学研究院}
  {合作导师: 田野}
\end{enumerate}


\cvsection{出版物与预印本}
\cvsubsection{论文}
\begin{itemize}
\item \textbf{S. Zhang}.
On the Newton polygons of twisted $L$-functions of binomials.
{\em Finite Fields Appl.} 80 (2022), Paper No. 102026, 20 pp.

\item J. Li, \textbf{S. Zhang}.
The $3$-class groups of $\mathbb{Q}(\sqrt[3]{p})$ and its normal closure.
{\em Math. Z.} 300 (2022), no. 1, 209--215.

\item J. Li, Y. Ouyang, Y. Xu, \textbf{S. Zhang}.
$\ell$-Class groups of fields in Kummer towers.
{\em Publ. Mat.} 66(2022), no. 1, 235--267.

\item \textbf{S. Zhang}.
The generating fields of two twisted Kloosterman sums.
{\em J. Univ. Sci. Technol. China} 51 (2021), no. 12, 879-888.

\item Y. Ouyang, \textbf{S. Zhang}.
Birch's lemma over global function fields.
{\em Proc. Amer. Math. Soc.} 145 (2017), no. 2, 577--584.

\item Y. Ouyang, \textbf{S. Zhang}.
Newton polygons of $L$-functions of polynomials $x^d+ax^{d-1}$ with $p\equiv -1 \bmod d$.
{\em Finite Fields Appl.} 37 (2016), 285--294.

\item Y. Ouyang, \textbf{S. Zhang}.
On second 2-descent and non-congruent numbers.
{\em Acta Arith.} 170 (2015), no. 4, 343--360.

\item Y. Ouyang, \textbf{S. Zhang}.
On non-congruent numbers with 1 modulo 4 prime factors.
{\em Sci. China Math.} 57 (2014), no. 3, 649--658.
\end{itemize}


\cvsubsection{预印本}
\begin{itemize}
\item \textbf{S. Zhang}.
On a comparison of Cassels pairings of different elliptic curves.
(2022), preprint.

\item Z. Wang, \textbf{S. Zhang}.
On the quadratic twist of elliptic curves with full $2$-torsion.
(2022), preprint.

\item \textbf{S. Zhang}.
On linearity of the periods of subtraction games.
(2021), submitted.

\item \textbf{S. Zhang}.
The distinctness and generating fields of twisted Kloosterman sums.
(2021), submitted.

\item \textbf{S. Zhang}.
The virtual period of the degree sequences of the exponential sums.
{\em Arxiv:} 2010.08342, preprint.
\end{itemize}


\cvsection{基金和奖励}
\cvsubsection{主持基金}
\begin{itemize}
\item 国家自然科学基金委青年科学基金(12001510)
\item 博士后科学基金(2017M611027)
\end{itemize}


\cvsubsection{参与基金}
\begin{itemize}
\item 中国科学技术大学青年创新基金(WK0010000061)
\item 国家自然科学基金委青年科学基金(11601255)
\item 国家自然科学基金委面上项目(11571328)
\item 国家自然科学基金委青年科学基金(11201445)
\item 国家自然科学基金委面上项目(11171317)
\end{itemize}


\cvsubsection{奖项}
\begin{itemize}
\item 2013年研究生国家奖学金
\item 2009年本科生国家奖学金
\item 2006年安徽省优秀青年学生
\end{itemize}


\cvsection{教学经历}
\cvsubsection{教学}
\begin{enumerate}
\item 合肥工业大学
  \begin{enumerate}
  \csitem{1400261B 复变函数与积分变换}{2022年秋}
  \csitem{034Y01 数学(下)}{2022年春}
  \end{enumerate}
\item 中国科学技术大学
  \begin{enumerate}
  \csitem{001548 复变函数B}{2020年秋}
  \csitem{MA05109 代数数论}{2020年春}
  \end{enumerate}
\end{enumerate}


\cvsubsection{助教}
\begin{enumerate}
\item 清华大学
  \begin{enumerate}
  \csitem{丘成桐清华数学夏令营, 线性代数}{2022年夏}
  \csitem{丘成桐清华数学夏令营, 线性代数}{2021年夏}
  \end{enumerate}
\item 合肥工业大学
  \begin{enumerate}
  \csitem{概率论与数理统计}{2021年秋}
  \end{enumerate}
\item 中国科学院大学
  \begin{enumerate}
  \csitem{代数与数论暑期学校}{2019年夏}
  \csitem{初等数论}{2016年秋}
  \end{enumerate}
\item 中国科学技术大学
  \begin{enumerate}
  \csitem{001356 代数学基础}{2014年秋}
  \csitem{001704 近世代数(H)}{2013年春}
  \csitem{001356 代数学基础}{2012年秋}
  \csitem{001010 近世代数}{2011年秋}
  \csitem{001012 复变函数}{2011年春}
  \end{enumerate}
\end{enumerate}


\cvsection{期刊审稿人}
\begin{itemize}
\item {\em International Journal of Number theory}
\end{itemize}


\cvsection{研究方向}
\begin{itemize}
\item 椭圆曲线与非同余数问题
\item 椭圆曲线与 Heegner 点
\item 指数和
\item 类群与岩泽理论
\item $p$ 进伽罗瓦表示与 $p$ 进霍奇理论
\end{itemize}


\medskip\emph{最后更新: 2022年11月22日}

\end{document}
