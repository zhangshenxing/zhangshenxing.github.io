\documentclass[aspectratio=169,handout]{ctexbeamer}
\usepackage{../latex/bamboo}
\setbeamertemplate{footline}[naviboxall]
\usepackage{bm}
\usepackage{extarrows}
\usepackage{mathrsfs}
\usepackage{stmaryrd}
\usepackage{multirow}
\usepackage{tikz}
\usepackage{color,calc}
\usepackage{caption}
\usepackage{subcaption}
\usepackage{extarrows}
\usepackage{mathtools}
\usepackage{nicematrix}
\usepackage{booktabs}
\RequirePackage{colortbl}
\newcommand{\topcolorrule}{\arrayrulecolor{second}\toprule}
\newcommand{\midcolorrule}{\arrayrulecolor{fourth}\midrule}
\newcommand{\bottomcolorrule}{\arrayrulecolor{second}\bottomrule}
\usepackage[normalem]{ulem}
% 配色
\definecolor{main}{RGB}{0,0,224}
\definecolor{second}{RGB}{224,0,0}
\definecolor{third}{RGB}{112,0,112}
\definecolor{fourth}{RGB}{0,128,0}
\definecolor{fifth}{RGB}{255,128,0}
% \definecolor{main}{RGB}{0,0,0}%
% \definecolor{second}{RGB}{0,0,0}%
% \definecolor{third}{RGB}{0,0,0}%
% \definecolor{fourth}{RGB}{0,0,0}%
% \definecolor{fifth}{RGB}{0,0,0}%
% 缩进
\renewcommand{\indent}{\hspace*{1em}}
\setlength{\parindent}{1em}
% 减少公式垂直间距, 配合\endgroup
% 优先使用该效果调整, 其次使用\small
\newcommand{\beqskip}[1]{\begingroup\abovedisplayskip=#1\belowdisplayskip=#1\belowdisplayshortskip=#1}
% 矩阵间距
\NiceMatrixOptions{cell-space-limits = 1pt}
% 文本色, 尽量少用 \boldsymbol
\let\alert\relax\newcommand\alert[1]{{\color{second}{\bf #1}}}
\renewcommand{\emph}[1]{{\color{main}{\bf #1}}}
\newcommand{\alertn}[1]{{\color{second}{#1}}}
\newcommand{\emphn}[1]{{\color{main}{#1}}}
\newcommand{\alertm}[1]{{\textcolor{second}{\boldsymbol{#1}}}}
\newcommand{\emphm}[1]{{\color{main}{\boldsymbol{#1}}}}
% 文字堆叠
\newcommand{\cnen}[2]{{$\overset{\text{{#2}}}{\text{#1}}$}}
% 非考试内容
\newcommand{\noexer}{\hfill\mdseries\itshape\color{black}\small 非考试内容}
% 枚举数字引用标志
\newcommand\enumnum[1]{{\mdseries\upshape\textcolor{main}{(#1)}}}
% 枚举引用
\newcommand\enumref[1]{\textcolor{main}{\upshape(\ref{#1})}}

% 枚举环境
\newcommand{\setenumtype}[1]{
	\ifstrequal{#1}{(1)}{%
		\renewcommand{\theenumi}{{\mdseries\upshape\textcolor{main}{(\arabic{enumi})}}}
	}{}
	\ifstrequal{#1}{(i)}{%
		\renewcommand{\theenumi}{{\mdseries\upshape\textcolor{main}{(\roman{enumi})}}}
	}{}
  \setcounter{enumi}{0}%
}
% (V1)
\newenvironment{enumV}[1][V]{%
	\renewcommand{\theenumi}{{\mdseries\upshape\textcolor{main}{(#1\arabic{enumi})}}}
  \setcounter{enumi}{0}%
	\renewcommand{\item}{%
		\pause\par\noindent%
	\refstepcounter{enumi}\theenumi{ }}}%
{\ignorespacesafterend}
% 分段枚举环境
\newenvironment{enumpar}[1][(1)]{%
	\setenumtype{#1}\renewcommand{\item}{%
		\ifnum\value{enumi}=0\pause\fi%
		\par%
	\refstepcounter{enumi}\theenumi{ }}}%
{\ignorespacesafterend}
% % 首项不分段枚举环境
\newenvironment{enumnopar}[1][(1)]{%
	\setenumtype{#1}\renewcommand{\item}{%
		\ifnum\value{enumi}=0\else\pause\fi%
    \ifnum\value{enumi}=0\hspace{-1em}\else\par\noindent\fi%
  \refstepcounter{enumi}\theenumi { }}}%
{\ignorespacesafterend}
% % 不分段枚举环境
\newenvironment{enuminline}[1][(1)]{%
	\setenumtype{#1}\renewcommand{\item}{%
		\ifnum\value{enumi}=0\fi%
	\refstepcounter{enumi}\theenumi { }}}%
{\ignorespacesafterend}


% 选择题
% \begin{exchoice}(4)
% 	() 选项
% \end{exchoice}
\RequirePackage{tasks}
\settasks{label-format=\mdseries\textcolor{main},label={(\arabic*)},label-width=1.5em}
\NewTasksEnvironment[label={\upshape(\Alph*)},label-width=1.5em]{exchoice}[()]
% TIKZ 设置
\usetikzlibrary{
	quotes,
	shapes.arrows,
	arrows.meta,
	positioning,
	shapes.geometric,
	overlay-beamer-styles,
	patterns,
	calc,
	angles,
	decorations.pathreplacing,
	backgrounds % 背景边框
}
\tikzset{
	background rectangle/.style={semithick,draw=fourth,fill=white,rounded corners},
  % arrow
	cstra/.style      ={-Stealth},        % right arrow
	cstla/.style      ={Stealth-},        % left arrow
	cstlra/.style     ={Stealth-Stealth}, % left-right arrow
	cstwra/.style     ={-Straight Barb},  % wide ra
	cstwla/.style     ={Straight Barb-},
	cstwlra/.style    ={Straight Barb-Straight Barb},
	cstnra/.style      ={-Latex, line width=0.1cm},
	cstmra/.style      ={-Latex, line width=0.05cm},
	cstmlra/.style     ={Latex-Latex, line width=0.05cm},
	cstaxis/.style        ={-Stealth, thick}, %坐标轴
  % curve
	cstcurve/.style       ={very thick}, %一般曲线
	cstdash/.style        ={thick, dash pattern= on 0.2cm off 0.05cm}, %虚线
  % dot
	cstdot/.style         ={radius=.07,fourth}, %实心点
	cstdote/.style        ={radius=.06, fill=white}, %空心点
  % fill
	cstfill/.style       ={fill=black!10},
	cstfille/.style      ={pattern=north east lines, pattern color=black},
	cstfill1/.style       ={fill=main!20},
	cstfille1/.style      ={pattern=north east lines, pattern color=main},
	cstfill2/.style        ={fill=second!20},
	cstfille2/.style       ={pattern=north east lines, pattern color=second},
	cstfill3/.style        ={fill=third!20},
	cstfille3/.style       ={pattern=north east lines, pattern color=third},
	cstfill4/.style        ={fill=fourth!20},
	cstfille4/.style       ={pattern=north east lines, pattern color=fourth},
	cstfill5/.style        ={fill=fifth!20},
	cstfille5/.style       ={pattern=north east lines, pattern color=fifth},
  % node
	cstnode/.style        ={fill=white,draw=black,text=black,rounded corners=0.2cm,line width=1pt},
	cstnode1/.style       ={fill=main!15,draw=main!80,text=black,rounded corners=0.2cm,line width=1pt},
	cstnode2/.style       ={fill=second!15,draw=second!80,text=black,rounded corners=0.2cm,line width=1pt},
	cstnode3/.style       ={fill=third!15,draw=third!80,text=black,rounded corners=0.2cm,line width=1pt},
	cstnode4/.style       ={fill=fourth!15,draw=fourth!80,text=black,rounded corners=0.2cm,line width=1pt},
	cstnode5/.style       ={fill=fifth!15,draw=fifth!80,text=black,rounded corners=0.2cm,line width=1pt}
}

% 平行四边形
\newcommand\parallelogram{\mathord{\text{%
	\tikz[baseline] \draw (0em,.1ex)-- ++(.9em,0ex)--++(.2em,1.2ex)--++(-.9em,0ex)--cycle;%
}}}
% 初等变换
\newcommand\simr{\mathrel{\overset{r}{\sim}}}
\newcommand\simc{\mathrel{\overset{c}{\sim}}}
\newcommand{\wsim}[3][3]{
  \mathrel{\overset{\nsmath{#2}}{\underset{\nsmath{#3}}{\scalebox{#1}[1]{$\sim$}}}}
}
\newcommand{\nsmath}[1]{\text{\normalsize{$#1$}}}% 放大初等变换文字
% 单位阵
\newcommand{\Emat}{\begin{matrix}1&&\\&\ddots&\\&&1\end{matrix}}
% 分块, 参数为结束点 '行+1', '列+1'
\newcommand{\augdash}[2]{\CodeAfter
\tikz \draw[cstdash,second] (1-|#2) -- (#1-|#2);}
% 反向省略号
\newcommand{\udots}{\mathinner{\mskip1mu\raise1pt\vbox{\kern7pt\hbox{.}}\mskip2mu\raise4pt\hbox{.}\mskip2mu\raise7pt\hbox{.}\mskip1mu}}
% 带圈数字
\newcommand{\circleno}[1]{\tikz[baseline=-1.1mm]\draw[main] (0,0) circle (.2);\hspace{-3mm}\textcolor{blue}{#1}}
% 线性方程组, \laeq[lclcclcl]{ ... }
\newcommand{\laeq}[2][l]{
	\setlength{\arraycolsep}{0pt}
	\left\{\begin{array}{#1}
	#2\end{array}\right.}
\newenvironment{laeqn}{
	\setcounter{equation}{0}
	\begin{numcases}{}}{
	\end{numcases}}
\newcommand{\vvdots}{\multicolumn{1}{c}{\vdots}}
% 缩小标题
\newcommand\resizet[1]{\resizebox{!}{#1\baselineskip}}
% 填空题/选择题/判断题
\NewDocumentCommand\fillblankframe{O{3em} O{0ex} m}{\uline{\makebox[#1]{\raisebox{#2}{\onslide<+->{\alertn{#3}}}}}}
\newcommand\fillbraceframe[1]{{\text{\upshape(\nolinebreak\hspace{0.5em minus 0.5em}\onslide<+->{\alertn{#1}}\hspace{0.5em minus 0.5em}\nolinebreak)}}}
\newcommand{\trueex}{$\checkmark$}
\newcommand{\falseex}{$\times$}

% \newcommand{\ansno}[1]{\par\noindent\emph{\thechapter.#1}}
\newcommand{\pp}[2]{\frac{\partial #1}{\partial #2}}
\newcommand{\dpp}[2]{\dfrac{\partial #1}{\partial #2}}

% \newcommand{\nounen}[2]{{\color{main}\kaishu\cnen{#1}{#2}}\index{{#1}}}
% \newcommand{\nounsen}[3]{{\color{main}\kaishu\cnen{#1}{#2}}\index{{#3}}}

% arrows
\newcommand\ra{\rightarrow}
\newcommand\lra{\longrightarrow}
\newcommand\la{\leftarrow}
\newcommand\lla{\longleftarrow}
\newcommand\sqra{\rightsquigarrow}
\newcommand\sqlra{\leftrightsquigarrow}
\newcommand\swap{\leftrightarrow}
\newcommand\inj{\hookrightarrow}
\newcommand\linj{\hookleftarrow}
\newcommand\surj{\twoheadrightarrow}
\newcommand\simto{\stackrel{\sim}{\longrightarrow}}
\newcommand\sto[1]{\stackrel{#1}{\longrightarrow}}
\newcommand\lsto[1]{\stackrel{#1}{\longleftarrow}}
\newcommand\xto{\xlongrightarrow}
\newcommand\xeq{\xlongequal}
\newcommand\luobida{\xeq{\text{洛必达}}}
\newcommand\djwqx{\xeq{\text{等价无穷小}}}
\newcommand\eqprob{\stackrel{\mathrm{P}}{=}}
\newcommand\lto{\longmapsto}
\renewcommand\vec[1]{\overrightarrow{#1}}

% decorations
\newcommand\wh{\widehat}
\newcommand\wt{\widetilde}
\newcommand\ov{\overline}
\newcommand\ul{\underline}
\newlength{\larc} % 弧
\NewDocumentCommand\warc{o m}{%
	\IfNoValueTF {#1}%
	{%
		\settowidth{\larc}{$#2$}%
		\stackrel{\rotatebox{-90}{\ensuremath{\left(\rule{0ex}{0.7\larc}\right.}}}{#2}%
	}%
	{%
		\stackrel{\rotatebox{-90}{\ensuremath{\left(\rule{0ex}{#1}\right.}}}{#2}%
	}%
}

% 括号
\newcommand\midcolon{\mathrel{:}}
\newcommand\bigmid{\bigm\vert}
\newcommand\Bigmid{\Bigm\vert}
\newcommand\biggmid{\biggm\vert}
\newcommand\Biggmid{\Biggm\vert}
\newcommand\abs[1]{\lvert#1\rvert}
\newcommand\bigabs[1]{\bigl\vert#1\bigr\vert}
\newcommand\Bigabs[1]{\Bigl\vert#1\Bigr\vert}
\newcommand\biggabs[1]{\biggl\vert#1\biggr\vert}
\newcommand\Biggabs[1]{\Biggl\vert#1\Biggr\vert}
% \newcommand\set[1]{\{#1\}}
\newcommand\bigset[1]{\bigl\{#1\bigr\}}
\newcommand\Bigset[1]{\Bigl\{#1\Bigr\}}
\newcommand\biggset[1]{\biggl\{#1\biggr\}}
\newcommand\Biggset[1]{\Biggl\{#1\Biggr\}}
% \newcommand\setm[2]{\{#1\mid #2\}}
\newcommand\bigsetm[2]{\bigl\{#1\bigmid #2\bigr\}}
\newcommand\Bigsetm[2]{\Bigl\{#1\Bigmid #2\Bigr\}}
\newcommand\biggsetm[2]{\biggl\{#1\biggmid #2\biggr\}}
\newcommand\Biggsetm[2]{\Biggl\{#1\Biggmid #2\Biggr\}}
\newcommand\pair[1]{\langle{#1}\rangle}
\newcommand\norm[1]{\!\parallel\!{#1}\!\parallel\!}
\newcommand\dbb[1]{\llbracket#1 \rrbracket}
\newcommand\floor[1]{\lfloor#1\rfloor}
% symbols
\renewcommand\le{\leqslant}
\renewcommand\ge{\geqslant}
\newcommand\vare{\varepsilon}
\newcommand\varp{\varphi}
\newcommand\liml{\lim\limits}
\newcommand\suml{\sum\limits}
\newcommand\ilim{\varinjlim\limits}
\newcommand\plim{\varprojlim\limits}
\newcommand\half{\frac{1}{2}}
\newcommand\mmid{\parallel}
\font\cyr=wncyr10\newcommand\Sha{\hbox{\cyr X}}
\newcommand\Uc{\stackrel{\circ}{U}\!\!}
\newcommand\hil[3]{\left(\frac{{#1},{#2}}{#3}\right)}
\newcommand\leg[2]{\Bigl(\frac{{#1}}{#2}\Bigr)}
\newcommand\aleg[2]{\Bigl[\frac{{#1}}{#2}\Bigr]}
\newcommand\stsc[2]{\genfrac{}{}{0pt}{}{#1}{#2}}

% categories
\newcommand\cA{{\mathsf{A}}}
\newcommand\cb{{\mathsf{b}}}
\newcommand\cB{{\mathsf{B}}}
\newcommand\cC{{\mathsf{C}}}
\newcommand\cD{{\mathsf{D}}}
\newcommand\cM{{\mathsf{M}}}
\newcommand\cR{{\mathsf{R}}}
\newcommand\cP{{\mathsf{P}}}
\newcommand\cT{{\mathsf{T}}}
\newcommand\cX{{\mathsf{X}}}
\newcommand\cx{{\mathsf{x}}}
\newcommand\cAb{{\mathsf{Ab}}}
\newcommand\cBT{{\mathsf{BT}}}
\newcommand\cBun{{\mathsf{Bun}}}
\newcommand\cCharLoc{{\mathsf{CharLoc}}}
\newcommand\cCoh{{\mathsf{Coh}}}
\newcommand\cComm{{\mathsf{Comm}}}
\newcommand\cEt{{\mathsf{Et}}}
\newcommand\cFppf{{\mathsf{Fppf}}}
\newcommand\cFpqc{{\mathsf{Fpqc}}}
\newcommand\cFunc{{\mathsf{Func}}}
\newcommand\cGroups{{\mathsf{Groups}}}
\newcommand\cGrpd{{\{\mathsf{Grpd}\}}}
\newcommand\cHo{{\mathsf{Ho}}}
\newcommand\cIso{{\mathsf{Iso}}}
\newcommand\cLoc{{\mathsf{Loc}}}
\newcommand\cMod{{\mathsf{Mod}}}
\newcommand\cModFil{{\mathsf{ModFil}}}
\newcommand\cNilp{{\mathsf{Nilp}}}
\newcommand\cPerf{{\mathsf{Perf}}}
\newcommand\cPN{{\mathsf{PN}}}
\newcommand\cRep{{\mathsf{Rep}}}
\newcommand\cRings{{\mathsf{Rings}}}
\newcommand\cSets{{\mathsf{Sets}}}
\newcommand\cStack{{\mathsf{Stack}}}
\newcommand\cSch{{\mathsf{Sch}}}
\newcommand\cTop{{\mathsf{Top}}}
\newcommand\cVect{{\mathsf{Vect}}}
\newcommand\cZar{{\mathsf{Zar}}}
\newcommand\cphimod{{\varphi\txt{-}\mathsf{Mod}}}
\newcommand\cphimodfil{{\varphi\txt{-}\mathsf{ModFil}}}

% font
\newcommand\rma{{\mathrm{a}}}
\newcommand\rmb{{\mathrm{b}}}
\newcommand\rmc{{\mathrm{c}}}
\newcommand\rmd{{\mathrm{d}}}
\newcommand\rme{{\mathrm{e}}}
\newcommand\rmf{{\mathrm{f}}}
\newcommand\rmg{{\mathrm{g}}}
\newcommand\rmh{{\mathrm{h}}}
\newcommand\rmi{{\mathrm{i}}}
\newcommand\rmj{{\mathrm{j}}}
\newcommand\rmk{{\mathrm{k}}}
\newcommand\rml{{\mathrm{l}}}
\newcommand\rmm{{\mathrm{m}}}
\newcommand\rmn{{\mathrm{n}}}
\newcommand\rmo{{\mathrm{o}}}
\newcommand\rmp{{\mathrm{p}}}
\newcommand\rmq{{\mathrm{q}}}
\newcommand\rmr{{\mathrm{r}}}
\newcommand\rms{{\mathrm{s}}}
\newcommand\rmt{{\mathrm{t}}}
\newcommand\rmu{{\mathrm{u}}}
\newcommand\rmv{{\mathrm{v}}}
\newcommand\rmw{{\mathrm{w}}}
\newcommand\rmx{{\mathrm{x}}}
\newcommand\rmy{{\mathrm{y}}}
\newcommand\rmz{{\mathrm{z}}}
\newcommand\rmA{{\mathrm{A}}}
\newcommand\rmB{{\mathrm{B}}}
\newcommand\rmC{{\mathrm{C}}}
\newcommand\rmD{{\mathrm{D}}}
\newcommand\rmE{{\mathrm{E}}}
\newcommand\rmF{{\mathrm{F}}}
\newcommand\rmG{{\mathrm{G}}}
\newcommand\rmH{{\mathrm{H}}}
\newcommand\rmI{{\mathrm{I}}}
\newcommand\rmJ{{\mathrm{J}}}
\newcommand\rmK{{\mathrm{K}}}
\newcommand\rmL{{\mathrm{L}}}
\newcommand\rmM{{\mathrm{M}}}
\newcommand\rmN{{\mathrm{N}}}
\newcommand\rmO{{\mathrm{O}}}
\newcommand\rmP{{\mathrm{P}}}
\newcommand\rmQ{{\mathrm{Q}}}
\newcommand\rmR{{\mathrm{R}}}
\newcommand\rmS{{\mathrm{S}}}
\newcommand\rmT{{\mathrm{T}}}
\newcommand\rmU{{\mathrm{U}}}
\newcommand\rmV{{\mathrm{V}}}
\newcommand\rmW{{\mathrm{W}}}
\newcommand\rmX{{\mathrm{X}}}
\newcommand\rmY{{\mathrm{Y}}}
\newcommand\rmZ{{\mathrm{Z}}}
\newcommand\bfa{{\mathbf{a}}}
\newcommand\bfb{{\mathbf{b}}}
\newcommand\bfc{{\mathbf{c}}}
\newcommand\bfd{{\mathbf{d}}}
\newcommand\bfe{{\mathbf{e}}}
\newcommand\bff{{\mathbf{f}}}
\newcommand\bfg{{\mathbf{g}}}
\newcommand\bfh{{\mathbf{h}}}
\newcommand\bfi{{\mathbf{i}}}
\newcommand\bfj{{\mathbf{j}}}
\newcommand\bfk{{\mathbf{k}}}
\newcommand\bfl{{\mathbf{l}}}
\newcommand\bfm{{\mathbf{m}}}
\newcommand\bfn{{\mathbf{n}}}
\newcommand\bfo{{\mathbf{o}}}
\newcommand\bfp{{\mathbf{p}}}
\newcommand\bfq{{\mathbf{q}}}
\newcommand\bfr{{\mathbf{r}}}
\newcommand\bfs{{\mathbf{s}}}
\newcommand\bft{{\mathbf{t}}}
\newcommand\bfu{{\mathbf{u}}}
\newcommand\bfv{{\mathbf{v}}}
\newcommand\bfw{{\mathbf{w}}}
\newcommand\bfx{{\mathbf{x}}}
\newcommand\bfy{{\mathbf{y}}}
\newcommand\bfz{{\mathbf{z}}}
\newcommand\bfA{{\mathbf{A}}}
\newcommand\bfB{{\mathbf{B}}}
\newcommand\bfC{{\mathbf{C}}}
\newcommand\bfD{{\mathbf{D}}}
\newcommand\bfE{{\mathbf{E}}}
\newcommand\bfF{{\mathbf{F}}}
\newcommand\bfG{{\mathbf{G}}}
\newcommand\bfH{{\mathbf{H}}}
\newcommand\bfI{{\mathbf{I}}}
\newcommand\bfJ{{\mathbf{J}}}
\newcommand\bfK{{\mathbf{K}}}
\newcommand\bfL{{\mathbf{L}}}
\newcommand\bfM{{\mathbf{M}}}
\newcommand\bfN{{\mathbf{N}}}
\newcommand\bfO{{\mathbf{O}}}
\newcommand\bfP{{\mathbf{P}}}
\newcommand\bfQ{{\mathbf{Q}}}
\newcommand\bfR{{\mathbf{R}}}
\newcommand\bfS{{\mathbf{S}}}
\newcommand\bfT{{\mathbf{T}}}
\newcommand\bfU{{\mathbf{U}}}
\newcommand\bfV{{\mathbf{V}}}
\newcommand\bfW{{\mathbf{W}}}
\newcommand\bfX{{\mathbf{X}}}
\newcommand\bfY{{\mathbf{Y}}}
\newcommand\bfZ{{\mathbf{Z}}}
\newcommand\BA{{\mathbb{A}}}
\newcommand\BB{{\mathbb{B}}}
\newcommand\BC{{\mathbb{C}}}
\newcommand\BD{{\mathbb{D}}}
\newcommand\BE{{\mathbb{E}}}
\newcommand\BF{{\mathbb{F}}}
\newcommand\BG{{\mathbb{G}}}
\newcommand\BH{{\mathbb{H}}}
\newcommand\BI{{\mathbb{I}}}
\newcommand\BJ{{\mathbb{J}}}
\newcommand\BK{{\mathbb{K}}}
\newcommand\BL{{\mathbb{L}}}
\newcommand\BM{{\mathbb{M}}}
\newcommand\BN{{\mathbb{N}}}
\newcommand\BO{{\mathbb{O}}}
\newcommand\BP{{\mathbb{P}}}
\newcommand\BQ{{\mathbb{Q}}}
\newcommand\BR{{\mathbb{R}}}
\newcommand\BS{{\mathbb{S}}}
\newcommand\BT{{\mathbb{T}}}
\newcommand\BU{{\mathbb{U}}}
\newcommand\BV{{\mathbb{V}}}
\newcommand\BW{{\mathbb{W}}}
\newcommand\BX{{\mathbb{X}}}
\newcommand\BY{{\mathbb{Y}}}
\newcommand\BZ{{\mathbb{Z}}}
\newcommand\CA{{\mathcal{A}}}
\newcommand\CB{{\mathcal{B}}}
\newcommand\CC{{\mathcal{C}}}
\providecommand\CD{{\mathcal{D}}}
\newcommand\CE{{\mathcal{E}}}
\newcommand\CF{{\mathcal{F}}}
\newcommand\CG{{\mathcal{G}}}
\newcommand\CH{{\mathcal{H}}}
\newcommand\CI{{\mathcal{I}}}
\newcommand\CJ{{\mathcal{J}}}
\newcommand\CK{{\mathcal{K}}}
\newcommand\CL{{\mathcal{L}}}
\newcommand\CM{{\mathcal{M}}}
\newcommand\CN{{\mathcal{N}}}
\newcommand\CO{{\mathcal{O}}}
\newcommand\CP{{\mathcal{P}}}
\newcommand\CQ{{\mathcal{Q}}}
\newcommand\CR{{\mathcal{R}}}
\newcommand\CS{{\mathcal{S}}}
\newcommand\CT{{\mathcal{T}}}
\newcommand\CU{{\mathcal{U}}}
\newcommand\CV{{\mathcal{V}}}
\newcommand\CW{{\mathcal{W}}}
\newcommand\CX{{\mathcal{X}}}
\newcommand\CY{{\mathcal{Y}}}
\newcommand\CZ{{\mathcal{Z}}}
\newcommand\RA{{\mathrm{A}}}
\newcommand\RB{{\mathrm{B}}}
\newcommand\RC{{\mathrm{C}}}
\newcommand\RD{{\mathrm{D}}}
\newcommand\RE{{\mathrm{E}}}
\newcommand\RF{{\mathrm{F}}}
\newcommand\RG{{\mathrm{G}}}
\newcommand\RH{{\mathrm{H}}}
\newcommand\RI{{\mathrm{I}}}
\newcommand\RJ{{\mathrm{J}}}
\newcommand\RK{{\mathrm{K}}}
\newcommand\RL{{\mathrm{L}}}
\newcommand\RM{{\mathrm{M}}}
\let\RN\relax\newcommand\RN{{\mathrm{N}}}
\newcommand\RO{{\mathrm{O}}}
\newcommand\RP{{\mathrm{P}}}
\newcommand\RQ{{\mathrm{Q}}}
\newcommand\RR{{\mathrm{R}}}
\newcommand\RS{{\mathrm{S}}}
\newcommand\RT{{\mathrm{T}}}
\newcommand\RU{{\mathrm{U}}}
\newcommand\RV{{\mathrm{V}}}
\newcommand\RW{{\mathrm{W}}}
\newcommand\RX{{\mathrm{X}}}
\newcommand\RY{{\mathrm{Y}}}
\newcommand\RZ{{\mathrm{Z}}}
\newcommand\msa{\mathscr{A}}
\newcommand\msb{\mathscr{B}}
\newcommand\msc{\mathscr{C}}
\newcommand\msd{\mathscr{D}}
\newcommand\mse{\mathscr{E}}
\newcommand\msf{\mathscr{F}}
\newcommand\msg{\mathscr{G}}
\newcommand\msh{\mathscr{H}}
\newcommand\msi{\mathscr{I}}
\newcommand\msj{\mathscr{J}}
\newcommand\msk{\mathscr{K}}
\newcommand\msl{\mathscr{L}}
\newcommand\msm{\mathscr{M}}
\newcommand\msn{\mathscr{N}}
\newcommand\mso{\mathscr{O}}
\newcommand\msp{\mathscr{P}}
\newcommand\msq{\mathscr{Q}}
\newcommand\msr{\mathscr{R}}
\newcommand\mss{\mathscr{S}}
\newcommand\mst{\mathscr{T}}
\newcommand\msu{\mathscr{U}}
\newcommand\msv{\mathscr{V}}
\newcommand\msw{\mathscr{W}}
\newcommand\msx{\mathscr{X}}
\newcommand\msy{\mathscr{Y}}
\newcommand\msz{\mathscr{Z}}
\newcommand\fa{{\mathfrak{a}}}
\newcommand\fb{{\mathfrak{b}}}
\newcommand\fc{{\mathfrak{c}}}
\newcommand\fd{{\mathfrak{d}}}
\newcommand\fe{{\mathfrak{e}}}
\newcommand\ff{{\mathfrak{f}}}
\newcommand\fg{{\mathfrak{g}}}
\newcommand\fh{{\mathfrak{h}}}
\newcommand\fii{{\mathfrak{i}}} % be careful about \fii
\newcommand\fj{{\mathfrak{j}}}
\newcommand\fk{{\mathfrak{k}}}
\newcommand\fl{{\mathfrak{l}}}
\newcommand\fm{{\mathfrak{m}}}
\newcommand\fn{{\mathfrak{n}}}
\newcommand\fo{{\mathfrak{o}}}
\newcommand\fp{{\mathfrak{p}}}
\newcommand\fq{{\mathfrak{q}}}
\newcommand\fr{{\mathfrak{r}}}
\newcommand\fs{{\mathfrak{s}}}
\newcommand\ft{{\mathfrak{t}}}
\newcommand\fu{{\mathfrak{u}}}
\newcommand\fv{{\mathfrak{v}}}
\newcommand\fw{{\mathfrak{w}}}
\newcommand\fx{{\mathfrak{x}}}
\newcommand\fy{{\mathfrak{y}}}
\newcommand\fz{{\mathfrak{z}}}
\newcommand\fA{{\mathfrak{A}}}
\newcommand\fB{{\mathfrak{B}}}
\newcommand\fC{{\mathfrak{C}}}
\newcommand\fD{{\mathfrak{D}}}
\newcommand\fE{{\mathfrak{E}}}
\newcommand\fF{{\mathfrak{F}}}
\newcommand\fG{{\mathfrak{G}}}
\newcommand\fH{{\mathfrak{H}}}
\newcommand\fI{{\mathfrak{I}}}
\newcommand\fJ{{\mathfrak{J}}}
\newcommand\fK{{\mathfrak{K}}}
\newcommand\fL{{\mathfrak{L}}}
\newcommand\fM{{\mathfrak{M}}}
\newcommand\fN{{\mathfrak{N}}}
\newcommand\fO{{\mathfrak{O}}}
\newcommand\fP{{\mathfrak{P}}}
\newcommand\fQ{{\mathfrak{Q}}}
\newcommand\fR{{\mathfrak{R}}}
\newcommand\fS{{\mathfrak{S}}}
\newcommand\fT{{\mathfrak{T}}}
\newcommand\fU{{\mathfrak{U}}}
\newcommand\fV{{\mathfrak{V}}}
\newcommand\fW{{\mathfrak{W}}}
\newcommand\fX{{\mathfrak{X}}}
\newcommand\fY{{\mathfrak{Y}}}
\newcommand\fZ{{\mathfrak{Z}}}

\renewcommand\bfa{{\boldsymbol{a}}}
\renewcommand\bfb{{\boldsymbol{b}}}
\renewcommand\bfc{{\boldsymbol{c}}}
\renewcommand\bfd{{\boldsymbol{d}}}
\renewcommand\bfe{{\boldsymbol{e}}}
\renewcommand\bff{{\boldsymbol{f}}}
\renewcommand\bfg{{\boldsymbol{g}}}
\renewcommand\bfh{{\boldsymbol{h}}}
\renewcommand\bfi{{\boldsymbol{i}}}
\renewcommand\bfj{{\boldsymbol{j}}}
\renewcommand\bfk{{\boldsymbol{k}}}
\renewcommand\bfl{{\boldsymbol{l}}}
\renewcommand\bfm{{\boldsymbol{m}}}
\renewcommand\bfn{{\boldsymbol{n}}}
\renewcommand\bfo{{\boldsymbol{o}}}
\renewcommand\bfp{{\boldsymbol{p}}}
\renewcommand\bfq{{\boldsymbol{q}}}
\renewcommand\bfr{{\boldsymbol{r}}}
\renewcommand\bfs{{\boldsymbol{s}}}
\renewcommand\bft{{\boldsymbol{t}}}
\renewcommand\bfu{{\boldsymbol{u}}}
\renewcommand\bfv{{\boldsymbol{v}}}
\renewcommand\bfw{{\boldsymbol{w}}}
\renewcommand\bfx{{\boldsymbol{x}}}
\renewcommand\bfy{{\boldsymbol{y}}}
\renewcommand\bfz{{\boldsymbol{z}}}
\renewcommand\bfA{{\boldsymbol{A}}}
\renewcommand\bfB{{\boldsymbol{B}}}
\renewcommand\bfC{{\boldsymbol{C}}}
\renewcommand\bfD{{\boldsymbol{D}}}
\renewcommand\bfE{{\boldsymbol{E}}}
\renewcommand\bfF{{\boldsymbol{F}}}
\renewcommand\bfG{{\boldsymbol{G}}}
\renewcommand\bfH{{\boldsymbol{H}}}
\renewcommand\bfI{{\boldsymbol{I}}}
\renewcommand\bfJ{{\boldsymbol{J}}}
\renewcommand\bfK{{\boldsymbol{K}}}
\renewcommand\bfL{{\boldsymbol{L}}}
\renewcommand\bfM{{\boldsymbol{M}}}
\renewcommand\bfN{{\boldsymbol{N}}}
\renewcommand\bfO{{\boldsymbol{O}}}
\renewcommand\bfP{{\boldsymbol{P}}}
\renewcommand\bfQ{{\boldsymbol{Q}}}
\renewcommand\bfR{{\boldsymbol{R}}}
\renewcommand\bfS{{\boldsymbol{S}}}
\renewcommand\bfT{{\boldsymbol{T}}}
\renewcommand\bfU{{\boldsymbol{U}}}
\renewcommand\bfV{{\boldsymbol{V}}}
\renewcommand\bfW{{\boldsymbol{W}}}
\renewcommand\bfX{{\boldsymbol{X}}}
\renewcommand\bfY{{\boldsymbol{Y}}}
\renewcommand\bfZ{{\boldsymbol{Z}}}

\newcommand\bma{{\boldsymbol{\alpha}}}
\newcommand\bmb{{\boldsymbol{\beta}}}
\newcommand\bmg{{\boldsymbol{\gamma}}}
\newcommand\bmd{{\boldsymbol{\delta}}}
\newcommand\bme{{\boldsymbol{\epsilon}}}
\newcommand\bmph{{\boldsymbol{\phi}}}
\newcommand\bmp{{\boldsymbol{\psi}}}
\newcommand\bmpi{{\boldsymbol{\pi}}}
\newcommand\bmx{{\boldsymbol{\xi}}}
\newcommand\bmet{{\boldsymbol{\eta}}}
\newcommand\bmL{{\boldsymbol{\Lambda}}}
\newcommand\bmS{{\boldsymbol{\Sigma}}}

% a
\newcommand\ab{{\mathrm{ab}}}
\newcommand\ad{{\mathrm{ad}}}
\newcommand\Ad{{\mathrm{Ad}}}
\newcommand\adele{ad\'{e}le}
\newcommand\Adele{Ad\'{e}le}
\newcommand\adeles{ad\'{e}les}
\newcommand\adelic{ad\'{e}lic}
\newcommand\AJ{{\mathrm{AJ}}}
\newcommand\alb{{\mathrm{alb}}}
\newcommand\Alb{{\mathrm{Alb}}}
\newcommand\alg{{\mathrm{alg}}}
\newcommand\an{{\mathrm{an}}}
\newcommand\ann{{\mathrm{ann}}}
\newcommand\Ann{{\mathrm{Ann}}}
\DeclareMathOperator\Arcsin{Arcsin}
\DeclareMathOperator\Arccos{Arccos}
\DeclareMathOperator\Arctan{Arctan}
\DeclareMathOperator\arsh{arsh}
\DeclareMathOperator\Arsh{Arsh}
\DeclareMathOperator\arch{arch}
\DeclareMathOperator\Arch{Arch}
\DeclareMathOperator\arth{arth}
\DeclareMathOperator\Arth{Arth}
\DeclareMathOperator\arccot{arccot}
\DeclareMathOperator\Arg{Arg}
\newcommand\arith{{\mathrm{arith}}}
\newcommand\Art{{\mathrm{Art}}}
\newcommand\AS{{\mathrm{AS}}}
\newcommand\Ass{{\mathrm{Ass}}}
\newcommand\Aut{{\mathrm{Aut}}}
% b
\newcommand\Bun{{\mathrm{Bun}}}
\newcommand\Br{{\mathrm{Br}}}
\newcommand\bs{\backslash}
\newcommand\BWt{{\mathrm{BW}}}
% c
\newcommand\can{{\mathrm{can}}}
\newcommand\cc{{\mathrm{cc}}}
\newcommand\cd{{\mathrm{cd}}}
\DeclareMathOperator\ch{ch}
\newcommand\Ch{{\mathrm{Ch}}}
\let\char\relax
\DeclareMathOperator\char{char}
\DeclareMathOperator\Char{char}
\newcommand\Chow{{\mathrm{CH}}}
\newcommand\circB{{\stackrel{\circ}{B}}}
\newcommand\cl{{\mathrm{cl}}}
\newcommand\Cl{{\mathrm{Cl}}}
\newcommand\cm{{\mathrm{cm}}}
\newcommand\cod{{\mathrm{cod}}}
\DeclareMathOperator\coker{coker}
\DeclareMathOperator\Coker{Coker}
\DeclareMathOperator\cond{cond}
\DeclareMathOperator\codim{codim}
\newcommand\cont{{\mathrm{cont}}}
\newcommand\Conv{{\mathrm{Conv}}}
\newcommand\corr{{\mathrm{corr}}}
\newcommand\Corr{{\mathrm{Corr}}}
\DeclareMathOperator\coim{coim}
\DeclareMathOperator\coIm{coIm}
\DeclareMathOperator\corank{corank}
\DeclareMathOperator\covol{covol}
\newcommand\cris{{\mathrm{cris}}}
\newcommand\Cris{{\mathrm{Cris}}}
\newcommand\CRIS{{\mathrm{CRIS}}}
\newcommand\crit{{\mathrm{crit}}}
\newcommand\crys{{\mathrm{crys}}}
\newcommand\cusp{{\mathrm{cusp}}}
\newcommand\CWt{{\mathrm{CW}}}
\newcommand\cyc{{\mathrm{cyc}}}
% d
\newcommand\Def{{\mathrm{Def}}}
\newcommand\diag{{\mathrm{diag}}}
\newcommand\diff{\,\mathrm{d}}
\newcommand\disc{{\mathrm{disc}}}
\newcommand\dist{{\mathrm{dist}}}
\renewcommand\div{{\mathrm{div}}}
\newcommand\Div{{\mathrm{Div}}}
\newcommand\dR{{\mathrm{dR}}}
\newcommand\Drin{{\mathrm{Drin}}}
% e
\newcommand\End{{\mathrm{End}}}
\newcommand\ess{{\mathrm{ess}}}
\newcommand\et{{\text{\'{e}t}}}
\newcommand\etale{{\'{e}tale}}
\newcommand\Etale{{\'{E}tale}}
\newcommand\Ext{\mathrm{Ext}}
\newcommand\CExt{\mathcal{E}\mathrm{xt}}
% f
\newcommand\Fil{{\mathrm{Fil}}}
\newcommand\Fix{{\mathrm{Fix}}}
\newcommand\fppf{{\mathrm{fppf}}}
\newcommand\Fr{{\mathrm{Fr}}}
\newcommand\Frac{{\mathrm{Frac}}}
\newcommand\Frob{{\mathrm{Frob}}}
% g
\newcommand\Ga{\mathbb{G}_a}
\newcommand\Gm{\mathbb{G}_m}
\newcommand\hGa{\widehat{\mathbb{G}}_{a}}
\newcommand\hGm{\widehat{\mathbb{G}}_{m}}
\newcommand\Gal{{\mathrm{Gal}}}
\newcommand\gl{{\mathrm{gl}}}
\newcommand\GL{{\mathrm{GL}}}
\newcommand\GO{{\mathrm{GO}}}
\newcommand\geom{{\mathrm{geom}}}
\newcommand\Gr{{\mathrm{Gr}}}
\newcommand\gr{{\mathrm{gr}}}
\newcommand\GSO{{\mathrm{GSO}}}
\newcommand\GSp{{\mathrm{GSp}}}
\newcommand\GSpin{{\mathrm{GSpin}}}
\newcommand\GU{{\mathrm{GU}}}
% h
\newcommand\hg{{\mathrm{hg}}}
\newcommand\Hk{{\mathrm{Hk}}}
\newcommand\HN{{\mathrm{HN}}}
\newcommand\Hom{{\mathrm{Hom}}}
\newcommand\CHom{\mathcal{H}\mathrm{om}}
% i
\newcommand\id{{\mathrm{id}}}
\newcommand\Id{{\mathrm{Id}}}
\newcommand\idele{id\'{e}le}
\newcommand\Idele{Id\'{e}le}
\newcommand\ideles{id\'{e}les}
\let\Im\relax
\DeclareMathOperator\Im{Im}
\DeclareMathOperator\im{im}
\newcommand\Ind{{\mathrm{Ind}}}
\newcommand\cInd{{\mathrm{c}\textrm{-}\mathrm{Ind}}}
\newcommand\ind{{\mathrm{ind}}}
\newcommand\Int{{\mathrm{Int}}}
\newcommand\inv{{\mathrm{inv}}}
\newcommand\Isom{{\mathrm{Isom}}}
% j
\newcommand\Jac{{\mathrm{Jac}}}
\newcommand\JL{{\mathrm{JL}}}
% k
\newcommand\Katz{\mathrm{Katz}}
\DeclareMathOperator\Ker{Ker}
\newcommand\KS{{\mathrm{KS}}}
\newcommand\Kl{{\mathrm{Kl}}}
\newcommand\CKl{{\mathcal{K}\mathrm{l}}}
% l
\newcommand\lcm{\mathrm{lcm}}
\newcommand\length{\mathrm{length}}
\DeclareMathOperator\Li{Li}
\DeclareMathOperator\Ln{Ln}
\newcommand\Lie{{\mathrm{Lie}}}
\newcommand\lt{\mathrm{lt}}
\newcommand\LT{\mathcal{LT}}
% m
\newcommand\mex{{\mathrm{mex}}}
\newcommand\MW{{\mathrm{MW}}}
\renewcommand\mod{\, \mathrm{mod}\, }
\newcommand\mom{{\mathrm{mom}}}
\newcommand\Mor{{\mathrm{Mor}}}
\newcommand\Morp{{\mathrm{Morp}\,}}
% n
\newcommand\new{{\mathrm{new}}}
\newcommand\Newt{{\mathrm{Newt}}}
\newcommand\nd{{\mathrm{nd}}}
\newcommand\NP{{\mathrm{NP}}}
\newcommand\NS{{\mathrm{NS}}}
\newcommand\ns{{\mathrm{ns}}}
\newcommand\Nm{{\mathrm{Nm}}}
\newcommand\Nrd{{\mathrm{Nrd}}}
\newcommand\Neron{N\'{e}ron}
% o
\newcommand\Obj{{\mathrm{Obj}\,}}
\newcommand\odd{{\mathrm{odd}}}
\newcommand\old{{\mathrm{old}}}
\newcommand\op{{\mathrm{op}}}
\newcommand\Orb{{\mathrm{Orb}}}
\newcommand\ord{{\mathrm{ord}}}
% p
\newcommand\pd{{\mathrm{pd}}}
\newcommand\Pet{{\mathrm{Pet}}}
\newcommand\PGL{{\mathrm{PGL}}}
\newcommand\Pic{{\mathrm{Pic}}}
\newcommand\pr{{\mathrm{pr}}}
\newcommand\Proj{{\mathrm{Proj}}}
\newcommand\proet{\text{pro\'{e}t}}
\newcommand\Poincare{\text{Poincar\'{e}}}
\newcommand\Prd{{\mathrm{Prd}}}
\newcommand\prim{{\mathrm{prim}}}
% r
\newcommand\Rad{{\mathrm{Rad}}}
\newcommand\rad{{\mathrm{rad}}}
\DeclareMathOperator\rank{rank}
\let\Re\relax
\DeclareMathOperator\Re{Re}
\newcommand\rec{{\mathrm{rec}}}
\newcommand\red{{\mathrm{red}}}
\newcommand\reg{{\mathrm{reg}}}
\newcommand\res{{\mathrm{res}}}
\newcommand\Res{{\mathrm{Res}}}
\newcommand\rig{{\mathrm{rig}}}
\newcommand\Rig{{\mathrm{Rig}}}
\newcommand\rk{{\mathrm{rk}}}
\newcommand\Ros{{\mathrm{Ros}}}
\newcommand\rs{{\mathrm{rs}}}
% s
\newcommand\sd{{\mathrm{sd}}}
\newcommand\Sel{{\mathrm{Sel}}}
\newcommand\sep{{\mathrm{sep}}}
\DeclareMathOperator{\sgn}{sgn}
\newcommand\Sh{{\mathrm{Sh}}}
\DeclareMathOperator\sh{sh}
\newcommand\Sht{\mathrm{Sht}}
\newcommand\Sim{{\mathrm{Sim}}}
\DeclareMathOperator\sinc{sinc}
\newcommand\sign{{\mathrm{sign}}}
\newcommand\SK{{\mathrm{SK}}}
\newcommand\SL{{\mathrm{SL}}}
\newcommand\SO{{\mathrm{SO}}}
\newcommand\Sp{{\mathrm{Sp}}}
\newcommand\Spa{{\mathrm{Spa}}}
\newcommand\Span{{\mathrm{Span}}}
\DeclareMathOperator\Spec{Spec}
\newcommand\Spf{{\mathrm{Spf}}}
\newcommand\Spin{{\mathrm{Spin}}}
\newcommand\Spm{{\mathrm{Spm}}}
\newcommand\srs{{\mathrm{srs}}}
\newcommand\rss{{\mathrm{ss}}}
\newcommand\ST{{\mathrm{ST}}}
\newcommand\St{{\mathrm{St}}}
\newcommand\st{{\mathrm{st}}}
\newcommand\Stab{{\mathrm{Stab}}}
\newcommand\SU{{\mathrm{SU}}}
\newcommand\Sym{{\mathrm{Sym}}}
\newcommand\sub{{\mathrm{sub}}}
\newcommand\rsum{{\mathrm{sum}}}
\newcommand\supp{{\mathrm{supp}}}
\newcommand\Supp{{\mathrm{Supp}}}
\newcommand\Swan{{\mathrm{Sw}}}
% t
\newcommand\td{{\mathrm{td}}}
\let\tanh\relax
\DeclareMathOperator\tanh{th}
\newcommand\tor{{\mathrm{tor}}}
\newcommand\Tor{{\mathrm{Tor}}}
\newcommand\tors{{\mathrm{tors}}}
\newcommand\tr{{\mathrm{tr}\,}}
\newcommand\Tr{{\mathrm{Tr}}}
\newcommand\Trd{{\mathrm{Trd}}}
\newcommand\TSym{{\mathrm{TSym}}}
\newcommand\tw{{\mathrm{tw}}}
% u
\newcommand\uni{{\mathrm{uni}}}
\newcommand\univ{\mathrm{univ}}
\newcommand\ur{{\mathrm{ur}}}
\newcommand\USp{{\mathrm{USp}}}
% v
\newcommand\vQ{{\breve \BQ}}
\newcommand\vE{{\breve E}}
\newcommand\Ver{{\mathrm{Ver}}}
\newcommand\vF{{\breve F}}
\newcommand\vK{{\breve K}}
\newcommand\vol{{\mathrm{vol}}}
\newcommand\Vol{{\mathrm{Vol}}}
% w
\newcommand\wa{{\mathrm{wa}}}
% z
\newcommand\Zar{{\mathrm{Zar}}}

% 适应国标
\newcommand{\ii}{\mathrm{i}}
\newcommand{\jj}{\mathrm{j}}
\newcommand{\kk}{\mathrm{k}}
\newcommand{\ee}{\mathrm{e}}
\usepackage{fixdif,derivative} % 提供微分运算符
\letdif{\delt}{Delta} % 微分运算符 Δ
\usepackage[nolimits]{cmupint} % 提供正体 \int, \oint
\usepackage{textgreek}% 提供正体希腊字母
\newcommand{\cpi}{{\text{\textpi}}}
\newcommand{\dirac}{{\text{\textdelta}}}
% 缩写
\DeclareMathOperator\PV{P.V. }
\newcommand{\dint}{\displaystyle\int}
\newcommand{\doint}{\displaystyle\oint}
\newcommand{\intT}{\displaystyle\int_{-\frac T2}^{\frac T2}}
\newcommand{\intf}{\displaystyle\int_0^{+\infty}}
\newcommand{\intff}{\displaystyle\int_{-\infty}^{+\infty}}
\newcommand{\sumf}[1]{\displaystyle\sum_{n=#1}^{\infty}}
\newcommand{\sumff}{\displaystyle\sum_{n=-\infty}^{+\infty}}

\newcommand{\bigdel}{\vspace{-\bigskipamount}}
\newcommand{\meddel}{\vspace{-\medskipamount}}
\newcommand{\smalldel}{\vspace{-\smallskipamount}}

\newcommand\invitedby[2][hfut]{
	\def\insertoffice{#2}
	\def\inserttitlegraphic{../image/logo/#1.png}
	\def\insertinstitutegraphic{../image/logo/#1name.png}
}
\email{zhangshenxing@hfut.edu.cn}
\website{https://zhangshenxing.github.io}
\date{\today}
\author{张神星}
\institute{合肥工业大学}
\AtBeginSection{}
\showprefixfalse
\website{}
\newcommand\frameoutline{{\setnaviboxempty\begin{frame}{提纲}\tableofcontents\end{frame}}}
\AtEndDocument{{\setnaviboxempty\begin{frame}\begin{center}\includegraphics[height=15mm]{../image/xie.png}\hspace{8mm}\includegraphics[height=15mm]{../image/xie.png}\end{center}\end{frame}}}

\RequirePackage[T1]{fontenc}
\setCJKsansfont[ItalicFont={KaiTi},BoldFont={LXGW ZhenKai}]{Source Han Sans HW SC}
\newfontface\cmunrm{cmunrm.otf}\newcommand\cmu[1]{{\cmunrm{#1}}}

\title{抓石子游戏中的数学问题}
\invitedby[bit]{北京理工大学}
\date{2025年5月7日}

\DeclareMathOperator*{\SUB}{{\text{\mdseries{\upshape\textsc{sub}}}}}
\renewcommand\ul[1]{{\color{main}\underline{#1}}}
\newcommand{\emphov}[2]{\emph{$\overset{\text{\clap{#2}}}{\text{#1}}$}}
\usepackage{makecell}
\usepackage{tikz}
\usetikzlibrary{
  hobby,
  intersections,
  quotes,
  shapes.arrows,
  arrows.meta,
  bending,
  positioning,
  shapes.geometric,
  overlay-beamer-styles,
  calc,
  angles,
  decorations.markings,
  decorations.pathreplacing,
  backgrounds,
  chains
}

\begin{document}


\section{Nim游戏}

\begin{frame}{抓石子游戏}
  \begin{itemize}
    \item \alertn{A}lice和\emphn{B}ob 在玩一个游戏, 他们从地上抓起一把石子, 然后从Alice开始, 轮流从石子堆中取走石子.
    \item 每个人每次可以取走$1\sim3$个石子, 最终谁把最后一颗石子取走, 谁就获得了游戏的胜利.
    \onslide<+->
    \begin{center}
      \begin{tikzpicture}[scale=.25]
        \def\r{.7}
        \def\dx{1}
        \def\dy{1.732}
        \fill (0,0) circle (\r);
        \fill ({\dx*2},0) circle (\r);
        \fill ({\dx*4},0) circle (\r);
        \fill ({\dx},{\dy}) circle (\r);
        \fill ({\dx*3},{\dy}) circle (\r);
        \fill ({\dx*2},{\dy*2}) circle (\r);
        \begin{scope}[shift={(9,0)},visible on=<4->]
          \draw[line width=1pt,->,red] (-4,{\dy})--(-1,{\dy})
          node[midway,above] {A};
          \fill (0,0) circle (\r);
          \fill ({\dx*2},0) circle (\r);
          \fill ({\dx*4},0) circle (\r);
          \fill ({\dx},{\dy}) circle (\r);
          \draw[red,dashed,very thick] ({\dx*3},{\dy}) circle (\r);
          \draw[red,dashed,very thick] ({\dx*2},{\dy*2}) circle (\r);
        \end{scope}
        \begin{scope}[shift={(18,0)},visible on=<5->]
          \draw[line width=1pt,->,blue] (-4,{\dy})--(-1,{\dy})
          node[midway,above] {B};
          \fill (0,0) circle (\r);
          \fill ({\dx*2},0) circle (\r);
          \fill ({\dx*4},0) circle (\r);
          \draw[blue,dashed,very thick] ({\dx},{\dy}) circle (\r);
        \end{scope}
        \begin{scope}[shift={(27,0)},visible on=<6->]
          \draw[line width=1pt,->,red] (-4,{\dy})--(-1,{\dy})
          node[midway,above] {A};
          \draw[red,dashed,very thick] (0,0) circle (\r);
          \draw[red,dashed,very thick] ({\dx*2},0) circle (\r);
          \draw[red,dashed,very thick] ({\dx*4},0) circle (\r);
          \draw (9,{\dy}) node[red] {A 获胜};
        \end{scope}
      \end{tikzpicture}
    \end{center}
    \onslide<+->\onslide<+->\onslide<+->
    \item 如果一开始石子的个数是 $4$ 的倍数. 那么每次 \alertn{A 取 $x$ 个}之后, \emphn{B} 只需要\emphn{取 $4-x$} 个, 就可以保证必胜.
    \item 如果一开始石子的个数不是 $4$ 的倍数, 那么A只需要取  $1\sim 3$ 个石子, 使得剩下的石子个数是 $4$ 的倍数即可获胜.
  \end{itemize}
\end{frame}


\begin{frame}{必胜条件}
  \begin{itemize}
    \item 可以看出, 只要A能将游戏状态变成\emph{后手必胜}, 那么原来的游戏就是\alert{先手必胜}.
    \item 如果无论 A 怎么操作, 都不能将游戏变成先手必胜, 那么这个游戏就是\emph{后手必胜}的.
    \item 如果初始有 $n$ 个石子, 令
    \[
      \CP(n)=\begin{cases}
        \color{red}1,&\color{red}\text{先手必胜};\\
        \color{blue}0,&\color{blue}\text{后手必胜}.
      \end{cases}
    \]
    \item 那么
    \[
      \CP(n)=1-\CP(n-1)\CP(n-2)\CP(n-3)
      =\begin{cases}
        \color{red}1,&\color{red}4\nmid n;\\
        \color{blue}0,&\color{blue}4\mid n.
      \end{cases}
    \]
    \item 这个序列($n\ge 0$)形如:
    \[
      {\color{red}0\color{blue}111\ 
      \color{red}0\color{blue}111\ 
      \color{red}0\color{blue}111\ }\cdots
    \]
  \end{itemize}
\end{frame}


\begin{frame}{必胜点}
  \begin{itemize}
    \item 我们将这个游戏记为 $\SUB(S)$, 其中 $S\subset \BN$ 表示每次可以取的石头个数.
    \item 由于有可能最后剩下的石子数量比 $S$ 中的最小元还要小, 所以我们将游戏规则改成\alert{谁不能取谁算输}更为合理.
    \onslide<+->
    \begin{center}
      \mode<beamer>{
      \begin{tikzpicture}
        \def\r{.3}
        \foreach \i in {0,1,...,5}
        \coordinate[minimum height=1cm] (A\i) at (\i,\r);
        \draw[visible on=<3->] (0,0) circle (\r) node {$0$};
        \begin{scope}[visible on=<4->]
          \draw[blue] (0,0) circle (\r) node {$0$}
          node[below=3mm] {$0$};
        \end{scope}
        \draw[visible on=<5->] (1,0) circle (\r) node {$1$};
        \draw[visible on=<6>,thick, ->] (A1) to [bend right=30] (A0);
        \draw[visible on=<6->,red] (1,0) circle (\r) node {$1$}
        node[below=3mm] {$1$};
        \draw[visible on=<7->] (2,0) circle (\r) node {$2$};
        \draw[visible on=<8-10>,thick, ->] (A2) to [bend right=30] (A0);
        \draw[visible on=<8-10>,thick, ->] (A2) to [bend right=30] (A1);
        \draw[visible on=<8->,red] (2,0) circle (\r) node {$2$};
        \draw[visible on=<9>,red] (2,0) node[below=7mm] {\small 可变成 $0,1$};
        \begin{scope}[visible on=<10->]
          \draw[red] (2,0) node[below=3mm] {$2$};
          \draw[red] (2,0) node[below=7mm,align=center] {\small 级\\必\\胜\\点};
        \end{scope}
        \begin{scope}[visible on=<11->]
          \draw[blue] (0,0) node[below=7mm,align=center] {\small 级\\必\\胜\\点};
          \draw[red] (1,0) node[below=7mm,align=center] {\small 级\\必\\胜\\点};
        \end{scope}
        \begin{scope}[visible on=<12>]
          \draw[thick, ->] (A3) to [bend right=30] (A0);
          \draw[thick, ->] (A3) to [bend right=30] (A1);
          \draw[thick, ->] (A3) to [bend right=30] (A2);
        \end{scope}
        \begin{scope}[visible on=<12->]
          \draw[red] (3,0) circle (\r) node {$3$} node[below=3mm] {$3$};
          \draw[red] (3,0) node[below=7mm,align=center] {\small 级\\必\\胜\\点};
        \end{scope}
        \begin{scope}[visible on=<13>]
          \draw[thick, ->] (A4) to [bend right=30] (A1);
          \draw[thick, ->] (A4) to [bend right=30] (A2);
          \draw[thick, ->] (A4) to [bend right=30] (A3);
        \end{scope}
        \begin{scope}[visible on=<13->]
          \draw[blue] (4,0) circle (\r) node {$4$} node[below=3mm] {$0$};
          \draw[blue] (4,0) node[below=7mm,align=center] {\small 级\\必\\胜\\点};
        \end{scope}
        \begin{scope}[visible on=<14->,red]
          \draw (5,0) circle (\r) node {$5$} node[below=3mm] {$1$};
          \draw (6,0) circle (\r) node {$6$} node[below=3mm] {$2$};
          \draw (7,0) circle (\r) node {$7$} node[below=3mm] {$3$};
          \draw[blue] (8,0) circle (\r) node {$8$} node[below=3mm] {$0$};
          \draw (5,0) node[below=7mm,align=center] {\small 级\\必\\胜\\点};
          \draw (6,0) node[below=7mm,align=center] {\small 级\\必\\胜\\点};
          \draw (7,0) node[below=7mm,align=center] {\small 级\\必\\胜\\点};
          \draw[blue] (8,0) node[below=7mm,align=center] {\small 级\\必\\胜\\点};
        \end{scope}
      \end{tikzpicture}}
      \mode<handout>{
      \begin{tikzpicture}[red]
        \def\r{.3}
        \draw[blue] (0,0) circle (\r) node {$0$} node[below=3mm] {$0$};
        \draw (1,0) circle (\r) node {$1$} node[below=3mm] {$1$};
        \draw (2,0) circle (\r) node {$2$} node[below=3mm] {$2$};
        \draw (3,0) circle (\r) node {$3$} node[below=3mm] {$3$};
        \draw[blue] (4,0) circle (\r) node {$4$} node[below=3mm] {$0$};
        \draw (5,0) circle (\r) node {$5$} node[below=3mm] {$1$};
        \draw (6,0) circle (\r) node {$6$} node[below=3mm] {$2$};
        \draw (7,0) circle (\r) node {$7$} node[below=3mm] {$3$};
        \draw[blue] (8,0) circle (\r) node {$8$} node[below=3mm] {$0$};
        \draw[blue] (0,0) node[below=7mm,align=center] {\small 级\\必\\胜\\点};
        \draw (1,0) node[below=7mm,align=center] {\small 级\\必\\胜\\点};
        \draw (2,0) node[below=7mm,align=center] {\small 级\\必\\胜\\点};
        \draw (3,0) node[below=7mm,align=center] {\small 级\\必\\胜\\点};
        \draw[blue] (4,0) node[below=7mm,align=center] {\small 级\\必\\胜\\点};
        \draw (5,0) node[below=7mm,align=center] {\small 级\\必\\胜\\点};
        \draw (6,0) node[below=7mm,align=center] {\small 级\\必\\胜\\点};
        \draw (7,0) node[below=7mm,align=center] {\small 级\\必\\胜\\点};
        \draw[blue] (8,0) node[below=7mm,align=center] {\small 级\\必\\胜\\点};
      \end{tikzpicture}}
    \end{center}
    \onslide<+->\onslide<+->\onslide<+->\onslide<+->\onslide<+->\onslide<+->\onslide<+->
    可以变成 $0\sim m-1$ 级必胜点的点, 叫做 $m$ 级必胜点.
  \end{itemize}
\end{frame}



\begin{frame}{Sprague-Grundy 序列}
  \begin{itemize}
    \item 如果 $n$ 个石子情形是 $m$ 级必胜点, 定义 $\CG_S(n)=m$,
    \onslide<+->
    并称该序列为 \emph{Sprague-Grundy 序列} (或 Nim 序列).
    \onslide<+->
    那么
    \[
      \CG_S(n)=\mex\{\CG_S(n-s): s\in S\},
    \]
    $\mex$ 是指不属于后面集合的最小的非负整数 (\alertn{m}inimal \alertn{ex}cept).
  \end{itemize}
\end{frame}


\begin{frame}{Nim 游戏及其变种}
  \onslide<+->
  实际上 Nim 游戏(抓石子游戏)有相当多的变种, 例如
  \begin{itemize}
    \item 有多个石子堆;
    \item 有无穷多种取法 ($S$ 无限);
    \item 高维情形 ($n$ 是向量, $S$ 是向量集合) 等等.
  \end{itemize}
  \onslide<+->
  我们今天只讨论 $S$ 有限的\emphov{一维一堆情形}{subtraction game}.

  \onslide<+->
  注意到 $\CG_{dS}(n)=\CG_S\bigl(\bigl[\frac nd\bigr]\bigr)$.
  \onslide<+->
  因此我们只需考虑 $S$ 的所有元素公因子为 $1$ 的情形.
\end{frame}


\begin{frame}{周期和预周期}
  \begin{itemize}
    \item 我们将集合 $S$ 中的元素从小到大排列, 即
    \[
      S=\{s_1,s_2,\dots,s_k\},\quad s_1<s_2<\cdots<s_k.
    \]
    \item 那么 $\CG(n)\le k$.
    \onslide<+->
    于是 S-G 序列中连续 $s_k$ 项形成的序列只有 $(k+1)^{s_k}$ 种可能, 从而由抽屉原理可知, 存在两个相同的 $s_k$ 项序列.
    \onslide<+->
    而 $\CG(n)$ 仅由它之前的 $s_k$ 项决定, 所以我们得到:
  \end{itemize}
  \onslide<+->
  \begin{proposition}
    序列 $\CG$ 是\emphov{最终周期}{ultimately periodic}的, 即存在整数 $p\ge 1,\ell\ge 0$ 使得 $\CG(n+p)=\CG(n),\forall n\ge \ell$.
  \end{proposition}
  \begin{itemize}
    \item 将最小的 $p$ 称为($\CG_S$ 或 $\SUB(S)$ 的)\emphov{周期}{period}, 最小的 $\ell$ 称为\emphov{预周期}{pre-period}.
  \end{itemize}
\end{frame}


\begin{frame}{周期和预周期(续)}
  \begin{itemize}
    \item 于是
    \[
      \CG=\CG(0)\CG(1)\CG(2)\cdots
      =\CG(0)\cdots\CG(\ell-1)\ul{\CG(\ell)\cdots\CG(\ell+p-1)}.
    \]
    这里 $\ul{\CH}=\CH\CH\cdots$ 表示无穷多个 $\CH$ 重复得到的序列.
    \item 不难说明, 满足 $\CG(n)=\CG(n+p),\ell\le\forall  n\le \ell+s_k$ 的最小的 $p$ 和 $\ell$ 就是周期和预周期.
    \item 因此对于任意集合 $S$, 很容易通过计算机来计算它的周期和预周期, 从而得到整个 S-G 序列.
    \item 显然 $p,\ell\le (k+1)^{s_k}$.
  \end{itemize}
\end{frame}


\section{已知的结论}

\begin{frame}{二元集合情形}
  \onslide<+->
  当 $k=\#S\le 2$ 时, $p$ 和 $\ell$ 都是已知的.
  \onslide<+->
  而即使是 $k=3$ 的情形, $p$ 和 $\ell$ 依然还不是完全知道.
  \onslide<+->
  我们将回顾已知的并给出一些新的结果.
  \begin{itemize}
    \item $\CG_{\{1\}}=\ul{01}$.
    \item $1\in S$ 不含偶数 $\iff \CG_S=\ul{01}$.
    \item 事实上, 如果 $S'=S\cup\{x+pt\}$, 其中 $x\in S,p$ 是 $\CG_S$ 周期, 则 $\CG_{S'}=\CG_S$.
    \item 设 $S=\{a,c=at+r\}, 0\le r<a$, 则
    \[
      \CG_S=\begin{cases}
        \ul{(0^a 1^a)^{t/2} 0^r 2^{a-r}1^r},&2\mid t;\\
        \ul{(0^a 1^a)^{(t+1)/2} 2^r},&2\nmid t,
      \end{cases},\quad
      \ell=0, p=c+a\ \text{或}\ 2a.
    \]
    这里 $\CH^t=\CH\cdots\CH$ 表示 $t$ 个 $\CH$ 重复得到的序列.
    \onslide<+->
    注意 $2\nmid t$ 时这里未必是最小循环节.
  \end{itemize}
\end{frame}


\begin{frame}{三元集合: $a=1,b$ 奇}
  \onslide<+->
  \begin{example}
    设 $S=\{1,b,c\},2\nmid b$.
    注意到 $\CG_\{1,b\}=\ul\CH, \CH=01$.
    我们有
    \begin{center}
      \begin{tabular}{cccc}
        \topcolorrule
        $c$ & $\CG_S$ & $\ell$ & $p$\\
        \midcolorrule
        奇数&$\ul\CH$ & $0$ & $2$\\
        偶数&$\ul{\CH^{c/2}(23)^{(b-1)/2}2}$ & $0$ & $c+b$\\
        \bottomcolorrule
      \end{tabular}
    \end{center}
  \end{example}
\end{frame}


\begin{frame}{三元集合: $a=1,b=2$}
  \onslide<+->
  \begin{example}
    设 $S=\{1,2,3t+r\},0\le r<3$.
    注意到 $\CG_\{1,2\}=\ul\CH, \CH=012$.
    我们有
    \begin{center}
      \begin{tabular}{cccc}
        \topcolorrule
        $r$ & $\CG_S$ & $\ell$ & $p$\\
        \midcolorrule
        $0$ & $\ul{(012)^t3}$ & $0$ & $c+1$\\
        $1,2$ & $\ul{012}$ & $0$ & $3$\\
        \bottomcolorrule
      \end{tabular}
    \end{center}
  \end{example}
\end{frame}


\begin{frame}{三元集合: $a=1,b=4$}
  \onslide<+->
  \begin{example}
    设 $S=\{1,4,c=5t+r\}, 0\le r<5$.
    注意到 $\CG_\{1,4\}=\ul{\CH}, \CH=01012$.
    我们有
    \begin{center}
      \begin{tabular}{cccc}
        \topcolorrule
        $r,c$ & $\CG_S$ & $\ell$ & $p$\\
        \midcolorrule
        $r=0, c=5$ & $\ul{\CH\,323}$ & $0$ & $8$\\
        $r=0, c>5$ & $\CH^t\,323013\ul{\CH^{t-1}012012}$ & $c+6$ & $c+1$\\
        $r=1,4$ & $\ul\CH$ & $0$ & $5$\\
        $r=2$ & $\ul{\CH^t\,012}$ & $0$ & $c+1$\\
        $r=3$ & $\ul{\CH^{t+1}\,32}$ & $0$ & $c+4$\\
        \bottomcolorrule
      \end{tabular}
    \end{center}
  \end{example}
\end{frame}


\begin{frame}{三元集合: $a=1,b\ge 6$ 偶}
  \onslide<+->
  \begin{proposition}[near]
    设 $S=\{1,b,c\}$, 其中 $b\ge 6$ 是偶数, $c=t(b+1)+r,0\le r\le b$.
    \onslide<+->{
    我们有
    \begin{center}
      \begin{tabular}{cccc}
        \topcolorrule
        $r$ & $\ell$ & $p$\\
        \midcolorrule
        $1,b$ & $0$ & $b+1$\\
        $[3,b-1]$ 是奇数 & $0$ & $c+b$\\
        $b-2$ & $0$ & $c+1$\\
        $c=b+1$ & $0$ & $2b$\\
        \begin{tikzpicture}[overlay,shift={(-.1,.2)}]
          \draw[decorate,decoration={brace,amplitude=5},thick] (0,-1.3)--(0,0);
          \draw (-1.5,-.65) node[align=left] {\small $c>b+1$\smallskip\\$r\le b-4$ 偶};
        \end{tikzpicture}
        $r>b-2t-2$&
        $\bigl(\frac{b-r}2-1\bigr)(c+b+2)-b$&
        $c+1$\\
        $r=b-2t-2$&
        $t(c+b+2)-b$&
        $b-1$\\
        $r<b-2t-2$&
        $t(c+b+2)-b$&
        $c+b$\\
        \bottomcolorrule
      \end{tabular}
      \smalldel
    \end{center}}
  \end{proposition}
  \begin{itemize}
    \item 可以看出在带 $1$ 的三元集情形, $p$ 和 $\ell$ 的形式与 $c$ 模 $\{1,b\}$ 的周期的同余类有关.
    \item 除去有限多种情形外, $c$ 在每一个同余类中, $p$ 和 $\ell$ 是 $c$ 的一次函数.
  \end{itemize}
\end{frame}


\begin{frame}{三元集合: $a=1,b\ge 6$ 偶(续)}
  \onslide<+->
  该情形 $\CG$ 序列较为复杂.
  \onslide<+->
  例如: 若 $0<r=2v<b-2t-2$, $b=2k$, 则
  \begin{center}
    \begin{tabular}{cl}
      \topcolorrule
      $i$ & $\CG\bigl((c+1)i+j\bigr), 0\le j\le c$\\
      \midcolorrule
    $0$ &
      $\CH^t\, (01)^v2$\\\hline
    $1$ &
      $(32)^{k-v-1}(01)^{v+1}2, \CH^{t-1} (01)^v0$\\\hline
    $2$ &
      $1(01)^{k-v-2}2(01)^{v+1}2, (32)^{k-v-2}(01)^{v+2}2, \CH^{t-2} (01)^v0$\\\hline
    $i$ &
      \makecell[l]{$1(01)^{k-v-2}2(01)^{v+1}0, \dots, 1(01)^{k-v-i+1}2(01)^{v+i-2}0, $\\
      $1(01)^{k-v-i}2(01)^{v+i-1}2, (32)^{k-v-i}(01)^{v+i}2, \CH^{t-i} (01)^v0$}\\\hline
    $t-1$ &
      \makecell[l]{$1(01)^{k-v-2}2(01)^{v+1}0, \dots, 1(01)^{k-v-t+2}2(01)^{v+t-3}0, $\\
      $1(01)^{k-v-t+1}2(01)^{v+t-2}2, (32)^{k-v-t+1}(01)^{v+t-1}2, \CH^1 (01)^v0$}\\\hline
    $t$ &
      \makecell[l]{$1(01)^{k-v-2}2(01)^{v+1}0, \dots, 1(01)^{k-v-t+1}2(01)^{v+t-2}0, $\\
      $1(01)^{k-v-t}2(01)^{v+t-1}2, (3$\emphn{$2)^{k-v-t}(01)^{v+t}2, (01)^v0$}}\\\hline
    $t+1$ &
      \makecell[l]{\emphn{$1(01)^{k-v-2}2(01)^{v+1}0, \dots, 1(01)^{k-v-t+1} 2(01)^{v+t-2}0$,}\\
      \emphn{$1(01)^{k-v-t}2(01)^{v+t-1}0, 1(01)^{k-v-t-1}2(01)^{v+t}$}$2, (32)^{k-v-t-1}01\cdots$}\\
      \bottomcolorrule
    \end{tabular}
  \end{center}
\end{frame}


\begin{frame}{更多的例子}
  \onslide<+->
  \begin{proposition}
    \begin{enumerate}
      \item 设 $S=\{a,2a,c=3at+r\},0\le r<3a$, 则
      \[
        \ell=\begin{cases}
          c+a-r,&0<r<a;\\
          0,&\text{其它情形},
        \end{cases}\quad 
        p=\begin{cases}
          3a/2,&r=a/2;\\
          3a,&a/2<r\le 2a;\\
          c+a,&\text{其它情形.}
        \end{cases}
      \]
      \item 设 $S=\{a,a+1,\dots,b-1,b,c=t(a+b)+r\},0\le r<a+b$, 则
      \[\ell=0,\quad p=\begin{cases}
        a+b,&a\le r\le b;\\
        c+a,&r=0\text{ 或 }r>b;\\
        c+b,&0<r<a.
      \end{cases}\]
    \end{enumerate}
  \end{proposition}
\end{frame}


\begin{frame}{五元集合的例子}
  \onslide<+->
  \begin{example}
    设 $S=\{2,3,5,7\}$, 则 $\CG_S=\ul{0^2 1^2 2^2 3^2 4}$ 周期为 $9$.
    对于 $11\le c\le 500$, $\SUB\bigl(S\cup\{c\}\bigr)$ 的预周期和周期为
    \[\ell_c=\begin{cases}
      2c-4,&c\equiv1\bmod{18};\\
      c+5,&c\equiv10\bmod{18};\\
      0,&\text{其它情形},
    \end{cases}\quad
      p_c=\begin{cases}
      c+2, &c\equiv0,8,9,10,17\bmod{18};\\
      4, &c\equiv1\bmod{18};\\
      9, &\text{其它情形}.
    \end{cases}\]
  \end{example}
\end{frame}


\section{主要猜想、结论和应用}
\begin{frame}{主要猜想结论}
  \onslide<+->
  由此我们猜测 $\SUB(S\cup\{c\})$ 周期和预周期关于 $c$ 是\emph{最终逐剩余类线性}的:
  \onslide<+->
  \begin{conjecture}
    固定集合 $S$. 存在正整数 $q, N$ 以及 $\alpha_r,\beta_r,\lambda_r,\mu_r,0\le r<q$,
    使得当 $c\ge N$ 且 $c\equiv r\bmod q$ 时,
      $\SUB(S\cup\{c\})$ 的预周期和周期分别是 $\alpha_r c+\beta_r$ 和 $\lambda_r c+\mu_r$.
  \end{conjecture}
  \onslide<+->
  \begin{theorem}
    上述猜想在如下情形成立:
    \begin{enumerate}
      \item $1\in S$ 且 $S$ 所有元素均为奇数;
      \item $S=\{1,b\}$;
      \item $S=\{a,2a\}$;
      \item $S=\{a,a+1,\dots,b-1,b\}$.
    \end{enumerate}
  \end{theorem}
\end{frame}


\begin{frame}{应用: 最终二分序列}
  \onslide<+->
  这个猜想可以指导我们寻找特定周期的 S-G 序列.
  \onslide<+->
  如果 $\CG_S$ 的周期为 $2$, 称 $\SUB(S)$ 是\emphov{最终二分}{ultimately bipartite}的.
  \onslide<+->
  可以证明如果 $\SUB(S)$ 是最终二分的, 则 $S$ 不含偶数.
  \onslide<+->
  \begin{example}
    设 $a\ge 3$ 是奇数. 如果 $S$ 是如下情形之一:
    \begin{itemize}[<*>]
      \item $S=\{3,5,9,\dots,2^a+1\}$;
      \item $S=\{3,5,2^a+1\}$;
      \item $S=\{a,a+2,2a+3\}$;
      \item $S=\{a,2a+1,3a\}$,
    \end{itemize}
    则 $\SUB(S)$ 是最终二分的.
  \end{example}
\end{frame}


\begin{frame}{应用: 最终二分序列}
  \onslide<+->
  根据上面的例子和猜想的启发, 我们发现了如下三元最终二分 $\SUB(S)$.
  \onslide<+->
  \begin{theorem}
    设 $a\ge 3$ 是奇数, $t\ge 1$.
    如果 $S$ 是如下情形之一:
    \begin{enumerate}[<*>]
      \item $S=\{a,a+2,(2a+2)t+1\}$;\label{enum:bi1} \hfill(来自 $\{a,a+2,2a+3\}$)
      \item $S=\{a,2a+1,(3a+1)t-1\}$; \hfill(来自 $\{a,2a+1,3a\}$)
      \item $S=\{a,2a-1,(3a-1)t+a-2\}$, \hfill(来自 $\{a,a+2,2a+3\}$)
    \end{enumerate}
    则 $\SUB(S)$ 是最终二分的.
  \end{theorem}
\end{frame}


\begin{frame}{应用: 最终二分序列}
  \onslide<+->
  例如情形 \enumnum1 的G-S序列开头为 ($a=2k+1$):
  \begin{center}\small
    \begin{tabular}{c@{}c@{}c@{}c@{}c@{}c@{}c@{}c@{}}
    \topcolorrule
      $i$ & \multicolumn{7}{l}{$\CG\bigl((a+1)(2t+1)i+j\bigr), 0\le j<(a+1)(2t+1)=c+a$}
    \\\midcolorrule
      $0$&
      $0^a1$&
      $[$&
      $1^{a-1}22$&
      $0^a1$&
      $]^{t-1}$&
      $1^{a-1}22$&
      $02^{a-3}331$
    \\\midcolorrule
      $1$&
      $030^{a-2}1$&
      $[$&
      $01^{a-2}21$&
      $020^{a-2}1$&
      $]^{t-1}$&
      $01^{a-2}21$&
      $0202^{a-5}321$
    \\\midcolorrule
      $i$&
      $(01)^{i-1}030^{a-2i}1$&
      $[$&
      $(01)^{i-1}01^{a-2i}21$&
      $(01)^{i-1}020^{a-2i}1$&
      $]^{t-1}$&
      $(01)^{i-1}01^{a-2i}21$&
      $(01)^{i-1}0202^{a-2i-3}321$
    \\\midcolorrule
      $k-1$&
      $(01)^{k-2}030^31$&
      $[$&
      $(01)^{k-2}01^321$&
      $(01)^{k-2}020^31$&
      $]^{t-1}$&
      $(01)^{k-2}01^321$&
      $(01)^{k-2}020321$
    \\\midcolorrule
      $k$&
      $(01)^{k-1}0301$&
      $[$&
      $(01)^{k-1}0121$&
      $(01)^{k-1}03$\emphn{$01$}&
      $]^{t-1}$&
      \emphn{$(01)^{k-1}0101$}&
      \emphn{$(01)^{k-1}0101$}
    \\\bottomcolorrule
    \end{tabular}
  \end{center}
\end{frame}


% \begin{frame}{应用: 最终二分序列}
%   \onslide<+->
%   例如情形 \enumnum1的G-S序列开头为:
%   \begin{center}
%     \begin{tabular}{cc@{}cc@{}cc@{}c}
%     \topcolorrule
%       $i$ & \multicolumn{5}{l}{$\CG\bigl((a+1)(2t+1)i+j\bigr), 0\le j<(a+1)(2t+1)=c+a$}\\
%     \midcolorrule
%       $0$&
%       \makecell{$0^a1$\\~}&
%       \makecell{$[$\\~}&
%       \makecell{$1^{a-1}22$\\$1^{a-1}22$}&
%       \makecell{$0^a1$\\$02^{a-3}331$}&
%       \makecell{$]^{t-1}$\\~}\\\hline
%       $1$&
%       \makecell{$030^{a-2}1$\\~}&
%       \makecell{$[$\\~}&
%       \makecell{$01^{a-2}21$\\$01^{a-2}21$}&
%       \makecell{$020^{a-2}1$\\$0202^{a-5}321$}&
%       \makecell{$]^{t-1}$\\~}\\\hline
%       $i$&
%       \makecell{$(01)^{i-1}030^{a-2i}1$\\~}&
%       \makecell{$[$\\~}&
%       \makecell{$(01)^{i-1}01^{a-2i}21$\\$(01)^{i-1}01^{a-2i}21$}&
%       \makecell{$(01)^{i-1}020^{a-2i}1$\\$(01)^{i-1}0202^{a-2i-3}321$}&
%       \makecell{$]^{t-1}$\\~}\\\hline
%       $k-1$&
%       \makecell{$(01)^{k-2}030^31$\\~}&
%       \makecell{$[$\\~}&
%       \makecell{$(01)^{k-2}01^321$\\$(01)^{k-2}01^321$}&
%       \makecell{$(01)^{k-2}020^31$\\$(01)^{k-2}020321$}&
%       \makecell{$]^{t-1}$\\~}\\\hline
%       $k$&
%       \makecell{$(01)^{k-1}0301$\\~}&
%       \makecell{$[$\\~}&
%       \makecell{$(01)^{k-1}0121$\\\emphn{$(01)^{k-1}0101$}}&
%       \makecell{$(01)^{k-1}03$\emphn{$01$}\\\emphn{$(01)^{k-1}0101$}}&
%       \makecell{$]^{t-1}$\\~}\\
%     \bottomcolorrule
%     \end{tabular}
%   \end{center}
%   % \onslide<+->
%   % 对于四元情形, 我们通过计算发现了当 $3\le a\le 25, c<500$ 且 $c\not\equiv \pm1 \bmod a$ 时, $\SUB(\{a,2a+1,3a,c\})$ 是最终二分的.
% \end{frame}


\end{document}



