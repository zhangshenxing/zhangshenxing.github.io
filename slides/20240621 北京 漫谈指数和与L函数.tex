	\documentclass[aspectratio=169,handout]{ctexbeamer}
\usepackage{amsmath,amssymb,amsthm}
\usepackage{bm}
\usepackage{extarrows}
\usepackage{mathrsfs}
\usepackage{stmaryrd}
\RequirePackage[T1]{fontenc} % 纯英文去掉此行
\setCJKsansfont[ItalicFont={KaiTi},BoldFont={SimHei}]{Microsoft YaHei} % 纯英文去掉此行
\usetheme{reed}
\renewcommand\emph[1]{{\color{structure.fg!50!blue}{#1}}}
\setlength{\parindent}{1.5em}
\renewcommand{\indent}{\hspace*{1.5em}}
\setbeamerfont{title}{size=\Large}
\setbeamerfont{frametitle}{size=\Large}

\AtBeginSection{
	\setbeamertemplate{subsection in footline}[normal]
	{
		\setnaviboxempty
		\frame[c]{\sectionpage}
	}
}
\title{漫谈指数和与 $L$ 函数}
\author{张神星 (合肥工业大学)}
\institute{首都师范大学}
\date{2024年6月22日}
\email{zhangshenxing@hfut.edu.cn}
\titlegraphic{misc/cnu.jpg}
\institutegraphic{misc/cnuname.png}
% 自定义主题配色
\definecolor{themecolor}{RGB}{28,73,136}
\setbeamercolor{theme}{fg=themecolor}
\setbeamersize{text margin left=1em,text margin right=1em}

\renewcommand{\labelenumi}{{\upshape(\arabic{enumi})}}
\newcommand\enumnum[1]{{\textcolor{fourth}{\mdseries\upshape{(#1)}}}}
\newcommand\peq{\mathrel{\phantom{=}}} % 用于对齐的等号幻影

\usepackage{tikz}
\usepackage{color,calc}
% TIKZ 设置
\usetikzlibrary{
	quotes,
	shapes.arrows,
	arrows.meta,
	positioning,
	shapes.geometric,
	patterns,
	calc,
	angles,
	decorations.pathreplacing,
	backgrounds % 背景边框
}
\tikzset{
	background rectangle/.style={semithick,draw=fourth,fill=white,rounded corners},
  % arrow
	cstra/.style      ={-Stealth},        % right arrow
	cstla/.style      ={Stealth-},        % left arrow
	cstlra/.style     ={Stealth-Stealth}, % left-right arrow
	cstwra/.style     ={-Straight Barb},  % wide ra
	cstwla/.style     ={Straight Barb-},
	cstwlra/.style    ={Straight Barb-Straight Barb},
	cstnarrow/.style      ={-Latex, line width=0.1cm}, %文本框间箭头
	cstaxis/.style        ={-Stealth, thick}, %坐标轴
  % curve
	cstcurve/.style       ={very thick}, %一般曲线
	cstdash/.style        ={thick, dash pattern= on 0.2cm off 0.05cm}, %虚线
  % dot
	cstdot/.style         ={radius=.08}, %实心点
	cstdote/.style        ={radius=.07, fill=white}, %空心点
  % fill
	cstfill/.style       ={fill=black!10},
	cstfille/.style      ={pattern=north east lines, pattern color=black},
	cstfill1/.style       ={fill=main!20},
	cstfille1/.style      ={pattern=north east lines, pattern color=main},
	cstfill2/.style        ={fill=second!20},
	cstfille2/.style       ={pattern=north east lines, pattern color=second},
	cstfill3/.style        ={fill=third!20},
	cstfille3/.style       ={pattern=north east lines, pattern color=third},
	cstfill4/.style        ={fill=fourth!20},
	cstfille4/.style       ={pattern=north east lines, pattern color=fourth},
	cstfill5/.style        ={fill=fifth!20},
	cstfille5/.style       ={pattern=north east lines, pattern color=fifth},
  % node
	cstnode/.style        ={fill=white,draw=black,text=black,rounded corners=0.2cm,line width=1pt},
	cstnode1/.style       ={fill=main!15,draw=main!80,text=black,rounded corners=0.2cm,line width=1pt},
	cstnode2/.style       ={fill=second!15,draw=second!80,text=black,rounded corners=0.2cm,line width=1pt},
	cstnode3/.style       ={fill=third!15,draw=third!80,text=black,rounded corners=0.2cm,line width=1pt},
	cstnode4/.style       ={fill=fourth!15,draw=fourth!80,text=black,rounded corners=0.2cm,line width=1pt},
	cstnode5/.style       ={fill=fifth!15,draw=fifth!80,text=black,rounded corners=0.2cm,line width=1pt}
}
\ExplSyntaxOn
\cs_new_protected:Npn \fpstepfromto#1#2#3 
  {% from, to, nums
    \fp_step_inline:nnnn {#1} { (#2-(#1))/(#3-1)*0.99 } {#2}
  }
\pgfmathdeclarefunction{nrand}{0}
  {% \tex_normaldeviate:D 生成均值为 0,标准差为 10000 的随机整数
    \tl_set:Nx \pgfmathresult { \fp_eval:n { \tex_normaldeviate:D/10000 } }
  }
\pgfmathdeclarefunction{rdv}{0}{\pgfmathparse{1+nrand/100}}
\ExplSyntaxOff
\newcommand{\randpts}[3][10]{
  \foreach\i in {0,1,...,#1}{
    \pgfmathparse{rdv}\let\rdv\pgfmathresult
    \coordinate (\i) at ({#2*\rdv*cos(360/#1*\i)},{#3*\rdv*sin(360/#1*\i)});
  }
}
\newcommand{\randep}[2]{
  \randpts{#1}{#2}
  \filldraw[cstcurve,main,cstfill3,smooth] plot coordinates {(0) (1) (2) (3) (4) (5) (6) (7) (8) (9) (0)};
}

\let\question\relax
\elegantnewtheorem{question}{问题}{defstyle}{que}

\newfontface\cmunrm{cmunrm.otf}
\newcommand\cmu[1]{{\cmunrm{#1}}}
\newcommand{\alert}[1]{\textcolor{main}{\bf #1}}
\renewcommand{\emph}[1]{\textcolor{second}{\bf #1}}
\newcommand{\cnen}[2]{{\kaishu$\overset{\text{{#2}}}{\text{#1}}$}}
\newcommand{\nounen}[2]{{\color{second}\kaishu\cnen{#1}{#2}}\index{{#1}}}
\newcommand{\noun}[1]{{\color{second}\kaishu #1}\index{{#1}}}
\newcommand{\nouns}[2]{{\color{second}\kaishu #1}\index{{#2}}}
\newcommand{\nounsen}[3]{{\color{second}\kaishu\cnen{#1}{#2}}\index{{#3}}}

\setcounter{tocdepth}{2}
\newfontfamily\couriernew{Courier New}
\lstset{language=[LaTeX]TeX,
	basicstyle=\couriernew,
  morekeywords={AUTHOR, KEY, TITLE, YEAR, PAGES, HOWPUBLISHED, URL, LANGUAGE},
  keywordstyle=\color{winered}
}




\RequirePackage{extarrows}
\usepackage[normalem]{ulem}
\NewDocumentCommand\fillblank{O{1cm} O{0cm} m}{\uline{\makebox[#1]{\raisebox{#2}{#3}}}}
\newcommand\fillbrace[1]{{(\nolinebreak\hspace{0.5em minus 0.5em}{#1}\hspace{0.5em minus 0.5em}\nolinebreak)}}
\newcommand\resizet[1]{\resizebox{!}{#1\baselineskip}}
\newcommand{\trueex}{$\checkmark$}
\newcommand{\falseex}{$\times$}

% arrows
\newcommand\ra{\rightarrow}
\newcommand\lra{\longrightarrow}
\newcommand\la{\leftarrow}
\newcommand\lla{\longleftarrow}
\newcommand\sqra{\rightsquigarrow}
\newcommand\sqlra{\leftrightsquigarrow}
\newcommand\inj{\hookrightarrow}
\newcommand\linj{\hookleftarrow}
\newcommand\surj{\twoheadrightarrow}
\newcommand\simto{\stackrel{\sim}{\longrightarrow}}
\newcommand\sto[1]{\stackrel{#1}{\longrightarrow}}
\newcommand\lsto[1]{\stackrel{#1}{\longleftarrow}}
\newcommand\xto{\xlongrightarrow}
\newcommand\xeq{\xlongequal}
\newcommand\luobida{\xeq{\text{洛必达}}}
\newcommand\djwqx{\xeq{\text{等价无穷小}}}
\newcommand\eqprob{\stackrel{\mathrm{P}}{=}}
\newcommand\lto{\longmapsto}
\renewcommand\vec[1]{\overrightarrow{#1}}

% decorations
\newcommand\wh{\widehat}
\newcommand\wt{\widetilde}
\newcommand\ov{\overline}
\newcommand\ul{\underline}
\newlength{\larc}
\NewDocumentCommand\warc{o m}{%
	\IfNoValueTF {#1}%
	{%
		\settowidth{\larc}{$#2$}%
		\stackrel{\rotatebox{-90}{\ensuremath{\left(\rule{0ex}{0.7\larc}\right.}}}{#2}%
	}%
	{%
		\stackrel{\rotatebox{-90}{\ensuremath{\left(\rule{0ex}{#1}\right.}}}{#2}%
	}%
}

% braces
\newcommand\set[1]{{\left\{#1\right\}}}
\newcommand\setm[2]{{\left\{#1\,\middle\vert\, #2\right\}}}
\newcommand\abs[1]{\left|#1\right|}
\newcommand\pair[1]{\langle{#1}\rangle}
\newcommand\norm[1]{\!\parallel\!{#1}\!\parallel\!}
\newcommand\dbb[1]{\llbracket#1 \rrbracket}
\newcommand\floor[1]{\lfloor#1\rfloor}

% symbols
\renewcommand\le{\leqslant}
\renewcommand\ge{\geqslant}
\newcommand\vare{\varepsilon}
\newcommand\varp{\varphi}
\newcommand\ilim{\varinjlim\limits}
\newcommand\plim{\varprojlim\limits}
\newcommand\half{\frac{1}{2}}
\newcommand\mmid{\parallel}
\font\cyr=wncyr10\newcommand\Sha{\hbox{\cyr X}}
\newcommand\Uc{\stackrel{\circ}{U}\!\!}
\newcommand\hil[3]{\left(\frac{{#1},{#2}}{#3}\right)}
\newcommand\leg[2]{\Bigl(\frac{{#1}}{#2}\Bigr)}
\newcommand\aleg[2]{\Bigl[\frac{{#1}}{#2}\Bigr]}
\newcommand\stsc[2]{\genfrac{}{}{0pt}{}{#1}{#2}}

% categories
\newcommand\cA{{\mathsf{A}}}
\newcommand\cb{{\mathsf{b}}}
\newcommand\cB{{\mathsf{B}}}
\newcommand\cC{{\mathsf{C}}}
\newcommand\cD{{\mathsf{D}}}
\newcommand\cM{{\mathsf{M}}}
\newcommand\cR{{\mathsf{R}}}
\newcommand\cP{{\mathsf{P}}}
\newcommand\cT{{\mathsf{T}}}
\newcommand\cX{{\mathsf{X}}}
\newcommand\cx{{\mathsf{x}}}
\newcommand\cAb{{\mathsf{Ab}}}
\newcommand\cBT{{\mathsf{BT}}}
\newcommand\cBun{{\mathsf{Bun}}}
\newcommand\cCharLoc{{\mathsf{CharLoc}}}
\newcommand\cCoh{{\mathsf{Coh}}}
\newcommand\cComm{{\mathsf{Comm}}}
\newcommand\cEt{{\mathsf{Et}}}
\newcommand\cFppf{{\mathsf{Fppf}}}
\newcommand\cFpqc{{\mathsf{Fpqc}}}
\newcommand\cFunc{{\mathsf{Func}}}
\newcommand\cGroups{{\mathsf{Groups}}}
\newcommand\cGrpd{{\{\mathsf{Grpd}\}}}
\newcommand\cHo{{\mathsf{Ho}}}
\newcommand\cIso{{\mathsf{Iso}}}
\newcommand\cLoc{{\mathsf{Loc}}}
\newcommand\cMod{{\mathsf{Mod}}}
\newcommand\cModFil{{\mathsf{ModFil}}}
\newcommand\cNilp{{\mathsf{Nilp}}}
\newcommand\cPerf{{\mathsf{Perf}}}
\newcommand\cPN{{\mathsf{PN}}}
\newcommand\cRep{{\mathsf{Rep}}}
\newcommand\cRings{{\mathsf{Rings}}}
\newcommand\cSets{{\mathsf{Sets}}}
\newcommand\cStack{{\mathsf{Stack}}}
\newcommand\cSch{{\mathsf{Sch}}}
\newcommand\cTop{{\mathsf{Top}}}
\newcommand\cVect{{\mathsf{Vect}}}
\newcommand\cZar{{\mathsf{Zar}}}
\newcommand\cphimod{{\varphi\txt{-}\mathsf{Mod}}}
\newcommand\cphimodfil{{\varphi\txt{-}\mathsf{ModFil}}}

% font
\newcommand\rma{{\mathrm{a}}}
\newcommand\rmb{{\mathrm{b}}}
\newcommand\rmc{{\mathrm{c}}}
\newcommand\rmd{{\mathrm{d}}}
\newcommand\rme{{\mathrm{e}}}
\newcommand\rmf{{\mathrm{f}}}
\newcommand\rmg{{\mathrm{g}}}
\newcommand\rmh{{\mathrm{h}}}
\newcommand\rmi{{\mathrm{i}}}
\newcommand\rmj{{\mathrm{j}}}
\newcommand\rmk{{\mathrm{k}}}
\newcommand\rml{{\mathrm{l}}}
\newcommand\rmm{{\mathrm{m}}}
\newcommand\rmn{{\mathrm{n}}}
\newcommand\rmo{{\mathrm{o}}}
\newcommand\rmp{{\mathrm{p}}}
\newcommand\rmq{{\mathrm{q}}}
\newcommand\rmr{{\mathrm{r}}}
\newcommand\rms{{\mathrm{s}}}
\newcommand\rmt{{\mathrm{t}}}
\newcommand\rmu{{\mathrm{u}}}
\newcommand\rmv{{\mathrm{v}}}
\newcommand\rmw{{\mathrm{w}}}
\newcommand\rmx{{\mathrm{x}}}
\newcommand\rmy{{\mathrm{y}}}
\newcommand\rmz{{\mathrm{z}}}
\newcommand\rmA{{\mathrm{A}}}
\newcommand\rmB{{\mathrm{B}}}
\newcommand\rmC{{\mathrm{C}}}
\newcommand\rmD{{\mathrm{D}}}
\newcommand\rmE{{\mathrm{E}}}
\newcommand\rmF{{\mathrm{F}}}
\newcommand\rmG{{\mathrm{G}}}
\newcommand\rmH{{\mathrm{H}}}
\newcommand\rmI{{\mathrm{I}}}
\newcommand\rmJ{{\mathrm{J}}}
\newcommand\rmK{{\mathrm{K}}}
\newcommand\rmL{{\mathrm{L}}}
\newcommand\rmM{{\mathrm{M}}}
\newcommand\rmN{{\mathrm{N}}}
\newcommand\rmO{{\mathrm{O}}}
\newcommand\rmP{{\mathrm{P}}}
\newcommand\rmQ{{\mathrm{Q}}}
\newcommand\rmR{{\mathrm{R}}}
\newcommand\rmS{{\mathrm{S}}}
\newcommand\rmT{{\mathrm{T}}}
\newcommand\rmU{{\mathrm{U}}}
\newcommand\rmV{{\mathrm{V}}}
\newcommand\rmW{{\mathrm{W}}}
\newcommand\rmX{{\mathrm{X}}}
\newcommand\rmY{{\mathrm{Y}}}
\newcommand\rmZ{{\mathrm{Z}}}
\newcommand\bfa{{\mathbf{a}}}
\newcommand\bfb{{\mathbf{b}}}
\newcommand\bfc{{\mathbf{c}}}
\newcommand\bfd{{\mathbf{d}}}
\newcommand\bfe{{\mathbf{e}}}
\newcommand\bff{{\mathbf{f}}}
\newcommand\bfg{{\mathbf{g}}}
\newcommand\bfh{{\mathbf{h}}}
\newcommand\bfi{{\mathbf{i}}}
\newcommand\bfj{{\mathbf{j}}}
\newcommand\bfk{{\mathbf{k}}}
\newcommand\bfl{{\mathbf{l}}}
\newcommand\bfm{{\mathbf{m}}}
\newcommand\bfn{{\mathbf{n}}}
\newcommand\bfo{{\mathbf{o}}}
\newcommand\bfp{{\mathbf{p}}}
\newcommand\bfq{{\mathbf{q}}}
\newcommand\bfr{{\mathbf{r}}}
\newcommand\bfs{{\mathbf{s}}}
\newcommand\bft{{\mathbf{t}}}
\newcommand\bfu{{\mathbf{u}}}
\newcommand\bfv{{\mathbf{v}}}
\newcommand\bfw{{\mathbf{w}}}
\newcommand\bfx{{\mathbf{x}}}
\newcommand\bfy{{\mathbf{y}}}
\newcommand\bfz{{\mathbf{z}}}
\newcommand\bfA{{\mathbf{A}}}
\newcommand\bfB{{\mathbf{B}}}
\newcommand\bfC{{\mathbf{C}}}
\newcommand\bfD{{\mathbf{D}}}
\newcommand\bfE{{\mathbf{E}}}
\newcommand\bfF{{\mathbf{F}}}
\newcommand\bfG{{\mathbf{G}}}
\newcommand\bfH{{\mathbf{H}}}
\newcommand\bfI{{\mathbf{I}}}
\newcommand\bfJ{{\mathbf{J}}}
\newcommand\bfK{{\mathbf{K}}}
\newcommand\bfL{{\mathbf{L}}}
\newcommand\bfM{{\mathbf{M}}}
\newcommand\bfN{{\mathbf{N}}}
\newcommand\bfO{{\mathbf{O}}}
\newcommand\bfP{{\mathbf{P}}}
\newcommand\bfQ{{\mathbf{Q}}}
\newcommand\bfR{{\mathbf{R}}}
\newcommand\bfS{{\mathbf{S}}}
\newcommand\bfT{{\mathbf{T}}}
\newcommand\bfU{{\mathbf{U}}}
\newcommand\bfV{{\mathbf{V}}}
\newcommand\bfW{{\mathbf{W}}}
\newcommand\bfX{{\mathbf{X}}}
\newcommand\bfY{{\mathbf{Y}}}
\newcommand\bfZ{{\mathbf{Z}}}
\newcommand\BA{{\mathbb{A}}}
\newcommand\BB{{\mathbb{B}}}
\newcommand\BC{{\mathbb{C}}}
\newcommand\BD{{\mathbb{D}}}
\newcommand\BE{{\mathbb{E}}}
\newcommand\BF{{\mathbb{F}}}
\newcommand\BG{{\mathbb{G}}}
\newcommand\BH{{\mathbb{H}}}
\newcommand\BI{{\mathbb{I}}}
\newcommand\BJ{{\mathbb{J}}}
\newcommand\BK{{\mathbb{K}}}
\newcommand\BL{{\mathbb{L}}}
\newcommand\BM{{\mathbb{M}}}
\newcommand\BN{{\mathbb{N}}}
\newcommand\BO{{\mathbb{O}}}
\newcommand\BP{{\mathbb{P}}}
\newcommand\BQ{{\mathbb{Q}}}
\newcommand\BR{{\mathbb{R}}}
\newcommand\BS{{\mathbb{S}}}
\newcommand\BT{{\mathbb{T}}}
\newcommand\BU{{\mathbb{U}}}
\newcommand\BV{{\mathbb{V}}}
\newcommand\BW{{\mathbb{W}}}
\newcommand\BX{{\mathbb{X}}}
\newcommand\BY{{\mathbb{Y}}}
\newcommand\BZ{{\mathbb{Z}}}
\newcommand\CA{{\mathcal{A}}}
\newcommand\CB{{\mathcal{B}}}
\newcommand\CC{{\mathcal{C}}}
\providecommand\CD{{\mathcal{D}}}
\newcommand\CE{{\mathcal{E}}}
\newcommand\CF{{\mathcal{F}}}
\newcommand\CG{{\mathcal{G}}}
\newcommand\CH{{\mathcal{H}}}
\newcommand\CI{{\mathcal{I}}}
\newcommand\CJ{{\mathcal{J}}}
\newcommand\CK{{\mathcal{K}}}
\newcommand\CL{{\mathcal{L}}}
\newcommand\CM{{\mathcal{M}}}
\newcommand\CN{{\mathcal{N}}}
\newcommand\CO{{\mathcal{O}}}
\newcommand\CP{{\mathcal{P}}}
\newcommand\CQ{{\mathcal{Q}}}
\newcommand\CR{{\mathcal{R}}}
\newcommand\CS{{\mathcal{S}}}
\newcommand\CT{{\mathcal{T}}}
\newcommand\CU{{\mathcal{U}}}
\newcommand\CV{{\mathcal{V}}}
\newcommand\CW{{\mathcal{W}}}
\newcommand\CX{{\mathcal{X}}}
\newcommand\CY{{\mathcal{Y}}}
\newcommand\CZ{{\mathcal{Z}}}
\newcommand\RA{{\mathrm{A}}}
\newcommand\RB{{\mathrm{B}}}
\newcommand\RC{{\mathrm{C}}}
\newcommand\RD{{\mathrm{D}}}
\newcommand\RE{{\mathrm{E}}}
\newcommand\RF{{\mathrm{F}}}
\newcommand\RG{{\mathrm{G}}}
\newcommand\RH{{\mathrm{H}}}
\newcommand\RI{{\mathrm{I}}}
\newcommand\RJ{{\mathrm{J}}}
\newcommand\RK{{\mathrm{K}}}
\newcommand\RL{{\mathrm{L}}}
\newcommand\RM{{\mathrm{M}}}
% \newcommand\RN{{\mathrm{N}}}
\newcommand\RO{{\mathrm{O}}}
\newcommand\RP{{\mathrm{P}}}
\newcommand\RQ{{\mathrm{Q}}}
\newcommand\RR{{\mathrm{R}}}
\newcommand\RS{{\mathrm{S}}}
\newcommand\RT{{\mathrm{T}}}
\newcommand\RU{{\mathrm{U}}}
\newcommand\RV{{\mathrm{V}}}
\newcommand\RW{{\mathrm{W}}}
\newcommand\RX{{\mathrm{X}}}
\newcommand\RY{{\mathrm{Y}}}
\newcommand\RZ{{\mathrm{Z}}}
\newcommand\msa{\mathscr{A}}
\newcommand\msb{\mathscr{B}}
\newcommand\msc{\mathscr{C}}
\newcommand\msd{\mathscr{D}}
\newcommand\mse{\mathscr{E}}
\newcommand\msf{\mathscr{F}}
\newcommand\msg{\mathscr{G}}
\newcommand\msh{\mathscr{H}}
\newcommand\msi{\mathscr{I}}
\newcommand\msj{\mathscr{J}}
\newcommand\msk{\mathscr{K}}
\newcommand\msl{\mathscr{L}}
\newcommand\msm{\mathscr{M}}
\newcommand\msn{\mathscr{N}}
\newcommand\mso{\mathscr{O}}
\newcommand\msp{\mathscr{P}}
\newcommand\msq{\mathscr{Q}}
\newcommand\msr{\mathscr{R}}
\newcommand\mss{\mathscr{S}}
\newcommand\mst{\mathscr{T}}
\newcommand\msu{\mathscr{U}}
\newcommand\msv{\mathscr{V}}
\newcommand\msw{\mathscr{W}}
\newcommand\msx{\mathscr{X}}
\newcommand\msy{\mathscr{Y}}
\newcommand\msz{\mathscr{Z}}
\newcommand\fa{{\mathfrak{a}}}
\newcommand\fb{{\mathfrak{b}}}
\newcommand\fc{{\mathfrak{c}}}
\newcommand\fd{{\mathfrak{d}}}
\newcommand\fe{{\mathfrak{e}}}
\newcommand\ff{{\mathfrak{f}}}
\newcommand\fg{{\mathfrak{g}}}
\newcommand\fh{{\mathfrak{h}}}
\newcommand\fii{{\mathfrak{i}}} % be careful about \fii
\newcommand\fj{{\mathfrak{j}}}
\newcommand\fk{{\mathfrak{k}}}
\newcommand\fl{{\mathfrak{l}}}
\newcommand\fm{{\mathfrak{m}}}
\newcommand\fn{{\mathfrak{n}}}
\newcommand\fo{{\mathfrak{o}}}
\newcommand\fp{{\mathfrak{p}}}
\newcommand\fq{{\mathfrak{q}}}
\newcommand\fr{{\mathfrak{r}}}
\newcommand\fs{{\mathfrak{s}}}
\newcommand\ft{{\mathfrak{t}}}
\newcommand\fu{{\mathfrak{u}}}
\newcommand\fv{{\mathfrak{v}}}
\newcommand\fw{{\mathfrak{w}}}
\newcommand\fx{{\mathfrak{x}}}
\newcommand\fy{{\mathfrak{y}}}
\newcommand\fz{{\mathfrak{z}}}
\newcommand\fA{{\mathfrak{A}}}
\newcommand\fB{{\mathfrak{B}}}
\newcommand\fC{{\mathfrak{C}}}
\newcommand\fD{{\mathfrak{D}}}
\newcommand\fE{{\mathfrak{E}}}
\newcommand\fF{{\mathfrak{F}}}
\newcommand\fG{{\mathfrak{G}}}
\newcommand\fH{{\mathfrak{H}}}
\newcommand\fI{{\mathfrak{I}}}
\newcommand\fJ{{\mathfrak{J}}}
\newcommand\fK{{\mathfrak{K}}}
\newcommand\fL{{\mathfrak{L}}}
\newcommand\fM{{\mathfrak{M}}}
\newcommand\fN{{\mathfrak{N}}}
\newcommand\fO{{\mathfrak{O}}}
\newcommand\fP{{\mathfrak{P}}}
\newcommand\fQ{{\mathfrak{Q}}}
\newcommand\fR{{\mathfrak{R}}}
\newcommand\fS{{\mathfrak{S}}}
\newcommand\fT{{\mathfrak{T}}}
\newcommand\fU{{\mathfrak{U}}}
\newcommand\fV{{\mathfrak{V}}}
\newcommand\fW{{\mathfrak{W}}}
\newcommand\fX{{\mathfrak{X}}}
\newcommand\fY{{\mathfrak{Y}}}
\newcommand\fZ{{\mathfrak{Z}}}

% a
\newcommand\ab{{\mathrm{ab}}}
\newcommand\ad{{\mathrm{ad}}}
\newcommand\Ad{{\mathrm{Ad}}}
\newcommand\adele{ad\'{e}le}
\newcommand\Adele{Ad\'{e}le}
\newcommand\adeles{ad\'{e}les}
\newcommand\adelic{ad\'{e}lic}
\newcommand\AJ{{\mathrm{AJ}}}
\newcommand\alb{{\mathrm{alb}}}
\newcommand\Alb{{\mathrm{Alb}}}
\newcommand\alg{{\mathrm{alg}}}
\newcommand\an{{\mathrm{an}}}
\newcommand\ann{{\mathrm{ann}}}
\newcommand\Ann{{\mathrm{Ann}}}
\DeclareMathOperator\Arcsin{Arcsin}
\DeclareMathOperator\Arccos{Arccos}
\DeclareMathOperator\Arctan{Arctan}
\DeclareMathOperator\arsh{arsh}
\DeclareMathOperator\Arsh{Arsh}
\DeclareMathOperator\arch{arch}
\DeclareMathOperator\Arch{Arch}
\DeclareMathOperator\arth{arth}
\DeclareMathOperator\Arth{Arth}
\DeclareMathOperator\arccot{arccot}
\DeclareMathOperator\Arg{Arg}
\newcommand\arith{{\mathrm{arith}}}
\newcommand\Art{{\mathrm{Art}}}
\newcommand\AS{{\mathrm{AS}}}
\newcommand\Ass{{\mathrm{Ass}}}
\newcommand\Aut{{\mathrm{Aut}}}
% b
\newcommand\Bun{{\mathrm{Bun}}}
\newcommand\Br{{\mathrm{Br}}}
\newcommand\bs{\backslash}
\newcommand\BWt{{\mathrm{BW}}}
% c
\newcommand\can{{\mathrm{can}}}
\newcommand\cc{{\mathrm{cc}}}
\newcommand\cd{{\mathrm{cd}}}
\DeclareMathOperator\ch{ch}
\newcommand\Ch{{\mathrm{Ch}}}
\let\char\relax
\DeclareMathOperator\char{char}
\DeclareMathOperator\Char{char}
\newcommand\Chow{{\mathrm{CH}}}
\newcommand\circB{{\stackrel{\circ}{B}}}
\newcommand\cl{{\mathrm{cl}}}
\newcommand\Cl{{\mathrm{Cl}}}
\newcommand\cm{{\mathrm{cm}}}
\newcommand\cod{{\mathrm{cod}}}
\DeclareMathOperator\coker{coker}
\DeclareMathOperator\Coker{Coker}
\DeclareMathOperator\cond{cond}
\DeclareMathOperator\codim{codim}
\newcommand\cont{{\mathrm{cont}}}
\newcommand\Conv{{\mathrm{Conv}}}
\newcommand\corr{{\mathrm{corr}}}
\newcommand\Corr{{\mathrm{Corr}}}
\DeclareMathOperator\coim{coim}
\DeclareMathOperator\coIm{coIm}
\DeclareMathOperator\corank{corank}
\DeclareMathOperator\covol{covol}
\newcommand\cris{{\mathrm{cris}}}
\newcommand\Cris{{\mathrm{Cris}}}
\newcommand\CRIS{{\mathrm{CRIS}}}
\newcommand\crit{{\mathrm{crit}}}
\newcommand\crys{{\mathrm{crys}}}
\newcommand\cusp{{\mathrm{cusp}}}
\newcommand\CWt{{\mathrm{CW}}}
\newcommand\cyc{{\mathrm{cyc}}}
% d
\newcommand\Def{{\mathrm{Def}}}
\newcommand\diag{{\mathrm{diag}}}
\newcommand\diff{\,\mathrm{d}}
\newcommand\disc{{\mathrm{disc}}}
\newcommand\dist{{\mathrm{dist}}}
\renewcommand\div{{\mathrm{div}}}
\newcommand\Div{{\mathrm{Div}}}
\newcommand\dR{{\mathrm{dR}}}
\newcommand\Drin{{\mathrm{Drin}}}
% e
\newcommand\End{{\mathrm{End}}}
\newcommand\ess{{\mathrm{ess}}}
\newcommand\et{{\text{\'{e}t}}}
\newcommand\etale{{\'{e}tale}}
\newcommand\Etale{{\'{E}tale}}
\newcommand\Ext{\mathrm{Ext}}
\newcommand\CExt{\mathcal{E}\mathrm{xt}}
% f
\newcommand\Fil{{\mathrm{Fil}}}
\newcommand\Fix{{\mathrm{Fix}}}
\newcommand\fppf{{\mathrm{fppf}}}
\newcommand\Fr{{\mathrm{Fr}}}
\newcommand\Frac{{\mathrm{Frac}}}
\newcommand\Frob{{\mathrm{Frob}}}
% g
\newcommand\Ga{\mathbb{G}_a}
\newcommand\Gm{\mathbb{G}_m}
\newcommand\hGa{\widehat{\mathbb{G}}_{a}}
\newcommand\hGm{\widehat{\mathbb{G}}_{m}}
\newcommand\Gal{{\mathrm{Gal}}}
\newcommand\gl{{\mathrm{gl}}}
\newcommand\GL{{\mathrm{GL}}}
\newcommand\GO{{\mathrm{GO}}}
\newcommand\geom{{\mathrm{geom}}}
\newcommand\Gr{{\mathrm{Gr}}}
\newcommand\gr{{\mathrm{gr}}}
\newcommand\GSO{{\mathrm{GSO}}}
\newcommand\GSp{{\mathrm{GSp}}}
\newcommand\GSpin{{\mathrm{GSpin}}}
\newcommand\GU{{\mathrm{GU}}}
% h
\newcommand\hg{{\mathrm{hg}}}
\newcommand\Hk{{\mathrm{Hk}}}
\newcommand\HN{{\mathrm{HN}}}
\newcommand\Hom{{\mathrm{Hom}}}
\newcommand\CHom{\mathcal{H}\mathrm{om}}
% i
\newcommand\id{{\mathrm{id}}}
\newcommand\Id{{\mathrm{Id}}}
\newcommand\idele{id\'{e}le}
\newcommand\Idele{Id\'{e}le}
\newcommand\ideles{id\'{e}les}
\let\Im\relax
\DeclareMathOperator\Im{Im}
\DeclareMathOperator\im{im}
\newcommand\Ind{{\mathrm{Ind}}}
\newcommand\cInd{{\mathrm{c}\textrm{-}\mathrm{Ind}}}
\newcommand\ind{{\mathrm{ind}}}
\newcommand\Int{{\mathrm{Int}}}
\newcommand\inv{{\mathrm{inv}}}
\newcommand\Isom{{\mathrm{Isom}}}
% j
\newcommand\Jac{{\mathrm{Jac}}}
\newcommand\JL{{\mathrm{JL}}}
% k
\newcommand\Katz{\mathrm{Katz}}
\DeclareMathOperator\Ker{Ker}
\newcommand\KS{{\mathrm{KS}}}
\newcommand\Kl{{\mathrm{Kl}}}
\newcommand\CKl{{\mathcal{K}\mathrm{l}}}
% l
\newcommand\lcm{\mathrm{lcm}}
\newcommand\length{\mathrm{length}}
\DeclareMathOperator\Li{Li}
\DeclareMathOperator\Ln{Ln}
\newcommand\Lie{{\mathrm{Lie}}}
\newcommand\lt{\mathrm{lt}}
\newcommand\LT{\mathcal{LT}}
% m
\newcommand\mex{{\mathrm{mex}}}
\newcommand\MW{{\mathrm{MW}}}
\renewcommand\mod{\, \mathrm{mod}\, }
\newcommand\mom{{\mathrm{mom}}}
\newcommand\Mor{{\mathrm{Mor}}}
\newcommand\Morp{{\mathrm{Morp}\,}}
% n
\newcommand\new{{\mathrm{new}}}
\newcommand\Newt{{\mathrm{Newt}}}
\newcommand\nd{{\mathrm{nd}}}
\newcommand\NP{{\mathrm{NP}}}
\newcommand\NS{{\mathrm{NS}}}
\newcommand\ns{{\mathrm{ns}}}
\newcommand\Nm{{\mathrm{Nm}}}
\newcommand\Nrd{{\mathrm{Nrd}}}
\newcommand\Neron{N\'{e}ron}
% o
\newcommand\Obj{{\mathrm{Obj}\,}}
\newcommand\odd{{\mathrm{odd}}}
\newcommand\old{{\mathrm{old}}}
\newcommand\op{{\mathrm{op}}}
\newcommand\Orb{{\mathrm{Orb}}}
\newcommand\ord{{\mathrm{ord}}}
% p
\newcommand\pd{{\mathrm{pd}}}
\newcommand\Pet{{\mathrm{Pet}}}
\newcommand\PGL{{\mathrm{PGL}}}
\newcommand\Pic{{\mathrm{Pic}}}
\newcommand\pr{{\mathrm{pr}}}
\newcommand\Proj{{\mathrm{Proj}}}
\newcommand\proet{\text{pro\'{e}t}}
\newcommand\Poincare{\text{Poincar\'{e}}}
\newcommand\Prd{{\mathrm{Prd}}}
\newcommand\prim{{\mathrm{prim}}}
% r
\newcommand\Rad{{\mathrm{Rad}}}
\newcommand\rad{{\mathrm{rad}}}
\DeclareMathOperator\rank{rank}
\let\Re\relax
\DeclareMathOperator\Re{Re}
\newcommand\rec{{\mathrm{rec}}}
\newcommand\red{{\mathrm{red}}}
\newcommand\reg{{\mathrm{reg}}}
\newcommand\res{{\mathrm{res}}}
\newcommand\Res{{\mathrm{Res}}}
\newcommand\rig{{\mathrm{rig}}}
\newcommand\Rig{{\mathrm{Rig}}}
\newcommand\rk{{\mathrm{rk}}}
\newcommand\Ros{{\mathrm{Ros}}}
\newcommand\rs{{\mathrm{rs}}}
% s
\newcommand\sd{{\mathrm{sd}}}
\newcommand\Sel{{\mathrm{Sel}}}
\newcommand\sep{{\mathrm{sep}}}
\DeclareMathOperator{\sgn}{sgn}
\newcommand\Sh{{\mathrm{Sh}}}
\DeclareMathOperator\sh{sh}
\newcommand\Sht{\mathrm{Sht}}
\newcommand\Sim{{\mathrm{Sim}}}
\newcommand\sign{{\mathrm{sign}}}
\newcommand\SK{{\mathrm{SK}}}
\newcommand\SL{{\mathrm{SL}}}
\newcommand\SO{{\mathrm{SO}}}
\newcommand\Sp{{\mathrm{Sp}}}
\newcommand\Spa{{\mathrm{Spa}}}
\newcommand\Span{{\mathrm{Span}}}
\DeclareMathOperator\Spec{Spec}
\newcommand\Spf{{\mathrm{Spf}}}
\newcommand\Spin{{\mathrm{Spin}}}
\newcommand\Spm{{\mathrm{Spm}}}
\newcommand\srs{{\mathrm{srs}}}
\newcommand\rss{{\mathrm{ss}}}
\newcommand\ST{{\mathrm{ST}}}
\newcommand\St{{\mathrm{St}}}
\newcommand\st{{\mathrm{st}}}
\newcommand\Stab{{\mathrm{Stab}}}
\newcommand\SU{{\mathrm{SU}}}
\newcommand\Sym{{\mathrm{Sym}}}
\newcommand\sub{{\mathrm{sub}}}
\newcommand\rsum{{\mathrm{sum}}}
\newcommand\supp{{\mathrm{supp}}}
\newcommand\Supp{{\mathrm{Supp}}}
\newcommand\Swan{{\mathrm{Sw}}}
\newcommand\suml{\sum\limits}
% t
\newcommand\td{{\mathrm{td}}}
\let\tanh\relax
\DeclareMathOperator\tanh{th}
\newcommand\tor{{\mathrm{tor}}}
\newcommand\Tor{{\mathrm{Tor}}}
\newcommand\tors{{\mathrm{tors}}}
\newcommand\tr{{\mathrm{tr}\,}}
\newcommand\Tr{{\mathrm{Tr}}}
\newcommand\Trd{{\mathrm{Trd}}}
\newcommand\TSym{{\mathrm{TSym}}}
\newcommand\tw{{\mathrm{tw}}}
% u
\newcommand\uni{{\mathrm{uni}}}
\newcommand\univ{\mathrm{univ}}
\newcommand\ur{{\mathrm{ur}}}
\newcommand\USp{{\mathrm{USp}}}
% v
\newcommand\vQ{{\breve \BQ}}
\newcommand\vE{{\breve E}}
\newcommand\Ver{{\mathrm{Ver}}}
\newcommand\vF{{\breve F}}
\newcommand\vK{{\breve K}}
\newcommand\vol{{\mathrm{vol}}}
\newcommand\Vol{{\mathrm{Vol}}}
% w
\newcommand\wa{{\mathrm{wa}}}
% z
\newcommand\Zar{{\mathrm{Zar}}}

\newcommand\ldb{\llbracket}
\newcommand\rdb{\rrbracket}
\newcommand\bchi{{\boldsymbol\chi}}
\newcommand\brho{{\boldsymbol\rho}}


\begin{document}

\section{指数和与 $L$ 函数的特征根}
\begin{frame}{指数和与 $L$ 函数}
设 $p$ 是素数, $\BF_q$ 是含有 $q=p^a$ 个元素的有限域.
设 $\psi:\BF_p\to \BQ(\zeta_p)$ 是一非平凡加性特征,
那么 $\psi_k=\psi\circ \Tr_{\BF_{q^k}/\BF_p}$ 是 $\BF_{q^k}$ 的加性特征.

对于多项式 $f(x)\in \BF_q[x]$, 定义\emph{指数和}
\[S_k(f):=\sum_{x\in\BF_{q^k}}\psi_k\bigl(f(x)\bigr)\in\BZ[\zeta_p]\]
以及\emph{$L$ 函数}
\[L(s,f):=\exp\left(\sum_k S_k(f)\frac{s^k}{k}\right)=\prod_{x\in\ov{\BF}_p}\Bigl(1-\psi_{\deg x}\bigl(f(x)\bigr)s^{\deg x}\Bigr)^{-1}.\]
我们关心,作为一个代数整数, $S_k(f)$ 的各种性质, 以及 $L$ 函数的各种性质.
\end{frame}


\begin{frame}{$L$ 函数的有理性}
\begin{theorem}[Dwork-Bombieri-Grothendick]
$L(s,f)$ 是有理函数.
\end{theorem} 
	\[L(s,f)=\frac{\prod_j (1-\beta_j s)}{\prod_i(1-\alpha_i s)},\quad \alpha_i\neq \beta_j
	\implies
	S_k(f)=\sum_i \alpha_i^k-\sum_j\beta_j^k.\]
称 $\alpha_i,\beta_j$ 为\emph{特征根}. 为了估计特征根, 我们需要 $\ell$ 进方法.

% 一般地, 设 $X$ 是一个概形, $X_\et$ 是(小) \etale\ site.
% 固定素数 $\ell\neq p$, 设 $E$ 是 $\BQ_\ell$ 的有限扩张.
% $X_\et$ 上系数为 $E$ 的 \emph{$\ell$ 进层}是这个 site 上的层 (实际上是 $\CO_E$ 有限商上的模层的逆向系, 态射扩充至 $E$), 使得在每个有限商上是可构造的. 若在每个有限商上是局部常值的, 则称之为 \emph{lisse} 的.
\end{frame}


\begin{frame}{$\ell$ 进方法}
设 $E\supseteq \mu_p$ 是一 $\ell\neq p$ 进域.
Deligne 在 ${\Ga}_{,\ov\BF_p}$ 上构造了一个局部自由秩 $1$ 的 $E$ 系数 $\ell$ 进 lisse 层 $\CF_\ell(f)$, 它满足
	\[L(s,f)=\prod_i \det\bigl(1-s \Frob,\RH^i_c\bigr)^{(-1)^{i+1}},\]
	\[S_k(f)=\sum_i (-1)^i\Tr(\Frob^k,\RH_c^i).\]
这里 $\Frob$ 是几何 Frobenius, $\RH_c^i=\RH_c^i\bigl({\Ga}_{,\ov\BF_p},\CF_\ell(f)\bigr)$ 是紧支撑上同调.
\end{frame}


\begin{frame}{$\ell$ 进方法(续)}
记 $\omega_{ij}$ 为 $\Frob$ 在 $\RH^i_c$ 上的特征值, 则
	\[S_k(f)=\sum_{ij}(-1)^i \omega_{ij}^k.\]

\begin{theorem}[Deligne]
$\omega_{ij}$ 是代数整数, 且存在整数 $0\le r_{ij}\le i$ 使得它的所有 $\BQ$ 共轭的绝对值均为 $q^{r_{ij}/2}$.
\end{theorem}
我们把 $r_{ij}$ 就叫做对应特征根 $\omega_{ij}$ 的权.

记 $b_i=\dim_{E}\RH^i_c$ 为 Betti 数, 
则
	\[|S_k(f)|\le \sum_i b_i q^{ki/2}.\]
\end{frame}



\begin{frame}{一般情形}
一般地, 设
\begin{itemize}
\item $V\subseteq \BA^N$ 是 $\BF_q$ 上的闭子簇,
\item $f\in \BF_q[V]$, $g\in\BF_q[V]^\times$,
\item $\chi$ 是 $\BF_q^\times$ 上乘性特征.
\end{itemize}
定义指数和和 $L$ 函数
	\[S_k=\sum_{x\in V(\BF_{q^k})}\psi_{k\log_p q}\bigl(f(x)\bigr)\chi\Bigl(\bfN_{\BF_{q^n}/\BF_q}\bigl(g(x)\bigr)\Bigr),\quad
	L(s,V,f)=\exp\left(\sum_k S_k\frac{s^k}k\right).\]
前述结论依然成立.
\begin{theorem}[Bombieri1978]
特征根个数不超过 $(4\max\set{\deg V+1,\deg f}+5)^{2N+1}$.
\end{theorem}
\end{frame}


\begin{frame}{Kloosterman 和}

\begin{example}[Deligne SGA4$\half$, Serre1977]
设 $\bchi=\set{\chi_1,\dots,\chi_n}$ 是 $\BF_q^\times$ 上的 $n$ 个特征, $a\in\BF_q^\times$.
定义 \emph{Kloosterman 和}
	\[\Kl_k=\sum_{x_1\cdots x_n=a\atop x_i\in\BF_{q^k}}\chi_1(x_1)\cdots\chi_n(x_n)\psi_k(x_1+\cdots+x_n).\]
它是 $V=V(X_1\cdots X_n-a)$ 上 $f=X_1+\cdots+X_n$ 的指数和.

此时 $L(s,V,f)^{(-1)^n}$ 是 $n$ 次多项式, 即
\[\Kl_k=(-1)^{n-1}\sum_{i=1}^n \omega_i^k,\]
且特征根权均为 $n-1$.
因此 $|\Kl_k|\le nq^{(n-1)k/2}$.
\end{example}
\end{frame}


\begin{frame}
\begin{example}[Serre1977]
设 $X$ 是一几何不可约仿射光滑曲线, $\wh X$ 为其对应的射影曲线, $X_\infty=\wh X-X$.
设 $X'\to X$ 是方程 $y^p-y=f(x)$ 给出的 $\BZ/p\BZ$ 平展覆盖, 并延拓至 $\wh X'\to \wh X$.
对于 $P\in X_\infty$, 如果该覆盖在 $P$ 处非分歧, 记 $n_P=0$; 否则记
\[n_P=1-\sup_{\varphi\in\BF_q(\wh X)} v_P(f-\varphi^p+\varphi)\ge 2.\]
那么
\begin{itemize}
	\item $L(s,X,f)$ 是多项式; $b_i=0, \forall i\neq 1$;
	\item $(X,f)$ 权全为 $1 \iff n_P\neq 0,\forall P$, 此时 $b_1=2g-2+\sum n_P\deg P$.
\end{itemize}
\end{example}
\end{frame}
	

\begin{frame}
\begin{example}[Serre1977]
对于仿射平面 $\BA^n$, $f\in\BF_q[x_1,\dots,x_n]$ 是 $d$ 次多项式且 $p\nmid d$, 有
\begin{itemize}
	\item $L(s,X,f)^{(-1)^{n-1}}$ 是多项式; $b_i=0,\forall i\neq n$;
	\item $(X,f)$ 权全为 $n$, $b_n=(d-1)^r$.
\end{itemize}
此时 $|S_k|\le (d-1)^nq^{nk/2}$.
\end{example}
\end{frame}
	

\section{$L$ 函数的牛顿折线}


\begin{frame}{指数和的变化}
现在我们来考虑指数和的 $p$ 进性质.

设 $f\in\BF_q[x]$ 是 $d$ 次多项式.
我们可以考虑更一般点. 设
\begin{itemize}
\item $\psi_m:\BZ_p\to\BC_p^\times$ 是一个阶为 $p^m$ 的加性特征;
\item $\omega^{-u}:\BF_q^\times\to\BC_p^\times$ 是一个乘性特征, 其中 $\omega$ 是 Teichm\"uller 提升, $0\le u\le q-2$.
\end{itemize}
定义
	\[
		S_{k,u}(f,\psi_m)=\sum_{x\in\BF_{q^k}^\times}\psi_m\left(\Tr_{\BQ_{q^k}/\BQ_p}\bigl(\hat f(\hat x)\bigr)\right)\omega^{-u}\left(\Nm_{\BF_{q^k}/\BF_q}(x)\right),
	\]
	\[
		L_u(s,f,\psi_m)=\exp\left(\sum_{k=1}^\infty S_{k,u}(f,\psi_m)\frac{s^m}m\right).
	\]
\end{frame}

\begin{frame}{$L$ 函数是多项式}
\begin{theorem}[Adolphson-Sperber, 李文卿, 刘春雷-魏达盛, 刘春雷]
如果 $p\nmid d=\deg f$, 则 $L_u(s,f,\psi_m)$ 是次数为 $p^{m-1}d$ 的多项式.
\end{theorem} 

记
	\[L_u(s,f,\psi_m)=\sum_{n=0}^{p^{m-1}d} a_n s^n=\prod_i(1-\alpha_i s),
	\quad S_{u,k}(f)=-\sum_i \alpha_i^k.\]
为了了解 $S_{u,k}(f)$ 的 $p$ 进性质, 我们需要了解 $\alpha_i$ 的赋值.
而它们正是该 $L$ 函数的牛顿折线的斜率, 其中牛顿折线是指所有
	\[\bigl(n,v_p(a_n)\bigr)\]
的下凸包.
\end{frame}


\begin{frame}{$T$ 进指数和和 $T$ 进 $L$ 函数}
为了统一考虑不同 $m$ 对应的牛顿折线, 我们引入 $T$ 进指数和和 $T$ 进 $L$ 函数:
	\[	S_{k,u}(f,T)=\sum_{x\in\BF_{q^k}^\times}(1+T)^{\Tr_{\BQ_{q^k}/\BQ_p}(\hat f(\hat x))}\omega^{-u}\left(\Nm_{\BF_{q^k}/\BF_q}(x)\right), \]
	\[
		L_u(s,f,T)=\exp\left(\sum_{k=1}^\infty S_{k,u}(f,T)\frac{s^k}{k}\right)\in 1+s\BZ_q\ldb T\rdb\ldb s\rdb.
	\]
我们有 $L_u(s,f,\psi_m)=L_u(s,f,\pi_m)$, 其中 $\pi_m=\psi_m(1)-1$.

定义特征函数
	\[C_u(s,f,T)=\prod_{j=0}^\infty L_u(q^j s,f,T)\in 1+s\BZ_q\ldb T\rdb\ldb s\rdb,\]
则
	\[	L_u(s,f,T)=\frac{C_u(s,f,T)}{C_u(qs,f,T)}.\]
\end{frame}


\begin{frame}{牛顿折线的关系}
记
\begin{itemize}
\item $\NP_{u,m}(f)=C_u(s,f,\pi_m)$ 的 $\pi_m^{a(p-1)}$ 进牛顿折线(不依赖 $\psi_m$), $a=\log_p q$;
\item $\NP_{u,T}(f)=C_u(s,f,T)$ 的 $T^{a(p-1)}$ 进牛顿折线.
\item $H_{[0,d],u}^\infty$ 为扭霍奇折线, 其斜率为 $\tfrac{n}{d}+\tfrac{1}{bd(p-1)}\sum_{k=1}^b u_k,\ n\in\BN,$
其中 $b$ 是满足 $p^bu\equiv u\bmod{(q-1)}$ 的最小正整数,
	\[u=u_0+u_1p+\cdots+u_{a-1}p^{a-1},\ 0\le u_i\le p-1.\]
\end{itemize}
这样规范化后的牛顿折线满足
	\[\NP_{u,m}(f)\ge \NP_{u,T}(f)\ge H_{[0,d],u}^\infty.\]
由定义可知 $\NP_{u,m}(f)$ 完全由它在 $[0,d-1]$ 上的值决定.
\end{frame}



\begin{frame}{$p$ 进 Artin-Hasse 函数}
不妨设
\[f(x)=\sum_{i=1}^d a_ix^i.\]
我们需要 $T$ 进 Dwork 迹公式来计算牛顿折线. 定义
	\[
	E(X)=\exp\left(\sum_{i=0}^\infty p^{-i}X^{p^i}\right)=\sum_{n=0}^\infty \lambda_n X^n\in\BZ_p\ldb X\rdb,
	\]
	\[
	E_f(X)=\prod_{i=1}^d E(\pi \hat a_i X^i)=\sum_{n=0}^\infty \gamma_n X^n,
	\]
则
	\[\gamma_k=\sum\pi^{x_1+\cdots+x_d}\prod_{i=1}^d\lambda_{x_i}\hat a_i^{x_i},\]
其中 $(x_1,\dots,x_d)$ 取遍 $\sum i x_i=k$ 的所有非负整数解.
\end{frame}

\begin{frame}{$T$ 进 Dwork 半线性算子}
定义
	\[\CB_u=\Bigl\{\sum_{v\in M_u} b_v \pi^{\frac{v}{d}}X^v\mid b_v\in\BZ_q\ldb\pi^{\frac{1}{d(q-1)}}\rdb\ra 0 (\pi\text{进})\Bigr\},\ M_u=\tfrac{u}{q-1}+\BN,\]
	\[\psi:\CB_u\lra \CB_{p^{-1}u},\
		\sum_{v\in M_u} b_v X^v\longmapsto \sum_{v\in M_{p^{-1}u}} b_{pv}X^v,\]
则
	\[\Psi:=\sigma^{-1}\circ\psi\circ E_f: \CB_u\to \CB_{p^{-1}u}\]
是一个半线性算子, 其中 $\sigma\in \Gal(\BQ_q/\BQ_p)$ 是 Frobenius.
那么它定义了 $\CB:=\bigoplus_{i=0}^{b-1}\CB_{p^iu}$ 上的算子, 且 $\Psi^a$ 是 $\BZ_q\ldb\pi^{\frac{1}{d(q-1)}}\rdb$ 线性的.
\end{frame}

\begin{frame}{$T$ 进 Dwork 迹公式}
\begin{theorem}
我们有
	\[C_u(s,f,T)=\det\Bigl(1-\Psi^a s \mid \CB_u/\BZ_q\ldb\pi^{\frac{1}{d(q-1)}}\rdb\Bigr).\]
因此 $C_u(s,f,T)$ 的 $T$ 进牛顿折线是
	\[\Bigl(n,\frac1b\ord_T(c_{abn})\Bigr),\ n\in\BN,\]
的凸包, 其中
	\[
	\det\Bigl(1-\Psi s\mid\CB/\BZ_p\ldb\pi^{\frac{1}{d(q-1)}}\rdb\Bigr)=\sum_{i=0}^\infty (-1)^n c_n s^n.
	\]
\end{theorem}
\end{frame}


\begin{frame}{矩阵表达}
记 $s_k\equiv p^k u\bmod{q-1}$, $0\le s_k\le q-2$.
设 $\xi_1,\dots,\xi_a$ 为 $\BQ_q/\BQ_p$ 的一组正规基, 则
	\[\Bigl\{\xi_v(\pi^{\frac1d}X)^{\frac{s_k}{q-1}+i}\Bigr\}_{(i,v,k)\in\BN\times I_a\times I_b}\]
是 $\CB/\BZ_p\ldb\pi^{\frac{1}{d(q-1)}}\rdb$ 的一组基, 对应的矩阵为
	\[
	\Gamma=\Bigl(\gamma_{(v,\frac{s_k}{q-1}+i),(w,\frac{s_\ell}{q-1}+j)}\Bigr)_{\BN\times I_a\times I_b}=\begin{pmatrix}
		0&\Gamma^{(1)}&0&\cdots&0\\
		0&0&\Gamma^{(2)}&\cdots&0\\
		\vdots&\vdots&\vdots&\ddots&\vdots\\
		0&0&0&\cdots&\Gamma^{(b-1)}\\
		\Gamma^{(b)}&0&0&\cdots&0
	\end{pmatrix}.
	\]
因此 $c_{bn}=\sum_{A\in\CA_n} \det(A)$, $\CA_n$ 为全体 $bn$ 阶主子式, 且 $A^{(k)}=A\cap \Gamma^{(k)}$ 均为 $n$ 阶.
\end{frame}

\begin{frame}{进一步化归}
我们有
\vspace{-2mm}
	\[\xi_w^{\sigma^{-1}}{\color{red}{\gamma_{(\frac{s_{k-1}}{q-1}+i,\frac{s_k}{q-1}+j)}}}^{\sigma^{-1}}=\sum_{u=1}^a {\color{red}{\gamma_{(v,\frac{s_{k-1}}{q-1}+i),(w,\frac{s_k}{q-1}+j)}}}\xi_v,\]
\vspace{-2mm}
其中
\vspace{-2mm}
	\[\gamma_{\color{red}{(\frac{s_{k-1}}{q-1}+i,\frac{s_k}{q-1}+j)}}
	=\pi^{\frac{s_k-s_{k-1}}{d(q-1)}+\frac{j-i}{d}}{\color{red}{\gamma_{pi-j+u_{-k}}}}.\]
于是
	\[
	\begin{split}
	&\ord_\pi\left(\gamma_{(v,\frac{s_{k-1}}{q-1}+i),(w,\frac{s_k}{q-1}+j)}\right)\ge \ord_\pi\left(\gamma_{(\frac{s_{k-1}}{q-1}+i,\frac{s_k}{q-1}+j)}\right)\\
	=&\frac{s_k-s_{k-1}}{d(q-1)}+\frac{j-i}{d}+\phi(pi-j+u_{-k}),
	\end{split}
	\]
其中 $\phi(n)=\min\set{x_1+\cdots+x_d\mid \sum ix_i=n,x_i\ge0}\in\BN\cup\set{+\infty}.$
\end{frame}


\begin{frame}{二项式情形的已知结果}
对于一般的多项式, 上述方法可以得到牛顿折线的下界.
对于二项式情形, 这个下界是否能否达到, 取决于对应赋值项的系数是否为零.

设 $f(x)=x^d+\lambda x^e$ 的情形. 由于 $(d,e)>1$ 时可以化归到扭的情形, 我们不妨设 $(d,e)=1$.
此时最低赋值项为
	\[\pm\Nm\left(\prod_{k=1}^b \hat\lambda^{(*)} h_{n,k}\right),\quad
	h_{n,k}:=\sum_{\tau\in S_{u_k,n}^\circ}\sgn(\tau)\prod_{i=0}^n\frac{1}{x_{u_k,i}^\tau!y_{u_k,i}^\tau!},
	\]
其中 $S_{u_k,n}^\circ$ 是 $\set{0,1,\dots,n}$ 上满足 $e^{-1}(pi-\tau(i)+u_k)\mod d$ 所有最小非负剩余之和达到最小的置换全体,
	\[dx_{u_k,i}^\tau+ey_{u_k,i}^\tau=pi-\tau(i)+u_k,0\le y_{u_k,i}^\tau\le d-1.\]
因此当且仅当所有的 $h_{n,k}\in\BZ_p^\times$ 时,
	\[\NP_{u,m}(f)=\NP_{u,T}(f)\]
达到应有的下界.
\end{frame}


\begin{frame}{二项式情形的已知结果}
如下情形是已知的 ($p\gg0$):
\begin{itemize}
\item $u=0$:
\begin{itemize}
\item $p\equiv 1\bmod d$, 此时 $\NP_{u,m}(f)=H_{[0,d],u}^\infty$.
\item $e=1$, 有很多人计算过, 不在此列举.
\item $e=d-1,p\equiv -1\bmod d$, 欧阳毅-张 2016.
\item $e=2,p\equiv 2\bmod d$, Zhang Qingjie-牛传择 2021.
\end{itemize}
\item 任意 $u$:
\begin{itemize}
	\item $e=1$, 刘春雷-牛传择 2011.
	\item $e=d-1$, 张 2022.
\end{itemize}
\end{itemize}
\end{frame}


\begin{frame}{二项式情形的已知结果}
例如, 当 $e=d-1$ 时, 若 $p>(d^2-d-1)\mathrm{order}(\omega^{-u})$, 我们有
	\[\begin{split}
	&h_{n,k}\equiv\det\left(\frac{1}{(-d^{-1}ev_i+u_k(1-d^{-1}e)-j)!(v_i+j)!}\right)\\
\equiv &\prod_{i=0}^n \frac{\left(d^{-1}e(i-t)+t\right)_i}{\bigl(-d^{-1}ev_i+u_k(1-d^{-1}e)\bigr)!\cdot(v_i+n)!} \cdot \prod_{0\le i<j\le n}(v_i-v_j)\not\equiv 0\mod p.
\end{split}\]
因此此时 $\NP_{u,m}(f)=\NP_{u,T}(f)=P_{u,e,d}.$
\end{frame}



\section{指数和的生成域}



\begin{frame}{Swan 导子}
为了刻画 lisse 层的一些性质, 我们需要 Swan 导子的概念.
设 $K$ 是完备离散赋值域, $I^{(x)},x\ge 0$ 为其高阶分歧群. 
对于分歧群 $P$ 的 $E$ 表示 $M$, 我们有满足如下性质的分解 $M=\oplus M(x)$, 
	\[M(0)=M^P,\quad M(x)^{I^{(x)}}=0,\quad M(x)^{I(y)}=M(x), \ y>x>0.\]
称 $M(x)\neq 0$ 的 $x$ 为 $M$ 的断点.
定义 $M$ 的 \emph{Swan 导子}为
	\[\Swan(M)=\sum x\dim M(x).\]
它总是一个整数.
\end{frame}



\begin{frame}{曲线}
令 $C$ 是特征 $p$ 完全域 $\BF$ 上一射影光滑几何连通代数曲线, $K=\BF(C)$ 为其函数域.
对于任意闭点 $x\in C(\BF)$, 我们有完备化 $K_x$.

对于非空开集 $U\subset C$, 我们有阿贝尔范畴等价
\begin{align*}
	\set{\text{$U$ 上的 lisse $E$ 层}}&\lra \cRep_E^c \pi_1(U,\ov\eta)\\
	\CF&\lto \CF_{\ov\eta}.
\end{align*}
	
由于基本群 $\pi_1(U,\ov\eta)$ 是伽罗瓦群 $\Gal(\ov K/K)$ 的商, 因此分解群 $D_x$ 作用在 $\CF_{\ov\eta}$ 上.
于是我们可以定义 $\CF$ 在 $x$ 处的 Swan 导子.
对于 $x\in U$, 由于惯性群 $I_x$ 作用平凡, 因此 $\Swan_x(\CF)=0$.

我们将会取 $C=\BP^1$ and $U=\Gm$.
\end{frame}



\begin{frame}{一个猜想}
\begin{conjecture}
若 $p$ 相对 $d$ 和 $\omega^{-u}$ 的阶都很大, 则 $\NP_{u,m}(f)=\NP_{u,T}(f)=P_{u,e,d}$.
\end{conjecture}
\end{frame}



{
	\setnaviboxempty
	\begin{frame}
		\begin{center}
			\huge \textbf{谢~谢!}
		\end{center}
	\end{frame}
}
\end{document}


