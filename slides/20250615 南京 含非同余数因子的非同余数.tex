\documentclass[handout,aspectratio=169]{ctexbeamer}
\usepackage{../latex/bamboo}
\definecolor{main}{RGB}{0,0,224}%
\definecolor{second}{RGB}{224,0,0}%
\definecolor{third}{RGB}{112,0,112}%
\definecolor{fourth}{RGB}{0,128,0}%
\definecolor{fifth}{RGB}{255,128,0}%

% Black Theme
% \definecolor{main}{RGB}{0,0,0}%
% \definecolor{second}{RGB}{0,0,0}%
% \definecolor{third}{RGB}{0,0,0}%
% \definecolor{fourth}{RGB}{0,0,0}%
% \definecolor{fifth}{RGB}{0,0,0}%

% Doremi Theme
% \definecolor{main}{RGB}{73,119,213}%
% \definecolor{second}{RGB}{220,60,105}%
% \definecolor{third}{RGB}{122,89,166}%
% \definecolor{fourth}{RGB}{229,143,52}%
% \definecolor{fifth}{RGB}{244,220,10}%

\usepackage{bm}
\usepackage{extarrows}
\usepackage{mathrsfs}
\usepackage{stmaryrd}
\usepackage{multirow}
\usepackage{tikz}
\usepackage{color,calc}
\usepackage{caption}
\usepackage{subcaption}
\usepackage{booktabs}
% \usepackage{wrapfig}
% 文本色
\renewcommand\emph[1]{{\color{main}{\bf #1}}}
\ifcsundef{alert}{
  \newcommand{\alert}[1]{\textcolor{second}{\bf #1}}
}{}
% 非考试内容
\newcommand{\noexer}{\hfill\mdseries\itshape\color{black}\small 非考试内容}
% 枚举数字引用标志
\newcommand\enumnum[1]{{\mdseries\upshape\textcolor{main}{(#1)}}}
% 缩进
\renewcommand{\indent}{\hspace*{1em}}
\setlength{\parindent}{1em}
% 减少公式垂直间距, 配合\endgroup
\newcommand\beqskip[1]{\begingroup\abovedisplayskip=#1\belowdisplayskip=#1\belowdisplayshortskip=#1}

\RequirePackage{tasks}
\settasks{label-format=\textcolor{main},label={(\arabic*)},label-width=1.5em}
% 选择题选项
\NewTasksEnvironment[label={\upshape(\Alph*)},label-width=1.5em]{taskschoice}[()]
% 一行显示多题答案
\NewTasksEnvironment[label={\arabic*.},label-width=1.5em]{tasksans}[()]
% TIKZ 设置
\usetikzlibrary{
	quotes,
	shapes.arrows,
	arrows.meta,
	positioning,
	shapes.geometric,
	overlay-beamer-styles,
	patterns,
	calc,
	angles,
	decorations.pathreplacing,
	backgrounds % 背景边框
}
\tikzset{
	background rectangle/.style={semithick,draw=fourth,fill=white,rounded corners},
  % arrow
	cstra/.style      ={-Stealth},        % right arrow
	cstla/.style      ={Stealth-},        % left arrow
	cstlra/.style     ={Stealth-Stealth}, % left-right arrow
	cstwra/.style     ={-Straight Barb},  % wide ra
	cstwla/.style     ={Straight Barb-},
	cstwlra/.style    ={Straight Barb-Straight Barb},
	cstnra/.style      ={-Latex, line width=0.1cm},
	cstmra/.style      ={-Latex, line width=0.05cm},
	cstmlra/.style     ={Latex-Latex, line width=0.05cm},
	cstaxis/.style        ={-Stealth, thick}, %坐标轴
  % curve
	cstcurve/.style       ={very thick}, %一般曲线
	cstdash/.style        ={thick, dash pattern= on 0.2cm off 0.05cm}, %虚线
  % dot
	cstdot/.style         ={radius=.08}, %实心点
	cstdote/.style        ={radius=.07, fill=white}, %空心点
  % fill
	cstfill/.style       ={fill=black!10},
	cstfille/.style      ={pattern=north east lines, pattern color=black},
	cstfill1/.style       ={fill=main!20},
	cstfille1/.style      ={pattern=north east lines, pattern color=main},
	cstfill2/.style        ={fill=second!20},
	cstfille2/.style       ={pattern=north east lines, pattern color=second},
	cstfill3/.style        ={fill=third!20},
	cstfille3/.style       ={pattern=north east lines, pattern color=third},
	cstfill4/.style        ={fill=fourth!20},
	cstfille4/.style       ={pattern=north east lines, pattern color=fourth},
	cstfill5/.style        ={fill=fifth!20},
	cstfille5/.style       ={pattern=north east lines, pattern color=fifth},
  % node
	cstnode/.style        ={fill=white,draw=black,text=black,rounded corners=0.2cm,line width=1pt},
	cstnode1/.style       ={fill=main!15,draw=main!80,text=black,rounded corners=0.2cm,line width=1pt},
	cstnode2/.style       ={fill=second!15,draw=second!80,text=black,rounded corners=0.2cm,line width=1pt},
	cstnode3/.style       ={fill=third!15,draw=third!80,text=black,rounded corners=0.2cm,line width=1pt},
	cstnode4/.style       ={fill=fourth!15,draw=fourth!80,text=black,rounded corners=0.2cm,line width=1pt},
	cstnode5/.style       ={fill=fifth!15,draw=fifth!80,text=black,rounded corners=0.2cm,line width=1pt}
}
\renewcommand{\labelenumi}{{\upshape(\arabic{enumi})}}
\newcommand\enumnum[1]{{\textcolor{fourth}{\mdseries\upshape{(#1)}}}}
\newcommand\peq{\mathrel{\phantom{=}}} % 用于对齐的等号幻影

\usepackage{tikz}
\usepackage{color,calc}
% TIKZ 设置
\usetikzlibrary{
	quotes,
	shapes.arrows,
	arrows.meta,
	positioning,
	shapes.geometric,
	patterns,
	calc,
	angles,
	decorations.pathreplacing,
	backgrounds % 背景边框
}
\tikzset{
	background rectangle/.style={semithick,draw=fourth,fill=white,rounded corners},
  % arrow
	cstra/.style      ={-Stealth},        % right arrow
	cstla/.style      ={Stealth-},        % left arrow
	cstlra/.style     ={Stealth-Stealth}, % left-right arrow
	cstwra/.style     ={-Straight Barb},  % wide ra
	cstwla/.style     ={Straight Barb-},
	cstwlra/.style    ={Straight Barb-Straight Barb},
	cstnarrow/.style      ={-Latex, line width=0.1cm}, %文本框间箭头
	cstaxis/.style        ={-Stealth, thick}, %坐标轴
  % curve
	cstcurve/.style       ={very thick}, %一般曲线
	cstdash/.style        ={thick, dash pattern= on 0.2cm off 0.05cm}, %虚线
  % dot
	cstdot/.style         ={radius=.08}, %实心点
	cstdote/.style        ={radius=.07, fill=white}, %空心点
  % fill
	cstfill/.style       ={fill=black!10},
	cstfille/.style      ={pattern=north east lines, pattern color=black},
	cstfill1/.style       ={fill=main!20},
	cstfille1/.style      ={pattern=north east lines, pattern color=main},
	cstfill2/.style        ={fill=second!20},
	cstfille2/.style       ={pattern=north east lines, pattern color=second},
	cstfill3/.style        ={fill=third!20},
	cstfille3/.style       ={pattern=north east lines, pattern color=third},
	cstfill4/.style        ={fill=fourth!20},
	cstfille4/.style       ={pattern=north east lines, pattern color=fourth},
	cstfill5/.style        ={fill=fifth!20},
	cstfille5/.style       ={pattern=north east lines, pattern color=fifth},
  % node
	cstnode/.style        ={fill=white,draw=black,text=black,rounded corners=0.2cm,line width=1pt},
	cstnode1/.style       ={fill=main!15,draw=main!80,text=black,rounded corners=0.2cm,line width=1pt},
	cstnode2/.style       ={fill=second!15,draw=second!80,text=black,rounded corners=0.2cm,line width=1pt},
	cstnode3/.style       ={fill=third!15,draw=third!80,text=black,rounded corners=0.2cm,line width=1pt},
	cstnode4/.style       ={fill=fourth!15,draw=fourth!80,text=black,rounded corners=0.2cm,line width=1pt},
	cstnode5/.style       ={fill=fifth!15,draw=fifth!80,text=black,rounded corners=0.2cm,line width=1pt}
}
\ExplSyntaxOn
\cs_new_protected:Npn \fpstepfromto#1#2#3 
  {% from, to, nums
    \fp_step_inline:nnnn {#1} { (#2-(#1))/(#3-1)*0.99 } {#2}
  }
\pgfmathdeclarefunction{nrand}{0}
  {% \tex_normaldeviate:D 生成均值为 0,标准差为 10000 的随机整数
    \tl_set:Nx \pgfmathresult { \fp_eval:n { \tex_normaldeviate:D/10000 } }
  }
\pgfmathdeclarefunction{rdv}{0}{\pgfmathparse{1+nrand/100}}
\ExplSyntaxOff
\newcommand{\randpts}[3][10]{
  \foreach\i in {0,1,...,#1}{
    \pgfmathparse{rdv}\let\rdv\pgfmathresult
    \coordinate (\i) at ({#2*\rdv*cos(360/#1*\i)},{#3*\rdv*sin(360/#1*\i)});
  }
}
\newcommand{\randep}[2]{
  \randpts{#1}{#2}
  \filldraw[cstcurve,main,cstfill3,smooth] plot coordinates {(0) (1) (2) (3) (4) (5) (6) (7) (8) (9) (0)};
}

\let\question\relax
\elegantnewtheorem{question}{问题}{defstyle}{que}

\newfontface\cmunrm{cmunrm.otf}
\newcommand\cmu[1]{{\cmunrm{#1}}}
\newcommand{\alert}[1]{\textcolor{main}{\bf #1}}
\renewcommand{\emph}[1]{\textcolor{second}{\bf #1}}
\newcommand{\cnen}[2]{{\kaishu$\overset{\text{{#2}}}{\text{#1}}$}}
\newcommand{\nounen}[2]{{\color{second}\kaishu\cnen{#1}{#2}}\index{{#1}}}
\newcommand{\noun}[1]{{\color{second}\kaishu #1}\index{{#1}}}
\newcommand{\nouns}[2]{{\color{second}\kaishu #1}\index{{#2}}}
\newcommand{\nounsen}[3]{{\color{second}\kaishu\cnen{#1}{#2}}\index{{#3}}}

\setcounter{tocdepth}{2}
\newfontfamily\couriernew{Courier New}
\lstset{language=[LaTeX]TeX,
	basicstyle=\couriernew,
  morekeywords={AUTHOR, KEY, TITLE, YEAR, PAGES, HOWPUBLISHED, URL, LANGUAGE},
  keywordstyle=\color{winered}
}




\RequirePackage{extarrows}
\usepackage[normalem]{ulem}
\NewDocumentCommand\fillblank{O{1cm} O{0cm} m}{\uline{\makebox[#1]{\raisebox{#2}{#3}}}}
\newcommand\fillbrace[1]{{(\nolinebreak\hspace{0.5em minus 0.5em}{#1}\hspace{0.5em minus 0.5em}\nolinebreak)}}
\newcommand\resizet[1]{\resizebox{!}{#1\baselineskip}}
\newcommand{\trueex}{$\checkmark$}
\newcommand{\falseex}{$\times$}

% arrows
\newcommand\ra{\rightarrow}
\newcommand\lra{\longrightarrow}
\newcommand\la{\leftarrow}
\newcommand\lla{\longleftarrow}
\newcommand\sqra{\rightsquigarrow}
\newcommand\sqlra{\leftrightsquigarrow}
\newcommand\inj{\hookrightarrow}
\newcommand\linj{\hookleftarrow}
\newcommand\surj{\twoheadrightarrow}
\newcommand\simto{\stackrel{\sim}{\longrightarrow}}
\newcommand\sto[1]{\stackrel{#1}{\longrightarrow}}
\newcommand\lsto[1]{\stackrel{#1}{\longleftarrow}}
\newcommand\xto{\xlongrightarrow}
\newcommand\xeq{\xlongequal}
\newcommand\luobida{\xeq{\text{洛必达}}}
\newcommand\djwqx{\xeq{\text{等价无穷小}}}
\newcommand\eqprob{\stackrel{\mathrm{P}}{=}}
\newcommand\lto{\longmapsto}
\renewcommand\vec[1]{\overrightarrow{#1}}

% decorations
\newcommand\wh{\widehat}
\newcommand\wt{\widetilde}
\newcommand\ov{\overline}
\newcommand\ul{\underline}
\newlength{\larc}
\NewDocumentCommand\warc{o m}{%
	\IfNoValueTF {#1}%
	{%
		\settowidth{\larc}{$#2$}%
		\stackrel{\rotatebox{-90}{\ensuremath{\left(\rule{0ex}{0.7\larc}\right.}}}{#2}%
	}%
	{%
		\stackrel{\rotatebox{-90}{\ensuremath{\left(\rule{0ex}{#1}\right.}}}{#2}%
	}%
}

% braces
\newcommand\set[1]{{\left\{#1\right\}}}
\newcommand\setm[2]{{\left\{#1\,\middle\vert\, #2\right\}}}
\newcommand\abs[1]{\left|#1\right|}
\newcommand\pair[1]{\langle{#1}\rangle}
\newcommand\norm[1]{\!\parallel\!{#1}\!\parallel\!}
\newcommand\dbb[1]{\llbracket#1 \rrbracket}
\newcommand\floor[1]{\lfloor#1\rfloor}

% symbols
\renewcommand\le{\leqslant}
\renewcommand\ge{\geqslant}
\newcommand\vare{\varepsilon}
\newcommand\varp{\varphi}
\newcommand\ilim{\varinjlim\limits}
\newcommand\plim{\varprojlim\limits}
\newcommand\half{\frac{1}{2}}
\newcommand\mmid{\parallel}
\font\cyr=wncyr10\newcommand\Sha{\hbox{\cyr X}}
\newcommand\Uc{\stackrel{\circ}{U}\!\!}
\newcommand\hil[3]{\left(\frac{{#1},{#2}}{#3}\right)}
\newcommand\leg[2]{\Bigl(\frac{{#1}}{#2}\Bigr)}
\newcommand\aleg[2]{\Bigl[\frac{{#1}}{#2}\Bigr]}
\newcommand\stsc[2]{\genfrac{}{}{0pt}{}{#1}{#2}}

% categories
\newcommand\cA{{\mathsf{A}}}
\newcommand\cb{{\mathsf{b}}}
\newcommand\cB{{\mathsf{B}}}
\newcommand\cC{{\mathsf{C}}}
\newcommand\cD{{\mathsf{D}}}
\newcommand\cM{{\mathsf{M}}}
\newcommand\cR{{\mathsf{R}}}
\newcommand\cP{{\mathsf{P}}}
\newcommand\cT{{\mathsf{T}}}
\newcommand\cX{{\mathsf{X}}}
\newcommand\cx{{\mathsf{x}}}
\newcommand\cAb{{\mathsf{Ab}}}
\newcommand\cBT{{\mathsf{BT}}}
\newcommand\cBun{{\mathsf{Bun}}}
\newcommand\cCharLoc{{\mathsf{CharLoc}}}
\newcommand\cCoh{{\mathsf{Coh}}}
\newcommand\cComm{{\mathsf{Comm}}}
\newcommand\cEt{{\mathsf{Et}}}
\newcommand\cFppf{{\mathsf{Fppf}}}
\newcommand\cFpqc{{\mathsf{Fpqc}}}
\newcommand\cFunc{{\mathsf{Func}}}
\newcommand\cGroups{{\mathsf{Groups}}}
\newcommand\cGrpd{{\{\mathsf{Grpd}\}}}
\newcommand\cHo{{\mathsf{Ho}}}
\newcommand\cIso{{\mathsf{Iso}}}
\newcommand\cLoc{{\mathsf{Loc}}}
\newcommand\cMod{{\mathsf{Mod}}}
\newcommand\cModFil{{\mathsf{ModFil}}}
\newcommand\cNilp{{\mathsf{Nilp}}}
\newcommand\cPerf{{\mathsf{Perf}}}
\newcommand\cPN{{\mathsf{PN}}}
\newcommand\cRep{{\mathsf{Rep}}}
\newcommand\cRings{{\mathsf{Rings}}}
\newcommand\cSets{{\mathsf{Sets}}}
\newcommand\cStack{{\mathsf{Stack}}}
\newcommand\cSch{{\mathsf{Sch}}}
\newcommand\cTop{{\mathsf{Top}}}
\newcommand\cVect{{\mathsf{Vect}}}
\newcommand\cZar{{\mathsf{Zar}}}
\newcommand\cphimod{{\varphi\txt{-}\mathsf{Mod}}}
\newcommand\cphimodfil{{\varphi\txt{-}\mathsf{ModFil}}}

% font
\newcommand\rma{{\mathrm{a}}}
\newcommand\rmb{{\mathrm{b}}}
\newcommand\rmc{{\mathrm{c}}}
\newcommand\rmd{{\mathrm{d}}}
\newcommand\rme{{\mathrm{e}}}
\newcommand\rmf{{\mathrm{f}}}
\newcommand\rmg{{\mathrm{g}}}
\newcommand\rmh{{\mathrm{h}}}
\newcommand\rmi{{\mathrm{i}}}
\newcommand\rmj{{\mathrm{j}}}
\newcommand\rmk{{\mathrm{k}}}
\newcommand\rml{{\mathrm{l}}}
\newcommand\rmm{{\mathrm{m}}}
\newcommand\rmn{{\mathrm{n}}}
\newcommand\rmo{{\mathrm{o}}}
\newcommand\rmp{{\mathrm{p}}}
\newcommand\rmq{{\mathrm{q}}}
\newcommand\rmr{{\mathrm{r}}}
\newcommand\rms{{\mathrm{s}}}
\newcommand\rmt{{\mathrm{t}}}
\newcommand\rmu{{\mathrm{u}}}
\newcommand\rmv{{\mathrm{v}}}
\newcommand\rmw{{\mathrm{w}}}
\newcommand\rmx{{\mathrm{x}}}
\newcommand\rmy{{\mathrm{y}}}
\newcommand\rmz{{\mathrm{z}}}
\newcommand\rmA{{\mathrm{A}}}
\newcommand\rmB{{\mathrm{B}}}
\newcommand\rmC{{\mathrm{C}}}
\newcommand\rmD{{\mathrm{D}}}
\newcommand\rmE{{\mathrm{E}}}
\newcommand\rmF{{\mathrm{F}}}
\newcommand\rmG{{\mathrm{G}}}
\newcommand\rmH{{\mathrm{H}}}
\newcommand\rmI{{\mathrm{I}}}
\newcommand\rmJ{{\mathrm{J}}}
\newcommand\rmK{{\mathrm{K}}}
\newcommand\rmL{{\mathrm{L}}}
\newcommand\rmM{{\mathrm{M}}}
\newcommand\rmN{{\mathrm{N}}}
\newcommand\rmO{{\mathrm{O}}}
\newcommand\rmP{{\mathrm{P}}}
\newcommand\rmQ{{\mathrm{Q}}}
\newcommand\rmR{{\mathrm{R}}}
\newcommand\rmS{{\mathrm{S}}}
\newcommand\rmT{{\mathrm{T}}}
\newcommand\rmU{{\mathrm{U}}}
\newcommand\rmV{{\mathrm{V}}}
\newcommand\rmW{{\mathrm{W}}}
\newcommand\rmX{{\mathrm{X}}}
\newcommand\rmY{{\mathrm{Y}}}
\newcommand\rmZ{{\mathrm{Z}}}
\newcommand\bfa{{\mathbf{a}}}
\newcommand\bfb{{\mathbf{b}}}
\newcommand\bfc{{\mathbf{c}}}
\newcommand\bfd{{\mathbf{d}}}
\newcommand\bfe{{\mathbf{e}}}
\newcommand\bff{{\mathbf{f}}}
\newcommand\bfg{{\mathbf{g}}}
\newcommand\bfh{{\mathbf{h}}}
\newcommand\bfi{{\mathbf{i}}}
\newcommand\bfj{{\mathbf{j}}}
\newcommand\bfk{{\mathbf{k}}}
\newcommand\bfl{{\mathbf{l}}}
\newcommand\bfm{{\mathbf{m}}}
\newcommand\bfn{{\mathbf{n}}}
\newcommand\bfo{{\mathbf{o}}}
\newcommand\bfp{{\mathbf{p}}}
\newcommand\bfq{{\mathbf{q}}}
\newcommand\bfr{{\mathbf{r}}}
\newcommand\bfs{{\mathbf{s}}}
\newcommand\bft{{\mathbf{t}}}
\newcommand\bfu{{\mathbf{u}}}
\newcommand\bfv{{\mathbf{v}}}
\newcommand\bfw{{\mathbf{w}}}
\newcommand\bfx{{\mathbf{x}}}
\newcommand\bfy{{\mathbf{y}}}
\newcommand\bfz{{\mathbf{z}}}
\newcommand\bfA{{\mathbf{A}}}
\newcommand\bfB{{\mathbf{B}}}
\newcommand\bfC{{\mathbf{C}}}
\newcommand\bfD{{\mathbf{D}}}
\newcommand\bfE{{\mathbf{E}}}
\newcommand\bfF{{\mathbf{F}}}
\newcommand\bfG{{\mathbf{G}}}
\newcommand\bfH{{\mathbf{H}}}
\newcommand\bfI{{\mathbf{I}}}
\newcommand\bfJ{{\mathbf{J}}}
\newcommand\bfK{{\mathbf{K}}}
\newcommand\bfL{{\mathbf{L}}}
\newcommand\bfM{{\mathbf{M}}}
\newcommand\bfN{{\mathbf{N}}}
\newcommand\bfO{{\mathbf{O}}}
\newcommand\bfP{{\mathbf{P}}}
\newcommand\bfQ{{\mathbf{Q}}}
\newcommand\bfR{{\mathbf{R}}}
\newcommand\bfS{{\mathbf{S}}}
\newcommand\bfT{{\mathbf{T}}}
\newcommand\bfU{{\mathbf{U}}}
\newcommand\bfV{{\mathbf{V}}}
\newcommand\bfW{{\mathbf{W}}}
\newcommand\bfX{{\mathbf{X}}}
\newcommand\bfY{{\mathbf{Y}}}
\newcommand\bfZ{{\mathbf{Z}}}
\newcommand\BA{{\mathbb{A}}}
\newcommand\BB{{\mathbb{B}}}
\newcommand\BC{{\mathbb{C}}}
\newcommand\BD{{\mathbb{D}}}
\newcommand\BE{{\mathbb{E}}}
\newcommand\BF{{\mathbb{F}}}
\newcommand\BG{{\mathbb{G}}}
\newcommand\BH{{\mathbb{H}}}
\newcommand\BI{{\mathbb{I}}}
\newcommand\BJ{{\mathbb{J}}}
\newcommand\BK{{\mathbb{K}}}
\newcommand\BL{{\mathbb{L}}}
\newcommand\BM{{\mathbb{M}}}
\newcommand\BN{{\mathbb{N}}}
\newcommand\BO{{\mathbb{O}}}
\newcommand\BP{{\mathbb{P}}}
\newcommand\BQ{{\mathbb{Q}}}
\newcommand\BR{{\mathbb{R}}}
\newcommand\BS{{\mathbb{S}}}
\newcommand\BT{{\mathbb{T}}}
\newcommand\BU{{\mathbb{U}}}
\newcommand\BV{{\mathbb{V}}}
\newcommand\BW{{\mathbb{W}}}
\newcommand\BX{{\mathbb{X}}}
\newcommand\BY{{\mathbb{Y}}}
\newcommand\BZ{{\mathbb{Z}}}
\newcommand\CA{{\mathcal{A}}}
\newcommand\CB{{\mathcal{B}}}
\newcommand\CC{{\mathcal{C}}}
\providecommand\CD{{\mathcal{D}}}
\newcommand\CE{{\mathcal{E}}}
\newcommand\CF{{\mathcal{F}}}
\newcommand\CG{{\mathcal{G}}}
\newcommand\CH{{\mathcal{H}}}
\newcommand\CI{{\mathcal{I}}}
\newcommand\CJ{{\mathcal{J}}}
\newcommand\CK{{\mathcal{K}}}
\newcommand\CL{{\mathcal{L}}}
\newcommand\CM{{\mathcal{M}}}
\newcommand\CN{{\mathcal{N}}}
\newcommand\CO{{\mathcal{O}}}
\newcommand\CP{{\mathcal{P}}}
\newcommand\CQ{{\mathcal{Q}}}
\newcommand\CR{{\mathcal{R}}}
\newcommand\CS{{\mathcal{S}}}
\newcommand\CT{{\mathcal{T}}}
\newcommand\CU{{\mathcal{U}}}
\newcommand\CV{{\mathcal{V}}}
\newcommand\CW{{\mathcal{W}}}
\newcommand\CX{{\mathcal{X}}}
\newcommand\CY{{\mathcal{Y}}}
\newcommand\CZ{{\mathcal{Z}}}
\newcommand\RA{{\mathrm{A}}}
\newcommand\RB{{\mathrm{B}}}
\newcommand\RC{{\mathrm{C}}}
\newcommand\RD{{\mathrm{D}}}
\newcommand\RE{{\mathrm{E}}}
\newcommand\RF{{\mathrm{F}}}
\newcommand\RG{{\mathrm{G}}}
\newcommand\RH{{\mathrm{H}}}
\newcommand\RI{{\mathrm{I}}}
\newcommand\RJ{{\mathrm{J}}}
\newcommand\RK{{\mathrm{K}}}
\newcommand\RL{{\mathrm{L}}}
\newcommand\RM{{\mathrm{M}}}
% \newcommand\RN{{\mathrm{N}}}
\newcommand\RO{{\mathrm{O}}}
\newcommand\RP{{\mathrm{P}}}
\newcommand\RQ{{\mathrm{Q}}}
\newcommand\RR{{\mathrm{R}}}
\newcommand\RS{{\mathrm{S}}}
\newcommand\RT{{\mathrm{T}}}
\newcommand\RU{{\mathrm{U}}}
\newcommand\RV{{\mathrm{V}}}
\newcommand\RW{{\mathrm{W}}}
\newcommand\RX{{\mathrm{X}}}
\newcommand\RY{{\mathrm{Y}}}
\newcommand\RZ{{\mathrm{Z}}}
\newcommand\msa{\mathscr{A}}
\newcommand\msb{\mathscr{B}}
\newcommand\msc{\mathscr{C}}
\newcommand\msd{\mathscr{D}}
\newcommand\mse{\mathscr{E}}
\newcommand\msf{\mathscr{F}}
\newcommand\msg{\mathscr{G}}
\newcommand\msh{\mathscr{H}}
\newcommand\msi{\mathscr{I}}
\newcommand\msj{\mathscr{J}}
\newcommand\msk{\mathscr{K}}
\newcommand\msl{\mathscr{L}}
\newcommand\msm{\mathscr{M}}
\newcommand\msn{\mathscr{N}}
\newcommand\mso{\mathscr{O}}
\newcommand\msp{\mathscr{P}}
\newcommand\msq{\mathscr{Q}}
\newcommand\msr{\mathscr{R}}
\newcommand\mss{\mathscr{S}}
\newcommand\mst{\mathscr{T}}
\newcommand\msu{\mathscr{U}}
\newcommand\msv{\mathscr{V}}
\newcommand\msw{\mathscr{W}}
\newcommand\msx{\mathscr{X}}
\newcommand\msy{\mathscr{Y}}
\newcommand\msz{\mathscr{Z}}
\newcommand\fa{{\mathfrak{a}}}
\newcommand\fb{{\mathfrak{b}}}
\newcommand\fc{{\mathfrak{c}}}
\newcommand\fd{{\mathfrak{d}}}
\newcommand\fe{{\mathfrak{e}}}
\newcommand\ff{{\mathfrak{f}}}
\newcommand\fg{{\mathfrak{g}}}
\newcommand\fh{{\mathfrak{h}}}
\newcommand\fii{{\mathfrak{i}}} % be careful about \fii
\newcommand\fj{{\mathfrak{j}}}
\newcommand\fk{{\mathfrak{k}}}
\newcommand\fl{{\mathfrak{l}}}
\newcommand\fm{{\mathfrak{m}}}
\newcommand\fn{{\mathfrak{n}}}
\newcommand\fo{{\mathfrak{o}}}
\newcommand\fp{{\mathfrak{p}}}
\newcommand\fq{{\mathfrak{q}}}
\newcommand\fr{{\mathfrak{r}}}
\newcommand\fs{{\mathfrak{s}}}
\newcommand\ft{{\mathfrak{t}}}
\newcommand\fu{{\mathfrak{u}}}
\newcommand\fv{{\mathfrak{v}}}
\newcommand\fw{{\mathfrak{w}}}
\newcommand\fx{{\mathfrak{x}}}
\newcommand\fy{{\mathfrak{y}}}
\newcommand\fz{{\mathfrak{z}}}
\newcommand\fA{{\mathfrak{A}}}
\newcommand\fB{{\mathfrak{B}}}
\newcommand\fC{{\mathfrak{C}}}
\newcommand\fD{{\mathfrak{D}}}
\newcommand\fE{{\mathfrak{E}}}
\newcommand\fF{{\mathfrak{F}}}
\newcommand\fG{{\mathfrak{G}}}
\newcommand\fH{{\mathfrak{H}}}
\newcommand\fI{{\mathfrak{I}}}
\newcommand\fJ{{\mathfrak{J}}}
\newcommand\fK{{\mathfrak{K}}}
\newcommand\fL{{\mathfrak{L}}}
\newcommand\fM{{\mathfrak{M}}}
\newcommand\fN{{\mathfrak{N}}}
\newcommand\fO{{\mathfrak{O}}}
\newcommand\fP{{\mathfrak{P}}}
\newcommand\fQ{{\mathfrak{Q}}}
\newcommand\fR{{\mathfrak{R}}}
\newcommand\fS{{\mathfrak{S}}}
\newcommand\fT{{\mathfrak{T}}}
\newcommand\fU{{\mathfrak{U}}}
\newcommand\fV{{\mathfrak{V}}}
\newcommand\fW{{\mathfrak{W}}}
\newcommand\fX{{\mathfrak{X}}}
\newcommand\fY{{\mathfrak{Y}}}
\newcommand\fZ{{\mathfrak{Z}}}

% a
\newcommand\ab{{\mathrm{ab}}}
\newcommand\ad{{\mathrm{ad}}}
\newcommand\Ad{{\mathrm{Ad}}}
\newcommand\adele{ad\'{e}le}
\newcommand\Adele{Ad\'{e}le}
\newcommand\adeles{ad\'{e}les}
\newcommand\adelic{ad\'{e}lic}
\newcommand\AJ{{\mathrm{AJ}}}
\newcommand\alb{{\mathrm{alb}}}
\newcommand\Alb{{\mathrm{Alb}}}
\newcommand\alg{{\mathrm{alg}}}
\newcommand\an{{\mathrm{an}}}
\newcommand\ann{{\mathrm{ann}}}
\newcommand\Ann{{\mathrm{Ann}}}
\DeclareMathOperator\Arcsin{Arcsin}
\DeclareMathOperator\Arccos{Arccos}
\DeclareMathOperator\Arctan{Arctan}
\DeclareMathOperator\arsh{arsh}
\DeclareMathOperator\Arsh{Arsh}
\DeclareMathOperator\arch{arch}
\DeclareMathOperator\Arch{Arch}
\DeclareMathOperator\arth{arth}
\DeclareMathOperator\Arth{Arth}
\DeclareMathOperator\arccot{arccot}
\DeclareMathOperator\Arg{Arg}
\newcommand\arith{{\mathrm{arith}}}
\newcommand\Art{{\mathrm{Art}}}
\newcommand\AS{{\mathrm{AS}}}
\newcommand\Ass{{\mathrm{Ass}}}
\newcommand\Aut{{\mathrm{Aut}}}
% b
\newcommand\Bun{{\mathrm{Bun}}}
\newcommand\Br{{\mathrm{Br}}}
\newcommand\bs{\backslash}
\newcommand\BWt{{\mathrm{BW}}}
% c
\newcommand\can{{\mathrm{can}}}
\newcommand\cc{{\mathrm{cc}}}
\newcommand\cd{{\mathrm{cd}}}
\DeclareMathOperator\ch{ch}
\newcommand\Ch{{\mathrm{Ch}}}
\let\char\relax
\DeclareMathOperator\char{char}
\DeclareMathOperator\Char{char}
\newcommand\Chow{{\mathrm{CH}}}
\newcommand\circB{{\stackrel{\circ}{B}}}
\newcommand\cl{{\mathrm{cl}}}
\newcommand\Cl{{\mathrm{Cl}}}
\newcommand\cm{{\mathrm{cm}}}
\newcommand\cod{{\mathrm{cod}}}
\DeclareMathOperator\coker{coker}
\DeclareMathOperator\Coker{Coker}
\DeclareMathOperator\cond{cond}
\DeclareMathOperator\codim{codim}
\newcommand\cont{{\mathrm{cont}}}
\newcommand\Conv{{\mathrm{Conv}}}
\newcommand\corr{{\mathrm{corr}}}
\newcommand\Corr{{\mathrm{Corr}}}
\DeclareMathOperator\coim{coim}
\DeclareMathOperator\coIm{coIm}
\DeclareMathOperator\corank{corank}
\DeclareMathOperator\covol{covol}
\newcommand\cris{{\mathrm{cris}}}
\newcommand\Cris{{\mathrm{Cris}}}
\newcommand\CRIS{{\mathrm{CRIS}}}
\newcommand\crit{{\mathrm{crit}}}
\newcommand\crys{{\mathrm{crys}}}
\newcommand\cusp{{\mathrm{cusp}}}
\newcommand\CWt{{\mathrm{CW}}}
\newcommand\cyc{{\mathrm{cyc}}}
% d
\newcommand\Def{{\mathrm{Def}}}
\newcommand\diag{{\mathrm{diag}}}
\newcommand\diff{\,\mathrm{d}}
\newcommand\disc{{\mathrm{disc}}}
\newcommand\dist{{\mathrm{dist}}}
\renewcommand\div{{\mathrm{div}}}
\newcommand\Div{{\mathrm{Div}}}
\newcommand\dR{{\mathrm{dR}}}
\newcommand\Drin{{\mathrm{Drin}}}
% e
\newcommand\End{{\mathrm{End}}}
\newcommand\ess{{\mathrm{ess}}}
\newcommand\et{{\text{\'{e}t}}}
\newcommand\etale{{\'{e}tale}}
\newcommand\Etale{{\'{E}tale}}
\newcommand\Ext{\mathrm{Ext}}
\newcommand\CExt{\mathcal{E}\mathrm{xt}}
% f
\newcommand\Fil{{\mathrm{Fil}}}
\newcommand\Fix{{\mathrm{Fix}}}
\newcommand\fppf{{\mathrm{fppf}}}
\newcommand\Fr{{\mathrm{Fr}}}
\newcommand\Frac{{\mathrm{Frac}}}
\newcommand\Frob{{\mathrm{Frob}}}
% g
\newcommand\Ga{\mathbb{G}_a}
\newcommand\Gm{\mathbb{G}_m}
\newcommand\hGa{\widehat{\mathbb{G}}_{a}}
\newcommand\hGm{\widehat{\mathbb{G}}_{m}}
\newcommand\Gal{{\mathrm{Gal}}}
\newcommand\gl{{\mathrm{gl}}}
\newcommand\GL{{\mathrm{GL}}}
\newcommand\GO{{\mathrm{GO}}}
\newcommand\geom{{\mathrm{geom}}}
\newcommand\Gr{{\mathrm{Gr}}}
\newcommand\gr{{\mathrm{gr}}}
\newcommand\GSO{{\mathrm{GSO}}}
\newcommand\GSp{{\mathrm{GSp}}}
\newcommand\GSpin{{\mathrm{GSpin}}}
\newcommand\GU{{\mathrm{GU}}}
% h
\newcommand\hg{{\mathrm{hg}}}
\newcommand\Hk{{\mathrm{Hk}}}
\newcommand\HN{{\mathrm{HN}}}
\newcommand\Hom{{\mathrm{Hom}}}
\newcommand\CHom{\mathcal{H}\mathrm{om}}
% i
\newcommand\id{{\mathrm{id}}}
\newcommand\Id{{\mathrm{Id}}}
\newcommand\idele{id\'{e}le}
\newcommand\Idele{Id\'{e}le}
\newcommand\ideles{id\'{e}les}
\let\Im\relax
\DeclareMathOperator\Im{Im}
\DeclareMathOperator\im{im}
\newcommand\Ind{{\mathrm{Ind}}}
\newcommand\cInd{{\mathrm{c}\textrm{-}\mathrm{Ind}}}
\newcommand\ind{{\mathrm{ind}}}
\newcommand\Int{{\mathrm{Int}}}
\newcommand\inv{{\mathrm{inv}}}
\newcommand\Isom{{\mathrm{Isom}}}
% j
\newcommand\Jac{{\mathrm{Jac}}}
\newcommand\JL{{\mathrm{JL}}}
% k
\newcommand\Katz{\mathrm{Katz}}
\DeclareMathOperator\Ker{Ker}
\newcommand\KS{{\mathrm{KS}}}
\newcommand\Kl{{\mathrm{Kl}}}
\newcommand\CKl{{\mathcal{K}\mathrm{l}}}
% l
\newcommand\lcm{\mathrm{lcm}}
\newcommand\length{\mathrm{length}}
\DeclareMathOperator\Li{Li}
\DeclareMathOperator\Ln{Ln}
\newcommand\Lie{{\mathrm{Lie}}}
\newcommand\lt{\mathrm{lt}}
\newcommand\LT{\mathcal{LT}}
% m
\newcommand\mex{{\mathrm{mex}}}
\newcommand\MW{{\mathrm{MW}}}
\renewcommand\mod{\, \mathrm{mod}\, }
\newcommand\mom{{\mathrm{mom}}}
\newcommand\Mor{{\mathrm{Mor}}}
\newcommand\Morp{{\mathrm{Morp}\,}}
% n
\newcommand\new{{\mathrm{new}}}
\newcommand\Newt{{\mathrm{Newt}}}
\newcommand\nd{{\mathrm{nd}}}
\newcommand\NP{{\mathrm{NP}}}
\newcommand\NS{{\mathrm{NS}}}
\newcommand\ns{{\mathrm{ns}}}
\newcommand\Nm{{\mathrm{Nm}}}
\newcommand\Nrd{{\mathrm{Nrd}}}
\newcommand\Neron{N\'{e}ron}
% o
\newcommand\Obj{{\mathrm{Obj}\,}}
\newcommand\odd{{\mathrm{odd}}}
\newcommand\old{{\mathrm{old}}}
\newcommand\op{{\mathrm{op}}}
\newcommand\Orb{{\mathrm{Orb}}}
\newcommand\ord{{\mathrm{ord}}}
% p
\newcommand\pd{{\mathrm{pd}}}
\newcommand\Pet{{\mathrm{Pet}}}
\newcommand\PGL{{\mathrm{PGL}}}
\newcommand\Pic{{\mathrm{Pic}}}
\newcommand\pr{{\mathrm{pr}}}
\newcommand\Proj{{\mathrm{Proj}}}
\newcommand\proet{\text{pro\'{e}t}}
\newcommand\Poincare{\text{Poincar\'{e}}}
\newcommand\Prd{{\mathrm{Prd}}}
\newcommand\prim{{\mathrm{prim}}}
% r
\newcommand\Rad{{\mathrm{Rad}}}
\newcommand\rad{{\mathrm{rad}}}
\DeclareMathOperator\rank{rank}
\let\Re\relax
\DeclareMathOperator\Re{Re}
\newcommand\rec{{\mathrm{rec}}}
\newcommand\red{{\mathrm{red}}}
\newcommand\reg{{\mathrm{reg}}}
\newcommand\res{{\mathrm{res}}}
\newcommand\Res{{\mathrm{Res}}}
\newcommand\rig{{\mathrm{rig}}}
\newcommand\Rig{{\mathrm{Rig}}}
\newcommand\rk{{\mathrm{rk}}}
\newcommand\Ros{{\mathrm{Ros}}}
\newcommand\rs{{\mathrm{rs}}}
% s
\newcommand\sd{{\mathrm{sd}}}
\newcommand\Sel{{\mathrm{Sel}}}
\newcommand\sep{{\mathrm{sep}}}
\DeclareMathOperator{\sgn}{sgn}
\newcommand\Sh{{\mathrm{Sh}}}
\DeclareMathOperator\sh{sh}
\newcommand\Sht{\mathrm{Sht}}
\newcommand\Sim{{\mathrm{Sim}}}
\newcommand\sign{{\mathrm{sign}}}
\newcommand\SK{{\mathrm{SK}}}
\newcommand\SL{{\mathrm{SL}}}
\newcommand\SO{{\mathrm{SO}}}
\newcommand\Sp{{\mathrm{Sp}}}
\newcommand\Spa{{\mathrm{Spa}}}
\newcommand\Span{{\mathrm{Span}}}
\DeclareMathOperator\Spec{Spec}
\newcommand\Spf{{\mathrm{Spf}}}
\newcommand\Spin{{\mathrm{Spin}}}
\newcommand\Spm{{\mathrm{Spm}}}
\newcommand\srs{{\mathrm{srs}}}
\newcommand\rss{{\mathrm{ss}}}
\newcommand\ST{{\mathrm{ST}}}
\newcommand\St{{\mathrm{St}}}
\newcommand\st{{\mathrm{st}}}
\newcommand\Stab{{\mathrm{Stab}}}
\newcommand\SU{{\mathrm{SU}}}
\newcommand\Sym{{\mathrm{Sym}}}
\newcommand\sub{{\mathrm{sub}}}
\newcommand\rsum{{\mathrm{sum}}}
\newcommand\supp{{\mathrm{supp}}}
\newcommand\Supp{{\mathrm{Supp}}}
\newcommand\Swan{{\mathrm{Sw}}}
\newcommand\suml{\sum\limits}
% t
\newcommand\td{{\mathrm{td}}}
\let\tanh\relax
\DeclareMathOperator\tanh{th}
\newcommand\tor{{\mathrm{tor}}}
\newcommand\Tor{{\mathrm{Tor}}}
\newcommand\tors{{\mathrm{tors}}}
\newcommand\tr{{\mathrm{tr}\,}}
\newcommand\Tr{{\mathrm{Tr}}}
\newcommand\Trd{{\mathrm{Trd}}}
\newcommand\TSym{{\mathrm{TSym}}}
\newcommand\tw{{\mathrm{tw}}}
% u
\newcommand\uni{{\mathrm{uni}}}
\newcommand\univ{\mathrm{univ}}
\newcommand\ur{{\mathrm{ur}}}
\newcommand\USp{{\mathrm{USp}}}
% v
\newcommand\vQ{{\breve \BQ}}
\newcommand\vE{{\breve E}}
\newcommand\Ver{{\mathrm{Ver}}}
\newcommand\vF{{\breve F}}
\newcommand\vK{{\breve K}}
\newcommand\vol{{\mathrm{vol}}}
\newcommand\Vol{{\mathrm{Vol}}}
% w
\newcommand\wa{{\mathrm{wa}}}
% z
\newcommand\Zar{{\mathrm{Zar}}}

\renewcommand\logo[1]{
	\def\inserttitlegraphic{../image/logo/#1.png}
	\def\insertinstitutegraphic{../image/logo/#1name.png}
}
\email{zhangshenxing@hfut.edu.cn}
\website{https://zhangshenxing.github.io}
\date{}
\logo{hfut}
\author{张神星}
\institute{合肥工业大学}
\AtBeginSection{}
\showprefixfalse
\website{}
\newcommand\frameoutline{{\setnaviboxempty\begin{frame}{提纲}\tableofcontents\end{frame}}}
\newcommand\framethanks{{\setnaviboxempty\begin{frame}\begin{center}\huge \textbf{谢~谢!}\end{center}\end{frame}}}

\RequirePackage[T1]{fontenc}
\setCJKsansfont[ItalicFont={KaiTi},BoldFont={LXGW ZhenKai}]{Source Han Sans HW SC}
\newfontface\cmunrm{cmunrm.otf}\newcommand\cmu[1]{{\cmunrm{#1}}}

\title{含非同余数因子的非同余数}
\invitedby[nju]{2025南京大学数论与自守表示研讨会}
\date{2025年6月15日}

\newcommand{\emphov}[2]{\emph{$\overset{\text{\clap{#2}}}{\text{#1}}$}}
\renewcommand\leg[2]{\Bigl(\dfrac{{#1}}{#2}\Bigr)}
\renewcommand\aleg[2]{\Bigl[\dfrac{{#1}}{#2}\Bigr]}

\usepackage{makecell}
\usepackage{tikz}
\usetikzlibrary{
	hobby,
	intersections,
	quotes,
	shapes.arrows,
	arrows.meta,
	bending,
	positioning,
	shapes.geometric,
	overlay-beamer-styles,
	calc,
	angles,
	decorations.markings,
	decorations.pathreplacing,
	backgrounds,
	chains
}

\begin{document}


\section{问题背景}

\begin{frame}{同余数问题}
	\begin{itemize}
		\item 同余数问题是一个古老的数学问题.
		\item 如果正整数 $n$ 可以表达为一个有理边长直角三角形的面积, 则称 $n$ 是\emphov{同余数}{congruent number}.
	\end{itemize}
	\onslide<+->
	\begin{center}
		\begin{tikzpicture}
			\begin{scope}[scale=.1]
				\draw (0,0)--(40,0)--(0,9)--cycle;
				\draw (7,4) node[color=main] {$\bf 5$};
				\draw (20,0) node[below] {$\frac{40}6$};
				\draw (0,4.5) node[left] {$\frac{9}6$};
				\draw (20,4.5) node[above] {$\frac{41}6$};
			\end{scope}
			\begin{scope}[scale=.8,shift={(6,0)}]
				\draw (0,0)--(4,0)--(0,3)--cycle;
				\draw (1,1) node[color=main] {$\bf 6$};
				\draw (2,0) node[below] {$4$};
				\draw (0,1.5) node[left] {$3$};
				\draw (2,1.5) node[above right] {$5$};
			\end{scope}
			\begin{scope}[scale=.15,shift={(60,0)}]
				\draw (0,0)--(28.8,0)--(0,17.5)--cycle;
				\draw (7,7) node[color=main] {$\bf 7$};
				\draw (14.4,0) node[below] {$\frac{288}{60}$};
				\draw (0,8.75) node[left] {$\frac{175}{60}$};
				\draw (14.4,8.75) node[above right] {$\frac{337}{60}$};
			\end{scope}
		\end{tikzpicture}
	\end{center}
\end{frame}


\begin{frame}{同余数与椭圆曲线}
	\begin{itemize}
		\item 显然我们只需要考虑无平方因子正整数.
		\item 设直角三角形的三条边分别为 $a,b,c$, 设 $x=\dfrac{n(a-c)}b,y=\dfrac{2nx}b$.
		\item 那么由勾股定理 $a^2+b^2=c^2$ 可得 $a=\dfrac{x^2-n^2}y,b=\dfrac{2nx}y$.
		\item 于是 $n=\dfrac12ab=\dfrac{n(x^3-n^2x)}{y^2}$.
		\item 换言之, $(x,y)$ 是椭圆曲线
		\[
			E_n:y^2=x^3-n^2x
		\]
		的一个满足 $y\ne 0$ 的有理点.
		\item 而 $E_n(\BQ)$ 全体构成有限生成交换群, 且挠群为
		\[
			E_n(\BQ)_{\tors}=E_n[2]=\{O,(0,0),(n,0),(-n,0)\}\cong(\BZ/2\BZ)^2.
		\]
		\item 故 $n$ 是同余数当且仅当该椭圆曲线的有理点全体 $E_n(\BQ)$ 构成无限群.
	\end{itemize}
\end{frame}


\begin{frame}{Selmer 群和Tate-Shafarevich群}
	\begin{itemize}
		\item 我们可以利用Selmer群来估计 $E_n(\BQ)$ 的秩.
		\item 通过对短正合列 $0\ra E_n[2]\ra E_n\sto{\times2} E_n\ra 0$ 取 $G_\BQ$ 上同调可得
		\[
			0\ra E_n(\BQ)/2E_n(\BQ)\ra \Sel_2(E_n)\ra \Sha(E_n)[2]\ra0,
		\]
		其中
		\begin{align*}
			\Sel_2(E_n)&:=\Ker\Bigl(\RH^1(G_\BQ,E_n[2])\ra\prod_v \RH^1(G_{\BQ_v},E_n)\Bigr),\\
			\Sha(E_n)&:=\Ker\Bigl(\RH^1(G_\BQ,E_n)\ra\prod_v \RH^1(G_{\BQ_v},E_n)\Bigr).
		\end{align*}
		\item 由此可得
		\[
			\rank_\BZ E_n(\BQ)\le \dim_{\BF_2}\Sel_2(E_n)-2=s_2(n),
		\]
		\item 其中
		\[
			s_2(n):=\dim_{\BF_2}\Sel'_2(E_n),\qquad
			\Sel'_2(E_n):=\frac{\Sel_2(E_n)}{E_n(\BQ)[2]}.
		\]
	\end{itemize}
\end{frame}


\begin{frame}{非同余数: $s_2(n)=0$ 情形}
	\begin{itemize}
		\item BSD猜想断言: 若 $n$ 是非同余数, 则 $L(E_n,1)\ne 0$, 从而 $n\equiv 1,2,3\bmod 8$.
		\item 当 $n\equiv 1,2,3\bmod 8$ 时, $s_2(n)$ 是偶数.
		\item 自然地, $s_2(n)=0$ 蕴含 $n$ 是非同余数且 $\Sha(E_n)[2^\infty]=0$.
		\item 此时由Tian-Yuan-Zhang (2017) 和Smith (2016), 它等价于
		\[
			\sum_{\substack{n=d_0d_1\cdots d_k\\ d_1\equiv\dots\equiv d_k\equiv 1\bmod 8\\ h_4(-d_i)=0,\forall i}}1=1\in\BF_2,
		\]
		且此时 BSD 猜想 $2$ 部分成立.
		\item 这里 $h_4(-d)=r_4(\CA_{-d})$ 是 $F_{-d}=\BQ(\sqrt{-d})$ 整数环 $\CO_{-d}$ 缩理想类群 $\CA_{-d}$ 的 $4$ 秩, 
		\[
			r_{2^a}(A):=\dim_{\BF_2}\Bigl(\frac{2^{a-1} A}{2^a A}\Bigr).
		\]
	\end{itemize}
\end{frame}


\begin{frame}{非同余数: $s_2(n)=2$ 情形}
	\onslide<+->
	\begin{theorem}[nearprev][{Wang 2016}]
		若 $n$ 是模 $4$ 余 $1$ 素数乘积, 则下述等价:
		\begin{itemize}[<*>]
			\item $n$ 是非同余数且 $\Sha(E_n)[2^\infty]\cong(\BZ/2\BZ)^2$;
			\item $h_4(-n)=1,h_8(-n)\equiv (d-1)/4\bmod 2$,
		\end{itemize}
		其中 $0<d\mid n$ 满足 $(d,-n)_v=1,\forall v, d\ne 1,n$, 或 $(2d,-n)_v=1,\forall v$.
	\end{theorem}
	这里 $(d,-n)_v$ 是希尔伯特符号.
	\onslide<+->
	\begin{theorem}[nearnext][{Wang-Zhang 2022}]
		若 $n$ 是模 $8$ 余 $\pm1$ 素数乘积, 则下述等价:
		\begin{itemize}[<*>]
			\item $n$ 是非同余数且 $\Sha(E_n)[2^\infty]\cong(\BZ/2\BZ)^2$;
			\item $h_4(-n)=1, h_8(-n)=0$.
		\end{itemize}
	\end{theorem}
\end{frame}


\begin{frame}{非同余数: $s_2(n)=2$ 情形}
	\onslide<+->
	\begin{theorem}[][{Zhang 2023}]
		\label{thm:3}
		若 $n$ 是模 $8$ 余 $\pm1$ 素数乘积, 则下述等价:
		\begin{itemize}[<*>]
			\item $2n$ 是非同余数且 $\Sha(E_{2n})[2^\infty]\cong(\BZ/2\BZ)^2$;
			\item $h_4(-n)=1,d\equiv 9\bmod 16$,
		\end{itemize}
		其中 $d\mid n$ 满足 $(d,n)_v=1,\forall v$ 且 $d\ne 1,d\equiv 1\bmod 4$.
	\end{theorem}
	\begin{itemize}
		\item 这实际上也等价于 $h_4(-n)=1,h_8(-n)+h_8(-2n)=1$.
		\item 此外, Qin (2021) 证明了当素数 $p\equiv 1\bmod 8$ 且 $r_8(K_2\CO_p)=0$ 时, $p$ 是非同余数.
		且若此时 $r_4(K_2\CO_p)=1$, 则 $\Sha(E_p/\BQ)[2^\infty]\cong(\BZ/4\BZ)^2$.
		\item 这里 $K_2$ 是Milnor $K$ 群 (或叫 tame kernel).
	\end{itemize}  
\end{frame}


\section{主要结果}
\begin{frame}{设定}
	\begin{itemize}
		\item 这些结论都是指定 $n$ 的素因子落在某个同余类中来研究.
		\item 我们想要考虑的问题略有不同, 我们希望从一个满足 $s_2(Q)=0$ 的非同余数 $Q$ 出发, 构造它的一个倍数 $n=PQ$, 使得 $n$ 依然是非同余数.
		\item 设 $P=p_1\cdots p_k$, 其中素因子 $p_i\equiv 1\bmod8$.
		\item 设 $Q=\gcd(2,Q) q_1\cdots q_\ell$.
	\end{itemize}  
\end{frame}


\begin{frame}{假设}
	\begin{itemize}
		\item 假设存在 $\BF_2$ 上的向量 $\bfu=(u_1,\dots,u_k)^\rmT$, $\bfv=(v_1,\dots,v_\ell)^\rmT$,
		\item 使得 $\sum_i u_i=0,\sum_j v_j=1$, $\aleg{p_i}{q_j}=u_i v_j$.
		\item 这里 $\aleg{p_i}{q_j}=\log\leg{p_i}{q_j}$ 是加性勒让德符号, 其中 $\log:\{\pm1\}\simto \BF_2$.
		\item 换言之, 定义矩阵
		\[
			\bfA_{2P}=\bfA_P:=([p_j,-P]_{p_i})\in M_k(\BF_2),\qquad(\text{每行元素之和为 $0$})
		\]
		\item 并类似定义 $\bfA_Q$, $\bfA_n$, 则
		\[
			\bfA_n=\begin{pmatrix}
				\bfA_P+\bfU_P&\bfu\bfv^\rmT\\
				\bfv\bfu^\rmT&\bfA_Q
			\end{pmatrix},
		\]
		其中 ${\bf1}^\rmT\bfu=0,{\bf1}^\rmT\bfv=1$, $\bfU_P=\diag\{u_1,\dots,u_k\}$.
	\end{itemize}
\end{frame}


\begin{frame}{主要结果: $s_2(n)=0$ 情形}
	\onslide<+->
	\begin{theorem}
		在前述假设下, 如下等价:
		\begin{itemize}[<*>]
			\item $n$ 是非同余数且 $\Sha(E_n)[2^\infty]=0$;
			\item $\bfA_P+\bfU_P$ 可逆.
		\end{itemize}
	\end{theorem}
\end{frame}


\begin{frame}{主要结果: $s_2(n)=2$ 情形}
	\onslide<+->
	\begin{theorem}
		在前述假设下, 如下等价:
		\begin{itemize}[<*>]
			\item $n$ 是非同余数且 $\Sha(E_n)[2^\infty]\cong(\BZ/2\BZ)^2$;
			\item $\corank(\bfA_P+\bfU_P)=1$ 且 $\aleg{\gamma}d=\aleg{\sqrt2+1}{d}+1$,
		\end{itemize}
		其中 $0<d\mid P$ 满足 $d\ne 1, [d,-P]_{p_i}=u_i,\forall p_i\mid d$; $[d,-P]_{p_i}=0,\forall p_i\mid \frac Pd$;  $(\alpha,\beta,\gamma)$ 是 $d\alpha^2+\frac{n}d\beta^2=4\gamma^2$ 的一组本原正整数解.
	\end{theorem}
	\onslide<+->
	这里, 本原正整数解是指满足 $\gcd(\alpha,\beta,\gamma)=1$ 的正整数解.
\end{frame}


\begin{frame}{推论: $s_2(n)=2,\bfu={\bf0}$ 情形}
	\onslide<+->
	若取 $\bfu={\bf0}$ 则我们得到:
	\onslide<+->
	\begin{corollary}
		在前述假设下, 若 $\aleg{p_i}{q_j}=0, \forall i,j$, 则如下等价:
		\begin{itemize}[<*>]
			\item $n$ 是非同余数且 $\Sha(E_n)[2^\infty]\cong(\BZ/2\BZ)^2$;
			\item $h_4(-P)=1$ 且 $\aleg{\gamma}P=h_8(-P)$;
			\item $h_4(-P)=1$ 且 $\aleg{\gamma}P=r_4(K_2\CO_P)$,
		\end{itemize}
		其中 $(\alpha,\beta,\gamma)$ 是 $P\alpha^2+Q\beta^2=4\gamma^2$ 的一组本原正整数解.
	\end{corollary}
\end{frame}


\begin{frame}{推论: $s_2(n)=2,\ell=0$ 情形}
	\onslide<+->
	若 $\ell=0$, 即 $Q=1,2$, 则:
	\onslide<+->
	\begin{corollary}
		设 $n$ 是模 $8$ 余 $1$ 素数乘积.
		\begin{enumerate}
			\item 下述等价:
			\begin{itemize}[<*>]
				\item $n$ 是非同余数且 $\Sha(E_n)[2^\infty]\cong(\BZ/2\BZ)^2$;
				\item $h_4(-n)=1$ 且 $h_8(-n)=0$;
				\item $r_4(K_2\CO_n)=0$.
			\end{itemize}
			\item 下述等价:
			\begin{itemize}[<*>]
				\item $2n$ 是非同余数且 $\Sha(E_{2n})[2^\infty]\cong(\BZ/2\BZ)^2$;
				\item $h_4(-n)=1$ 且 $h_8(-n)+h_8(-2n)=1$;
				\item $r_4(K_2\CO_{-2n})=0$.
			\end{itemize}
		\end{enumerate}
	\end{corollary}
\end{frame}


\begin{frame}{主要结果: $s_2(n)\ge 2$}
	\begin{theorem}
		假设前述条件以及 $\aleg{p_i}{q_j}=0, \forall i,j$.
		若存在分解 $P=f_1\cdots f_r$ 满足
		\begin{itemize}[<*>]
			\item $h_4(-f_i)=1,\forall i$;
			\item $\aleg{p}{p'}=0$, 其中 $p\mid f_i,p'\mid f_j$ 是任意素因子, $i\ne j$;
			\item $\aleg{\gamma_i}{f_j}=0$,$\forall i\ne j$; $\aleg{\gamma_i}{f_i}=h_8(-f_i)$,
		\end{itemize}
		则 $n$ 是非同余数且 $\Sha(E_n)[2^\infty]\cong(\BZ/2\BZ)^{2r}$,
		其中 $(\alpha_i,\beta_i,\gamma_i)$ 是  $f_i\alpha_i^2+\frac{n}{f_i}\beta_i^2=4\gamma_i^2$ 的一组本原正整数解.
	\end{theorem}
\end{frame}


\begin{frame}{推论: $s_2(n)\ge 2,\ell=0$ 情形}
	\begin{corollary}[near]
		设奇数 $n$ 的所有素因子均模 $8$ 余 $1$.
		\begin{enumerate}
			\item (Wang 2016) 若存在分解 $n=f_1\cdots f_r$ 使得
			\begin{itemize}[<*>]
				\item $h_4(-f_i)=1,h_8(-f_i)=0,\forall i$;
				\item $h_8(-n)=r$; 或 $h_8(-n)=r-1,[(2,\sqrt{-n})]\notin\CA_{-n}^4$;
				\item $\aleg{p}{p'}=0$, 其中 $p\mid f_i,p'\mid f_j$ 是任意素因子, $i\ne j$,
			\end{itemize}
			则 $n$ 是非同余数且 $\Sha(E_n)[2^\infty]\cong(\BZ/2\BZ)^{2r}$.
			\item 若存在分解 $n=f_1\cdots f_r$ 使得
			\begin{itemize}[<*>]
				\item $h_4(-f_i)=1,h_8(-f_i)=0,\forall i$;
				\item $h_8(-2n)=r$;
				\item $\aleg{p}{p'}=0$, 其中 $p\mid f_i,p'\mid f_j$ 是任意素因子, $i\ne j$,
			\end{itemize}
			则 $2n$ 是非同余数且 $\Sha(E_{2n})\cong(\BZ/2\BZ)^{2r}$.
			\meddel
		\end{enumerate}
	\end{corollary}
\end{frame}


\section{下降法}
\begin{frame}{2-下降法}
	\begin{itemize}
		\item 证明主要工具是\emphov{下降法}{descent method}.
		\item 根据2-下降法, $\Sel_2(E_n)$ 可等同于集合
		\[
			\{\Lambda=(d_1,d_2,d_3)\in(\BQ^\times/\BQ^{\times2})^3:
			D_\Lambda(\BA_\BQ)\ne \emptyset,d_1d_2d_3\equiv 1\bmod\BQ^{\times 2}\},
		\]
		\item 其中 $D_\Lambda$ 是齐性空间
		\[
			\begin{cases}
				H_1:& -nt^2+d_2u_2^2-d_3u_3^2=0,\\
				H_2:& -nt^2+d_3u_3^2-d_1u_1^2=0,\\
				H_3:& 2nt^2+d_1u_1^2-d_2u_2^2=0.
			\end{cases}
		\]
		\item 一般地 $E(\BQ)\ni (x,y)\mapsto (x-n,x+n,x)$,
		\item $O\mapsto (1,1,1),(n,0)\mapsto (2,2n,n),(-n,0)\mapsto (-2n,2,-n),(0,0)\mapsto (-n,n,-1)$.
	\end{itemize}
\end{frame}


\begin{frame}{Monsky 矩阵: 奇数 $n$}
	\begin{itemize}
		\item 通过对这些齐性空间可解性的分析, Monsky 将 $\Sel_2'(E_n)$ 与一 $\BF_2$ 上矩阵 $\bfM_n$ 的核联系起来.
		\item 当 $n=p_1\cdots p_k$ 是奇数时, $\Sel_2'(E_n)$ 中的元素可选取一代表元 $(d_1,d_2,d_3)$ 使得 $d_1,d_2, d_3$ 均为 $n$ 的正因子.
		\item 此时
		\[
			\begin{aligned}
				\Sel_2'(E_n)&\ra\Ker\bfM_n,\quad \bfM_n=\begin{pmatrix}
					\bfA_n+\bfD_{n,2}&\bfD_{n,2}\\
					\bfD_{n,2}&\bfA_n+\bfD_{n,-2}
				\end{pmatrix}\\
				(d_1,d_2,d_3)&\mapsto\begin{pmatrix}
					\psi_n(d_2)\\ \psi_n(d_1)
				\end{pmatrix},
			\end{aligned}
		\]
		\item 其中 $\psi_n(d):=\bigl(v_{p_1}(d),\dots,v_{p_k}(d)\bigr)^\rmT\in\BF_2^k$,
		\item $\bfD_{n,\varepsilon}:=\diag\Bigl\{\aleg{\varepsilon}{p_1},\dots,\aleg{\varepsilon}{p_k}\Bigr\}$.
	\end{itemize}
\end{frame}


\begin{frame}{Monsky 矩阵: 偶数 $2n$}
	\begin{itemize}
		\item 类似地, $\Sel_2'(E_{2n})$ 中的元素可选取一代表元 $(d_1,d_2,d_3)$ 使得 $d_1,d_2, d_3$ 均为 $n$ 的因子且$d_2>0$, $d_3\equiv1\bmod4$.
		\item 此时
		\[
			\begin{aligned}
				\Sel_2'(E_{2n})&\ra\Ker\bfM_{2n},\quad \bfM_{2n}=\begin{pmatrix}
					\bfA_n^\rmT+\bfD_2&\bfD_{n,-1}\\
					\bfD_{n,2}&\bfA_n+\bfD_{n,2}
				\end{pmatrix}\\
				(d_1,d_2,d_3)&\mapsto\begin{pmatrix}
					\psi_n(|d_3|)\\ \psi_n(d_2)
				\end{pmatrix},
			\end{aligned}
		\]
		\item 两种情形下均有
		\[
			s_2(n)=\dim_{\BF_2}\Sel_2'(E_n)=\corank \bfM_n.
		\]
	\end{itemize}
\end{frame}


\begin{frame}{Cassels 配对}
	\begin{itemize}
		\item 仅知道 $s_2(n)=\rank_\BZ E_n(\BQ)+\dim_{\BF_2}\Sha(E_n)[2]$ 还不足以得到非同余数.
		\item Cassels 在 $\Sel_2'(E_n)$ 上定义了一个(反)对称双线性型 $\pair{-,-}\in\BF_2$.
		\item 对于 $\Lambda\in\Sel_2(E_n)$, $H_i$ 局部均可解, 从而存在整体解 $Q_i$ (Hasse-Minkowski 原理).
		\item 令 $L_i$ 为一线性型, 使得它定义了 $H_i$ 在 $Q_i$ 处的切平面.
		\item 对于 $\Lambda'=(d'_1,d'_2,d'_3)\in\Sel_2(E_n)$, 定义
		\[
			\pair{\Lambda,\Lambda'}=\sum_v\pair{\Lambda,\Lambda'}_v\in\BF_2,
			\quad
			\pair{\Lambda,\Lambda'}_v
			=\sum_{i=1}^3 \bigl[L_i(P_v),d_i'\bigr]_v,
		\]
		\item 其中对 $\BQ$ 的任意素位 $v$, 选取 $P_v\in D_\Lambda(\BQ_v)$.
		\item 这是一个有限和: 当 $v\nmid 2\infty$, $H_i,L_i$ 系数的分母, 且 $D_\Lambda$ 和 $L_i=0$ 模 $v$ 后依然分别是亏格 $1$ 的曲线和它的一个切平面时, $\pair{-,-}_v=0$.
	\end{itemize}
\end{frame}


\begin{frame}{Cassels 配对的核}
	\onslide<+->
	\begin{lemma}[][Wang2016]
		$n$ 是非同余数且 $\Sha(E_n)[2^\infty]\cong(\BZ/2\BZ)^{s_2(n)}\iff$
		$\Sel_2'(E_n)$ 上 Cassels 配对非退化.
	\end{lemma}
	\begin{itemize}
		\item 由正合列
		\[
			0\ra E_n[2]\ra E_n[4] \sto{\times 2} E_n[2]\ra 0
		\]
		得到长正合列
		\[
			0\ra {E_n(\BQ)[2]}/{2E_n(\BQ)[4]}\ra \Sel_2(E_n)\ra \Sel_4(E_n)\ra \Im \Sel_4(E_n)\ra 0,
		\]
		\item 其中 $\Im \Sel_4(E_n)$ 是映射 $\Sel_4(E_n)\sto{\times 2} \Sel_2(E_n)$ 的像.
		\item 而 $\Sel_2(E_n)$ 上Cassels 配对的核就是这个像.
		\item 因此引理左侧等价于 $\#\Sel_2(E_n) = \#\Sel_4(E_n)$, 
		\item 等价于 $\Im \Sel_4(E_n)=E_n[2]\subseteq \Sel_2(E_n)$, 等价于引理右侧.
	\end{itemize}
\end{frame}


\section{结论证明}
\begin{frame}{Selmer 群的计算}
	\begin{itemize}
		\item 现在我们开始证明主要结果.
	\end{itemize}
	\onslide<+->
	\begin{lemma}
		在前述假设下, $s_2(n)=2\corank(\bfA_P+\bfU_P)$.
	\end{lemma}
	\begin{itemize}
		\item 我们只证明 $n$ 是奇数的情形, 偶数情形类似,
		\item 设
		\[
			\begin{pmatrix}
				\bfx\\\bfy\\\bfz\\\bfw
			\end{pmatrix}\in\Ker\bfM_n
			=\Ker\begin{pmatrix}
				\bfA_P+\bfU_P&\bfu\bfv^\rmT&\bfO_k&\\
				\bfv\bfu^\rmT&\bfA_Q+\bfD_{Q,2}&&\bfD_{Q,2}\\
				\bfO_k&&\bfA_P+\bfU_P&\bfu\bfv^\rmT\\
				&\bfD_{Q,2}&\bfv\bfu^\rmT&\bfA_Q+\bfD_{Q,-2}
			\end{pmatrix}.
		\]
	\end{itemize}
\end{frame}
	

\begin{frame}{Selmer 群的计算 (续)}
	\begin{itemize}
		\item 
		则
		\[
			(\bfA_P+\bfU_P)\bfx=\bfu\bfv^\rmT\bfy,\qquad
			(\bfA_P+\bfU_P)\bfz=\bfu\bfv^\rmT\bfw
		\]
		\[
			\bfM_Q\begin{pmatrix}
				\bfy\\\bfw
			\end{pmatrix}
			=\begin{pmatrix}
				\bfv\bfu^\rmT\bfx\\
				\bfv\bfu^\rmT\bfz
			\end{pmatrix}.
		\]
		\item 从 ${\bf1}^\rmT(\bfA_P+\bfU_P)\bfx={\bf1}^\rmT\bfu\bfv^\rmT\bfy$ 得到 $\bfu^\rmT\bfx=0$. \hfill (假设 ${\bf1}^\rmT\bfu=0$)
		\item 同理 $\bfu^\rmT\bfz=0$, 故 $\bfM_Q\begin{pmatrix}
				\bfy\\\bfw
			\end{pmatrix}={\bf0}$.
		\item 由于 $s_2(Q)=0$, $\bfM_Q$ 可逆, 从而 $\bfy=\bfw={\bf0}$,
		\item $\bfx,\bfz\in\Ker(\bfA_P+\bfU_P)$, $s_2(n)=2\corank(\bfA_P+\bfU_P)$.
		\item 由此可立得主要结论中 $s_2(n)=0$ 的情形.
	\end{itemize}
\end{frame}


\begin{frame}{Cassles 配对的计算}
	\onslide<+->
	\begin{proposition}
		设 $0<f_i,f_j\mid P$ 满足 $\gcd(f_i,f_j)=1$, $\psi_P(f_i),\psi_P(f_j)\in \Ker(\bfA_P+\bfU_P)$.
		令 $\Lambda_t=(f_t,1,f_t), \Lambda'_t=(f_t,f_t,1)$,
		那么
		\begin{align*}
			\pair{\Lambda_i',\Lambda_i}&=\aleg{\sqrt2+1}{f_i}+\aleg{\gamma_i}{f_i}=\aleg{\sqrt2+1}{f_i}+\aleg{\gamma_i'}{f_i},\\
			\pair{\Lambda_i',\Lambda_j}&=\aleg{\gamma_i}{f_j}=\aleg{\gamma_j'}{f_i},\\
			\pair{\Lambda_i',\Lambda_i'}&=\aleg{\gamma_i\gamma_i'}{f_i},\qquad
			\pair{\Lambda_i',\Lambda_j'}=\aleg{\gamma_i\gamma_i'}{f_j},
		\end{align*}
		其中 $(\alpha_i,\beta_i,\gamma_i),(\alpha_i',\beta_i',\gamma_i')$ 分别是方程 $f_i\alpha_i^2+\frac n{f_i}\beta_i^2=4\gamma_i^2$, $f_i\alpha_i'^2-\frac n{f_i}\beta_i'^2=4\gamma_i'^2$ 的本原正整数解.
	\end{proposition}
\end{frame}


\begin{frame}{Cassles 配对的计算 (续)}
	\onslide<+->
	\[
		D_{\Lambda_i}:
		\begin{cases}
			H_1:& -nt^2+u_2^2-f_i u_3^2=0,\\
			H_2:& -\frac n{f_i} t^2+u_3^2-u_1^2=0,\\
			H_3:& 2nt^2+f_i u_1^2-u_2^2=0.
		\end{cases}
	\]
	\begin{itemize}
		\item 取
		\begin{align*}
			Q_1&=(\beta_i',f_i\alpha_i',2\gamma_i')\in H_1(\BQ),&
			L_1&=\frac{n}{f_i}\beta_i' t-\alpha_i' u_2+2\gamma_i' u_3,\\
			\onslide<+->{Q_2}&\onslide<.->{=(0,1,-1)\in H_2(\BQ),}&
			\onslide<.->{L_2}&\onslide<.->{=u_3+u_1.}
		\end{align*}
		\item 根据假设不难得到
		\[
			\aleg{f_i}{q_s}=0,\quad
			\aleg{n/f_i}{p}=0,\forall p\mid f_i,\quad
			\aleg{f_i}{p}=0,\forall p\mid \frac{P}{f_i}.
		\]
	\end{itemize}
\end{frame}


\begin{frame}{Cassles 配对的计算 (再续)}  
	\begin{itemize}
		\item 对于 $v\mid f_i$, 取 $P_v=(t,u_1,u_2,u_3)=\bigl(1,\sqrt{-2\frac{n}{f_i}},0,\sqrt{-\frac{n}{f_i}}\bigr)$.
		这里根号取正负不影响最后的结果.
		\item $\bigl[L_1(P_v),f_t\bigr]_v
			=\bigl[\beta_i'\frac{n}{f_i}+2\gamma_i'\sqrt{-\frac{n}{f_i}},f_t\bigr]_v$
			\onslide<+->{$
			=\bigl[4\gamma_i'\sqrt{-\frac{n}{f_i}},f_t\bigr]_v
			=\bigl[\gamma_i'\sqrt{-\frac{n}{f_i}},f_t\bigr]_v$.}
		\item $\bigl[L_2(P_v),f_t\bigr]_v
			=\bigl[(\sqrt2+1)\sqrt{-\frac{n}{f_i}},f_t\bigr]_v$,$\quad$
			\onslide<+->{$\bigl[L_1L_2(P_v),f_t\bigr]_v
				=\bigl[(\sqrt2+1)\gamma_i',f_t\bigr]_v$.}
		\item 对于 $v\mid \frac P{f_i}$, 取 $P_v=(t,u_1,u_2,u_3)=\bigl(0,1,\sqrt{f_i},1\bigr)$.
		\item 类似可得 $\bigl[L_1L_2(P_v),f_t\bigr]_v=[\gamma_i',f_t]_v$.
		\item $\pair{\Lambda_i,\Lambda_i'}
				=\suml_{v\mid f_i}\bigl[(\sqrt2+1)\gamma_i',f_i\bigr]_v
				+\suml_{v\mid \frac{P}{f_i}}[\gamma_i',f_i]_v
				=\aleg{(\sqrt2+1)\gamma_i'}{f_i}$.
		\item $\pair{\Lambda_i,\Lambda_j'}
				=\suml_{v\mid f_i}\bigl[(\sqrt2+1)\gamma_i',f_j\bigr]_v
				+\suml_{v\mid \frac{P}{f_i}}[\gamma_i',f_j]_v
				=\aleg{\gamma_i'}{f_j}$.
		\item 其它情形类似.
	\end{itemize}
\end{frame}


\begin{frame}{主要结论: $s_2(n)=2$ 情形}
	\begin{itemize}
		\item 根据前面的计算 $s_2(n)=2\iff \corank(\bfA_P+\bfU_P)=1$.
		\item 此时 $\Sel_2'(E_n)$ 由 $\Lambda=(d,1,d),\Lambda'=(d,d,1)$ 生成, 其中 $\psi_P(d)\in\Ker(\bfA_P+\bfU_P)$.
		\item 于是 $\pair{\Lambda,\Lambda'}
		=\aleg{\sqrt2+1}d+\aleg{\gamma}{d}$.
	\end{itemize}
	\onslide<+->
	若进一步假设 $\bfu={\bf0}$, 则 $d=P$.
\end{frame}


\begin{frame}{高斯型理论}
	\begin{itemize}
		\item 为了将我们的结果与类群、$K_2$ 群联系起来, 我们回顾有关结论.
		\item 设 $n=p_1\cdots p_k\equiv 1\bmod 4$.
		\item 根据高斯型理论, $h_2(-n)=k+1$,  $h_4(-n)=\corank \bfR_{-n}-1$,
		\item 其中R\`edei 矩阵 $\bfR_{-n}=\begin{pmatrix}
			\bfA_n&\bfb_{n,2}\\
			\bfb_{n,-1}^\rmT&\aleg{2}{n}
		\end{pmatrix}$, $\bfb_{n,\varepsilon}=\bfD_{n,\varepsilon}{\bf1}$.
		\item 对于 $\theta_{-n}(d):=[(d,\sqrt{-n})]\in\CA_{-n}[2]$, $\theta_{-n}(d)\in \CA_{-n}^4\iff \bfb_{n,\gamma}\in \Im\bfR_{-n}'$.
		\item 这里 $\bfR_{-n}'$ 是 $\bfR_{-n}$ 去掉最后一行, $(\alpha,\beta,\gamma)$ 是 $d\alpha^2+\frac{n}{d}\beta^2=4\gamma^2$ 的本原正整数解.
		\item 对于 $h_{2^a}(-2n)$, 我们有类似结论.
	\end{itemize}
\end{frame}


\begin{frame}{$K_2$ 群}
	\begin{itemize}
		\item 根据 Browkin-Schinzel (1982) 和 Qin (1995) 的工作, 
		\item 当 $m=n,2n>2$ 时
		\[
			2^{r_4(K_2\CO_m)+1}=\#\{\bfx:\bfB_m\bfx=\bfb_{n,\pm1},\bfb_{n,\pm2},\bfb_{n,\pm\mu}\}.
		\]
		\item 当 $m=-n,-2n<-2$ 时
		\[
			\begin{aligned}	
				2^{r_4(K_2\CO_m)+2}&
				=\begin{cases}
					\#\{\bfx:\bfB_m\bfx={\bf0},\bfb_{n,2},\bfb_{n,\mu}\},&\text{若}\ \bfb_{n,-1}\notin\Im \bfB_m;\\
					2\#\{\bfx:\bfB_m\bfx={\bf0},\bfb_{n,2},\bfb_{n,\mu}\},&\text{若}\ \bfb_{n,-1}\in\Im \bfB_m.
				\end{cases}
			\end{aligned}
		\]
		\item 其中 $\bfB_m=\bfA_n+\bfD_{n,m/n}$.
	\end{itemize}
\end{frame}


\begin{frame}{类群, $K_2$ 群和矩阵 $\bfA_n$ 的联系}
	\begin{itemize}
		\item 设 $n$ 是模 $8$ 余 $1$ 素数乘积.
		\begin{itemize}
			\item $h_4(-n)=h_4(-2n)=\corank\bfA_n$.
			\item $r_4(K_2\CO_n)=0\iff h_4(-n)=1,h_8(-n)=0$.
			\item $r_4(K_2\CO_{-2n})=0$ $\iff$ $h_4(-n)=1,h_8(-n)+h_8(-2n)=1$.
			\item 若 $h_4(-n)=1$, 则 $h_8(-n)=1-\aleg{\sqrt2+1}{n}$,  $h_8(-2n)=1-\aleg{\sqrt2}{n}$, $r_4(K_2\CO_{-2n}),r_4(K_2\CO_n)\le 1$.
		\end{itemize}
		\item 由此可得 $s_2(n)=0,\bfu={\bf0}$ 情形的结论.
	\end{itemize}
\end{frame}


\begin{frame}{主要结论: $s_2(n)=2k$ 情形}
	\begin{itemize}
		\item 此时 $\bfA_P+\bfU_P=\bfA_P=\diag\bigl\{
				\bfA_{f_1},\cdots \bfA_{f_r}
			\bigr\}$.
		\item 由 $h_4(-f_i)=1$ 可知 $\corank \bfA_{f_i}=1$, $s_2(n)=2r$.
		\item 此时由 $\Ker\bfM_n$ 可知 $\Sel_2'(E_n)$ 由 $\Lambda_i=(f_i,1,f_i),\Lambda_i'=(f_i,f_i,1)$ 生成.
		\item 相应的Cassels配对对应矩阵
		\[
			\bfX=\begin{pmatrix}
				*&\bfB^\rmT+\bfC\\\bfB+\bfC&\bfB+\bfB^\rmT
			\end{pmatrix},
		\]
		\begin{align*}
			\bfB&
			=\Bigl(\aleg{\gamma_i}{f_j}\Bigr)_{r\times r}
			=\diag\Bigl\{h_8(-f_1),\dots,h_8(-f_r)\Bigr\},\\
			\bfC&
			=\diag\Bigl\{\aleg{\sqrt2+1}{f_1},\dots,\aleg{\sqrt2+1}{f_r}\Bigr\}
			=\diag\bigl\{1-h_8(-f_1),\dots,1-h_8(-f_r)\bigr\}.
		\end{align*}
		\item 因此 $\bfX=\begin{pmatrix}
			*&\bfI\\\bfI&\bfO
		\end{pmatrix}$ 可逆, Cassels 配对非退化.
	\end{itemize}
\end{frame}


\begin{frame}{推论: $s_2(2n)=2k$ 情形}
	\begin{itemize}
		\item 我们来看偶数 $2n$ 情形.
		\item 此时在假设条件下,
		\[
			\bfR_{-2n}=\diag\{\bfA_n,0\}
			=\diag\{\bfA_{f_1},\cdots \bfA_{f_r},0\}.
		\]
		\item 所以 $h_4(-2n)=r$ 且 $\CA_{-2n}[2]\cap \CA_{-2n}^2$ 由 $\theta_{-2n}(f_1),\dots,\theta_{-2n}(f_{r-1})$ 生成.
		\item 由 $h_8(-2n)=r$ 可知它们都属于 $\CA_{-2n}^4$.
		\item 从而 $\bfb_{n,\gamma_i}\in\Im \bfA_n$, 
		\[
			0={\bf1}^\rmT\bfb_{f_j,\gamma_i}=\aleg{\gamma_i}{f_j}.
		\]
		\item 奇数情形类似.
	\end{itemize}
\end{frame}


\end{document}



