% -*- coding: utf-8 -*-
\documentclass{beamer}
\mode<presentation>
{
  \usetheme{Madrid}
  %\setbeamertemplate{background canvas}[vertical shading][bottom=red!10,top=blue!10]
  %\setbeamertemplate{blocks}[rounded][shadow=true]
  %\setbeamertemplate{footline}[frame number]
  %\setbeamercovered{transparent}
  %\usefonttheme[onlysmall]{structurebold}
  % \usefonttheme[onlymath]{serif}
}


%\usepackage{CJKutf8}
\usepackage{ctex}
\usepackage{latexsym}
\usepackage{amssymb}
\usepackage{color}
\usepackage{amsmath}
\usepackage{graphicx}
\usepackage{cases}
\usepackage{wasysym}
\usepackage{amsthm}
\usepackage[all]{xy}
\usepackage{amsfonts}
\usepackage{fancyhdr}
\usepackage{beamerouterthemeprogressbarjma} 
\setlength{\parindent}{2em}
\setlength{\parskip}{1em}

% symbols
\newcommand\ra{\rightarrow}
\newcommand\lra{\longrightarrow}
\newcommand\la{\leftarrow}
\newcommand\lla{\longleftarrow}
\newcommand\sqra{\rightsquigarrow}
\newcommand\sqlra{\leftrightsquigarrow}
\newcommand\inj{\hookrightarrow}
\newcommand\surj{\twoheadrightarrow}
\newcommand\sto[1]{\stackrel{#1}{\longrightarrow}}
\newcommand\lsto[1]{\stackrel{#1}{\longleftarrow}}
\newcommand\simto{\sto{\sim}}
\newcommand\lto{\longmapsto}
\renewcommand\vec[1]{\overrightarrow{#1}}
\newcommand\ilim{\varinjlim\limits}
\newcommand\plim{\varprojlim\limits}
\newcommand\wh{\widehat}
\newcommand\wt{\widetilde}
\newcommand\ov{\overline}
\newcommand\ul{\underline}
\newcommand\vare{\varepsilon}
\newcommand\varp{\varphi}
\newcommand\imply{\Longrightarrow}
\newcommand\rj{\relbar\joinrel}
\newcommand\half{\frac{1}{2}}
\newcommand\mmid{\parallel}
\newcommand\ldb{\llbracket}
\newcommand\rdb{\rrbracket}
\newcommand\pppi{\frac{\partial\bar\partial}{\pi i}}
\newcommand\mmod{\underline\bmod}
\font\cyr=wncyr10 \newcommand\Sha{\hbox{\cyr X}}


% functions
\newcommand\fct[4]{\begin{split}#1 &\lra #2 \\ #3 &\lto #4\end{split}}
\newcommand\set[1]{\left\{#1\right\}}
\newcommand\pair[1]{\langle{#1}\rangle}
\newcommand\pmat[4]{\begin{pmatrix}#1 & #2 \\ #3 & #4\end{pmatrix} }
\newcommand\smat[4]{\left(\begin{smallmatrix}#1 & #2 \\ #3 & #4\end{smallmatrix}\right) }
\newcommand\bvec[2]{\begin{bmatrix}#1 \\ #2\end{bmatrix} }
\newcommand\norm[1]{\!\parallel\!{#1}\!\parallel\!}
\newcommand\env[2]{\begin{#1}{#2}\end{#1}}
\newcommand\envn[3]{\begin{#1}[#2]{#3}\end{#1}}
\newcommand\red[1]{\textcolor{red}{#1}} %Marker
\newcommand\hil[3]{\left(\frac{{#1},{#2}}{#3}\right)} %Hilbert symbol
\newcommand\leg[2]{\left(\frac{{#1}}{#2}\right)} %Legdred symbol
\newcommand\stsc[2]{\genfrac{}{}{0pt}{}{#1}{#2}} %atop


% categories
\newcommand\cA{{\mathsf{A}}}
\newcommand\cB{{\mathsf{B}}}
\newcommand\cC{{\mathsf{C}}}
\newcommand\cAb{{\mathsf{Ab}}}
\newcommand\cBT{{\mathsf{BT}}}
\newcommand\cBun{{\mathsf{Bun}}}
\newcommand\cCharLoc{{\mathsf{CharLoc}}}
\newcommand\cCoh{{\mathsf{Coh}}}
\newcommand\cComm{{\mathsf{Comm}}}
\newcommand\cEt{{\mathsf{Et}}}
\newcommand\cFppf{{\mathsf{Fppf}}}
\newcommand\cFpqc{{\mathsf{Fpqc}}}
\newcommand\cFunc{{\mathsf{Func}}}
\newcommand\cGroups{{\mathsf{Groups}}}
\newcommand\cGrpd{{\{\mathsf{Grpd}\}}}
\newcommand\cHo{{\mathsf{Ho}}}
\newcommand\cIso{{\mathsf{Iso}}}
\newcommand\cLoc{{\mathsf{Loc}}}
\newcommand\cMod{{\mathsf{Mod}}}
\newcommand\cModFil{{\mathsf{ModFil}}}
\newcommand\cNilp{{\mathsf{Nilp}}}
\newcommand\cPerf{{\mathsf{Perf}}}
\newcommand\cPN{{\mathsf{PN}}}
\newcommand\cRep{{\mathsf{Rep}}}
\newcommand\cRings{{\mathsf{Rings}}}
\newcommand\cSets{{\mathsf{Sets}}}
\newcommand\cStack{{\mathsf{Stack}}}
\newcommand\cSch{{\mathsf{Sch}}}
\newcommand\cTop{{\mathsf{Top}}}
\newcommand\cVect{{\mathsf{Vect}}}
\newcommand\cZar{{\mathsf{Zar}}}
\newcommand\cphimod{{\varphi\txt{-}\mathsf{Mod}}}
\newcommand\cphimodfil{{\varphi\txt{-}\mathsf{ModFil}}}

% font
\newcommand\rma{{\mathrm{a}}}  \newcommand\rmb{{\mathrm{b}}}  \newcommand\rmc{{\mathrm{c}}}  \newcommand\rmd{{\mathrm{d}}}
\newcommand\rme{{\mathrm{e}}}  \newcommand\rmf{{\mathrm{f}}}  \newcommand\rmg{{\mathrm{g}}}  \newcommand\rmh{{\mathrm{h}}}
\newcommand\rmi{{\mathrm{i}}}  \newcommand\rmj{{\mathrm{j}}}  \newcommand\rmk{{\mathrm{k}}}  \newcommand\rml{{\mathrm{l}}}
\newcommand\rmm{{\mathrm{m}}}  \newcommand\rmn{{\mathrm{n}}}  \newcommand\rmo{{\mathrm{o}}}  \newcommand\rmp{{\mathrm{p}}}
\newcommand\rmq{{\mathrm{q}}}  \newcommand\rmr{{\mathrm{r}}}  \newcommand\rms{{\mathrm{s}}}  \newcommand\rmt{{\mathrm{t}}}
\newcommand\rmu{{\mathrm{u}}}  \newcommand\rmv{{\mathrm{v}}}  \newcommand\rmw{{\mathrm{w}}}  \newcommand\rmx{{\mathrm{x}}} \newcommand\rmy{{\mathrm{y}}}  \newcommand\rmz{{\mathrm{z}}}

\newcommand\rmA{{\mathrm{A}}}  \newcommand\rmB{{\mathrm{B}}}  \newcommand\rmC{{\mathrm{C}}}  \newcommand\rmD{{\mathrm{D}}}
\newcommand\rmE{{\mathrm{E}}}  \newcommand\rmF{{\mathrm{F}}}  \newcommand\rmG{{\mathrm{G}}}  \newcommand\rmH{{\mathrm{H}}}
\newcommand\rmI{{\mathrm{I}}}  \newcommand\rmJ{{\mathrm{J}}}  \newcommand\rmK{{\mathrm{K}}}  \newcommand\rmL{{\mathrm{L}}}
\newcommand\rmM{{\mathrm{M}}}  \newcommand\rmN{{\mathrm{N}}}  \newcommand\rmO{{\mathrm{O}}}  \newcommand\rmP{{\mathrm{P}}}
\newcommand\rmQ{{\mathrm{Q}}}  \newcommand\rmR{{\mathrm{R}}}  \newcommand\rmS{{\mathrm{S}}}  \newcommand\rmT{{\mathrm{T}}}
\newcommand\rmU{{\mathrm{U}}}  \newcommand\rmV{{\mathrm{V}}}  \newcommand\rmW{{\mathrm{W}}}  \newcommand\rmX{{\mathrm{X}}}
\newcommand\rmY{{\mathrm{Y}}}  \newcommand\rmZ{{\mathrm{Z}}}

\newcommand\bfa{{\mathbf{a}}}  \newcommand\bfb{{\mathbf{b}}}  \newcommand\bfc{{\mathbf{c}}}  \newcommand\bfd{{\mathbf{d}}}
\newcommand\bfe{{\mathbf{e}}}  \newcommand\bff{{\mathbf{f}}}  \newcommand\bfg{{\mathbf{g}}}  \newcommand\bfh{{\mathbf{h}}}
\newcommand\bfi{{\mathbf{i}}}  \newcommand\bfj{{\mathbf{j}}}  \newcommand\bfk{{\mathbf{k}}}  \newcommand\bfl{{\mathbf{l}}}
\newcommand\bfm{{\mathbf{m}}}  \newcommand\bfn{{\mathbf{n}}}  \newcommand\bfo{{\mathbf{o}}}  \newcommand\bfp{{\mathbf{p}}}
\newcommand\bfq{{\mathbf{q}}}  \newcommand\bfr{{\mathbf{r}}}  \newcommand\bfs{{\mathbf{s}}}  \newcommand\bft{{\mathbf{t}}}
\newcommand\bfu{{\mathbf{u}}}  \newcommand\bfv{{\mathbf{v}}}  \newcommand\bfw{{\mathbf{w}}}  \newcommand\bfx{{\mathbf{x}}} \newcommand\bfy{{\mathbf{y}}}  \newcommand\bfz{{\mathbf{z}}}

\newcommand\bfA{{\mathbf{A}}}  \newcommand\bfB{{\mathbf{B}}}  \newcommand\bfC{{\mathbf{C}}}  \newcommand\bfD{{\mathbf{D}}}
\newcommand\bfE{{\mathbf{E}}}  \newcommand\bfF{{\mathbf{F}}}  \newcommand\bfG{{\mathbf{G}}}  \newcommand\bfH{{\mathbf{H}}}
\newcommand\bfI{{\mathbf{I}}}  \newcommand\bfJ{{\mathbf{J}}}  \newcommand\bfK{{\mathbf{K}}}  \newcommand\bfL{{\mathbf{L}}}
\newcommand\bfM{{\mathbf{M}}}  \newcommand\bfN{{\mathbf{N}}}  \newcommand\bfO{{\mathbf{O}}}  \newcommand\bfP{{\mathbf{P}}}
\newcommand\bfQ{{\mathbf{Q}}}  \newcommand\bfR{{\mathbf{R}}}  \newcommand\bfS{{\mathbf{S}}}  \newcommand\bfT{{\mathbf{T}}}
\newcommand\bfU{{\mathbf{U}}}  \newcommand\bfV{{\mathbf{V}}}  \newcommand\bfW{{\mathbf{W}}}  \newcommand\bfX{{\mathbf{X}}}
\newcommand\bfY{{\mathbf{Y}}}  \newcommand\bfZ{{\mathbf{Z}}}

\newcommand\BA{{\mathbb{A}}}  \newcommand\BB{{\mathbb{B}}}  \newcommand\BC{{\mathbb{C}}}  \newcommand\BD{{\mathbb{D}}}  \newcommand\BE{{\mathbb{E}}}
\newcommand\BF{{\mathbb{F}}}  \newcommand\BG{{\mathbb{G}}}  \newcommand\BH{{\mathbb{H}}}  \newcommand\BI{{\mathbb{I}}}  \newcommand\BJ{{\mathbb{J}}}
\newcommand\BK{{\mathbb{K}}}  \newcommand\BL{{\mathbb{L}}}  \newcommand\BM{{\mathbb{M}}}  \newcommand\BN{{\mathbb{N}}}  \newcommand\BO{{\mathbb{O}}}
\newcommand\BP{{\mathbb{P}}}  \newcommand\BQ{{\mathbb{Q}}}  \newcommand\BR{{\mathbb{R}}}  \newcommand\BS{{\mathbb{S}}}  \newcommand\BT{{\mathbb{T}}}
\newcommand\BU{{\mathbb{U}}}  \newcommand\BV{{\mathbb{V}}}  \newcommand\BW{{\mathbb{W}}}  \newcommand\BX{{\mathbb{X}}}  \newcommand\BY{{\mathbb{Y}}}
\newcommand\BZ{{\mathbb{Z}}}

\newcommand\CA{{\mathcal{A}}} \newcommand\CB{{\mathcal{B}}} \newcommand\CC{{\mathcal{C}}}\providecommand\CD{{\mathcal{D}}}\newcommand\CE{{\mathcal{E}}}
\newcommand\CF{{\mathcal{F}}} \newcommand\CG{{\mathcal{G}}} \newcommand\CH{{\mathcal{H}}} \newcommand\CI{{\mathcal{I}}} \newcommand\CJ{{\mathcal{J}}}
\newcommand\CK{{\mathcal{K}}} \newcommand\CL{{\mathcal{L}}} \newcommand\CM{{\mathcal{M}}} \newcommand\CN{{\mathcal{N}}} \newcommand\CO{{\mathcal{O}}}
\newcommand\CP{{\mathcal{P}}} \newcommand\CQ{{\mathcal{Q}}} \newcommand\CR{{\mathcal{R}}} \newcommand\CS{{\mathcal{S}}} \newcommand\CT{{\mathcal{T}}}
\newcommand\CU{{\mathcal{U}}} \newcommand\CV{{\mathcal{V}}} \newcommand\CW{{\mathcal{W}}} \newcommand\CX{{\mathcal{X}}} \newcommand\CY{{\mathcal{Y}}}
\newcommand\CZ{{\mathcal{Z}}}

\newcommand\RA{{\mathrm{A}}} \newcommand\RB{{\mathrm{B}}} \newcommand\RC{{\mathrm{C}}} \newcommand\RD{{\mathrm{D}}} \newcommand\RE{{\mathrm{E}}}
\newcommand\RF{{\mathrm{F}}} \newcommand\RG{{\mathrm{G}}} \newcommand\RH{{\mathrm{H}}} \newcommand\RI{{\mathrm{I}}} \newcommand\RJ{{\mathrm{J}}}
\newcommand\RK{{\mathrm{K}}} \newcommand\RL{{\mathrm{L}}} \newcommand\RM{{\mathrm{M}}} \newcommand\RN{{\mathrm{N}}} \newcommand\RO{{\mathrm{O}}}
\newcommand\RP{{\mathrm{P}}} \newcommand\RQ{{\mathrm{Q}}} \newcommand\RR{{\mathrm{R}}} \newcommand\RS{{\mathrm{S}}} \newcommand\RT{{\mathrm{T}}}
\newcommand\RU{{\mathrm{U}}} \newcommand\RV{{\mathrm{V}}} \newcommand\RW{{\mathrm{W}}} \newcommand\RX{{\mathrm{X}}} \newcommand\RY{{\mathrm{Y}}}
\newcommand\RZ{{\mathrm{Z}}}

\newcommand\msa{\mathscr{A}}  \newcommand\msb{\mathscr{B}}  \newcommand\msc{\mathscr{C}}  \newcommand\msd{\mathscr{D}}  \newcommand\mse{\mathscr{E}}
\newcommand\msf{\mathscr{F}}  \newcommand\msg{\mathscr{G}}  \newcommand\msh{\mathscr{H}}  \newcommand\msi{\mathscr{I}}  \newcommand\msj{\mathscr{J}}
\newcommand\msk{\mathscr{K}}  \newcommand\msl{\mathscr{L}}  \newcommand\msm{\mathscr{M}}  \newcommand\msn{\mathscr{N}}  \newcommand\mso{\mathscr{O}}
\newcommand\msp{\mathscr{P}}  \newcommand\msq{\mathscr{Q}}  \newcommand\msr{\mathscr{R}}  \newcommand\mss{\mathscr{S}}  \newcommand\mst{\mathscr{T}}
\newcommand\msu{\mathscr{U}}  \newcommand\msv{\mathscr{V}}  \newcommand\msw{\mathscr{W}}  \newcommand\msx{\mathscr{X}}  \newcommand\msy{\mathscr{Y}}
\newcommand\msz{\mathscr{Z}}

\newcommand\fa{{\mathfrak{a}}}\newcommand\fb{{\mathfrak{b}}}\newcommand\fc{{\mathfrak{c}}}\newcommand\fd{{\mathfrak{d}}}\newcommand\fe{{\mathfrak{e}}}
\newcommand\ff{{\mathfrak{f}}}\newcommand\fg{{\mathfrak{g}}}\newcommand\fh{{\mathfrak{h}}}\newcommand\fii{\mathfrak{i}} \newcommand\fj{{\mathfrak{j}}}
\newcommand\fk{{\mathfrak{k}}}\newcommand\fl{{\mathfrak{l}}}\newcommand\fm{{\mathfrak{m}}}\newcommand\fn{{\mathfrak{n}}}\newcommand\fo{{\mathfrak{o}}} \newcommand\fp{{\mathfrak{p}}}\newcommand\fq{{\mathfrak{q}}}\newcommand\fr{{\mathfrak{r}}}\newcommand\fs{{\mathfrak{s}}}\newcommand\ft{{\mathfrak{t}}}
\newcommand\fu{{\mathfrak{u}}}\newcommand\fv{{\mathfrak{v}}}\newcommand\fw{{\mathfrak{w}}}\newcommand\fx{{\mathfrak{x}}}\newcommand\fy{{\mathfrak{y}}}
\newcommand\fz{{\mathfrak{z}}}

\newcommand\fA{{\mathfrak{A}}}\newcommand\fB{{\mathfrak{B}}}\newcommand\fC{{\mathfrak{C}}}\newcommand\fD{{\mathfrak{D}}}\newcommand\fE{{\mathfrak{E}}}
\newcommand\fF{{\mathfrak{F}}}\newcommand\fG{{\mathfrak{G}}}\newcommand\fH{{\mathfrak{H}}}\newcommand\fI{{\mathfrak{I}}}\newcommand\fJ{{\mathfrak{J}}}
\newcommand\fK{{\mathfrak{K}}}\newcommand\fL{{\mathfrak{L}}}\newcommand\fM{{\mathfrak{M}}}\newcommand\fN{{\mathfrak{N}}}\newcommand\fO{{\mathfrak{O}}} \newcommand\fP{{\mathfrak{P}}}\newcommand\fQ{{\mathfrak{Q}}}\newcommand\fR{{\mathfrak{R}}}\newcommand\fS{{\mathfrak{S}}}\newcommand\fT{{\mathfrak{T}}}
\newcommand\fU{{\mathfrak{U}}}\newcommand\fV{{\mathfrak{V}}}\newcommand\fW{{\mathfrak{W}}} \newcommand\fX{{\mathfrak{X}}}\newcommand\fY{{\mathfrak{Y}}}\newcommand\fZ{{\mathfrak{Z}}}
% be careful about \fii

% a
\newcommand\ab{{\mathrm{ab}}}
\newcommand\ad{{\mathrm{ad}}}
\newcommand\Ad{{\mathrm{Ad}}}
\newcommand\adele{ad\'{e}le}
\newcommand\Adele{Ad\'{e}le}
\newcommand\adeles{ad\'{e}les}
\newcommand\adelic{ad\'{e}lic}
\newcommand\AJ{{\mathrm{AJ}}}
\newcommand\alb{{\mathrm{alb}}}
\newcommand\Alb{{\mathrm{Alb}}}
\newcommand\alg{{\mathrm{alg}}}
\newcommand\an{{\mathrm{an}}}
\newcommand\ann{{\mathrm{ann}}}
\newcommand\Ann{{\mathrm{Ann}}}
\newcommand\Arg{{\mathrm{Arg}}}
\newcommand\arith{{\mathrm{arith}}}
\newcommand\Art{{\mathrm{Art}}}
\newcommand\AS{{\mathrm{AS}}}
\newcommand\Ass{{\mathrm{Ass}}}
\newcommand\Aut{{\mathrm{Aut}}}
% b
\newcommand\Bun{{\mathrm{Bun}}}
\newcommand\Br{{\mathrm{Br}}}
\newcommand\bs{\backslash}
\newcommand\BWt{{\mathrm{BW}}}
% c
\newcommand\can{{\mathrm{can}}}
\newcommand\cc{{\mathrm{cc}}}
\newcommand\cd{{\mathrm{cd}}}
\newcommand\ch{{\mathrm{ch}}}
\newcommand\Ch{{\mathrm{Ch}}}
\newcommand\Char{{\mathrm{char}\,}}
\newcommand\Chow{{\mathrm{CH}}}
\newcommand\circB{{\stackrel{\circ}{B}}}
\newcommand\cl{{\mathrm{cl}}}
\newcommand\Cl{{\mathrm{Cl}}}
\newcommand\cm{{\mathrm{cm}}}
\newcommand\cod{{\mathrm{cod}}}
\newcommand\coker{{\mathrm{coker}\,}}
\newcommand\Coker{{\mathrm{Coker}\,}}
\newcommand\cond{{\mathrm{cond}}}
\newcommand\codim{{\mathrm{codim}}}
\newcommand\cont{{\mathrm{cont}}}
\newcommand\Conv{{\mathrm{Conv}}}
\newcommand\corr{{\mathrm{corr}}}
\newcommand\Corr{{\mathrm{Corr}}}
\newcommand\coim{{\mathrm{coim}\,}}
\newcommand\CoIm{{\mathrm{CoIm}\,}}
\newcommand\cris{{\mathrm{cris}}}
\newcommand\Cris{{\mathrm{Cris}}}
\newcommand\CRIS{{\mathrm{CRIS}}}
\newcommand\crit{{\mathrm{crit}}} % critical
\newcommand\crys{{\mathrm{crys}}}
\newcommand\cusp{{\mathrm{cusp}}} % cuspidal
\newcommand\CWt{{\mathrm{CW}}} % Witt covector
\newcommand\cyc{{\mathrm{cyc}}} % cyclotomic
% d
\newcommand\Def{{\mathrm{Def}}} % deformation space
\newcommand\diag{{\mathrm{diag}}} % diagonal matrix
\newcommand\diff{\mathop{}\!\mathrm{d}} % differential
\newcommand\disc{{\mathrm{disc}}} % discriminant
\renewcommand\div{{\mathrm{div}}} % divisor
\newcommand\Div{{\mathrm{Div}}} % divisor group
\newcommand\dR{{\mathrm{dR}}} % de Rham
% e
\newcommand\End{{\mathrm{End}}} % endomorphism
\newcommand\ess{{\mathrm{ess}}} % essential
\newcommand\et{\text{\'{e}t}} % etale
\newcommand\etale{\'{e}tale} % etale
\newcommand\Etale{\'{E}tale} % etale
\newcommand\Ext{\mathrm{Ext}} % extension
\newcommand\CExt{\mathcal{E}\mathrm{xt}} % extension functor
% f
\newcommand\Fil{{\mathrm{Fil}}}
\newcommand\Fix{{\mathrm{Fix}}}
\newcommand\fppf{{\mathrm{fppf}}}
\newcommand\Fr{{\mathrm{Fr}\;}}
\newcommand\Frob{{\mathrm{Frob}}}
% g
\newcommand\Ga{\mathbb{G}_a}
\newcommand\Gm{\mathbb{G}_m}
\newcommand\hGa{\widehat{\mathbb{G}}_{a}}
\newcommand\hGm{\widehat{\mathbb{G}}_{m}}
\newcommand\Gal{{\mathrm{Gal}}}
\newcommand\gl{{\mathrm{gl}}}
\newcommand\GL{{\mathrm{GL}}}
\newcommand\GO{{\mathrm{GO}}}
\newcommand\geom{{\mathrm{geom}}}
\newcommand\Gr{{\mathrm{Gr}}}
\newcommand\gr{{\mathrm{gr}}}
\newcommand\GSO{{\mathrm{GSO}}}
\newcommand\GSp{{\mathrm{GSp}}}
\newcommand\GSpin{{\mathrm{GSpin}}}
\newcommand\GU{{\mathrm{GU}}}
% h
\newcommand\hg{{\mathrm{hg}}}
\newcommand\Hk{{\mathrm{Hk}}}
\newcommand\HN{{\mathrm{HN}}}
\newcommand\Hom{{\mathrm{Hom}}}
\newcommand\CHom{\mathcal{H}\mathrm{om}}
% i
\newcommand\id{{\mathrm{id}}}
\newcommand\Id{{\mathrm{Id}}}
\newcommand\idele{id\'{e}le}
\newcommand\Idele{Id\'{e}le}
\newcommand\ideles{id\'{e}les}
\renewcommand\Im{{\mathrm{Im}\,}}
\newcommand\im{{\mathrm{im}\,}}
\newcommand\Ind{{\mathrm{Ind}}}
\newcommand\cInd{{\mathrm{c}\textrm{-}\mathrm{Ind}}}
\newcommand\ind{{\mathrm{ind}}}
\renewcommand\inf{{\mathrm{inf}}}
\newcommand\Int{{\mathrm{Int}}}
\newcommand\inv{{\mathrm{inv}}}
\newcommand\Isom{{\mathrm{Isom}}}
% j
\newcommand\Jac{{\mathrm{Jac}}}
\newcommand\JL{{\mathrm{JL}}}
% k
\newcommand\Katz{\mathrm{Katz}}
\newcommand\Ker{{\mathrm{Ker}\,}}
\newcommand\KS{{\mathrm{KS}}}
\newcommand\Kl{{\mathrm{Kl}}}
\newcommand\CKl{{\mathcal{K}\mathrm{l}}}
% l
\newcommand\lcm{\mathrm{lcm}}
\newcommand\length{\mathrm{length}}
\newcommand\Li{{\mathrm{Li}}}
\newcommand\Lie{{\mathrm{Lie}}}
\newcommand\lt{\mathrm{lt}}
\newcommand\LT{\mathcal{LT}}
% m
\newcommand\MW{{\mathrm{MW}}}
\renewcommand\mod{\, \mathrm{mod}\, }
\newcommand\mom{{\mathrm{mom}}}
\newcommand\Mor{{\mathrm{Mor}}}
\newcommand\Morp{{\mathrm{Morp}\,}}
% n
\newcommand\new{{\mathrm{new}}}
\newcommand\Newt{{\mathrm{Newt}}}
\newcommand\NP{{\mathrm{NP}}}
\newcommand\NS{{\mathrm{NS}}}
\newcommand\ns{{\mathrm{ns}}}
\newcommand\Nm{{\mathrm{Nm}}}
\newcommand\Nrd{{\mathrm{Nrd}}}
\newcommand\Neron{N\'{e}ron}
% o
\newcommand\Obj{{\mathrm{Obj}\,}}
\newcommand\odd{{\mathrm{odd}}}
\newcommand\old{{\mathrm{old}}}
\newcommand\op{{\mathrm{op}}}
\newcommand\Orb{{\mathrm{Orb}}}
\newcommand\ord{{\mathrm{ord}}}
% p
\newcommand\pd{{\mathrm{pd}}}
\newcommand\Pet{{\mathrm{Pet}}}
\newcommand\PGL{{\mathrm{PGL}}}
\newcommand\Pic{{\mathrm{Pic}}}
\newcommand\pr{{\mathrm{pr}}}
\newcommand\Proj{{\mathrm{Proj}}}
\newcommand\proet{\text{pro\'{e}t}}
\newcommand\Poincare{\text{Poincar\'{e}}}
\newcommand\Prd{{\mathrm{Prd}}}
\newcommand\prim{{\mathrm{prim}}}
% r
\newcommand\Rad{{\mathrm{Rad}}}
\newcommand\rank{{\mathrm{rank}}}
\renewcommand\Re{{\mathrm{Re}}}
\newcommand\rec{{\mathrm{rec}}}
\newcommand\reg{{\mathrm{reg}}}
\newcommand\res{{\mathrm{res}}}
\newcommand\Res{{\mathrm{Res}}}
\newcommand\rig{{\mathrm{rig}}}
\newcommand\Rig{{\mathrm{Rig}}}
\newcommand\rk{{\mathrm{rk}}}
\newcommand\Ros{{\mathrm{Ros}}}
\newcommand\rs{{\mathrm{rs}}}
% s
\newcommand\sd{{\mathrm{sd}}}
\newcommand\Sel{{\mathrm{Sel}}}
\newcommand\sep{{\mathrm{sep}}}
\newcommand\sgn{{\mathrm{sgn}}}
\newcommand\Sh{{\mathrm{Sh}}}
\newcommand\Sht{\mathrm{Sht}}
\newcommand\Sim{{\mathrm{Sim}}}
\newcommand\sign{{\mathrm{sign}}}
\newcommand\SL{{\mathrm{SL}}}
\newcommand\SO{{\mathrm{SO}}}
\newcommand\Sp{{\mathrm{Sp}}}
\newcommand\Spa{{\mathrm{Spa}}}
\newcommand\Spec{{\mathrm{Spec}\,}}
\newcommand\Spf{{\mathrm{Spf}}}
\newcommand\Spin{{\mathrm{Spin}}}
\newcommand\Spm{{\mathrm{Spm}}}
\newcommand\srs{{\mathrm{srs}}}
\newcommand\rss{{\mathrm{ss}}}
\newcommand\ST{{\mathrm{ST}}}
\newcommand\st{{\mathrm{st}}}
\newcommand\Stab{{\mathrm{Stab}}}
\newcommand\SU{{\mathrm{SU}}}
\newcommand\Sym{{\mathrm{Sym}}}
\newcommand\sub{{\mathrm{sub}}}
\newcommand\rsum{{\mathrm{sum}}}
\newcommand\supp{{\mathrm{supp}}}
\newcommand\Supp{{\mathrm{Supp}}}
\newcommand\Swan{{\mathrm{Sw}}}
\newcommand\suml{\sum\limits}
% t
\newcommand\td{{\mathrm{td}}}
\newcommand\tor{{\mathrm{tor}}}
\newcommand\Tor{{\mathrm{Tor}}}
\newcommand\tors{{\mathrm{tors}}}
\newcommand\tr{{\mathrm{tr}\,}}
\newcommand\Tr{{\mathrm{Tr}}}
\newcommand\Trd{{\mathrm{Trd}}}
\newcommand\TSym{{\mathrm{TSym}}}
\newcommand\tw{{\mathrm{tw}}}
% u
\newcommand\uni{{\mathrm{uni}}}
\newcommand\univ{\mathrm{univ}}
\newcommand\ur{{\mathrm{ur}}}
\newcommand\USp{{\mathrm{USp}}}
% v
\newcommand\vQ{{\breve \BQ}}
\newcommand\vE{{\breve E}}
\newcommand\Ver{{\mathrm{Ver}}}
\newcommand\vF{{\breve F}}
\newcommand\vK{{\breve K}}
\newcommand\vol{{\mathrm{vol}}}
\newcommand\Vol{{\mathrm{Vol}}}
% w
\newcommand\wa{{\mathrm{wa}}}
% z
\newcommand\Zar{{\mathrm{Zar}}}


\newcommand\bchi{{\boldsymbol\chi}}
\newcommand\brho{{\boldsymbol\rho}}
\renewcommand\emph[1]{{\textcolor{blue}{#1}}}

\begin{document}
\title{Kloosterman 和的生成域}
\author{张神星}
\institute{第八届全国数论会议\quad 江苏金坛}
\date{2021年6月30日}

\frame{
\titlepage
}

\section{指数和}

%\frame{\frametitle{目录}\tableofcontents}

\frame{\frametitle{指数和}
设 $p$ 是一个素数, $\BF_q$ 是含有 $q=p^d$ 个元素的有限域.
对于 $\BF_q$ 上的一元多项式 $f(x)$, 定义\emph{指数和}
	\[S_1(f):=\sum_{x\in\BF_q}\zeta_p^{\Tr(f(x))}\in\BZ[\zeta_p],\]
其中 $\Tr=\Tr_{\BF_q/\BF_p}$, $\zeta_p\in \mu_p$ 是一个选定的 $p$ 次本原单位根.

我们要问:
\begin{enumerate}
\item 作为一个复数, $|S_1(f)|=?$
\item 作为一个 $p$ 进数, $|S_1(f)|_p=?$
\item 作为一个代数(整)数, $\deg S_1(f)=?$
\end{enumerate}
}

\frame{
\frametitle{$L$ 函数}
前两个问题已经有相当多的文献中研究过, 包括本次会议也有关于指数和的估计的报告. 我们简要回顾下指数和的基本性质.

定义 $f$ 的 \emph{$L$ 函数}为
	\[L(t,f):=\prod_{x\in \ov{\BF}_p}\Big(1-\Tr_{\BF_q(x)/\BF_p}(f(x))t^{\deg x}\Big)^{-1}=\exp\Bigl(\sum_k S_k(f)\frac{t^k}{k}\Bigr)\]
其中 $S_k(f):=\sum_{x\in\BF_{q^k}}\zeta_p^{\Tr(f(x))}\in\BZ[\zeta_p]$.
 
\begin{theorem}[Dwork-Bombieri-Grothendick]
$L(t,f)$ 是有理函数.
\end{theorem} 
}

\frame{
\frametitle{$\ell$ 进层}
记
	\[L(t,f)=\frac{\prod_j (1-\beta_j t)}{\prod_i(1-\alpha_i t)},\quad \alpha_i\neq \beta_j,\]
则
	\[S_k(f)=\sum_i \alpha_i^k-\sum_j\beta_j^k.\]
我们称 $\alpha_i,\beta_j$ 为\emph{特征根}. 为了估计特征根, 我们需要 $\ell$ 进方法.

一般地, 设 $X$ 是一个概形, $X_\et$ 是(小) \etale\ site.
固定素数 $\ell\neq p$, 设 $E$ 是 $\BQ_\ell$ 的有限扩张.
$X_\et$ 上系数为 $E$ 的 \emph{$\ell$ 进层}是这个 site 上的层 (实际上是 $\CO_E$ 有限商上的模层的逆向系, 态射扩充至 $E$), 使得在每个有限商上是可构造的. 若在每个有限商上是局部常值的, 则称之为 \emph{lisse} 的.
}

\frame{
\frametitle{Swan 导子}
为了刻画 lisse 层的一些性质, 我们需要 Swan 导子的概念.
设 $K$ 是完备离散赋值域, $I^{(x)},x\ge 0$ 为其高阶分歧群. 
对于分歧群 $P$ 的 $E$ 表示 $M$, 我们有满足如下性质的分解 $M=\oplus M(x)$, 
	\[M(0)=M^P,\quad M(x)^{I^{(x)}}=0,\quad M(x)^{I(y)}=M(x), \ y>x>0.\]
称 $M(x)\neq 0$ 的 $x$ 为 $M$ 的断点.
定义 $M$ 的 \emph{Swan 导子}为
	\[\Swan(M)=\sum x\dim M(x).\]
它总是一个整数.
}

\frame{
\frametitle{曲线}
令 $C$ 是特征 $p$ 完全域 $\BF$ 上一射影光滑几何连通代数曲线, $K=\BF(C)$ 为其函数域.
对于任意闭点 $x\in C(\BF)$, 我们有完备化 $K_x$.

对于非空开集 $U\subset C$, 我们有阿贝尔范畴等价
	\[\fct{\set{\text{$U$ 上的 lisse $E$ 层}}}{\cRep_E^c \pi_1(U,\ov\eta)}{\CF}{\CF_{\ov\eta}.}\]
由于基本群 $\pi_1(U,\ov\eta)$ 是伽罗瓦群 $\Gal(\ov K/K)$ 的商, 因此分解群 $D_x$ 作用在 $\CF_{\ov\eta}$ 上.
于是我们可以定义 $\CF$ 在 $x$ 处的 Swan 导子.
对于 $x\in U$, 由于惯性群 $I_x$ 作用平凡, 因此 $\Swan_x(\CF)=0$.

我们将会取 $C=\BP^1$ and $U=\Gm$.
}

\frame{
\frametitle{$\ell$ 进方法}
假设 $\mu_p\subseteq E$.
Deligne 在 ${\Ga}_{,\ov\BF_p}$ 上构造了一个局部自由秩 $1$ 的 $\ell$ 进层 $\CF_\ell(f)$, 它满足
	\[L(t,f)=\prod_i \det(1-t \Frob,\RH^i_c)^{(-1)^{i+1}}\]
及由此
	\[S_k(f)=\sum_i (-1)^i\Tr(\Frob^k,\RH_c^i).\]
这里 $\Frob$ 是几何 Frobenius, $\RH_c^i=\RH_c^i({\Ga}_{,\ov\BF_p},\CF_\ell(f))$ 是紧支撑上同调.
}

\frame{
\frametitle{$\ell$ 进方法之续}
记 $\omega_{ij}$ 为 $\Frob$ 在 $\RH^i_c$ 上的特征值, 则
	\[S_k(f)=\sum_{ij}(-1)^i \omega_{ij}^k.\]
记 $B_i=\dim_{E}\RH^i_c$ 为 Betti 数.
\begin{theorem}[Deligne]
$\omega_{ij}$ 是代数整数, 且存在整数 $0\le r_{ij}\le i$ 使得它的所有 $\BQ$ 共轭的绝对值均为 $q^{r_{ij}/2}$.
\end{theorem}
由此
	\[|S_k|\le \sum_i B_i q^{ki/2}.\]
}

\frame{
\frametitle{一般情形}
更一般地, 设
\begin{enumerate}
\item $V$ 是 $\BF_q$ 上 $\BA^N$ 的闭子簇,
\item $\psi$ 是 $\BF_q$ 上非平凡加性特征, $\psi_k:=\psi\circ \Tr_{\BF_{q^k}/\BF_q}$,
\item $f$ 是 $V$ 上定义在 $\BF_q$ 上的正则函数,
\item $\chi$ 是 $\BF_q^\times$ 上乘性特征, $\chi_k:=\chi\circ\bfN_{\BF_{q^n}/\BF_q}$,
\item $g$ 是 $V$ 上定义在 $\BF_q$ 上的可逆正则函数.
\end{enumerate}
定义指数和
	\[S_k=\sum_{x\in V(\BF_{q^k})}\psi_k(f(x))\chi_k(g(x)),\]
则前面的方法仍然是有效的. 这时候, Bombieri 证明了特征根个数不超过
	\[(4\max\set{\deg V+1,\deg f}+5)^{2N+1}.\]
} 

\section{Kloosterman 层}

\frame{
\frametitle{Kloosterman 和}
取
	\[V=V(X_1\cdots X_n-a),\quad f=X_1+\cdots+X_n.\]
设 $\bchi=\set{\chi_1,\dots,\chi_n}$ 是无序的 $n$ 个乘性特征 $\chi_i:\BF_q^\times\to \mu_{q-1}$.
定义 \emph{Kloosterman 和}
	\[\Kl_n(\psi,\bchi,q,a)=\sum_{x_1\cdots x_n=a\atop x_i\in\BF_q}\chi_1(x_1)\cdots\chi_n(x_n)\psi\bigl(\Tr_{\BF_q/\BF_p}(x_1+\cdots+x_n)\bigr).\] 
此时特征根个数为 $n$ 个. 因此 $|\Kl_n|\le nq^{(n-1)/2}$.
}

\frame{
\frametitle{伽罗瓦作用}
但我们并不是想要对其大小进行估计, 而是想要知道第三个问题的答案, 即它生成的数域是哪个. 显然 $\Kl_n\in\BZ[\mu_{pc}]$, 其中
$c=\lcm_i \set{\ord(\chi_i)}$
整除 $q-1$. 
我们将伽罗瓦群表示为
	\[\Gal(\BQ(\mu_{pc})/\BQ)=\set{\sigma_t\tau_w\mid t\in (\BZ/p\BZ)^\times,w\in(\BZ/c\BZ)^\times},\]
其中
	\[\sigma_t(\zeta_p)=\zeta_p^t,\quad \sigma_t(\zeta_c)=\zeta_c,\]
	\[\tau_w(\zeta_p)=\zeta_p,\quad \tau_w(\zeta_c)=\zeta_c^w.\] 

容易看出
	\[\sigma_t\tau_w\Kl_n(\psi,\bchi,q,a)=\prod\bchi(t)^{-w}\Kl_n(\psi,\bchi^w,q,at^n).\]
因此我们需要研究两个 Kloosterman 和何时相差一个 $(q-1)$ 次单位根.
}

\frame{\frametitle{平凡特征}
若 $\bchi={\bf1}=\set{1,\dots,1}$ 均为平凡特征, 则易知
	\[a,b\ \text{共轭} \implies \Kl_n(\psi,{\bf1},q,a)=\Kl_n(\psi,{\bf1},q,b).\] 
当 $p>(2n^{2d}+1)^2$ (Fisher), 或 $p\ge(d-1)n+2$ 且	$p$ 不整除一个特定整数(万大庆)时, 反过来也是成立的. 一般猜测 $p\ge nd$ 就足够了.
在这些情形下,
	\[\deg \Kl_n(\psi,{\bf1},q,a)=\frac{p-1}{(p-1,n)}.\]
}

\frame{\frametitle{Kloosterman 层}
Deligne 和 Katz 在 $\Gm\otimes \BF_q$ 上定义了一个 lisse 层
	\[\CKl=\CKl_{n,q}(\psi,\bchi),\]
它满足如下性质
\begin{enumerate}
\item $\CKl$ 秩为 $n$, 权为纯 $n-1$.
\item 对任意 $a\in \BF_q^\times$,
	$\Tr\bigl(\Frob_a,\CKl_{\ov a})=(-1)^{n-1}\Kl_n(\psi,\bchi,q,a).$
\item $\CKl$ 在 $0$ 处温和 ($\Swan_0=0$).
\item $\CKl$ 在 $\infty$ 处完全野, $Swan_\infty=1$. 于是它的 $\infty$ 断点均为 $1/n$.
\end{enumerate}
}

\frame{
\frametitle{Fisher 的下降}
Fisher 给了 Kloosterman 层沿着有限域的扩张的下降.
对于 $a\in\BF_q^\times$, 他定义了 $\Gm\otimes\BF_p$ 上的一个 lisse 层 $\CF_a(\bchi)$, 使得
$\CF_a(\bchi)|\Gm\otimes\BF_q=\bigotimes\limits_{\sigma\in\Gal(\BF_q/\BF_p)} \bigl(t\mapsto \sigma(a)t^n\bigr)^*\CKl_n(\psi\circ\sigma^{-1},\bchi\circ\sigma^{-1})$
并满足如下性质:
\begin{enumerate}
\item $\CF_a(\bchi)$ 秩为 $n^d$, 权为纯 $d(n-1)$.
\item 对任意 $t\in\BF_p^\times$,
	$\Tr\bigl(\Frob_t,\CF_a(\bchi)_{\ov t}\bigr)=(-1)^{(n-1)d}\Kl_n(\psi,\bchi,q,at^n).$
\item $\CF_a(\bchi)$ 在 $0$ 处温和.
\item $\CF_a(\bchi)$ 的 $\infty$ 断点均不超过 $1$.
\end{enumerate}
}

\frame{\frametitle{关键的估计}
\begin{lemma}
设 $\CF,\CF'$ 是 $\Gm\otimes\BF_p$ 上秩均为 $r$, 权均为纯 $w$ 的 lisse 层.
假设存在单位根 $\lambda$ 使得对任意 $t\in \BF_p^\times$ 有
	\[\Tr\bigl(\Frob_t, \CF_{\ov t})=\lambda\Tr\bigl(\Frob_t, \CF'_{\ov t}).\]
设 $\CG$ 是 $\Gm\otimes\BF_p$ 上一几何不可约秩为 $s$, 权为纯 $w$ 的层, 使得 $\CG\mid \Gm\otimes\ov\BF_p$ 在 $\CF\mid \Gm\otimes\ov \BF_p$ 中恰好出现一次, 则它在 $\CF'\mid \Gm\otimes\ov \BF_p$ 中至少出现一次, 其中我们要求 $p>[2rs(M_0+M_\infty)+1]^2$, $M_\eta$ 是 $\CF\oplus \CF'$ 的最大 $\eta$ 断点.
\end{lemma}
}

\frame{\frametitle{关键的估计之证明概述}
反证法. 通过平移我们不妨设 $w=0$. 我们将 Lefschetz 迹公式应用到 $\CG^\vee\otimes\CF$ 和 $\CG^\vee\otimes \CF'$ 上,
	\[\sum_{i=0}^2(-1)^i\Tr\bigl(\Frob,\RH_c^i(\CG^\vee\otimes\CF)\bigr)
		=\lambda\sum_{i=0}^2(-1)^i\Tr\bigl(\Frob, \RH_c^i(\CG^\vee\otimes\CF')\bigr).\] 
我们有
	\[\RH^0_c=0=\RH^2_c(\CG^\vee\otimes\CF')\]
	$\RH^2_c(\CG^\vee\otimes\CF)$ 为 $1$ 维, 权为纯 $2$, $\RH^1_c$ 的权不超过 $1$ (Weil II).
结合 Euler-Poincar\'e 公式
	\[h^0_c(\CF)-h^1_c(\CF)+h^2_c(\CF)	=-\Swan_0(\CF)-\Swan_\infty(\CF)\]
 我们可以得到 $p$ 的估计.
}

\frame{\frametitle{Kummer 诱导的}
称 $\bchi$ 是 \emph{Kummer 诱导的}, 若存在非平凡特征 $\Lambda$ 使得作为无序数组 $\bchi=\bchi\Lambda:=\set{\chi_1\Lambda,\dots,\chi_n\Lambda}$.
此时, $\prod\bchi=\prod(\bchi\Lambda)=\Lambda^n\prod\bchi$. 因此 $\Lambda^n=1$. 

假设 $p>2n+1$ 且 $\bchi$ 不是 Kummer 诱导的, 则 $\CF_a(\bchi)$ 有一个重数为 $1$ 的最高权. 考虑对应的李代数  $\fg(\CF_a(\bchi))$ 表示, 它对应一个子层 $\CG_a(\bchi)$. 
而且这个子层是几何不可约的, 且在 $\CF_a(\bchi)|\Gm\otimes\ov\BF_p$ 中只出现一次.
}

\frame{\frametitle{推论}
\begin{corollary}
设 $a,b\in\BF_q^\times$, $\bchi$ 和 $\brho$ 为乘性特征 $\chi_i,\rho_j:\BF_q^\times\to\ov\BQ_\ell^\times$ 构成的 $n$ 元无序数组.
假设 $p>(2n^{2d}+1)^2$, $\bchi$ 不是 Kummer 诱导的, 且存在 $\lambda\in\mu_{q-1}$ 使得
	\[\Kl_n(\psi,\bchi,q,a)=\lambda\Kl_n(\psi,\brho,q,b).\]
则 $\CG_a(\bchi)\otimes \CL_{\prod \ov\bchi}\mid \Gm\otimes\ov \BF_p$ 在 $\CF_b(\brho)\otimes\CL_{\prod\ov \brho}\mid \Gm\otimes\ov \BF_p$ 中出现.
\end{corollary}

这里 $\CL_\chi$ 是 $\Gm\otimes\BF_p$ 上由 Lang torsor 定义的秩 $1$ lisse 层, 使得对任意 $t\in\BF_p^\times$,
	\[\Tr(\Frob_t,(\CL_\chi)_{\ov t})=\chi(t).\]
}

\frame{\frametitle{推论的证明}
令
	\[\CF=\CF_a(\bchi)\otimes\CL_{\prod\ov \bchi},\	\CF'=\CF_b(\brho)\otimes\CL_{\prod\ov\brho},\ \CG=\CG_a(\bchi)\otimes\CL_{\prod\ov\bchi}.\]
由于对任意 $t\in\BF_p^\times$, $\sigma_t\lambda=\lambda$, 因此
\[\begin{split}
&(-1)^{(n-1)d}\Tr\bigl(\Frob_t, \CF_{\ov t})
=\prod\ov\bchi(t)\cdot\Kl_n(\psi,\bchi,q,at^n)\\
=&\sigma_t\bigl(\Kl_n(\psi,\bchi,q,a)\bigr)
=\lambda\sigma_t\bigl(\Kl_n(\psi,\brho,q,b)\bigr)\\
=&\lambda\prod\ov\brho(t)\cdot\Kl_n(\psi,\brho,q,bt^n)
=(-1)^{(n-1)d}\lambda\Tr\bigl(\Frob_t, \CF'_{\ov t}).
\end{split}\]
应用前述引理即可, 其中 $r=s=n^d,M_0=0,M_\infty\le 1$.
}

\frame{\frametitle{Kloosterman 和的不同}
现在
	\[\CG_a(\bchi)\otimes \CL_{\prod \ov\bchi}\inj \CF_b(\brho)\otimes\CL_{\prod\ov \brho},\quad \CG_b(\brho)\otimes \CL_{\prod \ov\brho}\inj \CF_a(\bchi)\otimes\CL_{\prod\ov \bchi}.\]
我们有最高权 $\lambda_a(\bchi)=\lambda_b(\brho)$.
由此, 通过 Fisher 的论述可得:
\begin{theorem}
设 $a,b\in\BF_q^\times$.
假设 $\bchi,\brho$ 不是 Kummer 诱导的且它们都不是 $(\xi_1,\xi_1^{-1},1,\Lambda_2)\xi_2$ 型.
若 $p>(2n^{2d}+1)^2$ 以及存在 $\lambda\in\mu_{q-1}$ 使得
	\[\Kl_n(\psi,\bchi,q,a)=\lambda\Kl_n(\psi,\brho,q,b),\]
则存在 $\sigma\in\Gal(\BF_q/\BF_p)$ 和乘性特征 $\eta$, 使得 $\brho=\eta\cdot(\bchi\circ\sigma^{-1})$ 以及 $b=\sigma(a)$. 更进一步, 要么两个 Kloosterman 和均为零, 要么 $\eta(b)=\lambda^{-1}$.
\end{theorem}
}

\section{生成域}
\frame{\frametitle{非零性}
最后一步我们需要证明 Kloosterman 和非零.
\begin{theorem}
若 $p>(3n-1)C_\bchi-n$, 且对任意 $i,j$, $\chi_i=\chi_j$ 或它们不相差一个 $n$ 次(未必本原)特征, 则 $\Kl_n(\psi,\bchi,q,a)$ 非零.
其中
\begin{equation}
	C_\bchi=\max_{i,j}\lcm\bigl(\ord(\chi_i),\ord(\chi_j)\bigr)
\end{equation}
是任两个特征的阶的最小公倍数的最大值.
\end{theorem}
}

\frame{\frametitle{非零性的证明概述}
通过 $\BF_q^\times$ 上的傅里叶变换, 我们可以将 $\Kl_n$ 表达为高斯和的组合
	\[(q-1)\Kl_n(\psi,\bchi,q,a)=
	\sum_{m=0}^{q-2}\omega^m(a) \prod_{i=1}^n g(m+s_i),\]
其中我们固定一个 Teichm\"uller 特征 并记 $\chi_i=\omega^{s_i}$.
通过小心的估计, 我们可以证明存在唯一的 $m$ 使得 $\prod_{i=1}^ng(m+s_i)$ 最小, 由此可知非零性.

对于平凡特征情形, $m=0$.
}

\frame{\frametitle{生成域}
\begin{theorem}
若 $p>\max\set{(2n^{2d}+1)^2,(3n-1)C_\bchi-n}$ 且对任意 $i,j$, $\chi_i=\chi_j$ 或它们不相差一个 $n$ 次特征, 则 $\Kl_n(\psi,\bchi,q,a)$ 生成 $\BQ(\mu_{pc})^H$, 其中 $H$ 包含的 $\sigma_t\tau_w$ 满足: 存在整数 $\beta$ 和特征 $\eta$ 使得
  \[t=\lambda a_1^\beta,\lambda^{n_1}=1,\ \bchi^w=\eta\bchi^{q_1^\beta},\ \eta(a)=\prod\bchi^w(t).\]
这里, $n_1=(n,p-1)$, $q_1=\#\BF_p(a^{(p-1)/n_1})$, $a_1\in\BF_p^\times$ 满足 $a_1^{n/n_1}=\bfN_{\BF_{q_1}/\BF_p}(a^{(1-p)/n_1})=a^{(1-q_1)/n_1}$.
\end{theorem}
}


\frame{\frametitle{例子: $n=2$ 情形}
设 $\bchi=\set{1,\chi}$, 其中乘性特征 $\chi$ 的阶为 $c\neq 2$.
若 $p>\max\set{(2^{2d+1}+1)^2,5c-2)}$, 则 $\Kl(\psi,\bchi,p^d,a)$ 生成 $\BQ(\mu_{pc})^H$, 其中 
	\[H=
	\begin{cases}
		\pair{\tau_{q_1}\sigma_{a_1},\sigma_{-1},\tau_{-1}},
			&\text{若}\ \chi(-1)=1,\chi(a)=1;\\
		\pair{\tau_{-q_1}\sigma_{a_1},\sigma_{-1}},
			&\text{若}\ \chi(-1)=1,\chi(a)=\chi(a_1)=-1;\\
		\pair{\tau_{q_1^{\alpha}}\sigma_{a_1^\alpha},\sigma_{-1}},
			&\text{若}\ \chi(-1)=1,\chi(a)^\alpha\neq 1;\\
		\pair{\tau_{q_1}\sigma_{-a_1},\tau_{-1}\sigma_{-1}},
			&\text{若}\ \chi(-1)=-1, \chi(a)=\chi(a_1)=-1;\\
		\pair{\tau_{q_1}\sigma_{a_1},\tau_{-1}}
			&\text{若}\ \chi(-1)=-1, \chi(a)=1;\\
		\pair{\tau_{q_1}\sigma_{a_1},\tau_{-1}\sigma_{-1}},
			&\text{若}\ \chi(-1)=-1,\chi(a)=-1,\chi(a_1)=1;\\
		\pair{\tau_{q_1^{\alpha/2}}\sigma_{-a_1^{\alpha/2}}},
			&\text{若}\ \chi(-1)=-1,2\mid \alpha,\chi(a)\neq \pm1;\\
		\pair{\tau_{q_1^{\alpha}}\sigma_{a_1^\alpha}},
			&\text{若}\ \chi(-1)=-1,2\nmid \alpha,\chi(a)\neq \pm1.
	\end{cases}\]
 $q_1=\#\BF_p(a^{(1-p)/2}), a_1=a^{(1-q_1)/2}$, $\alpha$ 是 $\chi(a_1)\in\mu_{p-1}$ 的阶.
}

\frame{\frametitle{注记}
考虑 Kloosterman 和
	\[S_k=\Kl(\psi,\bchi\circ\bfN_{\BF_{q^k}/\BF_q},q^k,a).\]
若 $p>\max\set{(2n^{2dk}+1)^2,(3n-1)C_\bchi-n},$ 则 $\BQ(S_k)=\BQ(\mu_{pc})^H$, 其中 $H$ 包含的 $\sigma_t\tau_w$ 满足: 存在整数 $\beta$ 和 $\BF_q^\times$ 上的特征 $\eta$ 满足
	\[	t=\lambda a_1^\beta,\lambda^{n_1}=1,\quad \bchi^w=\eta\bchi^{q_1^\beta},\quad \eta(a)=\gamma\cdot\prod\bchi^w(t),\gamma^k=1.\]
于是 $\BQ(S_k)=\BQ(S_{k-c})$, 因为 $\gamma^c=1$.
}

\frame{\frametitle{注记之续}
由于 $L$ 函数
	\[L(T)=\exp\left(\sum_{k=1}^\infty \frac{T^k}{k} S_k\right)\]
是一个有理函数, 序列 $\set{S_k}_k$ 是一个线性递推序列.
万大庆和尹航证明了存在 $N$ 使得序列 $\set{\BQ(S_k)}_{k\ge N}$ 是周期的. 设 $r$ 为其周期, 则 $p>\max\set{\bigl(2n^{2d(N+r)}+1\bigr)^2,(3n-1)C_\bchi-n}$ 时, $\BQ(S_k)$ 由前文所描述.
因此我们需要将下界 $(2n^{2d}+1)^2$ 缩小, 并需要对 $r$ 和 $N$ 进行尽可能小的估计.
我们(大胆)猜测下 $p>3ndc$ 时成立.
}

\frame{
\begin{center}
\huge 感谢各位的倾听!
\end{center}
}









\end{document}
