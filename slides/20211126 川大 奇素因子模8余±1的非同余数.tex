% -*- coding: utf-8 -*-
\documentclass{beamer}
\usetheme{madrid}
\usepackage{beamerouterthemeprogressbarjma} 
% 恢复默认字体
%\renewcommand\mathfamilydefault{\rmdefault}


 %\setbeamertemplate{background canvas}[vertical shading][bottom=red!10,top=blue!10]
  %\setbeamertemplate{blocks}[rounded][shadow=true]
  %\setbeamertemplate{footline}[frame number]
  %\setbeamercovered{transparent}
  %\usefonttheme[onlysmall]{structurebold}
  % \usefonttheme[onlymath]{serif}

\definecolor{bistre}{rgb}{0.24, 0.17, 0.12} % 黑色
\definecolor{mygrey}{rgb}{0.52, 0.52, 0.51} % 灰色
\definecolor{lightblue}{rgb}{0.15, 0.15, 0.55}
\colorlet{main}{green!50!black}
\colorlet{text}{bistre!100!white}

% 不同元素指定不同颜色,fg是本身颜色,bg是背景颜色,!num!改变数值提供渐变色
%\setbeamercolor{title}{fg=main}
\setbeamercolor{frametitle}{fg=lightblue,bg=white}
\setbeamertemplate{frametitle}[default][center]
%\setbeamercolor{section in toc}{fg=text}
%\setbeamercolor{normal text}{fg=text}
%\setbeamercolor{block title}{fg=main,bg=mygrey!14!white}
%\setbeamercolor{block body}{fg=black,bg=mygrey!10!white}
%\setbeamercolor{qed symbol}{fg=main} % 证明结束后的框颜色
%\setbeamercolor{math text}{fg=black}

%\usepackage{CJKutf8}
\usepackage{ctex}
\usepackage{latexsym}
\usepackage{amssymb}
\usepackage{color}
\usepackage{amsmath}
\usepackage{graphicx}
\usepackage{cases}
\usepackage{wasysym}
\usepackage{amsthm}
\usepackage[all]{xy}
\usepackage{amsfonts}
\usepackage{fancyhdr}
\usepackage{stmaryrd}
\setlength{\parindent}{2em}
\setlength{\parskip}{1em}

\renewcommand{\labelenumi}{{\upshape(\arabic{enumi})}}
\newcommand\enumnum[1]{{\textcolor{fourth}{\mdseries\upshape{(#1)}}}}
\newcommand\peq{\mathrel{\phantom{=}}} % 用于对齐的等号幻影

\usepackage{tikz}
\usepackage{color,calc}
% TIKZ 设置
\usetikzlibrary{
	quotes,
	shapes.arrows,
	arrows.meta,
	positioning,
	shapes.geometric,
	patterns,
	calc,
	angles,
	decorations.pathreplacing,
	backgrounds % 背景边框
}
\tikzset{
	background rectangle/.style={semithick,draw=fourth,fill=white,rounded corners},
  % arrow
	cstra/.style      ={-Stealth},        % right arrow
	cstla/.style      ={Stealth-},        % left arrow
	cstlra/.style     ={Stealth-Stealth}, % left-right arrow
	cstwra/.style     ={-Straight Barb},  % wide ra
	cstwla/.style     ={Straight Barb-},
	cstwlra/.style    ={Straight Barb-Straight Barb},
	cstnarrow/.style      ={-Latex, line width=0.1cm}, %文本框间箭头
	cstaxis/.style        ={-Stealth, thick}, %坐标轴
  % curve
	cstcurve/.style       ={very thick}, %一般曲线
	cstdash/.style        ={thick, dash pattern= on 0.2cm off 0.05cm}, %虚线
  % dot
	cstdot/.style         ={radius=.08}, %实心点
	cstdote/.style        ={radius=.07, fill=white}, %空心点
  % fill
	cstfill/.style       ={fill=black!10},
	cstfille/.style      ={pattern=north east lines, pattern color=black},
	cstfill1/.style       ={fill=main!20},
	cstfille1/.style      ={pattern=north east lines, pattern color=main},
	cstfill2/.style        ={fill=second!20},
	cstfille2/.style       ={pattern=north east lines, pattern color=second},
	cstfill3/.style        ={fill=third!20},
	cstfille3/.style       ={pattern=north east lines, pattern color=third},
	cstfill4/.style        ={fill=fourth!20},
	cstfille4/.style       ={pattern=north east lines, pattern color=fourth},
	cstfill5/.style        ={fill=fifth!20},
	cstfille5/.style       ={pattern=north east lines, pattern color=fifth},
  % node
	cstnode/.style        ={fill=white,draw=black,text=black,rounded corners=0.2cm,line width=1pt},
	cstnode1/.style       ={fill=main!15,draw=main!80,text=black,rounded corners=0.2cm,line width=1pt},
	cstnode2/.style       ={fill=second!15,draw=second!80,text=black,rounded corners=0.2cm,line width=1pt},
	cstnode3/.style       ={fill=third!15,draw=third!80,text=black,rounded corners=0.2cm,line width=1pt},
	cstnode4/.style       ={fill=fourth!15,draw=fourth!80,text=black,rounded corners=0.2cm,line width=1pt},
	cstnode5/.style       ={fill=fifth!15,draw=fifth!80,text=black,rounded corners=0.2cm,line width=1pt}
}
\ExplSyntaxOn
\cs_new_protected:Npn \fpstepfromto#1#2#3 
  {% from, to, nums
    \fp_step_inline:nnnn {#1} { (#2-(#1))/(#3-1)*0.99 } {#2}
  }
\pgfmathdeclarefunction{nrand}{0}
  {% \tex_normaldeviate:D 生成均值为 0,标准差为 10000 的随机整数
    \tl_set:Nx \pgfmathresult { \fp_eval:n { \tex_normaldeviate:D/10000 } }
  }
\pgfmathdeclarefunction{rdv}{0}{\pgfmathparse{1+nrand/100}}
\ExplSyntaxOff
\newcommand{\randpts}[3][10]{
  \foreach\i in {0,1,...,#1}{
    \pgfmathparse{rdv}\let\rdv\pgfmathresult
    \coordinate (\i) at ({#2*\rdv*cos(360/#1*\i)},{#3*\rdv*sin(360/#1*\i)});
  }
}
\newcommand{\randep}[2]{
  \randpts{#1}{#2}
  \filldraw[cstcurve,main,cstfill3,smooth] plot coordinates {(0) (1) (2) (3) (4) (5) (6) (7) (8) (9) (0)};
}

\let\question\relax
\elegantnewtheorem{question}{问题}{defstyle}{que}

\newfontface\cmunrm{cmunrm.otf}
\newcommand\cmu[1]{{\cmunrm{#1}}}
\newcommand{\alert}[1]{\textcolor{main}{\bf #1}}
\renewcommand{\emph}[1]{\textcolor{second}{\bf #1}}
\newcommand{\cnen}[2]{{\kaishu$\overset{\text{{#2}}}{\text{#1}}$}}
\newcommand{\nounen}[2]{{\color{second}\kaishu\cnen{#1}{#2}}\index{{#1}}}
\newcommand{\noun}[1]{{\color{second}\kaishu #1}\index{{#1}}}
\newcommand{\nouns}[2]{{\color{second}\kaishu #1}\index{{#2}}}
\newcommand{\nounsen}[3]{{\color{second}\kaishu\cnen{#1}{#2}}\index{{#3}}}

\setcounter{tocdepth}{2}
\newfontfamily\couriernew{Courier New}
\lstset{language=[LaTeX]TeX,
	basicstyle=\couriernew,
  morekeywords={AUTHOR, KEY, TITLE, YEAR, PAGES, HOWPUBLISHED, URL, LANGUAGE},
  keywordstyle=\color{winered}
}




\RequirePackage{extarrows}
\usepackage[normalem]{ulem}
\NewDocumentCommand\fillblank{O{1cm} O{0cm} m}{\uline{\makebox[#1]{\raisebox{#2}{#3}}}}
\newcommand\fillbrace[1]{{(\nolinebreak\hspace{0.5em minus 0.5em}{#1}\hspace{0.5em minus 0.5em}\nolinebreak)}}
\newcommand\resizet[1]{\resizebox{!}{#1\baselineskip}}
\newcommand{\trueex}{$\checkmark$}
\newcommand{\falseex}{$\times$}

% arrows
\newcommand\ra{\rightarrow}
\newcommand\lra{\longrightarrow}
\newcommand\la{\leftarrow}
\newcommand\lla{\longleftarrow}
\newcommand\sqra{\rightsquigarrow}
\newcommand\sqlra{\leftrightsquigarrow}
\newcommand\inj{\hookrightarrow}
\newcommand\linj{\hookleftarrow}
\newcommand\surj{\twoheadrightarrow}
\newcommand\simto{\stackrel{\sim}{\longrightarrow}}
\newcommand\sto[1]{\stackrel{#1}{\longrightarrow}}
\newcommand\lsto[1]{\stackrel{#1}{\longleftarrow}}
\newcommand\xto{\xlongrightarrow}
\newcommand\xeq{\xlongequal}
\newcommand\luobida{\xeq{\text{洛必达}}}
\newcommand\djwqx{\xeq{\text{等价无穷小}}}
\newcommand\eqprob{\stackrel{\mathrm{P}}{=}}
\newcommand\lto{\longmapsto}
\renewcommand\vec[1]{\overrightarrow{#1}}

% decorations
\newcommand\wh{\widehat}
\newcommand\wt{\widetilde}
\newcommand\ov{\overline}
\newcommand\ul{\underline}
\newlength{\larc}
\NewDocumentCommand\warc{o m}{%
	\IfNoValueTF {#1}%
	{%
		\settowidth{\larc}{$#2$}%
		\stackrel{\rotatebox{-90}{\ensuremath{\left(\rule{0ex}{0.7\larc}\right.}}}{#2}%
	}%
	{%
		\stackrel{\rotatebox{-90}{\ensuremath{\left(\rule{0ex}{#1}\right.}}}{#2}%
	}%
}

% braces
\newcommand\set[1]{{\left\{#1\right\}}}
\newcommand\setm[2]{{\left\{#1\,\middle\vert\, #2\right\}}}
\newcommand\abs[1]{\left|#1\right|}
\newcommand\pair[1]{\langle{#1}\rangle}
\newcommand\norm[1]{\!\parallel\!{#1}\!\parallel\!}
\newcommand\dbb[1]{\llbracket#1 \rrbracket}
\newcommand\floor[1]{\lfloor#1\rfloor}

% symbols
\renewcommand\le{\leqslant}
\renewcommand\ge{\geqslant}
\newcommand\vare{\varepsilon}
\newcommand\varp{\varphi}
\newcommand\ilim{\varinjlim\limits}
\newcommand\plim{\varprojlim\limits}
\newcommand\half{\frac{1}{2}}
\newcommand\mmid{\parallel}
\font\cyr=wncyr10\newcommand\Sha{\hbox{\cyr X}}
\newcommand\Uc{\stackrel{\circ}{U}\!\!}
\newcommand\hil[3]{\left(\frac{{#1},{#2}}{#3}\right)}
\newcommand\leg[2]{\Bigl(\frac{{#1}}{#2}\Bigr)}
\newcommand\aleg[2]{\Bigl[\frac{{#1}}{#2}\Bigr]}
\newcommand\stsc[2]{\genfrac{}{}{0pt}{}{#1}{#2}}

% categories
\newcommand\cA{{\mathsf{A}}}
\newcommand\cb{{\mathsf{b}}}
\newcommand\cB{{\mathsf{B}}}
\newcommand\cC{{\mathsf{C}}}
\newcommand\cD{{\mathsf{D}}}
\newcommand\cM{{\mathsf{M}}}
\newcommand\cR{{\mathsf{R}}}
\newcommand\cP{{\mathsf{P}}}
\newcommand\cT{{\mathsf{T}}}
\newcommand\cX{{\mathsf{X}}}
\newcommand\cx{{\mathsf{x}}}
\newcommand\cAb{{\mathsf{Ab}}}
\newcommand\cBT{{\mathsf{BT}}}
\newcommand\cBun{{\mathsf{Bun}}}
\newcommand\cCharLoc{{\mathsf{CharLoc}}}
\newcommand\cCoh{{\mathsf{Coh}}}
\newcommand\cComm{{\mathsf{Comm}}}
\newcommand\cEt{{\mathsf{Et}}}
\newcommand\cFppf{{\mathsf{Fppf}}}
\newcommand\cFpqc{{\mathsf{Fpqc}}}
\newcommand\cFunc{{\mathsf{Func}}}
\newcommand\cGroups{{\mathsf{Groups}}}
\newcommand\cGrpd{{\{\mathsf{Grpd}\}}}
\newcommand\cHo{{\mathsf{Ho}}}
\newcommand\cIso{{\mathsf{Iso}}}
\newcommand\cLoc{{\mathsf{Loc}}}
\newcommand\cMod{{\mathsf{Mod}}}
\newcommand\cModFil{{\mathsf{ModFil}}}
\newcommand\cNilp{{\mathsf{Nilp}}}
\newcommand\cPerf{{\mathsf{Perf}}}
\newcommand\cPN{{\mathsf{PN}}}
\newcommand\cRep{{\mathsf{Rep}}}
\newcommand\cRings{{\mathsf{Rings}}}
\newcommand\cSets{{\mathsf{Sets}}}
\newcommand\cStack{{\mathsf{Stack}}}
\newcommand\cSch{{\mathsf{Sch}}}
\newcommand\cTop{{\mathsf{Top}}}
\newcommand\cVect{{\mathsf{Vect}}}
\newcommand\cZar{{\mathsf{Zar}}}
\newcommand\cphimod{{\varphi\txt{-}\mathsf{Mod}}}
\newcommand\cphimodfil{{\varphi\txt{-}\mathsf{ModFil}}}

% font
\newcommand\rma{{\mathrm{a}}}
\newcommand\rmb{{\mathrm{b}}}
\newcommand\rmc{{\mathrm{c}}}
\newcommand\rmd{{\mathrm{d}}}
\newcommand\rme{{\mathrm{e}}}
\newcommand\rmf{{\mathrm{f}}}
\newcommand\rmg{{\mathrm{g}}}
\newcommand\rmh{{\mathrm{h}}}
\newcommand\rmi{{\mathrm{i}}}
\newcommand\rmj{{\mathrm{j}}}
\newcommand\rmk{{\mathrm{k}}}
\newcommand\rml{{\mathrm{l}}}
\newcommand\rmm{{\mathrm{m}}}
\newcommand\rmn{{\mathrm{n}}}
\newcommand\rmo{{\mathrm{o}}}
\newcommand\rmp{{\mathrm{p}}}
\newcommand\rmq{{\mathrm{q}}}
\newcommand\rmr{{\mathrm{r}}}
\newcommand\rms{{\mathrm{s}}}
\newcommand\rmt{{\mathrm{t}}}
\newcommand\rmu{{\mathrm{u}}}
\newcommand\rmv{{\mathrm{v}}}
\newcommand\rmw{{\mathrm{w}}}
\newcommand\rmx{{\mathrm{x}}}
\newcommand\rmy{{\mathrm{y}}}
\newcommand\rmz{{\mathrm{z}}}
\newcommand\rmA{{\mathrm{A}}}
\newcommand\rmB{{\mathrm{B}}}
\newcommand\rmC{{\mathrm{C}}}
\newcommand\rmD{{\mathrm{D}}}
\newcommand\rmE{{\mathrm{E}}}
\newcommand\rmF{{\mathrm{F}}}
\newcommand\rmG{{\mathrm{G}}}
\newcommand\rmH{{\mathrm{H}}}
\newcommand\rmI{{\mathrm{I}}}
\newcommand\rmJ{{\mathrm{J}}}
\newcommand\rmK{{\mathrm{K}}}
\newcommand\rmL{{\mathrm{L}}}
\newcommand\rmM{{\mathrm{M}}}
\newcommand\rmN{{\mathrm{N}}}
\newcommand\rmO{{\mathrm{O}}}
\newcommand\rmP{{\mathrm{P}}}
\newcommand\rmQ{{\mathrm{Q}}}
\newcommand\rmR{{\mathrm{R}}}
\newcommand\rmS{{\mathrm{S}}}
\newcommand\rmT{{\mathrm{T}}}
\newcommand\rmU{{\mathrm{U}}}
\newcommand\rmV{{\mathrm{V}}}
\newcommand\rmW{{\mathrm{W}}}
\newcommand\rmX{{\mathrm{X}}}
\newcommand\rmY{{\mathrm{Y}}}
\newcommand\rmZ{{\mathrm{Z}}}
\newcommand\bfa{{\mathbf{a}}}
\newcommand\bfb{{\mathbf{b}}}
\newcommand\bfc{{\mathbf{c}}}
\newcommand\bfd{{\mathbf{d}}}
\newcommand\bfe{{\mathbf{e}}}
\newcommand\bff{{\mathbf{f}}}
\newcommand\bfg{{\mathbf{g}}}
\newcommand\bfh{{\mathbf{h}}}
\newcommand\bfi{{\mathbf{i}}}
\newcommand\bfj{{\mathbf{j}}}
\newcommand\bfk{{\mathbf{k}}}
\newcommand\bfl{{\mathbf{l}}}
\newcommand\bfm{{\mathbf{m}}}
\newcommand\bfn{{\mathbf{n}}}
\newcommand\bfo{{\mathbf{o}}}
\newcommand\bfp{{\mathbf{p}}}
\newcommand\bfq{{\mathbf{q}}}
\newcommand\bfr{{\mathbf{r}}}
\newcommand\bfs{{\mathbf{s}}}
\newcommand\bft{{\mathbf{t}}}
\newcommand\bfu{{\mathbf{u}}}
\newcommand\bfv{{\mathbf{v}}}
\newcommand\bfw{{\mathbf{w}}}
\newcommand\bfx{{\mathbf{x}}}
\newcommand\bfy{{\mathbf{y}}}
\newcommand\bfz{{\mathbf{z}}}
\newcommand\bfA{{\mathbf{A}}}
\newcommand\bfB{{\mathbf{B}}}
\newcommand\bfC{{\mathbf{C}}}
\newcommand\bfD{{\mathbf{D}}}
\newcommand\bfE{{\mathbf{E}}}
\newcommand\bfF{{\mathbf{F}}}
\newcommand\bfG{{\mathbf{G}}}
\newcommand\bfH{{\mathbf{H}}}
\newcommand\bfI{{\mathbf{I}}}
\newcommand\bfJ{{\mathbf{J}}}
\newcommand\bfK{{\mathbf{K}}}
\newcommand\bfL{{\mathbf{L}}}
\newcommand\bfM{{\mathbf{M}}}
\newcommand\bfN{{\mathbf{N}}}
\newcommand\bfO{{\mathbf{O}}}
\newcommand\bfP{{\mathbf{P}}}
\newcommand\bfQ{{\mathbf{Q}}}
\newcommand\bfR{{\mathbf{R}}}
\newcommand\bfS{{\mathbf{S}}}
\newcommand\bfT{{\mathbf{T}}}
\newcommand\bfU{{\mathbf{U}}}
\newcommand\bfV{{\mathbf{V}}}
\newcommand\bfW{{\mathbf{W}}}
\newcommand\bfX{{\mathbf{X}}}
\newcommand\bfY{{\mathbf{Y}}}
\newcommand\bfZ{{\mathbf{Z}}}
\newcommand\BA{{\mathbb{A}}}
\newcommand\BB{{\mathbb{B}}}
\newcommand\BC{{\mathbb{C}}}
\newcommand\BD{{\mathbb{D}}}
\newcommand\BE{{\mathbb{E}}}
\newcommand\BF{{\mathbb{F}}}
\newcommand\BG{{\mathbb{G}}}
\newcommand\BH{{\mathbb{H}}}
\newcommand\BI{{\mathbb{I}}}
\newcommand\BJ{{\mathbb{J}}}
\newcommand\BK{{\mathbb{K}}}
\newcommand\BL{{\mathbb{L}}}
\newcommand\BM{{\mathbb{M}}}
\newcommand\BN{{\mathbb{N}}}
\newcommand\BO{{\mathbb{O}}}
\newcommand\BP{{\mathbb{P}}}
\newcommand\BQ{{\mathbb{Q}}}
\newcommand\BR{{\mathbb{R}}}
\newcommand\BS{{\mathbb{S}}}
\newcommand\BT{{\mathbb{T}}}
\newcommand\BU{{\mathbb{U}}}
\newcommand\BV{{\mathbb{V}}}
\newcommand\BW{{\mathbb{W}}}
\newcommand\BX{{\mathbb{X}}}
\newcommand\BY{{\mathbb{Y}}}
\newcommand\BZ{{\mathbb{Z}}}
\newcommand\CA{{\mathcal{A}}}
\newcommand\CB{{\mathcal{B}}}
\newcommand\CC{{\mathcal{C}}}
\providecommand\CD{{\mathcal{D}}}
\newcommand\CE{{\mathcal{E}}}
\newcommand\CF{{\mathcal{F}}}
\newcommand\CG{{\mathcal{G}}}
\newcommand\CH{{\mathcal{H}}}
\newcommand\CI{{\mathcal{I}}}
\newcommand\CJ{{\mathcal{J}}}
\newcommand\CK{{\mathcal{K}}}
\newcommand\CL{{\mathcal{L}}}
\newcommand\CM{{\mathcal{M}}}
\newcommand\CN{{\mathcal{N}}}
\newcommand\CO{{\mathcal{O}}}
\newcommand\CP{{\mathcal{P}}}
\newcommand\CQ{{\mathcal{Q}}}
\newcommand\CR{{\mathcal{R}}}
\newcommand\CS{{\mathcal{S}}}
\newcommand\CT{{\mathcal{T}}}
\newcommand\CU{{\mathcal{U}}}
\newcommand\CV{{\mathcal{V}}}
\newcommand\CW{{\mathcal{W}}}
\newcommand\CX{{\mathcal{X}}}
\newcommand\CY{{\mathcal{Y}}}
\newcommand\CZ{{\mathcal{Z}}}
\newcommand\RA{{\mathrm{A}}}
\newcommand\RB{{\mathrm{B}}}
\newcommand\RC{{\mathrm{C}}}
\newcommand\RD{{\mathrm{D}}}
\newcommand\RE{{\mathrm{E}}}
\newcommand\RF{{\mathrm{F}}}
\newcommand\RG{{\mathrm{G}}}
\newcommand\RH{{\mathrm{H}}}
\newcommand\RI{{\mathrm{I}}}
\newcommand\RJ{{\mathrm{J}}}
\newcommand\RK{{\mathrm{K}}}
\newcommand\RL{{\mathrm{L}}}
\newcommand\RM{{\mathrm{M}}}
% \newcommand\RN{{\mathrm{N}}}
\newcommand\RO{{\mathrm{O}}}
\newcommand\RP{{\mathrm{P}}}
\newcommand\RQ{{\mathrm{Q}}}
\newcommand\RR{{\mathrm{R}}}
\newcommand\RS{{\mathrm{S}}}
\newcommand\RT{{\mathrm{T}}}
\newcommand\RU{{\mathrm{U}}}
\newcommand\RV{{\mathrm{V}}}
\newcommand\RW{{\mathrm{W}}}
\newcommand\RX{{\mathrm{X}}}
\newcommand\RY{{\mathrm{Y}}}
\newcommand\RZ{{\mathrm{Z}}}
\newcommand\msa{\mathscr{A}}
\newcommand\msb{\mathscr{B}}
\newcommand\msc{\mathscr{C}}
\newcommand\msd{\mathscr{D}}
\newcommand\mse{\mathscr{E}}
\newcommand\msf{\mathscr{F}}
\newcommand\msg{\mathscr{G}}
\newcommand\msh{\mathscr{H}}
\newcommand\msi{\mathscr{I}}
\newcommand\msj{\mathscr{J}}
\newcommand\msk{\mathscr{K}}
\newcommand\msl{\mathscr{L}}
\newcommand\msm{\mathscr{M}}
\newcommand\msn{\mathscr{N}}
\newcommand\mso{\mathscr{O}}
\newcommand\msp{\mathscr{P}}
\newcommand\msq{\mathscr{Q}}
\newcommand\msr{\mathscr{R}}
\newcommand\mss{\mathscr{S}}
\newcommand\mst{\mathscr{T}}
\newcommand\msu{\mathscr{U}}
\newcommand\msv{\mathscr{V}}
\newcommand\msw{\mathscr{W}}
\newcommand\msx{\mathscr{X}}
\newcommand\msy{\mathscr{Y}}
\newcommand\msz{\mathscr{Z}}
\newcommand\fa{{\mathfrak{a}}}
\newcommand\fb{{\mathfrak{b}}}
\newcommand\fc{{\mathfrak{c}}}
\newcommand\fd{{\mathfrak{d}}}
\newcommand\fe{{\mathfrak{e}}}
\newcommand\ff{{\mathfrak{f}}}
\newcommand\fg{{\mathfrak{g}}}
\newcommand\fh{{\mathfrak{h}}}
\newcommand\fii{{\mathfrak{i}}} % be careful about \fii
\newcommand\fj{{\mathfrak{j}}}
\newcommand\fk{{\mathfrak{k}}}
\newcommand\fl{{\mathfrak{l}}}
\newcommand\fm{{\mathfrak{m}}}
\newcommand\fn{{\mathfrak{n}}}
\newcommand\fo{{\mathfrak{o}}}
\newcommand\fp{{\mathfrak{p}}}
\newcommand\fq{{\mathfrak{q}}}
\newcommand\fr{{\mathfrak{r}}}
\newcommand\fs{{\mathfrak{s}}}
\newcommand\ft{{\mathfrak{t}}}
\newcommand\fu{{\mathfrak{u}}}
\newcommand\fv{{\mathfrak{v}}}
\newcommand\fw{{\mathfrak{w}}}
\newcommand\fx{{\mathfrak{x}}}
\newcommand\fy{{\mathfrak{y}}}
\newcommand\fz{{\mathfrak{z}}}
\newcommand\fA{{\mathfrak{A}}}
\newcommand\fB{{\mathfrak{B}}}
\newcommand\fC{{\mathfrak{C}}}
\newcommand\fD{{\mathfrak{D}}}
\newcommand\fE{{\mathfrak{E}}}
\newcommand\fF{{\mathfrak{F}}}
\newcommand\fG{{\mathfrak{G}}}
\newcommand\fH{{\mathfrak{H}}}
\newcommand\fI{{\mathfrak{I}}}
\newcommand\fJ{{\mathfrak{J}}}
\newcommand\fK{{\mathfrak{K}}}
\newcommand\fL{{\mathfrak{L}}}
\newcommand\fM{{\mathfrak{M}}}
\newcommand\fN{{\mathfrak{N}}}
\newcommand\fO{{\mathfrak{O}}}
\newcommand\fP{{\mathfrak{P}}}
\newcommand\fQ{{\mathfrak{Q}}}
\newcommand\fR{{\mathfrak{R}}}
\newcommand\fS{{\mathfrak{S}}}
\newcommand\fT{{\mathfrak{T}}}
\newcommand\fU{{\mathfrak{U}}}
\newcommand\fV{{\mathfrak{V}}}
\newcommand\fW{{\mathfrak{W}}}
\newcommand\fX{{\mathfrak{X}}}
\newcommand\fY{{\mathfrak{Y}}}
\newcommand\fZ{{\mathfrak{Z}}}

% a
\newcommand\ab{{\mathrm{ab}}}
\newcommand\ad{{\mathrm{ad}}}
\newcommand\Ad{{\mathrm{Ad}}}
\newcommand\adele{ad\'{e}le}
\newcommand\Adele{Ad\'{e}le}
\newcommand\adeles{ad\'{e}les}
\newcommand\adelic{ad\'{e}lic}
\newcommand\AJ{{\mathrm{AJ}}}
\newcommand\alb{{\mathrm{alb}}}
\newcommand\Alb{{\mathrm{Alb}}}
\newcommand\alg{{\mathrm{alg}}}
\newcommand\an{{\mathrm{an}}}
\newcommand\ann{{\mathrm{ann}}}
\newcommand\Ann{{\mathrm{Ann}}}
\DeclareMathOperator\Arcsin{Arcsin}
\DeclareMathOperator\Arccos{Arccos}
\DeclareMathOperator\Arctan{Arctan}
\DeclareMathOperator\arsh{arsh}
\DeclareMathOperator\Arsh{Arsh}
\DeclareMathOperator\arch{arch}
\DeclareMathOperator\Arch{Arch}
\DeclareMathOperator\arth{arth}
\DeclareMathOperator\Arth{Arth}
\DeclareMathOperator\arccot{arccot}
\DeclareMathOperator\Arg{Arg}
\newcommand\arith{{\mathrm{arith}}}
\newcommand\Art{{\mathrm{Art}}}
\newcommand\AS{{\mathrm{AS}}}
\newcommand\Ass{{\mathrm{Ass}}}
\newcommand\Aut{{\mathrm{Aut}}}
% b
\newcommand\Bun{{\mathrm{Bun}}}
\newcommand\Br{{\mathrm{Br}}}
\newcommand\bs{\backslash}
\newcommand\BWt{{\mathrm{BW}}}
% c
\newcommand\can{{\mathrm{can}}}
\newcommand\cc{{\mathrm{cc}}}
\newcommand\cd{{\mathrm{cd}}}
\DeclareMathOperator\ch{ch}
\newcommand\Ch{{\mathrm{Ch}}}
\let\char\relax
\DeclareMathOperator\char{char}
\DeclareMathOperator\Char{char}
\newcommand\Chow{{\mathrm{CH}}}
\newcommand\circB{{\stackrel{\circ}{B}}}
\newcommand\cl{{\mathrm{cl}}}
\newcommand\Cl{{\mathrm{Cl}}}
\newcommand\cm{{\mathrm{cm}}}
\newcommand\cod{{\mathrm{cod}}}
\DeclareMathOperator\coker{coker}
\DeclareMathOperator\Coker{Coker}
\DeclareMathOperator\cond{cond}
\DeclareMathOperator\codim{codim}
\newcommand\cont{{\mathrm{cont}}}
\newcommand\Conv{{\mathrm{Conv}}}
\newcommand\corr{{\mathrm{corr}}}
\newcommand\Corr{{\mathrm{Corr}}}
\DeclareMathOperator\coim{coim}
\DeclareMathOperator\coIm{coIm}
\DeclareMathOperator\corank{corank}
\DeclareMathOperator\covol{covol}
\newcommand\cris{{\mathrm{cris}}}
\newcommand\Cris{{\mathrm{Cris}}}
\newcommand\CRIS{{\mathrm{CRIS}}}
\newcommand\crit{{\mathrm{crit}}}
\newcommand\crys{{\mathrm{crys}}}
\newcommand\cusp{{\mathrm{cusp}}}
\newcommand\CWt{{\mathrm{CW}}}
\newcommand\cyc{{\mathrm{cyc}}}
% d
\newcommand\Def{{\mathrm{Def}}}
\newcommand\diag{{\mathrm{diag}}}
\newcommand\diff{\,\mathrm{d}}
\newcommand\disc{{\mathrm{disc}}}
\newcommand\dist{{\mathrm{dist}}}
\renewcommand\div{{\mathrm{div}}}
\newcommand\Div{{\mathrm{Div}}}
\newcommand\dR{{\mathrm{dR}}}
\newcommand\Drin{{\mathrm{Drin}}}
% e
\newcommand\End{{\mathrm{End}}}
\newcommand\ess{{\mathrm{ess}}}
\newcommand\et{{\text{\'{e}t}}}
\newcommand\etale{{\'{e}tale}}
\newcommand\Etale{{\'{E}tale}}
\newcommand\Ext{\mathrm{Ext}}
\newcommand\CExt{\mathcal{E}\mathrm{xt}}
% f
\newcommand\Fil{{\mathrm{Fil}}}
\newcommand\Fix{{\mathrm{Fix}}}
\newcommand\fppf{{\mathrm{fppf}}}
\newcommand\Fr{{\mathrm{Fr}}}
\newcommand\Frac{{\mathrm{Frac}}}
\newcommand\Frob{{\mathrm{Frob}}}
% g
\newcommand\Ga{\mathbb{G}_a}
\newcommand\Gm{\mathbb{G}_m}
\newcommand\hGa{\widehat{\mathbb{G}}_{a}}
\newcommand\hGm{\widehat{\mathbb{G}}_{m}}
\newcommand\Gal{{\mathrm{Gal}}}
\newcommand\gl{{\mathrm{gl}}}
\newcommand\GL{{\mathrm{GL}}}
\newcommand\GO{{\mathrm{GO}}}
\newcommand\geom{{\mathrm{geom}}}
\newcommand\Gr{{\mathrm{Gr}}}
\newcommand\gr{{\mathrm{gr}}}
\newcommand\GSO{{\mathrm{GSO}}}
\newcommand\GSp{{\mathrm{GSp}}}
\newcommand\GSpin{{\mathrm{GSpin}}}
\newcommand\GU{{\mathrm{GU}}}
% h
\newcommand\hg{{\mathrm{hg}}}
\newcommand\Hk{{\mathrm{Hk}}}
\newcommand\HN{{\mathrm{HN}}}
\newcommand\Hom{{\mathrm{Hom}}}
\newcommand\CHom{\mathcal{H}\mathrm{om}}
% i
\newcommand\id{{\mathrm{id}}}
\newcommand\Id{{\mathrm{Id}}}
\newcommand\idele{id\'{e}le}
\newcommand\Idele{Id\'{e}le}
\newcommand\ideles{id\'{e}les}
\let\Im\relax
\DeclareMathOperator\Im{Im}
\DeclareMathOperator\im{im}
\newcommand\Ind{{\mathrm{Ind}}}
\newcommand\cInd{{\mathrm{c}\textrm{-}\mathrm{Ind}}}
\newcommand\ind{{\mathrm{ind}}}
\newcommand\Int{{\mathrm{Int}}}
\newcommand\inv{{\mathrm{inv}}}
\newcommand\Isom{{\mathrm{Isom}}}
% j
\newcommand\Jac{{\mathrm{Jac}}}
\newcommand\JL{{\mathrm{JL}}}
% k
\newcommand\Katz{\mathrm{Katz}}
\DeclareMathOperator\Ker{Ker}
\newcommand\KS{{\mathrm{KS}}}
\newcommand\Kl{{\mathrm{Kl}}}
\newcommand\CKl{{\mathcal{K}\mathrm{l}}}
% l
\newcommand\lcm{\mathrm{lcm}}
\newcommand\length{\mathrm{length}}
\DeclareMathOperator\Li{Li}
\DeclareMathOperator\Ln{Ln}
\newcommand\Lie{{\mathrm{Lie}}}
\newcommand\lt{\mathrm{lt}}
\newcommand\LT{\mathcal{LT}}
% m
\newcommand\mex{{\mathrm{mex}}}
\newcommand\MW{{\mathrm{MW}}}
\renewcommand\mod{\, \mathrm{mod}\, }
\newcommand\mom{{\mathrm{mom}}}
\newcommand\Mor{{\mathrm{Mor}}}
\newcommand\Morp{{\mathrm{Morp}\,}}
% n
\newcommand\new{{\mathrm{new}}}
\newcommand\Newt{{\mathrm{Newt}}}
\newcommand\nd{{\mathrm{nd}}}
\newcommand\NP{{\mathrm{NP}}}
\newcommand\NS{{\mathrm{NS}}}
\newcommand\ns{{\mathrm{ns}}}
\newcommand\Nm{{\mathrm{Nm}}}
\newcommand\Nrd{{\mathrm{Nrd}}}
\newcommand\Neron{N\'{e}ron}
% o
\newcommand\Obj{{\mathrm{Obj}\,}}
\newcommand\odd{{\mathrm{odd}}}
\newcommand\old{{\mathrm{old}}}
\newcommand\op{{\mathrm{op}}}
\newcommand\Orb{{\mathrm{Orb}}}
\newcommand\ord{{\mathrm{ord}}}
% p
\newcommand\pd{{\mathrm{pd}}}
\newcommand\Pet{{\mathrm{Pet}}}
\newcommand\PGL{{\mathrm{PGL}}}
\newcommand\Pic{{\mathrm{Pic}}}
\newcommand\pr{{\mathrm{pr}}}
\newcommand\Proj{{\mathrm{Proj}}}
\newcommand\proet{\text{pro\'{e}t}}
\newcommand\Poincare{\text{Poincar\'{e}}}
\newcommand\Prd{{\mathrm{Prd}}}
\newcommand\prim{{\mathrm{prim}}}
% r
\newcommand\Rad{{\mathrm{Rad}}}
\newcommand\rad{{\mathrm{rad}}}
\DeclareMathOperator\rank{rank}
\let\Re\relax
\DeclareMathOperator\Re{Re}
\newcommand\rec{{\mathrm{rec}}}
\newcommand\red{{\mathrm{red}}}
\newcommand\reg{{\mathrm{reg}}}
\newcommand\res{{\mathrm{res}}}
\newcommand\Res{{\mathrm{Res}}}
\newcommand\rig{{\mathrm{rig}}}
\newcommand\Rig{{\mathrm{Rig}}}
\newcommand\rk{{\mathrm{rk}}}
\newcommand\Ros{{\mathrm{Ros}}}
\newcommand\rs{{\mathrm{rs}}}
% s
\newcommand\sd{{\mathrm{sd}}}
\newcommand\Sel{{\mathrm{Sel}}}
\newcommand\sep{{\mathrm{sep}}}
\DeclareMathOperator{\sgn}{sgn}
\newcommand\Sh{{\mathrm{Sh}}}
\DeclareMathOperator\sh{sh}
\newcommand\Sht{\mathrm{Sht}}
\newcommand\Sim{{\mathrm{Sim}}}
\newcommand\sign{{\mathrm{sign}}}
\newcommand\SK{{\mathrm{SK}}}
\newcommand\SL{{\mathrm{SL}}}
\newcommand\SO{{\mathrm{SO}}}
\newcommand\Sp{{\mathrm{Sp}}}
\newcommand\Spa{{\mathrm{Spa}}}
\newcommand\Span{{\mathrm{Span}}}
\DeclareMathOperator\Spec{Spec}
\newcommand\Spf{{\mathrm{Spf}}}
\newcommand\Spin{{\mathrm{Spin}}}
\newcommand\Spm{{\mathrm{Spm}}}
\newcommand\srs{{\mathrm{srs}}}
\newcommand\rss{{\mathrm{ss}}}
\newcommand\ST{{\mathrm{ST}}}
\newcommand\St{{\mathrm{St}}}
\newcommand\st{{\mathrm{st}}}
\newcommand\Stab{{\mathrm{Stab}}}
\newcommand\SU{{\mathrm{SU}}}
\newcommand\Sym{{\mathrm{Sym}}}
\newcommand\sub{{\mathrm{sub}}}
\newcommand\rsum{{\mathrm{sum}}}
\newcommand\supp{{\mathrm{supp}}}
\newcommand\Supp{{\mathrm{Supp}}}
\newcommand\Swan{{\mathrm{Sw}}}
\newcommand\suml{\sum\limits}
% t
\newcommand\td{{\mathrm{td}}}
\let\tanh\relax
\DeclareMathOperator\tanh{th}
\newcommand\tor{{\mathrm{tor}}}
\newcommand\Tor{{\mathrm{Tor}}}
\newcommand\tors{{\mathrm{tors}}}
\newcommand\tr{{\mathrm{tr}\,}}
\newcommand\Tr{{\mathrm{Tr}}}
\newcommand\Trd{{\mathrm{Trd}}}
\newcommand\TSym{{\mathrm{TSym}}}
\newcommand\tw{{\mathrm{tw}}}
% u
\newcommand\uni{{\mathrm{uni}}}
\newcommand\univ{\mathrm{univ}}
\newcommand\ur{{\mathrm{ur}}}
\newcommand\USp{{\mathrm{USp}}}
% v
\newcommand\vQ{{\breve \BQ}}
\newcommand\vE{{\breve E}}
\newcommand\Ver{{\mathrm{Ver}}}
\newcommand\vF{{\breve F}}
\newcommand\vK{{\breve K}}
\newcommand\vol{{\mathrm{vol}}}
\newcommand\Vol{{\mathrm{Vol}}}
% w
\newcommand\wa{{\mathrm{wa}}}
% z
\newcommand\Zar{{\mathrm{Zar}}}


\setbeamertemplate{navigation symbols}{}

\begin{document}
\title{关于奇素因子模 $8$ 余 $\pm1$ 的非同余数}
\author{张神星}
\institute{2021年四川大学青年数论论坛\\四川~成都}
\date{2021年11月26日}

\frame{
\titlepage
}

\section{背景}

\begin{frame}
\frametitle{同余数与同余椭圆曲线}
设 $n$ 是一个平方自由的正整数. 如果 $n$ 可以表为一个有理边长直角三角形的面积, 则称 $n$ 是一个同余数.  
这等价于椭圆曲线
\[E=E_n: y^2=x^3-n^2x\]
的 Mordell-Weil 秩至少为 $1$.  
记 $\Sel_2(E)$ 为 $E/\BQ$ 的 $2$-Selmer 群,
\[s_2(n):=\dim_{\BF_2}\left(\frac{\Sel_2(E)}{E(\BQ)[2]}\right)
		=\dim_{\BF_2}\Sel_2(E)-2.\]  
由长正合列
\[0\ra E(\BQ)/2E(\BQ)\ra \Sel_2(E)\ra \Sha(E/\BQ)[2]\ra0\]
可知
\[s_2(n)=\rank_\BZ E(\BQ)+\dim_{\BF_2}\Sha(E/\BQ)[2].\]
\end{frame}


\begin{frame}
\frametitle{已知的结果}
Monsky 证明了 $n\equiv 1,2,3\bmod8$ 时, $s_2(n)$ 总是偶数.  
显然 $s_2(n)=0$ 时, $n$ 是非同余数. 该情形由田野-袁新意-张寿武 (2017) 和 Smith (2016) 完全刻画. 

我们来考虑何时 $n$ 是非同余数且 $s_2(n)=2$. 记 $h_{2^a}(m)$ 为 $\BQ(\sqrt{m})$ 的缩理想类群的 $2^a$ 阶秩.  
如果 $n$ 的素因子均模 $4$ 余 $1$, 王章结证明了:
\begin{theorem}[王章结, 2016]
设 $n=p_1\cdots p_k\equiv 1\bmod 8$ 的素因子均模 $4$ 余 $1$, 则下述等价:
\begin{itemize}
\item[(i)] $h_4(-n)=1, h_8(-n)\equiv (d-1)/4\pmod 2$;
\item[(ii)] $\rank_\BZ E_n(\BQ)=0,\Sha(E_n/\BQ)[2^\infty]\cong(\BZ/2\BZ)^2.$
\end{itemize}
这里, 要么 $d\neq 1,n$ 是满足 $(d,-n)_v=1,\forall v$ 的 $n$ 的正因子; 要么 $d$ 是满足 $(2d,-n)_v=1,\forall v$ 的 $n$ 的正因子.
\end{theorem}
\end{frame}

\section{主要结果}
\begin{frame}
\frametitle{主要结果: 奇数情形}
\begin{theorem}
设 $n=p_1\cdots p_k\equiv 1\bmod 8$ 的素因子均模 $8$ 余 $\pm1$, 
则下述等价:
\begin{itemize}
\item[(i)] $h_4(-n)=1, h_4(n)=0, \leg{-\mu}{d}=-1$;
\item[(ii)] $\rank_\BZ E_n(\BQ)=0,\Sha(E_n/\BQ)[2^\infty]\cong(\BZ/2\BZ)^2.$
\end{itemize}
这里, $d\neq 1$ 是满足 $(d,n)_v=1,\forall v$ 的 $n$ 的正因子, $n=2\mu^2-\tau^2$, 其中 $\mu\equiv d\bmod4$.
\end{theorem} 

\begin{corollary}
设 $n=p_1\cdots p_k\equiv 1\bmod 8$ 的素因子均模 $8$ 余 $1$, 
则下述等价:
\begin{itemize}
\item[(i)] $r_4(K_2\CO_{\BQ(\sqrt{n})})=0$;
\item[(ii)] $\rank_\BZ E_n(\BQ)=0,\Sha(E_n/\BQ)[2^\infty]\cong(\BZ/2\BZ)^2.$
\end{itemize}
\end{corollary}
\end{frame}

\begin{frame}
\frametitle{秦厚荣的结果}
\begin{theorem}[秦厚荣, 2021]
设素数 $p\equiv 1\bmod 8$. 如果 $r_8(K_2\CO_{\BQ(\sqrt{p})})=0$, 则 $p$ 是非同余数.
\end{theorem}
实际上, 该情形下可以证明完整的BSD猜想成立.
\end{frame}

\begin{frame}
\frametitle{主要结果: 偶数情形}
\begin{theorem}
设 $n=2p_1\cdots p_k\equiv 1\bmod 8$ 的奇素因子均模 $8$ 余 $\pm1$, 
则下述等价:
\begin{itemize}
\item[(i)] $h_4(-n/2)=1, \leg{2-\sqrt{2}}{|d|}=-1$;
\item[(ii)] $\rank_\BZ E_n(\BQ)=0,\Sha(E_n/\BQ)[2^\infty]\cong(\BZ/2\BZ)^2.$
\end{itemize}
这里, $d\neq 1$ 是满足 $(d,n)_v=1,\forall v$ 的 $n$ 的模 $8$ 余 $1$ 的因子.
\end{theorem} 

\begin{corollary}
设 $n=2p_1\cdots p_k\equiv 1\bmod 8$ 的奇素因子均模 $8$ 余 $1$, 
则下述等价:
\begin{itemize}
\item[(i)] $r_4(K_2\CO_{\BQ(\sqrt{-n})})=0$;
\item[(ii)] $\rank_\BZ E_n(\BQ)=0,\Sha(E_n/\BQ)[2^\infty]\cong(\BZ/2\BZ)^2.$
\end{itemize}
\end{corollary}
\end{frame}


\section{主要工具}
\begin{frame}
\frametitle{Selmer 群与齐性空间}
我们来回顾下 Monsky 矩阵 [HeathBrown1994].
$\Sel_2(E_n)$ 可以表为
\[\set{\Lambda=(d_1,d_2,d_3)\in(\BQ^\times/\BQ^{\times2})^3:
D_\Lambda(\BA_\BQ)\neq \emptyset,d_1d_2d_3\equiv 1\bmod\BQ^{\times 2}},\] 
其中
\[D_\Lambda=
	\begin{cases}
		H_1:& -nt^2+d_2u_2^2-d_3u_3^2=0,\\
		H_2:& -nt^2+d_3u_3^2-d_1u_1^2=0,\\
		H_3:& 2nt^2+d_1u_1^2-d_2u_2^2=0.
	\end{cases}\]
\begin{align*}
O&\mapsto(1,1,1)&(n,0)&\mapsto(2,2n,n)\\
(-n,0)&\mapsto(-2n,2,-n)&(0,0)&\mapsto(-n,n,-1)\\
(x,y)&\mapsto(x-n,x+n,x)&&
\end{align*}
\end{frame}

\begin{frame}
\frametitle{辅助矩阵和向量}
记 $n'=n/(2,n)=p_1\cdots p_k$.
记 $[a,b]_v, \aleg{a}{b}\in\BF_2$ 分别为加性希尔伯特符号和雅克比符号. 
定义
\[\bfA=\bfA_{n'}=(a_{ij})_{k\times k}
	\quad\text{其中}\quad
	a_{ij}=[p_j,-n']_{p_i}=
	\begin{cases}
		\aleg{p_j}{p_i},&i\neq j;\\
		\aleg{n'/p_i}{p_i},&i=j,
	\end{cases}\] 
\[\bfD_\varepsilon=\diag\left\{\aleg{\varepsilon}{p_1},\dots,\aleg{\varepsilon}{p_k}\right\}\qquad \bfb_\varepsilon=\bfD_\varepsilon{\bf1}.\]
则 $\bfA{\bf1}={\bf0}, \rank(\bfA)\le k-1$.
\end{frame}

\begin{frame}
\frametitle{Monsky 矩阵: 奇数情形}
当 $n$ 为奇数时, $\Sel_2(E_n)/E_n(\BQ)[2]$ 中元素可表为 $(d_1,d_2,d_3)$, 其中 $d_1,d_2$ 均为 $n$ 的正因子.
于是 $D_\Lambda$ 局部处处可解等价于特定的希尔伯特符号的条件. 
我们有同构
\[\fct{\Sel_2(E_n)/E_n(\BQ)[2]}{\Ker\bfM_n}{(d_1,d_2,d_3)}{\left(\begin{smallmatrix}
\psi^{-1}(d_2)\\ \psi^{-1}(d_1)
\end{smallmatrix}\right),}\]
其中
\[\psi\bigl((\delta_1,\dots,\delta_k)^\rmT\bigr)=\prod_{\delta_j=1} p_j,\]
\[\bfM_n=\begin{pmatrix}
		\bfA+\bfD_2&\bfD_2\\
		\bfD_2&\bfA+\bfD_{-2}
	\end{pmatrix}.\]
\end{frame}

\begin{frame}
\frametitle{Monsky 矩阵: 偶数情形}
类似地, 当 $n$ 为偶数时, $\Sel_2(E_n)/E_n(\BQ)[2]$ 中元素可表为 $(d_1,d_2,d_3)$, 其中 $d_2,d_3$ 均为 $n$ 的因子, $d_2>0, d_3\equiv1\bmod4$. 
我们有同构
\[\fct{\Sel_2(E_n)/E_n(\BQ)[2]}{\Ker\bfM_n}{(d_1,d_2,d_3)}{\left(\begin{smallmatrix}
\psi^{-1}(|d_3|)\\ \psi^{-1}(d_2)
\end{smallmatrix}\right),}\]
其中
\[\bfM_n=\begin{pmatrix}
		\bfA^\rmT+\bfD_{2}&\bfD_{-1}\\
		\bfD_2&\bfA+\bfD_2
	\end{pmatrix}.\]
特别地,
\[s_2(n)=2k-\rank(\bfM_n),\qquad\forall n.\]
\end{frame}

\begin{frame}
\frametitle{Cassels 配对}
Cassels 在 $\BF_2$ 线性空间 $\Sel_2(E_n)/E_n(\BQ)[2]$ 上定义了一个(反)对称双线性型.  
对于 $\Lambda,\Lambda'$, 选择 $P=(P_v)\in D_\Lambda(\BA)$, $Q_i\in H_i(\BQ)$. 令 $L_i$ 为定义了 $H_i$ 在 $Q_i$ 处切平面的线性型, 定义
\[\pair{\Lambda,\Lambda'}=\prod_v \pair{\Lambda,\Lambda'}_v,\qquad\text{其中}\ \pair{\Lambda,\Lambda'}_v=\prod_{i=1}^3 \bigl(L_i(P_v),d_i'\bigr)_v.\]
它不依赖 $P$ 和 $Q_i$ 的选择. 
\begin{lemma}[Cassels1998, Lemma~7.2]
如果 $p\nmid 2\infty$, $H_i$ 和 $L_i$ 的系数均是 $p$ 进整数, 且模 $p$ 后, $\ov{D}_\Lambda$ 仍定义了一条亏格 $1$ 的曲线并带有切平面 $\ov{L}_i=0$, 则 $\pair{-,-}_p=+1$.
\end{lemma}
\end{frame}

\begin{frame}
\frametitle{类群与 R\'edei 矩阵}
设 $m\neq 0,1$ 为无平方因子整数, $F=\BQ(\sqrt{m})$, $C(F)$ 为缩理想类群. 我们将其判别式进行分解
\[D=p_1^*\cdots p_t^*,\qquad p^*=(-1)^{\frac{p-1}2}p\equiv1\bmod4,\ 2^*=-4,8,-8.\]
高斯型理论告诉我们
\[h_2(m)=t-1.\] 
R\'edei证明了
\[\fct{\theta:\set{\text{$D$ 的平方自由且属于 $\bfN F$ 的正因子}}}{C(F)[2]\cap C(F)^2}{d}{\cl[\fd]}\]
是 $2:1$ 的满射, 其中 $(d)=\fd^2$.
将其用矩阵语言表达, 则
\[h_4(m)=t-1-\rank(\bfR_m),\qquad
\bfR_m=([p_j,m]_{p_i})_{ij}.\]
\end{frame}


\begin{frame}
\frametitle{温核 $K_2\CO_F$}
记 $K_2\CO_F$ 为 $\CO_F$ 的 $K_2$ 群, 则我们有正合列
\[1\ra K_2\CO_F\ra K_2F\ra \prod_p \BF_p^\times\ra 1,\]
其中第二个箭头是温剩余符号: $(a,b)_p=(-1)^{v(a)v(b)}b^{v(a)}a^{-v(b)}\bmod p$. 

Browkin-Schinzel (1982) 证明了 $K_2\CO_F[2]$ 可由如下Steinberg符号生成:
\begin{itemize}
\item $\set{-1,d},d\mid m$;
\item $\{-1,u+\sqrt{m}\}$, 其中 $m=u^2-cw^2,c=-1,\pm2,u,w\in\BN$.
\end{itemize}
\end{frame}


\begin{frame}
\frametitle{温核的 $4$ 阶秩}
设 $r_4=r_4(K_2\CO_F)$. 秦厚荣 (1995) 证明了如下结果:

(i) $m>0$ 时, $2^{r_4+1}=\#V_1+\#V_2$, 其中 
\[V_1=\set{\psi(\bfd): \bfB\bfd=\bfb_{\pm1},\bfb_{\pm2}},\qquad
V_2=\set{\psi(\bfd): \bfB\bfd=\bfb_{\pm\mu}}.\]

(ii) $m<0$ 时, $2^{r_4+2}=\#V_1+\#V_2$, 其中 
\[\begin{split}
V_1&=\set{\psi(\bfd): \bfB\bfd={\bf0},\bfb_2}\cup\set{-\psi(\bfd): \bfB\bfd=\bfb_{-1},\bfb_{-2}},\\
V_2&=\set{\psi(\bfd): \bfB\bfd=\bfb_{\mu}}\cup\set{-\psi(\bfd): \bfB\bfd=\bfb_{-\mu}}.
\end{split}\]

这里 $\bfB=\bfA_{m'}+\bfD_{m/m'}$, $m'=\frac{|m|}{(2,m)}, m=2\mu^2-\lambda^2,\mu,\lambda\in\BN$.
\end{frame}

\section{证明过程}
\begin{frame}
\frametitle{约化到 Cassels 配对非退化}
设 $s_2(n)=2$.
正合列
\[0\ra E[2]\ra E[4]\sto{\times 2}E[2]\ra 0\]
诱导了
\[0\ra E(\BQ)[2]/2E(\BQ)[4]\ra \Sel_2(E)\ra \Sel_4(E)\ra \Sel_2(E)\ra\cdots.\] 
于是
\[\rank_\BZ E(\BQ)=0,\quad\Sha(E/\BQ)[2^\infty]\cong(\BZ/2\BZ)^2\]
当且仅当 $\#\Sel_2(E)=\#\Sel_4(E)$, 即 $\#\Im\Sel_4(E)=\#E(\BQ)[2]$. 而 Cassels 配对的核是 $\Im\Sel_4(E)/E(\BQ)[2]$, 因此我们只需刻画何时 Cassels 配对非退化.
\end{frame}

\begin{frame}
\frametitle{奇数情形: $s_2(n)=2$}
设 $n$ 的素因子均模 $8$ 余 $\pm1$, 此时
\[\bfD_2=\bfO\qquad 
\bfM_n=\begin{pmatrix}
\bfA&\\&\bfA+\bfD_{-1}
\end{pmatrix}.\]
由于 $\bfA{\bf1}=(\bfA^\rmT+\bfD_{-1}){\bf1}={\bf0}$, 我们有
\[\rank(\bfA)\le k-1,\quad \rank(\bfA+\bfD_{-1})\le k-1.\]
而
\[\bfR_{-n}=\left(\begin{smallmatrix}
\bfA&{\bf0}\\\bfb_{-1}^\rmT&0
\end{smallmatrix}\right),\quad
\bfR_n=\bfA+\bfD_{-1}\]
且 ${\bf1}^\rmT\bfA=\bfb_{-1}^\rmT$, 因此 $s_2(n)=2\iff h_4(-n)=1, h_4(n)=0$.
\end{frame}

\begin{frame}
\frametitle{奇数情形: 齐性空间}
设 $d\neq 1$ 是唯一满足 $(d,n)_v=1,\forall v$ 的 $n$ 的正因子. 换言之, $(\bfA+\bfD_{-1})\psi^{-1}(d)={\bf0}$. 于是 $E_n$ 的纯 $2$-Selmer 群为
\[\{(1,1,1), \Lambda=(1,n,n), \Lambda'=(d,1,d), (d,n,nd)\}.\]
\[D_\Lambda=\begin{cases}
H_1:& -t^2+u_2^2-u_3^2=0,\\
H_2:& -nt^2+nu_3^2-u_1^2=0,\\
H_3:& 2nt^2+u_1^2-nu_2^2=0.
\end{cases}\]
令 $n=2\mu^2-\tau^2$, 则 $\mu$ 是奇数且 $n=u^2-2w^2$, $u=2\mu-\tau, w=-\mu+\tau$. 选择
\begin{align*}
Q_1&=(0,1,1)\in H_1(\BQ),& L_1&=u_2-u_3,\\
Q_3&=(w,n,u)\in H_3(\BQ),& L_3&=2wt+u_1-uu_2.
\end{align*}
\end{frame}

\begin{frame}
\frametitle{奇数情形: Cassels 配对}
对于 $p\mid n$, 取 $(t,u_1,u_2,u_3)=(1,0,\sqrt{2},1)$, 其中 $\sqrt{2}\equiv -u/w\bmod p$,
\[L_1L_3(P_p)=(\sqrt{2}-1)(2w-\sqrt{2}u)\equiv 4(\sqrt{2}-1)w\equiv -4\mu\mod p,\]
\[\bigl(L_1L_3(P_p),d\bigr)_p=(-\mu,d)_p.\] 
对于 $p=2$, 取 $(t,u_1,u_2,u_3)=(0,\sqrt{n},1,-1)$,
\[\bigl(L_1L_3(P_2),d\bigr)_2=\bigl(2(\sqrt{n}-u),d\bigr)_2=(-\sqrt{n}-u,d)_2=(-\mu,d)_2.\]
由于 $\mu\equiv d\bmod 4$, 于是我们有 $(-\mu,d)_2=1$,
\[\pair{\Lambda,\Lambda'}=\prod_{p\mid d}(-\mu,d)_p=\leg{-\mu}{d}.\]
\end{frame}

\begin{frame}
\frametitle{奇数情形: 推论}
如果所有的 $p_i\equiv1\bmod8$, 则 $\bfD_{\pm1}=\bfD_{\pm2}=\bfO$, $d=n$. 令 $F=\BQ(\sqrt{n})$, 则
\[2^{r_4(K_2\CO_F)+1}=\#\set{\bfd: \bfA\bfd={\bf0}}+\#\set{\bfd:\bfA\bfd=\bfb_{|\mu|}=\bfb_\mu}.\]
由于 $\Ker \bfA\supseteq\set{{\bf0},{\bf1}}$, 因此 $r_4(K_2\CO_F)=0$ 当且仅当 $\rank (\bfA)=k-1$ 且 $\bfb_\mu\notin\Im\bfA$. 但此时
\[\Im\bfA=\set{\bfd:{\bf1}^\rmT\bfd={\bf0}},\]
因此 $\leg{\mu}{n}=-1$.
\end{frame}

\begin{frame}
\frametitle{偶数情形: $s_2(n)=2$}
设 $n=2n'$ 的奇素因子均模 $8$ 余 $\pm1$, 此时
\[\bfD_2=\bfO\qquad 
\bfM_n=\begin{pmatrix}
\bfA^\rmT&\bfD_{-1}\\
&\bfA
\end{pmatrix}.\]
由于 $\bfA{\bf1}={\bf0}$, 我们有 $\rank(\bfA)\le k-1$.
方程 $\bfM_n\left(\begin{smallmatrix}
\bfx\\\bfy
\end{smallmatrix}\right)={\bf0}$ 等价于
\[\bfA^\rmT \bfx=\bfD_{-1}\bfy,\quad \bfA\bfy={\bf0}.\]
若 $\bfy={\bf0}$, 则 $\bfA^\rmT\bfx={\bf0}$ 至少有两个解.
若 $\bfy={\bf1}$, 则 $\bfA^\rmT(\bfx+{\bf1})={\bf0}$ 至少有两个解. 因此 $s_2(n)=\dim_{\BF_2}\Ker\bfM_n\ge 2$ 且
\[s_2(n)=2\iff \rank(\bfA)=k-1\iff h_4(-n/2)=1.\]
\end{frame}

\begin{frame}
\frametitle{偶数情形: 齐性空间}
设 $d\neq 1$ 是唯一满足 $(d,n)_v=1,\forall v$ 的 $n$ 的模 $8$ 余 $1$ 的因子. 换言之, $\bfA^\rmT\psi^{-1}(|d|)={\bf0}$. 于是 $E_n$ 的纯 $2$-Selmer 群为
\[\{(1,1,1), \Lambda=(1,n',n'), \Lambda'=(d,1,d), (d,n',n'd)\}.\]
\[D_\Lambda=\begin{cases}
H_1:& -2t^2+u_2^2-u_3^2=0,\\
H_2:& -2n't^2+n'u_3^2-u_1^2=0,\\
H_3:& 4n't^2+u_1^2-n'u_2^2=0.
\end{cases}\] 
选择
\begin{align*}
Q_1&=(0,1,1)\in H_1(\BQ),& L_1&=u_2-u_3,\\
Q_3&=(n'-1,4n',2n'+2)\in H_3(\BQ),& L_3&=2(n'-1)t+2u_1-(n'+1)u_2.
\end{align*}
\end{frame}

\begin{frame}
\frametitle{偶数情形: Cassels 配对}
对于 $p\mid n$, 取 $(t,u_1,u_2,u_3)=(1,0,2,\sqrt{2})$, 
\[\bigl(L_1L_3(P_p),d\bigr)_p=\bigl(-2(2-\sqrt{2}),d\bigr)_p=(\sqrt{2}-2,d)_p.\] 
对于 $p=2$, 取 $(t,u_1,u_2,u_3)=(0,\sqrt{n'},1,-1)$,
\[L_1L_3(P_2)=2(2\sqrt{n'}-n'-1)=-2(\sqrt{n'}-1)^2,\]
\[\bigl(L_1L_3(P_2),d\bigr)_2=(-2,d)_2=(-1,d)_2.\]
因此
\[\pair{\Lambda,\Lambda'}=(-1,d)_2\prod_{p\mid d}(\sqrt{2}-2,d)_p=\leg{2-\sqrt{2}}{|d|}.\]
\end{frame}

\begin{frame}
\frametitle{偶数情形: 推论}
如果所有的 $p_i\equiv1\bmod8$, 则 $\bfD_{\pm1}=\bfD_{\pm2}=\bfO$, $d=n'$. 令 $F=\BQ(\sqrt{-n})$, $-n=2\mu^2-\lambda^2,\mu,\lambda\in\BN$, 则
\[2^{r_4(K_2\CO_F)+2}=2\#\set{\bfd: \bfA\bfd={\bf0}}+2\#\set{\bfd:\bfA\bfd=\bfb_\mu}.\]
由于 $\Ker \bfA\supseteq\set{{\bf0},{\bf1}}$, 因此 $r_4(K_2\CO_F)=0$ 当且仅当 $\rank (\bfA)=k-1$ 且 $\bfb_\mu\notin\Im\bfA$.  
但此时
\[\Im\bfA=\set{\bfd:{\bf1}^\rmT\bfd={\bf0}},\]
因此 $\leg{\mu}{n'}=-1$.
记 $u=\lambda-\mu, w=-\lambda/2+\mu$, 则 $n'=u^2-2w^2$,
\[\leg{w}{n'}=\leg{n'}{w'}=\leg{u^2-2w^2}{w'}=\leg{u^2}{w'}=1,\]
\[\leg{\mu}{n'}=\leg{u+2w}{n'}=\leg{(2\pm\sqrt{2})w}{n'}=\leg{2\pm\sqrt{2}}{n'}.\]
\end{frame}

\begin{frame}
\begin{center}
\huge 感谢各位的倾听!
\end{center}
\end{frame}


\end{document}


