\documentclass[aspectratio=169,handout]{beamer}
\usepackage{../latex/bamboo}
\setbeamertemplate{footline}[naviboxall]
\definecolor{main}{RGB}{0,0,224}%
\definecolor{second}{RGB}{224,0,0}%
\definecolor{third}{RGB}{112,0,112}%
\definecolor{fourth}{RGB}{0,128,0}%
\definecolor{fifth}{RGB}{255,128,0}%

% Black Theme
% \definecolor{main}{RGB}{0,0,0}%
% \definecolor{second}{RGB}{0,0,0}%
% \definecolor{third}{RGB}{0,0,0}%
% \definecolor{fourth}{RGB}{0,0,0}%
% \definecolor{fifth}{RGB}{0,0,0}%

% Doremi Theme
% \definecolor{main}{RGB}{73,119,213}%
% \definecolor{second}{RGB}{220,60,105}%
% \definecolor{third}{RGB}{122,89,166}%
% \definecolor{fourth}{RGB}{229,143,52}%
% \definecolor{fifth}{RGB}{244,220,10}%

\usepackage{bm}
\usepackage{extarrows}
\usepackage{mathrsfs}
\usepackage{stmaryrd}
\usepackage{multirow}
\usepackage{tikz}
\usepackage{color,calc}
\usepackage{caption}
\usepackage{subcaption}
\usepackage{booktabs}
% \usepackage{wrapfig}
% 文本色
\renewcommand\emph[1]{{\color{main}{\bf #1}}}
\ifcsundef{alert}{
  \newcommand{\alert}[1]{\textcolor{second}{\bf #1}}
}{}
% 非考试内容
\newcommand{\noexer}{\hfill\mdseries\itshape\color{black}\small 非考试内容}
% 枚举数字引用标志
\newcommand\enumnum[1]{{\mdseries\upshape\textcolor{main}{(#1)}}}
% 缩进
\renewcommand{\indent}{\hspace*{1em}}
\setlength{\parindent}{1em}
% 减少公式垂直间距, 配合\endgroup
\newcommand\beqskip[1]{\begingroup\abovedisplayskip=#1\belowdisplayskip=#1\belowdisplayshortskip=#1}

\RequirePackage{tasks}
\settasks{label-format=\textcolor{main},label={(\arabic*)},label-width=1.5em}
% 选择题选项
\NewTasksEnvironment[label={\upshape(\Alph*)},label-width=1.5em]{taskschoice}[()]
% 一行显示多题答案
\NewTasksEnvironment[label={\arabic*.},label-width=1.5em]{tasksans}[()]
% TIKZ 设置
\usetikzlibrary{
	quotes,
	shapes.arrows,
	arrows.meta,
	positioning,
	shapes.geometric,
	overlay-beamer-styles,
	patterns,
	calc,
	angles,
	decorations.pathreplacing,
	backgrounds % 背景边框
}
\tikzset{
	background rectangle/.style={semithick,draw=fourth,fill=white,rounded corners},
  % arrow
	cstra/.style      ={-Stealth},        % right arrow
	cstla/.style      ={Stealth-},        % left arrow
	cstlra/.style     ={Stealth-Stealth}, % left-right arrow
	cstwra/.style     ={-Straight Barb},  % wide ra
	cstwla/.style     ={Straight Barb-},
	cstwlra/.style    ={Straight Barb-Straight Barb},
	cstnra/.style      ={-Latex, line width=0.1cm},
	cstmra/.style      ={-Latex, line width=0.05cm},
	cstmlra/.style     ={Latex-Latex, line width=0.05cm},
	cstaxis/.style        ={-Stealth, thick}, %坐标轴
  % curve
	cstcurve/.style       ={very thick}, %一般曲线
	cstdash/.style        ={thick, dash pattern= on 0.2cm off 0.05cm}, %虚线
  % dot
	cstdot/.style         ={radius=.08}, %实心点
	cstdote/.style        ={radius=.07, fill=white}, %空心点
  % fill
	cstfill/.style       ={fill=black!10},
	cstfille/.style      ={pattern=north east lines, pattern color=black},
	cstfill1/.style       ={fill=main!20},
	cstfille1/.style      ={pattern=north east lines, pattern color=main},
	cstfill2/.style        ={fill=second!20},
	cstfille2/.style       ={pattern=north east lines, pattern color=second},
	cstfill3/.style        ={fill=third!20},
	cstfille3/.style       ={pattern=north east lines, pattern color=third},
	cstfill4/.style        ={fill=fourth!20},
	cstfille4/.style       ={pattern=north east lines, pattern color=fourth},
	cstfill5/.style        ={fill=fifth!20},
	cstfille5/.style       ={pattern=north east lines, pattern color=fifth},
  % node
	cstnode/.style        ={fill=white,draw=black,text=black,rounded corners=0.2cm,line width=1pt},
	cstnode1/.style       ={fill=main!15,draw=main!80,text=black,rounded corners=0.2cm,line width=1pt},
	cstnode2/.style       ={fill=second!15,draw=second!80,text=black,rounded corners=0.2cm,line width=1pt},
	cstnode3/.style       ={fill=third!15,draw=third!80,text=black,rounded corners=0.2cm,line width=1pt},
	cstnode4/.style       ={fill=fourth!15,draw=fourth!80,text=black,rounded corners=0.2cm,line width=1pt},
	cstnode5/.style       ={fill=fifth!15,draw=fifth!80,text=black,rounded corners=0.2cm,line width=1pt}
}
\newcommand\invitedby[2][hfut]{
	\def\insertoffice{#2}
	\def\inserttitlegraphic{../image/logo/#1.png}
	\def\insertinstitutegraphic{../image/logo/#1name.png}
}
\email{zhangshenxing@hfut.edu.cn}
\website{https://zhangshenxing.github.io}
\date{\today}
\author{Shenxing Zhang}
\institute{HFUT}
\AtBeginSection{}
\showprefixfalse
\website{}
\newcommand\frameoutline{{\setnaviboxempty\begin{frame}{Outline}\tableofcontents\end{frame}}}
\AtEndDocument{{\setnaviboxempty\begin{frame}\begin{center}\includegraphics[width=70mm]{../image/thankyou.png}\end{center}\end{frame}}}

\title{On the generating fields of Kloosterman sums}
\office{Liaocheng University}
\logo{lcu}
\date{2021-05-17}

\newcommand\bchi{{\boldsymbol\chi}}
\newcommand\brho{{\boldsymbol\rho}}

\begin{document}

\frameoutline

\section{Exponential Sums}

\begin{frame}{Exponential sums}
Let $f(x)\in\BF_q[x]$ be a polynomial over a finite field with $q=p^d$ elements, where $p$ is a rational prime. \pause
Define the exponential sum
	\[S_1(f):=\sum_{x\in\BF_q}\zeta_p^{\Tr(f(x))}\in\BZ[\zeta_p].\]
\pause A basic problem is 
\begin{enumerate}
\item as a complex number, $|S_1(f)|=?$
\item as a $p$-adic number, $|S_1(f)|_p=?$
\item as an algebraic number, $\deg S_1(f)=?$
\end{enumerate}
\end{frame}


\begin{frame}{$L$-function}
The first two questions have been studied extensively in the literature.
Define 
	\[L(t,f):=\prod_{x\in \ov{\BF}_p}\Big(1-\Tr_{\BF_q(x)/\BF_p}(f(x))t^{\deg x}\Big)^{-1}=\exp\Bigl(\sum_k S_k(f)\frac{t^k}{k}\Bigr)\]
where $S_k(f):=\sum_{x\in\BF_{q^k}}\zeta_p^{\Tr(f(x))}\in\BZ[\zeta_p].$
 \pause
\begin{theorem}[Dwork-Bombieri-Grothendick]
$L(t,f)$ is a rational function.
\end{theorem} \pause
Write
	\[L(t,f)=\frac{\prod_j (1-\beta_j t)}{\prod_i(1-\alpha_i t)}.\]
Then
	\[S_k(f)=\sum_i \alpha_i^k-\sum_j\beta_j^k.\]
\end{frame}


\begin{frame}{Sheaf}
How to estimate the characteristic roots $\alpha_i$ and $\beta_j$? We need $\ell$-adic method.
To describe it, let's recall the definition of sheaves. \pause

Given a topological space $X$, there is a site $\cTop(X)$ with 
\begin{enumerate}
\item objects: the open subsets of $X$;
\item morphisms:  the injection of open sets;
\item coverings: normal open coverings.
\end{enumerate} \pause
A sheaf $\CF$ on a topological space $X$ over a field $E$ is a contravariant functor $\cTop(X)^\op\to\cVect/E$, which can be uniquely glued locally. That's to say, for any open covering $U=\cup_i U_i$,
	\[\CF(U)\to \prod_i\CF(U_i)\rightrightarrows \prod_{i,j}\CF(U_i\cap U_j)\]
is exact.
\end{frame}


\begin{frame}{\Etale\ site}
Let $X$ be a scheme. Denote by $X_\et$ the site with
\begin{enumerate}
\item objects: \etale\ scheme $X'\to X$;
\item morphisms: \etale\ morphisms;
\item coverings: $\set{\varphi_i:X'_i\to X'}$ with $X'=\cup \varphi_i(X'_i)$.
\end{enumerate} \pause
Fix a prime $\ell\neq p$ and let $E$ be a finite extension of $\BQ_\ell$.
An $\ell$-adic sheaf is a sheaf on $X_\et$ over $E$ (which is constructible at every finite level).
\end{frame}


\begin{frame}{Swan conductor}
Let $K$ be c.d.v.f, with higher ramification groups $I^{(r)},r\ge 0$. \pause
For any $E$-representation $M$ of $P$, we have a decomposition $M=\oplus M(x)$, such that
	\[M(0)=M^P,\quad M(x)^{I^{(x)}}=0,\quad M(x)^{I(y)}=M(x), \ y>x>0.\]
	\pause We call $x$ a break if $M(x)\neq 0$. Define
	\[\Swan(M)=\sum x\dim M(x).\]
\end{frame}


\begin{frame}{Curves}
Let $C$ be a proper smooth geometrically connected curve over a perfect field $\BF$, with function field $K=\BF(C)$.
For any closed point $x\in C(\BF)$, we have the completion $K_x$.

\pause
For any non-empty open $U\subset C$, we have an equivalence of abelian categories
	\[\begin{split}
		\set{\text{lisse $E$-sheaves on $U$}}&\lra\cRep_E^c \pi_1(U,\ov\eta)\\
		\CF&\longmapsto \CF_{\ov\eta}.
	\end{split}\]
	\pause
Since $\pi_1(U,\ov\eta)$ is a quotient of $\Gal(\ov K/K)$, the decomposition group $D_x\subset \Gal(\ov K/K)$ acts on $\CF_{\ov\eta}$.
We can define Swan conductor of $\CF$ at $x$.
If $x\in U$, the action of $I_x$ is trivial.

We will take $\BF=\BF_p,C=\BP^1$ and $U=\Gm$.
\end{frame}


\begin{frame}{$\ell$-adic method}
Assume that $\mu_p\subseteq E$.
Deligne constructed a certain locally free of rank one $\ell$-adic sheaf $\CF_\ell(f)$ over $E$ on ${\Ga}_{,\ov\BF_p}=\Spec \ov\BF_p[X]$, such that
	\[L(t,f)=\prod_i \det(1-t \Frob,\RH^i_c)^{(-1)^{i+1}}\]
and
	\[S_k(f)=\sum_i (-1)^i\Tr(\Frob^k,\RH_c^i).\]
Here, $\Frob$ is the geometric Frobenius (inverse of $\alpha\mapsto \alpha^p$), $\RH_c^i=\RH_c^i({\Ga}_{,\ov\BF_p},\CF_\ell(f))$ is the compact cohomology.
\end{frame}


\begin{frame}{$\ell$-adic method, continue}
Denote by $\omega_{ij}$ the eigenvalues of $\Frob$ on $\RH^i_c$, then
	\[S_k(f)=\sum_{ij}(-1)^i \omega_{ij}^k.\]
Denote by $B_i=\dim_{E}\RH^i_c$ the Betti number.
\begin{theorem}[Deligne]
$\omega_{ij}$ is an algebraic integer and all its conjugates over $\BQ$ has same absolute value $q^{r_{ij}/2}$, where the weight $0\le r_{ij}\le i$ are integers.
\end{theorem}
Thus
	\[|S_k|\le \sum_i B_i q^{ki/2}.\]
\end{frame}


\begin{frame}{General case}
In general, 
\begin{enumerate}
\item $V$ a closed variety over $\BF_q$ of $\BA^N$,
\item $\psi$ a non-trivial additive character on $\BF_q$, $\psi_k=\psi\circ \Tr_{\BF_{q^k}/\BF_q}$,
\item $f$ a regular function on $V$ defined over $\BF_q$,
\item $\chi$ a multiplicative character on $\BF_q^\times$, $\chi_k=\chi\circ\bfN_{\BF_{q^n}/\BF_q}$,
\item $g$ an invertible regular function on $V$.
\end{enumerate} \pause
Define
	\[S_k=\sum_{x\in V(\BF_{q^k})}\psi_k(f(x))\chi_k(g(x)).\]
Then Deligne's results still hold in this case. Moreover, Bombieri proved that the number of  characteristic roots is at most
	\[(4\max\set{\deg V+1,\deg f}+5)^{2N+1}.\]
\end{frame}



\section{Kloosterman sheaves}

\begin{frame}{Kloosterman sums}
Now we will consider
	\[V=V(X_1\cdots X_n-a),\quad f=X_1+\cdots+X_n.\]
Let $\bchi=\set{\chi_1,\dots,\chi_n}$ be an unordered $n$-tuple of multiplicative characters $\chi_i:\BF_q^\times\to \mu_{q-1}$.
Define the Kloosterman sum as 
	\[\Kl_n(\psi,\bchi,q,a)=\sum_{x_1\cdots x_n=a\atop x_i\in\BF_q}\chi_1(x_1)\cdots\chi_n(x_n)\psi\bigl(\Tr_{\BF_q/\BF_p}(x_1+\cdots+x_n)\bigr).\] \pause

In this case, there are $n$ characteristic roots with same weight $n-1$. Hence $|\Kl_n|\le nq^{(n-1)/2}$.
\end{frame}


\begin{frame}{Galois action}
Clearly, $\Kl_n\in\BZ[\mu_{pc}]$, where
	\[c=\lcm_i \set{\ord(\chi_i)}\]
divides $q-1$. \pause
Write
	\[\Gal(\BQ(\mu_{pc})/\BQ)=\set{\sigma_t\tau_w\mid t\in (\BZ/p\BZ)^\times,w\in(\BZ/c\BZ)^\times},\]
where
	\[\sigma_t(\zeta_p)=\zeta_p^t,\quad \sigma_t(\zeta_c)=\zeta_c,\]
	\[\tau_w(\zeta_p)=\zeta_p,\quad \tau_w(\zeta_c)=\zeta_c^w.\] \pause
A basic observation tells
	\[\sigma_t\tau_w\Kl_n(\psi,\bchi,q,a)=\prod\bchi(t)^{-w}\Kl_n(\psi,\bchi^w,q,at^n).\]
To study the generating fields of $\Kl_n$, we need to consider the distinctness of different Kloosterman sums.

\end{frame}


\begin{frame}{Trivial character}
When $\bchi={\bf1}=\set{1,\dots,1}$ is trivial, it's easy to see that
	\[a,b\ \text{conjugate} \implies \Kl_n(\psi,{\bf1},q,a)=\Kl_n(\psi,{\bf1},q,b).\] \pause
When $p>(2n^{2d}+1)^2$ (Fisher), or $p\ge(d-1)n+2$ and	$p$ does not divide a certain integer (Wan), this is necessary. In general, it's conjectured that it's true when $p\ge nd$.
Thus 
	\[\deg \Kl_n(\psi,{\bf1},q,a)=\frac{p-1}{(p-1,n)}\]
under these conditions.

\end{frame}


\begin{frame}{Kloosterman sheaves}
For our purpose, we need a different sheaf.
Deligne and Katz defined the Kloosterman sheaf
	\[\CKl=\CKl_{n,q}(\psi,\bchi)\]
on $\Gm\otimes \BF_q=\Spec \BF_q[X,X^{-1}]$, with the following properties:
\begin{enumerate}
\item $\CKl$ is lisse (locally constant at every finite level) of rank $n$ and pure of weight $n-1$.
\item For any $a\in \BF_q^\times$,
	$\Tr\bigl(\Frob_a,\CKl_{\ov a})=(-1)^{n-1}\Kl_n(\psi,\bchi,q,a).$
\item $\CKl$ is tame at $0$ (Swan$=0$).
\item $\CKl$ is totally wild with Swan conductor $1$ at $\infty$. So all $\infty$-breaks are $1/n$.
\end{enumerate}

\end{frame}


\begin{frame}{Fisher's descent}
Fisher gave a descent of Kloosterman sheaves along an extension of finite fields.
For any $a\in\BF_q^\times$, he defined a lisse sheaf $\CF_a(\bchi)$ on $\Gm\otimes\BF_p$, such that
$\CF_a(\bchi)|\Gm\otimes\BF_q=\bigotimes\limits_{\sigma\in\Gal(\BF_q/\BF_p)} \bigl(t\mapsto \sigma(a)t^n\bigr)^*\CKl_n(\psi\circ\sigma^{-1},\bchi\circ\sigma^{-1})$.
\begin{enumerate}
\item $\CF_a(\bchi)$ is lisse of rank $n^d$ and pure of weight $d(n-1)$.
\item For any $t\in\BF_p^\times$,
	$\Tr\bigl(\Frob_t,\CF_a(\bchi)_{\ov t}\bigr)=(-1)^{(n-1)d}\Kl_n(\psi,\bchi,q,at^n).$
\item $\CF_a(\bchi)$ is tame at $0$ and its $\infty$-breaks are at most $1$.
\end{enumerate}

\end{frame}


\begin{frame}{Key lemma}
\begin{lemma}
Let $\CF,\CF'$ be lisse sheaves on $\Gm\otimes\BF_p$ of same rank $r$ and pure of the same weight $w$.
Assume that there is a root of unity $\lambda$ such that for any $t\in \BF_p^\times$, we have 
	\[\Tr\bigl(\Frob_t, \CF_{\ov t})=\lambda\Tr\bigl(\Frob_t, \CF'_{\ov t}).\]
Let $\CG$ be a geometrically irreducible sheaf of rank $s$ on $\Gm\otimes\BF_p$, pure of weight $w$, such that $\CG\mid \Gm\otimes\ov\BF_p$ occurs exactly once in $\CF\mid \Gm\otimes\ov \BF_p$.
Then $\CG\mid \Gm\otimes\ov \BF_p$ occurs at least once in $\CF'\mid \Gm\otimes\ov \BF_p$, provided that $p>[2rs(M_0+M_\infty)+1]^2$, where $M_\eta$ is the largest $\eta$-break of $\CF\oplus \CF'$.
\end{lemma}

\end{frame}


\begin{frame}{Key lemma, proof}
Assume not. 
Applying the Lefschetz Trace Formula to $\CG^\vee\otimes\CF$ and $\CG^\vee\otimes \CF'$, we have
	\[\sum_{i=0}^2(-1)^i\Tr\bigl(\Frob,\RH_c^i(\CG^\vee\otimes\CF)\bigr)
		=\lambda\sum_{i=0}^2(-1)^i\Tr\bigl(\Frob, \RH_c^i(\CG^\vee\otimes\CF')\bigr).\] \pause
Apply  Euler-Poincar\'e formula
	\[\begin{split}
	&h^0_c(\CF)-h^1_c(\CF)+h^2_c(\CF)\\
	=&\rank \CF\cdot \chi_c(\Gm\otimes\BF_p)-\Swan_0(\CF)-\Swan_\infty(\CF)\end{split}\]
to estimate $\Tr(\Frob,\RH^1_c)$ (weight $\le1$ by Weil II).

\end{frame}


\begin{frame}{Corollary}
The $n$-tuple $\bchi$ is called {\em Kummer-induced} if there exsists a non-trivial character $\Lambda$ such that $\bchi=\bchi\Lambda:=\set{\chi_1\Lambda,\dots,\chi_n\Lambda}$ as unordered $n$-tuples.
In this case, $\prod\bchi=\prod(\bchi\Lambda)=\Lambda^n\prod\bchi$ and thus $\Lambda^n=1$. \pause

Assume that $p>2n+1$ and $\bchi$ is not Kummer-induced. \pause
Then $\CF_a(\bchi)$ has a highest weight with multiplicity one.  \pause
Thus it has a subsheaf $\CG_a(\bchi)$ such that, as representations of the Lie algebra $\fg(\CF_a(\bchi))$, $\CG_a(\bchi)$ is the irreducible sub-representation with highest weight. \pause
Moreover, it is geometrically irreducible and occurs exactly once in $\CF_a(\bchi)$ over $\Gm\otimes\ov\BF_p$.

\end{frame}


\begin{frame}{Corollary, continue}
\begin{corollary}
Let $a,b\in\BF_q^\times$ and let $\bchi$ and $\brho$ be $n$-tuples of multiplicative characters $\chi_i,\rho_j:\BF_q^\times\to\ov\BQ_\ell^\times$.
Assume that $p>(2n^{2d}+1)^2$, $\bchi$ is not Kummer-induced	and 
	\[\Kl_n(\psi,\bchi,q,a)=\lambda\Kl_n(\psi,\brho,q,b)\]
for a fixed root of unity $\lambda\in\mu_{q-1}$.
Then $\CG_a(\bchi)\otimes \CL_{\prod \ov\bchi}\mid \Gm\otimes\ov \BF_p$ occurs at least once in $\CF_b(\brho)\otimes\CL_{\prod\ov \brho}\mid \Gm\otimes\ov \BF_p$.
\end{corollary}

Here $\CL_\chi$ is a rank one lisse sheaf on $\Gm\otimes\BF_p$ such that for $t\in\BF_p^\times$,
	\[\Tr(\Frob_t,(\CL_\chi)_{\ov t})=\chi(t).\]

\end{frame}


\begin{frame}{Corollary, proof}
Denote by
	\[\CF=\CF_a(\bchi)\otimes\CL_{\prod\ov \bchi},\	\CF'=\CF_b(\brho)\otimes\CL_{\prod\ov\brho},\ \CG=\CG_a(\bchi)\otimes\CL_{\prod\ov\bchi}.\]
For $t\in\BF_p^\times$, we have $\sigma_t\lambda=\lambda$ and thus
\[\begin{split}
&(-1)^{(n-1)d}\Tr\bigl(\Frob_t, \CF_{\ov t})
=\prod\ov\bchi(t)\cdot\Kl_n(\psi,\bchi,q,at^n)\\
=&\sigma_t\bigl(\Kl_n(\psi,\bchi,q,a)\bigr)
=\lambda\sigma_t\bigl(\Kl_n(\psi,\brho,q,b)\bigr)\\
=&\lambda\prod\ov\brho(t)\cdot\Kl_n(\psi,\brho,q,bt^n)
=(-1)^{(n-1)d}\lambda\Tr\bigl(\Frob_t, \CF'_{\ov t}).
\end{split}\]
Apply Lemma to $r=s=n^d,M_0=0,M_\infty\le 1$.

\end{frame}


\begin{frame}{Distinctness}
Now 
	\[\CG_a(\bchi)\otimes \CL_{\prod \ov\bchi}\inj \CF_b(\brho)\otimes\CL_{\prod\ov \brho},\quad \CG_b(\brho)\otimes \CL_{\prod \ov\brho}\inj \CF_a(\bchi)\otimes\CL_{\prod\ov \bchi}.\]
Thus the highest weight $\lambda_a(\bchi)=\lambda_b(\brho)$.
Derived from this, and combining Fisher's arguments,  we have:
\begin{theorem}[Z.]
Let $a,b\in\BF_q^\times$.
Assume that $\bchi,\brho$ are not Kummer-induced and neither of them is of type $(\xi_1,\xi_1^{-1},1,\Lambda_2)\xi_2$.
If $p>(2n^{2d}+1)^2$ and 
	\[\Kl_n(\psi,\bchi,q,a)=\lambda\Kl_n(\psi,\brho,q,b)\]
for some $\lambda\in\mu_{q-1}$, then there exists $\sigma\in\Gal(\BF_q/\BF_p)$ and a multiplicative character $\eta$, such that $b=\sigma(a)$ and $\brho=\eta\cdot(\bchi\circ\sigma^{-1})$ as unordered tuples. Moreover, either both Kloosterman sums vanish or $\eta(b)=\lambda^{-1}$.
\end{theorem}
\end{frame}

\section{Generating fields of Kloosterman sums}


\begin{frame}{Non-vanishingness}
The last step is to show the non-vanishingness.
\begin{theorem}
If $p>(3n-1)C_\bchi-n$ and for any $i,j$, $\chi_i=\chi_j$ if $\chi_i^n=\chi_j^n$, then $\Kl_n(\psi,\bchi,q,a)$ is nonzero.
Here 
\begin{equation}
	C_\bchi=\max_{i,j}\lcm\bigl(\ord(\chi_i),\ord(\chi_j)\bigr)
\end{equation}
is the supremum of least common multipliers of the orders of any two characters in $\bchi$.
\end{theorem}

\end{frame}


\begin{frame}{Non-vanishingness, continue}
We can express $\Kl_n$ as Gauss sums
	\[(q-1)\Kl_n(\psi,\bchi,q,a)=
	\sum_{m=0}^{q-2}\omega^m(a) \prod_{i=1}^n g(m+s_i)\]
by Fourier transform on $\BF_q^\times$, where $\chi_i=\omega^{s_i}$ for a Teichm\"uller character.
What we need to do is to proof there is a unique $m$ such that the valuation of $\prod_{i=1}^ng(m+s_i)$ is minimal.

\end{frame}


\begin{frame}{Main result}
\begin{theorem}[Z.]
If $p>\max\set{(2n^{2d}+1)^2,(3n-1)C_\bchi-n}$ and for any $i,j$, $\chi_i=\chi_j$ if $\chi_i^n=\chi_j^n$, then $\Kl_n(\psi,\bchi,q,a)$ generates $\BQ(\mu_{pc})^H$, where $H$ consists of those $\sigma_t\tau_w$ such that there exists an integer $\beta$ and a character $\eta$ satisfying
  \[t=\lambda a_1^\beta,\lambda^{n_1}=1,\ \bchi^w=\eta\bchi^{q_1^\beta},\ \eta(a)=\prod\bchi^w(t).\]
Here $n_1=(n,p-1)$, $q_1=\#\BF_p(a^{(p-1)/n_1})$ and $a_1\in\BF_p^\times$ such that $a_1^{n/n_1}=\bfN_{\BF_{q_1}/\BF_p}(a^{(1-p)/n_1})=a^{(1-q_1)/n_1}$.
\end{theorem}
\end{frame}


\begin{frame}{An example: $n=2$ case}
Let $\bchi=\set{1,\chi}$, where $\chi$ is a multiplicative character of order $c\neq 2$.
If $p>\max\set{(2^{2d+1}+1)^2,5c-2}$, then $\Kl(\psi,\bchi,p^d,a)$ generates $\BQ(\mu_{pc})^H$, where 
	\[H=
	\begin{cases}
		\pair{\tau_{q_1}\sigma_{a_1},\sigma_{-1},\tau_{-1}},
			&\text{if}\ \chi(-1)=1,\chi(a)=1;\\
		\pair{\tau_{-q_1}\sigma_{a_1},\sigma_{-1}},
			&\text{if}\ \chi(-1)=1,\chi(a)=\chi(a_1)=-1;\\
		\pair{\tau_{q_1^{\alpha}}\sigma_{a_1^\alpha},\sigma_{-1}},
			&\text{if}\ \chi(-1)=1,\chi(a)^\alpha\neq 1;\\
		\pair{\tau_{q_1}\sigma_{-a_1},\tau_{-1}\sigma_{-1}},
			&\text{if}\ \chi(-1)=-1, \chi(a)=\chi(a_1)=-1;\\
		\pair{\tau_{q_1}\sigma_{a_1},\tau_{-1}}
			&\text{if}\ \chi(-1)=-1, \chi(a)=1;\\
		\pair{\tau_{q_1}\sigma_{a_1},\tau_{-1}\sigma_{-1}},
			&\text{if}\ \chi(-1)=-1,\chi(a)=-1,\chi(a_1)=1;\\
		\pair{\tau_{q_1^{\alpha/2}}\sigma_{-a_1^{\alpha/2}}},
			&\text{if}\ \chi(-1)=-1,2\mid \alpha,\chi(a)\neq \pm1;\\
		\pair{\tau_{q_1^{\alpha}}\sigma_{a_1^\alpha}},
			&\text{if}\ \chi(-1)=-1,2\nmid \alpha,\chi(a)\neq \pm1.
	\end{cases}\]
 $q_1=\#\BF_p(a^{(1-p)/2}), a_1=a^{(1-q_1)/2}$ and $\alpha$ is the order of $\chi(a_1)\in\mu_{p-1}$.

\end{frame}


\begin{frame}{Remark}
Consider the Kloosterman sums
	\[S_k=\Kl(\psi,\bchi\circ\bfN_{\BF_{q^k}/\BF_q},q^k,a).\]
If $p>\max\set{(2n^{2dk}+1)^2,(3n-1)C_\bchi-n},$ then $\BQ(S_k)=\BQ(\mu_{pc})^H$, where $H$ consists of those $\sigma_t\tau_w$ such that there exists an integer $\beta$ and a character $\eta$ on $\BF_q^\times$ satisfying
	\[	t=\lambda a_1^\beta,\lambda^{n_1}=1,\quad \bchi^w=\eta\bchi^{q_1^\beta},\quad \eta(a)=\gamma\cdot\prod\bchi^w(t),\gamma^k=1.\]
Thus $\BQ(S_k)=\BQ(S_{k-c})$ since $\gamma^c=1$.

\end{frame}


\begin{frame}{Remark, continue}
The $L$-function 
	\[L(T)=\exp\left(\sum_{k=1}^\infty \frac{T^k}{k} S_k\right)\]
is a rational function.
Thus the sequence $\set{S_k}_k$ is linear recurrence sequence.
The sequence $\set{\BQ(S_k)}_{k\ge N}$ is periodic of period $r$ for some $N$ (Wan, Yin).
Thus if $p>\max\set{\bigl(2n^{2d(N+r)}+1\bigr)^2,(3n-1)C_\bchi-n}$,  the generating field of $S_k$ is determined by the previous equations for any $k$.
For this purpose, we need to decrease the bound $(2n^{2d}+1)^2$ and estimate the period $r$ and $N$.
We conjecture that $S_k$ has the predicted generating field if $p>3ndc$.
\end{frame}

\framethanks


\end{document}
